%%%%%%%%%%%%%%%%%%%%%%%%%%%%%%%%%%%%%%%%%%%%%%%%%%%%%%%%%%
%%
%%  PROJECT: CDRBCID - Cech-De-Rham Bicomplex in Diffeology
%%
%%  Created by Patrick Iglesias-Zemmour on 18/07/18.
%%  Copyright 2018 __MyCompanyName__.
%%
%%  All rights reserved.
%%
%%%%%%%%%%%%%%%%%%%%%%%%%%%%%%%%%%%%%%%%%%%%%%%%%%%%%%%%%%

%%%%%%%%%%%%%%%%%%%%%%%%%%%%%%%%%%%%%%%%%%%%%%%%%%%%%%%%%%
%%
%% MARK: § Front Matter
%%
%%%%%%%%%%%%%%%%%%%%%%%%%%%%%%%%%%%%%%%%%%%%%%%%%%%%%%%%%%

\documentclass[12pt,reqno]{amsart}

%%%%%%%%%%%%%%%%%%%%%%%%%%%%%%%%%%%%%%%%%%%%%%%%%%%%%%%%%%
%%
%% MARK: Macro file
%%
%%%%%%%%%%%%%%%%%%%%%%%%%%%%%%%%%%%%%%%%%%%%%%%%%%%%%%%%%%

%%%%%%%%%%%%%%%%%%%%%%%%%%%%%%%%%%%%%%%%%%%%%%%%%%%%%%%%%%
%%
%% MARK: Page Display
%%
%%%%%%%%%%%%%%%%%%%%%%%%%%%%%%%%%%%%%%%%%%%%%%%%%%%%%%%%%%

\parindent 0mm
\parskip 0.5em plus 1pt

%%%%%%%%%%%%%%%%%%%%%%%%%%%%%%%%%%%%%%%%%%%%%%%%%%%%%%%%%%
%%
%% MARK: Theorem Environments
%%
%%%%%%%%%%%%%%%%%%%%%%%%%%%%%%%%%%%%%%%%%%%%%%%%%%%%%%%%%%

\newtheoremstyle{article} % ⟨name⟩
{7pt} %  ⟨Space above⟩
{7pt} %  ⟨Space below⟩
{} %     ⟨Body font⟩
{0pt} %  ⟨Indent amount⟩
{\bf} %  ⟨Theorem head font⟩
{.~} %  ⟨Punctuation after theorem head⟩
{0pt} %  ⟨Space after theorem head⟩
{} %     ⟨Theorem head spec (can be left empty, meaning ‘normal’)⟩

\theoremstyle{article}
\newtheorem{article}{}

\def\artlabel[#1]{\textsc{#1}.}

\newcommand{\art}[1]{(\text{\textsection} \ref{#1})}
\newcommand{\xart}[2]{(\text{\textsection} \ref{#1}, #2)}

\renewenvironment{proof}{\noindent \textit{Proof.}} {\nolinebreak\hfill $\square$}

\renewcommand\thesection{\roman{section}}

%%%%%%%%%%%%%%%%%%%%%%%%%%%%%%%%%%%%%%%%%%%%%%%%%%%%%%%%%%
%%
%% MARK: Packages Figures
%%
%%%%%%%%%%%%%%%%%%%%%%%%%%%%%%%%%%%%%%%%%%%%%%%%%%%%%%%%%%

% Pour figures et diagrammes
% http://texdoc.net/texmf-dist/doc/latex/tikz-cd/tikz-cd-doc.pdf
\usepackage{tikz-cd}
\usetikzlibrary{calc}
\tikzcdset{
arrow style=tikz,
diagrams={>={Straight Barb[scale=0.8]}}
}

\newcommand{\rfl}[1]{\,{\buildrel{\displaystyle{#1}}\over{\hbox to 7mm{\rightarrowfill}}}\,}

\usepackage{yfonts}

\usepackage{cancel}

%%%%%%%%%%%%%%%%%%%%%%%%%%%%%%%%%%%%%%%%%%%%%%%%%%%%%%%%%%
%%
%% MARK: Letters
%%
%%%%%%%%%%%%%%%%%%%%%%%%%%%%%%%%%%%%%%%%%%%%%%%%%%%%%%%%%%

\def \A{\mathrm A}
\def \B{\mathrm B}
\def \C{\mathrm C}
\def \D{\mathrm D}
\def \E{\mathrm E}
\def \F{\mathrm F}
\def \G{\mathrm G}
\def \H{\mathrm H}
\def \I{\mathrm I}
\def \J{\mathrm J}
\def \K{\mathrm K}
\def \L{\mathrm L}
\def \M{\mathrm M}
\def \N{\mathrm N}
\def \O{\mathrm O}
\def \P{\mathrm P}
\def \Q{\mathrm Q}
\def \R{\mathrm R}
\def \S{\mathrm S}
\def \T{\mathrm T}
\def \U{\mathrm U}
\def \V{\mathrm V}
\def \W{\mathrm W}
\def \X{\mathrm X}
\def \Y{\mathrm Y}
\def \Z{\mathrm Z}


\newcommand{\CC}{{\bf C}}
\newcommand{\FF}{{\bf F}}
\newcommand{\HH}{{\bf H}}
\newcommand{\KK}{{\bf K}}
\newcommand{\NN}{{\bf N}}
\newcommand{\QQ}{{\bf Q}}
\newcommand{\RR}{{\bf R}}
\newcommand{\TT}{{\bf T}}
\newcommand{\ZZ}{{\bf Z}}

\newcommand{\cA}{{\mathcal A}}
\newcommand{\cB}{{\mathcal B}}
\newcommand{\cC}{{\mathcal C}}
\newcommand{\cD}{{\mathcal D}}
\newcommand{\cE}{{\mathcal E}}
\newcommand{\cF}{{\mathcal F}}
\newcommand{\cG}{{\mathcal G}}
\newcommand{\cI}{{\mathcal I}}
\newcommand{\cK}{{\mathcal K}}

\newcommand{\cM}{\mathcal{M}}

\newcommand{\cN}{\mathcal{N}}

\newcommand{\cO}{{\mathcal O}}
\newcommand{\cR}{{\mathcal R}}
\newcommand{\cS}{{\mathcal S}}
\newcommand{\cV}{{\mathcal V}}

\newcommand{\fB}{\textswab{B}}
\newcommand{\fF}{\textswab{F}}

\newcommand{\sC}{\mathsf{C}}
\newcommand{\sZ}{\mathsf{Z}}
\newcommand{\sB}{\mathsf{B}}
\newcommand{\sH}{\mathsf{H}}


%%%%%%%%%%%%%%%%%%%%%%%%%%%%%%%%%%%%%%%%%%%%%%%%%%%%%%%%%%
%%
%% MARK: Abreviations
%%
%%%%%%%%%%%%%%%%%%%%%%%%%%%%%%%%%%%%%%%%%%%%%%%%%%%%%%%%%%

\newcommand{\Cinfty}{{\mathcal C}^{\infty}}
\newcommand{\Cech}{\v{C}ech\ }
\newcommand{\class}{\mathrm{class}}
\DeclareMathOperator{\coker}{coker}

\DeclareMathOperator{\dom}{dom}
\newcommand{\ds}{{d\!s}}
\newcommand{\dt}{{d\!t}}

\newcommand{\dR}{\mathrm{dR}}

\DeclareMathOperator{\ev}{ev}
\newcommand{\exit}{\mathrm{out}}

\DeclareMathOperator{\Fl}{\mathsf{Fl}}

\DeclareMathOperator{\Hom}{Hom}

\newcommand{\init}{\mathrm{in}}
\newcommand{\id}{{\bf 1}}

\newcommand{\loc}{\text{loc}}

\DeclareMathOperator{\Maps}{Maps}

\newcommand{\pr}{{\rm pr}}
\newcommand{\Param}{{\rm Param}}
\DeclareMathOperator{\Paths}{Paths}

\DeclareMathOperator{\Supp}{Supp}

\newcommand{\ul}{\underline}

\DeclareMathOperator{\Val}{im}

\newcommand{\vB}{\check{\mathrm{B}}}
\newcommand{\vC}{\check{\mathrm{C}}}
\newcommand{\vH}{\check{\mathrm{H}}}
\newcommand{\vZ}{\check{\mathrm{Z}}}

%%%%%%%%%%%%%%%%%%%%%%%%%%%%%%%%%%%%%%%%%%%%%%%%%%%%%%%%%%%%%%%%%%%%%%%%%%%%%%%
%%
%% MARK: MACROS General Layout
%%
%%%%%%%%%%%%%%%%%%%%%%%%%%%%%%%%%%%%%%%%%%%%%%%%%%%%%%%%%%%%%%%%%%%%%%%%%%%%%%%

\usepackage[T1]{fontenc}
\usepackage[utf8x]{inputenc}

%%%%%%%%%%%%%%%%%%%%%%%%%%%%%%%%%%%%%%%%%%%%%%%%%%%%%%%%%%%%%%%%%%%%%%%%%%%%%%%
%%<<< Begin - Garamond style
%%%%%%%%%%%%%%%%%%%%%%%%%%%%%%%%%%%%%%%%%%%%%%%%%%%%%%%%%%%%%%%%%%%%%%%%%%%%%%%

\usepackage{microtype}
\usepackage[cal=scr,expert,frenchmath,uppercase = upright,greeklowercase = upright,greekfamily = didot, garamond]{mathdesign}
\usepackage{ebgaramond}
% Change the caligraphy to Euler characters
\usepackage[mathcal]{eucal}
\linespread{1.1}

%%%%%%%%%%%%%%%%%%%%%%%%%%%%%%%%%%%%%%%%%%%%%%%%%%%%%%%%%%%%%%%%%%%%%%%%%%%%%%%
%%>>> End - Garamond style or Charter style ?
%%%%%%%%%%%%%%%%%%%%%%%%%%%%%%%%%%%%%%%%%%%%%%%%%%%%%%%%%%%%%%%%%%%%%%%%%%%%%%%

\def \H{\mathrm H}
\def \L{\mathrm L}
\def \O{\mathrm O}
\let \epsilon=\varepsilon

%%%%%%%%%%%%%%%%%%%%%%%%%%%%%%%%%%%%%%%%%%%%%%%%%%%%%%%%%%
%%
%% Begin Document
%%
%%%%%%%%%%%%%%%%%%%%%%%%%%%%%%%%%%%%%%%%%%%%%%%%%%%%%%%%%%

\begin{document}
  
  \allowdisplaybreaks
  
  %%%%%%%%%%%%%%%%%%%%%%%%%%%%%%%%%%%%%%%%%%%%%%%%%%%%%%%%%%
  %%
  %% MARK: • Title & Data
  %%
  %%%%%%%%%%%%%%%%%%%%%%%%%%%%%%%%%%%%%%%%%%%%%%%%%%%%%%%%%%
  
  \title[\v{C}ech-De-Rham Bicomplex in Diffeology]{\v{C}ech-De-Rham Bicomplex in Diffeology}
  
  \author{Patrick Iglesias-Zemmour}
  
  %\date{\today}
  \date{January 31, 1922.}

  \address{Patrick Iglesias-Zemmour --- Einstein Institute of Mathematics,
  The Hebrew University of Jerusalem,
  Campus Givat Ram,
  9190401 Israel.}
  \email{piz@math.huji.ac.il}
  \urladdr{http://math.huji.ac.il/~piz}
  
  \keywords{Diffeology, \Cech Cohomology, De Rham Cohomology.}
  \subjclass{58A12,55N05,53C99}
  
  \maketitle
  
%  \tikz[overlay,remember picture]\n
%  {\n
%  \node at ($(current page.west)+(1.5,0)$) [rotate=90] {\Huge\textcolor{gray}{Draft Version -- \pdfcreationdate}}% \pdfcreationdate}};\n
%  }\n
  
  %%%%%%%%%%%%%%%%%%%%%%%%%%%%%%%%%%%%%%%%%%%%%%%%%%%%%%%%%%
  %%
  %% MARK: - § Abstract
  %%
  %%%%%%%%%%%%%%%%%%%%%%%%%%%%%%%%%%%%%%%%%%%%%%%%%%%%%%%%%%
  
  \begin{abstract}
    We propose a \Cech cohomology for diffeological spaces as the Hochschild cohomology of some characteristic gauge monoid.
    Then,
    we describe how this cohomology is connected with the De Rham cohomology,
    through a double-complex which we make explicit.
    We show that the first obstruction for the De Rham theorem in diffeology,
    which is the middle term of the exact sequence of low-degree terms of the  associated spectral sequence,
    can be interpreted as some group of $(\RR,+)$ principal fiber bundles.
    That introduces a first characteristic class for diffeological spaces which does not exist for manifolds.
  \end{abstract}
    
  %%%%%%%%%%%%%%%%%%%%%%%%%%%%%%%%%%%%%%%%%%%%%%%%%%%%%%%%%%
  %%
  %% MARK: - § Introduction
  %%
  %%%%%%%%%%%%%%%%%%%%%%%%%%%%%%%%%%%%%%%%%%%%%%%%%%%%%%%%%%
  
  \section*{Introduction}
  
  This paper introduces a general \emph{\Cech cohomology} for diffeological spaces,
  which is adapted to their specificity.
  In particular,
  it could not be a plain transliteration of what exists in topology,
  since diffeological spaces can be substantially non trivial even with a trivial topology.
  %
  It seems that a good approach in diffeology,
  for the \Cech cohomology,
  will be to be defined as the \emph{Hochschild cohomology} of a \emph{gauge monoid},
  associated with the specific generating family of \emph{round plots}.
  The computation in the case of the irrational torus shows that this \Cech cohomology achieves its purpose,
  even in this extreme case.
  Indeed,
  for the quotient $\T_\K = \RR/\K$,
  with $\K \subset \RR$ any strict subgroup,
  dense or not,
  we get $\vH^\star(\T_\K,\RR) = \sH^\star(\K,\RR)$,
  where the second term is the ordinary group cohomology of $\K$,
  with real coefficients.
  
  Then,
  we connect this \Cech cohomology with the De Rham cohomology through a double complex,
  in Weil's style \cite{Wei52}.
  We describe this complex,
  the associated two filtrations and the spectral sequence.
  Contrarily to the case of smooth manifolds,
  the spectral sequence does not split in general and contains the obstructions to the De Rham theorem in diffeology.
  We knew already
  ---~by a specific computation~---
  that the obstruction of the first De Rham homomorphism is the group of equivalent $(\RR,+)$ principal bundles,
  equipped with a flat connection \cite{Igl85}.
  Thanks to the \v{C}ech-De-Rham bicomplex,
  we find this obstruction as the middle term in the low-degree sequence of the spectral sequence,
  which gives to it a full-fledged geometrical status.
  
  We must acknowledge that the geometrical natures of the higher obstructions of the De Rham theorem still remain uninterpreted.
  It would be certainly interesting to pursue this matter further.
  
  \textsc{Note 1.} This paper is a translation,
  also a revision,
  of a preprint \cite{PI88-91} written in French,
  and issued first in 1988 by the Center for Theoretical Physics%
  \footnote{Luminy, in Marseille (France).}.
  It has never been published in a journal and remained a confidential document.
  I felt recently the need to publish it after it has been cited a couple of times in \cite{Gur14} and \cite{Kur19}.
  
  This preprint was the first attempt to understand \emph{what happens to the De Rham theorem in diffeology}?
  We know indeed that,
  in diffeology,
  the De Rham homomorphism is not necessarily an isomorphism.
  The first and simplest counter-example is the irrational torus $\T_\alpha = \RR/ \ZZ + \alpha \ZZ$ \cite{DonIgl83}.
  Its first cohomology group $\H^1(\T_\alpha,\RR)$
  ---~interpreted \emph{a priori} as $\Hom(\pi_1(\T_\alpha),\RR)$~---
  is equal to $\Hom(\ZZ^2,\RR) = \RR^2$.
  While its first De Rham cohomology group $\H^1_{\D\R}(\T_\alpha)$ is equal to $\RR$.
  There was one factor $\RR$ missing\dots
  
  \textsc{Note 2.} Diffeologies have been introduced by J.-M.~Souriau in the eighties,
  and began to be developed at that time by his students,
  \cite{Sou80},
  \cite{DonIgl83},
  \cite{Sou84-a},
  \cite{Sou84-b},
  \cite{Don84},
  \cite{Igl85}.
  A similar theory of ``differential spaces'' had been proposed by K.-T.~Chen in the seventies \cite{Che77}.
  
  \textsc{Acknowledgements:} I would like to thank the anonymous referee of the Israel Journal of Mathematics for his careful reading and corrections that improved the quality of the manuscript.
  
  %%%%%%%%%%%%%%%%%%%%%%%%%%%%%%%%%%%%%%%%%%%%%%%%%%%%%%%%%%
  %%
  %% MARK: - § Diffeological Spaces And Differential Forms
  %%
  %%%%%%%%%%%%%%%%%%%%%%%%%%%%%%%%%%%%%%%%%%%%%%%%%%%%%%%%%%
  
  \section{Diffeological Spaces And Differential Forms}
  
  For self consistency we recall first some definitions in diffeology we use in the following.
  Any detail can be found now in the textbook \cite{PIZ13}.
  %    Then,
  %    we reinterpret the definition of the differential forms in diffeology and introduce a few related constructions.
  
  %% MARK: • Diffeology And Diffeological Spaces
  \begin{article}\artlabel[Diffeology And Diffeological Spaces]
    \label{Diffeology-And-Diffeological-Spaces}
    We call \emph{Euclidean domain} any open subset of some Euclidean space $\RR^n$,
    for any integer $n$.
    We call \emph{parametrization} in a set $\X$,
    any map $\P \colon \U \to \X$,
    where $\U$ is a Euclidean domain.
    The set of all parametrizations in $\X$ is denoted by $\Param(\X)$.
    
    A \emph{diffeology} on a set $\X$ is the choice of a set $\cD$ of parametrizations in $\X$ which satisfies the following axioms.
    
    \begin{enumerate}
      \item[1.] \textsc{Covering}\,: $\cD$ contains the constant parametrizations.
      \item[2.] \textsc{Locality}\,: Let $\P$ be a parametrization in $\X$.
      If for all $r \in \dom(\P)$ there is an open neighbourhood $\V$ of $r$ such that $\P \restriction \V \in \cD$,
      then $\P \in \cD$.
      \item[3.] \textsc{Smooth compatibility}\,: For all $\P \in \cD$,
      for all $\F \in \Cinfty(\V,\dom(\P))$,
      where $\V$ is a Euclidean domain,
      $\P \circ \F \in \cD$.
    \end{enumerate}
    
    A set $\X$ equipped with a diffeology is a \emph{diffeological space}.
    The elements of $\cD$ are then called \emph{plots} of the diffeological space.
  \end{article}
  
  %% MARK: • Smooth Maps
  \begin{article}\artlabel[Smooth Maps]
    A map $f \colon \X \to \X'$ is said to be \emph{smooth} if for any plot $\P$ in $\X$,
    $f \circ \P$ is a plot in $\X'$.
    If $f$ is smooth,
    bijective,
    and if its inverse $f^{-1}$ is smooth,
    then $f$ is said to be a \emph{diffeomorphism}.
    
    Diffeological spaces and smooth maps constitute the category \{Diffeology\},
    stable by set-theoretic operations,
    whose isomorphisms are diffeomorphisms.
    
    The set of smooth map from $\X$ to $\X'$ can be equipped with a natural diffeology called \emph{functional diffeology}.
    That makes the category Cartesian closed.
    For example,
    the set of (smooth) paths in $\X$,
    that is,
    $$
    \Paths(\X) = \Cinfty(\RR,\X),
    $$
    is equipped with the functional diffeology.
  \end{article}
  
  %% MARK: • Differential Forms In Diffeology
  \begin{article}\artlabel[Differential Forms In Diffeology]
    \label{Differential-forms-in-diffeology}
    {A \emph{differential $k$-form} $\alpha$ on a diffeological space $\X$ is a mapping,
    that associates with every plot $\P \colon \U \to \X$ a smooth $k$-form $\alpha(\P)$ on $\U$,
    such that for any smooth parametrization $\F$ in $\U$,
    $$
    \alpha(\P \circ \F) = \F^*(\alpha(\P)).
    $$}
    The set of differential $k$-forms on $\X$ is a real vector space denoted by $\Omega^k(\X)$.
    The total complex is denoted by
    $$
    \Omega^\star(\X) = \bigoplus_{k \in \NN} \Omega^k(\X).
    $$
    Let $\X'$ be another diffeological space and $f \colon \X \to \X'$ be a smooth map,
    and let $\alpha' \in \Omega^k(\X')$.
    Then,
    the pullback of $\alpha'$ by $f$ is denoted by $f^*(\alpha')$,
    this is an element of $\Omega^k(\X)$ and it is defined by
    $$
    f^*(\alpha)(\P) = \alpha(f \circ \P).
    $$
    The exterior differential and the exterior product are defined as follows:
    $$
    (d_\dR\alpha)(\P) = d_\dR[\alpha(\P)]
    \quad \text{and} \quad
    (\alpha \wedge \beta)(\P) = \alpha(\P) \wedge \beta(\P),
    $$
    where the exterior De Rham differential $d_\dR$ is a linear operator mapping $\Omega^k(\X)$ to $\Omega^{k+1}(\X)$,
    for all $k \in \NN$;
    and the wedge product is a bilinear operator from $\Omega^k \times \Omega^\ell$ to $\Omega^{k+\ell}$.
    Then,
    $\Omega^\star(\X)$ is a differential graduated algebra over $\RR$,
    with unit the constant function equal to $1$.
    The graduated subalgebra of De Rham cocycles (closed forms) and the ideal of De Rham coboundaries (exact forms) are well defined,
    $$
    \Z_\dR^\star(\X) = \bigoplus_{k \in \NN} \Z_\dR^k(\X)
    \quad \text{and} \quad
    \B_\dR^\star(\X) = \bigoplus_{k \in \NN} \B_\dR^k(\X).
    $$
    Where:
    $\Z_\dR^k(\X) = \ker d_\dR \colon \Omega^k(\X) \to \Omega^{k+1}(\X)$,
    $\B_\dR^k(\X) = d_\dR\big(\Omega^{k-1}(\X)\big)$ and $\B_\dR^0(\X) =\{0\}$.
    The De Rham cohomology groups are then defined as usual:
    $$
    \H^k_\dR(\X) = \Z^k_\dR(\X)/\B_\dR(\X)
    \quad \text{and} \quad
    \H_\dR^\star(\X) = \bigoplus_{k \in \NN} \H^k_\dR(\X).
    $$
    
    \textsc{Note.} The De Rham cohomology is still homotopic invariant in diffeology.
    The proof uses the \emph{Chain-Homotopy Operator} \cite[\textsection 6.83]{PIZ13},
    which is a linear operator satisfying:
    $$
    \cK \colon \Omega^k(\X) \to \Omega^{k-1}(\Paths(\X))
    \quad \text{with} \quad
    \cK \circ d + d \circ \cK = \hat 1^* - \hat 0^*,
    $$
    where $\hat t(\gamma) = \gamma(t)$,
    for all $\gamma \in \Paths(\X)$.
    Precisely,
    
    \textsc{Proposition} \textit{Let $\X'$ and $\X$ be two diffeological spaces.
    Let $\alpha$ be a closed $k$-form on $\X$,
    with $k \geq 1$.
    Let $t \mapsto f_t$ be a smooth path in $\Cinfty(\X',\X)$,
    that is,
    a homotopy from $f_0$ to $f_1$.
    Then $f_1^*(\alpha) = f_0^*(\alpha) + d\beta$,
    with $\beta \in \Omega^{k-1}(\X)$.}
    
    It should be noted that the short proof that follows is essentially diffeological,
    since it uses fundamentally the notion of differential forms on the space of paths,
    which is not a manifold,
    except in trivial cases.
  \end{article}
  
  \begin{proof}
    The homotopy $t \mapsto f_t$ induces a smooth map $\phi \colon \X' \to \Paths(\X)$,
    with $\phi(x') = [t \mapsto f_t(x')]$.
    Then,
    the pullback by $\phi$ of the Chain-Homotopy Operator identity,
    applied to $\alpha$,
    gives $\phi^*(\cK(d\alpha)) + \phi^*(d(\cK(\alpha))) = \phi^*(\hat 1^*(\alpha)) - \phi^*(\hat 0^*(\alpha))$.
    That is,
    since $d\alpha = 0$ and $\phi^* \circ \hat t^* = (\hat t \circ \phi)^* = f_t^*$,
    $d(\phi^*(\cK\alpha)) = f_1^*(\alpha) - f_0^*(\alpha)$.
    That is,
    $f_1^*(\alpha) = f_0^*(\alpha) + d\beta$,
    with $\beta = \phi^*(\cK\alpha)$.
  \end{proof}
  
  %% MARK: • D-Topology
  \begin{article}\artlabel[D-Topology]
    {We call D-topology on a diffeological space $\X$ the finest topology that makes the plots continuous.
    A subset $\cO \subset \X$ is open for the D-topology if and only if $\P^{-1}(\cO)$ is open for all plots $\P$ \cite{Igl85}.}
  \end{article}
  
  %% MARK: • Generating Families
  \begin{article}\artlabel[Generating Families]
    Let $\fF$ be any family of parametrizations in a set $\X$.
    There exists a finest diffeology on $\X$ such that $\fF$ is made of plots in $\X$.
    This diffeology is said to be \emph{generated} by $\fF$,
    and $\fF$ is called a \emph{generating family}  \cite{Igl85}.
    Every diffeology admits generating families,
    at least the diffeology $\cD$ itself.
    
    The family \emph{covers} $\X$ if it is surjective on $\X$.
    Otherwise one adds the constant parametrizations, without changing the diffeology.
    That is why we consider only covering families.
    
    The plots $\P \colon \U \to \X$ of the generated family are the parametrizations such that for every point in $\U$,
    there exist $\F \in \fF$ and a smooth parametrization $\Q \in \Cinfty(\V,\U)$ such that,
    locally,
    $\P =_\loc \F \circ \Q$.
    
    The \emph{nebula} $\cN$ of the generating family $\fF$ is the sum of the domains of its elements,
    that is,
    $$
    \cN = \coprod_{\F \in \fF} \dom(\F) = \{ (\F, r) \mid \F \in \fF \text{ and } r \in \dom(\F) \}
    $$
    A plot in the nebula is any smooth parametrization with values,
    locally,
    into some fixed $\dom(\F)$.
    In other words,
    the set of indices $\F \in \fF$ is discrete for the sum diffeology.
    A nebula comes with its \emph{evaluation map} defined by:
    $$
    \ev \colon \cN \to \X
    \quad \text{with} \quad
    \ev(\F,r) = \F(r).
    $$
    The evaluation map is a subduction,
    that is,
    it realizes $\X$ as a quotient of $\cN$.
    In other words,
    every diffeological space is a quotient of such a \emph{bubbles space} by an equivalence.
  \end{article}
  
  %% MARK: • Diffeological Fiber Bundles
  \begin{article}\artlabel[Diffeological Fiber Bundles]
    A \emph{diffeological fibration},
    or \emph{fiber bundle},
    is a projection $\pi \colon \Y \to \X$,
    where $\X$ and $\Y$ are diffeological spaces,
    which is \emph{locally trivial along the plots}.
    That means that,
    there exists a diffeological space $\F$ such that for all plot $\P \colon \U \to \X$,
    the pullback $\pr_1 \colon \P^*(\Y) \to \U$ is locally trivial with fiber $\F$,
    where
    $$
    \P^*(\Y) = \{ (r,y) \in \U \times \Y \mid \P(r) = \pi(y \}.
    $$
    A \emph{principal fibration},
    or a \emph{principal fiber bundle},
    is a diffeological fiber bundle $\pi \colon \Y \to \X$ with a smooth free action of a diffeological group $\G$,
    such that the fibers of $\pi$ are the orbits of $\G$,
    and such that the local trivialisations of $\pr_1 \colon \P^*(\Y) \to \U$ intertwin the actions of $\G$ on each side.
    For more details see \cite{Igl85} and \cite{PIZ13},
    particularly \cite[\textsection 8.12 8.13]{PIZ13} for principal bundles.
  \end{article}
  
  %%%%%%%%%%%%%%%%%%%%%%%%%%%%%%%%%%%%%%%%%%%%%%%%%%%%%%%%%%
  %%
  %% MARK: - § Cech Cohomology On Diffeological Spaces
  %%
  %%%%%%%%%%%%%%%%%%%%%%%%%%%%%%%%%%%%%%%%%%%%%%%%%%%%%%%%%%
  \section{\Cech Cohomology On Diffeological Spaces}
  
  In this section,
  we propose a \Cech cohomology for diffeological spaces.
  A definition which tries to take in consideration the very specificity of diffeological spaces,
  compared with Euclidean domain or manifolds.
  Think for example to the irrational torus which has a trivial topology but which is highly not trivial from the point of view of diffeology \cite{DonIgl83}.
  
  Since we will build this cohomology as the Hochschild cohomology of some monoid,
  we recall this definition we can find%
  \footnote{We read in part X.5 (p. 291) of MacLane ``Homology''\,:
  ``The homology of a group (or a monoid) is a special case of the Hochschild homology of its group ring'',
  where Hochschild cohomology is introduced in \cite{Hoch45}.}
  in \cite{Mac67}.
  
  %% MARK: • Hochschild Cohomology
  \begin{article}\artlabel[Hochschild Cohomology]
    \label{Hochschild-Cohomology}
    Let $\cM$ be a monoid,
    and let $(f,a) \mapsto f^*(a)$ be a covariant action of $\cM$ on some Abelian group $\A$.
    For all $f,g \in \cM$ and all $a,b \in \A$,
    $$
    f^*(g^*(a)) = (g \circ f)^*(a)
    \quad \text{and} \quad
    f^*(a+b) = f^*(a) + f^*(b).
    $$
    The set $\sC^k(\cM,\A)$ of $k$-cochains from $\cM$ to $\A$ is defined as the space of maps $\sigma$ from $\cM^k$ to $\A$.
    $$
    \sC^k(\cM,\A) = \Maps(\cM^k,\A)
    \quad \text{and} \quad
    \sC^0(\cM,\A) = \A.
    $$
    The $i$-th simplicial faces of the cochain $\sigma$ are then defined by,
    $$
    \left \{ \begin{array}{lcl}
      \delta_0\sigma (f_0,\ldots,f_k) & = & f_0^*\big(\sigma(f_1,\ldots,f_k)\big) \\
      \delta_{i}\sigma (f_0,\ldots,f_k) & = &\sigma(f_0,\ldots,f_{i}\circ f_{i-1},\ldots,f_k),
      \quad  0<i\leq k  \\
      \delta_{k+1}\sigma (f_0,\ldots,f_k) & = & \sigma(f_0,\ldots,f_{k-1}) \\
    \end{array}\right.
    $$
    for $k \geq 1$. And for $k=0$,
    the two simplicial faces of the cochain $a$ are given by
    $$
    \delta_0(a)(f) = f^*(a)
    \quad \text{and} \quad
    \delta_1(a)(f) = a.
    $$
    The Hochschild coboundary operator is then defined by
    $$
    \delta \colon \sC^k(\cM,\A) \to \sC^{k+1}(\cM,\A)
    \quad \text{with} \quad
    \delta = \sum_{i=0}^{k+1} (-1)^i\delta_i.
    $$
    This coboundary operator satisfies as expected $\delta \circ \delta = 0$ and then defines a cohomology.
    The group of $k$-cocycles is denoted by $\sZ^k(\cM,\A)$,
    the group of $k$-boundaries is denoted by $\sB^k(\cM,\A)$,
    and the \emph{Hochschild cohomology groups} are denoted by $\sH^k(\cM,\A)$.
    $$
    \left \{ \begin{array}{lcl}
      \sZ^k(\cM,\A) & = & \ker[\delta \colon \sC^k(\cM,\A) \to \sC^{k+1}(\cM,\A)], \quad k\geq 0. \\[.3ex]
      \sB^k(\cM,\A) & = & \delta \big(\sC^{k-1}(\cM,\A)\big), \quad k>0, \quad \text{and} \quad \sB^0(\cM,\A) =0. \\[.3ex]
      \sH^k(\cM,\A) & = & \sZ^k(\cM,\A)/\sB^k(\cM,\A), \quad k \geq 0.
    \end{array}\right.
    $$
    
    \textsc{Remark.} The first Hochschild cohomology group has a particularly simple interpretation.
    One gets immediately,
    for a $0$-cochain $a$,
    $\delta(a)(f) = f^*(a) - a$.
    Thus,
    $a \in \sZ^0(\cM,\A)$ if and only if $f^*(a) - a = 0$,
    for all $f \in \cM$.
    And since $\sB^0(\cM,\A) = \{0\}$,
    we get
    $$
    \sH^0(\cM,\A) = \{ a \in \A \mid f^*(a) = a, \forall f \in \cM \}.
    $$
    This cohomology group is the subgroup in $\A$ made of fixed elements by the covariant action of the monoid $\cM$.
  \end{article}
  
  %% MARK: • Nebulae And Round Plots
  \begin{article}\artlabel[The Gauge Monoid of Round Plots]
    \label{The-Gauge-Monoid-of-Round-Plots}
    Let us introduce now the \emph{nebula of round plots} and the associated \emph{gauge monoid}.
    
    \textsc{Definition 1.} \textit{We call \emph{round parametrization},
    and \emph{round plots},
    every parametrization,
    and plot,
    defined on an open ball of some Euclidean domain.}
    
    The set of round plots of a diffeological space $\X$ is obviously a generating family.
    Let $\cD$ be the diffeology of $\X$,
    and let $\fB$ be the set of its round plots:
    $$
    \fB = \{ \phi \in \cD \mid \dom(\phi) \text{ is some Euclidean open ball $\B$} \}.
    $$
    We will denote by $\cB$ the nebula of the family $\fB$,
    that is,
    $$
    \cB = \coprod_{\phi \in \fB} \dom(\phi) = \{ (\phi, r) \mid \phi \in \fB \text{ and } r \in \dom(\phi) \},
    $$
    and $\ev \colon \cB \to \X$ be the evaluation map $\ev(\phi,r) = \phi(r)$.
    
    Let us,
    now,
    introduce the second main object of this study.
    
    \textsc{Definition 2.} \textit{We denote by $\cM$ and we call \emph{gauge monoid} (of the round plots),
    the set of smooth maps on $\cB$ that project,
    along the evaluation map,
    to the identity of $\X$,
    that is,
    $$
    \cM = \{ f \in \Cinfty(\cB,\cB) \mid \ev \circ f = \ev \};
    $$
    }
    and let
    $$
    f(\phi,r) = (f_\fB(\phi), f_\phi(r)),
    $$
    where $f \mapsto f_\fB \in \Maps(\fB,\fB)$ is the action of $\cM$ on the set of indices $\fB$,
    and $f_\phi \in \Cinfty(\dom(\phi), \dom(f_\fB(\phi)))$.
    
    \textsc{Note.} Actually we can restrict our attention to the sub-monoid
    $$
    \cM' = \{ f\in \cM\mid \#\, \Supp(f) < \infty \}
    $$
    of transformations with finite support,
    $\Supp(f) = \{ \phi \in \fB \mid f_\phi \neq \id_{\dom(\phi)} \}$.
    It seems that nowhere in this work this is necessary,
    however it could be necessary for some other developments.
    It is good to leave the door open for that.
  \end{article}
  
  %% MARK: • Differential Forms Revisited
  \begin{article}\artlabel[Differential Forms Revisited]
    \label{Differential-forms-Revisited}
    Let $\alpha \in \Omega^k(\X)$,
    its pullback
    $$
    \tilde\alpha = \ev^*(\alpha)
    $$
    is the smooth $k$-form on $\cB$ satisfying
    $$
    \tilde\alpha \restriction \{\phi\} \times \dom(\phi) = \alpha(\phi).
    $$
    And since for all $f \in \cM$,
    $\ev \circ f = \ev$,
    $$
    f^*(\tilde\alpha) = \tilde\alpha
    \quad \text{for all} \quad
    f \in \cM.
    $$
    The evaluation map $\ev$ defines a morphism of De Rham complex
    $$
    \ev^* \colon \Omega^\star(\X) \to \Omega^\star(\cB)
    $$
    which takes its values in the subspace
    $$
    \Omega^\star_\cM(\cB) = \{ \tilde\alpha \in \Omega^\star(\cB) \mid \forall f \in \cM, \ f^*(\tilde\alpha) = \tilde\alpha \}
    $$
    of smooth forms on $\cB$ invariant by the covariant action of $\cM$.
    Conversely:
    
    \textsc{Proposition.}~\textit{A smooth form $\tilde \alpha \in \Omega^k(\cB)$ is the pullback,
    by the evaluation map,
    of a differential form $\alpha$ on $\X$,
    if and only if it is invariant by the action of the gauge monoid $\cM$,
    that is
    $$
    f^*(\tilde\alpha) = \tilde\alpha \quad \text{for all} \quad f \in \cM.
    $$
    The form $\alpha \in \Omega^k(\X)$ such that $\tilde\alpha = \ev^*(\alpha)$ is then unique.}
    
    In other word,
    the pullback $\ev^*$ realizes a De Rham complex equivalence:
    $$
    \Omega^\star(\X) \simeq \Omega^\star_\cM(\cB).
    $$
    The complex of differential forms on $\X$ is then identified with a subcomplex of the ordinary complex of smooth forms on the bubbles space.
    
    \textsc{Remark.} It is interesting to notice that,
    in regard of the equivalence of the complexes above,
    the space $\X$ itself,
    which is a quotient of the nebula $\cB$ by the evaluation map,
    is also the quotient of $\cB$ by the action of the gauge monoid $\cM$,
    that is,
    $$
    \X \simeq \cB/\cM.
    $$
    Therefore,
    considering the special structure of the nebula $\cB$,
    it is not surprising that $\Omega^\star(\cB/\cM)$ identifies with $\Omega^\star_\cM(\cB)$ as its pullback by the evaluation map $\ev$,
    for which $\cM$ is the gauge transformations monoid.
    
    We need however to clarify what does means this equivalence above.
    The relation between two elements $(\phi,r)$ and $(\phi',r')$,
    defined by the existence of $f \in \cM$ such that $(\phi',r') = f(\phi,r)$,
    is reflexive and transitive but not symmetric.
    The quotient above is defined considering the equivalence relation generated by the symmetric closure of the action of the monoid $\cM$ on $\cB$.
    More precisely,
    $(\phi',r')$ and $(\phi,r)$ are equivalent if there exists a finite chain $(\phi_i,r_i)_{i=1}^\N$ of elements of $\cB$,
    and a sequence $(f_i)_{i=1}^{\N-1}$ of elements of $\cM$,
    such that $(\phi_1,r_1) = (\phi,r)$,
    $(\phi_\N,r_\N) = (\phi',r')$,
    and for all $i = 1\ldots N-1$,
    $(\phi_{i+1},r_{i+1}) = f_i(\phi_i,r_i)$ or $(\phi_i,r_i) = f_i(\phi_{i+1},r_{i+1})$.
    Actually,
    these chains reduce to alternating chains,
    by composing the successive elements of the chain that can be composed.
    Now,
    for all $x \in \X$,
    let $\hat x \in \fB$ with $\hat x \colon \RR^0 \to \X$ such that $\hat x(0) = x$.
    Then,
    for all $(\phi,r)$ such that $\phi(r)=x$,
    there exists $f \colon \{\hat x\} \times \RR^0 \to \{\phi\} \times \dom(\phi)$ mapping $0$ to $r$ and satisfying $\ev \circ f = \ev$.
    Hence,
    all $(\phi,r)$ such that $\phi(r) = x$ are equivalent to $\hat x$,
    and $\X \simeq \cB/\cM$.
  \end{article}
  
  \begin{proof}
    \begin{figure}[t]
      \includegraphics[width=.7\textwidth]{Figures/Form-alpha.pdf}
      %\vspace{-.25\baselineskip}
      \caption{The form $\alpha$ on $\X$.}
      \label{Form-alpha}
    \end{figure}
    Let $\tilde \alpha \in \Omega^k(\cB)$ such that $f^*(\tilde\alpha) = \tilde\alpha$ for all $f \in \cM$.
    Consider a plot $\P \colon \U \to \X$.
    The domain $\U$ admits a locally finite covers $\U = \cup_{i} \B_i$,
    where the $\B_i$ are open balls,
    for some set of indices.
    Hence,
    the restrictions $\phi_i = \P \restriction \B_i$ are compatibe round plots,
    defined on $\B_i$;
    that is,
    they coincide on the intersection of their domains.
    Let then $\alpha(\phi_i) = \tilde\alpha \restriction \{\phi_i\} \times \B_i \in \Omega^k(\B_i)$.
    We shall check first that the $\alpha(\phi_i)$ are compatible and defines a smooth $k$-form on $\U$.
    Let $r_0$ be some point in $\U$,
    since the covering is locally finite,
    the point $r_0$ belongs to only a finite number of balls of the cover.
    Let $r_0 \in \B_i \cap \B_j$,
    and let $\phi_i = \P \restriction \B_i$ and $\phi_j = \P \restriction \B_j$.
    Then,
    there exists a small ball $b \subset \B_i \cap \B_j$ such that $r \in b$,
    let $\psi = \P \restriction b$,
    it is a round plot.
    Let $f_i \colon \{\psi\} \times b \to \{\phi_i\} \times \B_i$ be the inclusion:
    $f_i(\{\psi\},r) = (\{\phi_i\},r)$,
    and $f_i$ is the identity elsewhere.
    Then,
    for all $r \in b$,
    $\ev(f_i(\{\psi\},r)) = \ev(\{\phi_i\},r) = \phi_i(r) = (\P \restriction \B_i)(r) = (\P \restriction b)(r) = \psi(r) = \ev(\{\psi\},r)$.
    Thus,
    $\ev \circ f_i = \ev$.
    Hence $f_i^*(\tilde\alpha \restriction \{\phi_i\} \times \B_i) = \tilde\alpha \restriction \{\psi\} \times b$,
    that is,
    $f_i^*(\alpha(\phi_i)) = \alpha(\phi_i)\restriction b = \tilde\alpha \restriction \{\psi\} \times b$.
    For the same reason,
    we have $\alpha(\phi_j)\restriction b = \tilde\alpha \restriction \{\psi\} \times b$.
    Thus,
    $\alpha(\phi_i) \restriction \B_i \cap \B_j = \alpha(\phi_j) \restriction \B_i \cap \B_j$.
    This property extends naturally to a finite intersection of balls.
    Hence,
    the family of differential forms $\alpha(\phi_i)$ is compatible,
    there is a smooth $k$-form $\alpha(\P)$,
    defined on $\U$,
    such that $\alpha(\phi_i) = \alpha(\P) \restriction \B_i$.
    Note that the form $\alpha(\P)$ does not depend of the covering:
    If we have another locally finite covering by balls $\B'_{i'}$ giving $\alpha'(\P)$,
    then the union of the two coverings is still a locally finite covering by balls $\B''_{i''}$ giving a form $\alpha''(\P)$,
    which by construction is equal to $\alpha(\P)$ and $\alpha'(\P)$.
    Therefore,
    the form $\tilde\alpha$ defines a mapping $\P \to \alpha(\P)$ for all plot $\P$ in $\X$,
    with $\alpha(\P) \in \Omega^k(\dom(\P))$.
    
    Let us check now that this mapping defines a differential $k$-form on $\X$,
    that is,
    $\alpha(\P \circ \F) = \F^*(\alpha(\P))$,
    for all smooth parametrizations $\F$ in $\U$.
    Let $\F \colon \V \to \U$ be a smooth parametrization.
    Let $\B_i$ be a ball of the cover above.
    Let $\V' = \F^{-1}(\B_i)$.
    The subset $\V'$ is an Euclidean domain and admits a locally finite covering by balls.
    Let $b$ be a ball of that covering.
    On the nebula,
    consider the map $\F_b \colon \{\P \circ \F \restriction b\}\times b \to \{\P \restriction \B_i\} \times \B_i$,
    such that $\F_b(\P \circ \F \restriction b, s) = (\P \restriction \B_i,F(s))$.
    One has,
    $\ev \circ \F_b(\P \circ \F \restriction b, s) = \ev(\P \restriction \B_i,F(s)) = \P(\F(s)) = \ev(\P \circ \F \restriction b, s)$.
    Thus,
    $\F_b^*(\tilde\alpha \restriction (\P \restriction \B_i)) = \tilde\alpha \restriction (\P \circ \F \restriction b)$,
    that is,
    $\F^*(\alpha(\P\restriction \B_i) = \alpha(\P \circ \F \restriction b)$.
    Because of the locality of pullback,
    and the locality of smooth forms on Euclidean domains,
    we get $\F^*(\alpha(\P)) = \alpha(\P \circ \F)$.
    Therefore,
    $\alpha$ defined above  is a differential form on $\X$.
    The differential form $\alpha$ associated with $\tilde\alpha$ is unique,
    because the evaluation map $\ev$ is a subduction \cite[\textsection 6.38 \& 6.39]{PIZ13}.
    %
    Note that,
    for a real function $\tilde f \colon \cB \to \RR$,
    invariant by $\cM$,
    this is even simpler.
    The function $f$ is defined on $\X$ by $f(x) = \tilde f(\hat x)$,
    where $\hat x \colon \{0\} \to \X$ is the plot $\hat x(0) = x$.
  \end{proof}
  
  %% MARK: • Cech Cohomology Of A Diffeological Space
  \begin{article}\artlabel[\Cech Cohomology Of A Diffeological Space]
    \label{Cech-Cohomology-Of-A-Diffeological-Space}
    Let $\X$ be a diffeological space.
    Let $\cB$ be its nebula of round plots.
    Let $\cM$ be the gauge monoid of the evaluation map $\ev \colon \cB \to \X$.
    Let $\R$ be an Abelian group and let
    $$
    \A = \Maps(\fB,\R).
    $$
    be the group of maps from the round plots to $\R$.
    It is convenient to see $\A$ as the group of locally constant maps from the nebula $\cB$ to $\R$
    \[
    \A = \{ \tau \colon \cB \to \R \mid \forall \phi \in \fB, \exists c_\phi \in \R, \tau \restriction \{\phi\} \times \dom(\phi) = c_\phi \},
    \]
    where $\{\phi\} \times \dom(\phi)$ is a component of $\cB$ and $c_\phi$ some element in $\R$.
    
    We denote by $\vC^k(\X,\R)$ the group of \Cech $k$-cochains on $\X$ with values in $\R$,
    \begin{equation}\tag{$\diamondsuit$}
      \vC^k(\X,\R) = \Maps(\cM^k, \A) = \Maps\big(\cM^k, \Maps(\fB,\R)\big).
    \end{equation}
    
    \textsc{Definition.} \textit{We define the $k$-th group of \Cech Cohomology of $\X$ with values in the Abelian group $\R$,
    as the $k$-th Hochschild cohomology group of the monoid $\cM$ with values in $\A$.
    We denote
    \begin{equation}\tag{$\heartsuit$}
      \vH^k(\X,\R) = \sH^k(\cM,\A)
      \quad \text{and} \quad
      \vH^\star(\X,\R) = \sH^\star(\cM,\A)
    \end{equation}
    }
    \textsc{Note.} It is possible to understand immediately the $\vH^0(\X,\RR)$ for any diffeological space.
    The \Cech $0$-cochains is the set $\vC^0(\X,\RR) = \Maps(\fB,\RR)$.
    That is,
    the real constant maps $c$ from the components of the nebula $\cB$ to $\RR$.
    Now,
    belonging to $\vH^0(\X,\RR)$ means $\delta c(f)=0$,
    that is,
    $f^*(c) = c$,
    which means that $c = \ev^*(s)$ with $s \in \Cinfty(\X,\RR)$ and $ds=0$.
    Hence,
    $\vH^0(\X,\RR)$ is the set of local constant real functions on $\X$,
    that is,
    \[
    \vH^0(\X,\RR) = \Maps(\pi_0(\X),\RR).
    \]
  \end{article}
  
  %% MARK: • Cech Cohomology Of Tori
  \begin{article}\artlabel[\Cech Cohomology Of Tori]
    \label{Cech-Cohomology-Of-Tori}
    We call \emph{diffeologial torus} or simply \emph{torus} any quotient $\T_\K = \RR^n/\K$,
    where $\K \subset \RR^n$ is a (diffeologically) discrete subgroup,
    that is,
    its subset diffeology is discrete
    ---~generated by constant parametrizations.
    For what follows,
    we will not need that $\K$ generates $\RR^n$,
    but obviously the term torus will be an abuse of language when it will not be the case.
    We say that $\T_\K$ is an \emph{irrational torus} when the subgroup $\K$ generates $\RR^n$ but is strictly larger than a lattice.
    Irrational tori have been the first case of significant diffeological spaces,
    with the case $\K = \ZZ + \alpha \ZZ \subset \RR$ and $\alpha \in \RR-\QQ$ \cite{DonIgl83}.
    
    Then,
    $\T_\K$ is a connected diffeological space,
    the projection
    $$
    \pi \colon \RR^n \to \T_\K,
    $$
    is the universal covering of $\T_\K$ \cite[\textsection 8.26]{PIZ13},
    and $\K$ is the first group of homotopy,
    its \emph{fundamental group} \cite[\textsection 5.15]{PIZ13},
    $$
    \tilde \T_\K = \RR^n
    \quad \text{and} \quad
    \pi_1(\T_\K) = \K.
    $$
    Since $\T_\K$ is connected,
    $$
    \vH^0(\T_\K,\RR) = \RR.
    $$
    In order to compute the \Cech cohomology of $\T_\K$,
    we remark that,
    since the plots $\phi \in \fB$ are defined on balls,
    thanks to the monodromy theorem \cite[\textsection 8.25]{PIZ13},
    there exists a smooth map $f \colon \B \to \RR^n$ that lifts $\phi \colon \B \to \T_\K$.
    That is,
    $$
    \phi = \pi \circ f.
    $$
    This map is unique as soon as we fix the image of one point in $\B$.
    The set of liftings of $\phi$ is indexed by $\K$.
    Now,
    let $\F \in \cM$,
    we denote by the same letter the action of $\F$ on $\cB$ and its action on the set of component $\pi_0(\cB) = \fB$.
    In other words for all $(\phi,r) \in \cB$,
    let
    $\F(\phi,r) = (\phi', r')$.
    That is,
    $$
    \F(\phi) = [\phi' \colon \B' \to \T_\K]
    \quad\text{with}\quad
    \phi' \circ \F = \phi.
    $$
    Let us pick a lifting $f'$ for $\phi'$.
    We have then,
    $$
    \phi' \circ \F = \phi, \quad \phi = \pi \circ f, \quad \phi' = \pi \circ f'.
    $$
    Hence,
    $$
    \pi \circ f' \circ \F = \pi \circ f \quad \Rightarrow \quad \exists k \in \K, \quad f' \circ \F = f + k,
    $$
    and this $k$ is unique.
    
    \begin{figure}[ht]
      \[
      \begin{tikzcd}[column sep=large,every label/.append style = {font = \small}]
        \B \arrow[rr, "\F"] \arrow[dd,swap, "f"] \arrow[dr, "\phi"] &       &  \B' \arrow[dd, "f'"] \arrow[dl, swap, "\phi'"]  \\
        & \T_\K &  \\
        \RR^n \arrow[rr, swap, "k"] \arrow[ur, swap, "\pi"]                  &       &  \RR^n \arrow[ul, "\pi"]
      \end{tikzcd}
      \]
      \vspace{-.5cm}
      \caption{Element of the Gauge Monoid of $\T_\K$.}
      \label{cohomology-of-the-torus}
    \end{figure}
    
    These relations are represented by the commutative diagram of Fig.\ref{cohomology-of-the-torus}.
    Now,
    choosing for each $\phi \in \fB$ a lifting $f$ in $\RR$,
    defines a map
    $$
    \kappa : \cM \to \Maps(\fB,\K) \quad\text{by}\quad k = \kappa(\F)(\phi).
    $$
    Note this:
    
    \begin{itemize}
    \item[--] The elements $k \in \K$ belong to $\cM$,
    since $\pi \colon \RR^n \to \T_\K$ is a round plot and $\pi \circ k = \pi$,
    that is,
    $k_\fB(\pi) = \pi$ and $k_\pi(x) = x+k$.
    \item[--] The liftings $f$ also belong to $\cM$ with $f(\phi,r) = (\pi, f(r))$,
    that is:
    $f_\fB(\phi) = \pi$ and $f_\phi(r) = f(r)$.
    \end{itemize}
    
    \textsc{Proposition.} {\it For every $p$-cocycle $\sigma \in \vZ^p(\T_\K,\RR)$,
    there exists a $(p-1)$-cochain $\epsilon$ such that
    $$
    \sigma(\F_1,\dots,\F_p)(\phi) = \sigma(k_1,\dots,k_p)(\pi) + \delta\epsilon(\F_1,\dots,\F_p)(\phi),
    $$
    with $\F_i \colon \phi_i \to \phi_{i+1}$,
    $\phi_1=\phi$,
    $k_i = \kappa(\F_i)(\phi_i)$,
    that is,
    $f_{i+1} \circ \F_i = f_i + k_i$,
    as it is shown by the diagram of Figure~\ref{Cocycle-Morphism-For-TK}.
    }
    %
    \begin{figure}[ht]
      \[
      \begin{tikzcd}[column sep=normal,every label/.append style = {font = \small}]
        \phi_1 \arrow[r, "\F_1"] \arrow[d,swap, "f_1"] & \phi_2 \arrow[r, "\F_2"] \arrow[d,swap, "f_2"] & \phi_3 \arrow[d, "f_3"] \arrow[r, "\F_3"] & \dots \arrow[r, "\F_{p-1}"] & \phi_p \arrow[d, swap, "f_p"] \arrow[r, "\F_p"] & \phi_{p+1} \arrow[d, "f_{p+1}"] \\
        \RR^n \arrow[r, swap, "k_1"] &  \RR^n \arrow[r, swap, "k_2"] & \RR^n \arrow[r, swap, "k_3"] & \dots \arrow[r,swap, "k_{p-1}"] & \RR^n \arrow[r, swap, "k_p"] & \RR^n
      \end{tikzcd}
      \]
      \vspace{-.5cm}
      \caption{Cocycle Morphism For $\T_\K$.}
      \label{Cocycle-Morphism-For-TK}
    \end{figure}
    
    \textsc{Corollary.} {\it The mapping that associates with all $\sigma \in \vC^p(\T_\K,\RR)$,
    its restriction $\sigma \restriction \K^p$ which belongs to $\C^p(\K,\RR) = \Maps(\K^p,\RR)$,
    defines a morphism in cohomology from $\vH^p(\T_\K, \RR)$ to $\sH^p(\K,\RR)$,
    which is an isomorphism:
    $$
    \vH^\star(\T_\K, \RR) \simeq \sH^\star(\K,\RR).
    $$
    Where $\sH^\star(\K,\RR)$ is the ordinary cohomology of the group $\K$ with real coefficients.}
  \end{article}
  
  \begin{proof}
    \begin{figure}[ht]
      \includegraphics[width=.6\textwidth]{Figures/Epsilon.pdf}
      %\vspace{-.33\baselineskip}
      \caption{Representing $\epsilon(\F_1,\dots,\F_{p-1})(\phi)$.}
      \label{Epsilon}
    \end{figure}
    Let $\sigma \in \vC^p(\T_\K,\RR)$.
    Let $[\F_i \colon \phi_i \to \phi_{i+1}] \in \cM$,
    with $\phi_1=\phi$, $k_i = \kappa(\F_i)(\phi_i)$ and the liftings $f_i \colon \B_i \to \RR^n$ such that $f_{i+1} \circ \F_i = f_i + k_i$,
    for $i=1,\dots,p$.
    Let us consider the sum
    $$
    %\S_p = \sum_{i=0}^{p} (-1)^i \, \delta\sigma([\F_1,F_{p-i}],f_{p-i+1},[k_{p-i+1},k_p])(\phi)
    \S_p = \sum_{i=0}^{p} (-1)^i \, \delta\sigma(\F_1,\dots,F_{p-i},f_{p-i+1},k_{p-i+1},\dots,k_p)(\phi)
    $$
    This sum begins with $\delta\sigma(\F_1,\dots,\F_p,f_{p+1})$ and ends with $(-1)^p\delta\sigma(f_1,k_1,\dots,k_p)$.
    Let us define $\epsilon \in \vC^{p-1}(\T_\K,\RR)$,
    by
    $$
    %\epsilon(\F_1,\dots,\F_{p-1})(\phi) = \sum_{i=0}^{p-1}(-1)^{i}\sigma([\F_1,\F_{p-1-i}],f_{p-i},[k_{p-i},k_{p-1}])(\phi),
    \epsilon(\F_1,\dots,\F_{p-1})(\phi) = \sum_{i=0}^{p-1}(-1)^{i}\sigma(\F_1,\dots,\F_{p-1-i},f_{p-i},k_{p-i},\dots,k_{p-1})(\phi),
    $$
    for $p > 1$,
    and for $p=1$\,:
    $$
    \epsilon \in \Maps(\fB,\RR) \quad\text{with}\quad \epsilon(\phi) = \sigma(f_1),
    $$
    with $\phi = \phi_1$ and $\pi \circ f_1 = \phi_1$.
    
    We shall show firstly that:
    $$
    \S_p = (-1)^{p+1}[\sigma(\F_1,\dots,\F_p)(\phi) - \sigma(k_1,\dots,k_p)(\pi)] + \delta\epsilon(\F_1,\dots,\F_p)(\phi).
    $$
    We shall begin by computing the example with $p=3$ to identify the recursivity of the computation of $\S_p$.
    But before that,
    we will change our notation and denote $\ul{k}_{i+1} = k_i$,
    then the commutativity of the squares in the diagram of Figure \ref{Cocycle-Morphism-For-TK} becomes $f_{i+1} \circ \F_i = \ul{k}_{i+1} \circ f_i$.
    In the definition of $\S_p$ above we will omit the prefix $\delta\sigma$ and the suffix $(\phi)$.
    And we will shall encode the sequences $\F_a,\dots,F_b,f_c,k_d,\dots,k_p$ as follows:
    
    \begin{itemize}
      \item[--] The $\F_i$ will be denote by their index $i$,
      $\F_1$ becomes $1$,
      etc.
      \item[--] The commas will be replaced by centered periods,
      for example: $\F_1, \F_2,\dots$ will be replaced by $1 \cdot 2 \cdots$.
      \item[--] The only element $f_\ell$ will be replaced by $[\ell]$.
      \item[--] The elements $\ul{k}_j$ will be replaced by their underlined index,
      for example $\ul{k}_3$ will be written $\ul3$.
      \item[--] The composite $F_{i+1} \circ \F_i$ will be replaced by $(i \cdot i+1)$.
      \item[--] The composite $f_{i+1} \circ \F_i$ will be replaced by $(i \cdot i+1]$.
      \item[--] The composite $\ul{k}_{i+1} \circ f_i$ will be replaced by $[i \cdot \ul{i+1})$.
      \item[--] The composite $\ul{k}_{i+1} \circ \ul{k}_{i}$ will be replaced by $(\ul{i} \cdot \ul{i+1})$.
    \end{itemize}
    
    The crossed-out index means that the index is omitted and the previous notations are completed by:
    
    \begin{itemize}
      \item[--] The monomial $\cancel 1 \cdot 2 \cdot 3 \cdot [4]\cdot(1)$ represents $\sigma(\F_2,\F_3,f_4)(\F_1(\phi) = \phi_2)$.
      \item[--] The monomial $\cancel{[1]}\cdot \ul{2} \cdot \ul{3} \cdot \ul{4} \cdot[1]$ represents $\sigma(\ul{k}_2,\ul{k}_3,\ul{k}_4)(f_1(\phi) = \pi)$.
    \end{itemize}
    
    In particular $\epsilon(\F_1,\dots,\F_{p-1})(\phi)$ is represented by the diagram in Figure \ref{Epsilon}.
    
    Then,
    the development of $S_3$ becomes then:
    $$
    \begin{array}{c@{\hspace{2pt}}c@{\hspace{2pt}}c@{\hspace{2pt}}c@{\hspace{2pt}}c@{\hspace{2pt}}c@{\hspace{2pt}}c@{\hspace{2pt}}c@{\hspace{2pt}}c@{\hspace{2pt}}c}
      + & \cancel 1\cdot 2\cdot 3\cdot [4]\cdot(1) & - & (1\cdot 2)\cdot 3\cdot [4] & + & 1\cdot (2\cdot 3)\cdot[4] & - & 1\cdot 2\cdot (3\cdot 4]   & + & 1\cdot 2\cdot 3 \\[.25cm]
      - & \cancel 1\cdot 2\cdot [3]\cdot \ul4\cdot(1) & + & (1\cdot 2)\cdot [3]\cdot \ul4 & - & 1\cdot (2\cdot 3]\cdot\ul4   & + & 1\cdot 2\cdot [3\cdot \ul4)   & - & 1\cdot 2\cdot [3] \\[.25cm]
      + & \cancel 1\cdot [2]\cdot \ul3\cdot \ul4\cdot(1) & - & (1\cdot 2]\cdot \ul3 \cdot \ul4  & + & 1\cdot [2 \cdot \ul3)\cdot\ul4  & - & 1\cdot [2]\cdot (\ul3\cdot \ul4) & + & 1\cdot [2]\cdot \ul3 \\[.25cm]
      - & \cancel{[1]}\cdot \ul2\cdot \ul3\cdot \ul4\cdot[1] & + & [1\cdot \ul2)\cdot \ul3 \cdot \ul4  & - & [1] \cdot (\ul2 \cdot \ul3)\cdot\ul4 & + & [1]\cdot \ul2\cdot (\ul3\cdot \ul4)& - & [1]\cdot \ul2\cdot \ul3
    \end{array}
    $$
    %
    According to our notations,
    the commutativity $f_{i+1} \circ \F_i = \ul{k}_{i+1} \circ f_i$ becomes $(i \cdot i+1] = [i\cdot \ul{i+1})$.
    Then,
    we cancel the two by two opposite terms:
    %
    $$
    \begin{array}{c@{\hspace{2pt}}c@{\hspace{2pt}}c@{\hspace{2pt}}c@{\hspace{2pt}}c@{\hspace{2pt}}c@{\hspace{2pt}}c@{\hspace{2pt}}c@{\hspace{2pt}}c@{\hspace{2pt}}c}
      + & \cancel 1\cdot 2\cdot 3\cdot [4]\cdot(1)                     & - & (1\cdot 2)\cdot 3\cdot [4]                   & + & 1\cdot (2\cdot 3)\cdot[4]                 & - & \cancel{1\cdot 2\cdot (3\cdot 4]}    & + & \fbox{$1\cdot 2\cdot 3$} \\[.25cm]
      - & \cancel 1\cdot 2\cdot [3]\cdot \ul4\cdot(1)                  & + & (1\cdot 2)\cdot [3]\cdot \ul4                & - & \xcancel{1\cdot (2\cdot 3]\cdot\ul4}      & + & \cancel{1\cdot 2\cdot [3\cdot \ul4)} & - & 1\cdot 2\cdot [3] \\[.25cm]
      + & \cancel 1\cdot [2]\cdot \ul3\cdot \ul4\cdot(1)               & - & \bcancel{(1\cdot 2]\cdot \ul3 \cdot \ul4}    & + & \xcancel{1\cdot [2 \cdot \ul3)\cdot\ul4}  & - & 1\cdot [2]\cdot (\ul3\cdot \ul4)     & + & 1\cdot [2]\cdot \ul3 \\[.25cm]
      - &  \fbox{$\cancel{[1]}\cdot \ul2\cdot \ul3\cdot \ul4\cdot[1]$} & + & \bcancel{[1\cdot \ul2)\cdot \ul3 \cdot \ul4} & - & [1] \cdot (\ul2 \cdot \ul3)\cdot\ul4      & + & [1]\cdot \ul2\cdot (\ul3\cdot \ul4)  & - & [1]\cdot \ul2\cdot \ul3
    \end{array}
    $$
    Then,
    we pull upwards and to the left, one degree, the terms that are below the diagonal of the canceled terms.
    That gives the following table:
    $$
    \begin{array}{c@{\hspace{2pt}}c@{\hspace{2pt}}c@{\hspace{2pt}}c@{\hspace{2pt}}c@{\hspace{2pt}}c@{\hspace{2pt}}c@{\hspace{2pt}}c@{\hspace{2pt}}c@{\hspace{2pt}}c}
      + & \cancel 1\cdot 2\cdot 3\cdot [4]\cdot(1)                     & - & (1\cdot 2)\cdot 3\cdot [4]              & + & 1\cdot (2\cdot 3)\cdot[4]             & - & 1\cdot 2\cdot [3]        & + & \fbox{$1\cdot 2\cdot 3$} \\[.25cm]
      - & \cancel 1\cdot 2\cdot [3]\cdot \ul4\cdot(1)                  & + & (1\cdot 2)\cdot [3]\cdot \ul4           & - & 1\cdot [2]\cdot (\ul3\cdot \ul4)      & + & 1\cdot [2]\cdot \ul3     &   &  \\[.25cm]
      + & \cancel 1\cdot [2]\cdot \ul3\cdot \ul4\cdot(1)               & - & [1] \cdot (\ul2 \cdot \ul3)\cdot\ul4    & + & [1]\cdot \ul2\cdot (\ul3\cdot \ul4)   & - &  [1]\cdot \ul2\cdot \ul3 &   &  \\[.25cm]
      - &  \fbox{$\cancel{[1]}\cdot \ul2\cdot \ul3\cdot \ul4\cdot[1]$} &  & &  &      &  &   &  &
    \end{array}
    $$
    Beginning with the framed elements,
    we get:
    $$
    + 1\cdot 2\cdot 3 - \cancel{[1]}\cdot \ul2\cdot \ul3\cdot \ul4\cdot[1] = \sigma(\F_1,\F_2,\F_3)(\phi) - \sigma(k_1,k_2,k_3)(\pi).
    $$
    Next,
    from left to right and top to bottom,
    the three columns give:
    \begin{eqnarray*}
      \allowdisplaybreaks
      \text{Column 1} & = &  + \cancel 1\cdot 2\cdot 3\cdot [4]\cdot(1) - \cancel 1\cdot 2\cdot [3]\cdot \ul4\cdot(1) + \cancel 1\cdot [2]\cdot \ul3\cdot \ul4\cdot(1)\\[.2cm]
      & = & + \sigma(\F_2,\F_3,f_4)(\phi_2) - \sigma(\F_2,f_3,\ul k_4)(\phi_2) + \sigma(f_2,\ul k_3,\ul k_4)(\phi_2)\\[.2cm]
      & = & + \epsilon (\F_2,\F_3)(\F_1(\phi))\\[.2cm]
      \text{Column 2} & = &  - (1\cdot 2)\cdot 3\cdot [4] + (1\cdot 2)\cdot [3]\cdot \ul4 - [1] \cdot (\ul2\cdot \ul3)\cdot\ul4 \\[.2cm]
      & = & - \sigma(\F_2\circ \F_1,\F_3,f_4)(\phi) + \sigma(\F_2 \circ \F_1,f_3, \ul k_4)(\phi) - \sigma(f_1,\ul k_3 \circ \ul k_2 ,\ul k_4)(\phi)\\[.2cm]
      & = & - \epsilon(\F_2 \circ \F_1,\F_3)(\phi) \\[.2cm]
      \text{Column 3} & = &  + 1 \cdot (2 \cdot 3) \cdot [4] - 1 \cdot [2] \cdot (\ul3 \cdot \ul4) + [1] \cdot \ul2 \cdot (\ul3\cdot \ul4) \\[.2cm]
      & = & + \sigma(\F_1,\F_3\circ\F_2,f_4)(\phi) - \sigma(\F_1,f_2,\ul k_4 \circ \ul k_3)(\phi) + \sigma(f_1,\ul k_2, \ul k_4 \circ \ul k_3)(\phi).\\[.2cm]
      & = & + \epsilon(\F_1,\F_3 \circ \F_2)(\phi) \\[.2cm]
      \text{Column 4} & = &  - 1 \cdot 2 \cdot [3] + 1 \cdot [2] \cdot \ul3 - [1] \cdot \ul2 \cdot \ul3 \\[.2cm]
      & = & - \sigma(\F_1,\F_2,f_3)(\phi) + \sigma(\F_1, f_2, \ul k_3)(\phi) - \sigma(f_1,\ul k_2 ,\ul k_3)(\phi) \\[.2cm]
      & = & - \epsilon(\F_1,\F_2)(\phi).
    \end{eqnarray*}
    Thus,
    $$
    \text{Columns 1 + 2 + 3 + 4} = \delta\epsilon(\F_1,\F_2,\F_3).
    $$
    And then:
    $$
    \S_3 = \sigma(\F_1,\F_2,\F_3)(\phi) - \sigma(k_1,k_2,k_3)(\pi) + \delta \epsilon(\F_1,\F_2,\F_3).
    $$
    Now, if $\sigma$ is a cocycle:
    $\delta\sigma = 0$,
    then we get
    $$
    \sigma(\F_1,\F_2,\F_3)(\phi) = \sigma(k_1,k_2,k_3)(\pi) - \delta \epsilon(\F_1,\F_2,\F_3)
    $$
    This particular case reveals the general computation,
    leading to the general formula above,
    which is outlined in the Figure \ref{Cech-computation}.
    The letters represent successive integers.
    The diagram on the right represent the sequence of elements of the monoid that occur in the evaluation of $\delta\epsilon$.
    
    \begin{figure}[htp]
      \includegraphics[width=.95\textwidth]{Figures/Cech-computation-v2.pdf}
      \vspace{-.25\baselineskip}
      \caption{Computing $\S_p$.}
      \label{Cech-computation}
    \end{figure}
    
    Consider now the restriction operation $\sigma \mapsto \sigma \restriction \K^p$,
    with $\sigma \in \vC^p(\T_\K,\RR)$.
    It commutes with the differential:
    $(\delta \sigma) \restriction \K^p = \delta (\sigma \restriction \K^p)$.
    The restriction $\sigma \restriction \K^p$ is a $p$-cochain of the group $\K$ with coefficients in $\RR$,
    and $\delta$ is exactly the coboundary operator for the group cohomology with real coefficient.
    Thus,
    $\sigma \mapsto \sigma \restriction \K^p$ gives a well defined morphism,
    $\class \colon \vH^p(\T_\K,\RR) \to \sH^p(\K, \RR)$.
    
    Let us check that $\class$ is surjective.
    Let $s \in \sZ^p(\K,\RR)$, and $\phi \in \fB$.
    We define
    $$
    \sigma(\F_1,\dots,\F_p)(\phi) = s(k_1,\dots, k_p)(\pi),
    $$
    with $\F_i \colon \phi_i \to \phi_{i+1}$,
    $\phi_1=\phi$,
    $k_i = \kappa(\F_i)(\phi_i)$,
    The cocycle property can be regarded as the juxtaposition of diagrams of type Fig.\ref{cohomology-of-the-torus}.
    
    Let us check that $\class$ is injective.
    Let $\sigma \in \vZ^p(\T_\K,\RR)$ such that $\class(\sigma) = 0$,
    that is,
    $\sigma \restriction \K^p = \delta s$,
    with $s \in \Maps(\K^{p-1},\RR)$.
    But,
    for all $(\F_1,\dots,\F_p)(\phi)\in \cM^p \times \fB$,
    $\sigma(\F_1,\dots,\F_p)(\phi) = (\sigma\restriction \K^p)(k_1,\dots,k_p)(\pi) + \delta \epsilon(\F_1,\dots,\F_p)$.
    Then,
    \begin{eqnarray*}
      \sigma(\F_1,\dots,\F_p)(\phi) & = & \delta s(k_1,\dots,k_p)(\pi) + \delta \epsilon(\F_1,\dots,\F_p) \\
      & = & \delta[s\circ \kappa^p + \epsilon](\F_1,\dots,\F_p)(\phi).
    \end{eqnarray*}
    Thus,
    $\class(\sigma) = 0$.
    Therefore,
    we conclude that,
    for all $p>0$,
    $\vH^p(\T_\K,\RR) = \sH^p(\K, \RR)$.
    In particular,
    for $p=1$,
    that coincides with $\Hom(\pi_1(\T_\K), \RR)$,
    a prior natural identification \cite{PI88-91}.
  \end{proof}
  
  %%%%%%%%%%%%%%%%%%%%%%%%%%%%%%%%%%%%%%%%%%%%%%%%%%%%%%%%%%
  %%
  %% MARK: - § Cech-De-Rham Cohomology Bicomplex
  %%
  %%%%%%%%%%%%%%%%%%%%%%%%%%%%%%%%%%%%%%%%%%%%%%%%%%%%%%%%%%
  \section{\Cech-De-Rham Cohomology Bicomplex}
  
  The relation between the De Rham cohomology and the \v{C}hech cohomology,
  defined above,
  is described by a differential bicomplex,
  also called a double complex.
  
  %% MARK: • The Cech-De-Rham Bicomplex
  \begin{article}\artlabel[The \Cech-De-Rham Bicomplex]
    Let $\X$ be a diffeological space.
    Let $\cB$ be its nebula of round plots.
    Let $\cM$ be the gauge monoid of the evaluation map $\ev \colon \cB \to \X$.
    Let $\Omega^k(\X)$ be the space of differential $k$-forms on $\X$,
    identified with the gauge invariant $k$-forms on $\cB$,
    as it has been established in \art{Differential-forms-in-diffeology}.
    $$
    \Omega^k(\X) = \{ \alpha \in \Omega^k(\cB) \mid f^*(\alpha) = \alpha, \forall f \in \cM \}.
    $$
    We define now the group of $p$-cochain of the monoid $\cM$ with values in the vector space $\Omega^q(\cB)$,
    and we denote
    $$
    \sC^{p,q} = \sC^p(\cM,\Omega^q(\cB)) = \Maps(\cM^p,\Omega^q(\cB)).
    $$
    And then,
    we get the bicomplex denoted by
    $$
    \sC^{\star,\star} = \bigoplus_{p,q \in \NN} \sC^{p,q},
    $$
    equipped with the differential operators $d$ and $\delta$ (see Figure \ref{bicomplex-CDR}),
    $$
    \left\{
    \begin{array}{lcl}
      d   & : & \sC^{p,q}\rightarrow \sC^{p,q+1}\\
      \delta  & : & \sC^{p,q}\rightarrow \sC^{p+1,q},
    \end{array}
    \right.
    $$
    
    \begin{figure}[th]
      $$
      \begin{tikzcd}[column sep=small,every label/.append style = {font = \small}]
        & \vdots                                            & \vdots                                          &   \vdots                                        &         \\
        (q=2) & \sC^{0,2} \arrow[u, "d"] \arrow[r, "\delta"] & \sC^{1,2} \arrow[u, "d"] \arrow[r,"\delta"] & \sC^{2,2} \arrow[u, "d"] \arrow[r,"\delta"] & \cdots  \\
        (q=1) & \sC^{0,1} \arrow[u, "d"] \arrow[r, "\delta"]  & \sC^{1,1} \arrow[u, "d"] \arrow[r,"\delta"] & \sC^{2,1} \arrow[u, "d"] \arrow[r,"\delta"] & \cdots  \\
        (q=0) & \sC^{0,0} \arrow[u, "d"] \arrow[r, "\delta"]  & \sC^{1,0} \arrow[u, "d"] \arrow[r,"\delta"] & \sC^{2,0} \arrow[u, "d"] \arrow[r,"\delta"] & \cdots  \\[-.5cm]
        & (p=0)                                             & (p=1)                                           & (p=2)                                           & \cdots
      \end{tikzcd}
      $$
      \vspace{-.5cm}
      \caption{\Cech-De-Rham Bicomplex}
      \label{bicomplex-CDR}
    \end{figure}
    
    The operator $d$ on $\sC^{p,q}$ is defined by:
    $$
    d\sigma(f_1,\dots,f_p) = (-1)^p d_\dR\big[\sigma(f_1,\dots,f_p)\big],
    $$
    where $d_\dR$ is the ordinary exterior differential.
    The operator $\delta$ has been defined above \art{Hochschild-Cohomology}.
    
    Now,
    since the two operators $d_\dR$ and $\delta$ commute,
    we get
    $$
    d \circ \delta + \delta \circ d = 0.
    $$
    
    \textsc{Definition.} \textit{The complex $\sC^{\star,\star}$ defined above,
    equipped with the two differential operators $d$ and $\delta$,
    becomes a \emph{bicomplex},
    according to S. Mac Lane \cite[\textsection XI.6]{Mac67}.
    We call it the \emph{\Cech-De-Rham Complex} of the diffeological space $\X$.
    The total differential operator is $\D =d+\delta$,
    and satisfies $\D \circ \D = 0$.}
    
    We shall discuss,
    in the following,
    the exact relationship there exists between the De Rham and \Cech cohomologies,
    thanks to this double complex.
  \end{article}
  
  %% MARK: • The Cech-De-Rham Homomorphisms
  \begin{article}\artlabel[The \Cech-De-Rham Homomorphisms]
    We consider first the case $k=0$.
    Let us remind that
    $$
    \left\{
    \begin{array}{rcl}
      \H^0_\dR(\X)  & = &  \{ s \in \Cinfty(\X, \RR) \mid ds = 0 \}, \\[.5ex]
      \vH^0(\X,\RR) & = & \{ \sigma \in \Maps(\fB,\RR) \mid \delta \sigma = 0 \}.
    \end{array}
    \right.
    $$
    Thus,
    any $s \in \Cinfty(\X, \RR)$ such that $ds = 0$ lifts on $\cB$ by a smooth function $\sigma$ that satisfies $d\sigma = 0$ and $\delta\sigma = 0$.
    Thus,
    $\sigma \in \vH^0(\X,\RR)$.
    That defines the De Rham homomorphism $h_0 \colon s \mapsto \sigma$,
    from $\H^0_\dR(\X)$ to $\vH^0(\X,\RR)$.
    Conversely,
    every $\sigma \in \Maps(\fB,\RR)$ is a smooth function on $\cB$ such that $d\sigma = 0$,
    if moreover $\delta\sigma = 0$ then $\sigma$ descends on $\X$ into a smooth function $s$ that also satisfies $ds=0$.
    Therefore,
    
    \textsc{Proposition.} {\it The first De Rham homomorphism is an isomorphism,
    what we summarize by the exact sequence:
    $$
    \begin{tikzcd}[column sep=small,every label/.append style = {font = \small}]
      0 \arrow[r] & \H_\dR^0(\X) \arrow[r,"h_0"] & \vH^0(\X,\RR) \arrow[r] & 0.
    \end{tikzcd}
    $$
    }
    %
    We consider now the case $k \geq 1$,
    and we shall define the series of \emph{\v{C}ech-De-Rham Homomorphisms}
    $$
    h_k \colon \H^k_\dR(\X) \to \vH^k(\X,\RR).
    $$
    Let
    $$
    \Z^{p,q} = \ker\big(d \colon \sC^{p,q}\to \sC^{p,q+1} \big) \cap \ker\big(\delta \colon \sC^{p,q} \to \sC^{p+1,q} \big),
    $$
    and
    $$
    \left\{
    \begin{array}{rcll}
      \B^{p,q} & = & d\big[\delta(\sC^{p-1,q-1}) \big] = \delta\big[d\big(\sC^{p-1,q-1}) \big], & p,q \geq 1 \\[.25cm]
      \B^{0,q} & = & d\big[\ker\delta\big(\sC^{0,q-1} \to \sC^{1,q-1}) \big], & q \geq 1 \\[.25cm]
      \B^{p,0} & = & \delta\big[\ker d\big(\sC^{p-1,0} \to \sC^{p-1,1})\big], & p \geq 1 \\[.25cm]
      \B^{0,0} & = & \{0\}.
    \end{array}\right.
    $$
    Since $\B^{p,q} \subset \Z^{p,q}$,
    we define
    $$
    \K^{p,q} = \Z^{p,q}/\B^{p,q}.
    $$
    We can already notice that,
    $$
    \K^{0,q} = \H_\dR^q(\X) \quad \text{and} \quad \K^{p,0} = \vH^p(\X,\RR).
    $$
    
    Now,
    the homomorphism $h_k$ will be built by using a descending staircase homomorphism.
    $$
    h_{p,q} \colon \K^{p,q} \to \K^{p+1,q-1},
    \quad \forall p \geq 0, \ q \geq 1.
    $$
    Consider an element $\kappa_{p,q} \in \K^{p,q}$,
    and let $\alpha_{p,q}$ representing $\kappa_{p,q}$,
    $$
    \kappa_{p,q} = \class(\alpha_{p,q}), \quad \alpha_{p,q} \in \sC^{p,q}
    \quad \text{with}\quad
    d\alpha_{p,q} = 0
    \quad \text{and}\quad
    \delta\alpha_{p,q} = 0.
    $$
    Let us recall that $\sC^{p,q} = \Maps(\cM^p,\Omega^q(\cB))$,
    and since every closed form on $\cB$ is exact:
    $$
    \exists \alpha_{p,q-1} \in \sC^{p,q-1}
    \quad \text{such that}\quad
    \alpha_{p,q} = d \alpha_{p,q-1}.
    $$
    Then,
    the cochain
    $$
    \alpha_{p+1,q-1} = \delta\alpha_{p,q-1}
    $$
    satisfies
    $$
    \alpha_{p+1,q-1} \in \sC^{p+1,q-1}
    \quad \text{with}\quad
    d\alpha_{p+1,q-1} = 0
    \quad \text{and}\quad
    \delta\alpha_{p+1,q-1} = 0.
    $$
    \textsc{Proposition.} {\it The class $\kappa_{p+1,q-1} = \class(\alpha_{p+1,q-1}) \in \K^{p+1,q-1}$ depends only on $\kappa_{p,q}$,
    and not of the choice or the representant $\alpha_{p,q}$,
    neither on the choice of the primitive $\alpha_{p,q-1}$.}
    
    Therefore,
    the construction being clearly additive,
    we described an homomorphism $h_{p,q} \colon \K^{p,q} \to \K^{p+1,q-1}$,
    for all $p \geq 0$ and $q \geq 1$.
    This construction is described by Figure \ref{The-Homomorphism-h1}.
    
    \begin{figure}[thb]
      $$
      \begin{tikzcd}[column sep=small,every label/.append style = {font = \small}]
        \sC^{p,q+1}                                                   & \                                                   & \   \\
        \sC^{p,q} \arrow[u, "d"] \arrow[r, "\delta"]    & \sC^{p+1,q}                                     & \   \\
        \sC^{p,q-1} \arrow[u, "d"] \arrow[r, "\delta"]  & \sC^{p+1,q-1} \arrow[u, "d"] \arrow[r,"\delta"] & \sC^{p+2,q-1}
      \end{tikzcd}
      \quad \quad
      \begin{tikzcd}[column sep=small,every label/.append style = {font = \small}]
        0                                                  &                    \                               & \  \\
        \alpha_{p,q} \arrow[u, "d"] \arrow[r, "\delta"]  & 0                                                    & \  \\
        \alpha_{p,q-1} \arrow[u, "d"] \arrow[r, "\delta"]  & \alpha_{p+1,q-1} \arrow[u, "d"] \arrow[r,"\delta"] & 0
      \end{tikzcd}
      $$
      \vspace{-.5cm}
      \caption{The Homomorphism $h_{p,q}$}
      \label{The-Homomorphism-h1}
    \end{figure}
    
    Now,
    let $\alpha \in \Omega^k(\X)$ be a closed form,
    $d\alpha = 0$.
    The smooth form $\alpha_{0,k} = \ev^*(\alpha)$ satisfies $d\alpha_{0,k}=0$ and $\delta \alpha_{0,k}=0$.
    Hence it belongs to $\Z^{0,k}$.
    Let $\kappa_{0,k} = \class(\alpha_{0,k}) \in \K^{0,k}$.
    Then,
    by applying the chain of homomorphisms
    $$
    \begin{tikzcd}[column sep=normal,every label/.append style = {font = \small}]
      \K^{0,k} \arrow[r, "h_{0,k}"]  & \K^{1,k-1} \arrow[r, "h_{1,k-1}"] & \cdots \arrow[r, "h_{k-2,2}"] & \K^{k-1,1} \arrow[r, "h_{k-1,1}"]  & \K^{k,0} \\[-.75cm]
      \kappa_{0,k} \arrow[r,mapsto]  & \kappa_{1,k-1} \arrow[r,mapsto] & \cdots \arrow[r,mapsto] & \kappa_{k-1,1} \arrow[r,mapsto] & \kappa_{k,0}
    \end{tikzcd}
    $$
    we get an element $\kappa_{0,k} \in \K^{k,0}$,
    $$
    \kappa_{k,0} = h_{k-1,1} \circ h_{k-2,2} \circ \cdots \circ h_{1,k-1} \circ h_{0,k} \big(\kappa_{0,k}\big).
    $$
    \textsc{Proposition} {\it The element $\kappa_{0,k}$ depends only on the class of $\alpha$ in $\H_\dR^k(\X)$,
    and the element $\kappa_{0,k}$ belongs to $\vH^k(\X,\RR)$.}
    
    \textsc{Definition.} {\it The homomorphism defined by the construction above
    $$
    h_k \colon \H_\dR^k(\X) \to \vH^k(\X,\RR),
    \quad \text{with} \quad
    h_k(\kappa_{0,k}) = \kappa_{k,0},
    $$
    will be called the \emph{$k$-th \v{C}ech-De-Rham homomorphism} of the space $\X$.}
    
    \textsc{Note.} For higher cohomology group,
    beginning at $k=1$,
    the De Rham homomorphism is not necessarily an isomorphism,
    in particular for tori $\T_\K = \RR/\K$.
    On the one hand,
    we have $\H^1_\dR(\T_\K) = \RR$.
    On the other hand $\vH^1(\K,\RR) = \Hom(\K,\RR)$ \art{Cech-Cohomology-Of-Tori}.
    Isomorphism happens only for $\K = a\ZZ$,
    that is,
    when $\T_\K \simeq \S^1$.
    Otherwise $\H^1_\dR(\T_\K) \not\simeq \vH^1(\T_K,\RR)$.
    To identify the kernel and the cokernel of the De Rham homomorphism is one important remaining question of this theory.
    $$
    \begin{tikzcd}[column sep=small,every label/.append style = {font = \small}]
      0 \arrow[r] & \ker(h_k) \arrow[r] & \H_\dR^k(\X) \arrow[r,"h_k"] & \vH^k(\X,\RR) \arrow[r] & \coker(h_k) \arrow[r] & 0.
    \end{tikzcd}
    $$
    That said,
    for $k=1$,
    $h_1$ is injective and we discuss a geometric interpretation of its cokernel in the next sections.
  \end{article}
  
  \begin{proof}
    Let $\kappa_{p,q} \in \K^{p,q}$,
    with $q\geq 1$.
    And let $\alpha_{p,q}$ representing $\kappa_{p,q}$.
    Thus $d\alpha_{p,q}=0$ and $\delta \alpha_{p,q}=0$.
    Since every closed form on $\cB$ is exact,
    we pick $\alpha_{p,q-1} \in \sC^{p,q-1}$ such that $d\alpha_{p,q-1} = \alpha_{p,q}$.
    And then,
    we define $\alpha_{p+1,q-1} = \delta\alpha_{p,q-1}$.
    Obviously,
    $\delta\alpha_{p+1,q-1}=0$.
    Now,
    $d\alpha_{p+1,q-1}=0 = d\delta\alpha_{p,q-1}= \pm \delta d \alpha_{p,q-1} = \delta \alpha_{p,q} = 0$.
    Hence,
    $\alpha_{p+1,q-1} \in \Z^{p+1,q-1}$ and we can define $\kappa_{p+1,q-1} = \class(\alpha_{p+1,q-1}) \in \K^{p+1,q-1}$.
    Now,
    if we chose another primitive $\alpha'_{p,q-1}$ such that $d\alpha'_{p,q-1} = \alpha_{p,q}$,
    then $d(\alpha'_{p,q-1} - \alpha_{p,q-1}) = 0$.
    And,
    
    (A) If $q \geq 2$,
    then there is $\beta \in \sC^{p,q-2}$ such that $\alpha'_{p,q-1} = \alpha_{p,q-1} + d\beta$.
    Hence,
    $\alpha'_{p+1,q-1} = \delta\alpha'_{p,q-1} = \delta \alpha_{p,q-1} + \delta d\beta$.
    Thus,
    $\kappa'_{p+1,q-1} = \class(\alpha'_{p+1,q-1}) = \class(\alpha_{p+1,q-1} + \delta d \beta) = \class(\alpha_{p+1,q-1}) = \kappa_{p+1,q-1}$.
    
    (B) If $q = 1$,
    then $\alpha_{p,1} = de_{p,0}$ with $e_{p,0} \in \sC^{p,0} = \Maps(\cM^p,\Omega^0(\cB))$,
    and $\kappa_{p+1,0} \in \K^{p+1,0} = \Z^{p+1,0}/\delta(\sC^{p+1,0})$.
    If we consider another primitive $e'_{p,0}$,
    then $e'_{p,0} = e_{p,0} + c$,
    where $c \in \sC^{p,0}$ with $dc=0$.
    Hence,
    $\alpha'_{p+1,0} = \delta e'_{p,0} = \delta e_{p,0} + \delta c = \alpha_{p+1,0} + \delta c$.
    But since $dc=0$,
    $\delta c \in \B^{p,0}$.
    Thus,
    $\kappa'_{p+1,0} = \class(\alpha_{p+1,0} + \delta c) = \class(\alpha_{p+1,0}) = \kappa_{p+1,0}$.
    
    Next,
    let $\alpha'^{p,q}$ represents the class $\kappa^{p,q}$.
    
    (A') If $p,q \geq 1$,
    then $\alpha'^{p,q} = \alpha^{p,q} + d\delta \epsilon^{p-1,q-1}$.
    We can choose the primitive $\alpha'^{p,q-1} = \alpha^{p,q-1} + \delta \epsilon^{p-1,q-1}$,
    which gives $\delta\alpha'^{p,q-1} = \delta\alpha^{p,q-1}$.
    
    (B') If $p = 0$ and $q\geq 1$,
    then $\alpha'^{0,q} = \alpha^{0,q} + d\epsilon^{0,q-1}$ with $\delta\epsilon^{0,q-1}= 0$.
    Then,
    again $\delta\alpha'^{0,q-1} = \delta\alpha^{0,q-1}$.
    
    Let us finish this proof by checking that $\K^{0,q} = \H_\dR^q(\X)$ and $\K^{p,0} = \vH^p(\X,\RR)$.
    So,
    let $\alpha^{0,q} \in \Z^{0,q}$.
    That means that $\alpha^{0,q} \in \Omega^q(\cB)$ and satisfies firstly $\delta\alpha^{0,q} = 0$,
    which means that $\alpha^{0,q} = \ev^*(\alpha)$ with $\alpha \in \Omega^q(\X)$,
    and $d\alpha^{0,q} = 0$ which means that $d\alpha = 0$.
    Thus $\Z^{0,q} \simeq \Z_\dR^q(\X)$.
    Now an element of $\B^{0,q}$ is a differential $d\beta^{0,q-1}$ with $\delta\beta^{0,q-1}=0$.
    Hence,
    $d\beta^{0,q-1}= d(\ev^*(\beta))$ with $\beta \in \B_\dR^q(\X)$.
    Therefore,
    $\K^{0,q} = \Z^{0,q}/\B^{0,q} = \Z_\dR^q(\X)/ \B_\dR^q(\X) = \H_\dR^q(\X)$.
    
    Next,
    let $\alpha^{p,0} \in \Z^{p,0}$.
    That means that $\alpha^{p,0} \in \sC^{p,0} = \Maps(\cM^p,\Omega^0(\X))$ and satisfies $d\alpha^{p,0} = 0$ and $\delta \alpha^{p,0} = 0$.
    From $d\alpha^{p,0} = 0$ we get that $\alpha^{p,0}(f_1,\dots,f_p)$ is constant on each component $\{\phi\} \times \dom(\phi)$,
    and then $\alpha^{p,0} \in \vC^p(\X,\RR)$.
    From $\delta \alpha^{p,0} = 0$ we deduce that $\alpha^{p,0} \in \vZ^p(\X,\RR)$.
    Now,
    an element of $\B^{p,0}$ is a differential $\delta\beta^{p-1,0}$ with $d\beta^{p-1,0} = 0$,
    that is,
    $\beta^{p-1,0} \in \vC^{p-1}$.
    Therefore,
    $\K^{p,0} = \Z^{p,0}/\B^{p,0} = \vZ^p(\X,\RR)/\vB^p(\X,\RR) = \vH^p(\X,\RR)$.
  \end{proof}
  
  %%%%%%%%%%%%%%%%%%%%%%%%%%%%%%%%%%%%%%%%%%%%%%%%%%%%%%%%%%
  %%
  %% MARK: - § The Spectral Sequence of the Bicomplex
  %%
  %%%%%%%%%%%%%%%%%%%%%%%%%%%%%%%%%%%%%%%%%%%%%%%%%%%%%%%%%%
  \section{The Spectral Sequence of the Bicomplex}
  
  
  %% MARK: • Cohomology Groups Of The Bicomplex
  \begin{article}\artlabel[Cohomology Groups Of The Bicomplex]
    \label{Cohomology-Groups-Of-The-Bicomplex}
    The cohomology groups of the bicomplex are defined and denoted as follows:
    %
    \begin{equation*}
      \begin{array}{l}
        \left\{\begin{array}{llcl}
          p \geq 1 & \H^{p,q}_{\delta} & = & \ker\big(\delta : \sC^{p,q}\rightarrow \sC^{p+1,q}\big) / \delta \big(\sC^{p-1,q}\big) \\[.3ex]
          p=0 &  \H^{0,q}_{\delta} & = & \ker\big(\delta : \sC^{0,q}\rightarrow \sC^{1,q}\big)
        \end{array}\right. \\[3ex]
        \left\{\begin{array}{llcl}
          q\geq 1 & \H^{p,q}_d & = & \ker\big(d : \sC^{p,q}\rightarrow \sC^{p,q+1} \big) / d \big(\sC^{p,q-1}\big) \\[.3ex]
          q=0 &  \H^{p,0}_d & = & \ker\big(d : \sC^{p,0}\rightarrow \sC^{p,1}\big)
        \end{array}\right .
      \end{array}
    \end{equation*}
    %
    Note that,
    by definition
    $$
    \sC^{0,q}=\Omega^q(\cB).
    $$
    Then,
    for all $\alpha \in \sC^{0,q}$ and all $f \in \cM$,
    $\delta(\alpha)(f) = f^*(\alpha) - \alpha$.
    Hence,
    thanks to the Proposition in \art{Differential-forms-in-diffeology}
    %
    \begin{equation*}
      \text{for all } q\in \NN \quad \H^{0,q}_\delta =\Omega^q(X).
    \end{equation*}
    %
    On the other hand,
    $\H^{p,0}_d$ is the kernel of the De Rham differential operator on $\sC^{p,0} = \Maps(\cM^p, \Omega^0(\cB))$.
    Thus,
    for $\sigma \colon \cM^p \to \Omega^0(\cB) = \Cinfty(\cB, \RR)$,
    $d_\dR\sigma = 0$ means that $\sigma(f_1,\dots,f_p)$ is constant on each component of $\cB$.
    Hence,
    $\sigma \in \sC^p(\cM, \A)$,
    where $\A$ is the group of locally constant real maps defined on $\cB$.
    And that is exactly the group of \Cech $p$-cochains of $\X$ with values in $\RR$,
    %
    \begin{equation*}
      \text{for all } p\in \NN\quad \H^{p,0}_d = \vC^p(X,\RR),
    \end{equation*}
  \end{article}
  
  %% MARK: • The Filtrations δ Of the Bicomplex
  \begin{article}\artlabel[The Filtrations $\delta$ Of the Bicomplex]
    The filtration $\delta$ is the set of sequences indexed by $q$,
    got by projecting the differential operator $\delta$ onto the cohomological quotients of two consecutive horizontal lines,
    in the previous diagram (Fig. \ref{bicomplex-CDR}).
    Replacing the nodes of the diagram by the corresponding $d$-cohomological groups.
    $$
    \begin{tikzcd}[column sep=small,every label/.append style = {font = \small}]
      \text{(Line $q > 0$)} \quad \cdots \arrow[r, "\delta"] & \ker[d \colon \sC^{p,q} \to \sC^{p,q+1}]/d[\sC^{p,q-1}] \arrow[r, "\delta"] &  \cdots
    \end{tikzcd}
    $$
    beginning with $p = 0,1,2$ etc.
    Now,
    $\sC^{p,q} = \Maps(\cM^p,\Omega^q(\X))$ and the exterior derivative $d$ applies to the $\Omega^q(\X)$-part.
    Since $\cB$ is the union of balls,
    and balls are contractible,
    every closed form is exact and all the terms of the $\delta$-sequence above vanishes except for the line indexed by $q = 0$,
    which becomes
    $$
    \begin{tikzcd}[column sep=small,every label/.append style = {font = \small}]
      %
      \ker[d \colon \sC^{0,0} \to \sC^{0,1}] \arrow[r, "\delta"] &  \ker[d \colon \sC^{1,0} \to \sC^{1,1}] \arrow[r,"\delta"] & \cdots
      %
    \end{tikzcd}
    $$
    Now,
    $$
    \sC^{p,0} = \Maps(\cM^p,\Omega^0(\cB))
    \quad \text{and} \quad
    \Omega^0(\cB) = \Cinfty(\cB,\RR).
    $$
    Then,
    $\ker[d \colon \sC^{p,0} \to \sC^{p,1}]$ is the set of locally constant maps from $\Maps(\cM^p,\Omega^0(\cB))$ to $\RR$,
    thus $\ker[d \colon \sC^{p,0} \to \sC^{p,1}]$ is just the set $\vC^p(\X,\RR)$ as defined by ($\diamondsuit$) in \art{Cech-Cohomology-Of-A-Diffeological-Space},
    which is also $ \H^{p,0}_d$ as stated just above.
    Hence,
    according to all that,
    $$
    \left\{
    \begin{array}{ll}
      q = 0 :    & \H_d^{0,0} \rfl{\delta}  \H_d^{1,0} \rfl{\delta}  \H_d^{2,0} \rfl{\delta}  \cdots \\[.25cm]
      q \geq 1 : & \H^{p,q}_d = \{0\}.
    \end{array}
    \right.
    $$
    The filtration $\delta$ is summarized by the diagram of Figure \ref{delta-Subcomplex}.
    \begin{figure}[ht]
      $$
      \begin{tikzcd}[column sep=small,every label/.append style = {font = \small}]
        \vdots                                             & \vdots                                            &   \vdots                                           &        \\[-.5cm]
        \{0\} \arrow[r,"\delta"]                           & \{0\}  \arrow[r, "\delta"]                        & \{0\} \arrow[r, "\delta"]                          & \cdots \\[-.5cm]
        \H_d^{0,1} \arrow[u ,equal] \arrow[r,"\delta"] & \H_d^{1,1} \arrow[u,equal] \arrow[r,"\delta"] & \H_d^{2,1} \arrow[u,equal] \arrow[r,"\delta"]  & \cdots \\[-.5cm]
        \vC^0(\X,\RR) \arrow[r,"\delta"]                   & \vC^1(\X,\RR)  \arrow[r, "\delta"]                & \vC^2(\X,\RR) \arrow[r, "\delta"]                  & \cdots \\[-.5cm]
        \H_d^{0,0} \arrow[u ,equal] \arrow[r,"\delta"] & \H_d^{1,0} \arrow[u,equal] \arrow[r,"\delta"] & \H_d^{2,0} \arrow[u,equal] \arrow[r,"\delta"]  & \cdots
      \end{tikzcd}
      $$
      \vspace{-.5cm}
      \caption{Subcomplex of the $\delta$-filtration.}
      \label{delta-Subcomplex}
    \end{figure}
    And, the \Cech subcomplex associated to the filtration $\delta$ is then denoted by
    $$
    \H^{\star,0}_d(X,\RR)= \bigoplus_{p\in\NN}\H^{p,0}_d
    \quad \text{with}\quad \H^{p,0}_d  = \vC^p(X,\RR).
    $$
  \end{article}
  
  %% MARK: • The Filtrations d Of the Bicomplex
  \begin{article}\artlabel[The Filtrations $d$ Of the Bicomplex]
    The filtration $d$ is the set of sequences indexed by $p$,
    got by projecting the differential operator $d$ onto the cohomological quotients of two consecutive vertical lines,
    in the previous diagram (Fig. \ref{bicomplex-CDR}).
    Replacing the nodes of the diagram by the corresponding $\delta$-cohomological groups.
    That is,
    for the node $(p,q)$,
    and thanks to \art{Cohomology-Groups-Of-The-Bicomplex},
    $$
    \left\{
    \begin{array}{ll}
      p = 0 :    & \H_\delta^{0,q} = \ker[\delta \colon \sC^{0,q} \to \sC^{1,q}] = \Omega^q(\X). \\[.25cm]
      p \geq 1 : & \H_\delta^{p,q} = \ker[\delta \colon \sC^{p,q} \to \sC^{p+1,q}]/\delta\big(\sC^{p-1,q}\big).
    \end{array}
    \right.
    $$
    Which gives the filtration summarized by the diagram of Figure \ref{d-Subcomplex}.
    
    \begin{figure}[thb]
      $$
      \begin{tikzcd}[column sep=small,every label/.append style = {font = \small}]
        \vdots                                           & \vdots                             &   \vdots                            &         \\
        \Omega^2(\X) = \H_\delta^{0,2} \arrow[u,"d"] & \H_\delta^{1,2} \arrow[u, "d"] & \H_\delta^{2,2} \arrow[u, "d"]  & \cdots  \\
        \Omega^1(\X) = \H_\delta^{0,1} \arrow[u,"d"] & \H_\delta^{1,1} \arrow[u, "d"] & \H_\delta^{2,1} \arrow[u, "d"]  & \cdots  \\
        \Omega^0(\X) = \H_\delta^{0,0} \arrow[u,"d"] & \H_\delta^{1,0} \arrow[u, "d"] & \H_\delta^{2,0} \arrow[u, "d"]  & \cdots
      \end{tikzcd}
      $$
      \vspace{-.5cm}
      \caption{Subcomplex of the $d$-filtration.}
      \label{d-Subcomplex}
    \end{figure}
    And,
    for all integer $p$,
    the De Rham subcomplex associated to the filtration $d$ is then denoted by
    $$
    \H^{p,\star}_\delta(X,\RR)= \bigoplus_{q\in\NN}\H^{p,q}_\delta
    \quad \text{with}\quad \H^{0,\star}_\delta = \Omega^\star(X).
    $$
  \end{article}
  
  %% MARK: • Bigraded Cohomolgy and Spectral Sequence
  \begin{article}\artlabel[Bigraded Cohomolgy Spectral Sequence]
    \label{Bigraded-Cohomolgy-Spectral-Sequence}
    The \v{C}ech-De-Rham double complex gives two bi-graded cohomology groups associated with the two filtrations $d$ and $\delta$,
    defining the two associated first quadrant spectral sequences.
    According to universal conventions,
    we have
    $$
    \E_0^{p,q} = \sC^{p,q},
    \quad ^d\E^{p,q}_1 = \H_\delta^{p,q},
    \quad ^\delta\E^{p,q}_1 = \H_d^{p,q},
    $$
    with
    $$
    ^d\E^{p,q}_2 = \H^q_d(\H^{p,\star}_\delta )
    \quad \text{and} \quad
    ^\delta\E^{p,q}_2 = \H^p_\delta(\H_d^{\star,q}).
    $$
    The \emph{De Rham bigraded group} is given by:
    $$
    \left\{
    \begin{array}{lcl}
      ^d\E^{p,q}_2 & = & \ker[d \colon \H^{p,q}_\delta \to \H^{p,q+1}_\delta]/ d\big( \H^{p,q-1}_\delta \big), \quad q \geq 1.  \\[.3ex]
      ^d\E^{p,0}_2 & = & \ker[d \colon \H^{p,0}_\delta \to \H^{p,1}_\delta].  \\[.3ex]
      ^d\E^{0,q}_2 & = & \H_\dR^q(\X).
    \end{array}
    \right .
    $$
    And the \emph{\Cech bigraded group} reduces to $\vH^\star(\X,\RR)$.
    Precisely,
    $$
    ^\delta\E^{p,q}_2 = 0
    \quad \text{for all} \quad q \geq 1,
    \quad \text{and} \quad
    ^\delta\E^{p,0}_2  =  \vH^p(\X,\RR).
    $$
    
    \textsc{Proposition.} {\it The total cohomology associated with the differential operator $\D = d + \delta$ converges to $\vH^\star(\X,\RR)$.
    The exact sequence of low-degree terms writes then
    %
    \begin{equation}\tag{$\clubsuit$}
      \begin{tikzcd}[column sep=1.2em,every label/.append style = {font = \small}]
        0 \arrow[rightarrow]{r} & \H^{1}_\dR(X) \arrow[r, "h_1"] & \vH^{1}(X,\RR) \arrow[r, "\eta_\init"] & {}^dE^{1,0}_2 \arrow[r, "\eta_\exit"] & \H^{2}_\dR(X) \arrow[r, "h_2"]  & \vH^{2}(X,\RR)
      \end{tikzcd}
    \end{equation}
    %
    where
    $$
    {}^d\E^{1,0}_2=\ker\big(d \colon \H^{1,0}_\delta  \rightarrow \H^{1,1}_\delta \big).
    $$
    }
    We shall interpret in the next articles the obstruction to the De Rham theorem represented by this term ${}^d\E^{1,0}_2$.
  \end{article}
  
  \begin{proof}
    Since this is a first quadrant spectral sequence,
    that is,
    $\sC^{p,q} = 0$ if $p<0$ or $q<0$,
    and since ${}^\delta\E^{p,q}_2 = 0$ for $q \geq 1$,
    the total cohomology of the bicomplex coincides with the \Cech cohomology \cite[Thm. 4.8.1]{God64}.
    Then,
    the exact sequence of low-degree terms is the translation of the five-term exact sequence of the general first quadrant spectral sequence:
    $$
    0 \to {}^d\E^{0,1}_2 \to \H^1 \to {}^d\E_2^{1,0} \to {}^d\E_2^{0,2} \to \H^2.
    $$
    See Thm. 4.5.1 \emph{op. cit.}
  \end{proof}
  
  %%%%%%%%%%%%%%%%%%%%%%%%%%%%%%%%%%%%%%%%%%%%%%%%%%%%%%%%%%
  %%
  %% MARK: - § Obstruction In The \Cech-De-Rham Morphism
  %%
  %%%%%%%%%%%%%%%%%%%%%%%%%%%%%%%%%%%%%%%%%%%%%%%%%%%%%%%%%%
  \section{Obstruction In The \Cech-De-Rham Morphism}
  
  %% MARK: •  Flows over Diffeological Spaces
  \begin{article}\artlabel[Flows over Diffeological Spaces]
    The first obstruction of the De Rham theorem involves a special kind of fiber bundle:
    
    \textsc{Definition.} {\it We call a principal fiber bundle over a diffeological space $\X$ with group the additive real numbers $(\RR, +)$ a \emph{flow} over $\X$.}
    
    Note that,
    if $\X$ is a manifold,
    the flows over $\X$ are trivial,
    since every fiber bundle on a manifold with contractible fiber has a global smooth section,
    and,
    every principal fiber bundle with a global smooth section is trivial.
    
    Now,
    let $\pi \colon \Y \to \X$ and $\pi' \colon \Y' \to \X$ two flows.
    The direct sum $\pi \oplus \pi'$ is the $(\RR^2,+)$ principal bundle over $\X$,
    with total space
    $$
    \Y \oplus \Y' = \{ (y,y') \in \Y \times \Y' \mid \pi(y) = \pi'(y') \}.
    $$
    The action of $\RR^2$ is the product of the two actions:
    for all $(t,t') \in \RR^2$ and $(y,y') \in \Y \oplus \Y'$,
    $(t,t')_{\Y \oplus \Y'}(y,y') = (t_\Y(y),t'_{\Y'}(y'))$.
    The subscript denotes the actions of $\RR$ or $\RR^2$ on the total spaces.
    Now,
    consider the anti-diagonal $\ul\Delta = \{(t,-t)\}_{t \in \RR}$.
    It acts on $\Y \oplus \Y'$ by $(t,-t)_{\Y \oplus \Y'}(y,y') = (t_\Y(y),-t_{\Y'}(y'))$.
    
    The quotient of  $\Y \oplus \Y'$ by the action of the anti diagonal $\ul\Delta$ is still a $(\RR,+)$ principal bundle over $\X$.
    We denote it by $\pi'' \colon \Y'' \to \X$,
    with
    $$
    \Y'' = \Y \oplus \Y' / \ul\Delta.
    $$
    Its elements are the classes of pairs $(y,y')$ that project on the same point $x \in \X$,
    that is:
    $$
    \pi''(\class(y,y')) = \pi(y) = \pi'(y').
    $$
    The action of $(\RR,+)$ on $\Y''$ is the projection of the residual action of $(\RR^2,+)$ on $\Y \oplus \Y'$,
    that is,
    $$
    t_{\Y''}(\class(y,y')) = \class(t_\Y(y),y') = \class(y,t_{\Y'}(y')).
    $$
    
    We denote by $\Fl(\X)$ the set of equivalence classes of flows over $\X$.
    The sum defined above passes to the set of classes:
    $$
    \class(\pi) + \class(\pi') = \class(\pi'').
    $$
    That operation is associative and gives to the set $\Fl(\X)$ a structure of Abelian group.
    The identity is the class of the trivial bundle $\pr_1 \colon \X \times \RR \to \X$.
    The inverse of the class of $\pi \colon \Y \to \X$ is the class of the same bundle,
    but with the opposite action of $\RR$ $t \colon y \mapsto {-t}_\Y(y)$.
    
    We comment the case of the irrational torus $\T_\alpha$ in \xart{The-First-Obstruction-To-De-Rham-Theorem}{Remark 1}.
  \end{article} % Flows over a Diffeological Space
  
  \begin{proof}
    The operation $\class(\pi) + \class(\pi')$ is obviously commutative since $\pi \oplus \pi'$ is clearly equivalent to $\pi' \oplus \pi$.
    The class of the trivial bundle $\pr_1 \colon \X \times \RR \to \RR$ is the identity,
    thanks to the map $(x,t,y) \mapsto t_\Y(y)$ that identifies $\class (\pr_1 \oplus \pi)$ with $\pi$.
    Now,
    let $\bar\pi \colon \overline\Y \to \X$ be the same principal bundle $\pi \colon \Y \to \X$,
    but with the opposite $(\RR,+)$-action.
    That is,
    $\overline\Y = \Y$ and $t_{\overline\Y}(y) = -t_\Y(y)$.
    Consider the smooth map $y \mapsto \class(y,y)$ from $\Y$ to $(\Y \oplus \overline\Y)/\ul\Delta$.
    It satisfies $t_\Y(y) \mapsto \class(t_\Y(y),t_\Y(y)) = \class(t_\Y(y),-t_{\overline\Y}(y)) = \class(y,y)$.
    Hence,
    it projects on $\X$ into a smooth section of $\pi \oplus \bar\pi$,
    which is therefore trivial.
    The associativity of the addition is left to the reader.
  \end{proof}
  
  %% MARK: • About Connections in Diffeology
  \begin{article}\artlabel[About Connections in Diffeology]
    The notion of connection in diffeology \cite{Igl85} is the second important construction that occurs in the identification of the first obstruction of the De Rham's theorem.
    To clarify the context,
    we recall in this section the main properties,
    for more details see \cite[\textsection 8.32]{PIZ13}.
    
    Consider a principal fiber bundle $\pi \colon \Y \to \X$,
    with group $\G$.
    Then,
    the projection $\pi_* \colon \Paths(\Y) \to \Paths(\X)$,
    with $\pi_*(\gamma) = \pi \circ \gamma$,
    is a principal fiber bundle with group $\Paths(\G)$.
    The group $\G$ injects in $\Paths(\G)$ by the constant paths.
    
    Roughly speaking,
    a connection $\Theta$ on the $\G$-principal fiber bundle $\pi \colon \Y \to \X$,
    is a reduction of the $\Paths(\G)$-principal fiber bundle $\pi_* \colon \Paths(\Y) \to \Paths(\X)$ to the subgroup $\G$ of constant paths,
    see \cite[A5.1(4) \& A5.2(1)]{Igl85}.
    Precisely,
    we identify the connection with a smooth projector defined from the tautological bundle of the space $\Paths_\loc(\Y)$ of local paths%
    \footnote{A local path in a diffeological space is any plot defined on some open interval of $\RR$.}
    in $\Y$,
    that is,
    the set of pairs $(\gamma,t)$ such that $\gamma \in \Paths_\loc(\Y)$ and $t \in \dom(\gamma)$,
    into the space of local paths in $\Y$,
    which satifies a set of natural properties:
    \begin{enumerate}
      \item[1.] \emph{Domain}. $\dom(\Theta(\gamma,t)) = \dom(\gamma)$.
      \item[2.] \emph{Lifting}. $\pi \circ \Theta(\gamma,t) = \pi \circ \gamma$.
      \item[3.] \emph{Basepoint}. $\Theta(\gamma,t)(t) = \gamma(t)$.
      \item[4.] \emph{Reduction}. $\Theta(a \cdot \gamma)(t) = a(t)_\Y \circ \Theta(\gamma,t)$,
      where $a \colon \dom(\gamma) \to \G$ is a smooth path and $a \cdot \gamma \colon s \mapsto a(s)_\Y(\gamma(s))$.
      \item[5.] \emph{Locality}. $\Theta(\gamma \circ f,s) = \Theta(\gamma, f(s)) \circ f$,
      where $f$ is any smooth local path with values in $\dom(\gamma)$.
      \item[6.] \emph{Projector}. $\Theta(\Theta(\gamma,t),t) = \Theta(\gamma,t)$
    \end{enumerate}
    The paths in the image of $\Theta$ are said to be \emph{horizontal}.
    Precisely,
    the path $\Theta(\gamma,t)$ is the \emph{horizontal projection} of $\gamma$,
    pointed at $t$.
    It is also the \emph{horizontal lifting} of the path $\pi \circ \gamma \in \Paths_\loc(\X)$,
    passing through $\gamma(t)$ at time $t$.
    
    Regarded roughly as a reduction of a principal fiber bundle,
    the connection $\Theta$ can be also interpreted as a smooth section of the quotient space $\pi_\star \colon  \Paths(\Y) \to \Paths(\Y)/\G$.
    The section $\bar\Theta \colon \Paths(\X) \to \Paths(\Y)/\G$ associates with each path $p$ in $\X$ the $\G$-orbit of horizontal liftings of $p$.
    $$
    \begin{tikzcd}[column sep=normal,every label/.append style = {font = \small}]
      \Paths(\Y)  \arrow[rr, "\class"] \arrow[rd, swap, "\pi_*"] &    & \Paths(\Y)/\G \arrow[dl,swap, "\pi_\star"] \\[.75cm]
      & \Paths(\X) \arrow[bend right=30, swap, ur, "\bar\Theta"]  &
    \end{tikzcd}
    $$
    Each path $\bar p \in \bar\Theta(p)$ is completely defined giving an origin $y \in \pi^{-1}(x)$,
    with $y = \bar p(o)$ and $x = p(o)$.
    The origin $o$ is actually arbitrary and the path $p$ can be considered local.
  \end{article}
  
  %% MARK: • Connections on Torus Bundles
  \begin{article}\artlabel[Connections on Torus Bundles]
    Let $\pi \colon \Y \to \X$ by a principal fiber bundle with group a torus $\T_\K = \RR/\K$,
    where $\K \subset \RR$ is a strict subgroup.
    Among all possible connections,
    we consider the special class defined by a \emph{connection form} $\lambda$.
    That is:
    
    \begin{enumerate}
      \item[1.] $\lambda \in \Omega^1(\Y)$.
      \item[2.] $\lambda$ is $\T_\K$ invariant,
      for all $\tau \in \T_\K$,
      $\tau^*(\lambda) = \lambda$.
      \item[3.] $\lambda$ is \emph{calibrated}\,:
      for all $y \in \Y$,
      $\hat y^*(\lambda) = \theta$,
      where $\hat y \colon \T_\K \to \Y$ is the orbit map $\hat y(\tau) = \tau(y)$,
      and $\theta = \class_*(\dt)$.
    \end{enumerate}
    
    Then,
    the connection associated with $\lambda$ is defined,
    according to the definition above,
    by:
    $$
    \Theta_\lambda(\gamma,t) \colon t' \mapsto \class\left( - \int_t^{t'} \lambda(\gamma)_s(1) \, \ds \right)_\Y \big(\gamma(t')\big)
    $$
    for all $\gamma \in \Paths(\Y)$,
    and where the $\Y$ in index denotes the action of $\T_\K$ on $\Y$,
    and $\class$ is the projection from $\RR$ to $\T_\K$.
    More details are given in \cite[\textsection 8.37]{PIZ13}.
    
    Horizontal paths are the paths $\underline\gamma$ that cancel the connection form:
    $\lambda(\underline\gamma)_t(1) = 0$ for all $t$.
    
    In particular,
    that applies to flows over diffeological spaces,
    the $(\RR,+)$-principal fiber bundles,
    where $\class$ in the formula above is just the identity.
    In that case,
    we will denote by $\Fl^\bullet(\X) \subset \Fl(\X)$ the subgroup of classes of flows that support an infinitesimal connection defined by a connection form $\lambda$.
  \end{article}
  
  %% MARK: • The First Obstruction To De Rham Theorem
  \begin{article}\artlabel[The First Obstruction To De Rham Theorem]
    \label{The-First-Obstruction-To-De-Rham-Theorem}
    The first obstruction to the De Rham theorem,
    relative to the \Cech cohomology,
    is the cokernel of the morphism $h_1$ in the exact sequence
    $$
    \begin{tikzcd}[column sep=1.2em,every label/.append style = {font = \small}]
      0 \arrow[rightarrow]{r} & \H^{1}_\dR(X) \arrow[r, "h_1"] & \vH^{1}(X,\RR) \arrow[r, "\eta_\init"] & {}^dE^{1,0}_2 \arrow[r, "\eta_\exit"] & \H^{2}_\dR(X) \arrow[r, "h_2"]  & \vH^{2}(X,\RR)
    \end{tikzcd}
    $$
    It identifies naturally with $\Val(\eta_\init) = \ker(\eta_\exit)$.
    Let us recall that:
    $$
    {}^dE^{1,0}_2 = \ker[d \colon \H_\delta^{1,0} \to \H_\delta^{1,1}]
    \text{ with }
    \left\{
    \begin{array}{l}
      \H_\delta^{1,0} = \ker(\delta \colon \sC^{1,0} \to \sC^{2,0}) / \delta(\sC^{0,0}) \\[5pt]
      \H_\delta^{1,1} = \ker(\delta \colon \sC^{1,1} \to \sC^{2,1}) / \delta(\sC^{0,1}).
    \end{array}
    \right.
    $$
    In order to describe the space ${}^dE^{1,0}_2$ and the two morphisms $\eta_\init$ and $\eta_\exit$,
    we will first interpret geometrically the space
    $$
    \H_\delta^{1,0} = \Z_\delta^{1,0}/\B_\delta^{1,0},
    \ \text{ with } \
    \Z_\delta^{1,0} = \ker(\delta \colon \sC^{1,0} \to \sC^{2,0})
    \ \text{ and } \
    \B_\delta^{1,0} = \delta(\sC^{0,0}).
    $$
    %
    Let us recall that $\sC^{1,0} = \Maps(\cM,\Omega^0(\cB))$ and $\sC^{0,0}=\Omega^0(\cB)$.
    By definition of the operator $\delta$,
    the space $\Z_\delta^{1,0}$ is made of maps $\tau$ from $\cM$ to $\Omega^0(\cB)$ such that,
    $$
    \tau(g \circ f) = f^*(\tau(g)) + \tau(f),
    \ \text{ for all $f$, $g$ in $\cM$.}
    $$
    And $\tau$ belongs to $\B_\delta^{1,0}$ if and only if there exists $\sigma \in \Omega^0(\cB)$ such that $\tau = \delta\sigma$,
    that is,
    $$
    \delta\sigma(f) = f^*(\sigma) - \sigma,
    \ \text{ for all $f \in \cM$}.
    $$
    Hence,
    $$
    \H_\delta^{1,0} = \{ \tau \mid \tau(g \circ f) = f^*(\tau(g)) + \tau(f) \} /
    \{ \tau \mid \tau(f) = f^*(\sigma) - \sigma \}.
    $$
    Now,
    every cocycle $\tau \in \Z_\delta^{1,0}$ defines an action of the monoid $\cM$ on $\cB \times \RR$,
    that lifts the action of $\cM$ on $\cB$. % described in \xart{Differential-forms-Revisited}{Remark}
    Let $f \in \cM$,
    $b=(\phi,r) \in \cB$ and $t \in \RR$,
    one has:
    $$
    f_\tau(b,t) = \big(f(b), t + \tau(f)(b)\big).
    $$
    Developed,
    according to notation of \art{The-Gauge-Monoid-of-Round-Plots},
    that gives:
    $$
    f_\tau(\phi,r,t) = \big(f_\fB(\phi), \phi_f(r), t + \tau(f)(\phi,r)\big).
    $$
    This binary relation $(b',t') = f_\tau(b,t)$,
    for some $f \in \cM$,
    extends to the equivalence relation $\sim_\tau$,
    generated by the symmetric closure of the action of the monoid $\cM$ on $\cB \times \RR$,
    that lifts the equivalence relation described in \xart{Differential-forms-Revisited}{Remark}.
    We shall denote by $\cB \times_\tau \RR$ the quotient $(\cB \times \RR)/ \sim_\tau$.
    We have,
    then,
    the commutative diagram:
    $$
    \begin{tikzcd}[column sep=large,every label/.append style = {font = \small}]
      \cB \times \RR  \arrow[r, "\class"] \arrow[d, swap, "\pr_1"]  & \Y_\tau = \cB \times_\tau \RR \arrow[d, "\pi_\tau"] \\[.75cm]
      \cB \arrow[r, swap, "\ev"]                                    & \X = \cB/\cM
    \end{tikzcd}
    $$
    where the projection $\pi_\tau$ is defined by:
    $$
    \pi_\tau(\class(b,t)) = \ev(b),
    \quad\text{that is,}\quad
    \pi_\tau(\class(\phi,r,t)) = \phi(r).
    $$
    
    \textsc{Theorem.} \textit{The projection $\pi_\tau \colon \Y_\tau \to \X$ is a flow over $\X$,
    with the following action of $(\RR,+)$\,:
    $$
    s_\Y\big(\class(b,t)\big) = \class(b,t+s),
    \text{ for all $s \in \RR$.}
    $$
    %
    \textnormal{1.} Cohomologous cocycles define equivalent principal fiber bundles.
    That is,
    the principal bundle defined by $\tau + \delta\sigma$ is equivalent to the principal bundle defined by $\tau$.
    That defines a map
    $$
    \Psi \colon \H_\delta^{1,0}(\X) \to \Fl(\X)
    \quad \text{by} \quad
    \class(\tau) \mapsto \class[\pi_\tau \colon \Y_\tau \to \X].
    $$
    \textnormal{2.} The map $\Psi$ is a group isomorphism and identifies $\H_\delta^{1,0}(\X)$ with $\Fl(\X)$. \\
    \textnormal{3.} Through this identification,
    the subspace ${}^dE^{1,0}_2 \subset \H_\delta^{1,0}(\X)$ becomes the subset $\Fl^\bullet(\X)$ of classes of flows that admit a connexion $1$-form $\lambda$.
    }
    $$
    \begin{tikzcd}[column sep=1.2em,row sep=1.2em, every label/.append style = {font = \small}]
      &                                &                                        & \Fl(\X)                                        &                                 &  \\
      0 \arrow[r,"\ "] & \H^{1}_\dR(X) \arrow[r, "h_1"] & \vH^{1}(X,\RR) \arrow[r, ""] & \Fl^\bullet(\X) \arrow[r, "c_1"] \arrow[u, ""] & \H^{2}_\dR(X) \arrow[r, "h_2"]  & \vH^{2}(X,\RR) \\
      &                                &                                        & 0  \arrow[u, ""]                                                &                                 &
    \end{tikzcd}
    $$
    \textit{
    \textnormal{4.} The morphism $\eta_\exit$ becomes the characteristic class $c_1 \colon \Fl^\bullet(\X) \to \H^{2}_\dR(X)$,
    that associates the principal bundle with the cohomology class of the curvature of the connexion form $\lambda$. \\
    \textnormal{5.} The obstruction of the first De Rham homomorphism identifies with the kernel of $c_1$,
    that is,
    the subspace of flows equipped with a flat connexion.
    These are the various quotients of the trivial flow on the universal covering $\tilde\X$,
    by a lifting of the action of the fundamental group $\pi_1(\X)$.
    }
    
    \textsc{Remark 1.} Since the real principal fiber bundles over a manifold are trivial,
    this obstruction vanishes in this case.
    This shows how the surjectivity of the first De Rham homomorphism,
    for manifolds,
    is related to the existence of global smooth sections of bundles with contractible fiber.
    
    \textsc{Remark 2.}  In the case $\X=\T_\alpha$,
    quotient of the $2$-torus $\T^2$ by the $1$-parameter subgroup of irrational slope $\alpha$,
    the group $\Fl(\T_\alpha)$ has been directly computed in \cite{Igl86},
    see also \cite[\textsection 8.39]{PIZ13}.
    It shows a huge difference in behavior between Diophantine numbers (far from rational numbers) and Liouville numbers (close to rational numbers).
    In the first case $\Fl(\T_\alpha) = \Fl^\bullet(\T_\alpha) = \RR$,
    and in the second case $\dim(\Fl(\T_\alpha))= \infty$.
    
    \textsc{Remark 3.} The irrationnal torus $\T_\alpha$ is equivalent to the quotient of $\RR$ by the subgroup $\ZZ + \alpha \ZZ$.
    It it always possible to equip any flow $\pi \colon \Y \to \T_\alpha$ with a connection.
    Indeed,
    we first lift any path in $\T_\alpha$ to the universal covering $\tilde\T_\alpha = \RR$,
    which is essentially unique up to an element of $\ZZ+\alpha\ZZ$.
    Then,
    we remark that the pullback $\pr_1 \colon \class^*(\Y) \to \RR$ is trivial because it is a $(\RR,+)$-principal bundle over $\RR$.
    Thus,
    we just lift horizontally,
    in the product $\tilde\T_\alpha \times \RR$,
    the lifting of the path in $\tilde\T_\alpha$ and push it into $\Y$ by the second projection.
    Now,
    according to the above,
    that shows that
    
    \textsc{Proposition.} \textit{For Liouville numbers $\alpha$,
    there exists flows over $\T_\alpha$ equipped with connections that cannot be defined by a differential $1$-form.}
    
    This is the first case of this kind in diffeology,
    and justify \emph{a posteriori} the introduction of general connections in the theory.
    
    \textsc{Remark 4.} In the general case,
    the morphism
    $$
    c_1 \colon \Fl^\bullet(\X) \to \H^{2}_\dR(X)
    $$
    maps the connection form $\lambda$ to the cohomology class of its curvature,
    that is,
    $\omega$ such that $d\lambda = \pi^*(\omega)$.
    It is a first genuine diffeological \emph{characteristic class},
    since it vanishes on manifolds.
  \end{article}
  
  \begin{proof}
    Let $\pi \colon \Y \to \X$ be a flow and let us consider its pullback $\pr_1 \colon \ev^*(\Y) \to \cB$ over $\cB$.
    Because $\pr_1$ is a $(\RR,+)$-principal bundle over a sum of open balls,
    it is trivial over each ball.
    There exists then,
    for each round plot $\phi \in \fB$,
    a smooth section $\rho \colon \dom(\phi) \to \ev^*(\Y)$ of the projection $\pr_1$.
    Let $f \in \cM$,
    the pullback of the flow $\pi \colon \Y \to \X$ by $\ev \circ f$ is identical to the pullback by $\ev$,
    since $\ev \circ f = \ev$.
    The fiber over $f(\phi,r)$ coincides with the fiber over $(\phi,r)$,
    for all $(\phi,r) \in \cB$.
    There exists then a smooth map $\tau \colon \cM \to \Cinfty(\cB, \RR)$ such that:
    $$
    \text{for all } b = (\phi,r) \in \cB, \quad \rho(f(b)) = \tau(f)(b) \cdot \rho(b),
    $$
    where the middle dot denotes the action of $\RR$ on the principal fiber bundle.
    It is then easy to check that $\tau$ is a Hochschild $1$-cocycle of the monoid $\cM$ with values in $\Cinfty(\cB,\RR) = \Omega^0(\cB)$.
    One checks also that if we change the section $\rho$,
    the cocycle $\tau$ changes by a coboundary $\sigma$,
    which is the difference of the two sections.
    We just defined a map $\Psi$ that associates to the class of the flow $\pi \colon \Y \to \X$,
    the class of $\tau$ in $\H_\delta^{1,0}(\X)$.
    $$
    \Psi \colon \Fl(\X) \to \H_\delta^{1,0}(\X)
    \quad\text{with}\quad
    \Psi\big(\class(\pi \colon \Y \to \X)\big) = \class(\tau).
    $$
    We showed also indirectly that $\Psi$ is injective.
    
    Conversely,
    as we claim,
    every $\tau \in \Z_\delta^{1,0}$ defines an action of $\cM$ on the product $\cB \times \RR$ that lifts the action of the monoid $\cM$ on $\cB$.
    The group $(\RR,+)$ acts freely on the quotient $\Y_\tau = \cB \times_\tau \RR$ by $t \cdot \class(b,s) = \class(b, s+t)$.
    That makes $\pi_\tau \colon \Y_\tau \to \X$ a $(\RR,+)$ principal bundle.
    To test if the projection $\pi_\tau$ is indeed a fiber bundle over a diffeological space,
    we have to check its pullbacks by the plots and verify that they are locally trivial.
    It suffice to consider round plots.
    So,
    let $\phi \in \fB$,
    $\phi^*(\Y_\tau) \simeq \{((\phi,r), \class(\phi',r',t')) \mid \phi(r) = \pi_\tau(\phi',r',t') = \phi'(r')\}$,
    but since $\phi(r) = \phi'(r')$ then there is $t \in \RR$ such that $\class(\phi,r,t) = \class(\phi',r',t')$.
    Thus,
    $\phi^*(\Y_\tau) \simeq \{(\phi,r,t) \mid r \in \dom(\phi) \text{ and } t \in \RR\} \simeq \dom(\phi) \times \RR$.
    Therefore,
    $\pi_\tau$ is a $(\RR,+)$-principal fiber bundle.
    
    The class of $\tau$ in $\H_\delta^{1,0}(\X)$ is obviously associated with the class of the flow $\pi_\tau \colon \Y_\tau \to \X$,
    by the map $\Psi$ defined above.
    A cohomologous cocycle would have given an equivalent flow over $\X$.
    The map $\Psi$ is then bijective,
    and gives a geometrical interpretation of the intermediate cohomology group $\H_\delta^{1,0}(\X)$ as the group of flows over $\X$,
    and here we get:
    $$
    \H_\delta^{1,0}(\X) \simeq \Fl(\X).
    $$
    It remains to identify now ${}^dE^{1,0}_2 = \ker[d \colon \H_\delta^{1,0} \to \H_\delta^{1,1}]$,
    among $\Fl(\X)$.
    That is,
    $\class(\tau)$ in $\H_\delta^{1,0}$ such that $d(\class(\tau))=0$.
    Let $\pi \colon \Y \to \X$ be a flow represented by $\tau$.
    Let $\rho$ be a section of $\pr_1 \colon \ev^*(\Y) \to \cB$ over some $b \in \fB$,
    as above.
    We know that,
    for all $f \in \cM$,
    $\rho(f(b)) = \tau(f)(b) \cdot \rho(b)$.
    Now,
    for every round plot $\P$ in $\Y$ is associated a round plot $\pi \circ \P$ in $\X$.
    For all $r \in \dom(\P)$,
    $\rho(\pi \circ \P,r)$ and $\P(r)$ belong to the same fiber.
    Therefore,
    there exists a smooth map $\nu(\P) \colon \dom(\P) \to \RR$ such that
    $$
    \text{for all } r \in \dom(P), \quad \rho(\pi \circ \P,r) = \nu(\P)(r) \cdot \P(r).
    $$
    On the other hand,
    the condition $d(\class(\tau))=0$ means $\class_\delta(d\tau)=0$,
    that is,
    there exists $\alpha \in \Omega^1(\cB)$ such that
    $$
    \text{for all } f \in \cM, \quad d\tau(f) = f^*(\alpha) - \alpha.
    $$
    We define then the mapping $\lambda$,
    that associates with very round plot $\P$ in $\Y$,
    the $1$-form
    $$
    \lambda(\P) = \alpha(\pi \circ \P) - d\nu(\P)
    $$
    on $\dom(\P)$.
    One can check that $\lambda$ defines on $\Y$ a differential $1$-form,
    and moreover that this $1$-form is a connection form.
    
    Conversely,
    on can verify that if $\pi \colon \Y \to \X$ can be equiped with a connection $1$-form,
    then $d(\class(\tau))=0$.
    Thus,
    the connecting space ${}^dE^{1,0}_2$ identies with the subset of flows over $\X$ that can be equipped with a connection $1$-form.
    In summary:
    $$
    \H_\delta^{1,0} \simeq \Fl(\X)
    \quad \text{and} \quad
    {}^dE^{1,0}_2 \simeq \Fl^\bullet(\X).
    $$
    As an application of this constructions,
    we consider the irrational torus $\T_\alpha = \RR/\ZZ + \alpha \ZZ$ \cite{DonIgl83}.
    In that case we compute directly
    $$
    \H_\dR^1(\T_\alpha) = \RR,
    \quad \vH^1(\T_\alpha,\RR) = \RR^2
    \quad\text{and}\quad
    \H_\dR^2(\T_\alpha) = 0.
    $$
    The exact sequence gives then $0 \to \RR \to \RR^2 \to \Fl^\bullet(\X) \to 0$.
    We deduce then that $\Fl^\bullet(\X) = \RR$.
  \end{proof}
  
  %%%%%%%%%%%%%%%%%%%%%%%%%%%%%%%%%%%%%%%%%%%%%%%%%%%%%%%%%%
  %%
  %% MARK: - Bibliography
  %%
  %%%%%%%%%%%%%%%%%%%%%%%%%%%%%%%%%%%%%%%%%%%%%%%%%%%%%%%%%%
  
  \begin{thebibliography}{DonIgl83}
    
    \bibitem[Che77]{Che77}
    Kuo-Tsai Chen.
    \newblock \emph{Iterated path integral}.
    \newblock {\em Bull. of Am. Math. Soc.}, 83(5):831--879, 1977.
    
    \bibitem[Don84]{Don84}
    Paul Donato.
    \newblock \emph{Rev\^etement et groupe fondamental des espaces diff\'e\-rentiels homog\`enes},
    \newblock {Th\`ese de doctorat d'\'etat, Universit\'e de Provence, Marseille, 1984}.
    
    \bibitem[DonIgl83]{DonIgl83}
    \newblock{Paul Donato and Patrick Iglesias.}
    \newblock\emph{Exemple de groupes différentiels : flots irrationnels sur le tore.}
    \newblock{Preprint CPT-83/P.1524, Centre de Physique Théorique, Marseille Luminy, July 1983.}
    %\newline
    \newblock{\scriptsize\verb!http://math.huji.ac.il/~piz/documents/EDGDFISLT.pdf!}
    
    \bibitem[DonIgl85]{DonIgl85}
    \bysame
    \newblock\emph{Exemple de groupes différentiels : flots irrationnels sur le tore.}
    \newblock{\em Comptes Rendus de l'Acad{\'e}mie des Sciences}, 301(4), Paris, 1985.
    
    \bibitem[God64]{God64}
    Roger Godement.
    \newblock \emph{Topologie alg\'ebrique et th\'eorie des faisceaux}.
    \newblock Hermann, Paris, 1964.
    
    \bibitem[Gur14]{Gur14}
    Serap Gürer
    \newblock \emph{Topologie algébrique des espaces difféologiques}.
    \newblock PhD Th\`ese Université Lille 1, 2014.
    
    \bibitem[Hoch45]{Hoch45}
    \emph{On the Cohomology Groups of an Associative Algebra}
    Gerhard Hochschild
    \newblock Annals of Mathematics, Second Series, Vol. 46, No. 1 (1945), pp. 58-67.
    
    \bibitem[Igl85]{Igl85}
    Patrick Iglesias.
    \newblock \emph{Fibr\'es diff\'eologiques et homotopie}.
    \newblock Th\`ese de doctorat d'\'etat, Universit\'e de Provence, F.~Marseille, 1985.
    
    \bibitem[Igl86]{Igl86}
    Patrick Iglesias.
    \newblock \emph{Diff\'eologie d'espace singulier et petits diviseurs}.
    \newblock {\em C.R.A.S.}, 302:519--522, Paris, 1986.
    
    \bibitem[PI88-91]{PI88-91}
    \bysame
    \newblock\emph{Bi-complexe cohomologique des espaces diff\'erentiables} \newline
    \newblock{\scriptsize\verb!http://math.huji.ac.il/~piz/documents/BCCED.pdf!}
    \newline
    \newblock{A revision of \emph{Une cohomologie de \Cech pour les espaces diff\'erentiables et sa relation à la cohomologie de De Rham}}
    \newblock{Preprint CPT -Marseille. CNRS CPT-88/P.2193.}
    \newline
    \newblock{\scriptsize\verb!http://math.huji.ac.il/~piz/documents/UCDCPLEDESRALCDDR.pdf!}
    
    \bibitem[PIZ13]{PIZ13}
    Patrick Iglesias-Zemmour
    \newblock\emph{Diffeology.}
    \newblock{Mathematical Surveys and Monographs. The American Mathematical Society, vol. 185, USA R.I. 2013.}
    
    \bibitem[Kur19]{Kur19}
    Katsuhiko Kuribayashi;
    \newblock \emph{Simplicial cochain algebras for diffeological spaces.}
    \newblock{preprint, arXiv:1902.10937.}
    
    \bibitem[Mac67]{Mac67}
    Saunders MacLane.
    \newblock \emph{Homology}.
    \newblock Springer Verlag, New-York Heidelberg Berlin, 1967.
    
    \bibitem[Sou80]{Sou80}
    Jean-Marie Souriau.
    \newblock \emph{Groupes diff\'erentiels}.
    \newblock Lecture notes in mathematics, 836:91--128, 1980.
    
    \bibitem[Sou84-a]{Sou84-a}
    \bysame
    \newblock \emph{Groupes diff\'erentiels et physique math\'e\-ma\-tique}.
    \newblock Collection travaux en cours, pp.75--79, 1984.
    
    \bibitem[Sou84-b]{Sou84-b}
    \bysame
    \newblock \emph{Un algorithme g{\'e}n{\'e}rateur de structures quantiques}.
    \newblock in Élie Cartan et les mathématiques d'aujourd'hui - Lyon, 25-29 juin 1984, Astérisque, no. S131 (1985), pp. 341-399.
    
    \bibitem[Wei52]{Wei52}
    André Weil
    \newblock \emph{Sur les théorèmes de De Rham.}
    \newblock{Comment. Math. Helv. 26(1952), pp. 119--145.}
    
  \end{thebibliography}
  
\end{document}

%%%%%%%%%%%%%%%%%%%%%%%%%%%%%%%%%%%%%%%%%%%%%%%%%%%%%%%%%%
%%
%% MARK: - Extra Text to Process
%%
%%%%%%%%%%%%%%%%%%%%%%%%%%%%%%%%%%%%%%%%%%%%%%%%%%%%%%%%%%
