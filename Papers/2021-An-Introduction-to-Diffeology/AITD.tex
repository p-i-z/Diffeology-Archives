%%%%%%%%%%%%%%%%%%%%%%%%%%%%%%%%%%%%%%%%%%%%%%%%%%%%%%%%%%
%%
%%  PROJECT: AITD
%%
%%  Created by Patrick Iglesias-Zemmour on 16/04/15.
%%  Copyright 2015-2020 PIZ.
%%
%%  All rights reserved.
%%
%%%%%%%%%%%%%%%%%%%%%%%%%%%%%%%%%%%%%%%%%%%%%%%%%%%%%%%%%%

\documentclass[11pt,reqno]{amsart}

%%%%%%%%%%%%%%%%%%%%%%%%%%%%%%%%%%%%%%%%%%%%%%%%%%%%%%%%%%
%%
%% MARK: ***MACROS FOR AITD***
%%
%%%%%%%%%%%%%%%%%%%%%%%%%%%%%%%%%%%%%%%%%%%%%%%%%%%%%%%%%%

%%%%%%%%%%%%%%%%%%%%%%%%%%%%%%%%%%%%%%%%%%%%%%%%%%%%%%%%%%
%%
%% MARK: MACROS Page Display
%%
%%%%%%%%%%%%%%%%%%%%%%%%%%%%%%%%%%%%%%%%%%%%%%%%%%%%%%%%%%

\parindent 0mm
\parskip .5ex plus 2pt

%%%%%%%%%%%%%%%%%%%%%%%%%%%%%%%%%%%%%%%%%%%%%%%%%%%%%%%%%%
%%
%% MARK: MACROS Environment
%%
%%%%%%%%%%%%%%%%%%%%%%%%%%%%%%%%%%%%%%%%%%%%%%%%%%%%%%%%%%

\newenvironment{proposition}{\textsc{Proposition}\textit}{}

%%%%%%%%%%%%%%%%%%%%%%%%%%%%%%%%%%%%%%%%%%%%%%%%%%%%%%%%%%
%%
%% MARK: MACROS Packages
%%
%%%%%%%%%%%%%%%%%%%%%%%%%%%%%%%%%%%%%%%%%%%%%%%%%%%%%%%%%%

\usepackage[usenames,dvipsnames]{color}

\usepackage{pdfpages}

\usepackage{graphicx}

\usepackage[margin=0pt, font=small, justification=centering, labelformat=empty]{caption}

\usepackage{pifont}

\usepackage[hidelinks]{hyperref}

% Pour figures et diagrammes
% http://texdoc.net/texmf-dist/doc/latex/tikz-cd/tikz-cd-doc.pdf
\usepackage{tikz-cd}
\usetikzlibrary{calc}
\tikzcdset{
arrow style=tikz,
diagrams={>={Straight Barb[scale=0.8]}}
}

%%%%%%%%%%%%%%%%%%%%%%%%%%%%%%%%%%%%%%%%%%%%%%%%%%%%%%%%%%%%%%%%%%%%%%%%%%%%%%%
%%
%% MARK: MACROS Style
%%
%%%%%%%%%%%%%%%%%%%%%%%%%%%%%%%%%%%%%%%%%%%%%%%%%%%%%%%%%%%%%%%%%%%%%%%%%%%%%%%

%\usepackage[french]{babel}

\usepackage{amssymb}
\usepackage{amscd}

\usepackage{verbatim}

\usepackage{url}

\usepackage{pifont}

%%%%%%%%%%%%%%%%%%%%%%%%%%%%%%%%%%%%%%%%%%%%%%%%%%%%%%%%%%%%%%%%%%%%%%%%%%%%%%%
%%
%% MARK: MACROS Articles
%%
%%%%%%%%%%%%%%%%%%%%%%%%%%%%%%%%%%%%%%%%%%%%%%%%%%%%%%%%%%%%%%%%%%%%%%%%%%%%%%%

\newtheoremstyle{article}% ⟨name⟩
{7pt}%  ⟨Space above⟩
{7pt}%  ⟨Space below⟩
{}%      ⟨Body font⟩
{0pt}%  ⟨Indent amount⟩
{\bf}%  ⟨Theorem head font⟩
{.\ }%  ⟨Punctuation after theorem head⟩
{0pt}%  ⟨Space after theorem head⟩
{}%     ⟨Theorem head spec (can be left empty, meaning ‘normal’)⟩

\theoremstyle{article}
\newtheorem{article}{}
\newtheorem{theorem}{Theorem}

\def\artlabel[#1]{{\textsc{#1}}.}
\newcommand{\artend}{\nolinebreak\hfill$\blacktriangleright$}

\renewenvironment{proof}{\noindent \textit{Proof.}} {\nolinebreak\hfill $\square$}

%%%%%%%%%%%%%%%%%%%%%%%%%%%%%%%%%%%%%%%%%%%%%%%%%%%%%%%%%%%%%%%%%%%%%%%%%%%%%%%
%%
%% MARK: MACROS Facilities
%%
%%%%%%%%%%%%%%%%%%%%%%%%%%%%%%%%%%%%%%%%%%%%%%%%%%%%%%%%%%%%%%%%%%%%%%%%%%%%%%%

\def\qmbox#1{\quad\mbox{#1}\quad}

%%%%%%%%%%%%%%%%%%%%%%%%%%%%%%%%%%%%%%%%%%%%%%%%%%%%%%%%%%%%%%%%%%%%%%%%%%%%%%%
%%
%% MARK: MACROS Bold faces
%%
%%%%%%%%%%%%%%%%%%%%%%%%%%%%%%%%%%%%%%%%%%%%%%%%%%%%%%%%%%%%%%%%%%%%%%%%%%%%%%%

\newcommand{\1}{{\hat 1}}
\newcommand{\0}{{\hat 0}}
\def \ends {\mathrm{ends}}

\renewcommand{\AA}{\mbox{\bf A}}
\newcommand{\BB}{\mbox{\bf B}}
\newcommand{\CC}{\mathbf C}
\newcommand{\DD}{\mbox{\bf D}}
\newcommand{\EE}{\mbox{\bf E}}
\newcommand{\FF}{\mbox{\bf F}}
\newcommand{\GG}{{\mathbf G}}
\newcommand{\HH}{\mbox{\bf H}}
\newcommand{\II}{\mbox{\bf I}}
\newcommand{\JJ}{\mbox{\bf J}}
\newcommand{\KK}{{\mathbf K}}
\newcommand{\LL}{\mbox{\bf L}}
\newcommand{\MM}{\mbox{\bf M}}
\newcommand{\NN}{{\mathbf N}}
\newcommand{\OO}{\mbox{\bf O}}
\newcommand{\PP}{{\mathbf P}}
\newcommand{\QQ}{\mathbf Q}
\newcommand{\RR}{\mathbf R}
\renewcommand{\SS}{\mbox{\bf S}}
\newcommand{\TT}{{\mathbf T}}
\newcommand{\UU}{\mbox{\bf U}}
\newcommand{\VV}{\mbox{\bf V}}
\newcommand{\WW}{\mbox{\bf W}}
\newcommand{\XX}{\mbox{\bf X}}
\newcommand{\YY}{\mbox{\bf Y}}
\newcommand{\ZZ}{\mathbf Z}

\newcommand{\rr}{{\bold r}}
\renewcommand{\tt}{{\bold t}}
\newcommand{\vv}{{\bold v}}
\newcommand{\xx}{{\bold x}}
\newcommand{\yy}{{\bold y}}

\newcommand{\bbeta}{\pmb{\beta}}

%%%%%%%%%%%%%%%%%%%%%%%%%%%%%%%%%%%%%%%%%%%%%%%%%%%%%%%%%%%%%%%%%%%%%%%%%%%%%%%
%%
%% MARK: MACROS Math calligraphy
%%
%%%%%%%%%%%%%%%%%%%%%%%%%%%%%%%%%%%%%%%%%%%%%%%%%%%%%%%%%%%%%%%%%%%%%%%%%%%%%%%

\newcommand{\cA}{{\mathcal A}}
\newcommand{\cB}{{\mathcal B}}
\newcommand{\cC}{{\mathcal C}}
\newcommand{\cD}{{\mathcal D}}
\newcommand{\cE}{{\mathcal E}}
\newcommand{\cF}{{\mathcal F}}
\newcommand{\cG}{{\mathcal G}}
\newcommand{\cH}{{\mathcal H}}
\newcommand{\cI}{{\mathcal I}}
\newcommand{\cJ}{{\mathcal J}}
\newcommand{\cK}{{\mathcal K}}
\newcommand{\cL}{{\mathcal L}}
\newcommand{\cM}{{\mathcal M}}
\newcommand{\cN}{{\mathcal N}}
\newcommand{\cO}{{\mathcal O}}
\newcommand{\cP}{{\mathcal P}}
\newcommand{\cQ}{{\mathcal Q}}
\newcommand{\cR}{{\mathcal R}}
\newcommand{\cS}{{\mathcal S}}
\newcommand{\cT}{{\mathcal T}}
\newcommand{\cU}{{\mathcal U}}
\newcommand{\cV}{{\mathcal V}}
\newcommand{\cW}{{\mathcal W}}
\newcommand{\cX}{{\mathcal X}}
\newcommand{\cY}{{\mathcal Y}}
\newcommand{\cZ}{{\mathcal Z}}

%%%%%%%%%%%%%%%%%%%%%%%%%%%%%%%%%%%%%%%%%%%%%%%%%%%%%%%%%%%%%%%%%%%%%%%%%%%%%%%
%%
%% MARK: MACROS Math letters
%%
%%%%%%%%%%%%%%%%%%%%%%%%%%%%%%%%%%%%%%%%%%%%%%%%%%%%%%%%%%%%%%%%%%%%%%%%%%%%%%%

\def \A{\mathrm A}
\def \B{\mathrm B}
\def \C{\mathrm C}
\def \D{\mathrm D}
\def \E{\mathrm E}
\def \F{\mathrm F}
\def \G{\mathrm G}
\def \H{\mathrm H}
\def \I{\mathrm I}
\def \J{\mathrm J}
\def \K{\mathrm K}
\def \L{\mathrm L}
\def \M{\mathrm M}
\def \N{\mathrm N}
\def \O{\mathrm O}
\def \P{\mathrm P}
\def \Q{\mathrm Q}
\def \R{\mathrm R}
\def \S{\mathrm S}
\def \T{\mathrm T}
\def \U{\mathrm U}
\def \V{\mathrm V}
\def \W{\mathrm W}
\def \X{\mathrm X}
\def \Y{\mathrm Y}
\def \Z{\mathrm Z}

%%%%%%%%%%%%%%%%%%%%%%%%%%%%%%%%%%%%%%%%%%%%%%%%%%%%%%%%%%%%%%%%%%%%%%%%%%%%%%%
%%
%% MARK: MACROS Math fraktur
%%
%%%%%%%%%%%%%%%%%%%%%%%%%%%%%%%%%%%%%%%%%%%%%%%%%%%%%%%%%%%%%%%%%%%%%%%%%%%%%%%

\newcommand{\fA}{{\mathfrak A}}
\newcommand{\fC}{{\mathfrak C}}
\newcommand{\fD}{{\mathfrak D}}
\newcommand{\fF}{{\mathfrak F}}
\newcommand{\fH}{{\mathfrak H}}
\newcommand{\fG}{{\mathfrak G}}
\newcommand{\fK}{{\mathfrak K}}
\newcommand{\fL}{{\mathfrak L}}
\newcommand{\fP}{{\mathfrak P}}
\newcommand{\fX}{{\mathfrak X}}

%%%%%%%%%%%%%%%%%%%%%%%%%%%%%%%%%%%%%%%%%%%%%%%%%%%%%%%%%%%%%%%%%%%%%%%%%%%%%%%
%%
%% MARK: MACROS Typography
%%
%%%%%%%%%%%%%%%%%%%%%%%%%%%%%%%%%%%%%%%%%%%%%%%%%%%%%%%%%%%%%%%%%%%%%%%%%%%%%%%

\renewcommand{\over}{\above .2pt}
\newcommand{\hs}{\hspace{2mm}}

%%%%%%%%%%%%%%%%%%%%%%%%%%%%%%%%%%%%%%%%%%%%%%%%%%%%%%%%%%%%%%%%%%%%%%%%%%%%%%%
%%
%% MARK: MACROS Abbreviations
%%
%%%%%%%%%%%%%%%%%%%%%%%%%%%%%%%%%%%%%%%%%%%%%%%%%%%%%%%%%%%%%%%%%%%%%%%%%%%%%%%

\newcommand{\Ad}{\mathrm{Ad}}

\newcommand{\class}{\mathrm{class}\,}

\newcommand{\CHK}{\mathrm{K}}
\newcommand{\Cinfty}{\mathcal{C}^\infty}

\newcommand{\Def}{\mathrm{def}\,}
\newcommand{\Diff}{\mathrm{Diff}}
\newcommand{\dom}{\mathrm{dom}}
\newcommand{\dR}{\mathrm{dR}}
\newcommand{\dt}{{dt}}

\newcommand{\ev}{\mathrm{ev}}

\newcommand{\grad}{\mathrm{grad}}

\newcommand{\Geod}{\mathrm{Geod}}
\newcommand{\GL}{\mathrm{GL}}

\newcommand{\Ham}{\mathrm{Ham}}
\newcommand{\Hom}{\mathrm{Hom}}

\newcommand{\id}{\mathbf 1}

\newcommand{\loc}{\mathrm{loc}}

\newcommand{\Loops}{\mathrm{Loops}\,}

\newcommand{\Maps}{\mathrm{Maps}}

\newcommand{\Mor}{\mathrm{Mor}}

\newcommand{\Obj}{\mathrm{Obj}}
\renewcommand{\O}{\mathop{\rm O}\nolimits}

\newcommand{\Paths}{\mathrm{Paths}\,}
\newcommand{\Params}{\mathrm{Params}\,}

\newcommand{\pr}{\mathrm{pr}}

\newcommand{\src}{\mathrm{src}}
\newcommand{\Sup}{\mathop{\rm Sup}\nolimits}
\newcommand{\Surf}{\mathrm{Surf}}
\newcommand{\Sq}{\mathrm{Sq}}

\newcommand{\tr}{\mathop{\rm tr}\nolimits}
\newcommand{\trg}{\mathrm{trg}}

\newcommand{\vect}[1]{\left(\begin{array}{c}#1\end{array}\right) }

\usepackage[greek,english]{babel}

%%%%%%%%%%%%%%%%%%%%%%%%%%%%%%%%%%%%%%%%%%%%%%%%%%%%%%%%%%%%%%%%%%%%%%%%%%%%%%%
%%
%% MARK: MACROS General Layout
%%
%%%%%%%%%%%%%%%%%%%%%%%%%%%%%%%%%%%%%%%%%%%%%%%%%%%%%%%%%%%%%%%%%%%%%%%%%%%%%%%

\usepackage[T1]{fontenc}
\usepackage[utf8]{inputenc}

%%%%%%%%%%%%%%%%%%%%%%%%%%%%%%%%%%%%%%%%%%%%%%%%%%%%%%%%%%%%%%%%%%%%%%%%%%%%%%%
%%<<< Begin - Utopia style
%%%%%%%%%%%%%%%%%%%%%%%%%%%%%%%%%%%%%%%%%%%%%%%%%%%%%%%%%%%%%%%%%%%%%%%%%%%%%%%

\usepackage{microtype}
\usepackage[cal=scr,uppercase = upright,greekfamily = didot,greeklowercase = upright,utopia]{mathdesign}
\usepackage[supspaced=.04em]{superiors}

\linespread{1.1}

\def \H{\mathrm H}
\def \L{\mathrm L}
\def \O{\mathrm O}
\let \epsilon=\varepsilon

%%%%%%%%%%%%%%%%%%%%%%%%%%%%%%%%%%%%%%%%%%%%%%%%%%%%%%%%%%
%%
%%
%% Mark:  ***DOCUMENT AITD***
%%
%%
%%%%%%%%%%%%%%%%%%%%%%%%%%%%%%%%%%%%%%%%%%%%%%%%%%%%%%%%%%

\begin{document}
  
  \title{An Introduction To Diffeology}
  \author{Patrick Iglesias-Zemmour}
  
  %%% À inclure pour le preprint
  \date{August 25, 2017 -- Revision May 15,2020} % \today}
  
  \address{Einstein Institute of Mathematics,
  The Hebrew University of Jerusalem,
  Campus Givat Ram,
  9190401 Israel.}
  \email{piz@math.huji.ac.il}
  \urladdr{http://math.huji.ac.il/~piz}
  
  %%%%  À inclure pour le preprint
    \thanks{Text of the talk given at the Conference \href{https://ercpqg-espace.sciencesconf.org}{\color{Violet}{``New Spaces in Mathematics \& Physics''}},
    September 28 – October 2, 2015
    Institut Henri Poincaré Paris,
    France}
  
  \maketitle
  
  %%%%%%%%%%%%%%%%%%%%%%%%%%%%%%%%%%%%%%%%%%%%%%%%%%%%%%%%%%
  %%
  %% MARK: DOCUMENT Abstract
  %%
  %%%%%%%%%%%%%%%%%%%%%%%%%%%%%%%%%%%%%%%%%%%%%%%%%%%%%%%%%%
  
  \begin{abstract}
    This text presents the basics of Diffeology and the main domains:
    Homotopy,
    Fiber Bundles,
    Quotients,
    Singularities,
    Cartan-de Rham Calculus
    ---~which form the core of differential geometry~---
    from the point of view of this theory.
    We show what makes diffeology special and relevant in regard to these traditional subjects.
  \end{abstract}
  
  %%%%%%%%%%%%%%%%%%%%%%%%%%%%%%%%%%%%%%%%%%%%%%%%%%%%%%%%%%
  %%
  %% MARK: DOCUMENT Introduction
  %%
  %%%%%%%%%%%%%%%%%%%%%%%%%%%%%%%%%%%%%%%%%%%%%%%%%%%%%%%%%%
  \section*{Introduction}
  
  
  Since its creation,
  in the early 1980s,
  \emph{Diffeology} has become an alternative,
  or rather a natural extension of traditional differential geometry.
  With its developments in higher homotopy theory,
  fiber bundles,
  modeling spaces,
  Cartan-de Rham calculus,
  moment map and symplectic%
  \footnote{About what it means ``to be symplectic'' in diffeology see \cite{PIZ10,PIZ16,PIZ17}.} program,
  for examples,
  diffeology now covers a large spectrum of traditional fields and deploys them from singular quotients to infinite-dimensional spaces
  ---~and mixes the two~---
  treating mathematical objects that are or are not strictly speaking manifolds,
  and other constructions,
  on an equal footing in a common framework.
  We shall see some of its achievements through a series of examples,
  chosen because they are not covered by the geometry of manifolds,
  because they involve infinite-dimensional spaces or singular quotients,
  or both.
  
  The growing interest in diffeology comes from the conjunction of two strong properties of the theory:
  
  \begin{enumerate}
    \item[1.] Mainly,
    the category \{Diffeology\} is stable under all set-theoretic constructions:
    sums, products, subsets,
    and quotients.
    One says that it is a complete and co-complete category.
    It is also Cartesian closed,
    the space of smooth maps in diffeology has itself a natural \emph{functional diffeology}.
    \item[2.] Just as importantly:
    quotient spaces,
    even non-Hausdorff,%
    \footnote{\label{note1}Trivial under the various generalizations of\, $\C^\infty$ differential geometry à la Sikorski or Frölicher \cite{KriMic97} etc.
    I am not considering the various algebraic generalizations that do not play on the same level of intuition and generality and do not concern exactly the same sets/objects,
    lattices instead of quotients etc.}
    get a natural non-trivial and meaningful diffeology.
    That is in particular the case of \emph{irrational tori},
    non-Hausdorff quotients of the real line by dense subgroups.
    They own,
    as we shall see,
    a non trivial diffeology,
    capturing faithfully the intrication of the subgroup into its ambient space.
    This crucial property will be the raison d'être of many new constructions,
    or wide generalizations of classical constructions,
    that cannot exist in almost all extensions of differential geometry.\textsuperscript{\ref{note1}}
  \end{enumerate}
  
  The treatment of any kind of singularities,
  maybe more than the inclusion of infinite-dimensional spaces,
  reveals how diffeology changes the way we understand smoothness and discriminates this theory among the various alternatives;
  see,
  for example,
  the use of dimension in diffeology \cite{PIZ07},
  which distinguishes between the different quotients $\RR^n\!/\O(n)$.
  
  The diagram in Figure \ref{The-Scope-of-Diffeology} shows the inclusivity of diffeology,
  with respect to differential constructions,
  in comparison with the classical theory.
  \begin{figure}[t]
    \includegraphics[width=.95\textwidth]{Figures/The-Scope-of-Diffeology.pdf}
    \vspace{-.25\baselineskip}
    \caption{Figure \ref{The-Scope-of-Diffeology} --- The scope of diffeology.}
    \label{The-Scope-of-Diffeology}
  \end{figure}
  
  \textsc{Connecting a Few Dots.}~The story began in the early 1980s,
  when Jean-Marie Souriau introduced his \emph{difféologies} in a paper titled ``Groupes Différentiels'' \cite{Sou80}.
  It was defined as a formal but light structure on groups,%
  \footnote{Compared to functional analysis heavy structures.}
  and it was designed for dealing easily with infinite-dimensional groups of diffeomorphisms,
  in particular the group of symplectomorphisms or quantomorphisms.
  He named the groups equipped with such a structure \emph{groupes différentiels},%
  \footnote{Which translates in to English as ``differential'' or ``differentiable groups.''}
  as announced in the title of his paper.%
  \footnote{Actually,
  difféologies are built on the model of K.-T. Chen's \emph{differentiable spaces} \cite{Che77},
  for which the structure is defined over convex Euclidean subsets instead of open Euclidean domains.
  That makes diffeology more suitable to extending differential geometry than Chen's differentiable spaces,
  which focus more on homology and cohomology theories.}
  His definition was made of five axioms that we can decompose today into a first group of three that gave later the notion of diffeology on arbitrary sets,
  and the last two,
  for the compatibility with the internal group multiplication.
  But it took three years,
  from 1980 to 1983,
  to separate the first three general axioms from the last two specific ones and to extract the general structure of \emph{espace différentiel} from the definition of \emph{groupe différentiel}.
  That was the ongoing work of Paul Donato on the covering of \emph{homogeneous spaces} of differential groups,
  for one part,
  and mostly our joint work on the \emph{irrational torus},
  which made urgent and unavoidable a formal separation between groups and spaces in the domain of Souriau's \emph{differential structures},
  as that gave a new spin to the theory.
  Actually,
  the first occurrence of the wording ``espace différentiel'',
  including quotes,
  appears in the paper ``Exemple de groupes différentiels~: flots irrationnels sur le tore,''
  published in July 1983 \cite{DonIgl83}.%
  \footnote{The final version has been published in the French Acad.  Sci. Proc. \cite{DonIgl85}.}
  The expression was used informally,
  for the purposes of the case,
  without giving a precise definition.
  The formal definition was published a couple of months later,
  in October 1983,
  in ``Groupes différentiels et physique mathématique'' \cite{Sou83}.
  It took then a couple of years to bring the theory of \emph{espaces différentiel} to a new level:
  with Souriau on generating quantum structures \cite{Sou84};
  with Donato's doctoral dissertation on covering of homogeneous spaces,
  defended in 1984 \cite{Don84},
  and with my doctoral dissertation,
  defended in 1985,
  in which I troduced higher homotopy theory and diffeological fiber bundles%
  \footnote{That is in this thesis that the first occurence of \emph{diffeological space} appears,
  in replacement of \emph{differential space} which would have been confused with other structures.}
  \cite{Igl85}.
  
  \bigskip
  
  What follows is an attempt to introduce the main constructions and results in diffeology,
  past and recent,
  through several meaningful examples.
  More details on the theory can be found in the textbook \emph{Diffeology} \cite{PIZ13} and in related papers.
  
  %\newpage
  
  %%%%%%%%%%%%%%%%%%%%%%%%%%%%%%%%%%%%%%%%%%%%%%%%%%%%%%%%%%
  %%
  %% MARK: DOCUMENT The Unexpected Example
  %%
  %%%%%%%%%%%%%%%%%%%%%%%%%%%%%%%%%%%%%%%%%%%%%%%%%%%%%%%%%%
  \section*{The Unexpected Example: The Irrational Torus}
  
  Let us begin with the \emph{irrational torus} $\T_\alpha$.
  At this time,
  in the early 1980s,
  physicists were interested in quantizing one-dimensional systems with a quasi-periodic potential,
  that is,
  a function $u$ from $\RR$ to $\RR$ which is the pullback of a smooth function $\U$ on a torus $\T^2 = \RR^2/\ZZ^2$ along a line of slope $\alpha$,
  with $\alpha \in \RR - \QQ$.
  If you prefer,
  $u(x)=\U(e^{2i\pi x},e^{2i\pi\alpha x})$,
  where $\U \in \Cinfty(\T^2,\RR)$.
  That problem has drawn the attention of physicists and some mathematicians to the question of the statute of the quotient space
  $$
    \T_\alpha = \T^2/\Delta_\alpha,
    $$
  where $\Delta_\alpha \subset \T^2$ is a one-parameter subgroup,
  the projection from $\RR^2$ of the line $y=\alpha x$,
  that is,
  $\Delta_\alpha = \{(e^{2i\pi x},e^{2i\pi\alpha x})\}_{x\in \RR}$.
  
  As a topological space,
  $\T_\alpha$ is trivial,
  because $\alpha$ is irrational and $\Delta_\alpha$ ``fills'' the torus,
  that is,
  its closure is $\T^2$.
  And a trivial topological space is of no help.
  The various differentiable approaches%
  \footnote{Sikorski,
  Frölicher,
  ringed spaces,
  etc.}
  lead also to dead ends;
  the only smooth maps from $\T_\alpha$ to $\RR$ are constant maps,
  because the composition with the projection $\pi \colon \T^2 \to \T_\alpha$ should be smooth.
  For these reasons,
  the irrational torus was regarded by everyone as an extremely \emph{singular} space.
  
  But when we think a little bit,
  the irrational torus $\T_\alpha$ is a group,
  moreover,
  an Abelian group,
  and there is nothing more regular and homogeneous than a group.
  And that was the point that made us,%
  \footnote{Paul Donato and I.}
  eager to explore $\T_\alpha$ through the approach of diffeologies.
  But diffeologies were invented to study infinite-dimensional groups,
  such as groups of symplectomorphisms,
  and it was not clear that they could be of any help for the study of such ``singular'' spaces as $\T_\alpha$,
  except that,
  because a diffeology on a set is defined by its \emph{smooth parameterizations},
  and because smooth parameterizations on $\T_\alpha$ are just (locally) the composites of smooth parameterizations into $\T^2$ by the projection $\pi \colon \T^2 \to \T_\alpha$,
  it was already clear that the diffeology of $\T_\alpha$ was not the \emph{trivial diffeology} made of all parameterizations,
  and neither was it the \emph{discrete diffeology} made only of locally constant parameterizations.
  Indeed,
  considering two smooth parameterizations%
  \footnote{A \emph{parameterization} in a set $\X$ is just any map $\P$ defined on some open subset $\U$ of some numerical space $\RR^n$ into $\X$.}
  $\P$ and $\P'$ in $\T^2$,
  there is a great chance that $\pi \circ \P$ and $\pi \circ \P'$ are different.
  And because $\P$ and $\P'$ are not any parameterizations but smooth ones,
  that makes $\T_\alpha$ neither trivial nor discrete.
  That was already a big difference with the traditional topological or differential approches we mentioned above that makes $\T_\alpha$ coarse.
  
  But how to measure this non triviality?
  To what ends did this new approach lead?
  That was the true question.
  We gave some answers in the paper ``Exemple de groupes différentiels: flots irrationnels sur le tore'' \cite{DonIgl83}.
  Thanks to what Paul Donato had already developed at this time on the covering of \emph{homogeneous differential spaces},
  and which made the core of his doctoral dissertation \cite{Don84},
  we could compute the fundamental group of $\T_\alpha$ and its universal covering $\tilde \T_\alpha$.
  We found that
  $$
    \pi_1(\T_\alpha) = \ZZ \times \ZZ \quad\text{and}\quad \tilde \T_\alpha = \RR,
    $$
  with $\pi_1(\T_\alpha)$ included in $\RR$ as $\ZZ + \alpha \ZZ$.
  That was a first insightful result showing the capability of diffeology concerning these spaces regarded ordinarily as (highly) singular.
  Needless to say,
  since then,
  they become completely admissible.
  
  In the second half of the 1970s,
  the quantum mechanics of quasi-periodic potentials hit the field of theoretical physics%
  \footnote{The most famous paper on the question was certainly from Dinaburg and Sinai,
  on ``The One-Dimensional Schr{\"o}dinger Equation with a Quasiperiodic Potential" \cite{DinSin75}.}
  and the question of the structure of the space of leaves of the linear foliation of the $2$-torus sparked a new level of interest.
  In particular,
  French theoretical physicists used the techniques of \emph{noncommutative geometry} developed by Alain Connes.
  So,
  the comparison between the two theories became a natural question.
  The notion of fundamental group or universal covering was missing at that time in noncom\-mutative geometry,%
  \footnote{And are still missing today,
  except for a few attempts to fill in the holes \cite{IZL16}.}
  so it was not with these invariants that we could compare the two approaches.
  That came eventually from the following result:
  
  \begin{theorem}[Donato-Iglesias, 1983]
    \emph{Two irrational tori $\T_\alpha$ and $\T_\beta$ are diffeomorphic if and only if $\alpha$ and $\beta$ are conjugate modulo $\GL(2,\ZZ)$,
    that is, if there exists a matrix
    $$
      \M = \begin{pmatrix} a & b \cr c & d \end{pmatrix} \in \GL(2,\ZZ) \quad\text{such that}\quad \beta = {a \alpha + b \over c\alpha + d}.
      $$
    }
  \end{theorem}
  
  That result had its correspondence in Connes's theory,
  due to Marc Rieffel \cite{Rie81}:
  the $\CC^*$-algebras associated with $\alpha$ and $\beta$ are Morita-equivalent if and only if $\alpha$ and $\beta$ are conjugate modulo $\GL(2,\ZZ)$.
  At this moment,
  it was clear that diffeology was a possible alternative to noncommutative geometry.
  Its advantage was to stay close to the special intuition and concepts developed by geometers along history.
  
  Actually,
  because of irrationality of  $\alpha$ and $\beta$,
  we showed that a diffeomorphism $\varphi \colon \T_\alpha \to \T_\beta$ could be fully lifted at the level of the covering $\RR^2$ of $\T^2$ into an affine diffeomorphism $\Phi(Z) = AZ + B$,
  where $\A$ preserves the lattice $\ZZ^2 \subset \RR^2$,
  that is,
  $\A \in \GL(2,\ZZ)$,
  and $\B \in \RR^2$.
  The fact that these natural isomorphisms are preserved \emph{a minima} in diffeology was of course an encouragement to continue the exploration of this example,
  and to push the test of diffeology even further.
  
  \medskip
  \begin{figure}[t]
    \includegraphics[width=.90\textwidth]{Figures/T2-MDelta.pdf}
    \vspace{-.25\baselineskip}
    \caption{Figure \ref{A-diffeomorphism-of-Talpha} --- A diffeomorphism from $\T_\alpha$ to $\T_\beta$.}
    \label{A-diffeomorphism-of-Talpha}
  \end{figure}
  
  And that is not all that could be said,
  and has been said,
  on irrational tori.
  Indeed,
  the category \{Diffeology\} has many nice properties:
  in particular,
  as we shall see in the following,
  it is Cartesian closed.
  That means in particular that the set of smooth maps between diffeological spaces has a natural diffeology.
  We call it the \emph{functional diffeology}.
  Thus any set of smooth maps between diffeological spaces has a fundamental group,
  in particular the group $\Diff(\T_\alpha)$ of diffeomorphisms of $\T_\alpha$.
  And its computation gives us another surprise:
  
  \begin{theorem}[Donato-Iglesias, 1983]
    \emph{The connected component of $\Diff(\T_\alpha)$ is $\T_\alpha$,
    acting by multiplication on itself.
    Its group of connected components is}
    $$
      \pi_0(\Diff(\T_\alpha)) =
      \left\{
      \begin{array}{l}
      \{\pm 1\} \times \ZZ \quad \text{if $\alpha$ is \emph{quadratic}} \\
      \{\pm 1\} \quad\text{otherwise}.
      \end{array}
      \right.
      $$
  \end{theorem}
  
  We recall that a number is quadratic if it is a solution of a quadratic polynomial with integer coefficients.
  That result was indeed discriminating,
  since it did not appear in any other theory pretending to extend ordinary differential geometry.
  That clearly showed that diffeology,
  even for such a twisted example,
  was subtle enough to distinguish between numbers, quadratic or not.
  This property was not without reminding us about the periodicity of the continued fraction of real numbers.
  
  This computation has been generalized on the $\pi_0(\Diff(\T_\H))$ \cite{IglLac90},
  where $\H \subset \RR^n$ is a totally irrational hyperplane and $\T_\H= \T^n/\H$.
  
  \begin{theorem}[Iglesias-Lachaud, 1990]
    \emph{Let $\H \subset \RR^n$ be a totally irrational hyperplane,
    that is,
    $\H \cap \ZZ^n = \{0\}$.
    Let $\T_\H = \T^n/\pr(\H)$,
    where $\pr$ is the projection from $\RR^n$ to $\T^n$.
    Let $w=(1,w_2,\dots,w_n)$ be the normalized $1$-form such that $\H = \ker(w)$.
    The coefficients $w_i$ are independent on $\QQ$.
    Let $\E_w = \QQ + w_2 \QQ + \dots + w_n \QQ \subset \RR$ be the
    $\QQ$-vector subspace of $\RR$ generated by the $w_i$.
    The subset
    $$
      \K_w = \{ \lambda \in \RR \mid \lambda \E_w \subset \E_w \}
      $$
    is an algebraic number field,
    a finite extension of $\QQ$.
    Then,
    $\pi_0(\Diff(\T_\H))$ is the group of the units of an order of $\K_w$.
    Thanks to the Dirichlet theorem,
    $$
      \pi_0(\Diff(\T_\H)) \simeq \{\pm 1\} \times \ZZ^{r+s-1},
      $$
    where $r$ and $s$ are the number of real and complex places of $\K_w$.
    }
  \end{theorem}
  
  Now,
  since we have seen what diffeology is capable of,
  we may have gotten your attention,
  and it is time to give some details on what exactly a \emph{diffeology} on an arbitrary set is.
  Then we shall see other applications of diffeology and examples,
  some of them famous.
  
  %%%%%%%%%%%%%%%%%%%%%%%%%%%%%%%%%%%%%%%%%%%%%%%%%%%%%%%%%%
  %%
  %% MARK:  DOCUMENT Category Diffeology
  %%
  %%%%%%%%%%%%%%%%%%%%%%%%%%%%%%%%%%%%%%%%%%%%%%%%%%%%%%%%%%
  \section*{What is a Diffeology?}
  
  It is maybe time to give a precise meaning to what we claimed on the (one-dimensionnal) irrational torus.
  As a preamble,
  let us say that,
  contrarily to many constructions in differential geometry,
  diffeologies are defined just on sets
  --~dry sets~--
  without any preexisting structure,
  neither topology nor anything else.
  That is important enough to be underlined,
  and that is also what makes the difference with the other approaches.
  A diffeology on a set $\X$ consists in declaring what parameterizations are smooth.
  Let us first introduce formally a fundamental word of this theory.
  
  \begin{article}\artlabel[Parametrization]
    We call \emph{parameterization} in a set $\X$ any map $\P \colon \U \to \X$ such that $\U$ is some open subset of a Euclidean space.
    If we want to be specific,
    we say that $\P$ is an $n$-parameterization when $\U$ is an open subset of $\RR^n$.
    The set of all parameterizations in $\X$ is denoted by $\Params(\X)$.
    \artend
  \end{article}
  
  Note that there is no condition of injectivity on $\P$,
  neither any topology precondition on $\X$ a priori.
  And also:
  we shall say \emph{Euclidean domain} for ``open subset of an Euclidean space.''
  Now,
  
  \begin{article}\artlabel[Definition of a diffeology]
    A \emph{diffeology} on $\X$ is any subset $\cD$ of $\Params(\X)$ that satisfies the following axioms:
    \begin{enumerate}
      \item[1.] \textsc{Covering}: $\cD$ contains the constant parameterizations.
      \item[2.] \textsc{Locality}: Let $\P$ be a parameterization in $\X$.
      If,
      for all $r \in \dom(\P)$,
      there is an open neighborhood $\V$ of $r$ such that $\P \restriction \V \in \cD$,
      then $\P \in \cD$.
      \item[3.] \textsc{Smooth compatibility}: For all $\P \in \cD$,
      for all $\F \in \Cinfty(\V,\dom(\P))$,
      where $\V$ is a Euclidean domain,
      $\P \circ \F \in \cD$.
    \end{enumerate}
    
    A space equipped with a diffeology is called a \emph{diffeological space}.
    The elements of the diffeology of a diffeological space are called the \emph{plots} of (or in) the space.%
    \footnote{There is a discussion about diffeology as a sheaf theory in \cite[Annex]{Igl87}.
    But we do not develop this formal point of view in general,
    because the purpose of diffeology is to minimize the technical tools in favour of a direct,
    more geometrical,
    intuition.}
    \artend
  \end{article}
  
  The first and foremost examples of diffeological spaces are the Euclidean domains,
  equipped with their \emph{smooth diffeology},
  that is,
  the ordinary smooth para\-metri\-za\-tions.
  Pick,
  for example,
  the smooth $\RR^2$:
  we have a diffeology on $\T^2 = \RR^2/\ZZ^2$ by lifting locally the parameterizations in $\RR^2$.
  That is,
  a plot of $\T^2$ will be a parameterization $\P \colon r \mapsto (z_r,z'_r)$ such that,
  for every point in the domain of $\P$,
  there exist two smooth parameterizations $\theta$ and $\theta'$, in $\RR$,
  defined in the neighborhood of this point,
  with $(z_r,z'_r)= (e^{2i\pi\theta(r)},e^{2i\pi\theta'(r)})$,
  that is,
  the usual diffeology that makes $\T^2$ the manifold we know.
  But that procedure can be extended naturally to $\T_\alpha$.
  Indeed,
  a parameterization $\P \colon r \mapsto \tau_r$ in $\T_\alpha$ is a plot if there exists locally,
  in the neighborhood of every point in the domain of $\P$,
  a parameterization $\zeta \colon r \mapsto (z_r,z'_r)$,
  such that $\tau_r= \pi(\zeta(r))$.
  That construction is summarized by the sequence of arrows:
  $$
    \RR^2 \xrightarrow{\quad \mbox{$\pr$} \quad} \T^2 \xrightarrow{\quad \mbox{$\pi$} \quad} \T_\alpha.
    $$
  That is exactly the diffeology we consider when we talk about the irrational torus.
  But this construction of diffeologies by \emph{pushforward} is actually one of the fundamental constructions of the theory.
  However,
  to go there,
  we need first to introduce an important property of diffeologies.
  
  \begin{article}\artlabel[Comparing diffeologies]
    Inclusion defines a partial order in diffeology.
    If $\cD$ and $\cD'$ are two diffeologies on a set $\X$,
    one says that $\cD$ is finer than $\cD'$ if $\cD \subset \cD'$,
    or $\cD'$ is coarser than $\cD$.
    Moreover,
    diffeologies are stable by intersection,
    which gives the following property:
    
    \textsc{Proposition}. \textit{This partial order,
    called fineness,
    makes the set of diffeologies on a set $\X$,
    a \emph{lattice}.
    That is,
    every set of diffeologies has an infimum and a supremum.
    As usual,
    the infimum of a family is obtained by intersection of the elements of the family,
    and the supremum is obtained by intersecting the diffeologies coarser than any element of the family.}
    
    The infimum of every diffeology,
    the finest diffeology,
    is called the \emph{discrete diffeology}.
    It consists of locally constant parameterizations.
    The supremum of every diffeology,
    the coarsest diffeology,
    is called the \emph{coarse} or \emph{trivial diffeology},
    and it is made of all the parameterizations.
    As we shall see,
    these bounds will be useful for defining diffeologies by properties (Boolean functions).
    \artend
  \end{article}
  
  Now,
  we can write the construction by pushforward:
  
  \begin{article}\artlabel[Pushing forward diffeologies] Let $f \colon \X \to \X'$ be a map,
    and let $\X$ be a diffeological space,
    with diffeology $\cD$.
    Then,
    there exists a finest diffeology on $\X'$ such that $f$ is smooth.
    It is called the \emph{pushforward} of the diffeology of $\X$.
    We denote it by $f_*(\cD)$.
    If $f$ is surjective,
    its plots are the parameterizations $\P$ in $\X'$ that can be written $\Sup_i f \circ \P_i$,
    where the $\P_i$ are plots of $\X$ such that the $f \circ \P_i$ are compatible,
    that is,
    coincide on the intersection of their domains,
    and $\Sup$ denotes the smallest common extension of the family $\{f\circ\P_i\}_{i\in \cI}$.
    
    In particular,
    the diffeology of $\T^2$ is the pushforward of the smooth diffeology of $\RR^2$ by $\pr$,
    and the diffeology on $\T_\alpha$ is the pushforward of the diffeology of $\T^2$ by $\pi$,
    or,
    equivalently,
    the pushforward of the smooth diffeology of $\RR^2$ by the projection $\pi \circ \pr$.
    
    \textsc{Note 1}. Let $\pi \colon \X \to \X'$ be a map between diffeological spaces.
    We say that $\pi$ is a \emph{subduction} if it is surjective and if the pushforward of the diffeology of $\X$ coincides with the diffeology of $\X'$.
    In particular $\pr \colon \RR^2 \to \T^2$ and $\pi \colon \T^2 \to \T_\alpha$ are two subductions.
    Subductions make a subcategory,
    since the composite of two subductions is again a subduction.
    \artend
    
    \textsc{Note 2}. Let $\X$ be a diffeological space and $\sim$ be an equivalence relation on $\X$.
    Let $\Q = \X/\!\!\sim$ be the quotient set,%
    \footnote{I regard always a quotient set as a subset of the set of all subsets of a set.}
    that is,
    $$
      \Q = \{\class(x) \mid x \in \X \} \quad\text{and}\quad \class(x) = \{ x' \mid x' \sim x \}.
      $$
    The pushforward on  $\Q$ of the diffeology of $\X$ by the projection $\class$ is called the \emph{quotient diffeology}.
    Equipped with the quotient diffeology,
    $\Q$ is called the \emph{quotient space} of $\X$ by $\sim$.
    This is the first important property of the category \{Diffeology\},
    it is closed by quotient.
  \end{article}
  
  Then,
  after having equipped the irrational tori with a diffeology (actually the quotient diffeology),
  we would compare different irrational tori with respect to diffeomorphisms.
  For that,
  we need a precise definition.
  
  \begin{article}\artlabel[Smooth maps]
    Let $\X$ and $\X'$ be two diffeological spaces.
    A map $f \colon \X \to\X'$ is \emph{smooth} if,
    for any plot $\P$ in $\X$,
    $f \circ \P$ is a plot in $\X'$.
    The set of smooth maps from $\X$ to $\X'$ is denoted,
    as usual,
    by $\Cinfty(\X,\X')$.
    
    \textsc{Note 1.} The composition of smooth maps is smooth.
    Diffeological spaces,
    together with smooth maps,
    make a category that we denote by \{Diffeology\}.
    
    \textsc{Note 2}. The isomorphisms of the category \{Diffeology\} are the bijective maps,
    smooth as well as their inverses. They are called \emph{diffeomorphisms}.
    \artend
  \end{article}
  
  \begin{article}\artlabel[What about manifolds]
    The time has come to make a comment on manifolds.
    Every manifold is naturally a diffeological space;
    its plots are the smooth parameterizations in the usual sense.
    That makes the category \{Manifolds\} a full and faithful subcategory of \{Diffeology\}.
    But we should insist on diffeology not be understood as a generalization of the theory of manifolds.
    It happens that,
    between many other things,
    diffeology extends the theory of manifolds,
    but its true nature is to extend the differential calculus on domains in Euclidean spaces,
    and is the way it should be regarded.
    \artend
  \end{article}
  
  Let us continue to explore our example of the irrational torus $\T_\alpha$.
  We still have to describe its fundamental group and its universal covering.
  Actually,
  the way it was treated in the founding paper \cite{DonIgl83} used special definitions adapted only to groups and homogeneous spaces,
  because at this time,
  diffeology was only about groups.
  It was clear at this moment that considering diffeology only on groups was insufficient,
  and that we missed a real independent theory of fiber bundles and homotopy in diffeology.
  That was the content of my doctoral dissertation ``Fibrations difféologiques et homotopie'' \cite{Igl85}.
  
  Let us begin with the covering thing.
  A covering is a special kind of fiber bundle with a discrete fiber.
  But all these terms must be understood in the sense of diffeology,
  especially the word \emph{discrete}.
  Let me give an example:
  
  \textsc{Proposition.}~\textit{The rational numbers $\QQ$ are discrete in $\RR$.}
  
  This is completely natural for everyone,
  except that this is false as far as topology is concerned.
  But we are talking diffeology:
  if we equip $\QQ$ with the diffeology \emph{induced} by $\RR$,
  it is not difficult to prove%
  \footnote{A nice application of the intermediate value theorem.}
  that $\QQ$ is discrete,
  that is,
  its diffeology is discrete.
  And that is the meaning we want to give to be a \emph{discrete subset} of a diffeological space.
  Well,
  we still have to elaborate a little bit about \emph{induced diffeology}.
  
  \begin{article}\artlabel[Pulling back diffeologies]
    Let $f \colon \X \to \X'$ be a map,
    and let $\X'$ be a diffeological space with diffeology $\cD'$.
    Then,
    there exists a coarsest diffeology on $\X$ such that $f$ is smooth.
    It is called the \emph{pullback} of the diffeology of $\X'$.
    We denote it by $f^*(\cD')$.
    Its plots are the parameterizations $\P$ in $\X$ such that $f \circ \P$ is a plot of $\X'$.
    
    In particular,
    that gives to any subset $\A \subset \X$,
    where $\X$ is a diffeological space,
    a \emph{subset diffeology},
    that is,
    $j^*(\cD)$,
    where $j \colon \A \to \X$ is the inclusion and $\cD$ is the diffeology of $\X$.
    A subset equipped with the subset diffeology is called a \emph{diffeological subspace}.
    Now it is clear what is meant by a \emph{discrete subset} of a diffeological space:
    it is a subset such that its induced diffeology is discrete.
    
    \textsc{Note}.
    Let $j \colon \X \to \X'$ be a map between diffeological spaces.
    We say that $j$ is an \emph{induction} if $j$ is injective and if the pullback of the diffeology of $\X'$ coincides with the diffeology of $\X$.
    For example,
    in the case of the irrational torus,
    the injection $t \mapsto (e^{2i\pi t},e^{2i\pi\alpha t})$ from $\RR$ to $\T^2$ is an induction.
    That means precisely that if $r \mapsto (z_r,z'_r)$ is smooth in $\T^2$ but takes its values in $\Delta_\alpha$,
    then there exists a smooth parameterization $r \mapsto t_r$ in $\RR$ such that $z_r = e^{2i\pi t_r}$ and $z'_r = e^{2i\pi \alpha t_r}$.
    \artend
  \end{article}
  
  Then,
  with this construction,
  the category \{Diffeology\},
  which was closed by quotient,
  is also closed by inclusion.
  
  Now,
  continuing with our example,
  there are two reasons we need a good concept of fiber bundle in diffeology.
  
  \begin{enumerate}
    \item[1.] Coverings of diffeological spaces should be defined as fiber bundles with discrete fiber.
    (We shall see then that we are even able to build a universal covering,
    unique up to isomorphism,
    for every diffeological space.)
    
    \item[2.] If we look closely at the projection $\pi \colon \T^2 \to \T_\alpha$,
    we observe that this looks like a fiber bundle with fiber $\Delta_\alpha \simeq \RR$.
    Thus,
    if the long homotopy sequence could apply to diffeological fiber bundles,
    we would get immediately $\pi_1(\T_\alpha) = \pi_1(\T^2) = \ZZ^2$,
    since the fiber $\RR$ is contractible.
  \end{enumerate}
  
  %%%%%%%%%%%%%%%%%%%%%%%%%%%%%%%%%%%%%%%%%%%%%%%%%%%%%%%%%%
  %%
  %% MARK: DOCUMENT Fiber Bundles
  %%
  %%%%%%%%%%%%%%%%%%%%%%%%%%%%%%%%%%%%%%%%%%%%%%%%%%%%%%%%%%
  \section*{Fiber Bundles}
  
  Of course,
  the classical definition of locally trivial fiber bundles is powerless here,
  since $\T_\alpha$ has a trivial topology.
  The situation is more subtle~--~we are looking for a definition%
  \footnote{The definition of fiber bundles in diffeology
  ---~their two equivalent versions~---
  has been introduced in ``Fibrés difféologiques et homotopie'' \cite{Igl85}.}
  that satisfies the following two conditions:
  
  \begin{enumerate}
    \item[1.] The quotient of a diffeological group by any subgroup is a diffeological fibration,
    whatever the subgroup is.
    
    \item[2.] For diffeological fibrations,
    the long exact homotopy sequence applies.
  \end{enumerate}
  
  Since we refer to diffeological groups,
  we have to clarify their definition.
  
  \begin{article}\artlabel[Diffeological groups]
    A diffeological group is a group $\G$ that is also a diffeological space such that,
    the multiplication $(g,g') \mapsto gg'$ and the inversion $g \mapsto g^{-1}$ are smooth.
    \artend
  \end{article}
  
  That needs a comment on the product of diffeological spaces,
  since we refer to the multiplication in a diffeological group $\G$ which is defined on the product $\G \times \G$.
  
  
  \begin{article}\artlabel[Product of diffeological spaces]
    Let $\{\X_i\}_{i \in \cI}$ be any family of diffeological spaces.
    There exists on the product $\X = \prod_{i \in \cI} \X_i$ a coarsest diffeology such that every projection $\pi_i \colon \X \to \X_i$ is smooth.
    It is called the \emph{product diffeology}.
    A plot in $\X$ is just a parameterization $ r \mapsto (x_{i,r})_{i\in \cI}$ such that each $r \mapsto x_{i,r}$ is a plot of $\X_i$.
    
    \textsc{Note}. The category \{Diffeology\} is then closed by products.
    \artend
  \end{article}
  
  There are two equivalent definitions of diffeological bundles;
  the following one is the pedestrian version.
  
  \begin{article}\artlabel[Diffeological fiber bundles]
    Let $\pi \colon \Y \to \X$ be a map with $\X$ and $\Y$ two diffeological spaces.
    We say that $\pi$ is a fibration,
    with fiber $\F$,
    if,
    for every plot $\P \colon \U \to \X$,
    the pullback
    $$
      \pr_1\colon \P^*(Y) \to \U \quad\text{with}\quad \P^*(\Y) = \{(r,y) \in \U \times \Y \mid \P(r) = \pi(y) \},
      $$
    is locally trivial with fiber $\F$.
    That is,
    every point in $\U$ has an open neighborhood $V$ such that there exists a diffeomorphism $\phi\colon \V\times \F \to \pr_1^{-1}(\V) \subset \P^*(\Y)$,
    with $\pr_1 \circ \phi = \pr_1$\,:
    $$
      \begin{tikzcd}[column sep=large, row sep=large, every label/.append style = {font = \small}]
      \V\times\F \arrow[dr, swap, "\pr_1"] \arrow[r, "\phi"] &  \P^*(\Y) \restriction \V \arrow[d, "\pr_1"] \arrow[r, "\pr_2"]  & \Y \arrow[d,"\pi"] \\
      {} & \V \arrow[r, swap, "\P \restriction \V"] & \X
      \end{tikzcd}
      $$
    This definition extends the usual definition of smooth fiber bundles in differential geometry.
    But it contains more:
    
    \textsc{Proposition}. \textit{The projection $\pi \colon \G \to \G/\H$,
    where $\G$ is a diffeological group and $\H \subset \G$ is any subgroup,
    is a diffeological fibration.}
    
    We have now the formal framework where $\pi\colon \T^2 \to \T_\alpha$ is a legitimate fiber bundle.
    This definition of fiber bundle satisfies also the long sequence of homotopy;
    we shall come back to that subject later.
    
    Since we have a definition of fiber bundles,
    we inherit naturally the notion of a diffeological covering:
    
    \textsc{Definition of Coverings}.
    \textit{A covering of a diffeological space $\X$ is a diffeological fibration $\pi\colon \hat \X \to \X$ with a discrete fiber.}
    
    \textsc{Note}. The projection $\pi\colon \RR \to \RR/(\ZZ + \alpha \ZZ)$ is a simply connected covering,
    since $\RR$ is a diffeological group and $\ZZ + \alpha \ZZ$ is a subgroup.
    The fact that $\RR/(\ZZ + \alpha \ZZ)$ is diffeomorphic to $\T_\alpha$ is an exercise left to the reader.
    \artend
  \end{article}
  
  There is an alternative to the definition of fiber bundles involving a groupoid of diffeomorphisms.
  This alternate definition is maybe less intuitive,
  but it is more internal to diffeology.
  It is based on the existence of a natural diffeology on the set of smooth maps between diffeological spaces.
  
  \begin{article}\artlabel[The functional diffeology]
    Let $\X$ and $\X'$ be two diffeological spaces.
    There exists on $\Cinfty(\X,\X')$ a coarsest diffeology such that the \emph{evaluation map}
    $$
      \ev \colon \Cinfty(\X,\X') \times \X \to \X' \quad\text{defined by}\quad \ev(f,x) = f(x),
      $$
    is smooth.
    Thus diffeology is called the \emph{functional diffeology}.
    
    \textsc{Note 1.} A parameterization $r \mapsto f_r$ is a plot for the functional diffeology if the map $(r,x) \mapsto f_r(x)$ is smooth.
    
    \textsc{Note 2.} There exists a natural diffeomorphism between $\Cinfty(\X,\Cinfty(\X',\X''))$ and $\Cinfty(\X \times \X',\X')$.
    That makes the category \{Diffeology\} \emph{Cartesian closed},
    which is a pretty nice property.
    \artend
  \end{article}
  
  \begin{article}\artlabel[Fiber bundles: the groupoid approach]
    Let $\pi \colon \Y \to \X$ be a map with $\X$ and $\Y$ two diffeological spaces.
    Consider the groupoid $\KK$ whose objects are the points of $\X$ and the arrows from $x$ to $x'$ are the diffeomorphisms from $\Y_x$ to $\Y_{x'}$,
    where the preimages $\Y_x = \pi^{-1}(x)$ are equipped with the subset diffeology.
    
    There is a \emph{functional diffeology} on $\KK$ that makes it a \emph{diffeological groupoid}.
    The set $\Obj(\KK) = \X$ is obviously equipped with its own diffeology.
    Next,
    a parameterization $r \mapsto \phi_r$ in $\Mor(\KK)$,
    defined on a domain $\U$,
    will be a plot if
    \begin{enumerate}
      \item[1.] $r \mapsto (\src(\phi_r),\trg(\phi_r))$ is a plot of $\X \times \X$.
      \item[2.] The maps $\ev \colon \U_\src \to \Y$ and $\overline \ev \colon \U_\trg \to \Y$,
      defined by $\ev \,(r,y) = \phi_r(y)$ and $\overline\ev \, (r,y) = \phi_r^{-1}(y)$,
      on $\U_\src = \{ (r,y) \in \U \times \Y \mid y \in \Def(\phi_r) \}$ and $\U_\trg =\{ (r,y) \in \U \times \Y \mid y \in \Def(\phi_r^{-1}) \}$,
      are smooth,
      where these two sets are equipped with the subset diffeology of the product $\U \times \Y$.
    \end{enumerate}
    We have,
    then,
    the following theorem \cite{Igl85}\,:
    
    \textsc{Theorem}. \textit{The map $\pi$ is a fibration if and only if the characteristic map
    $$
      \chi \colon \Mor(\KK) \to \X \times \X, \quad \text{defined by} \quad \chi(f) = (\src(f),\trg(f)),
      $$
    is a subduction.}
    
    Thanks to this approach,
    we can construct,
    for every diffeological fiber bundle,
    a principal fiber bundle
    ---~by spliting the groupoid~---
    with which the fiber bundle is associated.
    It is possible then to refine this construction and define fiber bundles with structures (e.g., linear).
    \artend
  \end{article}
  
  The next step concerning fiber bundles will be to establish the long homotopy sequence,
  but that requires preliminary preparation,
  beginning with the definition of the homotopy groups.
  
  %%%%%%%%%%%%%%%%%%%%%%%%%%%%%%%%%%%%%%%%%%%%%%%%%%%%%%%%%%
  %%
  %% MARK: DOCUMENT Homotopy Theory
  %%
  %%%%%%%%%%%%%%%%%%%%%%%%%%%%%%%%%%%%%%%%%%%%%%%%%%%%%%%%%%
  \section*{Homotopy Theory}
  
  The basis of homotopy begins by understanding what it means to be \emph{homotopic},
  that is,
  to share  the same ``place'' (\textgreek{τόπος}) or ``component''.
  
  \begin{article}\artlabel[Homotopy and connexity]
    Let $\X$ be a diffeological space;
    we denote by $\Paths(\X)$ the space of (smooth) paths in $\X$,
    that is,
    $\Cinfty(\RR,\X)$.
    The ends of a path $\gamma$ are denoted by
    $$
      \0(\gamma) = \gamma (0), \quad \1(\gamma) = \gamma(1) \quad\text{and}\quad \ends(\gamma)=(\gamma(0),\gamma(1)).
      $$
    We say that two points $x$ and $x'$ are \emph{connected} or \emph{homotopic} if there exists a path $\gamma$ such that $x=\0(\gamma)$ and $x'=\1(\gamma)$.
    
    To be connected defines an equivalence relation whose equivalence classes are called \emph{connected components},
    or simply \emph{components}.
    The set of components is denoted by $\pi_0(\X)$.
    
    \textsc{Proposition}. \textit{
    The space $\X$ is the \emph{sum} of its connected components,
    that is,
    $$
      \X= \coprod_{\X_i \in \pi_0(\X)} \X_i.
      $$
    Moreover,
    the partition in connected components is the finest partition of $\X$ that makes $\X$ the sum of its parts.}
    \artend
  \end{article}
  
  It is time to give the precise definition of the sum of diffeological spaces that founds the previous proposition.
  
  \begin{article}\artlabel[Sum of diffeological spaces]
    Let $\{\X_i\}_{i \in \cI}$ be a familly of diffeological spaces.
    There exists a finest diffeology on the sum $\X = \coprod_{\X_i \in \cI} \X_i$ such that each injection $j_i = x \mapsto (i,x)$,
    from $\X_i$ to $\X$,
    is smooth.
    We call this the \emph{sum diffeology}.
    The plots of $\X$ are the parameterization $r \mapsto (i_r,x_r)$ such that $r \mapsto i_r$ is locally constant.
    \artend
  \end{article}
  
  With that definition,
  we close one of the most interesting aspects of the category \{Diffeology\}.
  This category is stable by all the set-theoretic constructions:
  sum, product, part, quotient.
  It is a complete and co-complete category,
  every direct or inverse limit of diffeological spaces having their natural diffeology.
  Moreover,
  the category is Cartesian closed.
  
  That is by itself very interesting,
  as a generalization of the smooth category of Euclidean domains.
  But there are a few other generalizations that have these same properties (Frölicher spaces, for example).
  What really makes the difference is that,
  these nice properties apply to a category that includes non trivially extremely singular spaces,
  as we have seen with the irrational tori,
  and that is what makes \{Diffeology\} so unique.
  But let us now come back to the homotopy theory.
  
  \begin{article}\artlabel[The fundamental group and coverings]
    Let $\X$ be a connected diffeological space,
    that is,
    $\pi_0(\X) = \{\X\}$.
    Let $x \in \X$ be some point.
    We denote by $\Loops(\X,x) \subset \Paths(\X)$ the subspaces of \emph{loops} in $\X$,
    based at $x$,
    that is,
    the subspace of paths $\ell$ such that $\ell(0)=\ell(1)=x$.
    As a diffeological space,
    with its functional diffeology,
    this space has a set of components.
    We define the \emph{fundamental group}%
    \footnote{Formally speaking,
    the homotopy groups are objects of the category \{Pointed Sets\}.
    In particular,
    $\pi_0(\X,x) = (\pi_0(\X),x)$.}
    of $\X$,
    at the point $x$,
    as
    $$
      \pi_1(\X,x) = \pi_0(\Loops(\X,x), \hat x),
      $$
    where $\hat x$ is the constant loop $t\mapsto x$.
    The group multiplication on $\pi_1(\X,x)$ is defined as usual,
    by concatenation:
    $$
      \tau \cdot \tau' = \class \bigg[t \mapsto \left\{
      \begin{array}{l l }
      \ell(2t) & \text{if} \quad t \leq1/2 \\
      \ell'(2t-1) & \text{if} \quad t>1/2
      \end{array}
      \right.\bigg],
      $$
    where $\tau = \class(\ell)$ and $\tau' = \class(\ell')$.
    The inverse is given by $\tau^{-1} = \class[t \mapsto \ell(1-t)]$.
    Actually,
    we work with stationary paths for the concatenation to be smooth.
    A stationary path is a path that is constant on a small interval around $0$ and also around $1$.
    We have now a few main results.
    
    \textsc{Proposition}.
    \textit{The groups $\pi_1(\X,x)$ are conjugate to each other when $x$ runs over $\X$.
    We denote $\pi_1(\X)$ their type.}
    
    We say that $\X$ is \emph{simply connected} if its fundamental group is trivial,
    $\pi_1(\X) = \{0\}$.
    Now we have the following results.
    
    \textsc{Theorem [Universal Covering]}. \textit{Every connected diffeological space $\X$ has a unique
    ---~up to isomorphism~---
    simply connected covering $\pi \colon \tilde \X \to \X$.
    It is a principal fiber bundle with group $\pi_1(\X)$.
    It is called the \emph{universal covering};
    every other connected covering is a quotient of $\tilde \X$ by a subgroup of $\pi_1(\X)$.}
    
    Actually,
    the universal covering is \emph{half} of the \emph{Poincaré groupoid},
    quotient of the space $\Paths(\X)$ by fixed-end homotopy relation \cite[Section 5.15]{PIZ13}.
    This is the second meaningful construction of groupoid in the development of diffeology.
    
    \textsc{Monodromy theorem}. \textit{Let $f \colon \Y \to \X$ be a smooth map,
    where $\Y$ is a simply connected diffeological space.
    With the notations above,
    there exists a unique lifting $\tilde f \colon \Y \to \tilde \X$ once we fix $\tilde f(y) = \tilde x$,
    with $x=f(y)$ and $\tilde x \in \pi^{-1}(x)$.}
    
    These are the more relevant constructions and results,
    familiar to the differential geometer,
    concerning the fundamental group on diffeological spaces.
    \artend
  \end{article}
  
  \begin{article}\artlabel[Two examples of coverings]
    Let us now come back to our irrational torus $\T_\alpha = \T^2/\Delta_\alpha = [\RR^2/\ZZ^2]/\Delta_\alpha$,
    that is,
    $\T_\alpha = \RR^2/[\ZZ^2 \times \{(x,\alpha x)\}_{x \in \RR}] = [\RR^2/\{(x,\alpha x)\}_{x \in \RR}]/\ZZ^2$.
    The quotient $\RR^2/\{(x,\alpha x)\}_{x \in \RR}$ can be realized by $\RR$ with the projection $(x,y) \mapsto y-\alpha x$.
    Then,
    the action of $\ZZ^2$ on $\RR^2$ induces the action on $\ZZ^2$ on $\RR$ by $(n,m)(x,y)=(x+n,y+m) \mapsto y+m - \alpha(x+n)= y-\alpha x + (m-\alpha n)$.
    That is,
    $(n,m) \colon t \mapsto t + m-\alpha n$.
    Therefore,
    $\T_\alpha \simeq \RR/\ZZ + \alpha \ZZ$.
    Since $\RR$ is simply connected,
    thanks to the theorem on existence and unicity of the universal covering,
    $\tilde \T_\alpha = \RR$ and the $\pi_1(\T_\alpha)$ injects in $\RR$ as $\ZZ + \alpha \ZZ$.
    Of course,
    there is no need for all these sophisticated tools to get this result,
    as we have seen in \cite{DonIgl83}.
    But this shows how the pedestrian computation integrates seamlessly the general theory.
    
    Another example%
    \footnote{It was first elaborated by Paul Donato in his dissertation \cite{Don84}.
    We just reinterpret it with our tools.}
    of this theory,
    is one in infinite dimensions:
    the universal covering of $\Diff(\S^1)$.
    The group is equipped with its functional diffeology.
    Let $f$ be a diffeomorphism of $\S^1 = \RR/\ZZ$,
    and assume that $f$ fixes $1$.
    Let $\pi \colon t \mapsto e^{2i\pi t}$,
    from $\RR$ to $\S^1$,
    be the universal covering.
    The composite $f \circ \pi$ is a plot.
    Thanks to the monodromy theorem,
    since $\RR$ is simply connected,
    $f \circ \pi$ has a unique smooth lifting $\tilde f \colon \RR \to \RR$,
    such that $\tilde f(0)=0$\,:
    $$
      \begin{tikzcd}[column sep=large,every label/.append style = {font = \small}]
      \RR \arrow[r, "\tilde f"] \arrow[d, swap, "\pi"] & \RR \arrow[d,"\pi"] \\
      \S^1 \arrow[r, swap, "f"] & \S^1
      \end{tikzcd}
      $$
    Then,
    $\pi \circ \tilde f = f \circ \pi$ implies that $\tilde f(t+1) = \tilde f(t) +k$,
    $k \in \ZZ$,
    but $f$ cannot be injective unless $k=\pm 1$.
    Next,
    $f$ being a diffeomorphism implies that $\tilde f$ is a diffeomorphism of $\RR$,
    that is,
    a strictly increasing or decreasing function.
    Assume that $\tilde f$ is increasing;
    then $\tilde f(t+1) = \tilde f(t) + 1$.
    Thus,
    the \emph{positive diffeomorphisms} of $\S^1$ are the quotient of the increasing diffeomorphisms of $\RR$ satisfying that condition.
    Now,
    let $\tilde f_s(t)=f(t) + s(t-\tilde f(t))$,
    with $s \in [0,1]$.
    We still have $\tilde f_s(t+1) = \tilde f_s(t)+1$ and $\tilde f_s'(t)= s + (1-s)\tilde f'(t)$,
    which still is positive.
    Thus,
    since $\tilde f_0(t)=\tilde f(t)$ and $\tilde f_1(t)=t$,
    the group
    $$
      \widetilde{\Diff}_+(\S^1) = \{ \tilde f \in \Diff_+(\RR) \mid \tilde f(t+1) = \tilde f(t) +1 \}
      $$
    is contractible,
    hence simply connected.
    It is the universal covering of $\Diff_+(\S^1)$,
    the group  of the positive diffeomorphisms of $\S^1$.
    The monodromy theorem indicates that there are only $\ZZ$ different liftings $\tilde f$ of a given diffeomorphism $f$.
    Hence,
    $\pi_1(\Diff_+(\S^1)) = \ZZ$.
    \artend
  \end{article}
  
  \begin{article}\artlabel[Higher homotopy groups]
    Let $\X$ be a diffeological space and $x \in \X$.
    Since $\Loops(\X,x)$ is a diffeological space that contains $\hat x \colon t \mapsto x$,
    there is no obstruction to defining the \emph{higher homotopy groups} by recursion:
    $$
      \pi_n(\X,x) = \pi_{n-1}(\Loops(\X,x),\hat x), \quad n \geq 1.
      $$
    One can also define the recursion of diffeological spaces:
    $$
      \begin{array}{c}
      \X_0 = \X, \ x_0 = x \in \X_0 \ ; \
      \X_1 = \Loops(\X_0,x_0), \ x_1 = [t \mapsto x_0] \in \X_1 \ ; \ \dots \\
      \dots \ ; \ \ \X_n = \Loops(\X_{n-1}, x_{n-1}),\ x_n = [t \mapsto x_{n-1}] \in \X_n \ ; \  \dots
      \end{array}
      $$
    Thus,
    $\pi_n(\X,x) = \pi_{n-1}(\Loops(\X,x),\hat x)$,
    that is,
    $\pi_n(\X,x) = \pi_{n-1}(\X_1, x_1) = \dots = \pi_1(\X_{n-1},x_{n-1}) =  \pi_0(\X_n,x_n)$.
    Since $\pi_n(\X,x)$ is the fundamental group of a diffeological space,
    it is a group.
    And that is the formal definition of the \emph{$n$-th homotopy group}%
    \footnote{It is Abelian for $n=2$.}
    of $\X$ at the point $x$.
    
    We can feel in particular here,
    the benefits of considering all these spaces
    --~$\X$,
    $\Paths(\X)$,
    $\Loops(\X)$,
    and so on~--
    on an equal footing.
    Being all diffeological spaces,
    the recursion does not need any supplementary construction than the ones already defined.
    \artend
  \end{article}
  
  \begin{article}\artlabel[The homotopy sequence of a fiber bundle]
    One of the most important properties of diffeological fiber bundles is their long homotopy sequence.
    Let $\pi \colon \Y \to \X$ be a fiber bundle with fiber $\F$.
    Then,
    there is a long exact sequence of group homomorphisms \cite{Igl85},
    $$
      \begin{array}{l}
      \cdots \to \pi_n(\F) \to \pi_n(\Y) \to \pi_n(\X) \to \pi_{n-1}(\F) \to \cdots \\
      \cdots \to \pi_0(\F)\, \to \pi_0(\Y)\, \to \pi_0(\X) \to 0.
      \end{array}
      $$
    As usual in these cases,
    if the fiber is homotopy trivial,
    then the base space has the homotopy of the total space.
    And that is what happens for the irrational torus $\pi_k(\T_\alpha)= \pi_k(\T^2)$,
    $k \in \NN$.
    
    Let us consider another example,
    in infinite dimensions this time.
    Let $\S^\infty$ be the \emph{infinite-dimensional sphere} in the Hilbert space $\cH=\ell^2(\CC)$.
    We equip first $\cH$ with the fine diffeology of vector space \cite[Section 3.7]{PIZ13}.
    Then,
    we prove that $\S^\infty \subset \cH$,
    equipped with the subset diffeology,
    is contractible \cite[Section 4.10]{PIZ13}.
    Then we consider the \emph{infinite projective space} $\CC\P^\infty=\S^\infty/\S^1$,
    where $\S^1$ acts on the $\ell^2$ sequences by multiplication.
    The projection $\pi \colon \S^\infty \to \CC\P^\infty$ is then a diffeological principal fibration with fiber $\S^1$.
    The homotopy exact sequence gives then $\pi_2(\CC\P^\infty) = \ZZ$ and $\pi_k(\CC\P^\infty) = 0$ if $k \neq 2$,
    which is what we expected.
    
    That would prove if necessary that we can work on singular constructions or infinite-dimensional spaces,
    using the same tools and the same intuition as when we deal with ordinary differential geometry.
    \artend
  \end{article}
  
  \begin{article}\artlabel[Connections on fiber fundles and homotopy invariance]
    Let $\pi\colon \Y \to \X$ be a principal fiber bundle with group $\G$.
    That fabricates a new principal fibration $\pi_* \colon \Paths(\Y) \to \Paths(\X)$ with structure group $\Paths(\G)$.
    Roughly speaking,
    a \emph{connection} on $\pi$ is a reduction of this \emph{paths fiber bundle} to the subgroup $\G \subset \Paths(\G)$,
    consisting of constant paths \cite[Section 8.32]{PIZ13}.
    We require for this reduction to satisfy a few axioms:
    locallity (sheaf condition on the interval of $\RR$),
    compatibility with concatenation, and so on.
    The main point is that once we have a path $\gamma$ in $\X$ and a point $y$ over $x = \gamma(t)$,
    there exists a unique \emph{lift} $\tilde\gamma$ of $\gamma$ such that $\tilde\gamma(t) = y$,
    this is called the \emph{horizontal lift}.
    Moreover,
    if $y'=g_\Y(y)$, the lift $\gamma'$ of $\gamma$ passing at $y'$ at the time $t$ is the shifted $\gamma' = g_\Y \circ \gamma$.
    That property is exactly what we call a reduction of $\pi_*$ to $\G$.
    
    An important consequence of the existence of a connection on a principal fiber bundle is the homotopy invariance of pullbacks.
    
    \textsc{Proposition}.
    \textit{Let $\pi\colon \Y \to \X$ be a principal fiber bundle with group $\G$,
    equipped with a connection.
    Let $t \mapsto f_t$ be a smooth path in $\Cinfty(\X',\X)$,
    where $\X'$ is any diffeological space.
    Then,
    the pullbacks $\pr_0 \colon f_0^*(\Y) \to \X'$ and $\pr_1 \colon f_1^*(\Y) \to \X'$ are equivalent.}
    
    \textsc{Corollary}. \textit{Any diffeological fiber bundle equipped with a connection over a contractible space is trivial.}
    
    We know that this is always true in ordinary differential geometry,
    because every principal bundle over a manifold can be equipped with a connection.
    \artend
  \end{article}
  
  \begin{article}\artlabel[The group of flows of a space]
    Connections are usually defined in ordinary differential geometry by a differential form with values in some Lie algebra.
    As we have seen,
    that is not the way chosen in diffeology,
    for a few good reasons.
    First,
    for such an important property as the homotopy invariance of pullbacks,
    a broad definition of connection is enough.
    Moreover, we have no indisputable concept of Lie algebra in diffeology,%
    \footnote{Even for the moment map,
    in symplectic diffeology,
    we do not need the definition of a Lie algebra,
    as we shall see later on.}
    and choosing one definition rather than another would link a universal concept,
    such as \emph{parallel transport},
    to an arbitrary choice.
    
    And there is also a new diffeological construction where the difference between general connections and form-valued connections is meaningful enough to justify a posteriori our choice.
    That is the computation of the \emph{group of flows} over the irrational torus  \cite[Section 8.39]{PIZ13}.
    Let us begin with a definition:
    
    \textsc{Definition}. \textit{We shall call \emph{flow} over a diffeological space $\X$,
    any $(\RR,+)$-principal bundle over $\X$.}
    
    There is an additive operation on the set $\text{Flows}(\X)$ of (equivalence classes of) flows.
    Let $a=\class(\pi \colon \Y \to \X)$ and $a'=\class(\pi' \colon \Y' \to \X)$ be two classes of flows.
    Consider the pullback $\pi^*(\Y') = \{ (y,y') \in \Y \times \Y' \mid \pi(y) = \pi'(y') \}$.
    It is a $(\RR^2,+)$ principal bundle over $\X$ by $(y,y') \mapsto \pi(y)=\pi'(y')$.
    Let $\Y''$ be the quotient of $\pi^*(\Y')$ by the antidiagonal action of $\RR$,
    that is,
    $\underline{t}(y,y') = (t_\Y(y), -t_{\Y'}(y'))$.
    And let $\pi'' \colon \Y'' \to \X$ be the projection $\pi''(\class(y,y')) = \pi(y)=\pi'(y')$.
    We define then $a + a' = a''$ with $a''=\class(\pi''\colon \Y'' \to \X)$.
    
    The set $\text{Flows}(\X)$,
    equipped with this addition,
    is an Abelian group.
    The neutral element is the class of the trivial bundle,
    and the inverse of a flow is the same bundle but with the inverse action of $\RR$.
    Note that this group is a kind of Picard group on a diffeological space,
    but with $\RR$ instead of $\S^1$ as structure group,
    and if this group doesn't appear in ordinary differential geometry,
    it is because every principal bundle with fiber $\RR$ over a manifold is trivial.
    But that is not the case in diffeology,
    and we know one such nontrivial bundle,
    the irrational torus $\pi \colon \T^2 \to \T_\alpha$.
    
    Let $\pi \colon \Y \to \T_\alpha$ be a flow.
    Consider the pullback $\pr_1 \colon \pr^*(\Y) \to \RR$,
    where $\pr \colon \RR \to \T_\alpha$ is the universal covering.
    It is an $\RR$-principal fiber bundle over $\RR$,
    so it is trivial.
    Let $\phi \colon \RR \times \RR \to \pr^*(\Y)$ be an isomorphism.
    Thus $\Y \simeq \RR \times \RR/\pr_2 \circ \phi$.
    But $\pr_2 \circ \phi$ is any lifting on the second factor of $\RR \times \RR$,
    of the action of $\ZZ \oplus \alpha \ZZ$ on the first one:
    $$
      \begin{tikzcd}[column sep=large, row sep=large, every label/.append style = {font = \small}]
      \RR \times \RR \arrow[r, "\phi"] \arrow[dr, swap, "\pr_1"] & \pr^*(\Y) \arrow[d,"\pr_1"] \arrow[r,"\pr_2"] & \Y \arrow[d,"\pi"] \\
      & \RR \arrow[r, swap, "\pr"] & \T_\alpha
      \end{tikzcd}
      $$
    A lifting of a subgroup $\Gamma \subset \RR$ on the second factor of $\RR \times \RR$,
    where $\Gamma$ acts by translation on the first factor,
    is given,
    for all $k \in \Gamma$ and $(x,\tau) \in \RR \times \RR$,
    by
    $$
      k \colon (x,t) \mapsto (x + k, t + \tau(k)(x)),
      $$
    where $\tau \colon \Gamma \to \Cinfty(\RR)$ is a cocycle satisfying
    $$
      \tau(k + k')(x) = \tau(k)(x + k') + \tau(k')(x).
      $$
    Two cocycles $\tau$ and $\tau'$ define the same flow if they differ from a coboundary $\delta \sigma$:
    $$
      \tau'(k)(x) = \tau(k)(x) + \sigma(x+k) - \sigma(x), \quad \text{with} \quad \sigma \in \Cinfty(\RR).
      $$
    In other words,
    $$
      \text{Flows}(\RR/\Gamma) = \H^1(\Gamma,\Cinfty(\RR)).
      $$
    Applied to $\Gamma = \ZZ + \alpha \ZZ$,
    that gives $\text{Flows}(\T_\alpha)$ equivalent to the group of real $1$-periodic functions $f$,
    after some normalization,
    modulo the relation:
    $$
      f \sim f' \quad \text{if} \quad f'(x) = f(x) + g(x+\alpha) - g(x).
      $$
    This relation is known as the \emph{small divisors Arnold's cohomology relation}.
    The solution depends on the arithmetic of $\alpha$:
    if $\alpha$ is a diophantine or a Liouville number,
    $\text{Flows}(\T_\alpha)$ is one-dimensional or $\infty$-dimensional.%
    \footnote{As $\H^1(\Gamma,\Cinfty(\RR))$,
    the group $\text{Flows}(\RR/\Gamma)$ is obviously a real vector space.}
    
    Moreover,
    every flow $\pi \colon \Y \to \T = \RR / \Gamma$ defined by a cocycle $\tau$,
    can be naturally equipped with the connection associated with the covering $\pr \colon \RR \to \T$ \cite[Section 8.36]{PIZ13}.
    However,
    not all these bundles support a connection form \cite[Section 8.37]{PIZ13},
    only those whose cocycle $\tau$ defining $\pi$ is equivalent to a homomorphism from $\Gamma$ to $\RR$;
    see \cite[Exercise 139]{PIZ13}.
    In other words,
    if the cocycle $\tau$ is not cohomologous to a homomorphism,
    then there is no connection that can be defined by a connection form.
    In particular,
    for $T_\alpha$,
    the only flow equipped with connections defined by a connection form is the Kronecker flow (with arbitrary speeds).
    \artend
  \end{article}
  
  %%%%%%%%%%%%%%%%%%%%%%%%%%%%%%%%%%%%%%%%%%%%%%%%%%%%%%%%%%
  %%
  %% MARK: DOCUMENT Modeling Diffeology
  %%
  %%%%%%%%%%%%%%%%%%%%%%%%%%%%%%%%%%%%%%%%%%%%%%%%%%%%%%%%%%
  \section*{Modeling Diffeology}
  
  Now we have seen a few constructions in diffeology and applications to unusual situations:
  singular quotients and infinite-dimensions spaces.
  It will be interesting to revisit some constructions of differential geometry and see what diffeology can do with them.
  
  \begin{article}\artlabel[Manifolds]
    Every manifold owns a natural diffeology,
    for which the plots are the smooths parameterizations.
    There is a definition internal to the category diffeology:%
    \footnote{We could use too the concept of generating families;
    see \cite[Section 1.66]{PIZ13}.}
    
    \textsc{Definition}. \textit{An $n$-manifold is a diffeological space locally diffeomorphic to $\RR^n$ at each point.}
    
    With this definition,
    as diffeological spaces,
    \{Manifolds\} form a full subcategory of the category \{Diffeology\}.
    \artend
  \end{article}
  
  Of course,
  this definition needs a precise use of the wording \emph{locally diffeomorphic}.
  
  \begin{article}\artlabel[Local smooth maps, D-topology and so on]
    Very soon after the initial works in diffeology,
    it was clear that we needed to enrich the theory with local considerations,
    which were missing until then.
    To respect the spirit of diffeology,
    I defined directly the concept of local smoothness \cite{Igl85},
    as follows:
    
    \textsc{Definition}. \textit{Let $\X$ and $\X'$ be two diffeological spaces.
    Let $f$ be a map from a subset $\A \subset \X$ into $\X'$.
    We say that $f$ is \emph{local smooth} if,
    for each plot $\P$ in $\X$,
    $f \circ \P$ is a plot of $\X'$.}
    
    Note that $f \circ \P$ is defined on $\P^{-1}(\A)$,
    and a first condition for $f \circ \P$ to be a plot of $\X'$ is that $\P^{-1}(\A)$ be open.
    That leads immediately to a second definition:
    
    \textsc{Definition}. \textit{A subset $\A \subset \X$ will be said to be \emph{D-open} if $\P^{-1}(\A)$ is open for all plots $\P$ in $\X$.
    The D-open subsets in $\X$ define a topology on $\X$ called the \emph{D-topology}.}
    
    We have,
    then,
    the following proposition linking these two definitions:
    
    \textsc{Proposition}. \textit{A map $f$ defined on a subset $\A \subset \X$ to $\X'$ is local smooth if and only if:
    $\A$ is D-open,
    and $f \colon \A \to \X'$ is smooth when $\A$ is equipped with the subset diffeology.}
    
    To avoid misunderstanding and signify that $f$ is local smooth
    ---~not just smooth for the subset diffeology~---
    we note $f\colon \X \supset \A \to \X$.
    
    Next,
    since we have local smooth maps,
    we have \emph{local diffeomorphisms} too.
    
    \textsc{Definition}. \textit{We say that $f \colon \X \supset \A \to \X'$ is a \emph{local diffeomorphism} if $f$ is injective,
    if it is local smooth,
    and if its inverse $f^{-1} \colon \X' \supset f(\A) \to \X$ is local smooth.
    We say that $f$ is a local diffeomorphism at $x \in \X$ if there is a superset $\A$ of $x$ such that $f\restriction \A \colon \X \supset \A \to \X'$ is a local diffeomorphism.}
    
    In particular,
    these definitions give a precise meaning to the sentence ``the space $\X$ is locally diffeomorphic to $\X'$ at each/some point''.
    
    That was the beginning of the new concept of \emph{local diffeology} which guides everything modeling in diffeology.
    \artend
  \end{article}
  
  \begin{article}\artlabel[Orbifold as diffeologies]
    The word \emph{orbifold} was coined by Thurston \cite[Chapter 13]{Thu78} in 1978 as a replacement for \emph{V-manifold},
    a structure invented by Ichiro Satake in 1956 \cite{Sat56}.
    
    \begin{figure}[th]
      \centerline{\includegraphics[width=.475\textwidth]{Figures/TearDrop-Satake.pdf}}
      \caption{Figure \ref{fig-teardrop-satake} --- The Teardrop as Satake's Orbifold.}
      \label{fig-teardrop-satake}
    \end{figure}
    
    These new objects have been introduced to describe the smooth structure of spaces that look like manifolds,
    except around a few points,
    where they look like quotients of Euclidean domains by finite linear groups.
    Satake captured the smooth structure around the singularities by a family of compatible \emph{local uniformizing systems} defining the orbifold.%
    \footnote{We will not discuss this construction here.
    The original description by Satake is found in \cite{Sat56},
    and a discussion of this definition is in \cite{IKZ10}.}
    Figure \ref{fig-teardrop-satake} gives an idea about what would be an uniformizing system for the ``teardrop'' with one conic singularity.
    
    The main problem with Satake's definition is that it does not lead to a
    satisfactory notion of smooth maps between orbifolds,
    and therefore prevents the conception of a category of orbifolds.
    Indeed,
    in \cite[p. 469]{Sat57},
    Satake writes this footnote:
    \begin{quote}
      ``The notion of $C^\infty$-map thus defined is inconvenient in the point that a composite of two $C^\infty$-maps defined in a different choice of defining families is not always a $C^\infty$ map.''
    \end{quote}
    For a mathematician,
    that is very annoying.
    
    Considering orbifolds as diffeologies solved the problem.
    Indeed,
    in \cite{IKZ10},
    we defined a \emph{diffeological orbifold} by a modelling process in the same spirit as for smooth manifolds
    
    \textsc{Definition}.
    \textit{An orbifold is a diffeological space that is localy diffeomorphic,
    at each point,
    to some quotient space} $\RR^n/\Gamma$,
    \textit{for some finite subgroup $\Gamma$ of the linear group} $\GL(n,\RR)$,
    depending possibly on the point.
    
    Figure \ref{The-Teardrop} gives an idea about what is a diffeological orbifold:
    the teardrop as some diffeology on the sphere $\S^2$.
    
    \begin{figure}[th]
      \includegraphics[width=.95\textwidth]{Figures/TearDrop.pdf}
      \vspace{-.25\baselineskip}
      \caption{Figure \ref{The-Teardrop} --- The teardrop as diffeology.}
      \label{The-Teardrop}
    \end{figure}
    
    Once this definition is given,
    we could prove (\emph{op. cit.}) that,
    according to this definition,
    every Satake defining family of a local uniformizing system was associated with a diffeology of orbifold and,
    conversely,
    that every diffeological orbifold was associated with a Satake defining family of a local uniformizing system.
    And we proved that these constructions are inverse to each other,
    modulo equivalence.
    
    Thus,
    the diffeology framework fulfilled Satake's program by embedding orbifolds into diffeological spaces,
    and providing them naturally with good,
    workable,
    smooth mappings.
    
    The difficulty met by Satake is subtle and can be explained as follows:
    he tried to define smooth maps between orbifolds as maps that have equivariant liftings on the level of Euclidean domains,
    before quotienting.
    But the embedding of orbifolds into \{Diffeology\} shows that,
    if that is indeed satisfied for local diffeomorphisms
    (see \cite[Lemma 20, 21, 22]{IKZ10}),
    it is not necessarily the case for ordinary smooth maps,
    as this counter-example 25 shows.
    
    Consider the cone orbifolds $\cQ_n = \RR^2/\ZZ_n$,
    and let $f \colon \RR^2 \to \RR^2$,
    $$
      f(x,y) = \begin{cases}
      0 & \text{ if } r > 1 \text{ or } r = 0 \\
      e^{-1/r} \rho_n(r) (r,0) & \text{ if } \frac{1}{n+1} < r \leq \frac{1}{n}
      \text{ and $n$ is even } \\
      e^{-1/r} \rho_n(r) (x,y) & \text{ if } \frac{1}{n+1} < r \leq \frac{1}{n}
      \text{ and $n$ is odd},
      \end{cases}
      $$
    where $r = \sqrt{x^2 + y^2}$ and $\rho_n$ is a smooth non-zero real function which is zero outside the interval $\mathopen]1/(n+1),1/n\mathclose[$.
    Then,
    for all integers $m$ dividing $n$,
    $f$ projects onto a smooth map $\phi \colon \cQ_m \to \cQ_n$ that cannot be lifted locally in an equivariant smooth map,
    over a neighborhood of $0$.
    Again,
    diffeology structured the problem in a way that almost solved it.
    \artend
  \end{article}
  
  \begin{article}\artlabel[Noncommutative geometry \& diffeology: the case of \break orbifolds]
    The question of a relation between diffeology and noncommutative geometry appeared immediately with the study of the irrational torus \cite{DonIgl83}.
    The condition of diffeomorphy between two irrational tori $\T_\alpha$ and $\T_\beta$,
    that is,
    $\alpha$ and $\beta$ conjugate modulo $\GL(2,\ZZ)$,
    coincided clearly in noncommutative geometry,
    with the Morita-equivalent of the $\CC^*$-algebras associated with the foliations \cite{Rie81}.
    That suggested a structural relationship between diffeology and noncommutative geometry deserving to be explored,
    but that question had been left aside since then.
    
    We recently reopened the case of this relationship,
    considering orbifolds.
    And we exhibited a simple construction associating naturally,
    with every diffeological orbifold a $\CC^*$-algebra,
    such that two diffeomorphic orbifolds give two Morita-equivalent algebras \cite{IZL16}.
    %
    Before continuing,
    we need to recall a general definition \cite[Sections 1.66 and 1.76]{PIZ13}.
    
    \textsc{Definition}. \textit{Let $\X$ be a diffeological space.
    We define the \emph{nebula} of a set $\cF$ of plots in $\X$ as the diffeological sum
    $$
      \cN = \coprod_{\P \in \cF} \dom(\P) = \{ (\P,r) \mid \P \in \cF, \ r \in \dom(\P) \},
      $$
    where each domain is equipped with its standard smooth diffeology.
    Then,
    we define the \emph{evaluation map}
    $$
      \ev \colon \cN \to \X \quad\text{by}\quad \ev \colon (\P,r) \mapsto \P(r).
      $$
    We say that $\cF$ is a \emph{generating family} of $\X$ if $\ev$ is a subduction.}
    
    Now,
    let $\cQ$ be an orbifold.
    A local diffeomorphism $\F$ from a quotient $\RR^n/\Gamma$ to $\cQ$ will be called a \emph{chart}.
    A set $\cA$ of charts whose images cover $\cQ$ will be called an \emph{atlas}.
    With every atlas $\cA$ is associated a special generating family $\cF$ by considering the strict lifting of the charts $\F \in \cA$ to the corresponding $\RR^n$.
    Precisely,
    let $\pi_\Gamma = \RR^n \to \RR^n/\Gamma$, $\F \colon \RR^n/\Gamma \supset \dom(\F) \to \cQ$;
    then $\cF = \{\F \circ \pi_\Gamma\}_{\F \in \cA}$.
    We call $\cF$ the \emph{strict generating family} associated with $\cA$,
    and we denote by $\cN$ its nebula.
    
    Next,
    we consider the groupoid $\GG$ whose objects are the points of the nebula $\cN$,
    and the arrows,
    the germs of local diffeomorphisms $\varphi$ of $\cN$ that project on the identity by $\ev$,
    that is,
    $\ev \circ \varphi = \ev$.
    We call $\GG$ the \emph{structure groupoid} of $\cQ$.
    Then,
    for a suitable but natural \emph{functional diffeology} on $\GG$,
    we have the following \cite{IZL16}:
    
    \textsc{Theorem 1.}\ \textit{The groupoid $\GG$ is Hausdorff and etale.
    The groupoids associated with different atlases are equivalent as categories.}
    
    Hence,
    thanks to the etale property,
    we can associate a $\CC^*$-algebra $\fA$ with $\GG$ by the process described by Renaud in \cite{Ren80}.
    We have,
    then,
    
    \textsc{Theorem2.}\ \textit{The groupoids associated with different atlases of an orbifold are equivalent in the sense of Muhly--Renault--Williams \cite{MRW87}.
    Therefore,
    their $\CC^*$-algebras are Morita-equivalent.}
    
    This Morita-equivalence between the $\CC^*$-algebras associated with different atlases of the orbifold is the condition required to make this construction meaningful and categorical.
    That construction is the first bridge between diffeology and noncommutative geometry;
    it gives an idea where and how diffeology and noncommutative geometry respond to each other,
    at least at the level of orbifolds.%
    \footnote{There is a more general subcategory of \{Diffeology\} for which such a construction leads to the same conclusion.
    It is the subject of a work in progress.}
    \begin{center}
      \begin{tikzcd}
        \{\text{Diffeology}\} \supset \{\text{Orbifolds}\}
        \arrow[dr, red, start anchor={south east}, end anchor={north west}]
        & & \text{$\CC^*$-Algebras} \\
        & \text{Groupoids} \arrow[ur, red, start anchor={north east}, end anchor={south west}]
        &
      \end{tikzcd}
    \end{center}
    
    As an example,
    let us consider the simple orbifold $\Delta_1 = \RR/\{\pm 1\}$.
    The structure of the orbifold is represented by the pushforward of the standard diffeology from $\RR$ to $[0,\infty[$,
    by the square map $\text{sqr} \colon t \mapsto t^2$.
    The singleton $\cF =\{ \text{sqr} \}$ is a strict generating family,
    and the structure groupoid $\GG$ is the groupoid of the action of $\Gamma = \{\pm 1\}$,
    that is,
    $$
      \Obj(\GG) = \RR \qmbox{and} \Mor(\GG) = \{ (t,\varepsilon,\varepsilon t) \mid \varepsilon = \pm 1 \} \simeq \RR \times \{\pm 1\}.
      $$
    A continuous function $f$ on $\Mor(\GG)$ to $\CC$ is a pair of functions $f=(a,b)$,
    where $a(t)=f(t,1)$ and $b(t)=f(t,-1)$.
    With this convention,
    the algebra of the orbifold is then represented by a submodule of $\M_2(\CC) \otimes \cC^0(\RR,\CC)$,
    $$
      f = (a,b) \mapsto \M = \bigg[t \mapsto \begin{pmatrix} a(t) & b(-t) \cr b(t) & a(-t) \end{pmatrix}\bigg],
      $$
    with $\M^*(t) = [{}^\tau\M(t)]^*$. The superscript $\tau$ represents the transposition,
    and the asterisk represents the complex conjugation element by element.
    
    Note that we still trace the orbifold in the characteristic polynomial $\P_\M(\lambda)$,
    which is invariant by the action of $\{\pm 1\}$,
    and is then defined on the orbifold $\Delta_1$ itself,
    $\P_\M(\lambda) \colon t \mapsto \lambda^2 - \lambda \tr(\M(t)) + \det(\M(t))$,
    where $\tr(\M(t)) = a(t) + a(-t)$ and $\det(\M(t)) = a(t)a(-t) - b(t)b(-t)$ are obviously $\{\pm 1 \}$ invariant.
    \artend
  \end{article}
  
  \begin{article}\artlabel[Manifolds with boundary and corners]
    \label{Manifolds-with-Boundary-and-Corners}
    One day,
    in 2007,
    I received an e-mail from a mathematical physicist,
    wondering how diffeology behaves around the corners\dots
    Here is an excerpt:
    
    \begin{quote} ``I have just one worry about this theory.
      I found it very difficult to check that there's a diffeology on the closed interval $[0,1]$ such that a smooth function $f \colon [0,1] \to \RR$ is smooth in the usual sense,
      even at the endpoints. \ $\dots$
      This problem would become very easy to solve using Chen's definition of smooth space,
      which allows for plots whose domain is any convex subset of $\RR^n$.''
    \end{quote}
    
    And our colleague to end his remark by this wish,
    that people interested in diffeology ``are hoping you could either solve this problem in the context of diffeologies, or switch to Chen's definition\dots''.
    
    The good news is that it is not necessary to give up diffeology to be happy.
    There is indeed a diffeology on $[0,1]$ such that ``a smooth function $f \colon [0,1] \to \RR$ is smooth in the usual sense''.
    And that diffeology is simply the subset diffeology,
    precisely \cite[Section 4.13]{PIZ13},
    
    \textsc{Theorem}. \textit{Let $\H_n \subset \RR^n$ be the half-space defined by $x_1 \geq 0$,
    equipped with the subset diffeology.
    Let $f \in \Cinfty(\H_n,\RR)$;
    then there exists a smooth function $\F$,
    defined on an open superset of $\H_n$ in $\RR^n$,
    such that $f = \F \restriction \H^n$.}
    
    This proposition is a consequence of a famous Whitney theorem on extension of smooth even functions \cite{Whi43}.
    By the way,
    it gives a solid basis to the vague concept of ``smooth in the usual sense'',
    as to be smooth in the usual sense means then to be smooth for the diffeology.
    
    We can investigate further and characterize the local diffeomorphisms of half-spaces \cite[Section 4.14]{PIZ13}\,:
    
    \textsc{Theorem}. \textit{A map $f \colon \A \to \H_n$, with $\A \subset \H_n$, is a
    local diffeomorphism for the subset diffeology if
    and only if: $\A$ is open in $\H_n$, $f$ is injective,
    $f(\A \cap \partial \H_n) \subset \partial \H_n$, and for
    all $x \in \A$ there exist an open ball $\cB \subset \RR^n$
    centered at $x$ and a local diffeomorphism $\F \colon \cB \to
      \RR^n$ such that $f$ and $\F$ coincide on $\cB \cap \H_n$.}
    
    Thanks to this theorem,
    it is then easy to include the manifolds with boundary into the category \{Diffeology\},
    in the same way we included the categories \{Manifolds\} and \{Orbifolds\}.
    
    \textsc{Definition}. \textit{An $n$-dimensional \emph{manifold with boundary} is a diffeological space $\X$ which is diffeomorphic,
    at each point,
    to the half-space $\H_n$.
    We say that $\X$ is \emph{modeled} on $\H_n$.}
    
    Thanks to the previous theorems,
    it is clear that this definition covers completely,
    and not more,
    the usual definition of manifold with boundary one can find for example in \cite{GuiPol74} or \cite{Lee06}.
    In other words,
    the ordinary category \{Manifolds with Boundary\} is a natural full subcategory of \{Diffeology\},
    the category \{Manifolds\} being itself a full subcategory of \{Manifolds with Boundary\}.
    
    Moreover,
    we have a similar result for the subset diffeology of corners,
    thanks to a Schwartz theorem%
    \footnote{In this case,
    it is a simple corollary of the Whitney theorem \cite[Remark p. 310]{Whi43}.}
    \cite{Sch75}\,:
    
    \textsc{Theorem}. \textit{Let $\K^n$ be the positive $n$-corner in $\RR^n$ defined by $x_i \geq 0$,
    with $i = 1, \dots, n$,
    equipped with the subset diffeology.
    Let $f \in \Cinfty(\K^n,\RR)$;
    then there exists a smooth function $\F$,
    defined on an open superset of $\K^n$ in $\RR^n$,
    such that $f = \F \restriction \K^n$.}
    
    Actually,
    this property extends to any differential $k$-form on $\K^n$:
    it is the restriction of a smooth $k$-form on $\RR^n$  \cite{GIZ17-a}.
    
    As for manifolds with boundary,
    it is natural to define the $n$-dimensional \emph{manifolds with corners} as diffeological spaces that are locally diffeomorphic to $\K^n$ at each point.
    We get naturally then the category \{Manifolds with Corners\} as a new subcategory of \{Diffeology\}.
    
    Back to the alternative Chen versus Souriau:
    the three axioms of diffeologies are indeed identical to the preceding three axioms of Chen's differentiable spaces \cite{Che77},
    except for the domains of plots that are open instead of being convex.
    Chen's spaces had been introduced with homology and cohomology in mind,
    and that is why he chose the convex subsets as domains for his plots.
    On another side,
    the choice of open subsets positions diffeology as a competitor to differential geometry itself.
    And now,
    the fact that smooth maps for half-spaces or corners,
    equipped with the subset diffeology,
    coincide with what was guessed heuristically to describe manifolds with boundary or corners,
    is another confirmation that there is no need to amend the theory in any way.
    For example,
    we could define smooth simplices in diffeological spaces as smooth maps from the standard simplices,
    equipped with its subset diffeology.
    And that would cover the usual situation in differential geometry.
    \artend
  \end{article}
  
  \begin{article}\artlabel[Fr\"olicher spaces as reflexives diffeological spaces]
    We recall that a Fr\"olicher structure on a set $\X$ is defined by a pair of sets $\cF \subset \Maps(\X,\RR)$ and $\cC \subset \Maps(\RR,\X)$ that satisfies the double condition:
    \begin{eqnarray*}
      &\cC = \{ c \in \Maps(\RR,\X) \mid \cF \circ c \subset \Cinfty(\RR,\RR) \}, \\
      &\cF = \{ f \in \Maps(\X,\RR) \mid f \circ \cC \subset \Cinfty(\RR,\RR) \}.
    \end{eqnarray*}
    A set $\X$ equipped with a Fr\"olicher structure is called a Fr\"olicher space \cite{KriMic97}.
    Now,
    let $\X$ be a diffeological space,
    $\cD$ be its diffeology,
    and $\Cinfty(\X,\R)$ be its set of real smooth maps.
    
    \textsc{Definition.} \textit{We say that $\X$ is \emph{reflexive} if $\cD$ coincides with the coarsest diffeology on $\X$,
    for which the set of real smooth maps is exactly $\Cinfty(\X,\R)$.}
    
    Then,
    thanks to Boman's theorem \cite{Bom67},
    one can show%
    \footnote{The concept of reflexive space has been suggested by Yael Karshon,
    and we established the equivalence with Fr\"olicher spaces,
    together with Augustin Batubenge and Jordan Watts,
    at a seminar in Toronto in 2010.}
    that a Fr\"olicher space,
    equipped with the coarsest diffeology such that the elements of $\cF$ are smooth,
    is reflexive.
    And one can check conversely that a reflexive diffeological space satisfies the Fr\"olicher conditions above;
    see \cite[Exercises 79 and 80]{PIZ13}.
    In other words,
    
    \textsc{Proposition.} \textit{The category of Fr\"olicher spaces coincides with the subcategory of reflexive diffeological spaces.}
    \artend
  \end{article}
  
  %%%%%%%%%%%%%%%%%%%%%%%%%%%%%%%%%%%%%%%%%%%%%%%%%%%%%%%%%%
  %%
  %% MARK: DOCUMENT Cartan, de Rham Calculus
  %%
  %%%%%%%%%%%%%%%%%%%%%%%%%%%%%%%%%%%%%%%%%%%%%%%%%%%%%%%%%%
  
  \section*{Cartan--de Rham Calculus}
  
  With fiber bundles and homotopy,
  differential calculus is one of the most developed domains in diffeology.
  We begin first with the definition of a differential form on a diffeological space.
  
  \begin{article}\artlabel[Differential forms]
    Let $\U \subset \RR^n$;
    we denote by $\Lambda^k(\RR^n)$ the vector space of linear $k$-forms on $\RR^n$, $k \in \NN$.
    We call \emph{smooth $k$-form} on $\U$ any smooth map $a\colon \U \to \Lambda^k(\RR^n)$.
    
    Now,
    let $\X$ be a diffeological space.
    
    \textsc{Definition}. \textit{We call \emph{differential $k$-form} on $\X$ any map $\alpha$ that associates,
    with every plot $\P \colon \U \to \X$,
    a smooth $k$-form $\alpha(\P)$ on $\U$ that satisfies the compatibility condition:
    $$
      \alpha(\P \circ \F) = \F^*(\alpha(\P)),
      $$
    for all smooth parameterizations $\F$ in $\U$.}
    
    The set of $k$-forms on $\X$ is denoted by $\Omega^k(\X)$.
    Note that $\Omega^0(\X) = \Cinfty(\X,\RR)$.
    
    Note also that one can consider an $n$-domain $\U$ as a diffeological space;
    in this case,
    a differential $k$-form $\alpha$ is immediately identified with its value $a = \alpha(\id_\U)$ on the identity,
    $\alpha \in \Omega^k(\U)$ and $a \in \Cinfty(\U, \Lambda^k(\RR^n))$.
    
    The two main operations on the differential forms on a diffeological space are as follows:
    
    \begin{enumerate}
      \item[1.] \textsc{The pullback.} Let $f \in \Cinfty(\X',\X)$ be a smooth map between diffeological spaces,
      and let $\alpha \in \Omega^k(\X)$.
      Then $f^*(\alpha) \in \Omega^k(\X')$ is the $k$-form defined by
      $$
        [f^*(\alpha)](\P') = \alpha (f \circ \P'),
        $$
      for all plots $\P'$ in $\X'$.
      \item[2.] \textsc{The exterior derivative.} Let $\alpha \in \Omega^k(\X)$;
      its exterior derivative $d\alpha \in \Omega^{k+1}(\X)$ is defined by
      $$
        [d\alpha](\P) =d[\alpha(\P)],
        $$
      for all plots $\P$ in $\X$.
    \end{enumerate}
    
    Then,
    we have a de Rham complex $\Omega^*(\X)$,
    with an endomorphism $d$ that satisfies
    $$
      d \circ d = 0, \quad\text{and}\quad
      \left\{ \begin{array}{l}
      \Z^*_\dR(\X) = \ker(d \colon \Omega^*(\X) \to \Omega^{*+1}(\X)) \\
      \B^*_\dR(\X) = d(\Omega^{*-1}(\X)) \subset \Z^*_\dR(\X).
      \end{array} \right.
      $$
    This defines a de Rham cohomology series of groups
    $$
      \quad \H^k_\dR(\X) = \Z^k_\dR(\X) / \B^k_\dR(\X).
      $$
    Note that this series begins with $k=0$,
    for which $\B^0_\dR(\X) = \{0\}$.
    
    The first cohomology group $\H^0_\dR(\X)$ is easy to compute.
    The differential $d\!f$ of a smooth function $f \in \Omega^0(\X)$ vanishes if and only if $f$ is constant on the connected components of $\X$.
    Thus,
    $\H_\dR(\X)$ is the real vector space generated by $\pi_0(\X)$,
    that is,
    $\text{Maps}(\pi_0(\X),\RR)$.
    \artend
  \end{article}
  
  \begin{article}\artlabel[Quotienting differential forms]
    One of the main procedures on differential forms is \emph{quotienting forms}.
    I mean the following:
    let $\X$ and $\X'$ be two diffeological spaces,
    and let $\pi\colon \X \to \X'$ be a subduction.
    The following criterion \cite{Sou84} identifies $f^*(\Omega^*(\X'))$ into $\Omega^*(\X)$\,:
    
    \textsc{Proposition}. \textit{Let $\alpha \in \Omega^*(\X)$.
    There exists $\beta \in \Omega^*(\X')$ such that $\alpha = \pi^*(\beta)$ if and only if,
    for all pairs of plots $\P$ and $\P'$ in $\X$,
    if $\pi \circ \P = \pi \circ \P'$,
    then $\alpha(\P) = \alpha(\P')$.}
    
    That proposition helps us to compute $\H_\dR(\T_\alpha)$,
    for example.
    Actually,
    the criterion above has a simple declination for coverings.
    
    \textsc{Proposition}. \textit{Let $\X$ be a diffeological space and $\pi \colon \tilde \X \to \X$ its universal covering.
    Let $\tilde \alpha \in \Omega^*(\tilde X)$.
    There exists $\alpha \in \Omega^*(\X)$ such that $\tilde \alpha = \pi^*(\alpha)$ if and only if,
    $\tilde \alpha$ is invariant by $\pi_1(\X)$,
    that is,
    $k^*(\tilde \alpha) = \tilde \alpha$ for all $k \in \pi_1(\X)$.}
    
    Indeed,
    for the criterion above,
    $\pi \circ \P = \pi \circ \P'$ if and only if,
    locally on each ball in the domain of the plots,
    there exists an element $k \in \pi_1(\X)$ such that $\P' = k \circ \P$ on the ball.
    
    Now,
    let us apply this criterion to any irrational torus $\T = \RR/\Gamma$,
    where $\Gamma$ is a strict dense subgroup of $\RR$.
    A $1$-form $\tilde \alpha \in \RR$ writes $\tilde a(x)\, d\!x$.
    It is invariant by $\Gamma$ if and only if $a(x) = a$ is constant.
    Let $\theta$ be the $1$-form whose pullback is $d\!x$;
    then
    $$
      \Omega^1(\T) = \RR \theta \quad\text{and}\quad \H^1_\dR(\T) = \RR.
      $$
    Obviously,
    $\H^0_\dR(\T) = \RR$ and $\H^k_\dR(\T) = \{0\}$ if $k>1$.
    \artend
  \end{article}
  
  \begin{article}\artlabel[Parasymplectic form on the space of geodesics]
    It is well known that,
    if the space $\Geod(\M)$ of (oriented) \emph{geodesic trajectories} (aka unparametrized geodesics) of a Riemannian manifold $(\M,g)$ is a manifold,
    then this manifold is naturally symplectic for the quotient of the presymplectic form defining the geodesic flow.
    A famous example is the geodesics of the sphere $\S^2$,
    for which the space of geodesics is also $\S^2$, equipped with the standard surface element.%
    \footnote{For a judicious choice of constant.}
    In this case,
    the mapping from the unit bundle $\U\S^2$ to $\Geod(\S^2)$ is realized by the moment map of the rotations:
    $$
      \ell \colon \U\S^2 = \{(x,u)\in \S^2 \times \S^2 \mid u \cdot x = 0 \} \to \Geod(\S^2) \ \text{with}\  \ell(x,u) = x \wedge u.
      $$
    Now,
    what about the space of geodesics of the $2$-torus $\T^2 = \RR^2/\ZZ^2$,
    for example?
    It is certainly not a manifold because of the mix of closed and unclosed geodesics.
    And about the canonical symplectic structure,
    does it remain something from it?
    And what?
    That is exactly the kind of question diffeology is able to answer.
    
    The geodesics of $\T^2$ are the characteristics of the differential $d\lambda$ of the Liouville $1$-form $\lambda$ on $\U\T^2$,
    associated with the ordinary Euclidean product:
    $$
      \lambda (\delta y) = u \cdot \delta x, \quad\text{with}\quad y = (x,u) \in \U\T^2 \quad\text{and}\quad \delta y \in \T_y(\U\T^2).
      $$
    And then,
    $$
      \Geod(\T^2) = \bigg\{ \pr\{x+tu\}_{t\in \RR} \times \{u\} \subset \T^2 \times \S^1 \mid (x,u) \in \U\T^2 \bigg\}.
      $$
    The direction of the geodesic $\pr_2 \colon ( \pr\{x+tu\}_{t\in \RR},u) \mapsto u$ is a natural projection on $\S^1$\,:
    $$
      \begin{tikzcd}[column sep={3.5em,between origins}, row sep=2.5em, every label/.append style = {font = \small}]
      \U\T^2 \arrow[rr, "\pi"] \arrow[dr,swap,"\pr_2"]       &        & \Geod(\T^2) \arrow[dl,"\pr_2"] \\
      & \S^1   &
      \end{tikzcd}
      $$
    The fiber $\pr_2^{-1}(u) \subset \Geod(\T^2)$,
    is the torus $\T_u$ of all lines with slope $u$.
    We have seen that,
    depending on whether the slope is rational or not,
    we get a circle or an irrational torus.
    As we claimed,
    $\Geod(\T^2)$ equipped with the quotient diffeology of $\U\T^2$ is not a manifold.
    However,
    there exists on $\Geod(\T^2)$ a closed $2$-form $\omega$ such that $d\lambda = \pi^*(\omega)$ \cite{PIZ16-b}.
    We say that $\Geod(\T^2)$ is parasymplectic.%
    \footnote{The meaning of the word \emph{symplectic} in diffeology is still under debate.
    That is why I use the wording \emph{parasymplectic} to indicate a closed $2$-form.}
    
    Actually,
    this example is just a special case of the general situation \cite{PIZ16-c}.
    
    \textsc{Theorem}. \textit{Let $\M$ be a Riemannian manifold.
    Let $\Geod(\M)$ be the space of geodesics,
    defined as the characteristics of the canonical presymplectic $2$-form $d\lambda$ on the unit bundle $\U\M$.
    Then,
    there exists a closed $2$-form $\omega$ on $\Geod(\M)$ such that $d\lambda = \pi^*(\omega)$.}
    
    This result is a direct application of the criterion above on quotienting forms.
    It is strange that we had to wait so long to clarify this important point,
    which should have been one of the first results in diffeology.
    \artend
  \end{article}
  
  \begin{article}\artlabel[Differential forms on manifolds with corners]
    We claimed previously that the smooth maps $f \colon \K_n \to \RR$,
    where $\K_n$ is the $n$-dimensional corner,
    are the restrictions of smooth functions defined on some open neighborhood of $\K_n$ in $\RR^n$.
    We have more \cite{GIZ16-a}\,:
    
    \textsc{Proposition}. \textit{Let $\omega \in \Omega^k(\K^n)$ be a differential $k$-form on $\K^n$.
    Then,
    there exists a smooth $k$-form $\bar\omega$ defined on some open neighborhood of $\K^n \subset \RR^n$ such that $\omega = \bar\omega \restriction \K^n$.}
    
    This proposition has,
    then,
    a corollary (\emph{op. cit.})\,:
    
    \textsc{Theorem}. \textit{Let $\M$ be a smooth manifold.
    Let $\W \subset \M$,
    equipped with the subset diffeology,
    be a submanifold with boundary and corners.
    Any differential form on $\W$ is the restriction of a smooth form defined on an open neighborhood.}
    
    That closes the discussion about the compatibility between diffeology and manifolds with boundary and corners for any question relative to the de Rham complex.
    \artend
  \end{article}
  
  \begin{article}\artlabel[The problem with the de Rham homomorphism]
    Let us focus on $1$-forms,
    precisely,
    the $1$-forms on $\T_\alpha$,
    to take an example.
    The integration on paths defines the first de Rham homomorphism.
    Let $\epsilon$ be a closed $1$-form on $\T_\alpha$
    (actually any $1$-form,
    since they are all closed).
    Consider the map
    $$
      \gamma \mapsto \int_\gamma \epsilon = \int_0^1 \epsilon(\gamma)_x(1) \, d\!x,
      $$
    where $\gamma \in \Paths(\T_\alpha)$.
    Because $\epsilon$ is closed and because the integral depends only on the fixed end homotopy class of $\gamma$,
    restricted to $\Loops(\T_\alpha)$,
    this integral defines a homomorphism from $\pi_1(\T_\alpha)$ to $\RR$.
    And since the integral on a loop of a differential $d\!f$ vanishes,
    the integral depends only on the cohomology class of $\epsilon$.
    Thus,
    we get the first de Rham homomorphism:%
    \footnote{See \cite[Section 6.74]{PIZ13} for the general construction and for the justifications needed.}
    $$
      h \colon \H^1_\dR(\T_\alpha) \to \Hom(\pi_1(\T_\alpha),\RR) \quad\text{defined by}\quad \epsilon \mapsto [\ell \mapsto \int_\ell \epsilon].
      $$
    Now,
    since $\H^1_\dR(\T_\alpha) = \RR$ and $\Hom(\pi_1(\T_\alpha),\RR)= \RR^2$,
    the de Rham homomorphism cannot be an isomorphism,
    as it is the case for Euclidean domains,
    or more generally ordinary manifolds.
    Actually,
    it is precisely given by
    $$
      h \colon a \mapsto [(n,m) \mapsto a(n+\alpha m)],
      $$
    where $a \in \H^1_\dR(\T_\alpha)$ is represented by $a \dt$ on $\RR$,
    and $\pi_1(\T_\alpha) = \ZZ + \alpha \ZZ \subset \RR$.
    
    This hiatus is specific to diffeology,
    versus differential geometry.
    It is,
    however,
    still true that,
    for any diffeological space,
    the first de Rham homomorphism is injective, and we can interpret geometrically its cokernel.%
    \footnote{For differential forms in higher degree,
    this is still a work in progress.}
    
    Let $\tilde\T_\alpha(=\RR)$ denote the universal covering of $\T_\alpha$.
    Consider then a homomorphism $\rho$ from $\pi_1(\T_\alpha)$ to $(\RR,+)$.
    Then,
    build the associated bundle $\pr \colon \tilde \T_\alpha \times_\rho \RR \to \T_\alpha$,
    where $\pi_1(\T_\alpha)$ acts diagonally on the product $\tilde \T_\alpha \times \RR$.
    That is,
    for all $(x,t) \in \tilde\T_\alpha \times \RR$ and all $k \in \pi_1(\T_\alpha)$,
    $k \colon (x,t) \mapsto (x+k, t + \rho(k))$.
    Let $\class \colon \tilde \T_\alpha \times \RR \to \tilde \T_\alpha \times_\rho \RR$ be the projection.
    $$
      \begin{tikzcd}[column sep=large,every label/.append style = {font = \small}]
      \tilde \T_\alpha \times \RR \arrow[r, "\class"] \arrow[d, swap, "\pr_1"] & \tilde \T_\alpha \times_\rho \RR \arrow[d,"\pr"] \\
      \tilde\T_\alpha \arrow[r, swap, "\pi"] & \T_\alpha
      \end{tikzcd}
      $$
    
    The right down arrow,
    $\pr \colon \class(x,t) \mapsto \pi(x)$ is a principal $(\RR,+)$ fiber bundle for the action $s \colon \class(x,t) \mapsto \class(x, t+s)$.
    This principal fiber bundle has a natural connection induced by the connection of the universal covering.
    Pick a path $\gamma$ in $\T_\alpha$ and a point $\tilde x$ over $x=\gamma(t)$;
    there exists a unique lifting $\tilde\gamma$ such that $\tilde\gamma(t)=\tilde x$.
    Then,
    we define the horizontal lifting $\bar\gamma$ of $\gamma$ passing through $\class(\tilde x,s)$ by $\bar\gamma(t') = \class(\tilde\gamma(t'), t'-t + s)$.
    This connection is,
    by construction,
    flat.
    Indeed,
    the subspace $\{\class(\tilde x, 0) \mid \tilde x \in \tilde\T_\alpha \}$ is a reduction of the principal fiber bundle $\pr \colon \tilde \T_\alpha \times_\rho \RR \to \T_\alpha$ to the group $\pi_1(\T_\alpha)/ \ker(\rho)$.
    Now,
    in \cite[Section 8.30]{PIZ13},
    we prove that a homomorphism $\rho\colon \pi_1(\T_\alpha) \to \RR$ gives a trivial fiber bundle $\pr$ if and only if $\rho$ is the de Rham homomorphism of a closed $1$-form $\epsilon$.
    And eventually,
    we prove the following:
    
    \textsc{Proposition}. \textit{The cokernel of the de Rham homomorphism is equivalent to the set of equivalence classes of $(\RR,+)$-principal bundle over $\T_\alpha$,
    equipped with a flat connection.
    This result is actually general for any diffeological space $\X$.}
    
    Note that this analysis puts on an equal footing the surjectivity of the first de Rham homomorphism and the triviality of principal $(\RR,+)$-principal bundles over manifolds.
    That deserved to be noticed.
    \artend
  \end{article}
  
  \begin{article}\artlabel[The chain-homotopy operator]
    The \emph{chain-homotopy operator} $\CHK$ is a fundamental construction in differential calculus \cite[Section 6.83]{PIZ13}.
    It is related in particular to integration of closed differential forms,
    homotopic invariance of de Rham cohomology,
    and the moment map in symplectic geometry,
    as we shall see in the following.
    
    Let $\X$ be a diffeological space;
    there exists a smooth linear operator
    $$
      \CHK\colon \Omega^p(\X) \to \Omega^{p-1}(\Paths(\X)) \quad\text{with}\quad p \geq 1,
      $$
    that satisfies
    $$
      \CHK \circ d +d  \circ \CHK = \1^* - \0^*,
      $$
    where $\0,\1\colon \Paths(\X) \to \X$ are defined by $\hat t(\gamma) = \gamma(t)$.
    
    Explicitly,
    let $\alpha$ be a $p$-form of $\X$, with $p>1$, and
    $\P \colon \U \to \Paths(\X)$ be an $n$-plot. The value of
    $\CHK\alpha$ on the plot $\P$, at the point $r \in \U$,
    evaluated on $(p-1)$ vectors $(v)_{i=2}^p = (v_2)\dots(v_{p})$ of
    $\RR^n$, is given by
    $$
      \CHK\alpha\,(\P)_r(v)_{i=2}^p =
      \int_0^1 \alpha \left( \vect{t \\ r} \mapsto \P(r)(t)
      \right)_{t \choose r} \vect{1 \\ 0}
      \vect{0 \\ v_i}_{i=2}^p \dt.
      $$
    For $p=1$,
    $$
      \CHK \alpha \colon \gamma \mapsto \int_\gamma \alpha = \int_0^1 \alpha(\gamma)_t \,\dt
      $$
    is the usual integration along the paths.
    
    Note that there is no equivalent to this operator in ordinary differential geometry since there is no concept of differential forms on the space of paths,
    even for a manifold.
    Of course,
    there are a few bypasses,
    but none is as direct or efficient as the operator $\CHK$,
    as we shall see now.
    \artend
  \end{article}
  
  \begin{article}\artlabel[Homotopic invariance of de Rham cohomology]
    Consider an homotopy $t \mapsto f_t$ in $\Paths(\Cinfty(\X,\X'))$,
    where $\X$ and $\X'$ are two diffeological spaces.
    
    \textsc{Proposition}. \textit{Let $\alpha \in \Omega^k(\X')$ and $d\alpha = 0$.
    Then,
    $f_1^*(\alpha) = f_0^*(\alpha) + d\beta$,
    with $\beta \in \Omega^{k-1}(\X)$.}
    
    In other words,
    the de Rham cohomology is homotopic invariant:
    $$
      \class(f_1^*(\alpha)) = \class(f_0^*(\alpha)) \in \H_\dR^k(\X).
      $$
    Let us prove this rapidly.
    Consider the smooth map $\varphi \colon \X \to \Paths(\X')$ defined by $\varphi(x) = [t \mapsto f_t(x)]$.
    Take the pullback of the chain-homotopy identity,
    $\varphi^*(\CHK(d\alpha) + d(\CHK\alpha)) = \varphi^*(\1^*(\alpha)) - \varphi^*(\0^*(\alpha))$.
    That is,
    $\varphi^*(d(\CHK\alpha)) = d(\varphi^*(\CHK\alpha))= (\1 \circ \varphi)^*(\alpha) - (\0 \circ \varphi)^*(\alpha)$.
    But,
    $\hat t \circ \varphi = f_t$,
    thus $f_1^*(\alpha) -f_0^*(\alpha) = d\beta$ with $\beta = \varphi^*(\CHK\alpha)$.
    
    That is one of the most striking uses of this chain-homotopy operator,
    and it proves at the same time how one can take advantage of diffeology,
    even in a traditional course on differential geometry.
    \artend
  \end{article}
  
  \begin{article}\artlabel[Integration of closed $1$-forms]
    Consider a manifold $\M$ and a closed $1$-form $\alpha$ on $\M$.
    We know that if $\alpha$ is integral,
    that is,
    its integral on every loop is a multiple of some number called the \emph{period},
    then there exists a smooth function $f$ from $\M$ to the circle $\S^1$ such that $\alpha = f^*(\theta)$,
    where $\theta$ is the canonical length element.
    This specific construction has an ultimate generalization in diffeology that avoids the integral condition~--~and that is what diffeology is for,
    indeed.
    
    Let $\alpha$ be a closed $1$-form on a connected diffeological space $\X$.
    Consider the equivalence relation on $\Paths(\X)$ defined by
    $$
      \gamma \sim \gamma' \quad\text{if}\quad \ends(\gamma) = \ends(\gamma') \quad\text{and}\quad \int_\gamma \alpha = \int_{\gamma'} \alpha.
      $$
    The quotient $\fX_\alpha = \Paths(\X)/\!\sim$ is a groupoid for the addition%
    \footnote{Actually,
    this addition is defined on the stationary paths,
    but since the space of stationary paths is a deformation retract of the space of paths,
    that does not really matter.}
    $\class(\gamma) + \class(\gamma') = \class(\gamma \vee \gamma')$,
    when $\gamma(1) = \gamma'(0)$.
    Because the integral of $\alpha$ on $\gamma$ does not depend on its fixed-endpoints homotopy class,
    the groupoid $\fX_\alpha$ is a covering groupoid,
    a quotient of the Poincaré groupoid $\fX$.
    Let $\F_\alpha \colon \fX_\alpha \to \RR$ be $\F_\alpha(\class(\gamma)) = \CHK\alpha(\gamma) = \int_\gamma \alpha$.
    
    $$
      \begin{tikzcd}[column sep=2.5em, row sep={5em,between origins}, every label/.append style = {font = \small}]
      \Paths(\X) \arrow[dr, swap, "\ends"] \arrow[r, "\class"] &  \fX \arrow[d] \arrow[r, "\class"]  & \fX_\alpha \arrow[r,"\F_\alpha"] \arrow[ld] & \RR \arrow[d,"\pr"] \\
      {} & \X \times \X \arrow[rr,swap,"f_\alpha"]  & {} & \T_\alpha = \RR/\P_\alpha
      \end{tikzcd}
      $$
    
    The function $\F$ integrates $\alpha$ on $\fX_\alpha$,
    in the sense that,
    thanks to the chain-homotopy identity,
    $d\F_\alpha = \1^*(\alpha) - \0^*(\alpha)$.
    Let $o \in \X$ be a point defined as an \emph{origin} of $\X$,
    and define $\X_\alpha$ to be the subspace of $\fX_\alpha$ made of classes of paths with origin $o$.
    Let $\pi = \1\restriction \X_\alpha$;
    then $d\F = \pi^*(\alpha)$,
    with $\F = \F_\alpha \restriction \X_\alpha$.
    The covering $\X_\alpha$ of $\X$ is the smallest covering where the pullback of $\alpha$ is exact.
    We call it the integration covering of $\alpha$.
    Its structure group $\P_\alpha$ is the \emph{group of periods} of $\alpha$,
    that is,
    $$
      \P_\alpha = \bigg\{ \int_\ell \alpha \mid \ell \in \Loops(\X) \bigg\}.
      $$
    
    \textsc{Proposition}. \textit{If the group of periods $\P_\alpha$ is a strict subgroup of $\RR$,
    then $\pr \colon \RR \to \T_\alpha = \RR/\P_\alpha$ is a covering,
    and there exists a smooth map $f\colon \X \to \T_\alpha$ such that $f^*(\theta) = \alpha$,
    with $\theta$ being the projection of $\dt$ on $\T_\alpha$ by $\pr$.}
    
    This proposition \cite[Section 8.29]{PIZ13} is the ultimate generalization of the integration of integral $1$-forms in differential geometry,
    allowed by diffeology.
    We can notice that the usual condition of second countability for manifolds is in fact a sufficient precondition.
    The real obstruction,
    valid for manifolds as well as for general diffeological spaces,
    is that the group of periods $\P_\alpha$ is discrete in $\RR$,
    and that is equivalent to being a strict subgroup of $\RR$.
    \artend
  \end{article}
  
  %%%%%%%%%%%%%%%%%%%%%%%%%%%%%%%%%%%%%%%%%%%%%%%%%%%%%%%%%%
  %%
  %% MARK: DOCUMENT Symplectic Diffeology
  %%
  %%%%%%%%%%%%%%%%%%%%%%%%%%%%%%%%%%%%%%%%%%%%%%%%%%%%%%%%%%
  \section*{Symplectic Diffeology}
  
  During the 1990s a lot of symplectic-like geometry situations were explored,
  essentially in infinite-dimensional spaces,
  but not only,
  with more or less success.
  What these attempts at generalization missed was a uniform framework of concepts and vocabulary,
  precise definitions framing the context of their studies.
  Each example came with its own heuristic and context,
  for example,
  the nature of the moment maps were not clearly stated:
  for the case of the moment of imprimitivity \cite{Zie96},
  it was a function with values the Dirac delta functions (distributions);
  for another example involving the connections of a torus bundle \cite{Dnl99},
  it was the curvature of the connection.
  And Elisa Prato's quasifolds \cite{Pra01},
  for which the moment map is defined on a space that is not a legitimate manifold but a singular quotient,
  adds up to these infinite-dimensional examples.
  
  What we shall see now is how diffeology is the missing framework,
  where all these examples find their places,
  are treated on an equal footing,
  and give what we are waiting for from them.
  
  The general objects of interest will be arbitrary closed $2$-forms on diffeological spaces,
  for which we introduce this new terminology:
  
  \textsc{Definition}. \textit{We call \emph{parasymplectic space}%
  \footnote{The quality of being \emph{symplectic} or \emph{presymplectic} will be discussed and get a precise meaning.
  The word \emph{parasymplectic} seemed free and appropriate to denote a simple closed $2$-form.}
  any diffeological space $\X$,
  equipped with a closed $2$-form $\omega$.}
  
  Next,
  we consider the group of \emph{symmetries} (or \emph{automorphisms}) of $\omega$,
  denoted by $\Diff(\X,\omega)$.
  The pseudogroup $\Diff_\loc(\X,\omega)$ of local symmetries will play some role too.
  Then,
  to introduce the \emph{moment map} for any group of symmetries $\G$,
  we need to clarify some vocabulary and notations:%
  \footnote{Remember that a diffeological group is a group that is a diffeological space such that the multiplication and the inversion are smooth.}
  
  \textsc{Definition}. \textit{We shall call \emph{momentum}%
  \footnote{Plural \emph{momenta}.}
  of a diffeological group $\G$ any left-invariant $1$-form.
  We denote by $\cG^*$ its space of \emph{momenta},
  that is,
  $$
    \cG^* = \{\varepsilon \in \Omega^1(\G) \mid \mathrm{L}(g)^*(\varepsilon) = \varepsilon, \mbox{ for all } g \in \G \}.
    $$
  The set $\cG^*$ is obviously a real vector space%
  \footnote{It is also a diffeological vector space for the functional diffeology,
  but we shall not discuss that point here.}.}
  
  \begin{article}\artlabel[The moment map]
    Let $(\X,\omega)$ be a parasymplectic space and  $\G$ be a diffeological group.
    A \emph{symmetric action} of $\G$ on $(\X,\omega)$ is a smooth morphism $g \mapsto g_\X$ from $\G$ to $\Diff(\X,\omega)$,
    where $\Diff(\X,\omega)$ is equipped with the functional diffeology.
    That is,
    $$
      \text{for all $g\in \G$}, \quad g_\X^*(\omega)=\omega.
      $$
    
    Now,
    to grab the essential nature of the moment map,
    which is a map from $\X$ to $\cG^*$,
    we need to understand it in the simplest possible case.
    That is,
    when $\omega$ is exact,
    $\omega = d\alpha$,
    and when $\alpha$ is also invariant by $\G$,
    $g_\X^*(\alpha) = \alpha$.
    In these conditions,
    the moment map is given by
    $$
      \mu \colon \X \to \cG^*
      \quad
      \text{with}
      \quad
      \mu(x) = \hat x^*(\alpha),
      $$
    where $\hat x \colon \G \to \X$ is the \emph{orbit map} $\hat x (g)= g_\X(x)$.
    We check immediately that,
    since $\alpha$ is invariant by $\G$,
    $\hat x^*(\alpha)$ is left invariant by $\G$,
    and therefore $\mu(x) \in \cG^*$.
    
    But,
    as we know,
    not all closed $2$-forms are exact,
    and even if they are exact,
    they do not necessarily have an invariant primitive.
    We shall see now,
    how we can generally come to a situation,
    so close to the simple case above,
    that,
    modulo some minor subtleties,
    we can build a good moment map in all cases.
    
    Let us consider now the general case,
    with $\X$ connected.
    Let $\CHK$ be the chain-homotopy operator, defined previously.
    Then,
    the differential $1$-form $\K\omega$,
    defined on $\Paths(\X)$,
    satisfies $d[\CHK\omega] = (\1^* - \0^*)(\omega)$,
    and $\CHK\omega$ is invariant by $\G$ \cite[Section 6.84]{PIZ13}.
    Considering $\bar\omega = (\1^* - \0^*)(\omega)$ and $\bar\alpha = \CHK\omega$,
    we are in the simple case:
    $\bar\omega = d\bar\alpha$ and $\bar\alpha$ invariant by $\G$.
    We can apply the construction above and then we define the \emph{paths moment map} by
    $$
      \Psi \colon \Paths(\X) \to \cG^*
      \quad \mbox{with} \quad
      \Psi(\gamma) = \hat \gamma^*(\CHK\omega),
      $$
    where $\hat \gamma \colon \G \to \Paths(\X)$ is the orbit map $\hat \gamma(g)= g_\X \circ \gamma$ of the path $\gamma$.
    
    The paths moment map is additive with respect to the concatenation,
    $$
      \Psi(\gamma \vee \gamma') = \Psi(\gamma) + \Psi(\gamma'),
      $$
    and it is equivariant by $\G$,
    which acts by composition on $\Paths(\X)$,
    and by coadjoint action on $\cG^*$.
    That is,
    for all $g,k \in \G$ and $\epsilon \in \cG^*$,
    $$
      \Ad(g)\colon k \mapsto gkg^{-1} \quad\text{and}\quad \Ad_*(g)\colon \epsilon \mapsto \Ad(g)_*(\epsilon) =  \Ad(g^{-1})^*(\epsilon).
      $$
    
    Then,
    we define the \emph{holonomy} of the action of $\G$ on $\X$ as the subgroup
    $$
      \Gamma = \{\Psi(\ell) \mid \ell \in \Loops(\X) \} \subset \cG^*.
      $$
    The group $\Gamma$ is made of (closed) $\Ad_*$-invariant momenta.
    But $\Psi(\ell)$ depends only on the homotopy class of $\ell$,
    so then $\Gamma$ is a homomorphic image of $\pi_1(\X)$,
    more precisely,
    its abelianized.
    
    \textsc{Definition}. \textit{If $\Gamma = \{0\}$,
    we say that the action of $\G$ on $(\X,\omega)$ is \emph{Hamiltonian}.
    The holonomy $\Gamma$ is the obstruction for the action of the group $\G$ to be Hamiltonian.}
    
    Now,
    we can push forward the paths moment map on $\cG^*/\Gamma$,
    as suggested by the commutative diagram
    $$
      \begin{tikzcd}[column sep=large, row sep=large, every label/.append style = {font = \small}]
      \Paths(\X) \arrow[d, swap, "\ends"] \arrow[r, "\Psi"] &  \cG^* \arrow[d, "\class"] \\
      \X \times \X \arrow[r, swap, "\psi"] & \cG^*\!\!/\Gamma
      \end{tikzcd}
      $$
    and we get then the \emph{two-points moment map}\,:
    $$
      \psi(x,x') = \class(\Psi(\gamma)) \in \cG^*\!\!/\Gamma, \quad\text{for any path $\gamma$ such that}\quad \ends(\gamma) = (x,x').
      $$
    The additivity of $\Psi$ becomes the Chasles's cocycle condition on $\psi$\,:
    $$
      \psi(x,x') + \psi(x',x'') = \psi(x,x'').
      $$
    
    The group $\Gamma$ is invariant by the coadjoint action.
    Thus,
    the coadjoint action passes to the quotient group $\cG^*\!\!/\Gamma$,
    and $\psi$ is a natural group-valued moment map,
    equivariant for this quotient coadjoint action.
    
    Because $\X$ is connected,
    there exists always a map
    $$
      \mu \colon \X \to \cG^*\!\!/\Gamma
      \quad \mbox{such that} \quad
      \psi(x,x') = \mu(x') - \mu(x).
      $$
    The solutions of this equation are given by
    $$
      \mu(x)=\psi(x_0,x) + c,
      $$
    where $x_0$ is a chosen point in $\X$ and $c$ is a constant.
    These are the \emph{one-point moment maps}.
    But these moment maps $\mu$ are a priori no longer equivariant.
    Their variance introduces a $1$-cocycle $\theta$ of $\G$ with values in $\cG^*\!\!/\Gamma$ such that
    $$
      \mu(g(x))= \Ad_*(g)(\mu(x)) + \theta(g), \quad \text{with} \quad \theta(g) = \psi(x_0,g(x_0)) + \Delta c(g),
      $$
    where $\Delta c$ is the coboundary due to the constant $c$ in the choice of $\mu$.
    We say that the action of $\G$ on $(\X,\omega)$ is \emph{exact} when the cocycle $\theta$ is trivial.
    Defining
    $$
      \Ad_*^\theta (g) \colon \nu \mapsto \Ad_*(g)(\nu) + \theta(g), \quad\text{then}\quad \Ad_*^\theta (gg') = \Ad_*^\theta (g) \circ \Ad_*^\theta (g').
      $$
    The cocycle property of $\theta$,
    that is,
    $\theta(gg') = \Ad_*(g)(\theta(g')) + \theta(g)$,
    makes $\Ad_*^\theta$ an action of $\G$ on the group $\cG^*\!\!/\Gamma$.
    This action is called the \emph{affine action}.
    For the affine action,
    the moment map $\mu$ is equivariant:
    $$
      \mu(g(x))= \Ad_*^\theta(g)(\mu(x)).
      $$
    
    This construction extends to the category \{Diffeology\},
    the moment map for manifolds introduced by Souriau in \cite{Sou70}.
    When $\X$ is a manifold and the action of $\G$ is Hamiltonian,
    they are the standard moment maps he defined there.
    The remarkable point is that none of the constructions brought up above involves differential equations,
    and there is no need for considering a possible Lie algebra either.
    That is a very important point.
    The momenta appear as invariant $1$-forms on the group,
    naturally,
    without intermediaries,
    and the moment map as a map in the space of momenta.
    
    Note that the group of automorphisms $\G_\omega = \Diff(\X,\omega)$ is a legitimate diffeological group.
    The above constructions apply and give rise to universal objects:
    \emph{universal momenta} $\cG_\omega^*$,
    \emph{universal path moment map} $\Psi_\omega$,
    \emph{universal holonomy} $\Gamma_\omega$,
    \emph{universal two-points moment map} $\psi_\omega$,
    \emph{universal moment maps} $\mu_\omega$,
    and \emph{universal Souriau's cocycles} $\theta_\omega$.
    
    A \emph{parasymplectic action} of a diffeological group $\G$ is a smooth morphism $h \colon \G \to \G_\omega$,
    and the objects,
    associated with $\G$,
    introduced by the above moment maps constructions,
    are naturally subordinate to their universal counterparts.
    
    We shall illustrate this construction by two examples in the next paragraphs.
    More examples can be found in \cite[Sections 9.27~--~9.34]{PIZ13}
    \artend
  \end{article}
  
  \begin{article}\artlabel[Example $1$: The moment of imprimitivity]
    Consider the cotangent space $\T^*\M$ of a manifold $\M$,
    equipped with the standard symplectic form $\omega = d\lambda$\,,
    where $\lambda$ is the Liouville form.
    Let $\G$ be the Abelian group $\Cinfty(\M,\RR)$.
    Consider the action of $\G$ on $\T^*\M$ defined by
    $$
      f \colon (x,a) \mapsto (x, a - d\!f_x),
      $$
    where $x \in \M$,
    $a \in \T^*_x\M$,
    and $df_x$ is the differential of $f$ at the point $x$.
    Then,
    the moment map is given by
    $$
      \mu \colon (x,a) \mapsto d[f \mapsto f(x)] = d[\delta_x],
      $$
    where $\delta_x$ is the Dirac distribution $\delta_x(f)=f(x)$.
    
    Since $\delta_x$ is a smooth function on $\Cinfty(\M,\RR)$,
    its differential is a $1$-form.
    Let us check that this $1$-form is invariant:
    
    Let $h \in \Cinfty(\M,\RR)$,
    $\L(h)^*(\mu(x)) = \L(h)^*(d[\delta_x])= d[\L(h)^*(\delta_x)]=d[\delta_x \circ \L(h)]$,
    but $\delta_x \circ \L(h) \colon f \mapsto \delta_x(f+h) = f(x) + h(x)$.
    Then,
    $d[\delta_x \circ \L(h)] = d[f \mapsto f(x) + h(x)] = d[f \mapsto f(x)] = \mu(x)$.
    
    We see that in this case,
    the moment map identifies with a function with values distributions but still has the definite formal statute of a map into the space of momenta of the group of symmetries.
    
    Moreover,
    this action is Hamiltonian and exact.
    This example,
    generalized to diffeological space,
    is developped in \cite[Exercise 147]{PIZ13}.
    \artend
  \end{article}
  
  \begin{article}\artlabel[Example $2$: The $1$-forms on a surface]
    Let $\Sigma$ be a closed surface,
    oriented by a  $2$-form $\Surf$.
    Consider  $\Omega^1(\Sigma)$,
    the infinite-dimensional vector space  of $1$-forms on $\Sigma$,
    equipped with the functional diffeology.
    Let $\omega$ be the antisymmetric bilinear map defined on $\Omega^1(\Sigma)$ by
    $$
      \omega(\alpha,\beta) = \int_\Sigma \alpha\wedge \beta,
      $$
    for all $\alpha$, $\beta$ in $\Omega^1(\Sigma)$.%
    \footnote{Since the exterior product $\alpha \wedge\beta$ is a $2$-form of $\Sigma$,
    there exists $\varphi \in \Cinfty(\Sigma,\RR)$ such that $\alpha \wedge \beta = \varphi \times \Surf$.
    By definition,
    $\int_\Sigma \alpha \wedge \beta = \int_\Sigma \varphi \times \Surf$.}
    With this bilinear form is naturally
    associated a differential $2$-form $\omega$ on $\Omega^1(\Sigma)$,
    defined by
    $$
      \omega(\P)_r(\delta r,\delta' r) =
      \int_\Sigma {\partial \P(r) \over \partial r}(\delta r)
      \wedge
      {\partial \P(r) \over \partial r}(\delta' r),
      $$
    for all $n$-plots $\P \colon \U \to \X$,
    for all  $r \in \U$, $\delta r$ and $\delta' r$ in
    $\RR^n$.
    Moreover,
    $\omega$ is the differential of the $1$-form $\lambda$ on $\Omega^1(\Sigma)$\,:
    $$
      \omega = d\lambda
      \qmbox{with}
      \lambda(\P)_r(\delta r) =
      \frac{1}{2} \int_\Sigma \P(r)\wedge {\partial \P(r) \over \partial r}(\delta r).
      $$
    Define now the action of the additive group $\Cinfty(\Sigma,\RR)$ on $\Omega^1(\Sigma)$ by
    $$
      f \colon \alpha \mapsto  \alpha + df.
      $$
    Then,
    $\Cinfty(\Sigma,\RR)$ acts by symmetry on $(\Omega^1(\Sigma), \omega)$,
    for all $f$ in $\Cinfty(\Sigma,\RR)$,
    $f^*(\omega) = \omega$.
    
    The moment map of $\Cinfty(\Sigma,\RR)$ on $\Omega^1(\Sigma)$ is given (up to a constant) by
    $$
      \mu \colon \alpha \mapsto d \bigg[ f \mapsto \int_\Sigma
      f \times d\alpha\bigg].
      $$
    The moment map is invariant by the action of $\Cinfty(\Sigma,\RR)$,
    that is,
    exact,
    and Hamiltonian.
    And here again,
    the moment map is a function with values distributions.
    
    Now,
    $\mu(\alpha)$ is fully characterized by $d\alpha$.
    This is why we find in the literature on the subject,
    that the moment map for this action is the exterior derivative
    (or curvature, depending on the authors) $\alpha \mapsto d\alpha$.
    As we see again in this example that,
    diffeology  gives a precise meaning by procuring a unifying context.
    One can find the complete conputation of this example in \cite[Section 9.27]{PIZ13}.
    \artend
  \end{article}
  
  \begin{article}\artlabel[Symplectic manifolds are coadjoint orbits]
    Because symplectic forms of manifolds have no local invariants,
    as we know thanks to Darboux's theorem,
    they have a huge group of automorphisms.
    This group is big enough to be transitive \cite{Boo69},
    so that the universal moment map identifies the symplectic manifold with its image,
    which,
    by equivariance,
    is a coadjoint orbit (affine or not) of its group of symmetries.
    In other words,
    coadjoint orbits are the universal models of symplectic manifolds.
    
    Precisely,
    let $\M$ be a connected Hausdorff manifold,
    and let $\omega$ be a closed $2$-form on $\M$.
    Let $\G_\omega = \Diff(\M, \omega)$ be its group of symmetries and $\cG^*_\omega$ its space of momenta.
    Let $\Gamma_\omega$ be the holonomy,
    and $\mu_\omega$ be a universal moment map with values in $\cG^*_\omega/\Gamma_\omega$.
    We have,
    then,
    the following:
    
    \textsc{Theorem}. \textit{The form $\omega$ is symplectic,
    that is non-degenerate,
    if and only if:
    \begin{enumerate}
      \item[1.] the group $\G_\omega$ is transitive on $\M$;
      \item[2.] the  universal moment map $\mu_\omega \colon \M \to \cG^*_\omega/\Gamma_\omega$ is injective.
    \end{enumerate}
    }
    
    This theorem is proved in \cite[Section 9.23]{PIZ13},
    but let us make some comments on the keys elements.
    Consider the closed $2$-form $\omega = (x^2 + y^2) \, dx \wedge dy$;
    one can show that it has an injective universal moment map $\mu_\omega$.
    But its group $\G_\omega$ is not transitive,
    since $\omega$ is degenerate in $(0,0)$,
    and only at that point.
    Thus,
    the transitivity of $\G_\omega$ \cite{Boo69} is necessary.
    
    Let us give some hint about the sequel of the proof.
    Assume $\omega$ is symplectic.
    Let $m_0, m_1 \in \M$ and $p$ be a path connecting these points.
    For all $f \in \Cinfty(\M,\RR)$ with compact support,
    let
    $$
      \F \colon t \mapsto e^{t\grad_\omega(f)}
      $$
    be the exponential of the symplectic gradient of the $f$.
    Then,
    $\F$ is a 1-parameter group of automorphisms,
    and its value on $\Psi_\omega(p)$ is:
    $$
      \Psi_\omega(p)(\F) = [f(m_1) - f(m_0)] \times \dt.
      $$
    Now,
    if $\mu_\omega(m_0)=\mu_\omega(m_1)$,
    then there exists a loop $\ell$ in $\M$ such that $\Psi_\omega(p)=\Psi_\omega(\ell)$.
    Applied to the $1$-plot $\F$,
    we deduce $f(m_1) = f(m_0)$ for all $f$.
    Therefore $m_0=m_1$,
    and $\mu_\omega$ is injective.
    
    Conversely,
    let us assume that $\G_\omega$ is transitive,
    and $\mu_\omega$ is injective.
    By transitivity,
    the rank of $\omega$ is constant.
    Now,
    let us assume that $\omega$ is degenerate,
    that is,
    $\dim (\ker(\omega)) > 0$.
    Since the distribution $\ker(\omega)$ is integrable,
    given two different points $m_0$ and $m_1$ in a characteristic,
    there exists a path $p$ connecting these two points and drawn entirely in the characteristic,
    that is,
    such that $dp(t)/\dt \in \ker(\omega)$ for all $t$.
    But that implies $\Psi_\omega(p)=0$ \cite[Section 9.20]{PIZ13}.
    Hence,
    $\mu_\omega(m_0) = \mu_\omega(m_1)$.
    But we assumed $\mu_\omega$ is injective.
    Thus,
    $\omega$ is nondegenerate,
    that is,
    symplectic.
    \artend
  \end{article}
  
  \begin{article}\artlabel[Hamiltonian diffeomorphisms]
    Let $(\X,\omega)$ be a parasymplectic diffeological space.
    We have seen above that the universal moment map $\mu_\omega$ takes its values into the quotient $\cG^*_\omega/\Gamma_\omega$,
    where the holonomy $\Gamma_\omega \subset \cG^*_\omega$ is a subgroup made of closed $\Ad_*$-invariant $1$-forms on $\G$.
    This group $\Gamma_\omega$ is the very obstruction for the action of the group of symmetries $\G_\omega = \Diff(\X,\omega)$ to be Hamiltonian.
    
    \textsc{Proposition}. \textit{There exists a largest connected subgroup $\Ham(\X,\omega) \subset \G_\omega$ with vanishing holonomy.
    This group is called the group of \emph{Hamiltonian diffeomorphisms}.
    Every Hamiltonian smooth action on $(\X,\omega)$ factorizes through $\Ham(\X,\omega)$.}
    
    Hence,
    the moment map $\bar\mu_\omega$ with respect to the action of $\H_\omega = \Ham(\X,\omega)$ takes its values in the vector space of momenta $\cH^*_\omega$.
    One can prove also that the Hamiltonian moment map $\bar\mu_\omega$ is still injective when $\X$ is a manifold.
    It is probably an embedding in the diffeological sense \cite[Section 2.13]{PIZ13},
    but that has still to be proved.
    
    Perhaps most interesting is how this group is built.
    Let $\G_\omega^\circ$ be the neutral component of $\G_\omega$,
    and $\pi \colon \tilde \G_\omega^\circ \to \G_\omega^\circ$ be its universal covering.%
    \footnote{For universal covering of diffeological groups,
    see \cite[Section 7.10]{PIZ13}.}
    For all $\gamma \in \Gamma_\omega$,
    let $\F_\gamma \in \Hom^\infty(\tilde \G_\omega^\circ,\RR)$ be the primitive of $\pi^*(\gamma)$,
    that is,
    $d\F_\gamma = \pi^*(\gamma)$ and $\F_\gamma(\id_{\tilde \G_\omega^\circ}) = 0$.
    Next,
    let
    $$
      \F \colon \tilde \G_\omega^\circ \to \RR^{\Gamma_\omega} = \prod_{\gamma \in \Gamma_\omega} \RR
      \quad\text{defined by}\quad
      \F(\tilde g) = (\F_\gamma(\tilde g))_{\gamma \in \Gamma_\omega}.
      $$
    The map $\F$ is a smooth homomorphism,
    where $\RR^{\Gamma_\omega}$ is equipped with the product diffeology.
    Then \cite[Section 9.15]{PIZ13},
    $$
      \Ham(\X,\omega) = \pi (\ker (\F)).
      $$
    This definition gives the same group of Hamiltonian diffeomorphisms when $\X$ is a manifold \cite[Section 9.16]{PIZ13}.
    
    Now,
    let $\P_\gamma$ be the group of periods of $\gamma \in \Gamma_\omega$,
    that is,
    $$
      \P_\gamma = \F(\pi_1(\G_\omega^\circ,\id_{\G_\omega^\circ})) = \bigg\{ \int_\ell \gamma \mid \ell \in \Loops(\G_\omega^\circ,\id) \bigg\}.
      $$
    Then,
    let
    $$
      \PP_\omega = \prod_{\gamma \in \Gamma_\omega} \P_\gamma
      \quad\text{and}\quad
      \TT_\omega = \prod_{\gamma \in \Gamma_\omega} \T_\gamma
      \quad\text{with}\quad
      \T_\gamma = \RR/\P_\gamma.
      $$
    Each homomorphism $\F_\gamma$ projects onto a smooth homomorphism $f_\gamma \colon \G_\omega^\circ \to \T_\gamma$,
    and the homomorphism $\F$ projects onto a smooth homomorphism $f \colon \G_\omega^\circ \to \TT_\omega$,
    according to the following commutative diagram of smooth homomorphisms:
    $$
      \begin{tikzcd}[column sep=large, row sep=large, every label/.append style = {font = \small}]
      \tilde \G_\omega^\circ \arrow[d, swap, "\pi"] \arrow[r, "\F"]  & \RR^{\Gamma_\omega} \arrow[d,"\pr"] \\
      \G_\omega^\circ \arrow[r, swap, "f"] & \TT_\omega
      \end{tikzcd}
      $$
    with $\ker(\pi) = \pi_1(\G_\omega^\circ,\id_{\G_\omega^\circ})$,
    $\ker(\pr) = \PP_\omega = \F(\ker(\pi))$.
    We get,
    then,
    $$
      \Ham(\X,\omega) = \ker (f),
      $$
    an alternative,
    somewhat intrinsic,
    definition of the group of Hamiltonian diffeomorphisms as the kernel of the \emph{holonomy homomorphism} $f$.
    \artend
  \end{article}
  
  %%%%%%%%%%%%%%%%%%%%%%%%%%%%%%%%%%%%%%%%%%%%%%%%%%%%%%%%%%
  %%
  %% MARK: DOCUMENT In Conclusion
  %%
  %%%%%%%%%%%%%%%%%%%%%%%%%%%%%%%%%%%%%%%%%%%%%%%%%%%%%%%%%%
  \section*{In Conclusion}
  
  We have proposed in this text to switch from the rigid category \{Manifolds\} to
  the flexible category \{Diffeology\},
  which is now well developed.
  This is a category closed under every set-theoretic operation,
  complete,
  co-complete,
  and Cartesian closed,
  which includes on an equal footing:
  manifolds,
  singular quotients,
  and infinite-dimensional spaces.
  It is an ideal situation already,
  from the pure point of view of categoricians,
  and mainly the reason for the interest of the theory in that discipline;
  see for example
  ``Model Category Structures'',
  \cite{ChrWu14} \cite{Kih19},
  or ``Differentiable Homotopy Theory''%
  \footnote{Opposed to intrinsic geometric homotopy theory,
  which we described previously in this chapter.}
  \cite{IwaNob19}.%
  \footnote{The Japanese school is very productive in these fields these days,
  using diffeology as a tool or a general framework.
  One can consult other papers on the subject,
  already published or not,
  for examples \cite{SYH18, Kih19-b, Har19, HaSh19, Kur19}.}
  
  But of course,
  the primary interest of diffeology lies first and foremost in its very strength in geometry.
  The geometer will find pleasant and useful the flexibility of diffeology,
  to extend in a unique formal and versatile framework,
  different constructions in various fields,
  without inventing each time a heuristic framework that momentarily satisfies its needs.
  For example,
  the construction of the moment map and the integration of any closed $2$-form on any diffeological space \cite[Section 8.42]{PIZ13},
  are the prerequisite for an extension of symplectic geometry on spaces that are not manifolds,
  but that have bursted into mathematical physics these last decades with the problems of quantization and field theory.
  
  Then,
  beyond all these circumstances and technicalities,
  what does diffeology have to offer on a more formal or conceptual level?
  The answer lies partly in Felix Klein's Erlangen program \cite{Kle72}\,:
  
  \begin{quote}
    The totality of all these transformations we designate as the {\it principal group} of space-transformations;
    {\it geometric properties are not changed by the transformations of the principal group}.
    And,
    conversely,
    {\it geometric properties are characterized by their remaining invariant under the transformations of the principal group}...
    
    As a generalization of geometry arises then the following comprehensive problem:
    
    {\it Given a manifoldness and a group of transformations of the same;
    to investigate the configurations belonging to the manifoldness with regard to such properties as are not altered by the transformations of the group.}
  \end{quote}
  
  As we know,
  these considerations are regarded by mathematicians as the modern understanding of the word/concept of {\em geometry}.
  A geometry is given as soon as a space and a group of transformations of this space are given.%
  \footnote{Jean-Marie Souriau reduces the concept of geometry to the group itself \cite{Sou03}
  But this is an extreme point of view I am not confident to share,
  for several reasons.}.
  
  Consider Euclidean geometry,
  defined by the group of Euclidean transformations,
  our {\em principal group},
  on the Euclidean space.
  We can interpret,
  for example,
  the {\em Euclidean distance} as the invariant associated with the action of the Euclidean group on the set of pairs of points.
  We can superpose a pair of points onto another pair of points,
  by an Euclidean transformation,
  if and only if the distance between the points is the same for the two pairs.%
  \footnote{We could continue with the case of triangles and other elementary constructions~--~circles,
  parallels etc.~--~and a comparison between Euclidean and symplectic geometry,
  for example,
  from a strict Kleinian point of view.
  See the discussion in \cite{PIZ02}.}
  Hence,
  {\em geometric properties} or {\em geometric invariants} can be regarded as the orbits of the {\em principal group} in some spaces built on top of the principal space,
  and also as fixed/invariant points,
  since an orbit is a fixed point in the set of all the subsets of that space.
  In brief,
  what emerges from these considerations suggested by Felix Klein's principle is the following:
  
  \textsc{Claim.}~A geometry is associated with/defined by a {\em principal group} of transformations of some space,
  according to Klein's statement.
  The various geometric properties/invariants are described by the various actions of the principal group on spaces built on top of the principal space:
  products,
  sets of subsets,
  and so on.
  Each one of these properties,
  embodied as orbits,
  stabilizers,
  quotients,
  and so on,
  captures a part of this geometry.
  
  Now,
  how does diffeology fits to this context?
  
  \hspace{1em}- One can regard a diffeological space as the collection of the plots that gives its structure.
  That is the {\em passive approach}.
  
  \hspace{1em}- Or we can look at the space through the action of its group of diffeomorphisms:%
  \footnote{We consider more precisely the action of the pseudo-group of local diffeomorphisms.}
  on itself,
  but also on its powers or parts or maps.
  That is the {\em active approach}.
  
  This dichotomy appears already for manifolds,
  where the change of coordinates (transition functions of an atlas) is the passive approach.
  The active approach,
  as the action of the group of diffeomorphisms,
  is often neglected,
  and there are a few reasons for that.
  Among them,
  the group of diffeomorphisms is not a Lie group {\em stricto sensu}~--~it does not fit in the category \{Manifolds\}~--~and that creates a psychological issue.
  A second reason is that its action on the manifold itself is transitive~--%
  \footnote{Generally,
  manifolds are regarded as connected,
  Haussdorf,
  and second countable.}
  there are no immediate invariants,
  one having first to consider some secondary/subordinate spaces to make the first invariants appear.
  
  These obstacles,
  psychological or real,
  vanish in diffeology.
  First of all,
  the group of diffeomorphisms is naturally a diffeological group.
  And here is an example of space,
  where the action of the group of diffeomorphisms,
  the principal group in the sense of Klein,
  captures a good preliminary image of its geometry.%
  \footnote{And not just of its topology.}
  
  \bigskip
  \begin{figure}[ht]
    \includegraphics[width=.475\textwidth]{Figures/Square.pdf}
    \vspace{-0.5\baselineskip}
    \caption{Figure \ref{the-Square} --- A diffeomorphism of the square.}
    \label{the-Square}
  \end{figure}
  \bigskip
  
  Consider the square $\Sq = \mathopen[-1,+1\mathclose]^2$,
  equipped with the subset diffeology.
  Its decomposition in orbits,
  by its group of diffeomorphisms,
  is the following:
  
  \begin{enumerate}
    \item[1.] the 4-corners-orbit;
    \item[2.] the 4-edges-orbit;
    \item[3.] the interior-orbit.
  \end{enumerate}
  
  Any diffeomorphism preserves separately the interior of the square and its border,
  which is a consequence of the D-topology.%
  \footnote{A diffeomorphism is in particular an homeomorphism for the D-topology.
  We can use that, or homotopy.}
  But the fact that a diffeomorphism of the square cannot map a corner into the interior of an edge is a typical smooth property
  (see \cite{GIZ16-b, GIZ17-a},
  and some comments on more general stratified spaces \cite{GIZ17-b}).
  
  On the basis of this simple example,
  we can experiment the Klein's principle with the group of diffeomorphisms of a non-transitive diffeological space.
  The square being naturally an object of the theory,
  there is no need for heuristic extension here.%
  \footnote{See paragraph \ref{Manifolds-with-Boundary-and-Corners} above.}
  
  \textsc{Claim.} \textit{Hence,
  considering the group of diffeomorphisms of a diffeological space as its principal group,
  we can look at diffeology as the formal framework that makes {\em differential geometry},
  the geometry ---~in the sense of Felix Klein~--- of the group of diffeomorphisms.
  Or possibly,
  the (larger) pseudogroup of local diffeomorphisms.}
  
  That principle,
  in the framework of diffeology,
  can be regarded as the definitive expression of Souriau's point of view,
  developed in his paper ``Les groupes comme universaux'' \cite{Sou03}.
  
  Because diffeology is a such large and stable category that encompasses satisfactorily so many various situations,
  from singular quotients to infinite dimensions,
  mixing even these cases \cite{PIZ16},
  one can believe that this interpretation of diffeology fulfills its claim and answers in some sense to the second part of the title of this book.
    
  %%%%%%%%%%%%%%%%%%%%%%%%%%%%%%%%%%%%%%%%%%%%%%%%%%%%%%%%%%
  %%
  %% MARK: THE BIBLIOGRAPHY
  %%
  %%%%%%%%%%%%%%%%%%%%%%%%%%%%%%%%%%%%%%%%%%%%%%%%%%%%%%%%%%
  
\begin{thebibliography}{KriMic97}
    
    \bibitem[Bom67]{Bom67}
    {Jan Boman.}
    {\em Differentiability of a function and of its compositions with functions of one variable.}
    {Mathematica Scandinavica, 20, pp. 249--268 (1967)}.
    
    \bibitem[Boo69]{Boo69}
    {William M. Boothby.}
    {\em Transitivity of the automorphisms of certain geometric structures.}
    {Trans. Amer. Math. Soc., vol. 137, pp. 93--100 (1969).}
    
    \bibitem[Che77]{Che77}
    {Kuo-Tsai Chen.}
    {\em Iterated path integrals.}
    {Bull. of Am. Math. Soc., 83(5), pp. 831--879 (1977)}.
    
     \bibitem[ChrWu14]{ChrWu14}
     {Dan Christensen and Enxin Wu.}
     {\em The Homotopy Theory Of Diffeological Spaces.}
     {N. Y. J. Math. 20, 1269-1303 (2014).}

    \bibitem[DS75]{DinSin75}
    Efim Dinaburg and Yaacov Sinai.
    {\em The one-dimensional schr{\"o}dinger equation with a quasiperiodic potential.}
    {Funct. Anal. Appl., vol. 9, pp. 279---289 (1975).}
    
    \bibitem[Dnl99]{Dnl99}
    {Simon K. Donaldson.}
    {\em Moment maps and diffeomorphisms.}
    {Asian Journal of Math, 3, pp. 1--16 (1999).}
    
    \bibitem[DonIgl83]{DonIgl83}
    {Paul Donato and Patrick Iglesias.}
    {\em Exemple de groupes différentiels : flots irrationnels sur le tore.}
    {Preprint CPT-83/P.1524, Centre de Physique Théorique, Marseille (July 1983).}
    {Published in {\em Comptes Rendus de l'Acad{\'e}mie des Sciences}, 301(4), Paris (1985).}
    \newline
    \newblock{\scriptsize\verb!http://math.huji.ac.il/~piz/documents/EDGDFISLT.pdf!}
    
    \bibitem[Don84]{Don84}
    {Paul Donato.}
    {\em Rev\^etement et groupe fondamental des espaces diff\'e\-rentiels homog\`enes.}
    {Th\`ese de doctorat d'\'etat, Universit\'e de Provence, Marseille (1984).}
    
    \bibitem[GuiPol74]{GuiPol74}
    {Victor Guillemin and Alan Pollack.}
    {\em Differential topology.}
    {Prentice Hall, New Jersey (1974)}.
    
    \bibitem[GIZ16-a]{GIZ16-a}
    {Serap Gürer and Patrick Iglesias-Zemmour.}
    {\em $k$-Forms on Half-Spaces.}
    {Blog post (2016).}
    \newline
    \newblock{\scriptsize\verb|http://math.huji.ac.il/~piz/documents/DBlog-Rmk-kFOHS.pdf|}
    
    \bibitem[GIZ16-b]{GIZ16-b}
    \bysame
    {\em Diffeomorphisms Of The Square.}
    {Blog post (2016).}
    \newline
    \newblock{\scriptsize\verb|http://math.huji.ac.il/~piz/documents/DBlog-Rmk-DOTS.pdf|}
    
    \bibitem[GIZ17-a]{GIZ17-a}
    \bysame
    {\em Differential Forms On Manifolds With Boundary and Corners.}
    {ePrint (2017).}
    \newline
    \newblock{\scriptsize\verb|http://math.huji.ac.il/~piz/documents/OMWBAC.pdf|}
    
    \bibitem[GIZ17-b]{GIZ17-b}
    \bysame
    {\em Differential Forms On Stratified Spaces I \& II.}
    {Bulletin of the Australian Mathematical Society, 98, (2018), pp. 319-330; 99 (2019), pp. 311–318.}
    \newline
    \newblock{\scriptsize\verb|http://math.huji.ac.il/~piz/documents/DFOSS.pdf|}
    \newblock{\scriptsize\verb|http://math.huji.ac.il/~piz/documents/DFOSS-A.pdf|}

  \bibitem[Har19]{Har19}
    Tadayuki Haraguchi.
   {\em Homotopy structures of smooth CW complexes.}
    {preprint, arXiv:1811.06175.}

    \bibitem[HaSh19]{HaSh19}
    Tadayuki Haraguchi, Kazuhisa Shimakawa;
    {\em A model structure on the category of diffeological spaces.}
    {\em preprint, arXiv:1311.5668.}

    \bibitem[Igl85]{Igl85}
    {Patrick Iglesias.}
    {\em Fibrations diff{\'e}ologique et homotopie.}
    {Th{\`e}se d'{\'e}tat, Universit{\'e} de Provence, Marseille (1985).}
    
    \bibitem[Igl87]{Igl87}
     \bysame
    {\em Connexions et diff{\'e}ologie.}
    {In {Aspects dynamiques et topologiques des groupes infinis de transformation de la m{\'e}canique},
    vol.~25 of {\em Travaux en cours}, pp. 61--78, Hermann, Paris (1987).}
    \newline
    \newblock{\scriptsize\verb|http://math.huji.ac.il/~piz/documents/CED.pdf|}
    
    \bibitem[IglLac90]{IglLac90}
    Patrick Iglesias and Gilles Lachaud.
    {\em Espaces diff\'erentiables singuliers et corps de nombres alg\'ebriques.}
    Ann. Inst. Fourier, Grenoble, 40(1), pp. 723--737 (1990).
    
    \bibitem[PIZ02]{PIZ02}
    Patrick Iglesias-Zemmour.
    {\em Aperçu des Origines de la Mécanique Symplectique.}
    In Actes du Colloque “Histoire des géométries”, vol. I, Maison des  Sciences de l'Homme, Paris (2004).
    \newline
    \newblock{\scriptsize\verb|http://math.huji.ac.il/~piz/documents/AOGS-MSH.pdf|}
    
    \bibitem[PIZ07]{PIZ07}
    \bysame
    {\em Dimension in diffeology.}
    {Indagationes Mathematicae, 18(4) (2007).}
    
    \bibitem[PIZ10]{PIZ10}
     \bysame
    {\em The moment maps in diffeology.}
    {Memoirs of the American Mathematical Society, vol. 207, Am. Math. Soc., Providence RI (2010).}
    
    \bibitem[PIZ13]{PIZ13}
     \bysame
     {\em Diffeology.}
    {Mathematical Surveys and Monographs, vol. 185, Am. Math. Soc., Providence RI (2013).}
    \newline
    \newblock{\scriptsize\verb|http://www.ams.org/bookstore-getitem/item=SURV-185|}
    
    \bibitem[PIZ13-a]{PIZ13-a}
     \bysame
    {\em Variations of integrals in diffeology.}
    {Canad. J. Math., vol. 65 (6) (2013) pp. 1255–1286.}
    
    \bibitem[PIZ16]{PIZ16}
     \bysame
    {\em Example of singular reduction in symplectic diffeology.}
    {Proc. Amer. Math. Soc., 144(2):1309--1324 (2016).}
    
    \bibitem[PIZ16-b]{PIZ16-b}
     \bysame
    {\em The Geodesics Of The 2-Torus.}
    {Blog post (2016).}
    \newline
    \newblock{\scriptsize\verb|http://math.huji.ac.il/~piz/documents/DBlog-Rmk-TGOT2T.pdf|}
    
    \bibitem[PIZ16-c]{PIZ16-c}
     \bysame
    {\em Diffeology Of The Space of Geodesics.}
    {Work in progress.}
    %\newline
    %\newblock{\scriptsize\verb|http://math.huji.ac.il/~piz/documents/DBlog-Rmk-PSSOTSOG.pdf|}
    
    \bibitem[PIZ17]{PIZ17}
     \bysame
    {\em Every Symplectic Manifolds Is A Coadjoint Orbit.}
    {In Proceedings of Science, volume 224 — Frontiers of Fundamental Physics 14 (FFP14) - Mathemarical Physics (2017).}
    \newline
    \newblock{\scriptsize\verb|http://math.huji.ac.il/~piz/documents/ESMIACO.pdf|}
    
    \bibitem[IKZ10]{IKZ10}
    {Patrick Iglesias, Yael Karshon and Moshe Zadka.}
    {\em Orbifolds as Diffeology.}
    {Transactions of the American Mathematical Society, 362(6), pp. 2811--2831. Am. Math. Soc., Providence RI (2010).}
    
    \bibitem[IZL16]{IZL16}
    {Patrick Iglesias-Zemmour and Jean-Pierre Laffineur.}
    {\em Noncommutative Geometry and Diffeologies: The Case of Orbifolds.}
    {Journal of Noncommutative Geometry. Volume 12, Issue 4 (2018) pp. 1551–1572.}
    \newline
    \newblock{\scriptsize\verb|http://math.huji.ac.il/~piz/documents/CSAADTCOO.pdf|}
    
    \bibitem[IwaNob19]{IwaNob19}
    Norio Iwase and Nobuyuki Izumida.
    {\em Mayer-Vietoris sequence for differentiable/diffeological spaces.}
    {Algebraic Topology and Related Topics (Mohali, 2017), Trends in Mathematics, Birkh\"auser (2019) pp. 123-151.}

    \bibitem[Kih19]{Kih19}
    Hiroshi Kihara;
    {\em Model category of diffeological spaces.}
    {J. Homotopy Relat. Struct. 14 (2019), no. 1, 51-90.}

    \bibitem[Kih19-b]{Kih19-b}
    Hiroshi Kihara.
    {\em Quillen equivalences between the model categories of smooth spaces, simplicial sets, and arc-gengerated spaces.}
    {preprint, arXiv:1702.04070.}

    \bibitem[Kle72]{Kle72}
    {Felix Klein.}
    {\em Vergleichende Betrachtungen über neuere geometrische Forschungen.}
    {Math. Ann., 43 (1893), pp. 63–100.}
    {\& Verlag von Andreas Deichert, pp. 6-8 (1872).}
    \newline
    \newblock{\scriptsize\verb|http://www.deutschestextarchiv.de/book/view/klein_geometrische_1872|}
    
    \bibitem[KriMic97]{KriMic97}
    {Andreas Kriegl and Peter Michor.}
    {\em The convenient setting of global analysis.}
    {Mathematical Surveys and Monographs, vol. 53., Am. Math. Soc., Providence RI (1997).}

  \bibitem[Kur19]{Kur19}
    Katsuhiko Kuribayashi;
    {\em Simplicial cochain algebras for diffeological spaces.}
    {preprint, arXiv:1902.10937.}

    \bibitem[Lee06]{Lee06}
    {John M. Lee},
    {\em Introduction to Smooth Manifolds.}
    {Graduate Texts in Mathematics, Springer Verlag, New York (2006)}.
    
    \bibitem[MRW87]{MRW87}
    {Paul Muhly, Jean Renault and Dana Williams.}
    {\em Equivalence And Isomorphism For Groupoid $\CC^*$-Algebras.}
    {J. Operator Theory 17, no 1 pp. 3--22 (1987).}
    
    \bibitem[Omo86]{Omo86}
    {Stephen Malvern Omohundro.}
    {\em Geometric Perturbation Theory in Physics.}
    {World Scientific (1986).}
    
    \bibitem[Pra01]{Pra01}
    {Elisa Prato,}
    {\textit{Simple Non--Rational Convex Polytopes via Symplectic Geometry}.}
    {Topology, {40}, pp. 961--975 (2001).}
    
    \bibitem[Ren80]{Ren80}
    {Jean Renault.}
    {\textit{A groupoid approach to C*-Algebras}.}
    {Lecture notes in Mathematics (793), Springer-Verlag Berlin Heidelberg New-York (1980).}
    
    \bibitem[Rie81]{Rie81}
    {Marc Rieffel.}
    {\em {$C^*$}-algebras associated with irrationnal rotations.}
    {Pacific Journal Of Mathematics Vol. 93, No. 2, Tokyo (1981).}

    \bibitem[Sat56]{Sat56}
    {Ichiro Satake.}
    {\textit{On a generalization of the notion of manifold}.}
    {Proceedings of the National Academy of Sciences, 42, pp. 359--363 (1956).}
    
    \bibitem[Sat57]{Sat57}
     \bysame
    {\emph{The Gauss-Bonnet Theorem for V-manifolds}.
    Journal of the Mathematical Society of Japan, (9)4, pp. 464--492 (1957).}
    
    \bibitem[Sch75]{Sch75}
    {Gerald Schwarz.}
    {\em Smooth Functions Invariant under the Action of a Compact Lie Group},
    {Topology, 14, pp. 63--68 (1975).}

    \bibitem[SYH18]{SYH18}
    Kazuhiko Shimakawa, Kohei Yoshida, Tadayuki Haraguchi;
    {\em Homology and cohomology via enriched bifunctors.}
    {Kyushu J. Math. 72 (2018), no. 2, 239-252.}

    \bibitem[Sou70]{Sou70}
    {Jean-Marie Souriau.}
    {\em Structure des syst\`emes dynamiques.}
    {Dunod Editions, Paris (1970).}
    
    \bibitem[Sou80]{Sou80}
     \bysame
    {\em Groupes diff\'erentiels.}
    {Lect. Notes in Math., 836, pp. 91--128,}
    {Springer Verlag, New-York (1980).}
    
    \bibitem[Sou83]{Sou83}
     \bysame
    {\em Groupes différentiels et physique mathématique.}
    {Preprint CPT-83/P.1547, Centre de Physique Théorique, Marseille (October 1983).}
    \newline
    \newblock{\scriptsize\verb!http://math.huji.ac.il/~piz/documents-others/JMS-GDEPM-1983.pdf!}
    
    \bibitem[Sou84]{Sou84}
     \bysame
    {\em Un algorithme g{\'e}n{\'e}rateur de structures quantiques.}
    {Preprint CPT-84/PE.1694, Centre de Physique Th{\'e}orique, Marseille (1984).}
    
    \bibitem[Sou03]{Sou03}
     \bysame
    {\em Les groupes comme universaux.}
    In Dominique Flament, editor, {\em Histoires de g{\'e}om{\'e}tries},
    Documents de travail ({\'E}quipe F2DS). Fondation Maison des Sciences de
    l'Homme (2003).
    \newline
    \newblock{\scriptsize\verb!http://semioweb.msh-paris.fr/f2ds/docs/geo_2002/Document02_Souriau1.pdf!}
    
    \bibitem[Thu78]{Thu78}
    {William Thurston.}
    \textit{The Geometry and Topology of Three-Manifolds} (Chap. 13).
    {Princeton University lecture notes, years 1978--1981.}
    \newline
    \newblock{\scriptsize\verb|http://library.msri.org/books/gt3m/PDF/13.pdf|}
    
    \bibitem[Whi43]{Whi43}
    {Hassler Whitney.}
    {\textit{Differentiable even functions}.}
    {Duke Mathematics Journal, 10(1), pp. 159--160 (1943)}.
    
    \bibitem[Zie96]{Zie96}
    {Fran\c cois Ziegler.}
    {\em Th\'eorie de Mackey symplectique, in M\'ethode des orbites et repr\'esentations quantiques.}
    {Th\`ese de doctorat d'Uni\-ver\-sit\'e, Universit\'e de Provence, Marseille (1996).}
    \newline
    \newblock{\scriptsize\verb!http://arxiv.org/pdf/1011.5056v1.pdf!}
    
  \end{thebibliography}
  
    %%%%%%%%%%%%%%%%%%%%%%%%%%%%%%%%%%%%%%%%%%%%%%%%%%%%
  % Fin du document
  %%%%%%%%%%%%%%%%%%%%%%%%%%%%%%%%%%%%%%%%%%%%%%%%%%%%
  
\end{document}
