\documentclass[11pt,a4paper]{amsart}

%--------------------------------------------------------------------------
% Packages
%--------------------------------------------------------------------------
\usepackage[utf8]{inputenc}
\usepackage[T1]{fontenc}
\usepackage[french]{babel} % Document is in French
\usepackage{amssymb}
\usepackage{amsmath}
\usepackage{amscd}
\usepackage{tikz-cd}
\usepackage[hidelinks]{hyperref}

\usepackage{subcaption} %side by side diagrams
\usepackage[margin=0pt, font=small, justification=centering, labelformat=empty]{caption} %customises captions
\usepackage{floatrow}
\usepackage{mathrsfs} % fancy maths letters

%--------------------------------------------------------------------------
% Page Layout & Typography
%--------------------------------------------------------------------------
\parindent 0mm
\parskip .5ex plus 2pt
\linespread{1.1}

\parindent 0mm
\parskip .5ex plus 2pt

\renewcommand{\thesection}{\Roman{section}}
\numberwithin{equation}{subsection}

\usepackage{microtype}
\usepackage[cal=scr,uppercase = upright,greekfamily = didot,greeklowercase = upright,utopia]{mathdesign}
\usepackage[supspaced=.04em]{superiors}

\linespread{1.1}

%--------------------------------------------------------------------------
% Theorem Styles
%--------------------------------------------------------------------------
\theoremstyle{plain}
\newtheorem{proposition}{Proposition}[section]
\newtheorem{theoreme}[proposition]{Théorème}
\newtheorem{lemme}[proposition]{Lemme}

\theoremstyle{definition}
\newtheorem{definition}[proposition]{Définition}
\newtheorem{remarque}[proposition]{Remarque}

%--------------------------------------------------------------------------
% Macros
%--------------------------------------------------------------------------
\newcommand{\RR}{\mathbf{R}}
\newcommand{\ZZ}{\mathbf{Z}}
\newcommand{\CC}{\mathbf{C}}
\newcommand{\Cinfty}{C^\infty}
\newcommand{\Diff}{\mathrm{Diff}}
\newcommand{\Arc}{\mathrm{Arc}}
\newcommand{\Hom}{\mathrm{Hom}}
\newcommand{\Mor}{\mathrm{Mor}}
\newcommand{\dom}{\mathrm{dom}}
\newcommand{\im}{\mathrm{im}}
\newcommand{\id}{\mathrm{id}}

% Specific to this paper
\newcommand{\DL}{\mathrm{DL}} % Applications locales
\newcommand{\ReDomains}{\mathfrak{R}} % The set of domains R

%--------------------------------------------------------------------------
% Title Data
%--------------------------------------------------------------------------

\begin{document}
  
  \title{Connexions et Difféologie}
  
  \author{Patrick Iglesias}
  \address{Centre de Physique Théorique, CNRS Luminy Case 907, 13288 Marseille Cedex 9, France}
  \thanks{*) Présenté au Colloque International de Géométrie Symplectique, Lyon, Mai 1986.}
  \thanks{**) Laboratoire Propre du CNRS.}
  
  \date{Octobre 1986}
  
  \maketitle
  
  \begin{abstract}
    \textbf{Résumé :} Où il est question de fibrés difféologiques,
    de connexions et de l'ambiguïté de Gribov.
    
    \vspace{0.5em}
    
    \noindent \textbf{Abstract:} Where it is question of diffeological fibrations,
    connections and Gribov's ambiguity.
  \end{abstract}
  
  % ---- Introduction
  \section*{0 - Introduction}
  
  On a été amené,
  en physique,
  à étudier l'espace de toutes les connexions d'un fibré principal $\pi: X \to M$ (noté $C(\pi)$) considéré comme un espace de "champs" (théorie de Yang-Mills).
  En fait cet espace est plutôt un espace de "potentiels",
  l'action du groupe $J$ des transformations de jauge sur $C(\pi)$ étant inobservable.
  
  Ceci amène naturellement à étudier l'espace $C(\pi)/J$ et la projection $\Pi: C(\pi) \to C(\pi)/J$.
  Des études ont été faites sur cette application qui utilisent les méthodes topologiques (voir [6][5]).
  La question posée est la suivante :
  
  \begin{center}
    \emph{$\Pi$ est elle une fibration et, dans ces conditions, cette fibration est-elle triviale ?}
  \end{center}
  
  Ce qui physiquement s'interprèterait par l'existence d'une jauge globale.
  Nous allons voir que la difféologie est particulièrement bien adaptée pour traiter ce genre de question.
  Nous étudierons ici le cas $M$ non-compact,
  cas généralement ignoré par la topologie (pour le traitement difféologique du cas $M$ compact voir [3]).
  Faisons remarquer encore que l'approche traditionnelle qui suppose la variété de base (l'espace-temps) compacte cache des hypothèses de différentiabilité à l'infini qui n'ont aucune raison d'être vérifiées et qui sont en fait liées à un choix particulier de carte.
  Le traitement difféologique de cette question élimine cet inconvénient.
  
  Ce travail est une illustration de la théorie des espaces difféologiques.
  Cette théorie a été introduite par J.M. Souriau [7] pour aborder l'étude des groupes de dimension infinie.
  Elle a été ensuite développée par P. Donato en ce qui concerne les espaces homogènes [1],
  où il a notamment construit des revêtements d'espaces homogènes du groupe des symplectomorphismes de $\RR^{2n}$.
  Cette théorie a encore été appliquée à l'étude de ce qu'on a coutume d'appeler les "espaces singuliers",
  ce qui a permis de montrer qu'elle était aussi tout à fait adaptée à ce genre de question.
  L'étude de l'enroulement irrationnel du tore $T^2$ fait notamment apparaitre le groupe $GL(2,\ZZ)$,
  met en évidence les irrationnels quadratiques,
  et même les irrationnels diophantiens [2][3].
  
  Afin de répondre à certaines questions posées par la théorie des champs (comme l'ambiguïté de Gribov que nous traitons ici) il fallait développer l'aspect fibration de cette théorie [3].
  A cet effet nous avons introduit la \emph{micro-trivialité} qui étend la trivialité locale des fibrés de dimension finie.
  Cette extension préserve la suite exacte d'homotopie qui est un outil puissant en théorie des fibrés et que nous utilisons dans le problème qui nous intéresse.
  
  Ce travail contient aussi une ébauche de la théorie des connexions des fibrés difféologiques.
  Nous avons préféré le point de vue du relèvement des chemin plutôt que celui des sous-espaces horizontaux.
  
  Nous donnons au passage quelques résultats sur le groupe des transformations de jauge d'un fibré principal obtenu uniquement à partir de ces méthodes difféologiques,
  notamment sur la suite $1 \to J \to \text{Auto} \to \Diff$.
  Ainsi que quelques résultats généraux sur les fibrés difféologiques "connexables".
  
  % --- Rappels de difféologie
  \section{Rappels de difféologie}
  
  % --- Principes de difféologie
  \subsection{Principes de difféologie}
  
  Rappelons qu'une difféologie sur un ensemble $X$ est définie par la donnée d'un faisceau d'applications locales $\DL(\ReDomains, X)$ défini sur les ouverts de la réunion disjointe $\ReDomains$ des puissances finies de $\RR$ et à valeurs dans $X$,
  et vérifiant :
  \begin{equation} \label{I.1.1}
    \DL(\ReDomains, X) \text{ est couvrant } \heartsuit
  \end{equation}
  \begin{center}
    $\DL(\ReDomains, X)$ est saturé pour la composition par le faisceau $C^\infty(\ReDomains)$ $\diamondsuit$
  \end{center}
  
  Où \emph{couvrant} signifie que la réunion des images des éléments de $\DL(\ReDomains, X)$ est égal à $X$.
  Nous renvoyons le lecteur à [7] [1] et [3] pour plus de détails,
  à la fois pour la définition des espaces difféologiques et pour celles des notions de base comme les subductions,
  inductions,
  difféologie image,
  difféologie réciproque,
  etc...
  Nous proposons en annexe une présentation des difféologies basée sur le langage des faisceaux et que le lecteur pourra considérer comme une suite de remarques.
  
  % --- Les fibrés difféologiques
  \subsection{Les fibrés difféologiques}
  
  Nous décrivons ici,
  rapidement,
  la théorie des \emph{fibrés difféologiques},
  pour plus de détail le lecteur pourra se reporter à [3].
  
  Considérons deux espaces difféologiques $X$ et $M$ et une application différentiable $\pi$ de $X$ à $M$.
  L'ensemble des difféomorphismes des sous espaces difféologiques de $X$,
  images réciproques par $\pi$ des points de la base,
  constitue un groupoïde pour la composition :
  \begin{equation} \label{I.2.1}
    \Gamma(\pi) = \{ f \in \Diff(X_m, X_{m'}) \mid (m,m') \in M \times M \}
  \end{equation}
  On notera $\chi$ l'\emph{application caractéristique} du groupoïde,
  c'est à dire :
  \begin{equation} \label{I.2.2}
    \chi : \Gamma(\pi) \to M \times M \quad \chi(f) = (m,m') \iff f \in \Diff(X_m, X_{m'})
  \end{equation}
  $\Gamma(\pi)$ est muni de la difféologie de groupoïde la moins fine rendant la valuation différentiable :
  \begin{equation} \label{I.2.3}
    \text{valuation} : (f,x) \mapsto f(x) \text{ avec } x \in \dom(f)
  \end{equation}
  Le domaine de l'application valuation est l'image réciproque de l'application source définie sur $\Gamma(\pi)$ à valeur dans $M$ par :
  \begin{equation} \label{I.2.4}
    \text{source}(f) = m \iff \dom(f) = X_m
  \end{equation}
  L'application différentiable $\pi$ est appelée alors \emph{fibration difféologique} si l'application caractéristique $\chi$ est une subduction de $\Gamma(\pi)$ sur $M \times M$.
  
  Cette définition difféologique des fibrations s'interprète encore en terme de \emph{micro-trivialité} :
  l'application $\pi$ est une fibration si et seulement si l'image réciproque de $X$ par toute plaque de la base $M$ est localement triviale (au dessus de l'ouvert de $\ReDomains$ domaine de définition de la plaque en question).
  
  Il est clair que toutes les fibres d'une fibration sont difféomorphes entre elles,
  le type $F$ de cet espace difféologique est appelé évidemment \emph{fibre-type} de la fibration.
  Le groupe des difféomorphismes $\Diff(F)$ de la fibre type est appelé le \emph{groupe structural} de la fibration.
  Il existe une notion de réduction de fibration difféologique :
  étant donné un sous-groupoïde difféologique $K$ de $\Gamma(\pi)$ on dit que $\pi$ est réductible à $K$ si la restriction de $\chi$ à $K$ est encore une subduction.
  Le groupe structural est alors plus petit que $\Diff(F)$,
  c'est le groupe des difféomorphismes de $F$ dans $K$.
  
  La notion de fibration principale s'étend naturellement aux espaces difféologiques grâce à la proposition suivante,
  dont on trouvera la démonstration dans [3] :
  
  \subsubsection*{\emph{I.2.5.} Proposition.}
  Soit $G$ un groupe difféologique agissant différentiablement sur un espace difféologique $X$.
  Soit $L$ l'application de $G \times X$ dans $X \times X$ définie par :
  $$
    L(g,x) = (x, g(x)) \quad (g,x) \in G \times X \quad \heartsuit
    $$
  La projection canonique $\pi$ de $X$ sur $X/G$ est une fibration difféologique de fibre type $G$ si et seulement si $L$ est une induction.
  On dit dans ce cas que $\pi$ est une \emph{fibration difféologique principale}.
  
  On peut démontrer alors qu'au groupoïde structural de toute fibration difféologique est attaché de façon canonique un fibré (difféologique) principal appelé \emph{fibré principal structural} auquel le fibré initial est associé.
  
  De nombreuses constructions standards de la théorie classique des fibrés passent aux fibrés difféologiques,
  par exemple :
  l'image réciproque d'un fibré difféologique est encore un fibré difféologique etc...
  Une autre notion passe aux fibrés difféologiques :
  la suite exacte d'homotopie.
  
  % --- Homotopie difféologique
  \subsection{Homotopie difféologique}
  
  La théorie des espaces difféologiques a l'avantage de permettre une présentation synthétique de la théorie de l'homotopie.
  Cette théorie est l'étude des propriétés de connexité d'espaces de lacets.
  Il suffit de remarquer que ces espaces sont naturellement des espaces difféologiques.
  
  Etant donné un espace difféologique $X$,
  on appelle partition en composantes de $X$ la plus fine partition qui fait de $X$ la somme (difféologique) de ses parties.
  Les éléments de cette partition sont appelés \emph{composantes} de $X$,
  l'ensemble des composantes de $X$ est noté $\pi_0(X)$.
  Un espace est \emph{connexe} s'il ne possède qu'une composante,
  composante que l'on note $X$.
  Un espace difféologique est alors connexe.
  Cette définition coïncide avec la connexité par arcs différentiables et la connexité de la D-topologie (voir [3]).
  Considérons un autre espace difféologique $Y$ et soit $\DL(X,Y)$ le faisceau des applications localement différentiables de $X$ à $Y$.
  Cet espace possède une difféologie canonique que l'on appelle \emph{difféologie fonctionnelle} et définie par :
  
  Soit $\Gamma$ l'ensemble des couples $(f,x)$ tels que $f \in \DL(X,Y)$ et $x \in \dom(f)$,
  $\Gamma \subset \DL(X,Y) \times X$,
  la difféologie fonctionnelle (ouverte) est définie comme la moins fine difféologie de $\DL(X,Y)$ qui fasse de la valuation :
  $(f,x) \mapsto f(x)$ une application différentiable locale.
  
  Considérons maintenant l'espace $\Arc(X) = \DL(\RR, X)$ des applications différentiables de $\RR$ dans $X$,
  muni de sa difféologie d'espace fonctionnel.
  Notons $\chi$ l'application (source, but) définie sur $\Arc(X)$ par :
  \begin{equation} \label{I.3.1}
    \chi : c \mapsto (c(0), c(1))
  \end{equation}
  $\chi$ partage $\Arc(X)$ en classes d'équivalences (les arcs ayant mêmes origine et extrémité),
  on munit alors $\Arc(X)$ de la difféologie feuilletée par $\chi$ et on note $\Arc_\chi(X)$ l'espace des composantes connexes de $\Arc(X)$ pour la difféologie feuilletée par $\chi$,
  muni de la difféologie image de la difféologie fonctionnelle de $\Arc(X)$ (Ouf!).
  L'application $\chi$ se factorise sur $\Arc_\chi(X)$ et on montre que $\Arc_\chi(X)$ est un groupoïde difféologique de base $X$ pour la juxtaposition des arcs,
  c'est le \emph{groupoïde fondamental} de $X$,
  son groupe structural est le \emph{premier groupe d'homotopie} de $X$,
  il est noté $\pi_1(X)$,
  tandis que son fibré principal est le \emph{revêtement universel} de $X$.
  
  Le groupe $\pi_1(X)$ est isomorphe au groupe $\pi_1(X, x_0)$ des composantes connexes de l'espace des lacets différentiables de $X$ d'origine $x_0$ qui est noté $L(X, x_0)$.
  On établit sur l'espace des lacets d'origine $x_0$ la récurrence suivante :
  \begin{equation} \label{I.3.2}
    L_n(X, x_0) = L(L_{n-1}(X, x_0), x_0^n), \quad x_0^n = [t \mapsto x_0^{n-1}] \quad \spadesuit
  \end{equation}
  $$
    L_1(X, x_0) = L(X, x_0), \quad x_0^1 = x_0 \quad \diamondsuit
    $$
  Où l'indice supérieur $0$ dans $L_{n-1}^0(X, x_0)$ signifie la composante connexe de $x_0^{n-1}$.
  Les groupes d'homotopie supérieurs sont alors évidemment définis par :
  \begin{equation} \label{I.3.3}
    \pi_n(X, x_0) = \pi_0(L_n(X, x_0))
  \end{equation}
  La définition de l'homotopie relative n'offre pas plus de difficultés (pour plus de précision voir [3]).
  Un grand nombre de théorèmes classiques de la théorie de l'homotopie ont leur traduction en termes d'homotopie difféologique,
  notamment la propriété d'exactitude de la suite d'homotopie d'un couple $(X,A)$ où $A$ est un sous-espace difféologique de $X$ :
  \begin{multline} \label{I.3.4}
    \dots \to \pi_n(A, x_0) \to \pi_n(X, x_0) \to \pi_n(X, A, x_0) \to \pi_{n-1}(A, x_0) \to \dots \\
    \dots \to \pi_1(X,A,a) \to \pi_0(A, x_0)  \to \pi_0(X, x_0).
  \end{multline}
  Avec $x_0 \in A$.
  Ainsi, on peut démontrer l'exactitude de la suite d'homotopie des fibrés difféologiques :
  soit $(X, M, \pi)$ une fibration difféologique de fibre type $F = \pi^{-1}(m_0)$, $m_0 \in M$ et $x_0 \in F$ :
  \begin{multline} \label{I.3.5}
    \dots \to \pi_n(F, x_0) \to \pi_n(X, x_0) \to \pi_n(M, m_0) \to \pi_{n-1}(F, x_0) \to \dots\\
    \dots \to \pi_0(F, x_0) \to \pi_0(X, x_0) \to \pi_0(M, m_0).
  \end{multline}
  Comme exemples de fibrés difféologiques il y a évidemment tous les fibrés de variétés,
  mais aussi tous les quotients de groupes difféologiques par des sous-groupes.
  Ceci introduit des fibrés de dimension infinie comme des quotients singuliers.
  
  En particulier le quotient du tore $T^2$ par le flot irrationnel de pente $\alpha$,
  que nous notons $T_\alpha$,
  est un fibré difféologique.
  Son homotopie est celle de $T^2$ puisque la fibre est contractile,
  et son revêtement universel est $\RR$ dans lequel son groupe d'homotopie $\ZZ^2$ se réalise par $\ZZ + \alpha\ZZ$ (voir [2][3]).
  C'est notamment en classifiant les fibrés difféologiques principaux de groupe structural $\RR$ au dessus de $T_\alpha$ qu'apparaît la différence entre les nombres diophantiens et ceux de Liouville [3].
  
  % --- Connexions sur les fibrés difféologiques
  \section{Connexions sur les fibrés difféologiques}
  
  % --- Schéma général de connexions
  \subsection{Schéma général de connexions.}
  
  Les connexions ont été introduites initialement comme des procédés de relèvement de chemin.
  Elles s'identifiaient à la notion de transport parallèle.
  C'est suivant cette idée que nous allons définir les connexions difféologiques.
  
  Soit $\pi$ une application différentiable de $X$ dans $M$,
  considérons le morphisme $\Arc$ qui à $X$ associe $\Arc(X)$ (resp. à $M$ associe $\Arc(M)$) et à $\pi$ associe $\pi_*$ :
  \begin{equation} \label{II.1.1}
    \pi_* : \Arc(X) \to \Arc(M) \text{ avec } \pi_*(c) = \pi \circ c
  \end{equation}
  Les espaces $\Arc(X)$ et $\Arc(M)$ sont munis de leur difféologie d'espaces fonctionnels,
  $\pi_*$ est une application différentiable.
  Considérons le morphisme source $s = (s_X, s_M)$ qui à tout arc associe son origine :
  \begin{equation} \label{II.1.2}
    s_X, s_M : c \to c(0)
  \end{equation}
  Nous noterons $\kappa$ le morphisme canonique de l'application $\pi_*$ à l'image réciproque $\pi_*: X_s \to \Arc(M)$ de $\pi$ par $s$ (diagramme II.1),
  $X_s$ est défini par :
  \begin{equation} \label{II.1.3}
    X_s = \{ (c,x) \in \Arc(M) \times X \mid s_X(c) = \pi(x) \} \quad \clubsuit
  \end{equation}
  \begin{equation} \label{II.1.4}
    \pi_s(c,x) = c \quad s_\kappa(c,x) = x \quad \diamondsuit
  \end{equation}
  \begin{equation} \label{II.1.5}
    \forall c \in \Arc(X) \quad \kappa(c) = (\pi_*(c), c(0))
  \end{equation}
  Nous noterons $\sigma = (\sigma_X, \sigma_M)$ la section du morphisme $s$ définie par :
  $$
    \sigma_X : x \to x = [t \to x], \quad \sigma_M : m \to m = [t \to m]
    $$
    
  \begin{figure}[h!]
    \tikzset{column sep=small, ampersand replacement=\&}
\begin{floatrow}
    \centering
    \ffigbox{
    \begin{tikzcd}
    \Arc(X) \arrow[ddr, swap, "\pi_+"] \arrow[dr, "\kappa"] \arrow[drr, "s_X"]  \&  {}  \& {} \\
        {} \& X_s \arrow[r, swap, "s_\pi"] \arrow[d, "s_{\pi_s}"] \& X \arrow[d,"\pi"] \\
        {} \& \Arc(M) \arrow[r,"s_M"] \& M
    \end{tikzcd}
    }{\caption{Diagramme II.1.}}
    \ffigbox{
    \begin{tikzcd}
        \Arc(X)  \&  {}  \& X \arrow[ll, "\sigma_X"] \arrow[dl,"\sigma_{X_s}"] \\
        {} \& X_s \arrow[ul,"\gamma"] \& {}
    \end{tikzcd}
    }{\caption{Diagramme II.2.}}
    \end{floatrow}
\end{figure}

  Nous appellerons \emph{connexion d'arcs différentiables} ou tout simplement \emph{connexion} toute section (différentiable) du morphisme $\kappa$ qui fait commuter le diagramme II.2,
  où $\sigma_{X_s}$ est la section de $\pi_s$ déduite de $\sigma$.
  En clair une connexion $\gamma$ est une application différentiable de $X_s$ dans $\Arc(X)$ telle que :
  \begin{equation} \label{II.1.6}
    \forall (c,x) \in X_s, \quad \pi \circ \gamma(c,x) = c, \quad \gamma(c,x)(0) = x \quad \diamondsuit
  \end{equation}
  $$
    \forall m \in M \quad \forall x \in \pi^{-1}(m) \quad \gamma(m,x) = x \quad \heartsuit
    $$
  $\gamma$ sera encore appelé \emph{relèvement des arcs} (ou chemins) de la base,
  nous dirons que $\gamma(c,x)$ est le \emph{relevé horizontal} de $c$ au point $x$.
  
  On peut évidemment définir des connexions aux propriétés particulières,
  par exemple des connexions locales définies alors à partir du faisceau $\DL(\RR, X)$ et telles que le relevé en un point ne dépende que du germe du chemin en ce point.
  La définition des connexions peut s'interpréter comme un procédé catégorique d'inversion de flèches dans les catégories possédant des images réciproques pour les couples de flèches du type $\bullet \rightarrow \bullet \leftarrow \bullet$.
  
  Les connexions auxquelles on a affaire d'habitude sont des connexions d'ordre 1,
  le relevé d'un chemin en tout point ne dépendant que du jet d'ordre 1 du chemin au point image de la base.
  Il est clair que toutes les situations particulières sont envisageables :
  connexion non linéaire d'ordre 1,
  connexions d'ordre 2 etc... .
  
  % --- Connexion sur un fibré difféologique principal
  \subsection{Connexion sur un fibré difféologique principal.}
  
  Dans le cas particulier où le fibré de départ est un fibré principal de groupe structural $G$,
  on impose en plus au relèvement $\gamma$ de vérifier la propriété d'équivariance suivante :
  \begin{equation} \label{II.2.1}
    \forall g \in G \quad \forall (c,x) \in X_s \quad \gamma(c, g(x)) = g(\gamma(c,x))
  \end{equation}
  Où la notation $g(\gamma(c,x))$ signifie pour tout $t \in \RR : g(\gamma(c,x)(t)) = g(\gamma(c,x)(t))$,
  nous noterons d'ailleurs $g(c)$ l'action de $g \in G$ sur les courbes de $X$,
  elle consiste à translater tous les points de $c$ par $g$.
  Cette hypothèse supplémentaire se justifie lorsqu'on considère la catégorie des fibrés principaux.
  
  Comme nous venons de le voir le groupe $G$ agit naturellement sur $\Arc(X)$,
  il agit en fait par l'intermédiaire de $\Arc(G)$ dans lequel il s'induit $G \to \Arc(G)$ :
  \begin{equation} \label{II.2.2}
    \forall \xi \in \Arc(G) \quad \forall c \in \Arc(X) \quad \xi(c) = [t \to \xi(t)(c(t))]
  \end{equation}
  Le lecteur pourra vérifier que le quotient $\Arc(X)/\Arc(G)$ est égal (difféologiquement) à $\Arc(M)$ et que la projection $\pi_*$ est une fibration principale.
  
  L'existence d'une connexion principale de $\pi$ s'interprète comme une section du fibré quotient $\pi/G$ d'espace total $\Arc(X)/G$ et de base $\Arc(M)$.
  Cette section définit encore une application différentiable $\psi$ de $\Arc(X)$ dans $\Arc(G)/G$ vérifiant les propriétés suivantes :
  \begin{equation} \label{II.2.3}
    \forall \xi \in \Arc(G) \quad \forall c \in \Arc(X) \quad \psi(\xi(c)) = \xi(\psi(c)) \quad \heartsuit
  \end{equation}
  $$
    \forall x \in X \quad \psi(x) = 1_G \quad \diamondsuit
    $$
  Où $\xi(\psi(c))$ désigne l'action à gauche de $\xi \in \Arc(G)$ sur le quotient à droite $\Arc(G)/G$.
  Si on identifie $\Arc(G)/G$ avec $\Arc(G, 1_G)$ l'espace des arcs de $G$ d'origine l'identité,
  l'équation II.2.3 $\heartsuit$ s'écrit alors :
  \begin{equation} \label{II.2.4}
    \forall \xi \in \Arc(G) \quad \forall c \in \Arc(X) \quad \psi(\xi(c)) = \xi \cdot \psi(c) \cdot \xi(0)^{-1}
  \end{equation}
  Le fibré réduit est alors égal à $\psi^{-1}(1_G)$.
  Les éléments de ce sous-fibré seront appelés les \emph{arcs horizontaux} (relativement à $\psi$).
  
  Nous définirons donc l'espace des connexions de la fibration $\pi$ comme l'espace des applications différentiables $\psi$ vérifiant II.2.3,
  nous le noterons $C(\pi)$.
  Cet espace peut évidemment être muni de sa difféologie d'espace fonctionnel mais nous verrons plus loin qu'il peut posséder d'autres difféologies toutes aussi utiles.
  
  Supposons que l'ensemble $C(\pi)$ soit non-vide et soit $\psi_0 \in C(\pi)$,
  considérons alors l'application $T$ définie par :
  \begin{equation} \label{II.2.5}
    \forall \psi \in C(\pi) \quad T(\psi) = \psi_0^{-1} \cdot \psi
  \end{equation}
  $T(\psi)$ appartient alors au groupe $H(\pi)$ défini comme l'ensemble des applications différentiables $\phi$ de $\Arc(X)$ dans $\Arc(G, 1_G)$ vérifiant les propriétés suivantes :
  \begin{equation} \label{II.2.6}
    \forall \xi \in \Arc(G) \quad \forall c \in \Arc(X) \quad \phi(\xi(c)) = \xi(0) \cdot \phi(c) \cdot \xi(0)^{-1} \quad \heartsuit
  \end{equation}
  $$
    \forall x \in X \quad \phi(x) = 1_G \quad \diamondsuit
    $$
  Il est alors facile de vérifier que $T$ est une bijection.
  Si $C(\pi)$ a été muni d'une difféologie celle-ci se transporte naturellement sur $H(\pi)$ par image directe.
  
  % --- Quelques propriétés des fibrations difféologiques connexables
  \subsection{Quelques propriétés des fibrations difféologiques connexables.}
  
  L'adjectif \emph{connexable} s'appliquera évidemment aux fibrés dont l'espace des connexions est non-vide.
  Considérons maintenant deux fibrations (non principales) $\pi: X \to M$ et $p: Y \to N$.
  Notons $\Mor(\pi, p)$ l'espace des morphismes de $\pi$ vers $p$,
  c'est à dire l'espace des couples $(F, f)$ où $F \in D(X, Y)$ et $f \in D(M, N)$ tel que $p \circ F = f \circ \pi$.
  $\Mor(\pi, p)$ sera muni de la difféologie de partie du produit $D(X, Y) \times D(M, N)$.
  Il est immédiat de constater que l'application $\Pr: \Mor(\pi, p) \to D(M, N)$ définie par $\Pr(F, f) = f$ est différentiable.
  $\Pr$ n'est pas à priori surjective.
  En étendant la définition,
  que nous avons donné,
  des connexions aux applications qui ne sont pas nécessairement des fibrations et qui ne sont pas nécessairement surjectives,
  nous allons associer à toute connexion $\gamma$ de $p$ une connexion naturelle de $\Pr$.
  
  Soit $\phi$ un arc (différentiable) de $D(M, N)$ d'origine $f$,
  et soit $F$ un point de $D(X, Y)$ se projetant sur $f$.
  Nous pouvons associer à tout $x \in X$ un arc $\phi_x$ défini par :
  \begin{equation} \label{II.3.1}
    \phi_x = [t \to \phi(t)(\pi(x))]
  \end{equation}
  Cet arc est évidemment différentiable.
  Il a pour origine $p \circ F(x)$,
  nous pouvons associer alors à tout $t \in \RR$ et à tout $x \in X$ la valeur au point $t$ du relevé par $\gamma$ de l'arc $\phi_x$ d'origine $F(x)$.
  Ce qui nous permet de construire l'arc $\Phi$ par :
  \begin{equation} \label{II.3.2}
    \forall t \in \RR \quad \Phi(t) = [x \to \gamma([s \to \phi(s)(\pi(x))], F(x))(t)]
  \end{equation}
  On peut alors vérifier que l'application $\Gamma: (\phi, F) \mapsto \Phi$ que nous venons de construire est une connexion de la projection $\Pr$.
  
  On vérifie que cette construction s'adapte au cas des fibrés difféologiques principaux.
  On suppose maintenant que $p = \pi$ et que $\pi$ est un fibré difféologique principal de groupe structural $G$.
  On peut montrer dans ce cas que si on se restreint aux automorphismes de fibrés ($F$ est alors un difféomorphisme équivariant de $X$ et $f$ un difféomorphisme de la base) l'arc $\Phi$ que nous avons construit par la formule II.4.2 est entièrement contenu dans $\text{Auto}(\pi)$.
  C'est à dire $\Gamma$ est en fait une connexion de la projection $\Pr^*: \text{Auto}(\pi) \to \Diff(M)$.
  En remarquant alors que $1_X$ se projette sur $1_M$ on déduit alors que la composante connexe $\Diff(M)^0$ de l'identité de $\Diff(M)$ est entièrement couverte par la projection $\Pr^*$.
  En notant alors $\text{Auto}(\pi)^* = \Pr^{*-1}(\Diff(M)^0)$ et $J$ le groupe des transformations de jauge du fibré,
  on a la suite exacte d'homomorphismes :
  \begin{equation} \label{II.3.3}
    1_X \to J \to \text{Auto}(\pi)^* \to \Diff(M)^0 \to 1_M
  \end{equation}
  En appliquant cette construction à l'image réciproque d'un fibré $\pi: X \to M$ par une application différentiable $f \in D(N, M)$,
  on peut montrer que si $A \subset M$ est une rétraction stricte de $M$ alors $\pi^{-1}(A) \subset X$ est une rétraction stricte de $X$.
  De façon générale on montre que si $[t \to f_t]$ est une homotopie différentiable entre $f_0$ et $f_1$ les images réciproques de $\pi$ par $f_0$ et $f_1$ sont isomorphes,
  il s'ensuit que tout fibré difféologique connexable dont la base est contractile est trivial.
  Ce théorème est classique dans la théorie des variétés,
  l'hypothèse connexable n'étant pas utile puisque dans le cas des variétés tout fibré est connexable.
  
  % --- Structure fibrée de l'espace des connexions
  \section{Structure fibrée de l'espace des connexions}
  
  Nous allons maintenant étudier la question posée par l'ambiguïté de Gribov.
  L'invariance de certaines théories de jauge par les transformations de jauge amène à penser que dans certains cas qu'il appelle une jauge globale pour effectuer ses calculs.
  Or Gribov a montré que pour certains cas le choix généralement effectué conduisait à une ambiguïté de la définition de la jauge.
  La question se trouvait posée :
  existe-t-il de telles jauges ?
  Pour pouvoir répondre à une telle question il fallait d'abord définir le cadre mathématique dans lequel elle se pose.
  Des tentatives ont été faites dans le cadre de la topologie $C^\infty$ [5] [6].
  Les choix sont toujours délicats lorsqu'il s'agit d'espaces de dimension infinie,
  nous proposons d'utiliser la théorie des fibrés difféologiques.
  Il apparaitra dans le traitement du sujet que nous devons faire nous aussi un choix de difféologie,
  plus adapté au problème que nous traitons que la simple difféologie fonctionnelle usuelle.
  
  Nous supposerons donc par la suite que $X$ et $M$ sont des variétés,
  que $\pi$ est un fibré principal de groupe structural de Lie $G$ et que $M$ est non compacte.
  Nous noterons $B(\pi)$ le quotient $C(\pi)/J$.
  
  % --- Difféologie compacte et groupe des transformations de jauge à support compact
  \subsection{Difféologie compacte et groupe des transformations de jauge à support compact.}
  
  Considérons une transformation de jauge $A$, $A \in \Diff(X)$ et vérifie pour tout $g \in G$ :
  $A(g(x)) = g(A(x))$ où $x \in X$.
  On peut donc définir une application $\alpha$ de $X$ dans $G$ telle que $A(x) = \alpha(x)(x)$,
  $\alpha$ vérifie alors la propriété d'équivariance suivante
  \begin{equation} \label{III.1.1}
    \forall g \in G \quad \forall x \in X \quad \alpha(g(x)) = g \cdot \alpha(x) \cdot g^{-1}
  \end{equation}
  Réciproquement tout application $\alpha$ vérifiant III.1.1 définit une transformation de jauge de $\pi$.
  Nous identifierons donc $J$ avec les applications de ce type.
  
  Afin d'éviter les problèmes du comportement à l'infini nous nous restreindrons aux groupe $J_K$ des transformations de jauge à support compact,
  c'est à dire :
  $\alpha$ est constante et vaut $1_G$ en dehors de l'image réciproque $\pi^{-1}(S)$ d'un compact $S$ de la base $M$,
  appelé support de $\alpha$ et noté $\text{Supp}(\alpha)$ (en fait le plus petit compact en dehors duquel $\alpha$ est l'élément neutre).
  
  L'étude de ce problème montre que nous sommes forcé de faire un choix particulier quant à la difféologie de l'espace des connexion $C(\pi)_K$ ainsi qu'à celle de $J_K$.
  Cette difféologie que nous avons appelé \emph{difféologie compacte controlée} est obtenue comme limite inductive de difféologies indexées par des compacts,
  procédé que nous allons rapidement décrire.
  
  Soit $I$ une famille d'indices filtrants croissants et soit $X_i$, $i \in I$, une famille d'espaces difféologiques.
  Soit $\{f_j^i\}_{j<i}$ une famille d'applications différentiables de $X_j$ dans $X_i$, vérifiant les propriétés :
  \begin{equation} \label{III.1.2}
    \forall i \in I \quad f_i^i = 1_{X_i} \quad \heartsuit
  \end{equation}
  $$
    \forall k < j < i \quad f_j^i \circ f_k^j = f_k^i \quad \diamondsuit
    $$
  Définissons sur la somme difféologique des $X_i$ la relation d'équivalence suivante :
  \begin{equation} \label{III.1.3}
    (j,x) \equiv (i,y) \iff \exists k \in I \ j < k \ i < k \ f_j^k(y) = f_i^k(x)
  \end{equation}
  Le quotient difféologique de la somme des $X_i$ par cette relation d'équivalence est la \emph{limite inductive} de la famille $X_i$,
  cet espace est généralement noté $\varinjlim X_i$.
  
  Considérons alors la famille d'indices définie par les compacts de $M$ pour la relation d'ordre naturelle de l'inclusion et posons $J_S$ le groupe des transformations de jauge à support compact $S$.
  Munissons $J_S$ de sa difféologie fonctionnelle,
  il est facile de vérifier,
  qu'en tant qu'ensemble,
  la limite inductive des $J_S$ est égale au groupe des transformations de jauge à support compact.
  La difféologie dont est alors munie sera appelée \emph{difféologie compacte controlée}.
  Muni de cette difféologie ce groupe sera noté $J_K^\infty$.
  Ses plaques vérifient la propriété suivante :
  \begin{multline} \label{III.1.4}
    \forall \phi \in \DL(\ReDomains,\ J_K^\infty),\ \forall r_0 \in \dom(\phi),\ \exists \text{Vois}(r_0), \\
    \exists S \text{ compact tel que : } \forall r \in V \quad \text{Supp}(\phi(r)) \subset S
\end{multline}
  Où $\text{Vois}(r_0)$ désigne l'ensemble des voisinages de $r_0$.
  En d'autres termes le support de la famille de transformations de jauge $\phi(r)$ est borné au voisinage de tout $r$.
  
  % --- Difféologie compacte et espace des connexions
  \subsection{Difféologie compacte et espace des connexions}
  
  La difféologie compacte controlée se définit aussi sur l'espace des connexions $C(\pi)$,
  il faut tout d'abord feuilleter cet espace par une relation d'équivalence dont les classes sont définies par le comportement à l'infini des connexions.
  Définissons donc,
  pour tout compact $S$ de $M$ la relation d'équivalence $\approx_S$ :
  
  Deux connexions $\psi$ et $\psi'$ seront $\approx_S$-équivalentes si les relevés par $\psi$ et $\psi'$ de tout arc de $M$ entièrement situés à l'extérieur de $S$ coïncident.
  Nous dirons alors que $\psi$ et $\psi'$ coïncident à l'extérieur de $S$.
  
  Etant données deux connexions $\approx_S$-équivalentes $\psi$ et $\psi'$,
  le produit $\psi^{-1} \cdot \psi'$,
  élément du groupe $H(\pi)$ défini au paragraphe II.2,
  est à support compact.
  C'est à dire $\psi^{-1} \cdot \psi'$ est constant et égal à l'identité pour tout arc $c$ de $X$ dont la projection par $\pi$ est entièrement à l'extérieur de $S$.
  
  Notons $C(\pi)_S$ l'espace $C(\pi)$ muni de la difféologie feuilletée par $\approx_S$.
  La limite inductive des espaces $C(\pi)_S$ sera noté $C(\pi)_K^\infty$.
  En tant qu'ensemble cet espace est égal à $C(\pi)$,
  il est par contre partagé en classes caractérisées par le comportement à l'infini des connexions,
  ces classes sont celles de la relation d'équivalence $\approx$ définie par :
  
  Deux connexions sont $\approx$-équivalente si elles coïncident à l'extérieur d'un compact.
  
  Les plaques de $C(\pi)_K^\infty$ vérifient la propriété suivante :
  \begin{multline} \label{III.2.1}
    \forall \phi \in \DL(\ReDomains,\ C(\pi)_K^\infty),\ \forall r_0 \in \dom(\phi) : \exists S \text{ compact de } M,\\
     \exists V \in \text{Vois}(r_0) \text{ tel que:}
    \forall r \in V \quad \phi(r_0) \approx_S \phi(r)
  \end{multline}
  Si on choisit,
  pour chaque classe d'équivalence $\sigma$ de la relation $\approx$,
  un représentant $\psi_\sigma$,
  les applications $T_\sigma$ définies sur les classes $\sigma$ par :
  $T_\sigma(\psi) = (\psi_\sigma)^{-1} \cdot \psi$,
  sont des bijection sur le groupe $H(\pi)_K$ des éléments de $H(\pi)$ à support compact.
  Le lecteur pourra vérifier que les images directes de la difféologie de chaque classe $\sigma$ par $T_\sigma$ coïncident avec la difféologie compacte controlée de $H(\pi)_K$,
  définie de façon analogue à celle de $J_K^\infty$.
  En somme l'espace des connexions $C(\pi)_K^\infty$ est partagé en classes qui sont des réunions de composantes connexes,
  chacune difféomorphe au groupe $H(\pi)_K^\infty$ (c'est à dire $H(\pi)_K$ muni de la difféologie compacte controlée).
  Chaque classe représentant un comportement à l'infini de connexion (le germe à l'infini).
  
  Avec ces difféologies,
  le lecteur vérifiera que l'action du groupe des transformations de jauges à support compact sur l'espace des connexions est différentiable :
  l'application de $C(\pi)_K^\infty \times J(\pi)_K^\infty$ dans $C(\pi)_K^\infty$ définie par
  \begin{multline} \label{III.2.2}
    \forall (\psi, \alpha) \in C(\pi)_K^\infty \times J(\pi)_K^\infty \quad \alpha(\psi) = [c \to \alpha(c) \cdot \psi(c) \cdot \alpha(c(0))^{-1}] \\
    \alpha(c) \cdot \psi(c) \cdot \alpha(c(0))^{-1}(t) = \alpha(c(t)) \cdot \psi(c(t)) \cdot \alpha(c(0))^{-1}
  \end{multline}
  est différentiable.
  De plus l'action de $J(\pi)_K^\infty$ se fait à l'intérieur de chacune des classes de la relation d'équivalence $\approx$,
  c'est à dire $\alpha(\psi) \approx \psi$.
  Donc le quotient $C(\pi)_K^\infty / J(\pi)_K^\infty$ est égal à la somme (difféologique) des quotients de chaque classe de $\approx$ par $J(\pi)_K^\infty$.
  
  % --- Fibration de l'espace des connexions
  \subsection{Fibration de l'espace des connexions}
  
  Il ne nous reste plus qu'à démontrer que la projection :
  \begin{equation} \label{III.3.3}
    \Pi : C(\pi)_K^\infty \to B(\pi)_K^\infty = C(\pi)_K^\infty / J(\pi)_K^\infty
  \end{equation}
  est une fibration difféologique.
  Considérons alors l'application $F$ définie sur $C(\pi)_K^\infty \times J(\pi)_K^\infty$ dans $C(\pi)_K^\infty \times C(\pi)_K^\infty$ par :
  \begin{equation} \label{III.3.4}
    F(\psi, \alpha) = (\alpha(\psi), \psi)
  \end{equation}
  Nous allons montrer que $F$ est une induction,
  ce qui prouvera que $\Pi$ est une fibration principale (I.2.5).
  Commençons par l'injectivité :
  
  Supposons $F(\psi, \alpha) = F(\psi', \alpha')$ ceci implique $\psi = \psi'$ et $\alpha(\psi) = \alpha'(\psi)$,
  c'est à dire :
  \begin{multline} \label{III.3.5}
    \forall c \in \Arc(X) \quad \forall t \in \RR : \alpha(c(t)) \cdot \psi(c)(t) \cdot \alpha(c(0))^{-1}
    \\ = \alpha'(c(t)) \cdot \psi(c)(t) \cdot \alpha'(c(0))^{-1}
  \end{multline}
  Puisque $\alpha$ et $\alpha'$ sont à support compact choisissons un point $m_0 \in M$ à l'extérieur de la réunion des supports,
  puisque $M$ est connexe choisissons pour tout $x \in X$ un arc $c$ dans $M$ d'origine $m_0$ et d'extrémité $\pi(x)$ et soit $c$ son relevé par la connexion $\psi$,
  en appliquant III.3.5 on obtient $\alpha(x) = \alpha'(x)$,
  ce qui prouve l'injectivité de $F$.
  
  Montrons maintenant que $F$ est une induction :
  
  Soit $r \to (\psi_r, \alpha_r)$ une application définie sur un ouvert $\Omega$ de $\ReDomains$ à valeurs dans $C(\pi)_K^\infty \times J(\pi)_K^\infty$ telle que la composition par $F$ soit différentiable.
  On déduit alors évidemment que $[r \to \psi_r]$ est différentiable et que $[r \to \alpha_r(\psi_r)]$ est différentiable.
  Considérons alors un point $r_0 \in \dom([r \to \alpha_r])$,
  on a par raison de différentiabilité :
  \begin{multline} \label{III.3.6}
    \exists V \in \text{Vois}(r_0) \text{ et } S \text{ compact de } M \text{ tel que :} \\
    \forall r \in V \quad \forall c \in \Arc(X) : \im(\pi \circ c) \subset M-S \implies \psi_r(c) = \psi_{r_0}(c)
  \end{multline}
    
  \begin{multline} \label{III.3.7}
    \exists V' \in \text{Vois}(r_0) \text{ et } S' \text{ compact de } M \text{ tel que :} \\
    \forall r \in V' \quad \forall c \in \Arc(X) : \text{codom}(\pi \circ c) \subset M-S \\
    \implies \alpha_r(c) \cdot \psi_r(c) \cdot \alpha_r(c(0))^{-1} = \alpha_{r_0}(c) \cdot \psi_{r_0}(c) \cdot \alpha_{r_0}(c(0))^{-1}
    \end{multline}
  Posons alors $V'' = V \cap V'$ et $S'' = S \cap S'$,
  III.3.6 et III.3.7 peuvent s'écrire en remplaçant $V$ et $V'$ par $V''$.
  Ceci implique alors en posant $\beta_r(x) = \alpha_{r_0}(x)^{-1} \cdot \alpha_r(x)$ et en conservant les hypothèses des formules précédentes :
  \begin{equation} \label{III.3.8}
    \psi_r(c) = \psi_{r_0}(c) \text{ et } \beta_r(c) \cdot \psi_{r_0}(c) \cdot \beta_r(c(0))^{-1} = \psi_{r_0}(c)
  \end{equation}
  Mais $\beta_r$ est une transformation de jauge à support compact,
  son support est contenu dans $\text{Supp}(\alpha_{r_0}) \cup \text{Supp}(\alpha_r)$.
  Soit $r \in V''$ et soit $m_0 \in M - (\text{Supp}(\beta_r) \cup S'')$,
  Puisque $M$ est connexe et $S''$ compact il existe un arc $c$ dans $M$ joignant $m_0$ à $m$ et entièrement contenu dans $M-S''$.
  Notons alors $c$ un relevé par $\psi_{r_0}$ ($\psi_{r_0}(c) = 1_G$),
  grâce à ce que nous venons de voir on a $\beta_r(c(1)) = 1_G$,
  on déduit donc que $\forall x \in \pi^{-1}(M-S'') \ \beta_r(x) = 1_G$.
  Ce qui permet donc d'écrire :
  \begin{equation} \label{III.3.9}
    \forall r_0 \in \dom([r \to \alpha_r]) \exists V \in \text{Vois}(r_0) \text{ et } S \text{ compact de } M \text{ tel que }
  \end{equation}
  $$
    \forall r \in V \quad \text{Supp}(\alpha_r) \subset S
    $$
  Nous venons donc de démontrer une partie de la différentiabilité de l'application $[r \to \alpha_r]$,
  il s'agit maintenant de montrer qu'elle est différentiable pour la difféologie fonctionnelle.
  Pour ceci nous nous plaçons dans un voisinage $V$ de $r_0$ pour lequel on a $\forall r \in V \ \text{Supp}(\alpha_r) \subset S$,
  $S$ compact de $M$.
  Soit $m_0$ un point de $S$,
  le lecteur pourra vérifier la proposition suivante :
  
  $\forall [r \to x_r]$ définie sur $V$ et à valeurs dans $X$,
  il existe une application différentiable $[r \to c_r]$ définie sur $V$ et à valeurs dans $\Arc(X)$ telle que :
  $\psi_r(c_r) = 1_G$, $c_r(0) \in X_{m_0}$ et $c_r(1) = x_r$.
  
  D'où on déduit que $[r \to \alpha_r]$ est localement différentiable donc différentiable.
  En conclusion nous venons de montrer que l'application $F$ définie en III.3.4 est une induction et donc :
  
  \subsubsection*{\emph{III.3.8} Théorème}
  L'application $\Pi: C(\pi)_K^\infty \to B(\pi)_K^\infty = C(\pi)_K^\infty / J(\pi)_K^\infty$ est une fibration difféologique principale de groupe structural $J(\pi)_K^\infty$.
  
  % --- Conséquences homotopiques
  \subsection{Conséquences homotopiques}
  
  Nous avons vu au paragraphe III.2 que l'espace des connexions $C(\pi)_K^\infty$ est une somme de composantes chacune difféomorphe au groupe $H(\pi)_K^\infty$ (définition II.2.6).
  Chaque composante étant caractérisée par le germe à l'infini d'un de ses éléments.
  L'arc défini par
  \begin{equation} \label{III.4.1}
    \forall \eta \in H(\pi)_K^\infty \quad [s \to \eta_s] \text{ avec } \eta_s(c)(t) = \eta(c)(st) \quad \forall t \in \RR
  \end{equation}
  est une rétraction du groupe sur l'identité.
  Le lecteur pourra vérifier que,
  grâce à la compacité du segment $[0,1]$ cette rétraction est différentiable au sens de la difféologie compacte controlée.
  Donc chaque classe en question est une composante connexe de $C(\pi)_K^\infty$ chaque composante étant contractile.
  Nous avons vu d'autre part que le quotient de $C(\pi)_K^\infty$ par $J(\pi)_K^\infty$ est égal à la somme difféologique du quotient de chacune de ces composantes par $J(\pi)_K^\infty$.
  Soit $\Sigma$ une composante de $C(\pi)_K^\infty$,
  si la fibration était triviale nous devrions avoir $\pi_n(J(\pi)_K^\infty) \oplus \pi_n(B(\pi)_K^\infty) = 0$ or l'application de la suite exacte d'homotopie indique que $\pi_{n+1}(B(\pi)_K^\infty) \simeq \pi_n(J(\pi)_K^\infty)$ donc on déduit la proposition suivante :
  
  % ---
  \subsubsection*{\emph{III.4.2.} Proposition}
  Pour que le fibré $\Pi: C(\pi)_K^\infty \to B(\pi)_K^\infty = C(\pi)_K^\infty / J(\pi)_K^\infty$ soit non trivial il suffit qu'un des groupes d'homotopie du groupe $J(\pi)_K^\infty$ soit non nul.
  
  Considérons alors l'exemple particulier de $M = \RR^4$,
  le groupe $J(\pi)_K^\infty$ s'identifie avec le groupe de courant $D(\RR^4, \infty; G, 1_G)_K^\infty$.
  On peut montrer que cet espace est homotopiquement équivalent à l'espace $D(S^4, N; G, 1_G)$.
  On en déduit $\pi_n(J(\pi)_K^\infty) = \pi_{n+4}(G)$.
  On voit alors que pour $G = SU(2)$ le fibré ne peut pas être trivial,
  puisque $\pi_4(SU(2)) = \ZZ_2$.
  
  % --- Annexe
  \section{Annexe}
  
  \numberwithin{equation}{section}
  \renewcommand{\thesection}{A}
  
  Nous présentons ici une série de remarques sur la définition des difféologies,
  inspirée du vocabulaire de la théorie des faisceaux.
  Bien que ce ne soit pas absolument nécessaire,
  nous nous sommes placé à priori dans la catégorie des espaces topologiques,
  simplement pour faciliter l'exposé.
  
  Définissons d'abord le vocabulaire utilisé :
  
  Nous appellerons \emph{application locale} continue de $X$ à $Y$ (ou simplement application locale) toute application continue définie sur un ouvert d'un espace topologique $X$ à valeur dans un espace topologique $Y$.
  Nous noterons $\dom(f)$ le domaine de définition de l'application locale $f$ et $\im(f)$ son domaine de valeurs.
  Un ensemble d'applications locales sera appelé : \emph{une famille d'applications locales}.
  Nous dirons que la famille d'applications $F$ de $X$ vers $Y$ est \emph{couvrante} si :
  \begin{equation} \label{A.1}
    \bigcup_{f \in F} \im(f) = F
  \end{equation}
  Etant donnés trois ensembles $X$, $Y$ et $Z$,
  deux applications :
  $f$ de $X$ vers $Y$ et $g$ de $Y$ vers $Z$,
  le domaine de définition de $g \circ f$ est défini par :
  \begin{equation} \label{A.2}
    \dom(g \circ f) = f^{-1}(\dom(g))
  \end{equation}
  Si $f^{-1}(\dom(g)) = \varnothing$ nous écrirons $g \circ f = 1_\varnothing$ où $1_\varnothing$ est l'application vide.
  
  Etant données deux familles $F$ et $G$ d'applications locales,
  respectivement de $X$ vers $Y$ et de $Y$ vers $Z$,
  nous noterons $G \circ F$ la famille d'applications locales :
  \begin{equation} \label{A.3}
    G \circ F = \{ g \circ f \mid g \in G \text{ et } f \in F \}
  \end{equation}
  D'autre part nous définirons la \emph{famille unité} associé à tout espace topologique $X$ par :
  \begin{equation} \label{A.4}
    \mathfrak{I}_X = \{ 1_U \mid U \in \text{Ouv}(X) \}
  \end{equation}
  Où $\text{Ouv}(X)$ désigne l'ensemble des ouverts de la topologie de $X$.
  Nous rappelons que les \emph{pré-faisceaux} et \emph{faisceaux} d'applications locales sont des familles $F$ d'applications locales vérifiant les propriétés suivantes ($\spadesuit$ pour les pré-faisceaux, $\spadesuit$ et $\heartsuit$ pour les faisceaux) :
  \begin{equation} \label{A.5}
    \forall f \in F \quad \forall U \in \text{Top}(A) \quad f|_U \in F \quad \spadesuit
  \end{equation}
  $$
    \forall f \quad \forall r \in A : \exists V \in \text{Vois}(r) \text{ tel que } f|_V \in F \implies f \in F \quad \heartsuit
    $$
  Il existe sur les familles d'applications locales définies dans un même espace topologique une relation d'ordre que nous notons $\le$,
  définie naturellement par :
  \begin{equation} \label{A.6}
    F \le F' \iff F \subset F'
  \end{equation}
  Nous dirons alors que $F$ est \emph{plus fine} que $F'$.
  Il est facile de vérifier que toute famille d'applications locales $F$ définie dans un espace topologique $X$ est contenue dans un pré-faisceau le plus fin $F^\circ$.
  Nous dirons que c'est le \emph{pré-faisceau engendré} par $F$.
  Il est clair que si $F$ est déjà un pré-faisceau $F^\circ = F$.
  $F^\circ$ est défini par :
  \begin{equation} \label{A.7}
    F^\circ = F \circ \mathfrak{I}_X
  \end{equation}
  De même tout pré-faisceau $F$ engendre un faisceau (le faisceau le plus fin contenant $F$),
  que nous noterons $*(F)$,
  obtenu par un procédé que nous pourrions appeler :
  procédé de faisceautisation et qui consiste à construire toutes les bornes supérieures des familles compatibles d'éléments de $F$.
  Rappelons à cette occasion qu'une \emph{famille compatible} d'éléments de $F$ est un ensemble d'applications $\{f_i \mid f_i \in F, i \in I\}$,
  tel que deux éléments quelconques de cette famille coïncident sur l'intersection de leur domaine de définition.
  Le Sup désigne le plus petit prolongement commun.
  \begin{equation} \label{A.8}
    *(F) = \{ \text{Sup}(f_i) \mid \{f_i\} \text{ famille compatible} \}
  \end{equation}
  Soient deux faisceaux d'applications locales $F: X \to Y$ et $G: Y \to Z$,
  le produit $G \circ F$ n'est pas à priori un faisceau,
  il en engendre un grâce à la construction précédente,
  nous noterons $G * F$ le faisceau ainsi obtenu.
  \begin{equation} \label{A.9}
    G * F = *(G \circ F)
  \end{equation}
  Lorsque la famille $G$ est réduite à un seul élément $g$ nous noterons $g \circ F$ et $g * F$ à la place,
  respectivement de $G \circ F$ et $G * F$.
  Compte tenu de ces définitions nous avons la propriété suivante,
  dont nous laisserons la démonstration aux soins du lecteur :
  
  \subsubsection*{\emph{A.10} Proposition :} \setcounter{equation}{10}
  L'opération $*$ est associative.
  Soient $X$ et $Y$ deux espaces topologiques et $F$ un faisceau d'applications locales définie de $X$ vers $Y$,
  soit $\mathfrak{I}_X$ et $\mathfrak{I}_Y$ les faisceaux unités (A.4) associé respectivement à $X$ et $Y$.
  Nous avons :
  \begin{equation} \label{A.11}
    F * \mathfrak{I}_X = \mathfrak{I}_Y * F = F
  \end{equation}
  Grâce aux définitions précédentes et à la proposition A.10,
  nous sommes en mesure de définir la catégorie (plutôt "métacatégorie" au sens de S.MacLane [4]) des \emph{faisceaux d'applications locales} que nous noterons $\mathfrak{I}$,
  dont les objets sont les espaces topologiques et les flèches (ou morphismes) sont les faisceaux d'application locales pour la loi de composition $*$ définie en A.8.
  
  Le lecteur vérifiera que l'opération $*$ définie en A.8 définit un foncteur de la catégorie Top des espaces topologiques dans la catégorie $\mathfrak{I}$.
  A chaque espace topologique $X$,
  objet de Top,
  est associé $X$ objet de $\mathfrak{I}$ et à chaque application continue $f: X \to Y$ est associée le faisceau $*(f) = f \circ \mathfrak{I}_X$.
  
  Nous notons $\ReDomains$ la somme topologique des puissances finies de $\RR$ :
  Nous noterons $\DL(\ReDomains)$ le faisceau des applications $C^\infty$ défini sur $\ReDomains$ et à valeurs dans $\ReDomains$.
  Considérons alors un espace topologique $X$ et un faisceau $D$ d'application locales définies de $\ReDomains$ dans $X$.
  Nous dirons que $D$ est une \emph{difféologie} de $X$ si les conditions (A.13) sont vérifiées.
  Muni d'une difféologie $D$ un espace $X$ sera appelé espace difféologique.
  \begin{equation} \label{A.13}
    D \text{ est couvrante (A.1)} \spadesuit
  \end{equation}
  $$
    D * \DL(\ReDomains) = D \diamondsuit
    $$
  C'est ici que cette présentation des espaces difféologiques diffère de la définition standard dans la mesure où il n'est pas nécessaire de supposer $X$ espace topologique (Voir[7]).
  Il existe de toute façon sur $X$ une topologie la plus fine rendant tous les éléments de $D$ continus,
  elle est appelée la D-Topologie.
  Pour rester en accord avec les notations usuelles $D$ sera aussi noté $\DL(\ReDomains, X)$.
  Les éléments de $\DL(\ReDomains, X)$ sont appelés les \emph{plaques de la difféologie} de $X$ ou plus simplement les plaques de $X$.
  
  La catégorie des espaces difféologiques peut être décrite en adaptant à nos besoins la construction de la théorie générale des "Comma Catégorie" (Catégories virgules) (voir [4]) :
  
  Soit $*: \text{Top} \to \mathfrak{I}$ le foncteur en question,
  $\ReDomains$ étant un objet de $\mathfrak{I}$,
  nous noterons $(\ReDomains \le *)$ la catégorie qui a pour objets les couples $<\DL(\ReDomains, X), X>$ où $\DL(\ReDomains, X)$ est une difféologie de $X$ et pour flèches de $\DL(\ReDomains, X)$ dans $\DL(\ReDomains, Y)$ les applications $h: X \to Y$ telles que $*(h) * \DL(\ReDomains, X) \le \DL(\ReDomains, Y)$.
  Nous dirons que $h$ est \emph{différentiable}.
  La catégorie standard $(\ReDomains \downarrow *)$ des "objets *-sous $\ReDomains$" (voir [4]) étant définie en remplaçant dans ce qui précède :
  $*(h) * \DL(\ReDomains, X) \le \DL(\ReDomains, Y)$ par $*(h) * \DL(\ReDomains, X) = \DL(\ReDomains, Y)$.
  
  En clair : une application $h$ d'un espace difféologique $X$ dans un espace difféologique $Y$ est différentiable, si :
  \begin{equation} \label{A.14}
    f * \DL(\ReDomains, X) \le \DL(\ReDomains, Y)
  \end{equation}
  La catégorie $(\ReDomains \downarrow *)$ s'interprète en fait comme la catégorie des subductions (voir [7]).
  Etant donné un espace difféologique $X$ et une application continue $h$ de $X$ dans un espace topologique $Y$,
  il est facile de vérifier que $h * \DL(\ReDomains, X)$ est une difféologie,
  c'est la difféologie image de $X$ par $h$.
  Si $Y$ possède une difféologie $\DL(\ReDomains, Y)$ et si elle coïncide avec $h * \DL(\ReDomains, X)$ on dit que $h$ est une subduction,
  ceci revient à dire que $h$ est une flèche de la catégorie $(\ReDomains \downarrow *)$.
  
  Les notions d'images réciproques et d'inductions sont les notions duales de celles-ci.
  Pour le reste des notions de bases des difféologies nous renvoyons le lecteur à [7] [1] [3].
  
  \section{BIBLIOGRAPHIE}
  
  \begin{enumerate}
    \item[1] P.Donato -Thèse doctorat, Université de Provence, Octobre 1984
    \item[2] P.Donato P.Iglesias- Exemple de groupes difféologiques: Flots irrationnels sur le tore. C.R. Acad. Sc. Paris, t. 301, Série I, N° 4, 1985.
    \item[3] P.Iglesias
    \begin{itemize}
      \item[i)] Classification des fibrés différentiels principaux de groupe structural $(\RR,+)$ sur les tores de Denjoy-Poincarré, Preprint, CPT-84/P.1690 (1984)
      \item[ii)] Fibrés difféologiques et Homotopie. Thèse doctorat, Université de Provence, Nov 1985.
    \end{itemize}
    \item[4] S.M.Lane -Catégories for the working mathemetician, Graduate texts in Math. 5,Springer Verlag (1971)
    \item[5] M.Daniel C.M.Viallet-Geometrical fetting of gauge theory of Yang-Mills type. Preprint LPTHE 7903
    \item[6] I.MSinger-Some remarks on the Gribov Ambiguity, Comm. in Math. Phys., 60, pp 7-12 (1978)
    \item[7] J.M.Souriau
    \begin{itemize}
      \item[i)] Groupes différentiels, Lect. notes in Math., Spinger Verlag, 836 (1981)
      \item[ii)] Un algorithme générateur de structures quantiques, Preprint, CPT-84/PE.1694 (1984)
      \item[iii)] Groupes différentiels et physique mathématique in "Feuilletage et quantification géométrique", pp75-79, Colection Travaux en cours, Hermann, Paris (1984)
    \end{itemize}
  \end{enumerate}
  
\end{document}
