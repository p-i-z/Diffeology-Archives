\documentclass[10pt,reqno,twoside]{amsart}

%%%%%%%%%%%%%%%%%%%%%%%%%%%%%%%%%%%%%%%%%%%%%%%%%%%%%%%%%%
%% MARK: Packages
%%%%%%%%%%%%%%%%%%%%%%%%%%%%%%%%%%%%%%%%%%%%%%%%%%%%%%%%%%

\usepackage{amsmath}
\usepackage{amssymb}
\usepackage{amsthm}
\usepackage[utf8]{inputenc}
\usepackage[T1]{fontenc}
\usepackage{lmodern} % Modern font
\usepackage[hidelinks]{hyperref}
\usepackage{graphicx}
\usepackage{tikz-cd} % For diagrams

% TikZ settings for diagrams
\usetikzlibrary{calc, positioning, arrows.meta}
\tikzcdset{
arrow style=tikz,
diagrams={>={Straight Barb[scale=0.8]}}
}

% Display style

\parindent 0mm
\parskip .5ex plus 2pt

\usepackage[cal=scr,sfscaled=false,frenchmath,uppercase = upright,greeklowercase = upright,utopia]{mathdesign}
\linespread{1.1}

\allowdisplaybreaks

\def \epsilon{\varepsilon}

%%%%%%%%%%%%%%%%%%%%%%%%%%%%%%%%%%%%%%%%%%%%%%%%%%%%%%%%%%
%% MARK: Macros & Shortcuts
%%%%%%%%%%%%%%%%%%%%%%%%%%%%%%%%%%%%%%%%%%%%%%%%%%%%%%%%%%

% --- Text Shortcuts ---
\newcommand{\qmbox}[1]{\quad\mbox{#1}\quad}
\newcommand{\ie}{\textit{i.e.}}
\newcommand{\cf}{\textit{cf.}}

\newcommand{\mymatrix}[1]{\begin{pmatrix} #1 \end{pmatrix}}

% --- Standard Sets ---
\newcommand{\RR}{\mathbf{R}}
\newcommand{\ZZ}{\mathbf{Z}}
\newcommand{\CC}{\mathbf{C}}
\newcommand{\NN}{\mathbf{N}}
\newcommand{\QQ}{\mathbf{Q}}

% --- Calligraphic Letters (formerly \euls) ---
\newcommand{\cA}{\mathcal{A}}
\newcommand{\cB}{\mathcal{B}}
\newcommand{\cC}{\mathcal{C}}
\newcommand{\cD}{\mathcal{D}}
\newcommand{\cE}{\mathcal{E}}
\newcommand{\cF}{\mathcal{F}}
\newcommand{\cG}{\mathcal{G}}
\newcommand{\cH}{\mathcal{H}}
\newcommand{\cI}{\mathcal{I}}
\newcommand{\cJ}{\mathcal{J}}
\newcommand{\cK}{\mathcal{K}}
\newcommand{\cL}{\mathcal{L}}
\newcommand{\cM}{\mathcal{M}}
\newcommand{\cN}{\mathcal{N}}
\newcommand{\cO}{\mathcal{O}}
\newcommand{\cP}{\mathcal{P}}
\newcommand{\cQ}{\mathcal{Q}}
\newcommand{\cR}{\mathcal{R}}
\newcommand{\cS}{\mathcal{S}}
\newcommand{\cT}{\mathcal{T}}
\newcommand{\cU}{\mathcal{U}}
\newcommand{\cV}{\mathcal{V}}
\newcommand{\cW}{\mathcal{W}}
\newcommand{\cX}{\mathcal{X}}
\newcommand{\cY}{\mathcal{Y}}
\newcommand{\cZ}{\mathcal{Z}}

% --- Bold Letters ---
\newcommand{\GG}{\mathbf{G}}
\newcommand{\HH}{\mathbf{H}}
\newcommand{\KK}{\mathbf{K}}
\newcommand{\TT}{\mathbf{T}}
\newcommand{\UU}{\mathbf{U}}
\newcommand{\VV}{\mathbf{V}}
\newcommand{\WW}{\mathbf{W}}
\newcommand{\XX}{\mathbf{X}}
\newcommand{\YY}{\mathbf{Y}}
\renewcommand{\ZZ}{\mathbf{Z}}

% --- Operators & Math Symbols ---
\DeclareMathOperator{\Param}{Param}
\DeclareMathOperator{\dom}{dom}
\DeclareMathOperator{\id}{\mathbf{1}}
\DeclareMathOperator{\pr}{pr}
\DeclareMathOperator{\ev}{ev}
\DeclareMathOperator{\Diff}{Diff}
\DeclareMathOperator{\Hom}{Hom}
\DeclareMathOperator{\Imm}{Imm}
\DeclareMathOperator{\Stab}{St}
\DeclareMathOperator{\Ab}{Ab}
\DeclareMathOperator{\Ad}{Ad}
\DeclareMathOperator{\Paths}{Paths}
\DeclareMathOperator{\Loops}{Loops}
\DeclareMathOperator{\comp}{comp}
\DeclareMathOperator{\class}{class}
\DeclareMathOperator{\bounds}{ends}
\DeclareMathOperator{\flip}{flip}
\DeclareMathOperator{\Taut}{Taut}
\DeclareMathOperator{\Liouv}{Liouv}
\DeclareMathOperator{\Value}{value}
\DeclareMathOperator{\Values}{val}
\DeclareMathOperator{\Surf}{Surf}
\DeclareMathOperator{\const}{const}
\DeclareMathOperator{\GL}{GL}
\DeclareMathOperator{\Str}{Str}
\DeclareMathOperator{\Ham}{Ham}
\DeclareMathOperator{\grad}{grad}
\DeclareMathOperator{\Sp}{Sp}
\DeclareMathOperator{\Isom}{Isom}
\DeclareMathOperator{\Sympl}{Sympl}
\DeclareMathOperator{\Orth}{Orth}

% --- Special Commands ---
\newcommand{\norm}[1]{\Vert #1 Vert}
\newcommand{\modulus}[1]{\vert #1 \vert}
\newcommand{\Cinfty}{C^\infty}
\newcommand{\undemi}{\frac{1}{2}}
\newcommand{\unsurdeuxi}{\frac{1}{2i}}
\newcommand{\but}{\hat{1}}
\newcommand{\source}{\hat{0}}
\newcommand{\DLie}{\mathcal{L}} % Lie derivative
\newcommand{\Gomega}{G_\omega}
\newcommand{\idGomega}{G_\omega^\circ}
\newcommand{\tidGomega}{\widetilde{G}_\omega^\circ}
\newcommand{\scal}[2]{\langle #1 \mid #2 \rangle}

\newcommand{\eK}{\mathbf{\sf K}}
\newcommand{\ek}{\mathbf{k}}
\newcommand{\ef}{\mathbf{f}}
\newcommand{\et}{\mathbf{t}}
\newcommand{\eF}{\mathbf{F}}

\newcommand{\R}{R}
\renewcommand{\L}{L}
\newcommand{\D}{D}
\newcommand{\B}{B}
\newcommand{\Z}{Z}
\newcommand{\T}{T}
\newcommand{\E}{E}

% --- Theorem Environments ---
\theoremstyle{plain}
\newtheorem{theorem}{Theorem}[section]
\newtheorem{lemma}[theorem]{Lemma}
\newtheorem{proposition}[theorem]{Proposition}
\newtheorem{corollary}[theorem]{Corollary}

\theoremstyle{definition}
\newtheorem{definition}[theorem]{Definition}
\newtheorem{example}[theorem]{Example}
\newtheorem{remark}[theorem]{Remark}
\newtheorem{note}[theorem]{Note}

% Map the old 'article' environment to subsection for structure
\newcommand{\article}[1]{\subsection{#1}}
\newcommand{\art}[1]{art.~\ref{#1}}

%%%%%%%%%%%%%%%%%%%%%%%%%%%%%%%%%%%%%%%%%%%%%%%%%%%%%%%%%%
%% MARK: Document Info
%%%%%%%%%%%%%%%%%%%%%%%%%%%%%%%%%%%%%%%%%%%%%%%%%%%%%%%%%%

\title{The Moment Maps in Diffeology}
\author{Patrick Iglesias-Zemmour}
\address{The Hebrew University of Jerusalem, Einstein Institute, Campus Givat Ram, 91904 Jerusalem, Israel.}
\email{piz@math.huji.ac.il}
\date{}

\subjclass[2000]{53C99, 53D30, 53D20}
\keywords{Diffeology, Moment Map, Symplectic Geometry}

%%%%%%%%%%%%%%%%%%%%%%%%%%%%%%%%%%%%%%%%%%%%%%%%%%%%%%%%%%
%% MARK: Begin Document
%%%%%%%%%%%%%%%%%%%%%%%%%%%%%%%%%%%%%%%%%%%%%%%%%%%%%%%%%%

\begin{document}
  
  \begin{abstract}
    This memoir presents a generalization of the moment maps to the category \{Diffeology\}.
    This construction applies to every smooth action of any diffeological group $G$ preserving a closed 2-form $\omega$,
    defined on some diffeological space $X$.
    In particular,
    that reveals a universal construction,
    associated to the action of the whole group of automorphisms $\Diff(X,\omega)$.
    By considering directly the space of momenta of any diffeological group $G$,
    that is the space $\cG^*$ of left-invariant 1-forms on $G$,
    this construction avoids any reference to Lie algebra or any notion of vector fields,
    or does not involve any functional analysis.
    These constructions of the various moment maps are illustrated by many examples,
    some of them originals and others suggested by the mathematical literature.
  \end{abstract}
  
  \maketitle
  
  \tableofcontents
  
  %%%%%%%%%%%%%%%%%%%%%%%%%%%%%%%%%%%%%%%%%%%%%%%%%%%%%%%%%%
  \section{Introduction}
  %%%%%%%%%%%%%%%%%%%%%%%%%%%%%%%%%%%%%%%%%%%%%%%%%%%%%%%%%%
  
  The moment map has been introduced in the 1970's in Souriau's work about the structure of dynamical systems \cite{Sou70}.
  It is the tool by excellence for dealing with symmetries in symplectic,
  or pre-symplectic geometry.
  But,
  in recent decades,
  the necessity appeared to extend the notion of symplectic formalism and moment maps,
  outside the usual framework of manifolds,
  to include constructions in infinite dimension
  --- spaces of connections of principal bundles, spaces of functions etc. ---
  or to include singular spaces
  --- orbifolds, singular symplectic reduction spaces etc..
  
  In this paper,
  we shall use the category \{Diffeology\} as the framework for such a generalization.
  We know already that diffeology is suitable to describe,
  in a unique and satisfactory way,
  manifolds or infinite dimensional spaces,
  as well as singular quotients.
  But,
  if diffeology excels with covariant objects,
  as differential forms,
  it is more subtle when it is question of contravariant objects like vector fields,
  Lie algebra\footnote{Several authors, beginning with Souriau, proposed some generalizations of Lie algebra in diffeology. But, it does not seem to exist a unique good choice. Such generalizations rely actually on the kind of problem treated.},
  kernel etc..
  Thus,
  in order to build a good diffeological theory of the moment map,
  and to avoid useless debates,
  we need to get freed from everything related to contravariant geometrical objects.
  
  Actually,
  the notion of moment map is not really an object of the symplectic world,
  but relates more generally to the category of space equipped with closed 2-forms.
  The non-degeneracy condition is secondary and can be skipped first from the data.
  This has been underlined explicitly by Souriau in his symplectic formulation of Noether's theorem,
  which involves pre-symplectic manifolds.
  On symplectic manifolds,
  Noether's theorem is empty.
  So,
  the moment map is just an object of the world of differential closed form,
  and there is no reason a priori that it could not be extended to diffeology which has a very well developed framework for De Rham's calculus.
  
  Now,
  in order to generalize the moment map in diffeology,
  we need to understand its meaning in the simplest possible case.
  Let $M$ be a manifold equipped with a closed 2-form $\omega$.
  And,
  let $G$ be a Lie group acting smoothly on $M$ and preserving $\omega$.
  That is,
  $g_M^*(\omega) = \omega$ for all elements $g$ of $G$,
  where $g_M$ denotes the action of $g$ on $M$.
  Let us assume that $\omega$ is exact,
  $\omega = d \lambda$,
  and moreover that $\lambda$ is also invariant by the action of $G$.
  So,
  for every point $m$ of $M$,
  the pullback of $\lambda$,
  by the orbit map $\hat m : g \mapsto g_M(m)$ is a left-invariant 1-form of $G$.
  That is,
  an element of the dual of the Lie algebra $\cG^*$.
  The map,
  $\mu : m \mapsto \hat m^*(\lambda)$ is exactly the moment map of the action of $G$ on the pair $(M,\omega)$
  (at least one of the moment maps, since they are defined up to constants).
  As we can see,
  this construction does not involve really the Lie algebra of $G$ but the space $\cG^*$ of left-invariant 1-forms on $G$.
  Since this space is well defined in diffeology,
  we have just to replace ``manifold'' by ``diffeological space'',
  and ``Lie group'' by ``diffeological group'',
  and everything works the same.
  So,
  let us change the manifold $M$ for a diffeological space\footnote{The space $X$ will be assumed to be connected, as many results need this hypothesis.} $X$,
  and let $G$ be some diffeological group.
  Let us continue to denote the space of left-invariant 1-forms on $G$ by $\cG^*$,
  even if the star does not refer a priori to some duality,
  and let us call it simply the {\em space of momenta} of the group $G$.
  Note that the group $G$ continues to act on $\cG^*$ by pullback of its adjoint action $\Ad : (g,k) \mapsto gkg^{-1}$,
  so we don't lose the notions of coadjoint action and coadjoint orbits.
  
  So,
  if we got the good space of momenta,
  which is the space where the moment maps are assumed to take their values,
  the problem remains that not every $G$-invariant closed 2-form is exact.
  And moreover,
  even if such form is exact,
  there is no reason,
  for some of its primitives to be $G$-invariant.
  We shall pass over this difficulty by introducing an intermediary,
  on which we can realize the simple case described above.
  This intermediary is the space $\Paths(X)$,
  of all the smooth paths of $X$,
  where the group $G$ acts naturally by composition.
  And since $\Paths(X)$ carries a natural functional diffeology,
  it is legitimate to consider its differential forms,
  and this is what we do.
  By integrating $\omega$ along the paths,
  we get a differential 1-form defined on $\Paths(X)$,
  and invariant by the action of $G$.
  The exact tool used here is the chain-homotopy operator $\eK$ \cite{Piz05}.
  The 1-form $\Lambda = \eK\omega$,
  defined on $\Paths(X)$,
  is a $G$-invariant primitive of the 2-form $\Omega = (\but^* - \source^*)(\omega)$,
  where $\but$ and $\source$ map every path of $X$ to its ends.
  Thus,
  thanks to the construction described above,
  we get a moment map $\Psi$ for the 2-form $\Omega = d\Lambda$ and the action of $G$ on $\Paths(X)$.
  But,
  this {\em paths moment map\/} $\Psi$ is not the one we are waiting for.
  We need to push it down on $X$,
  or moreover on $X \times X$.
  Now,
  if we get this way a {\em 2-points moment map\/} $\psi$ well defined on $X \times X$,
  it doesn't take anymore its value in $\cG^*$,
  as does $\Psi$,
  but in the quotient $\cG^*\!/\Gamma$,
  where $\Gamma$ is the image by $\Psi$ of all the loops of $X$.
  Fortunately,
  $\Gamma = \Psi(\Loops(X))$ is a subgroup of $(\cG^*,+)$ and depends on the loops only through their free homotopy classes.
  In other words,
  $\Gamma$ is an homomorphic image of the fundamental group $\pi_1(X)$ of $X$,
  or more precisely of its abelianized.
  Well,
  it is not a big deal to have the moment map taking its values in some quotient of the space of momenta,
  we can live with that.
  Especially if the group $\Gamma$ is invariant under the coadjoint action of $G$,
  which is actually the case\footnote{More precisely, the elements of $\Gamma$ are not just elements of $\cG^*$ but are moreover closed, and therefore invariant, each of them, by the coadjoint action of $G$.}.
  But,
  we are not completely done.
  The usual moment map is not a 2-points function,
  but a 1-point function.
  So,
  we have to extract our usual moment maps from this 2-points function $\psi$.
  This is quite easy,
  thanks to its very definition,
  the moment map $\Psi$ satisfies an additive property for juxtaposition of paths.
  And,
  the moment map $\psi$ inherits this property as a cocycle condition:
  for any three point $x$, $x'$ and $x''$ of $X$ we have $\psi(x,x') + \psi(x',x'') = \psi(x,x'')$.
  Hence,
  for $X$ connected,
  there exists always a map $\mu$ such that $\psi(x,x') = \mu(x') - \mu(x)$.
  And,
  any two such maps differ just by a constant.
  So,
  we get finally our wanted set of {\em moment maps\/} $\mu$,
  defined in the diffeological framework.
  The only difference,
  with the simplest case described above,
  is that the moment maps take their values in some quotient of the space of momenta,
  instead of the space of momenta itself.
  But,
  this is in fact already the case in the classical theory.
  It doesn't appear explicitly because people focus more on hamiltonian actions than just on symplectic actions.
  Actually,
  the group $\Gamma$ represents the very obstruction,
  for the action of $G$ on $(X,\omega)$,
  to be {\em hamiltonian}.
  We shall call $\Gamma$,
  the {\em holonomy} of the action of $G$.
  
  Now,
  let us come back to some properties of the various moment maps introduced above.
  The paths moment maps $\Psi$ and its projection $\psi$ are equivariant with respect to the action of $G$ on $X$ and the coadjoint action of $G$ on $\cG^*$,
  or the projection of the coadjoint action on $\cG^*\!/\Gamma$.
  But this is not anymore the case for the moments maps $\mu$.
  The variance of the maps $\mu$ reveals a family of cocycles $\theta$ from $G$ to $\cG^*\!/\Gamma$ differing just by coboundaries,
  and generalizing {\em Souriau's cocycles\/} \cite{Sou70}.
  This class of cocycles $\sigma$ belongs to the cohomology group $H^1(G, \cG^*\!/\Gamma)$,
  and will be called {\em Souriau's class} of the action of $G$ of $(X, \omega)$.
  Souriau's class $\sigma$ is precisely the obstruction for the 2-points moment map $\psi$ to be exact,
  that is for some moment map $\mu$ to be equivariant.
  Moreover,
  in parallel with the classical situation,
  every Souriau's cocycle $\theta$ defines a new action of $G$ on $\cG^*\!/\Gamma$,
  which we still call the affine coadjoint action (associated to $\theta$).
  And,
  the image of a moment maps $\mu$ is a collection of coadjoint orbits for this action.
  We call these orbits,
  the $(\Gamma,\theta)$-coadjoint orbits of $G$.
  Two different cocycles give two families of orbits translated by the same constant.
  
  Let us remark that the holonomy group $\Gamma$ and Souriau's class $\sigma$ appear clearly on a different level of meaning,
  the first one is responsible of the non hamiltonian character of the action of $G$,
  and the second characterizes the lack of equivariance of the moment maps.
  
  Well,
  until now we didn't use all the facilities offered by the diffeological framework.
  Since we do not restrict ourselves to the category of Lie groups,
  nothing prevents us to consider the group of all the {\em automorphisms} of the pair $(X,\omega)$.
  That is,
  the group $\Diff(X,\omega)$ of all the diffeomorphisms of $X$,
  preserving $\omega$.
  This group is a natural diffeological group,
  acting smoothly on $X$.
  Thus,
  everything built above applies to $\Diff(X,\omega)$,
  and every other action preserving $\omega$,
  of any diffeological group,
  pass through $\Diff(X,\omega)$,
  and through the associated object of the theory developed here.
  Therefore,
  considering the whole group of automorphisms of the closed 2-form $\omega$ of $X$,
  we get a natural notion of universal moment maps $\Psi_\omega$, $\psi_\omega$ and $\mu_\omega$,
  universal holonomy $\Gamma_\omega$,
  universal Souriau's cocycles $\theta_\omega$,
  and universal Souriau's class $\sigma_\omega$.
  By the way,
  this universal construction suggests a simple and new characterization,
  for any diffeological space $X$ equipped with a closed 2-form $\omega$,
  of the group of {\em hamiltonian diffeomorphisms} $\Ham(X,\omega)$,
  as the largest connected subgroup of $\Diff(X,\omega)$ whose holonomy vanishes.
  
  It is interesting to notice that,
  contrary to the original constructions \cite{Sou70} and most of its generalizations,
  the theory described above is essentially global,
  more or less algebraic,
  do not refer to any differential,
  or partial differential,
  equation and do not involve any notion of vector field or functional analysis techniques.
  
  I give,
  at the end of the memoir,
  several examples involving diffeological groups which are not Lie groups,
  or involving diffeological spaces which are not manifolds.
  We can see how the general theory applies to the singular ``symplectic irrational tori'' for which topology is irrelevant.
  These general constructions of moment maps are also applied to a few examples in infinite dimension,
  and an example which mixes finite and infinite dimensions.
  Finally,
  two examples of orbifolds are also examined.
  These examples show without any doubt the ability of this theory to treat correctly,
  in a unique framework,
  avoiding heuristic arguments,
  the large variety of situations we can find in the mathematical literature today.
  For infinite dimensional (heuristic) examples,
  see Donaldson's paper \cite{Dnl99}.
  By the way,
  I developed on purpose some tedious computations,
  even if it is boring,
  just to show diffeology at work.
  I mean,
  to show that diffeology is not just a formalism,
  but a working calculus method too.
  
  Considering the classical case of a closed 2-form $\omega$ defined on a manifold $M$,
  we show in particular that $\omega$ is non degenerate if and only if the group $\Diff(M,\omega)$ is transitive on $M$ and if a universal moment maps $\mu_\omega$ is injective.
  In other words,
  symplectic manifolds are identified,
  by the universal moment maps,
  to some coadjoint orbits (in our general sense) of their group of symplectomorphisms.
  This idea that ``every symplectic manifold is a coadjoint orbit'' is not new,
  it is suggested by a well known classification theorem for symplectic homogeneous Lie group actions \cite{Kir74} \cite{Kos70} \cite{Sou70},
  and has been stated already in a different context \cite{Omo86}.
  What is new here is that diffeology make this statement rigorous without the use of any functional analysis tools.
  
  In conclusion,
  beside the point that the construction developed in this memoir is a first step in the elaboration of the {\em symplectic diffeology program\/},
  I would emphasize the fact that,
  since \{Manifolds\} is a full and faithful subcategory of \{Diffeology\},
  all the constructions developed here apply to manifolds and give a faithful description of the classical theory of moment maps.
  As we have seen,
  there is no mention,
  and no use,
  of Lie algebra or vector fields in this exposition.
  This reveal the fact that these objects are also superfluous in the traditional approach,
  and can be avoided.
  And,
  I would add,
  they should be avoided.
  No just because then,
  they can be extended to larger categories,
  but because the use of contravariant object hide the deep fact that the theory of moment maps is a pure covariant theory.
  For example,
  we know that since coadjoint orbits of Lie groups are symplectic they are even dimensional.
  This is often regarded as a miracle,
  since it is not necessarily the case for adjoint orbits.
  But if we think that Lie algebra have little to do with the space of momenta of a Lie group,
  there is no more miracle,
  just different behaviors for different objects,
  which is unsurprising.
  Moreover I would add,
  but this can appear as more or less subjective,
  that avoiding all this va-et-vient between Lie algebra and dual of Lie algebra,
  the diffeological approach of the moment maps is much more simpler,
  and even deeper,
  than the classical approach.
  Compare for example Souriau's cocycle constructions in the original ``Structure des syst\`emes dynamiques'' \cite{Sou70} and in this memoir.
  The only crucial property used here is connectedness,
  that is the existence of enough smooth paths connecting points in spaces.
  
  Now,
  this constructions,
  in particular the new diffeological symplectic framework it suggests,
  come together with a lot of new questions which have not be answered here.
  And I hope I'll develop some of them in future works.
  
  \bigskip{\sc Note} --- Diffeology is a maximal extension of the local category of smooth real domains.
  It contains by the way,
  fully and faithfully,
  the category of manifolds.
  Diffeology has been introduced by J.-M Souriau at the beginning of the 1980s \cite{Sou81},
  and it is a variant of the theory of K.-T. Chen's {\em differentiable spaces\/} introduced few years before \cite{Che77}.
  Since then,
  the theory has been enhanced by some authors.
  The reader is assumed to be familiar with diffeology even if we remind some basics constructions in the first Section.
  For an comprehensive report on diffeology see \cite{Piz05}.
  
  %%%%%%%%%%%%%%%%%%%%%%%%%%%%%%%%%%%%%%%%%%%%%%%%%%%%%%%%%%
  \section{Few words about diffeology}
  %%%%%%%%%%%%%%%%%%%%%%%%%%%%%%%%%%%%%%%%%%%%%%%%%%%%%%%%%%
  
  This is a reminder of the few diffeological notions we will use in the following.
  More details about these constructions,
  and proofs,
  can be found in \cite{Piz05}.
  
  \article{Domains and parametrizations}
  \label{Domains-and-parametrizations}
  We call {\em numerical space} any power of the real numbers $\RR$,
  and we call {\em numerical domain},
  or simply {\em domain\/},
  any open set of any numerical domain.
  If $U$ is a domain of $\RR^n$,
  we say that $U$ is an {\em $n$-domain}.
  Let $X$ be a set,
  we call {\em parametrization in $X$} any map defined on some numerical domain with values in $X$.
  The set of all the parametrizations in $X$ is denoted by $\Param(X)$.
  For any parametrization $P : U \to X$,
  the numerical domain $U$ is called the {\em domain} of $P$ and is denoted by $\dom(P)$.
  If $U$ is an $n$-domain we say that $P$ is a {\em $n$-parametrization\/}.
  
  \article{Diffeology and diffeological spaces}
  \label{Diffeology-and-diffeological-spaces}
  Let $X$ be a set.
  A {\em diffeology on $X$} is a set $\cD$ of parametrizations in $X$,
  that is ${\cD} \subset \Param(X)$,
  such that
  \begin{enumerate}
    \item[D1.] {\em Covering}\hspace{1em} Every point of $X$ is contained in the range of some $P\in \cD$.
    \item[D2.] {\em Locality}\hspace{1em} If $P \in\Param(X)$ and if for any $r \in \dom(P)$ there exists a domain $V$ such that $r \in V \subset \dom(P)$ and $P \restriction V \in \cD$,
    then $P\in {\cD}$.
    \item[D3.] {\em Smooth compatibility}\hspace{1em} If $P \in \cD$ and $F$ is a $\Cinfty$ mapping from some domain $V$ to $\dom(P)$, then $P \circ F \in {\cD}$.
  \end{enumerate}
  Equipped with a diffeology $\cD$,
  $X$ is a {\em diffeological space}.
  To make it short,
  the elements of the diffeology are called the {\em plots} of the diffeological space.
  So,
  the plots of a diffeological space are the elements of its diffeology.
  Note that the definition of a diffeology does not assume any pre-existing structure on the underlying set.
  
  \article{Smooth maps and diffeomorphisms}
  \label{Differentiable-maps-and-diffeomorphisms}
  Let $X$ and $X'$ be two sets equipped with the diffeologies $\cD$ and $\cD'$ respectively.
  A map $F : X \to Y$ is said to be {\em smooth} if for each $P \in \cD$ we have $F \circ P \in \cD'$.
  The set of smooth maps from $X$ to $Y$ is denoted by $\cC^{\infty}(X,Y)$.
  A bijective map $F : X \to Y$ is said to be a {\em diffeomorphism} if both $F$ and $F^{-1}$ are smooth.
  The set of diffeomorphisms of $X$ is a group denoted by ${\rm Diff}(X)$.
  Diffeological spaces are the objects of the category \{Diffeology\} whose morphisms are {\em smooth maps},
  and isomorphisms are {\em diffeomorphisms}.
  
  \article{Quotients and subspaces}
  \label{Quotient-and-subspaces}
  The category \{Diffeology\} is stable by set theoretic operations.
  Products,
  sums of diffeological spaces are naturally diffeological spaces,
  but also quotient and subsets.
  Let $\sim$ be any equivalence relation on a diffeological space $X$,
  let $Q = X/\!\!\sim$ and $\pi : X \to Q$ be the projection.
  There exists a natural {\em quotient diffeology} on $Q$,
  for which $\pi$ is smooth,
  defined by the parametrizations which can be lifted locally along $\pi$ by elements of $\cD$.
  That is,
  a parametrization $P : U \to Q$ is a plot if and only if for each $r \in U$ there exists a domain $V$ containing $r$ and a plot $\phi : V \to X$ such that $P \restriction V = \pi \circ \phi$.
  On the other hand,
  there exists on every subset $A \subset X$ a natural {\em subset diffeology},
  for which the inclusion is smooth,
  defined by the elements of $\cD$ which take their values in $A$.
  In the first case,
  the map $\pi : X \to Q$ is a {\em subduction},
  and in the second case the injection $j_A : A \to X$ is an {\em induction}.
  
  \article{Functional diffeology}
  \label{Functional-diffeology}
  Let $X$ and $X'$ be two diffeological spaces.
  There exists on $\Cinfty(X,X')$ a diffeology called the {\em functional diffeology} whose plots are parametrizations $P$ such that $(r,x) \mapsto P (r)(x)$,
  defined on $\dom(P) \times X$ to $X'$ is smooth.
  This diffeology is the {\em coarsest\/} (e.g. largest) diffeology such that the {\em evaluation map} $(f,x) \mapsto f(x)$,
  from $\Cinfty(X,X') \times X$ to $X'$,
  is smooth.
  In particular,
  the set of paths $\Cinfty(\RR,X)$,
  denoted by $\Paths(X)$,
  is naturally a diffeological space,
  equipped with the functional diffeology.
  
  \article{Differential forms}
  \label{Differential-forms}
  Let $X$ be a diffeological space.
  A {\em differential $k$-form} on $X$,
  for $k \geq 0$,
  is a mapping $\alpha$ which associates to each plot $P$ of $X$ a {\em smooth $k$-form} on $\dom(P)$.
  That is,
  if $P$ is an $n$-plot,
  $\alpha(P)$ belongs to $\Cinfty(\dom(P), \Lambda^k(\RR^n))$.
  And satisfying the following compatibility condition:
  for any plot $P$ of $X$ and for any smooth parametrization $F : V \to \dom(P)$,
  $$\alpha(P \circ F) = F^*(\alpha(P)).$$
  The space $\Omega^k(X)$ of differential $k$-forms on $X$ is naturally a vector space.
  It carries also a natural diffeology called again {\em functional diffeology} for which the ordinary vectorial operations are smooth.
  A parametrization $r \mapsto \alpha_r$ of $\Omega^k(X)$,
  defined on a domain $U$,
  is a plot for this functional diffeology if and only if for any $n$-plot $P : V \to X$,
  the parametrization $(r,s) \mapsto \alpha_r(P)_s$,
  defined on $U \times V$ with values in $\Lambda^k(\RR^n)$,
  is smooth.
  
  Note that,
  if it is necessary for a differential form to check the compatibility condition on all the plots of the space,
  two differential $k$-forms coincide if and only if they coincide on the $k$-plots.
  In other words,
  the value of a differential $k$-form is characterized by its values on the $k$-plots.
  
  The {\em exterior differential} of a $k$-form $\alpha$ is the differential $(k+1)$-form defined by
  $$d\alpha(P) = d(\alpha(P)).$$
  Let $f : X \to X'$ be a smooth map between diffeological spaces,
  let $\alpha'$ be a differential $k$-form on $X'$,
  the pullback $f^*(\alpha')$ is the differential $k$-form on $X$ defined by $f^*(\alpha')(P) = \alpha'(f \circ P)$.
  The exterior differential and the pullback are linear and smooth operations.
  
  Let $F : \cI \to \Diff(X)$ be a 1-plot defined on a open interval and centered at the identity $\id_X$,
  that is $0 \in \cI$ and $F(0) = \id_X$.
  Let $\alpha$ be a differential $k$-form on $X$,
  with $k>0$.
  The {\em contraction} $i_{F}(\alpha)$ of $\alpha$ by $F$ is the $(k-1)$-differential form defined by
  $$
    i_{F}(\alpha)(P)_r(v_2, \ldots, v_{k}) =
    \alpha\bigg[\mymatrix{t \cr r} \mapsto
    F(t)(P(r))\bigg]_{0 \choose r} \mymatrix{1 &
    0 & \cdots & 0 \cr 0 &v_2 & \cdots & v_{k}},
    $$
  where $P$ is any plot of $X$,
  $r \in \dom(P)$,
  and $v_2,\ldots,v_{k}$ are any $k-1$ vectors of $\RR^n$,
  $n$ being the dimension of the plot $P$.
  
  Let us continue with the 1-plot $F : \cI \to \Diff(X)$ defined on $\cI$ and centered at $\id_X$.
  Let $\alpha$ be a differential $k$-form on $X$,
  with $k \geq 0$.
  There exists a differential $k$-form on $X$,
  called the {\em Lie derivative} of $\alpha$ by $F$,
  defined by
  $$
    \DLie_F(\alpha)(P)_r =
    \left.{\partial
    \alpha(F(t) \circ P)_r \over \partial t}\right|_{t=0}
    $$
  for every $n$-plot $P$ and every $r \in \dom(P)$.
  Note that $\alpha(F(t) \circ P)$ is just $F(t)^*(\alpha)(P)$,
  and regarded as a function of $t$ is smooth from $\cI$ to $\Lambda^k(\RR^n)$,
  so the derivative with respect to $t$ makes sense.
  Now,
  the so called classical {\em Cartan formula} extends to diffeology and we have,
  for any differential $k$ form $\alpha$,
  with $k>0$,
  $$
    \DLie_F(\alpha) = d[i_F(\alpha)] + i_F(d\alpha).
    $$
  
  Let us fix now some vocabulary we shall use in the later paragraphs.
  We call {\em automorphism} of a differential $k$-form $\alpha$ on $X$ any diffeomorphism $\varphi$ of $X$ which preserves $\alpha$,
  that is $\varphi^*(\alpha) = \alpha$.
  The set of all the automorphisms of the form $\alpha$ is a group denoted by $\Diff(X,\alpha)$,
  $$
    \Diff(X,\alpha) = \{ \varphi \in \Diff(X) \mid \varphi^*(\alpha) = \alpha \}.
    $$
  The group $\Diff(X,\alpha)$ will be called {\em the group of automorphisms of $\alpha$},
  and any of its subgroups will be called {\em a group of automorphisms of $\alpha$}.
  
  \article{Chain-Homotopy operator}
  \label{Chain-Homotopy-operator}
  Let $X$ be a diffeological space.
  Let $\source$ and $\but$ be the maps defined on $\Paths(X)$ to $X$ by
  $$\source(p) = p(0) \qmbox{and} \but(p) = p(1).$$
  There exists a smooth linear operator $\cK$,
  called {\em Chain-Homotopy operator} such that,
  for any integer $k>0$,
  $$
    \cK : \Omega^k(X) \to \Omega^{k-1}(\Paths(X))
    \qmbox{and} \cK \circ d + d \circ \cK = \but^* - \source^*.
    $$
  The value of the chain-homotopy operator $\cK$ on a differential $k$-form $\alpha$ is given by the following formulas.
  For $k = 1$,
  $\cK\alpha$ is a real function
  $$
    \cK(\alpha)(p) = \int_0^1 \alpha(p)_t(1) \ dt
    \qmbox{with} \alpha \in \Omega^1(X) \qmbox{and} p \in \Paths(X).
    $$
  For $k>1$,
  let $P : U \to \Paths(X)$ be a $n$-plot,
  let $r \in U$ and let $v_2,\ldots,v_{k}$ be $k-1$ vectors of $\RR^n$,
  so
  $$
    (\eK\alpha)(P)_r(v_2, \ldots,v_{k}) =
    \int_0^1 \alpha \left[ \mymatrix{s \cr r} \mapsto P (r)(s
    + t) \right]_{0 \choose r}\mymatrix{1 & 0 &
    \cdots & 0 \cr 0 & v_2 & \cdots & v_{k}} \
    dt.
    $$
  The chain-homotopy operator satisfies a natural equivariance relation.
  Let $X'$ be another diffeological space and $f \in \Cinfty(X,X')$.
  Let $f_* : \Paths(X) \to \Paths(X')$ be the natural map $f_* : p \mapsto f \circ p$.
  Let $\cK_X$ and $\cK_{X'}$ be the chain-homotopy operators associated to $X$ and $X'$,
  so
  $$
    \cK_X \circ f^* = (f_*)^* \circ \cK_{X'}.
    $$
  In particular,
  if $X = X'$ and if $f$ preserves a differential $k$-form $\alpha$,
  that is $f^*(\alpha) = \alpha$,
  then $f_*$ preserves the differential $(k-1)$-form $\cK(\alpha)$,
  that is $(f_*)^*(\cK\alpha) = \cK\alpha$.
  
  %%%%%%%%%%%%%%%%%%%%%%%%%%%%%%%%%%%%%%%%%%%%%%%%%%%%%%%%%%
  \section{Diffeological groups and momenta}
  %%%%%%%%%%%%%%%%%%%%%%%%%%%%%%%%%%%%%%%%%%%%%%%%%%%%%%%%%%
  
  Diffeological groups have been first introduced as ``groupes différentiels'' by Souriau in \cite{Sou81} \cite{Sou84}.
  They are,
  with respect to diffeological spaces,
  what Lie groups are to manifolds.
  We remind here their definition.
  Then,
  we propose a diffeological equivalent of the ``dual of the Lie algebra'' as the space of invariant 1-forms on the group.
  We don't consider any duality with a putative diffeological Lie algebra.
  This is the simpler and the more natural way to work with coadjoint action and coadjoint orbits in diffeology.
  
  \article{Diffeological groups}
  \label{Diffeological-groups}
  Let $G$ be a group equipped with a diffeology $\cD$.
  We say that $G$ is a {\em diffeological group},
  or $\cD$ is a {\em group diffeology},
  if and only if the multiplication as well as the inversion are smooth.
  That is,
  $$
[(g,g')\mapsto gg'] \in \Cinfty(G\times G,G) \quad \mbox{and}
\quad [g\mapsto g^{-1}] \in \Cinfty(G, G).
    $$
  Note that if $G$ is a standard manifold,
  this definition is nothing but the definition of Lie groups.
  Note that any subgroup of a diffeological group,
  equipped with the subset diffeology,
  is a diffeological group.
  As well,
  the quotient of any diffeological group by a normal subgroup is a diffeological group for the quotient diffeology.
  We denote by $\Hom^\infty(G,G')$ the space of smooth homomorphisms from $G$ to another diffeological group $G'$.
  
  An important example of diffeological group is the groups of all the diffeomorphisms of a diffeological space $X$,
  equipped with the {\em functional diffeology of group of diffeomorphisms}.
  This diffeology is the coarsest group diffeology on $\Diff(X)$ such that the evaluation map $(f,x) \mapsto f(x)$ is smooth.
  A parametrization $P : U \to \Diff(X)$ is a plot if and only if the maps $(r,x) \mapsto P (r)(x)$ and $(r,x) \mapsto P (r)^{-1}(x)$ are smooth.
  
  \article{Covering diffeological groups}
  \label{Covering-diffeological-groups}
  Let $\hat G$ and $G$ be two diffeological groups.
  We say that a subduction $\pr : \hat G \to G$ is a {\em group covering} if and only if $\pr$ is an homomorphism and the fiber $K = \pr^{-1}(\id_G)$ is discrete\footnote{Let us remind that {\em discrete} means that the plots (here the plots for the subset diffeology) are locally constant.}.
  Let $G$ be a connected diffeological group.
  Its universal covering $\tilde G$ has a natural structure of diffeological group such that the subduction $\pi : \tilde G \to G$ is an homomorphism.
  The first homotopy group $\pi_1(G) = \ker(\pi)$ is a discrete invariant subgroup of $\tilde G$,
  so $\pi$ is a group covering.
  Any other connected covering $\pr : \hat G \to G$ is the quotient of the universal covering by a subgroup $K$ of $\pi_1(G)$.
  If the subgroup $K$ is normal then $\pr$ is a group covering.
  
  \begin{proof}
    This property has been stated originally in \cite{Sou84} \cite{Don84},
    but let us remind the general construction given in \cite{Igl85}.
    Let $X$ be a connected diffeological space,
    let $x_0$ be a point of $X$,
    chosen at the base point.
    Let $\Paths(X,x_0)$ be the space of paths starting at $x_0$.
    First of all,
    the end map $\but : p \mapsto p(1)$,
    defined on $\Paths(X,x_0)$ is a subduction.
    The quotient of $\Paths(X,x_0)$ by the fixed ends homotopy relation is exactly the universal covering pointed by the constant map $\hat x_0 : t \mapsto x_0$,
    over the pointed space $(X, x_0)$.
    The fiber over $x_0$ is the homotopy group $\pi_1(X, x_0)$.
    Now if $X = G$ we choose the identity $\id_G$ as base point.
    Thus,
    the multiplication of paths $(p,p') \mapsto [t \mapsto p(t) \cdot p'(t)]$ defines on $\tilde G$ a group multiplication such that the projection $\pi : \tilde G \to G$,
    defined by $\pi(\class(p)) = \but(p)$,
    is an homomorphism.
    The kernel of this morphism is clearly the fiber over $\id_G$,
    that is $\pi_1(G)$.
    Now,
    the kernel of an homomorphism is always an invariant subgroup.
    And,
    since $\pi$ is a covering,
    $\pi^{-1}(\id_G)$ is discrete.
    This last points are general results of the diffeological theory of homotopy \cite{Igl85}.
  \end{proof}
  
  \article{Smooth actions of a diffeological group}
  \label{Smooth-actions-of-a-diffeological-group}
  Let $G$ be a diffeological group.
  Let $X$ be a diffeological space.
  Let the group $\Diff(X)$,
  of all the diffeomorphisms of $X$,
  be equipped with the functional diffeology of group of diffeomorphisms.
  A {\em smooth action} of $G$ on $X$,
  or simply an {\em action} of $G$ on $X$,
  is a smooth homomorphism $\rho$ from $G$ to $\Diff(X)$,
  that is $\rho \in \Hom^\infty(G, \Diff(X))$.
  Let us fix or remind some vocabulary used in the following.
  \begin{enumerate}
    \item We says that the action is {\em effective} if $\ker(\rho) = \{\id_G\}$.
    \item The {\em orbits} of $G$ are the subsets $\rho(G)(x) = \{ \rho(g)(x) \mid g \in G \}$, where $x \in X$.
    \item We call {\em orbit maps} of a point $x \in X$, the smooth map $\hat x : G \to X$, defined by $\hat x : g \mapsto \rho(g)(x)$.
    \item The {\em stabilizer} $\Stab_\rho(x)$ of a point $x \in X$ is the subgroup of $G$ defined by the equation $\hat x(g) = x$, $g \in G$.
    \item We say that $X$ is {\em homogeneous} for the action $\rho$ of $G$, or that $X$ is an {\em homogeneous space} of $G$, for $\rho$, if and only if the orbit map $\hat x$ of some point $x \in X$ is a subduction, thus for every point. In this case, $\hat x$ is a principal fibration \cite{Igl85} with structure group the stabilizer $\Stab_\rho(x)$. That is $X \simeq G /\Stab_\rho(x)$, where $g' \sim g h$ with $h \in \Stab_\rho(x)$.
  \end{enumerate}
  Let $\alpha$ be a differential $k$-form on $X$.
  We say that $G$ {\em acts by automorphisms} on $(X, \alpha)$ if $\rho$ takes it values in $\Diff(X,\alpha)$.
  That is,
  if $\rho(G)$ is a group of automorphisms of the differential form $\alpha$.
  
  \article{Covering smooth actions}
  \label{Covering-smooth-actions}
  Let $X$ be a connected diffeological space.
  Let $G$ be a connected diffeological group.
  Let $\rho : G \to \Diff(X)$ be a smooth action of $G$ on $X$.
  Thus,
  $\rho$ takes its values in the identity component $\Diff(X)^\circ = \comp(\id_X) \subset \Diff(X)$.
  So,
  there exists a unique smooth action $\tilde \rho$ of the universal covering $\tilde G$ of $G$ on the universal covering $\tilde X$ of $X$,
  covering $\rho$.
  \[
  \begin{tikzcd}[column sep=large, row sep=large, every label/.append style = {font = \small}]
    \widetilde{G} \arrow[r, "\tilde{\rho}"] \arrow[d, "\pi_G"'] & \widetilde{\Diff(X)^\circ} \arrow[d, "\pi_{\Diff(X)}"] \\
    G \arrow[r, "\rho"'] & \Diff(X)^\circ
  \end{tikzcd}
  \]
  
  \begin{proof}
    The map $\rho \circ \pi$ is smooth and $\widetilde G$ is simply connected.
    So,
    thanks to the monodromy theorem \cite{Igl85},
    there exists a unique lifting $\tilde \rho$ of $\rho \circ \pi$ mappings the identity of $\tilde G$ to the identity of $\widetilde{ \Diff(X)^\circ}$.
    Now,
    this lifting is an homomorphism because its restriction on $\ker(\pi_G)$ and its projection $\rho$ are both homomorphisms.
  \end{proof}
  
  \article{Left, right and adjoint actions of a group onto itself}
  \label{Left-right-and-adjoint-actions-of-a-group-onto-itself}
  Let $G$ be a diffeological group.
  We denote by $\L(g)$ and $\R(g)$ the {\em left} and {\em right actions} of $G$ onto itself.
  $$ \mbox{For all } g \in G, \quad \left\{
    \begin{array}{l}
    {\L(g)} : g' \mapsto gg' \\
    {\R(g)} : g' \mapsto g'g.
    \end{array}
    \right.
    $$
  Note that the ``right action'' is in fact an anti-action.
  That is,
  $\R(gg') = \R(g') \circ \R(g)$.
  The {\em adjoint action} of $G$ onto itself is denoted by $\Ad$,
  and is defined by:
  $$ \mbox{For all } g \in G, \quad \Ad(g):k \mapsto gkg^{-1} = \L(g) \circ \R(g^{-1})(k).
    $$
  The maps $\L$ and $\Ad$ are smooth homomorphisms from $G$ to $\Diff(G)$,
  equipped with the diffeology of group of diffeomorphisms.
  The map $\R$ is a smooth anti-homomorphism from $G$ to $\Diff(G)$.
  
  \article{Momenta of a diffeological group}
  \label{Momenta-of-a-diffeological-group}
  We call {\em left momentum}
  --- or simply {\em momentum} ---
  of a diffeological group $G$,
  any 1-form of $G$,
  invariant by the left action of $G$ onto itself.
  We denote by $\cG^*$ the {\em space of momenta} of $G$.
  The space of momenta of a diffeological group is naturally a diffeological vector space,
  equipped with the functional diffeology.
  So,
  $$ \cG^* = \{ \alpha \in \Omega^1(G) \mid \mbox{For all } g \in G, \ \L(g)^*(\alpha) = \alpha \}.
    $$
  Note that,
  in spite of what the notation $\cG^*$ suggests,
  the space of momenta of a diffeological group is not defined by some duality.
  This notation is chosen here just to remind us the connection with the dual of the Lie algebra in the case of Lie groups.
  
  \article{Momenta and connectedness}
  \label{Momenta-and-connectedness}
  Let $G$ be a diffeological group.
  Let $G^\circ$ be the identity component of $G$,
  that is $G^\circ = \comp(\id_G) \subset G$.
  So,
  the pullback $j^* : \cG^* \to {\cG^\circ}^{\raisebox{-3.2pt}{\scriptsize *}}$ of the injection $j : G^\circ \to G$ is an isomorphism.
  This property is quite natural but needed to be checked up in our context of diffeological groups.
  
  {\sc Note} --- Said differently,
  the space of momenta of a connected diffeological group,
  or any of its extensions by a discrete group,
  coincide.
  In particular,
  the only momentum of a discrete group is the zero momentum.
  
  \begin{proof}
    Let us check first the injectivity.
    Let $\alpha \in \cG^*$ such that $j^*(\alpha) = 0$,
    and let $P : U \to G$ be a plot.
    Let $r_0 \in U$ and let $\B \subset U$ be a small open ball centered at $r_0$.
    Let $g_0 = P (r_0)$.
    Since $\B$ is connected,
    since $\L(g_0^{-1}) \circ P (r_0) = \id_G$,
    and thanks to the smoothness of group operations,
    the parametrization $Q = [\L(g_0^{-1}) \circ P] \restriction \B$ is a plot of $G^\circ$.
    So,
    $\alpha(Q) = 0$.
    But,
    $\alpha(Q) = \alpha(\L(g_0^{-1}) \circ (P \restriction \B)) = \L(g_0^{-1})^*(\alpha)(P \restriction \B) = \alpha(P \restriction \B)$.
    Thus,
    $\alpha(P \restriction \B) = 0$.
    Since $\alpha$ vanishes locally at each point of $U$,
    $\alpha = 0$.
    And,
    $j^*$ is injective.
    Now,
    let us prove the surjectivity.
    Let $\alpha \in {\cG^\circ}^{\raisebox{-3.2pt}{\scriptsize *}}$.
    For any component $G_i$ of $G$,
    let us choose an element $g_i \in G_i$,
    and the identity for the identity component.
    Let $P : U \to G$ be a plot,
    an let us assume that $U$ is connected.
    So,
    $P (U)$ is contained in one connected component of $G$,
    let us say the component $G_i$.
    Let us define then,
    $\bar \alpha(P) = \alpha(\R(g_i^{-1}) \circ P)$.
    Since $\R(g_i^{-1}) \circ P (r) \in G^\circ$ for all $r \in U$,
    this is well defined.
    Now,
    since any plot is the sum of its restrictions on the components of its domain,
    the map $\bar \alpha$ extends naturally to every plot of $G$.
    Now,
    let $P : U \to G$ be a plot,
    let $V$ be a domain,
    and let $F \in \Cinfty(V,U)$.
    Let $s_0 \in V$,
    let $V_0$ be the component of $s_0$ in $V$,
    let $r_0 = F(s_0)$,
    and let $U_0$ be the component of $r_0$ in $U$.
    Let $G_i$ be the component of $P \circ F(s_0) = P (r_0)$ in $G$.
    We have,
    $\bar \alpha((P \circ F) \restriction V_0) = \bar \alpha((P \restriction U_0) \circ (F \restriction V_0)) = \alpha(\R(g_i^{-1}) \circ (P \restriction U_0) \circ (F \restriction V_0)) = \alpha([\R(g_i^{-1}) \circ (P \restriction U_0)] \circ (F \restriction V_0)) = (F \restriction V_0)^*[\alpha(\R(g_i^{-1}) \circ (P \restriction U_0)] = (F \restriction V_0)^*[\bar \alpha(P \restriction U_0)]$.
    So locally,
    $\bar \alpha (F \circ P) =_{\rm loc} F^*(\bar \alpha(P))$.
    And if it is satisfied locally,
    it is satisfied globally,
    thus $\bar \alpha (F \circ P) = F^*(\bar \alpha(P))$.
    The map $\bar \alpha$ is a well defined differential 1-form on $G$.
    Now,
    let us check that $\bar \alpha$ is invariant by left multiplication.
    Let $g \in G$,
    let $P : U \to G$ be a plot,
    let $r_0 \in U$,
    let $U_0$ be the component of $r_0$ in $U$,
    let $G_i$ be the component of $P (r_0)$ in $G$,
    so $P (U_0) \subset G_i$.
    We have,
    $\L(g)^*(\bar \alpha(P \restriction U_0)) = \bar \alpha(\L(g) \circ (P \restriction U_0)) = \alpha(\R(g_i^{-1}) \circ \L(g) \circ (P \restriction U_0)) = \alpha( \L(g) \circ \R(g_i^{-1}) \circ (P \restriction U_0)) = [\L(g)^*(\alpha)] (\R(g_i^{-1}) \circ (P \restriction U_0)) = \alpha (\R(g_i^{-1}) \circ (P \restriction U_0)) = \bar \alpha (P \restriction U_0)$.
    So locally,
    $\L(g)^*(\bar \alpha)(P) =_{\rm loc} \bar \alpha(P)$,
    and therefore globally.
    So,
    $\L(g)^*(\bar \alpha) = \bar \alpha$,
    thus $\bar \alpha$ is an element of $\cG^*$,
    which coincide with $\alpha$ on $G^\circ$.
  \end{proof}
  
  \article{Momenta of coverings of diffeological groups}
  \label{Momenta-of-covering-of-diffeological-groups}
  Let $G$ be a diffeological group,
  let $\pr : \hat G \to G$ be some group covering,
  see \ref{Covering-diffeological-groups}.
  Let $\cG^*$ and $\hat \cG^*$ be the spaces of momenta of $G$ and $\hat G$.
  So,
  the pullback $\pr^* : \cG^* \to \hat \cG^*$ is a smooth linear isomorphism.
  
  \begin{proof}
    Thanks to \ref{Momenta-and-connectedness},
    it is sufficient to assume that $\hat G$ and $G$ are connected.
    And thanks to \ref{Covering-diffeological-groups},
    it is sufficient to prove this for the universal covering $\pi : \tilde G \to G$.
    Now,
    $\pi^*$ is obviously linear,
    let us show that $\pi^*$ is surjective.
    Let $\tilde\alpha \in \widetilde\cG^*$.
    The group $G$ is isomorphic to $\widetilde G/\pi_1(G)$,
    with respect to the left action of $\pi_1(G)$.
    That is $\tilde g \sim k \tilde g$,
    for all $k \in \pi_1(G)$.
    Now,
    let $\tilde \alpha \in \widetilde\cG^*$,
    $\tilde \alpha$ is left invariant by $\widetilde G$,
    thus by $\pi_1(G)$.
    That is,
    for all $k \in \pi_1(G)$,
    $\L(k)^*(\tilde \alpha) = \tilde \alpha$.
    But,
    since $\pi_1(G) = \ker(\pi)$ is discrete,
    this is sufficient for the existence of a 1-form $\alpha$ on $G$ such that $\tilde \alpha = \pi^*(\alpha)$.
    Now,
    let $\tilde g \in \widetilde G$ and $g = \pi(\tilde g)$.
    Since $\pi$ is an homomorphism,
    $\pi \circ \L(\tilde g) = \L(g) \circ \pi$.
    So,
    on one hand we have $\L(\tilde g)^*(\tilde \alpha) = \L(\tilde g)^*(\pi^*(\alpha)) = (\pi \circ \L(\tilde g))^*(\alpha) = (\L(g) \circ \pi)^*(\alpha) = \pi^*(\L(g)^*(\alpha))$.
    And,
    on the other hand,
    we have $\L(\tilde g)^*(\tilde \alpha) = \tilde \alpha = \pi^*(\alpha)$.
    Hence,
    $\pi^*(\L(g)^*(\alpha)) = \pi^*(\alpha)$.
    But,
    since $\pi$ is a subduction,
    $\L(g)^*(\alpha) = \alpha$.
    Thus,
    $\alpha \in \cG^*$,
    and the map $\pi^*$ is surjective.
    Now,
    let $\tilde \alpha$ and $\tilde \beta$ be such that $\pi^*(\tilde \alpha) = \pi^*(\tilde \beta)$.
    But,
    since $\pi$ is a subduction,
    $\tilde \alpha = \tilde \beta$.
    Finally,
    $\pi^*$ is injective.
    Finally,
    since the pullback is a smooth operation,
    $\pi^* : \cG^* \to \widetilde \cG^*$ is a smooth linear isomorphism.
  \end{proof}
  
  \article{Linear coadjoint action and coadjoint orbits}
  \label{Linear-coadjoint-action}
  Let $G$ be a diffeological group and let $\cG^*$ be the space of its momenta.
  The pushforward $\Ad(g)_*(\alpha)$ of a momentum $\alpha \in \cG^*$,
  by the adjoint action of any element $g$ of $G$,
  is again a momentum of $G$,
  that is again a left-invariant 1-form.
  This defines a linear smooth action of $G$ on $\cG^*$ called {\em coadjoint action},
  and denoted by $\Ad_*$.
  $$
    \Ad_* : (g,\alpha) \mapsto \Ad(g)_*(\alpha) = \Ad(g^{-1})^*(\alpha).
    $$
  We check immediately that for all $g$, $g'$ in $G$,
  $\Ad_*(gg') = \Ad_*(g) \circ \Ad_*(g')$,
  and that $\Ad_*(g)$ is linear.
  Note that,
  since $\alpha$ is left-invariant,
  $\Ad_*(g)(\alpha) = \R(g)^*(\alpha)$.
  
  The orbit of $\alpha$ by $G$ is by definition a {\em coadjoint orbit\/} of $G$,
  and it will be denoted by
  $$
    \cO_\alpha \mbox{ or } \Ad_*(G)(\alpha) = \{ \Ad_*(g)(\alpha) \mid g\in G \}.
    $$
  The orbit $\cO_\alpha$ can be regarded as a subset of $\cG^*$,
  but also as the quotient of the group $G$ by the stabilizer of the moment $\alpha$,
  $$
    \cO_\alpha \simeq G/\Stab_{G}(\alpha), \mbox{ with }
    \Stab_{G}(\alpha) = \{ g \in G \mid \Ad(g)_*(\alpha) = \alpha \}.
    $$
  {\sc Note} --- The orbit $\cO_\alpha$ can be equipped with the subset diffeology of the functional diffeology of $\cG^*$,
  or with the quotient diffeology of $G$.
  There is no reason a priori that these two diffeologies coincide.
  But it could be interesting however to understand in which conditions they do.
  
  \article{Affine coadjoint actions and $(\Gamma,\theta)$-coadjoint orbits}
  \label{Affine-coadjoint-actions-and-orbits}
  Let $G$ be a diffeological group,
  and $\cG^*$ be the space of its momenta.
  Let $\Gamma \subset \cG^*$ be a subgroup of $(\cG^*,+)$,
  invariant by the coadjoint action $\Ad_*$.
  That is,
  for all $g \in G$,
  $$
    \Ad_*(g)(\Gamma) \subset \Gamma.
    $$
  So,
  the coadjoint action of $G$ on $\cG^*$ project to the quotient $\cG^*\!/\Gamma$,
  regarded as an abelian group,
  on a smooth action.
  Let us denote this action by $\Ad_*^\Gamma$.
  For every $g \in G$ and $\tau \in \cG^*\!/\Gamma$,
  $$
    \Ad_*^\Gamma(g)(\tau) = \class(\Ad_*(g)(\mu))
    \qmbox{with} \tau = \class(\mu) \in \cG^*\!/\Gamma.
    $$
  Now,
  let $\theta$ be a smooth map from $G$ to the space $\cG^*\!/\Gamma$,
  such that for any pair $g$ and $g'$ of elements of $G$,
  $$
    \theta(g g') = \Ad_*^\Gamma(g)(\theta(g')) + \theta(g).
    $$
  Such maps are formally known,
  in the literature as twisted 1-cocycles of $G$ with values in $\cG^*\!/\Gamma$ \cite{Kir74}.
  We shall call them cocycles of $G$,
  with values in $\cG^*\!/\Gamma$,
  or simply $(\cG^*\!/\Gamma)$-cocycles.
  A cocycle $\theta$ is a coboundary if and only if there exists a constant $c \in \cG^*\!/\Gamma$,
  such that $\theta = \Delta c$,
  with
  $$
    \Delta c : g \mapsto \Ad_*^\Gamma(g)(c) -c.
    $$
  Cocycles modulo coboundaries define a cohomology group denoted by $H^1(G,\cG^*\!/\Gamma)$.
  Every such cocycle $\theta$ defines a new action of $G$ on $\cG^*\!/\Gamma$ by
  $$
    \Ad^{\Gamma,\theta}_* : (g, \tau) \mapsto \Ad_*^\Gamma(g)(\tau) + \theta(g).
    $$
  The orbits for these actions will be called the {\em $(\Gamma,\theta)$-coadjoint orbits} of $G$.
  If $\Gamma = \{0\}$ we shall call them simply $\theta$-coadjoint orbits.
  If $\theta = 0$ we shall call them simply $\Gamma$-coadjoint orbits.
  And,
  if $\Gamma = \{0\}$ and $\theta = 0$ we find again the ordinary coadjoint orbits defined in \ref{Linear-coadjoint-action}.
  
  \article{Closed momenta of a diffeological group}
  \label{Closed-momenta-of-a-diffeological-group}
  Let $G$ be a diffeological group,
  and let $\cG^*$ be its space of momenta.
  Let us denote by $\cZ$ the subset of closed momenta of $G$,
  and by $\cB$ the subset of exact momenta of $G$.
  That is,
  $$
    \cZ = \Z^1_{\D\R}(G) \cap \cG^* \qmbox{and} \cB = \B^1_{\D\R}(G) \cap \cG^*.
    $$
  1) Let us assume that $G$ is connected,
  and let $\tilde G$ be its universal covering.
  By factorization,
  the chain-homotopy operator defines a canonical De Rham isomorphism $\ek$,
  from the space of closed momenta $\cZ$ to the vector space $\Hom^\infty(\tilde G,\RR)$.
  That is,
  for all $\zeta \in \cZ$,
  $$
    \ek(\zeta) = [\tilde g \mapsto \cK\zeta(p)], \qmbox{where}
    \cK\zeta(p) = \int_{p} \zeta
    \qmbox{and} \tilde g = \class(p).
    $$
  Here,
  we have denoted by $\class(p)$ the fixed ends homotopy class of the path $p \in \Paths(G,\id_G)$.
  The subspace of exact momenta $\cB$ identifies,
  through the isomorphism $\ek$,
  to the subspace $\Hom^\infty(G,\RR)$.
  $$
    \cZ \simeq \Hom^\infty(\tilde G,\RR)
    \qmbox{and}
    \cB \simeq \Hom^\infty(G,\RR).
    $$
  
  2) Let $G$ be any diffeological group connected or not.
  Let $\zeta \in \cG^*$,
  if $\zeta$ is closed then $\zeta$ is $\Ad_*$ invariant.
  $$
    \mbox{For all } \zeta \in \cG^*, \ d\zeta = 0 \
    \Rightarrow \ \Ad_*(g)(\zeta) = \zeta, \mbox{ for all } g \in G.
    $$
  {\sc Note} --- Every homomorphism from a diffeological group $G$ to an abelian group factorizes through the {\em abelianized group} $\Ab(G) = G/[G,G]$,
  where $[G,G]$ is the normal subgroup of the commutators of the group $G$.
  So actually,
  $\cZ \simeq \Hom^\infty(\Ab(\tilde G),\RR)$ and $\cB \simeq \Hom^\infty(\Ab(G),\RR)$.
  %
  \begin{proof}
    1) Let $\pi : \tilde G \to G$ be the universal covering defined in \ref{Covering-diffeological-groups}.
    Since $\tilde G$ is simply connected,
    every closed 1-form is exact \cite{Piz05}.
    Thus,
    for every $\zeta \in \cZ$,
    the pullback $\pi^*(\zeta)$ is exact.
    So,
    let $F$ be a primitive of $\pi^*(\alpha)$,
    that is $dF = \pi^*(\alpha)$.
    We can even fix uniquely $F$ by choosing $F(\id_{\tilde G}) = 0$.
    Actually $F$ is defined by integrating the form $\zeta$ along the paths starting at the identity,
    that is $F = \ek(\zeta)$.
    Since $\alpha$ is left-invariant and since the projection $\pi$ commutes with the left actions,
    on $G$ and $\tilde G$,
    $\pi^*(\alpha)$ is left invariant.
    So,
    for every $\tilde g \in \tilde G$,
    $d[F \circ \L(\tilde g)] = dF$.
    Since $\tilde G$ is connected,
    for every $\tilde g$, $\tilde g'$ in $\tilde G$,
    $F(\tilde g \tilde g') = F(\tilde g') + f(\tilde g)$.
    Where $f$ is a smooth real function.
    But since $F(\id_G) = 0$,
    $f(\tilde g) = F(\tilde g)$,
    and $F$ is a smooth homomorphism from $\tilde G$ to $\RR$.
    So,
    for every closed momentum $\zeta \in \cZ$,
    there exists a unique homomorphism $F \in \Hom^\infty(\tilde G, \RR)$ such that $\zeta = \pi_*(dF)$.
    The homomorphism $\ek$ is thus injective,
    and it is obviously surjective.
    Now,
    if $\zeta$ is exact,
    that is if $\zeta = df$,
    then $F = \pi^*(f)$.
    So,
    $\ek(\cB) = \pi^*(\Hom^\infty(G,\RR)) \simeq \Hom^\infty(G,\RR)$.
    
    2) Thanks to \ref{Momenta-and-connectedness} we can assume that $G$ is connected.
    Now,
    for every $\tilde g$, $\tilde g'$ in $\tilde G$,
    $F(\tilde g \tilde g' \tilde g^{-1}) = F(\tilde g')$.
    That is,
    $F \circ \Ad(\tilde g) = \Ad(\tilde g)^*(F) = F$,
    for all $\tilde g \in \tilde G$.
    So,
    $d[\Ad(\tilde g)^*(F)] = dF$,
    or $\Ad^*(\tilde g)(\pi^*(\zeta)) = \pi^*(\zeta)$,
    or $(\pi \circ \Ad(\tilde g))^*(\zeta) = \pi^*(\zeta)$.
    But $\pi \circ \Ad(\tilde g) = \Ad(g) \circ \pi$,
    where $g = \pi(\tilde g)$.
    So,
    $\pi^*(\Ad(g)^*(\zeta)) = \pi^*(\zeta)$.
    And since $\pi$ is a subduction,
    $\Ad(g)^*(\zeta) = \zeta$.
    That is,
    $\Ad_*(g)(\zeta) = \zeta$.
  \end{proof}
  
  \article{Equivalence between right and left momenta}
  \label{Equivalence-between-right-and-left-momenta}
  Let $G$ be a diffeological group,
  and let $\cG^\star$ denote the space of {\em right momenta} of the group $G$.
  That is,
  the space of 1-forms of $G$,
  invariant by the right multiplication.
  $$ \cG^\star = \{ \alpha \in \Omega^1(G) \mid \mbox{For all } g \in G, \ \R(g)^*(\alpha) = \alpha \}.
    $$
  There exists a natural linear isomorphism $\flip : \cG^* \to \cG^\star$ equivariant with respect to the coadjoint action.
  That is,
  the following diagram commutes.
  \[
  \begin{tikzcd}[column sep=large, row sep=large, every label/.append style = {font = \small}]
    \cG^* \arrow[r, "\flip"] \arrow[d, "\Ad_*(g)"'] & \cG^\star \arrow[d, "\Ad_*(g)"] \\
    \cG^* \arrow[r, "\flip"'] & \cG^\star
  \end{tikzcd}
  \]
  In other words,
  there is no reason to prefer left or right momenta of a diffeological group.
  The particularization of left momenta comes because we are dealing with actions of groups and not anti-actions.
  
  \begin{proof}
    Let us denote by a dot the multiplication in $G$.
    Let $\alpha$ be any left $p$-momentum of $G$.
    Let $P : U \to G$ be a $n$-plot.
    Let $\bar \alpha(P)$ be defined by
    $$ \bar \alpha(P)(r) = \alpha\left[s \mapsto P (s) \cdot P (r)^{-1}\right](s=r).
      $$
    where $r$ belongs to $U$.
    Let us show that $\bar{\alpha}$ defines a $p$-form of $G$.
    First of all let us remark that $\bar \alpha(P)$ is the restriction of the 1-form $\alpha((s,r) \mapsto P (s)\cdot P (r)^{-1})$ to the diagonal $s=r$.
    Thus,
    $\bar \alpha(P)$ is a smooth 1-form of $U$.
    
    Now,
    let us prove that $\bar\alpha$ is a well defined 1-form on $G$,
    according to the definition of differential forms in diffeology.
    let $F : V \to U$ be a smooth $m$-parametrization.
    Let $v$ be a point of $V$,
    and $\delta v$ be a vector of $\RR^m$.
    We have:
    \begin{eqnarray*}
      \bar{\alpha}(P \circ F)_v(\delta v)
      & = & \alpha\left[s \mapsto (P \circ F)(s) \cdot (P \circ F)(v)^{-1}\right]_v(\delta v) \\
      & = & \alpha\left[s \mapsto F(s) \mapsto (P \circ F)(s) \cdot (P \circ F)(v)^{-1}\right]_v(\delta v) \\
      & = & \alpha\left[s \mapsto r = F(s) \mapsto P (r) \cdot P (F(v))^{-1}\right]_v(\delta v) \\
      & = & \alpha\left(\left[r \mapsto P (r) \cdot P (F(v))^{-1}\right] \circ F\right)_v(\delta v) \\
      & = & {F}^*\left[\alpha\left(r \mapsto P (r) \cdot P (F(v))^{-1}\right)\right]_v(\delta v) \\
      & = & \alpha\left[r \mapsto P (r) \cdot P (F(v))^{-1}\right]_{F(v)} (\D(F)(v)(\delta v)) \\
      & = & \bar{\alpha}(P)_{F(v)}(\D(F)(v)(\delta v)) \\
      & = & {F}^*\left[\bar \alpha(P)\right]_v(\delta v).
    \end{eqnarray*}
    Then,
    let us check that $\bar{\alpha}$ is right-invariant,
    that is $\bar \alpha \in \cG^\star$.
    For all $g \in G$,
    we have:
    \begin{eqnarray*}
      \R(g)^*(\bar\alpha)(P)_r(\delta r)
      & = & \bar\alpha(\R(g) \circ P)_r(\delta r) \\
      & = & \alpha\left[s \mapsto (\R(g) \circ P)(s) \cdot (\R(g) \circ P)(r)^{-1}\right]_r(\delta r) \\
      & = & \alpha\left[s \mapsto P (s) \cdot g \cdot (P (r) \cdot g)^{-1}\right]_r(\delta r) \\
      & = & \alpha\left[s \mapsto P (s) \cdot g \cdot g^{-1} \cdot P (r)^{-1}\right]_r(\delta r) \\
      & = & \alpha\left[s \mapsto P (s) \cdot P (r)^{-1}\right]_r(\delta r) \\
      & = & \bar{\alpha}(P)_r(\delta r)
    \end{eqnarray*}
    So,
    we have defined a map $\flip : \alpha \mapsto \bar \alpha$,
    from $\cG^*$ to $\cG^\star$.
    Let us prove now that $\flip$ is bijective.
    Let $\beta = \bar \alpha$.
    Let $P : U \to G$ be a plot,
    and let us define $\bar \beta$ by
    $$\bar \beta(P)(r) = \beta [s \mapsto P (r)^{-1} \cdot P (s)](s=r),
      $$
    for all $r \in U$.
    So,
    we have:
    \begin{eqnarray*}
      \bar\beta(P)(r)
      & = & \beta \left[s \mapsto P (r)^{-1} \cdot P (s)\right](s=r) \\
      & = & \bar \alpha \left[s \mapsto P (r)^{-1} \cdot P (s)\right](s=r) \\
      & = & \alpha \left[s \mapsto P (r)^{-1} \cdot P (s) \cdot P (r)^{-1} \cdot P (r) \right](s=r) \\
      & = & \alpha \left[s \mapsto P (r)^{-1} \cdot P (s) \right](s=r) \\
      & = & \L(P (r)^{-1})^*(\alpha) \left[s \mapsto P (s) \right](s=r) \\
      & = & \alpha(P)(r).
    \end{eqnarray*}
    Hence,
    $\bar \beta = \alpha$.
    Thus,
    $\flip$ is bijective.
    And,
    $\flip$ is clearly linear.
    Therefore,
    $\flip$ is a linear isomorphism from $\cG^*$ to $\cG^\star$.
    It is easy to check that it is a smooth isomorphism.
    
    Finally,
    let us check that $\flip$ is equivariant under the coadjoint action.
    Let $\alpha \in \cG^*$,
    let $\P: U \to G$ be a plot and $r \in U$.
    On one hand we have,
    \begin{eqnarray*}
      \flip[\Ad(g)^*(\alpha)](P)_r &=& \flip[\R(g)^*(\alpha)](P)_r \\
      & = & \R(g)^*(\alpha)[s \mapsto P (s) \cdot P (r)^{-1}]_r\\
      & = & \alpha(s \mapsto P (s) \cdot P (r)^{-1} \cdot g)_r.
    \end{eqnarray*}
    And,
    on the other hand:
    \begin{eqnarray*}
      [\Ad(g)^*(\flip(\alpha))](P)_r & = & [\L(g)_*(\flip(\alpha))](P)_r \\
      & = & \flip(\alpha)(\L(g^{-1}) \circ P)_r \\
      & = & \alpha[s \mapsto (\L(g^{-1}) \circ P)(s) \cdot (\L(g^{-1}) \circ P)(r))^{-1}]_r \\
      & = & \alpha[s \mapsto g^{-1} \cdot P (s) \cdot P (r)^{-1} \cdot g]_r \\
      & = & \L(g^{-1})^*(\alpha)[s \mapsto P (s) \cdot P (r)^{-1} \cdot g]_r \\
      & = & \alpha[s \mapsto P (s) \cdot P (r)^{-1} \cdot g]_r
    \end{eqnarray*}
    Therefore,
    $\flip \circ \Ad(g)^* = \Ad(g)^* \circ \flip$ for all $g \in G$.
  \end{proof}
  
  %%%%%%%%%%%%%%%%%%%%%%%%%%%%%%%%%%%%%%%%%%%%%%%%%%%%%%%%%%
  \section{The paths moment map}
  %%%%%%%%%%%%%%%%%%%%%%%%%%%%%%%%%%%%%%%%%%%%%%%%%%%%%%%%%%
  
  We shall now introduce the notion of moment map step by step.
  The first step consists to define the {\em paths moment map}.
  
  \article{Definition of the paths moment map}
  \label{Definition-of-the-paths-moment-map}
  Let $X$ be a diffeological space and $\omega$ be a closed 2-form defined on $X$.
  Let $G$ be a diffeological group and $\rho : G \to \Diff(X)$ be a smooth action.
  Let us denote by the same letter the natural action of $G$ on $\Paths(X)$,
  induced by the action $\rho$ of $G$ on $X$.
  That is,
  for all $g \in G$,
  for all $p \in \Paths(X)$,
  $$
    \rho(g)(p) = \rho(g) \circ p = [t \mapsto \rho(g)(p(t))].
    $$
  Let us assume now that the action $\rho$ of $G$ on $X$ preserves $\omega$.
  That is,
  for all $g \in G$,
  $$
    \rho(g)^*(\omega) = \omega \qmbox{or} \rho \in \Hom^\infty(G, \Diff(X,\omega)).
    $$
  Let $\cK$ be the chain-homotopy operator,
  so $\cK \omega$ is a 1-form of $\Paths(X)$,
  and the action of $G$ on $\Paths(X)$ preserves the 1-form $\cK \omega$.
  This is a consequence of the variance of the chain-homotopy operator,
  see \ref{Chain-Homotopy-operator}.
  Thus,
  for all $g \in G$,
  $$
    \rho(g)^*(\cK \omega) = \cK \omega.
    $$
  Now,
  let $p$ be any paths of $X$,
  and let $\hat p : G \to \Paths(X)$ be the orbit map.
  So,
  the pullback $\hat p^*(\cK \omega)$ is a left-invariant 1-form of $G$,
  that is an element of $\cG^*$.
  The map
  $$
    \Psi : \Paths(X) \to \cG^* \qmbox{defined by} \Psi(p) = \hat p^*(\cK \omega),
    $$
  is smooth with respect to the functional diffeology,
  $\Psi \in \Cinfty(\Paths(X), \cG^*)$.
  The map $\Psi$ will be called the {\em paths moment map}.
  
  \article{Evaluation of the paths moment map}
  \label{Evaluation-of-the-paths-moment-map}
  Let $X$ be a diffeological space and $\omega$ be a closed 2-form defined on $X$.
  Let $G$ be a diffeological group and $\rho$ be a smooth action of $G$ on $X$,
  preserving $\omega$.
  Let $p$ be a path in $X$.
  Thanks to the explicit expression of the chain-homotopy operator given in \ref{Chain-Homotopy-operator},
  we get the evaluation of the momentum $\Psi(p)$ on any $n$-plot $P$ of $G$,
  \begin{equation}
    \label{eq:heartsuit}
    \Psi(p)(P)_r(\delta r) =
    \int_0^1 \omega \left[ \mymatrix{s \cr u} \mapsto
    (\rho \circ P)(u) (p(s + t)) \right]_{\left({s=0 \atop
    u=r}\right)}\mymatrix{1 \cr 0} \mymatrix{0 \cr \delta r} dt,
  \end{equation}
  for all $r$ in $\dom(P)$ and all $\delta r$ in $\RR^n$.
  Now,
  as a differential 1-form,
  $\Psi(p)$ is characterized by its values on the $1$-plots \cite{Piz05}.
  So,
  let $f : t \mapsto f_t$ be a $1$-plot of $G$ centered at the identity $\id_G$,
  that is $f \in \Paths(G)$ and $f(0) = \id_{G}$.
  For any $t \in \RR$,
  let $F_t$ be the path in $\Diff(X,\omega)$ --- centered at the identity $\id_X$ --- defined by
  $$
    F_t : s \mapsto \rho( f_t^{-1} \circ f_{t+s}).
    $$
  So,
  we have
  \begin{equation}
    \label{eq:clubsuit}
    \Psi(p)(f)_t(1) = - \int_p i_{F_t}(\omega) = - \int_0^1 i_{F_t}(\omega)(p)_s(1) ds,
  \end{equation}
  where $i_{F_t}(\omega) $ is the contraction of $\omega$ by $F_t$,
  see \ref{Differential-forms}.
  
  But,
  as an invariant 1-form on $G$ the moment $\Psi(p)$ is characterized by its {\em value at the identity},
  that is for $t=0$,
  \begin{equation}
    \label{eq:diamondsuit}
    \Psi(p)(f)_0(1) = - \int_p i_F(\omega) = - \int_0^1 i_F(\omega)(p)_t(1) \ dt \qmbox{with} F = \rho \circ f.
  \end{equation}
  
  {\sc Note} --- Let $f \in \Hom^\infty(\RR,G)$,
  so $\Psi(p)(f)$ is an invariant $1$-form on $\RR$ whose coefficient is just $\int_p i_F(\omega)$.
  That is,
  $$
    \Psi(p)(f) = h_f(p) \times dt \qmbox{where} h_f(p) = - \int_p i_F(\omega).
    $$
  The smooth map $h_f : \Paths(X) \to \RR$ is the {\em hamiltonian} of $f$,
  or the hamiltonian of the 1-parameter group $f(\RR)$.
  Note also that,
  the map $h : \Hom^\infty(\RR,G) \to \Cinfty(\Paths(X),\RR)$,
  defined above,
  is smooth.
  
  \begin{proof}
    Let us prove (\ref{eq:heartsuit}).
    Let us remind that for every $p \in \Paths(X)$ and every $g \in G$,
    $\hat p (g) = \rho(g)(p) = [t \mapsto \rho(g)(p(t))]$.
    So,
    by definition
    \begin{eqnarray*}
      \Psi(p)(P)_r(\delta r) &=& \hat p^*(\cK\omega)_r(\delta r) \\
      &=& \cK\omega(\hat p \circ P)_r(\delta r) \\
      &=& \int_0^1 \omega \bigg[\mymatrix{s \cr r} \mapsto \hat p \circ P (r)(s+t) \bigg]_{\left({0 \atop r}\right)} \mymatrix{1 \cr 0} \mymatrix{0 \cr \delta r} dt \\
      &=& \int_0^1 \omega \bigg[\mymatrix{s \cr r} \mapsto (\rho \circ P)(r)(p(s+t)) \bigg]_{\left({0 \atop r}\right)} \mymatrix{1 \cr 0} \mymatrix{0 \cr \delta r} dt.
    \end{eqnarray*}
    Let us prove (\ref{eq:clubsuit}).
    Let us apply the general formula (\ref{eq:heartsuit}) for $P = f$.
    Introducing $u' = u-t$ and $s'' = s+s'$,
    using the compatibility property of $\omega(P \circ Q) = Q^*(\omega(P))$ and the $\rho(f_t)$ invariance of $\omega$,
    we get
    \begin{eqnarray*}
      \Psi(p)(f)_t(1) &=& \int_0^1 \omega \left[ \mymatrix{s \cr u} \mapsto \rho(f_u)(p(s + s')) \right]_{\left({s=0 \atop u=t}\right)}\mymatrix{1 \cr 0} \mymatrix{0 \cr 1} ds' \\
      &=& \int_0^1 \omega \left[ \mymatrix{s'' \cr u'} \mapsto \rho(f_{t+u'})(p(s'')) \right]_{\left({s''=s' \atop u'=0}\right)}\mymatrix{1 \cr 0} \mymatrix{0 \cr 1} ds' \\
      &=& \int_0^1 \omega \left[ \mymatrix{s'' \cr u'} \mapsto \rho(f_t \circ f_t^{-1} \circ f_{t+ u'})(p(s'')) \right]_{\left({s''= s' \atop u'=0}\right)}\mymatrix{1 \cr 0} \mymatrix{0 \cr 1} ds' \\
      &=& \int_0^1 \omega \left[ \mymatrix{s'' \cr u'} \mapsto \rho(f_t) \bigg(F_t(u')(p(s''))\bigg) \right]_{\left({s''=s' \atop u'=0}\right)}\mymatrix{1 \cr 0} \mymatrix{0 \cr 1} ds' \\
      &=& \int_0^1 \omega \left[ \mymatrix{s'' \cr u'} \mapsto F_t(u')(p(s'')) \right]_{\left({s''=s' \atop u'=0}\right)}\mymatrix{1 \cr 0} \mymatrix{0 \cr 1} ds' \\
      &=& \int_0^1 \omega \left[ \mymatrix{u' \cr s''} \mapsto F_t(u')(p(s'')) \right]_{\left({u'=0 \atop s''=s'}\right)}\mymatrix{0 \cr 1} \mymatrix{1 \cr 0} ds' \\
      &=& - \int_0^1 \omega \left[ \mymatrix{u' \cr s''} \mapsto F_t(u')(p(s'')) \right]_{\left({u'=0 \atop s''=s'}\right)}\mymatrix{1 \cr 0} \mymatrix{0 \cr 1} ds' \\
      &=& - \int_0^1 i_{F_t}(\omega)(p)_{s'}(1) ds' \\
      &=& - \int_p i_{F_t}(\omega).
    \end{eqnarray*}
    Let us prove the Note.
    Let $f \in \Hom^\infty(\RR,G)$.
    By definition of differential forms and pullbacks,
    $\Psi(p)(f) = f^*(\Psi(p))$,
    but since $f$ is an homomorphism from $\RR$ to $\Diff(X,\omega)$ and $\Psi(p)$ is a left-invariant 1-form on $\Diff(X,\omega)$,
    $f^*(\Psi(p))$ is an invariant 1-form of $\RR$,
    so $\Psi(p)(f) = f^*(\Psi(p)) = a \times dt$,
    for some real $a$.
    So,
    $\Psi(p)(f)_r = \Psi(p)(f)_0(1) \times dt = h_f(p) \times dt$,
    with $h_f(p) = \Psi(p)(f)_0(1) = -\int_p i_F(\omega)$,
    and $dt$ is the canonical 1-form on $\RR$.
  \end{proof}
  
  \article{Variance of the paths moment map}
  \label{Variance-of-the-paths-moment-map}
  Let $X$ be a diffeological space and $\omega$ be a closed 2-form defined on $X$.
  Let $G$ be a diffeological group and $\rho$ be a smooth action of $G$ on $X$,
  preserving $\omega.$ The paths moment map $\Psi$,
  defined in \ref{Definition-of-the-paths-moment-map},
  is equivariant under the action of $G$.
  That is,
  for all $g \in G$,
  $$
    \Psi \circ \rho(g)_* = \Ad(g)_* \circ \Psi.
    $$
  
  \begin{proof}
    Let us denote here the orbit map $\hat p$ of every path $p \in \Paths(X)$ by $\L(p)$.
    That is,
    $\L(p)(g) = \rho(g)_*(p) = \rho(g) \circ p$.
    So,
    $\Psi(\rho(g)_*(p)) = \Psi(\rho(g) \circ p) = (\L(\rho(g) \circ p)^*(\cK\omega)$.
    But,
    $\L(\rho(g) \circ p)(g') = \rho(g')(\rho(g) \circ p) = \rho(g'g)\circ p = \L(p)(g'g) = \L(p)\circ \R(g)(g')$.
    Thus,
    $\L(\rho(g) \circ p) = \L(p)\circ \R(g)$,
    and $\Psi(\rho(g)_*(p)) = (\L(p) \circ \R(g))^*(\cK\omega) = \R(g)^*(\L(p)^*(\cK(p)) = \R(g)^*(\Psi(p))$.
    But since $\Psi(p)$ is left-invariant,
    $\R(g)^*(\Psi(p)) = \Ad(g)_*(\Psi(p))$,
    and $\Psi(\rho(g)_*(p)) = \Ad(g)_*(\Psi(p))$.
  \end{proof}
  
  \article{Additivity of the paths moment map}
  \label{Additivity-of-the-paths-moment-map}
  Let $X$ be a diffeological space and $\omega$ be a closed 2-form defined on $X$.
  Let $G$ be a diffeological group and $\rho$ be a smooth action of $G$ on $X$,
  preserving $\omega$.
  The paths moment map $\Psi$,
  defined in \ref{Definition-of-the-paths-moment-map},
  satisfies the following additive property:
  for any two juxtaposable paths $p$ and $p'$ in $X$,
  $$
    \Psi( p \vee p') = \Psi(p) + \Psi(p') \qmbox{and} \Psi(\bar p) = - \Psi(p), \qmbox{with} \bar p (t) = p(1-t).
    $$
  
  \begin{proof}
    This is a direct application of the expression given in \ref{Evaluation-of-the-paths-moment-map} (\ref{eq:diamondsuit}),
    and of the additivity of the integral of differential form on paths.
  \end{proof}
  
  \article{Differential of the paths moment map}
  \label{Differential-of-a-paths-momentum}
  Let $X$ be a diffeological space and $\omega$ be a closed 2-form defined on $X$.
  Let $G$ be a diffeological group and $\rho$ be a smooth action of $G$ on $X$,
  preserving $\omega$.
  Let $p$ be a path in $X$.
  So,
  the exterior differential of the paths momentum $\Psi(p)$ is given by
  $$
    d(\Psi(p)) = \hat x_1^*(\omega) - \hat x_0^*(\omega),
    $$
  where $x_0=p(0)$ and $x_1 = p(1)$,
  and the $\hat x_i$ denote the orbit maps.
  
  \begin{proof}
    This is a direct application of the main property of the chain-homotopy operator,
    $d \circ \cK + \cK \circ d = \but^* - \source^*$.
    Since $d\omega = 0$,
    we have $d(\cK \omega) = \but^*(\omega) - \source^*(\omega)$,
    composed with $\hat p^*$,
    we get $\hat p^* \circ d(\cK \omega) = \hat p^* \circ \but^*(\omega) - \hat p^* \circ \source^*(\omega)$.
    That is $d(\hat p^*(\cK \omega)) = (\but \circ \hat p)^*(\omega) - (\source \circ \hat p)^*(\omega)$.
    Thus,
    $d(\Psi(p)) = \hat x_1^*(\omega) - \hat x_0^*(\omega)$.
  \end{proof}
  
  \article{Homotopic invariance of the paths moment map}
  \label{Homotopic-invariance-of-the-paths-moment-map}
  Let $X$ be a diffeological space and $\omega$ be a closed 2-form defined on $X$.
  Let $G$ be a diffeological group and $\rho$ be a smooth action of $G$ on $X$,
  preserving $\omega$.
  Let $p_0$ and $p_1$ be any two paths in $X$.
  If $p_0$ and $p_1$ are fixed ends homotopic,
  then $\Psi(p_0) = \Psi(p_1)$.
  
  \begin{proof}
    Let $s \mapsto p_s$ be a fixed ends homotopy connecting $p_0$ to $p_1$,
    for example let $p_s(0)=x_0$ and $p_s(1) = x_1$,
    for all $s$.
    Let $f$ be a 1-plot of $G$ centered at the identity $\id_G$,
    that is $f(0) = \id_G$,
    and let $F = \rho \circ f$.
    We use the fact that the moment of paths is characterized by its value at the identity,
    $\Psi(p_s)(f)_0(1) = -\displaystyle\int_{p_s} i_F(\omega)$,
    see \ref{Evaluation-of-the-paths-moment-map} (\ref{eq:diamondsuit}).
    Let us differentiate this equality with respect to $s$,
    $$
      {\partial \over \partial s}\bigg(\Psi(p_s)(f)_0(1)\bigg) = - \delta \int_{p_s} i_F (\omega), \qmbox{with} \delta = {\partial \over \partial s}.
      $$
    The variation of the integral of differential forms on chains gives
    $$
      \delta \int_{p_s} i_F (\omega) =
      \int_0^1 d\,[i_F(\omega)] ( \delta p_s ) +
      \bigg[ i_F(\omega) ( \delta
      p_s)\bigg]_{\raisebox{1pt}{\scriptsize
      0}}^{\raisebox{-2pt}{\scriptsize
      1}}.
      $$
    See \cite{Piz05} for the definition of $\delta p_s$ and for the proof of this formula in diffeology.
    Since the homotopy $s \mapsto p_s$ is a fixed end homotopy,
    $\delta p_s(0) = 0$ and $\delta p_s(1) = 0$,
    thus the second summand of the right term vanishes.
    Now,
    the Cartan formula writes $\DLie_F(\omega) = d[i_F(\omega)]+i_F(d\omega)$,
    see \ref{Differential-forms}.
    But $\omega$ is invariant under the action of $G$,
    so $\DLie_F(\omega) =0$,
    and since $d\omega = 0$ we get $d[i_F(\omega)] = \DLie_F(\omega) = 0$.
    So,
    $\delta {\displaystyle \int_{p_s}} i_F (\omega) = 0$ and $\Psi(p_0) = \Psi(p_s) = \Psi(p_1)$,
    for all $s$.
  \end{proof}
  
  \article{The holonomy group}
  \label{The-holonomy-group}
  Let $X$ be a connected diffeological space,
  and let $\omega$ be a closed 2-form defined on $X$.
  Let $G$ be a diffeological group and $\rho$ be a smooth action of $G$ on $X$,
  preserving $\omega$.
  Let $\Psi$ be the paths moment map defined in \ref{Definition-of-the-paths-moment-map}.
  We define the {\em holonomy} $\Gamma$ of the action $\rho$ as
  $$
    \Gamma = \{ \Psi(\ell) \mid \ell \in \Loops(X) \}.
    $$
  \begin{enumerate}
    \item The holonomy $\Gamma$ is an additive subgroup of the subspace of closed momenta, $\Gamma \subset \cZ$ (see \ref{Closed-momenta-of-a-diffeological-group}). That is, for every elements $\gamma$ and $\gamma'$ of $\Gamma$,
    $$
      d\gamma = 0 \qmbox{and} \gamma - \gamma' \in \Gamma.
      $$
    \item The paths moment map $\Psi$, restricted to $\Loops(X)$, factorizes through an homomorphism from $\pi_1(X)$ to $\cG^*$. Thus, $\Gamma$ is an homomorphic image of $\pi_1(X)$, or its abelianized $\Ab(\pi_1(X))$.
    \item In particular, every element $\gamma$ of $\Gamma$ is invariant by the coadjoint action of $G$ on $\cG^*$. For all $g$ in $G$,
    $$
      \Ad_*(g)(\gamma) = \gamma.
      $$
  \end{enumerate}
  The holonomy $\Gamma$ is the obstruction for the action $\rho$ to be ``hamiltonian''.
  Precisely,
  the action of $G$ on $X$ will be said to be {\em hamiltonian} if and only if $\Gamma = \{0\}$.
  Note that,
  if the group $G$ has no $\Ad_*$-invariant 1-forms except 0,
  the action $\rho$ is necessarily hamiltonian,
  see \ref{Closed-momenta-of-a-diffeological-group}.
  
  \begin{proof}
    We get immediately that $\gamma \in \Gamma$ is closed,
    by application of the differential of a path momentum:
    for all path $p \in \Paths(X)$,
    $d(\Psi(p)) = \hat x_1^*(\omega) -\hat x_0^*(\omega)$,
    where $x_0=p(0)$ and $x_1=p(1)$,
    see \ref{Differential-of-a-paths-momentum}.
    So,
    for any loop $\ell$ of $X$,
    $\ell(0) = \ell(1)$ and $d(\Psi(\ell)) = 0$.
    Now,
    let $x_0$ be any point of $X$.
    Thanks to \ref{Homotopic-invariance-of-the-paths-moment-map},
    for every loop $\ell \in \Loops(X,x_0)$,
    the momentum $\Psi(\ell)$ depends on $\ell$ only through the its homotopy class.
    So $\Gamma$ is the image of $\pi_1(X,x_0)$.
    And,
    thanks to the additive property of $\Psi$,
    see \ref{Additivity-of-the-paths-moment-map},
    the map $\class(\ell) \mapsto \Psi(\ell)$ is an homomorphism.
    Now,
    since $X$ is connected,
    for every other point $x_1$ of $X$,
    there exists a path $c$ connecting $x_0$ to $x_1$,
    and let $\bar c = t \mapsto c(1-t)$.
    Thanks to the additive property,
    $\Psi(\bar c \vee \ell \vee c) = \Psi(\bar c) + \Psi(\ell) + \Psi(c) = - \Psi(c) + \Psi(\ell) + \Psi(c) = \Psi(\ell)$.
    And,
    since the map $\class(\ell) \mapsto \class(\bar c \vee \ell \vee c)$ is a conjugation from $\pi_1(X,x_0)$ to $\pi_1(X,x_1)$,
    $\Gamma$ is the same homomorphic image of $\pi_1(X,x)$,
    for every point $x \in X$.
    So,
    we proved the points 1 and 2,
    the third one is a direct consequence of \ref{Closed-momenta-of-a-diffeological-group}.
  \end{proof}
  
  %%%%%%%%%%%%%%%%%%%%%%%%%%%%%%%%%%%%%%%%%%%%%%%%%%%%%%%%%%
  \section{The 2-points moment map}
  %%%%%%%%%%%%%%%%%%%%%%%%%%%%%%%%%%%%%%%%%%%%%%%%%%%%%%%%%%
  
  The definition of the paths moment map leads immediately to the {\em $2$-points moment map}.
  The 2-points moment map satisfies a cocycle condition inherited from the additive property of the paths moment map.
  This is the second step in our general construction.
  
  \article{Definition of the 2-points moment map}
  \label{Definition-of-the-2-points-moment-map}
  Let $X$ be a connected diffeological space and $\omega$ be a closed 2-form defined on $X$.
  Let $G$ be a diffeological group and $\rho$ be a smooth action of $G$ on $X$,
  preserving $\omega$.
  Let $\Psi$ be the paths moment map and $\Gamma$ be the holonomy of the action $\rho$,
  see \ref{Definition-of-the-paths-moment-map} and \ref{The-holonomy-group}.
  So,
  there exists a smooth map $\psi : X \times X \to \cG^* / \Gamma$ such that the following diagram commutes.
  \[
  \begin{tikzcd}
    \Paths(X) \arrow[r, "\Psi"] \arrow[d, "\bounds"'] & \cG^* \arrow[d, "\pr"] \\
    X \times X \arrow[r, "\psi"'] & \cG^*/\Gamma
  \end{tikzcd}
  \]
  where $\pr$ is the canonical projection from $\cG^*$ onto its quotient,
  and $\bounds = \source \times \but$,
  that is $\bounds(p) = (p(0),p(1))$.
  The map $\psi \in \Cinfty(X \times X, \cG^*\!/\Gamma)$ will be called the {\em 2-points moment map}.
  \begin{enumerate}
    \item The 2-points moment map $\psi$ satisfies the Chasles cocycle relation, for any three points $x$, $x'$, $x''$ of $X$,
    \begin{equation}
      \label{eq:heartsuit2}
      \psi(x,x'') = \psi(x,x') + \psi(x',x'') .
    \end{equation}
    \item The 2-points moment map $\psi$ is equivariant under the action of $G$. That is, for any $g \in G$, and any pair of points $x$ and $x'$ of $X$,
    $$
      \psi(\rho(g)(x), \rho(g)(x')) =
      \Ad_*^\Gamma(g)(\psi(x,x')).
      $$
  \end{enumerate}
  {\sc Note.} T. Ratiu and A. Weinstein have kindly pointed out that Condevaux,
  Dazord and Molino \cite{CDM88} proposed a similar construction in the case where $X$ is a manifold,
  $G$ is a Lie group,
  and $\Gamma$ is closed in $\cG^*$.
  
  \begin{proof}
    By construction $\psi$ is defined by $\psi(x,x') = \class_\Gamma(\Psi(p))$,
    where $p \in \Paths(X)$,
    $x = p(0)$,
    $x' = p(1)$,
    and $\class_\Gamma(\alpha)$ denotes the class of $\alpha \in \cG^*$ in $\cG^*\!/\Gamma$.
    The map $\psi$ is smooth simply by general properties of subductions in diffeology.
    Now,
    the first point is a direct consequence of the additive property of the paths moment map,
    see \ref{Additivity-of-the-paths-moment-map}.
    The second point is a direct consequence of the equivariance of the paths moment map of the $\Ad_*$ invariance of $\Gamma$,
    see \ref{Variance-of-the-paths-moment-map},
    and of the definition of the $\Ad_*^\Gamma$ action,
    see \ref{Affine-coadjoint-actions-and-orbits}.
  \end{proof}
  
  %%%%%%%%%%%%%%%%%%%%%%%%%%%%%%%%%%%%%%%%%%%%%%%%%%%%%%%%%%
  \section{The moment maps}
  %%%%%%%%%%%%%%%%%%%%%%%%%%%%%%%%%%%%%%%%%%%%%%%%%%%%%%%%%%
  
  From the construction of the paths moment map of \ref{Definition-of-the-paths-moment-map} and the 2-points moment map of \ref{Definition-of-the-2-points-moment-map} we get the notion of (1-point) moment map.
  This is the third step of our general construction,
  and the generalization of the notion of moment map coming from classical symplectic geometry.
  
  \article{Definition of the moment maps}
  \label{Definition-of-the-moment-maps}
  Let $X$ be a connected diffeological space and let $\omega$ be a closed 2-form defined on $X$.
  Let $G$ be a diffeological group and $\rho$ be a smooth action of $G$ on $X$,
  preserving $\omega$.
  Let $\psi$ be the 2-points moment map defined in \ref{Definition-of-the-2-points-moment-map}.
  There exists always a smooth map $\mu : X \to \cG^*\!/\Gamma$,
  called a {\em primitive} of $\psi$,
  such that,
  for any two points $x$ and $x'$ of $X$,
  $$
    \psi(x,x') = \mu(x') - \mu(x).
    $$
  For every point $x_0 \in X$,
  for every constant $c \in \cG^*\!/\Gamma$,
  the map $\mu$ defined by
  $$
    \mu(x) = \psi(x_0,x) + c.
    $$
  is a primitive of $\psi$.
  Every primitive $\mu$ of $\psi$ is of this kind,
  and any two primitive $\mu$ and $\mu'$ of $\psi$ differ only by a constant.
  
  The 2-points moment map $\psi$ will be said to be {\em exact} if there exists a primitive $\mu$,
  {\em equivariant} by the action of $G$.
  That is,
  if there exists a primitive $\mu$ such that
  $$
    \mu \circ \rho(g) = \Ad_*^\Gamma(g) \circ \mu,
    $$
  for all $g \in G$.
  The primitives $\mu$ of $\psi$,
  equivariant or not,
  will be called the {\em moment maps}\footnote{These maps should have been called the {\em 1-point moment maps\/}. But to conform with the usual denomination we chose to call them simply {\em moment maps}.}.
  
  {\sc Note} --- By the identity (\ref{eq:heartsuit2}) of \ref{Definition-of-the-2-points-moment-map},
  $\psi$ is a $1$-cocycle of the $G$-equivariant cohomology of $X$ with coefficients in $\cG^*\!/\Gamma$,
  twisted by the coadjoint action.
  Two cocycles $\psi$ and $\psi'$ are cohomologous if and only if,
  there exists a smooth equivariant map $\mu : X \to \cG^*\!/\Gamma$,
  such that $\psi'(x,x') = \psi(x,x') + \Delta \mu(x,x')$ where $\Delta \mu(x,x') = \mu(x') - \mu(x)$,
  $\Delta \mu$ is a coboundary.
  So,
  the 2-points moment map $\psi$ defines a class belonging to $H^1_G(X,\cG^*\!/\Gamma)$ which depends only on the form $\omega$ and the action $\rho$ of $G$ on $X$.
  If the moment map $\psi$ is exact,
  that is if $\class(\psi) = 0$,
  we shall say that the action $\rho$ of $G$ on $X$ is {\em exact},
  with respect to $\omega$.
  In this case,
  there exists a point $x_0$ of $X$ and a constant $c$ such that $\mu : x \mapsto \psi(x_0,x) + c$ is an equivariant primitive for $\psi$.
  
  \begin{proof}
    Let $x_0$ be a chosen point of $X$.
    Since $X$ is connected,
    for any $x \in X$ there exists always a path $p \in X$ such that $p(0) = x_0$ and $p(1) = x$.
    Thus,
    defining $\mu(x) = \psi(x_0,x) = \class(\Psi(p))$,
    and thanks to the cocycle properties of $\psi$,
    we have $\psi(x,x') = \psi(x,x_0) + \psi(x_0,x') = \psi(x_0,x') - \psi(x_0,x) = \mu(x') - \mu(x)$.
    Now,
    since $\psi$ is smooth,
    $\mu$ is smooth.
    Therefore,
    the equation $\psi(x',x) = \mu(x') - \mu(x)$ has always a solution in $\mu$.
    
    Now,
    let $\mu$ and $\mu'$ be two primitives of $\psi$.
    For each pair $x$, $x'$ of points of $X$ we have $\mu'(x') - \mu'(x) = \mu(x') - \mu(x)$.
    That is,
    $\mu'(x') - \mu(x') = \mu'(x) - \mu(x)$.
    So,
    the map $x \mapsto \mu'(x) - \mu(x)$ is constant.
    There exists $c \in \cG^*\!/\Gamma$ such that $\mu'(x) - \mu(x) = c$,
    that is $\mu'(x) = \mu(x) + c$.
    
    Since,
    the maps $x \mapsto \psi(x_0,x)$,
    where $x_0$ is a fixed point of $X$,
    is a special solution of the equation in $\mu$,
    $\psi(x',x) = \mu(x') - \mu(x)$,
    any solution writes $\mu(x) = \psi(x_0,x) + c$ for some point $x_0 \in X$ and some constant $c \in \cG^*\!/\Gamma$.
  \end{proof}
  
  \article{Souriau's cocycles}
  \label{Souriau-cocycles}
  Let $X$ be a connected diffeological space and $\omega$ be a closed 2-form defined on $X$.
  Let $G$ be a diffeological group and $\rho$ be a smooth action of $G$ on $X$,
  preserving $\omega$.
  Let $\psi$ be the 2-points moment map defined in \ref{Definition-of-the-2-points-moment-map} and let $\mu$ be a primitive of $\psi$ as defined in \ref{Definition-of-the-moment-maps}.
  So there exists a map $\theta \in \Cinfty(G,\cG^*\!/\Gamma)$ such that
  $$
    \mu(\rho(g)(x)) = \Ad_*^\Gamma(g)(\mu(x)) + \theta(g).
    $$
  The map $\theta$ is a $(\cG^*\!/\Gamma)$-cocycle,
  as defined in \ref{Affine-coadjoint-actions-and-orbits}.
  For all $g,g' \in G$,
  $$
    \theta(gg') = \Ad_*^\Gamma(g)(\theta(g')) + \theta(g).
    $$
  We shall call the cocycle $\theta$,
  {\em Souriau's cocycle} of the moment $\mu$.
  \begin{enumerate}
    \item Two Souriau's cocycles $\theta$ and $\theta'$, associated to two moment maps $\mu$ and $\mu'$ are {\em cohomologous\/}.
    That is, they differ by a {\em coboundary}
    $$
      \Delta c : g \mapsto \Ad_*^\Gamma(g)(c) -c, \qmbox{where} c \in \cG^*\!/\Gamma.
      $$
    \item For the affine coadjoint action of $G$ on $\cG^*\!/\Gamma$ defined by $\theta$, see \ref{Affine-coadjoint-actions-and-orbits},
    the moment map $\mu$ is equivariant. For all $g \in G$,
    $$
      \mu \circ \rho(g) = \Ad_*^{\Gamma,\theta}(g) \circ \mu.
      $$
    \item For every cocycle $\theta$, associated to some moment $\mu$, there exists always a point $x_0 \in X$ and a constant $c \in \cG^*\!/\Gamma$ such that,
    for all $g$ in $G$
    $$
      \theta(g) = \psi(x_0,\rho(g)(x_0)) + \Delta c(g).
      $$
    \item The cohomology class $\sigma$ of $\theta$ belongs to a cohomology group denoted by \linebreak
    $H^1(G,\cG^*\!/\Gamma)$.
    And, it depends only on the cohomology class of the 2-points moment map $\psi$. This class $\sigma$ will be called {\em Souriau's cohomology class}.
  \end{enumerate}
  {\sc Note 1} --- Let $x_0$ by some point of $X$.
  The 2-moment map (1-cocycle) $\psi$ defines a 1-cocycle $f$ from $G$ to $\cG^*\!/\Gamma$ by $f(g,g') = \psi(\rho(g)(x_0),\rho(g')(x_0))$.
  The cocycle $f'$ associated to another point $x_0'$ will differ just by a coboundary.
  So,
  Souriau's cocycle $\sigma$ represents just the class of this pullback $f = \hat x_0^*(\psi)$ by the orbit map $\hat x_0$,
  where $\hat x_0^* : H^1_\rho(X,\cG^*\!/\Gamma) \to H^1(G,\cG^*\!/\Gamma)$.
  And,
  by the way,
  depends only of the restriction of $\omega$ on any one orbit of $G$ on $X$.
  So,
  a good choice of the point $x_0$ can simplify sometimes the computation of $\sigma$.
  
  {\sc Note 2} --- The nature of the action $\rho$ has strong consequences on Souriau's class.
  For example,
  thanks to the third item,
  if the group $G$ has a fixed point $x_0$,
  that is $\rho(g)(x_0) = x_0$ for all $g$ in $G$,
  then Souriau's class vanishes.
  So,
  the cocycle $\psi$ is exact,
  and there exists an equivariant primitive $\mu$ of $\psi$.
  
  \begin{proof}
    Thanks to \ref{Definition-of-the-moment-maps},
    every moment map $\mu$ writes $\mu(x) = \psi(x_0,x) +c$,
    where $x_0$ is some fixed point of $X$ and $c \in \cG^*\!/\Gamma$.
    So,
    \begin{multline*}
      \mu(\rho(g)(x)) - \Ad_*^\Gamma(g)(\mu(x)) = \psi(x_0,\rho(g)(x)) + c - \Ad_*^\Gamma(g)(\psi(x_0,x) + c) \\
      = \psi(x_0,\rho(g)(x)) + c - \Ad_*^\Gamma(g)(\psi(x_0,x)) - \Ad_*^\Gamma(g)(c) \\
      = \psi(x_0,\rho(g)(x)) - \psi(\rho(g)(x_0),\rho(g)(x)) - \Delta c(g) \\
      = \psi(x_0,\rho(g)(x)) + \psi(\rho(g)(x),\rho(g)(x_0)) - \Delta c(g) = \psi(x_0,\rho(g)(x_0)) - \Delta c(g).
    \end{multline*}
    Therefore,
    $\mu(\rho(g)(x)) - \Ad_*^\Gamma(g)(\mu(x))$ is constant with respect to $x$.
    That proves the points 1) and 4).
    Now,
    the variance of $\theta$ with respect to the multiplication of $G$ is a classical result of cohomology (see for example \cite{Kir74}).
    It is then obvious that two moment maps $\mu$ and $\mu'$ differing just by a constant,
    the associated cocycles $\theta$ and $\theta'$ differ by a coboundary.
    The remaining items are just the results of elementary,
    or well known,
    algebraic computations.
  \end{proof}
  
  %%%%%%%%%%%%%%%%%%%%%%%%%%%%%%%%%%%%%%%%%%%%%%%%%%%%%%%%%%
  \section{The moment maps for exact 2-forms}
  %%%%%%%%%%%%%%%%%%%%%%%%%%%%%%%%%%%%%%%%%%%%%%%%%%%%%%%%%%
  
  The special case where the closed 2-form is the exterior differential of an invariant 1-form deserves a special care,
  since it justifies the constructions above,
  by analogy with the moment maps of classical symplectic geometry.
  
  \article{The exact case}
  \label{The-exact-case}
  Let $X$ be a connected diffeological space and let $\omega$ be a closed 2-form defined on $X$.
  Let $G$ be a diffeological group and $\rho$ be a smooth action of $G$ on $X$,
  preserving $\omega$.
  Let us assume that $\omega = d\alpha$ and that $\alpha$ is also invariant under the action of $G$,
  that is $\rho(g)^*(\alpha) = \alpha$ for all $g$ in $G$.
  Let $\Psi$ be the paths moment map defined in \ref{Definition-of-the-paths-moment-map},
  and $\psi$ be the 2-points moment map defined in \ref{Definition-of-the-2-points-moment-map}.
  So,
  for every $p \in \Paths(X)$
  $$
    \Psi(p) = \psi(x,x') = \hat x_1^*(\alpha) - \hat x_0^*(\alpha),
    $$
  where $x_1 = p(1)$ and $x_0 = x_0$.
  Moreover,
  the 2-points moment map $\psi$ is exact,
  and every equivariant moment map is cohomologous to
  $$
    \mu : x \mapsto \hat x^*(\alpha).
    $$
  The action of $G$ is hamiltonian,
  $\Gamma = \{0\}$ and exact $\sigma = 0$,
  see art.~\ref{The-holonomy-group} and art.~\ref{Souriau-cocycles}.
  So,
  this shows in particular the coherence of the general constructions developed until now.
  
  \begin{proof}
    By definition of the paths moment map,
    $\Psi(p) = \hat p^*(\cK \omega)$.
    So,
    $\Psi(p) = \hat p^*(\cK(d\alpha))$.
    But,
    $\cK(d\alpha) + d(\cK \alpha) = \but^*(\alpha) - \source^*(\alpha)$,
    thus $\cK(d\alpha) = \hat p^*[\but^*(\alpha) - \source^*(\alpha) - d(\cK\alpha)]$.
    And,
    $\Psi(p) = (\but \circ \hat p)^*(\alpha) - (\source \circ \hat p)^*(\alpha) - d[\hat p^* (\cK(\alpha))]$.
    But,
    $\but \circ \hat p = \hat x_1$,
    and $\source \circ \hat p = \hat x_0$.
    So $\Psi(p) = \hat x_1^*(\alpha)-\hat x_0^*(\alpha) -d[\hat p^*(\cK\alpha)]$.
    Now,
    $\cK\alpha$ is the real function
    $$
      \cK \alpha : p \mapsto \int_p \alpha.
      $$
    Since $\hat p^*(\cK\alpha) = \cK\alpha \circ \hat p$,
    we have for all $g \in G$,
    $$
      \cK\alpha(\hat p(g)) = \int_{\rho(g) \circ p} \alpha
      = \int_p \rho(g)^*(\alpha) = \int_p \alpha.
      $$
    So,
    the function $\hat p^*(\cK\alpha): G \to \RR$ is constant and equal to $\int_p \alpha$.
    So,
    $d [\hat p^*(\cK\alpha)] = 0$,
    and $\Psi(p) = \hat x_1^*(\alpha) - \hat x_0^*(\alpha)$.
    Thus,
    $\Psi(p) = \psi(x_0,x_1)$ and $\Gamma = \{0\}$.
    
    Now,
    the function $\mu: x \mapsto \hat x^*(\alpha)$ is clearly a primitive of $\psi$.
    That is,
    $\psi(x_0,x_1) = \mu(x_1) - \mu(x_0)$.
    But $\R(\rho(g)(x)) = \hat x \circ \R(g)$,
    where $\R(\rho(g)(x))$ denotes the orbit map of $\rho(g)(x)$,
    with $g \in G$.
    So,
    $\mu(\rho(g)(x)) = (\hat x \circ \R(g))^*(\alpha) = \R(g)^*(\hat x^*(\alpha)) = \R(g)^*(\mu(x)) = \Ad_*(g)(\mu(x))$.
    Thus,
    $\mu$ is an equivariant primitive of $\psi$.
    And,
    Souriau's class $\sigma$ vanishes.
  \end{proof}
  
  %%%%%%%%%%%%%%%%%%%%%%%%%%%%%%%%%%%%%%%%%%%%%%%%%%%%%%%%%%
  \section{Functoriality of the moment maps}
  %%%%%%%%%%%%%%%%%%%%%%%%%%%%%%%%%%%%%%%%%%%%%%%%%%%%%%%%%%
  
  We inspect now,
  the behavior of the moment maps and the various associated objects under natural transformations.
  
  \article{Images of the moment maps by morphisms}
  \label{Images-of-the-moment-maps-by-morphisms}
  Let $X$ be a connected diffeological space and $\omega$ be a closed 2-form defined on $X$.
  Let $G$ be a diffeological group and $\rho$ be a smooth action of $G$ on $X$,
  preserving $\omega$.
  Let $G'$ be another diffeological group,
  and let $h : G' \to G$ be a smooth homomorphism.
  Let $\rho' = \rho \circ h$ be the induced action of $G'$ on $X$.
  Let us remind that the pullback $h^* : \cG^* \to \cG'^{*}$ is a linear smooth map.
  \begin{enumerate}
    \item Let $\Psi : \Paths(X) \to \cG$, and $\Psi' : \Paths(X) \to \cG'$ be the paths moment map with respect to the actions of $G$ and $G'$ on $X$. So, $\Psi' = h^* \circ \Psi$.
    \item Let $\Gamma$ and $\Gamma'$ be the holonomy groups with respect to the actions of $G$ and $G'$ on $X$. So, $\Gamma' = \ h^*(\Gamma)$.
    \item The linear map $h^*$ projects on a smooth homomorphism $h^*_\Gamma : \cG/\Gamma \to \cG'^{*} / \Gamma'$, such that the following diagram commutes.
    \[
    \begin{tikzcd}
      \cG^* \arrow[r, "h^*"] \arrow[d, "\pr"'] & \cG'^* \arrow[d, "\pr'"] \\
      \cG^*/\Gamma \arrow[r, "h^*_\Gamma"'] & \cG'^*/\Gamma'
    \end{tikzcd}
    \]
    \item Let $\psi$ and $\psi'$ be the 2-points moment maps with respect to the actions of $G$ and $G'$. So, $\psi' = h^*_\Gamma \circ \psi$.
    \item Let $\mu$ be a moment map relative to the action $\rho$ of $G$. So $\mu' = h^*_\Gamma \circ \mu$ is a moment map relative to the action $\rho'$ of $G'$.
    \item Let $\mu$ be a moment map relative to the action $\rho$ of $G$, and let $\mu' = \mu \circ h^*_\Gamma$ be the associated moment map relative to the action $\rho'$ of $G'$. So, the associated Souriau's cocycles satisfy $\theta' = h^*_\Gamma \circ \theta \circ h$, summarized by the following commutative diagram.
    \[
    \begin{tikzcd}
      G' \arrow[r, "h"] \arrow[d, "\theta'"'] & G \arrow[d, "\theta"] \\
      \cG'^*/\Gamma' & \cG^*/\Gamma \arrow[l, "h^*_\Gamma"']
    \end{tikzcd}
    \]
    Said differently, if $\theta$ is Souriau's cocycle associated to a moment $\mu$ of the action $\rho$ of $G$,
    and $\mu'$ is a moment of the action $\rho'$ of $G'$,
    so $\theta'$ and $h^*_\Gamma \circ \theta \circ h$ are cohomologous.
  \end{enumerate}
  {\sc Note} --- Thanks to the identification between the space of momenta of a diffeological group and any of its extensions by a discrete group,
  stated in \ref{Momenta-and-connectedness},
  the moment maps of the action of a group or the moment map of the restriction of this action to its identity component coincide.
  Said differently,
  the moment maps doesn't say anything about actions of discrete groups.
  
  \begin{proof}
    To avoid confusion,
    let us denote by $\R(p)$ and $\R'(p)$ the orbit maps of $G$ and $G'$ of $p \in \Paths(X)$.
    That is,
    $\R(p)(g) = \rho(g) \circ p$ and $\R'(p)(g) = \rho'(g) \circ p$.
    So,
    we have,
    $\R'(p)(g) = \rho'(g) \circ p = \rho(h(g)) \circ p = (\R(p) \circ h)(g))$.
    Thus,
    $\R'(p) = \R(p) \circ h$.
    
    1. By definition of the paths moment map,
    we have $\Psi'(p) = \R'(p)^*(\cK \omega) = (\R(p) \circ h)^*(\cK \omega) = h^*(\R(p)^*(\cK \omega)) = h^*(\Psi(p))$.
    Thus,
    $\Psi' = h^* \circ \Psi$.
    
    2. Since $\Gamma' = \Psi'(\Loops(X))$,
    and thanks to item 1,
    we have $\Gamma' = h^*(\Gamma)$.
    
    3. The map $h_\Gamma^*$ is defined by $\class_\Gamma(\alpha) \mapsto \class_{\Gamma'}(h^*(\alpha))$,
    for all $\alpha \in \cG^*$.
    If $\beta = \alpha + \gamma$,
    with $\gamma \in \Gamma$,
    then $h^*(\beta) = h^*(\alpha) + \gamma'$,
    with $\gamma' = h^*(\gamma) \in \Gamma'$ (item 2).
    So,
    $\class_{\Gamma'}(h^*(\beta)) =\class_{\Gamma'}(h^*(\alpha))$.
    And,
    $h_\Gamma^*$ is well defined.
    Thanks to the linearity of $h^*$,
    $h_\Gamma^*$ is clearly an homomorphism.
    And,
    for $\cG^*\!/\Gamma$ and $\cG'^*/\Gamma'$ equipped with the quotient diffeologies,
    $h_\Gamma^*$ is naturally smooth.
    
    4. With to the notations above,
    $\psi$ and $\psi'$ are defined by,
    $\pr \circ \Psi = \psi \circ \bounds$ and $\pr' \circ \Psi' = \psi' \circ \bounds$,
    where $\bounds(p) = \source \times \but (p) = (p(0),p(1))$,
    with $p \in \Paths(X)$.
    So,
    by item 1 and 3,
    we have $\pr' \circ h^* \circ \Psi = h_\Gamma^* \circ \psi \circ \pr$.
    That is,
    $\pr' \circ \Psi' = (h_\Gamma^* \circ \psi) \circ \pr$.
    So,
    $h_\Gamma^* \circ \psi = \psi'$.
    
    5. Let $\mu' = h_\Gamma^* \circ \mu$,
    and let $x,y \in X$.
    So,
    $\mu'(y) - \mu'(x) = h_\Gamma^* \circ \mu(y) - h_\Gamma^* \circ \mu(y) = h_\Gamma^*(\mu(y)-\mu(x)) = h_\Gamma^*\circ \psi(y,x) = \psi'(y,x)$.
    So,
    $\mu'$ is a moment map for the action $\rho'$ of $G$.
    
    6. According to art.~\ref{Souriau-cocycles},
    there exists a point $x_0 \in X$ such that,
    for all $g' \in G'$,
    $\theta'(g') = \psi'(x_0,\rho'(g')(x_0))$.
    So,
    thanks to the previous items we have,
    $\theta'(g') = (h_\Gamma^* \circ \psi)(x_0,\rho( h (g'))(x_0)) = h_\Gamma^*(\psi(x_0,\rho( h (g'))(x_0))) = h_\Gamma^*(\theta(h(g'))) = (h_\Gamma^* \circ \theta \circ h) (g')$.
    Thus,
    we get $\theta' = h_\Gamma^* \circ \theta \circ h$
  \end{proof}
  
  \article{Pushing forward moment maps}
  \label{Pushing-forward- moment-maps}
  Let $X$ and $X'$ be two connected diffeological spaces.
  Let $\omega$ and $\omega'$ be two closed 2-forms defined respectively on $X$ and $X'$.
  Let $G$ be a diffeological group,
  let $\rho$ be a smooth action of $G$ on $X$,
  preserving $\omega$,
  and let $\rho'$ be a smooth action of the same group $G$ on $X'$,
  preserving $\omega'$.
  Let $f: X \to X'$ be a smooth map such that $\omega = f^*(\omega')$,
  and $f \circ \rho(g) = \rho'(g) \circ f$,
  for all $g \in G$.
  \begin{enumerate}
    \item Let $f_* : \Paths(X) \to \Paths(X')$ defined by $f_*(p) = f \circ p$. So, the paths moment maps $\Psi$ and $\Psi'$ relative to the action $\rho$ and $\rho'$ are related by
    $$
      \Psi = \Psi' \circ f_*,
      $$
    and the associated holonomy groups $\Gamma$ and $\Gamma'$ satisfy
    $$
      \Gamma = \{ \Psi'(f \circ \ell) \mid \ell \in \Loops(X) \} \subset\Gamma'.
      $$
    \item Let $\phi: \cG^*\!/\Gamma \to \cG^*\!/\Gamma'$ be the projection induced by the inclusion $\Gamma \subset\Gamma'$. Let $\psi$ and $\psi'$ be the 2-points moment maps relative to the actions $\rho$ and $\rho'$. So, for all pairs of points $x_1$, $x_2$ of $X$,
    $$ \psi'(f(x_1),f(x_2)) = \phi(\psi(x_1,x_2)).
      $$
    \item For every moment map $\mu$ relative to the action $\rho$, there exists a moment map $\mu'$ relative to the action $\rho'$, such that
    $$
      \mu' \circ f = \phi \circ \mu.
      $$
    \item Let $\theta$ and $\theta'$ be two Souriau's cocycles relative to the actions $\rho$ and $\rho'$. So, the map $\phi \circ \theta$ is a Souriau cocycle, cohomologous to $\theta'$. Thus, the two Souriau's classes $\sigma$ and $\sigma'$ satisfy $\sigma' = \phi_*(\sigma)$. Where $\phi_*$ denotes the action of $\phi$ on cohomology, $\phi_*(\class(\theta)) = \class(\phi \circ \theta)$.
  \end{enumerate}
  
  \begin{proof}
    1. By definition $\Psi(p) = \hat p^*(\cK\omega)$,
    that is $ \Psi(p) = \hat p^*(\cK(f^*(\omega')))$.
    And thanks to the variance of the chain-homotopy operator $\cK \circ f^* = (f_*)^* \circ \cK'$,
    see \ref{Chain-Homotopy-operator},
    we have $\Psi(p) = \hat p^* \circ (f_*)^* (\cK'\omega') = (f_* \circ \hat p)^*(\cK'\omega')$.
    But,
    for all $g \in G$,
    $f_* \circ \hat p(g) = f \circ \rho(g) \circ p = \rho'(g) \circ f \circ p = \hat p'(g)$,
    where $p' = f \circ p$.
    So,
    $\Psi(p) = \hat p'^*(\cK' \omega') = \Psi'(p') = \Psi'(f_*(p))$.
    Therefore,
    $\Psi = \Psi' \circ f_*$.
    Now,
    by definition of the holonomy groups,
    $\Gamma = \Psi(\Loops(X)) = \Psi'(f_*(\Loops(X)))$,
    and since $f_*(\Loops(X)) \subset \Loops(X')$,
    we get $\Gamma \subset \Gamma'$.
    
    2. Since $\Gamma \subset \Gamma'$,
    the map $\phi : \class_\Gamma(\alpha) \mapsto \class_{\Gamma'}(\alpha)$,
    from $\cG^*\!/\Gamma \to \cG^*\!/\Gamma'$,
    is well defined.
    Now,
    let $x'_1 = f(x_1)$ and $x'_2 = f(x_2)$,
    there exists $p \in \Paths(X)$ connecting $x_1$ to $x_2$.
    So the path $f_*(p)$ connects $x'_1$ to $x'_2$.
    Thus,
    by definition of $\psi'$,
    $\psi'(x'_1,x'_2) = \class_{\Gamma'}(\Psi'(p')) = \class_{\Gamma'}(\Psi'\circ f_*(p))$,
    and thanks to the first item:
    \begin{multline*}
      \class_{\Gamma'}(\Psi'(p')) = \class_{\Gamma'}(\Psi(p)) = \phi(\class_\Gamma(\Psi(p))).
      \text{ But, } \class_\Gamma(\Psi(p)) = \psi(x_1,x_2). \\
      \text{ Thus, }
      \psi'(x'_1,x'_2) = \phi(\psi(x_1,x_2)),
      \text{ that is, }
      \psi'(f(x_1), f(x_2)) = \psi(x_1,x_2).
    \end{multline*}
    3. According to \ref{Definition-of-the-moment-maps},
    for every moment map $\mu$ there exists a point $x_0 \in X$ and a constant $c \in \cG^*\!/\Gamma$ such that $\mu(x) = \psi(x_0,x) + c$ .
    Let us define $\mu'$ by $\mu'(x') = \psi'(x'_0,x') + c'$,
    where $x'_0 = f(x_0)$ and $c' = \phi(c)$.
    So,
    thanks to the item 2,
    $\psi'(f(x_0),f(x)) = \phi(\psi(x_0,x))$,
    so $\mu'(f(x)) = \phi(\psi(x_0,x)) + \phi(c) = \phi(\psi(x_0,x) + c) = \phi(\mu(x))$.
    Thus,
    $\mu'$ satisfies $\mu' \circ f = \phi \circ \mu$.
    
    4. Let $\theta$ be a Souriau cocycle for the action $\rho$.
    According to art.~\ref{Souriau-cocycles},
    $\theta$ is cohomologous to $\vartheta : g \mapsto \psi(x_0, \rho(g)(x))$,
    where $x_0$ is some point of $X$.
    So,
    let $x'_0 = f(x_0)$,
    and $\vartheta' : g \mapsto \psi'(x'_0, \rho'(g)(x'_0))$.
    Thus,
    $\vartheta'(g) = \psi'(f(x_0), \rho'(g)(f(x_0))) = \psi'(f(x_0), f(\rho(g)(x_0))) = \phi(\psi(x_0,\rho(g)(x_0))) = \phi \circ \vartheta(g)$.
    Now since all Souriau's cocycles,
    with respect to a given action of $G$,
    are cohomologous,
    the cocycle $\theta'$ is cohomologous to $\vartheta'$,
    and then cohomologous to $\phi \circ \vartheta$,
    and thus to $\phi \circ \theta$.
    Therefore,
    $\sigma' = \class(\theta') = \class(\phi \circ \theta) = \phi_*(\class(\theta)) = \phi_*(\sigma)$.
  \end{proof}
  
  %%%%%%%%%%%%%%%%%%%%%%%%%%%%%%%%%%%%%%%%%%%%%%%%%%%%%%%%%%
  \section{The universal moment maps}
  %%%%%%%%%%%%%%%%%%%%%%%%%%%%%%%%%%%%%%%%%%%%%%%%%%%%%%%%%%
  
  The theory of moment maps developed in the previous paragraph applies in particular to the whole group of automorphisms $\Diff(X,\omega)$ of a closed 2-form $\omega$ defined on a diffeological space $X$.
  We will describe,
  in this paragraph,
  the relationships between the ``universal'' moment maps and associated objects obtained by considering the whole group $\Diff(X,\omega)$ and the equivalent objects associated to a smooth action of some other group $G$ on $X$,
  preserving $\omega$.
  
  \article{Universal moment maps}
  \label{Universal-moment-maps}
  Let $X$ be a connected diffeological space and let $\omega$ be a closed 2-form defined on $X$.
  Let us remind that the group $\Diff(X,\omega)$ of all the automorphisms of $(X,\omega)$ is equipped with the functional diffeology of group of diffeomorphisms.
  Let us denote also this group by $G_\omega$.
  Every constructions defined above,
  the moment space,
  the paths moment map,
  the holonomy group,
  the 2-points moment map,
  the moment maps,
  Souriau's cocycle and Souriau's class,
  apply for $G_\omega$.
  We shall distinguish these objects by the index $\omega$.
  So,
  we denote by $\cG^*_\omega$ the momenta space of $G_\omega$,
  by $\Psi_\omega : \Paths(X) \to \cG^*_\omega$ the paths moment map,
  by $\Gamma_\omega = \Psi_\omega(\Loops(X))$ the holonomy group,
  by $\psi_\omega$ the 2-points moment map,
  by $\mu_\omega$ the moment maps,
  by $\theta_\omega$ Souriau's cocycles,
  and by $\sigma_\omega$ Souriau's class.
  Since $G_\omega$ and its action on $X$ are uniquely defined by $\omega$,
  these objects depend only on the 2-form $\omega$.
  
  Now,
  let $G$ be a diffeological group and $\rho$ be a smooth action of $G$ on $X$,
  preserving $\omega$.
  That is,
  a smooth homomorphism $\rho$ from $G$ to $G_\omega$.
  The values of the various objects $\Psi$, $\Gamma$, $\psi$, $\mu$, $\theta$,
  with respect to the action $\rho$ of $G$ on $X$,
  depend only on $\rho^*$, $\Psi_\omega$, $\Gamma_\omega$, $\psi_\omega$, $\mu_\omega$, and $\theta_\omega$,
  as described in \ref{Images-of-the-moment-maps-by-morphisms}.
  And,
  we have:
  $$
    \left\{
    \begin{array}{rcl}
    \Psi & = & \rho^* \circ \Psi_\omega \\
    \Gamma & = & \rho^*(\Gamma_\omega) \\
    \psi & = & \rho_{\Gamma_\omega}^* \circ \psi_\omega
    \end{array}
    \right.
    \quad \& \quad
    \left\{
    \begin{array}{rcl}
    \mu & \simeq & \rho_{\Gamma_\omega}^* \circ \mu_\omega \\
    \theta & \simeq & \rho_{\Gamma_\omega}^* \circ
    \theta_\omega \circ \rho.
    \end{array}
    \right.
    $$
  In this sense the objects $G_\omega$, $\Gamma_\omega$, $\Psi_\omega$, $\Gamma_\omega$, $\psi_\omega$, $\mu_\omega$, $\theta_\omega$ and $\sigma_\omega$ are {\em universal}.
  So,
  we shall call $\Psi_\omega$ the {\em universal paths moment map},
  $\Gamma_\omega$ the {\em universal holonomy},
  $\psi_\omega$ the {\em universal 2-points moment map},
  $\mu_\omega$ the {\em universal moment maps},
  $\theta_\omega$ {\em universal Souriau's cocycles},
  and $\sigma_\omega$ {\em universal Souriau's class} of $\omega$.
  
  Note that in particular,
  this gives us a notion of {\em hamiltonian spaces},
  those for which,
  for one reason or another,
  the universal holonomy is trivial $\Gamma_\omega = \{0\}$.
  
  \article{The group of hamiltonian diffeomorphisms}
  \label{The-group-of-hamiltonian-diffeomorphisms}
  Let $X$ be a connected diffeological space equipped with a closed 2-form $\omega$.
  There exists a largest connected subgroup $\Ham(X,\omega) \subset \Diff(X,\omega)$ whose action is hamiltonian,
  that is whose holonomy vanishes.
  The elements of $\Ham(X,\omega)$ are called {\em hamiltonian diffeomorphisms\/}.
  An action $\rho$ of a diffeological group $G$ on $X$ is hamiltonian if and only if,
  restricted to the identity component of $G$,
  $\rho$ takes its values in $\Ham(X,\omega)$.
  
  The construction of $\Ham(X,\omega)$ is actually given as follows.
  Let us denote by $G_\omega$ the group $\Diff(X,\omega)$ and by $\idGomega$ its identity component.
  Let $\pi : \tidGomega \to \idGomega$ be the universal covering.
  Since the universal holonomy $\Gamma_\omega$ is made up of closed momenta,
  according to \ref{Closed-momenta-of-a-diffeological-group} every $\gamma \in \Gamma_\omega$ defines a unique homomorphism $\ek(\gamma)$ from $\tidGomega$ to $\RR$ such that $\pi^*(\gamma) = d[\ek(\gamma)]$.
  Let
  $$
    \widehat H_\omega = \bigcap_{\gamma \in \Gamma_\omega} \ker(\ek(\gamma)),
    $$
  and let $\widehat H_\omega^\circ$ be its identity component.
  So,
  $$
    \Ham(X,\omega) = \pi(\widehat H_\omega^\circ).
    $$
  {\sc Note 1} --- The map $\ef : \tidGomega \to \Hom(\pi_1(X),\RR)$ defined by $\ef(\tilde g) = [ \tau \mapsto \ek(\gamma)(\tilde g)]$,
  with $\tau = \class(\ell)$ and $\gamma = \Psi(\ell)$,
  is an homomorphism.
  And,
  $\widehat H_\omega = \ker(\ef)$.
  In classical symplectic geometry,
  the image $\eF = \Values(\ef)$ is called,
  by some authors,
  the {\em group of flux} of $\omega$.
  
  {\sc Note 2} --- Since to be hamiltonian for a group of automorphisms depends only on its connected component,
  see \ref{Momenta-and-connectedness} and \ref{Momenta-of-covering-of-diffeological-groups},
  any extension $H \subset \Diff(X,\omega)$ of $\Ham(X,\omega)$,
  such that $H / \Ham(X,\omega)$ is discrete\footnote{Where $H$ and $\Ham(X,\omega)$ are equipped with the subset diffeology of the functional diffeology of $\Diff(X,\omega)$.},
  is hamiltonian.
  In particular $\pi(\widehat H_\omega)$ is hamiltonian,
  or if $\Gamma_\omega = \{0\}$ then $\Diff(X,\omega)$ is hamiltonian,
  and $\Ham(X,\omega)$ is the identity component of $\Diff(X,\omega)$.
  
  {\sc Note 3} --- Let us choose a point $x_0$ in $X$ and let $\mu$ be the moment map with respect to the group $\Ham(X,\omega)$,
  defined by $\mu(x_0) = 0$.
  Let $f$ be a 1-parameter subgroup of $\Ham(X,\omega)$.
  Applying the note of \ref{Evaluation-of-the-paths-moment-map},
  we get for all $x \in X$ the expression of $\mu(x)$,
  evaluated on $f$
  $$
    \mu(x)(f) = h_f(x) \times dt \qmbox{with} h_f(x) = - \int_{x_0}^x i_f(\omega).
    $$
  The smooth function $h_f : X \to \RR$ is the {\em hamiltonian} (vanishing at $x_0$) of the 1-parameter subgroup $f$.
  
  \begin{proof}
    Let us remark,
    first of all,
    that for every $\gamma \in \Gamma_\omega$,
    $\pi^*(\gamma) \restriction \widehat H_\omega = 0$.
    Indeed,
    $\pi^*(\gamma) \restriction \widehat H_\omega = d[\ek(\gamma)] \restriction \widehat H_\omega = d[\ek(\gamma)\restriction \widehat H_\omega]$.
    But,
    by the very definition of $\widehat H_\omega$,
    $\ek(\gamma)\restriction \widehat H_\omega = 0$,
    so $\pi^*(\gamma) \restriction \widehat H_\omega = 0$.
    
    a) Let us prove that the holonomy of $\Ham(X,\omega)$ is trivial.
    Let $H_\omega = \pi(\widehat H_\omega)$ and let us denote by $j_{H_\omega}$ the inclusion $H_\omega \subset G_\omega$,
    by $j_{\widehat H_\omega}$ the inclusion $\widehat H_\omega \subset \tidGomega$,
    and by $\pi_{H_\omega} : \widehat H_\omega \to H_\omega$ the projection.
    So,
    $ j_{H_\omega} \circ \pi_{H_\omega} = \pi \circ j_{\widehat H_\omega}$.
    Let $\Gamma_{H_\omega}$ be the holonomy of $H_\omega$,
    so according to \ref{Images-of-the-moment-maps-by-morphisms},
    $\Gamma_{H_\omega} = j_{H_\omega}^*(\Gamma_\omega)$.
    Thus,
    for every $\bar \gamma \in \Gamma_{H_\omega}$ there exists $\gamma \in \Gamma_\omega$ such that $\bar \gamma = \gamma \restriction H_\omega = j_{H_\omega}^*(\gamma)$.
    So,
    for all $\bar \gamma \in \Gamma_{H_\omega}$,
    $\pi_{H_\omega}^* (\bar \gamma) = \pi_{H_\omega}^* (j_{H_\omega}^*(\gamma)) = (j_{H_\omega} \circ \pi_{H_\omega})^*(\gamma) = (\pi \circ j_{\widehat H_\omega})^*(\gamma) = j_{\widehat H_\omega}^*(\pi^*(\gamma)) = \pi^*(\gamma) \restriction \widehat H_\omega$.
    But,
    $\pi^*(\gamma) \restriction \widehat H_\omega = 0$,
    so $\pi_{H_\omega}^*(\bar \gamma) = 0$.
    And since $\pi_{H_\omega}$ is a subduction,
    $\bar \gamma = 0$.
    Therefore,
    the holonomy of $H_\omega$ vanishes,
    $\Gamma_{H_\omega} = \{0\}$.
    
    b) Let us prove now that every connected subgroup $H \subset G_\omega$ whose action is hamiltonian is a subgroup of $\Ham(X,\omega)$.
    Let $\widehat H = \pi^{-1}(H)$ and $\widehat H^\circ$ be its identity component.
    Let $j_H$ be the inclusion $H \subset G_\omega$,
    and $j_{\widehat H^\circ}$ be the inclusion $\widehat H^\circ \subset \tidGomega$.
    Let $\pi_{H} = \pi \restriction \widehat H^\circ$.
    So,
    $j_H \circ \pi_H = \pi \circ j_{\widehat H^\circ}$.
    Let $\Gamma_{H}$ be the holonomy of $H$.
    Since $\Gamma_{H} = j_{H}^*(\Gamma_\omega)$ and $\Gamma_H = \{0\}$,
    for all $\gamma \in \Gamma_\omega$,
    $j_H^*(\gamma) = 0$.
    Thus,
    for all $\gamma \in \Gamma_\omega$,
    $\pi_H^*(j_H^*(\gamma)) = 0$.
    But,
    $ \pi_H^*(j_H^*(\gamma)) = (j_H \circ \pi_H)^*(\gamma) = (\pi \circ j_{\widehat H^\circ})^*(\gamma) = j_{\widehat H^\circ}^*(\pi^*(\gamma)) = \pi^*(\gamma) \restriction \widehat H^\circ$.
    So,
    for all $\gamma \in \Gamma_\omega$,
    $\pi^*(\gamma) \restriction \widehat H^\circ = 0$.
    But $\pi^*(\gamma) = d[\ek(\gamma)]$,
    hence $d[\ek(\gamma) \restriction \widehat H^\circ] = 0$.
    So,
    since $H^\circ$ is connected,
    $\ek(\gamma)$ is constant on $\widehat H^\circ$,
    and since $\ek(\gamma)$ is an homomorphism to $\RR$,
    this constant is necessarily $0$.
    Thus,
    $\widehat H^\circ \subset \ker(\ek(\gamma))$,
    for all $\gamma \in \Gamma_\omega$,
    that is $\widehat H^\circ \subset \widehat H_\omega$.
    But,
    since $H^\circ$ is connected $\widehat H^\circ \subset \widehat H_\omega^\circ \subset H_\omega$ and thus $H = \pi(\widehat H^\circ) \subset \Ham(X,\omega) = \pi(\widehat H_\omega^\circ)$.
  \end{proof}
  
  \article{Time-dependent hamiltonian}
  \label{Time-dependent-hamiltonian}
  Let $X$ be a connected diffeological space and $\omega$ be a closed 2-form defined on $X$.
  A diffeomorphism $f$ of $X$ belongs to $\Ham(X,\omega)$ if and only if:
  \begin{enumerate}
    \item There exists a smooth path $t \mapsto f_t$ in $\Diff(X,\omega)$ connecting the identity $\id_M = f_0$ to $f = f_1$.
    \item There exists a smooth path $t \mapsto \Phi_t$ in $\Cinfty(X,\RR)$ such that for all $t$,
    $$
      i_{F_t}(\omega) = - d\Phi_t \qmbox{with} F_t : s \mapsto f_t^{-1} \circ f_{t+s}.
      $$
  \end{enumerate}
  According to the tradition of classical symplectic geometry,
  the path $t \mapsto \Phi_t$ can be called a {\em time-dependent hamiltonian} of the 1-parameter family of hamiltonian diffeomorphisms $t \mapsto f_t$.
  
  \begin{proof}
    Let us assume first that $f$ satisfies the condition above.
    That is,
    there exists a smooth path $t \mapsto f_t$ in $\Diff(X,\omega)$ such that $f_0 = \id_M$,
    $f_1 = f$,
    and there exists a smooth path $t \mapsto \Phi_t$ in $\Cinfty(X,\RR)$ such that $i_{F_t}(\omega) = - d\Phi_t$ for all $t$ where $F_t : s \mapsto f_t^{-1} \circ f_{t+s}$.
    Let us remind that $\Ham(X, \omega) = \pi(\widehat H_\omega^\circ)$,
    with $\widehat H_\omega^\circ$ the identity component of $\widehat H_\omega = \cap_{\gamma \in \Gamma_\omega} \ker(\ek(\gamma))$,
    and let $\tilde f \in G_\omega^\circ$ be the homotopy class of the path $t \mapsto f_t$,
    notations of \ref{The-group-of-hamiltonian-diffeomorphisms}.
    So,
    let $\gamma \in \Gamma_\omega$,
    that is $\gamma = \Psi_\omega(\ell)$ where $\ell$ is some loop in $M$.
    By definition,
    we have
    $$
      \ek(\gamma)(\tilde f) = \int_{[t \mapsto f_t]} \gamma
      = \int_{[t \mapsto f_t]} \Psi_\omega(\ell) = \int_0^1
      \Psi_\omega(\ell)([t \mapsto f_t])_t(1) dt
      $$
    Now,
    thanks to \ref{Evaluation-of-the-paths-moment-map} (\ref{eq:clubsuit}),
    we have
    $$
      \Psi_\omega(\ell)([t \mapsto f_t])_t(1) = - \int_\ell i_{F_t}(\omega) = \int_\ell d\Phi_t = \int_{\partial \ell} \Phi_t = 0.
      $$
    So,
    $\ek(\gamma)(\tilde f) = 0$ for all $\gamma \in \Gamma_\omega$ and $\tilde f$ belongs to $\widehat H_\omega$ and more precisely in the identity component of $\widehat H_\omega$.
    Therefore $f \in \Ham(X,\omega)$.
    
    Conversely,
    let $f \in \Ham(M, \omega)$.
    Since $\Ham(M,\omega)$ is connected there exists a path $t \mapsto f_t$ in $\Ham(M, \omega)$ connecting $\id_M$ to $f$.
    And,
    since the projection $\pi \restriction \widehat H_\omega^\circ : \widehat H_\omega^\circ \to \Ham(M,\omega)$ is a covering,
    there exists a (unique) lifting $t \mapsto \tilde f_t$ of $t \mapsto f$ in $\widehat H_\omega^\circ$,
    along $\pi \restriction \widehat H_\omega^\circ$,
    such that $\tilde f_0 = \id_{\widehat H_\omega}$.
    This lifting is actually given by $\tilde f_t = \class(p_t)$,
    with $p_t : s \mapsto f_{st}$.
    So,
    for all $t$,
    $\tilde f_t \in \widehat H_\omega^\circ \subset \widehat H_\omega = \cap_{\gamma \in \Gamma_\omega} \ker(\ek(\gamma))$.
    That is,
    for all $\gamma \in \Gamma_\omega$,
    $\ek(\gamma)(\tilde f_t) = 0$,
    or in other words,
    for all $\ell \in \Loops(M)$,
    $\ek(\Psi_\omega(\ell))(\tilde f_t) = 0$.
    But,
    \begin{eqnarray*}
      \ek(\Psi_\omega(\ell))(\tilde f_t) &=& \int_{p_t} \Psi_\omega(\ell) \\
      &=& \int_0^1 \Psi_\omega(\ell)(s \mapsto f_{st})_s(1) ds \\
      &=& \int_0^1 \Psi_\omega(\ell)(s \mapsto st \mapsto f_{st})_s(1) ds \\
      &=& \int_0^1 [\Psi_\omega(\ell)(u \mapsto f_u)]_{u=st} \bigg({d st \over ds}\bigg) ds \\
      &=& \int_0^t \Psi_\omega(\ell)(u \mapsto f_u)_u(1) du.
    \end{eqnarray*}
    So,
    $$
      \ek(\Psi_\omega(\ell))(\tilde f_t) = 0 \quad \Rightarrow \quad {1 \over t} \int_0^t \Psi_\omega(\ell)(u \mapsto f_u)_u(1) du = 0,
      $$
    and taking the limit for $t \to 0$ we get,
    $$
      \ek(\Psi_\omega(\ell))(\tilde f_t) = 0 \quad \Rightarrow \quad \Psi_\omega(\ell)(t \mapsto f_t)_t(1) = 0.
      $$
    But,
    $\Psi_\omega(\ell)([t \mapsto f_t])_t(1) = - \displaystyle\int_\ell i_{F_t}(\omega)$,
    see \ref{Evaluation-of-the-paths-moment-map} (\ref{eq:clubsuit}).
    So,
    for all $t$ and all $\ell \in \Loops(X)$
    $$
      \int_\ell i_{F_t}(\omega) = 0.
      $$
    But $F_t$ is a path in $\Diff(X,\omega)$ centered at the identity,
    so the Lie derivative of $\omega$ by $F_t$ vanishes,
    and applying the Cartan formula given in \ref{Differential-forms},
    we get
    $$
      \DLie_{F_t} \omega = 0 \quad \Rightarrow \quad d[i_{F_t}(\omega)] + i_{F_t}(d\omega) = d[i_{F_t}(\omega)] = 0.
      $$
    So,
    the 1-form $i_{F_t}(\omega)$ is closed and its integral on any loop $\ell$ of $X$ vanishes,
    therefore $i_{F_t}(\omega)$ is exact \cite{Piz05}.
    Thus,
    for all real $t$ there exists a real function $\Phi_t \in \Cinfty(X,\RR)$ such that $i_{F_t}(\omega) = - d\Phi_t$.
    The fact that $t \mapsto \Phi_t$ is a smooth map from $\RR$ to $\Cinfty(X,\RR)$,
    for the functional diffeology,
    is a consequence of the explicit construction of the function $\Phi_t$ by integration along the paths,
    see \cite{Piz05}.
  \end{proof}
  
  %%%%%%%%%%%%%%%%%%%%%%%%%%%%%%%%%%%%%%%%%%%%%%%%%%%%%%%%%%
  \section{About Symplectic Manifolds}
  %%%%%%%%%%%%%%%%%%%%%%%%%%%%%%%%%%%%%%%%%%%%%%%%%%%%%%%%%%
  
  The case of symplectic manifolds $(M, \omega)$ deserves a special care:
  any universal moment map $\mu_\omega$ is injective and therefore identifies $M$ with a coadjoint orbit
  --- in the general sense given in \ref{Affine-coadjoint-actions-and-orbits} ---
  of $\Diff(M,\omega)$.
  
  \article{Value of the moment maps for manifolds}
  \label{Value-of-the-moment-maps-for-manifolds}
  Let $M$ be a connected manifold equipped with a closed 2-form $\omega$.
  In this context,
  the paths moment map $\Psi_\omega$ takes a special expression.
  Let $p$ be a path in $M$,
  let $F : U \to \Diff(M,\omega)$ be a $n$-plot,
  we have
  \begin{equation}
    \label{eq:diamondsuit2}
    \Psi_\omega(p)(F)_r(\delta r) = \int_0^1 \omega_{p(t)}(\dot p(t), \delta p(t)) \ dt
  \end{equation}
  for all $r$ in $U$ and $\delta r$ in $\RR^n$,
  where $\delta p$ is the lifting in the tangent space $TM$ of the path $p$,
  defined by
  \begin{equation}
    \label{eq:heartsuit3}
    \delta p(t) = [\D(F(r))(p(t))]^{-1} {\partial F(r)(p(t)) \over \partial r} (\delta r).
  \end{equation}
  
  \begin{proof}
    By definition,
    $\Psi(p)(F) = \hat p^*(\cK \omega)(F) = \cK \omega (\hat p \circ F)$.
    The explicit expression of the operator $\cK$ given in \ref{Chain-Homotopy-operator},
    applied to the plot $\hat p \circ F : r \mapsto F(r) \circ p$ of $\Paths(X)$,
    gives
    $$
      (\eK\omega)(\hat p \circ F)_r(\delta r) = \int_0^1 \omega \left[ \mymatrix{s \cr u} \mapsto (\hat p \circ F)(u)(s + t) \right]_{s=0 \choose u=r}\mymatrix{1 \cr 0} \mymatrix{0 \cr \delta r} dt.
      $$
    But $(\hat p \circ F)(u)(s + t) = F(u)(p(s+t))$,
    let us denote temporarily by $\Phi_t$ the plot $(s,u) \mapsto F(u)(p(s+t))$,
    so $F(u)(p(s+t))$ writes $\Phi_t(s,u)$.
    Now,
    let us denote by $\cI$ the integrand of the right term of this expression.
    We have,
    \begin{eqnarray*}
      \cI & = & \omega \left[ \mymatrix{s \cr u} \mapsto \Phi_t(s,u) \right]_{s=0 \choose u=r}\mymatrix{1 \cr 0} \mymatrix{0 \cr \delta r}\\
      & = & \Phi_t^*(\omega)_{0 \choose r} \mymatrix{1 \cr 0} \mymatrix{0 \cr \delta r} \\
      & = & \omega_{\Phi_t{{0 \choose r}}} \left(\D(\Phi_t)_{0 \choose r} \mymatrix{1 \cr 0},\D(\Phi_t)_{0 \choose r} \mymatrix{0 \cr \delta v}\right) \\
      & =& \omega_{F(r)(p(t))} \left({\partial \over \partial s} \bigg\{ F(r)(p(s+t))\bigg\}_{s=0}, {\partial \over \partial r} \bigg\{F(r)(p(t)) \bigg\} (\delta r)\right).
    \end{eqnarray*}
    But,
    $$
      {\partial \over \partial s} \bigg\{ F(r)(p(s+t))\bigg\}_{s=0} = \D(F(r))(p(t))\bigg({\partial p(s+t) \over \partial s}\bigg|_{s=0}\bigg) = \D(F(r))(p(t))(\dot p(t)).
      $$
    So,
    using this last expression and the fact that $F$ is a plot of $\Diff(M,\omega)$,
    that is for all $r$ in $U$,
    $F(r)^*\omega = \omega$,
    we have
    \begin{eqnarray*}
      \omega \left[\mymatrix{s \cr u} \mapsto \Phi_t(s,u)\right]_{s=0 \choose u=r} & = & \omega_{F(r)(p(t))} \bigg(\D(F(r))(p(t))(\dot p(t)) , {\partial F(r)(p(t)) \over \partial r} (\delta r)\bigg) \\
      &=& \omega_{p(t)} \bigg(\dot p(t) , [\D(F(r))(p(t))]^{-1}{\partial F(r)(p(t)) \over \partial r} (\delta r)\bigg) \\
      &=& \omega_{p(t)} (\dot p(t) , \delta p(t)).
    \end{eqnarray*}
    Therefore,
    $\Psi_\omega(p)(F)_r(\delta r) = \cK \omega (\hat p \circ F)_r(\delta r) = \displaystyle{\int_0^1} \omega_{p(t)}(\dot p(t), \delta p(t)) \ dt$.
  \end{proof}
  
  \article{The paths moment maps for symplectic manifolds}
  \label{The-paths-moment-maps-for-symplectic-manifolds}
  Let $M$ be a Hausdorff manifold and $\omega$ be a non degenerate closed 2-form defined on $M$.
  Let $m_0$ and $m_1$ be two points of $M$ connected by a path $p$.
  Let $f \in \Cinfty(M,\RR)$ with compact support.
  Let $F$ be the exponential of the symplectic gradient\footnote{Let us remind that the symplectic gradient is defined by $\omega(\grad_\omega(f),\cdot) = -df$.} $\grad_\omega(f)$,
  $F$ is a 1-plot of $\Diff(M,\omega)$,
  and precisely a 1-parameter homomorphism.
  So,
  the universal paths moment map $\Psi_\omega$,
  computed at the path $p$,
  evaluated to the 1-plot $F$,
  is the constant 1-form of $\RR$,
  $$
    \Psi_\omega(p)(F) = [f(m_1) -f(m_0)] \times dt \qmbox{with} F :t \mapsto e^{t \grad_\omega(f)},
    $$
  and $dt$ the standard 1-form of $\RR$.
  Note that we are in the special case where $F$ is actually a 1-parameter homomorphism of $\Ham(M,\omega) \subset \Diff(M,\omega)$,
  and the function $f$ is one {\em hamiltonian} of $F$.
  
  \begin{proof}
    Let us remark that,
    in our case,
    the lift $\delta p$ defined by (\ref{eq:heartsuit3}) of \ref{Value-of-the-moment-maps-for-manifolds} writes simply
    $$
      \delta p(t) = [\D(e^{r \xi})(p(t))]^{-1} {\partial e^{r \xi}(p(t)) \over \partial r} (\delta r) = \xi(p(t)) \times \delta r \qmbox{with} \xi = \grad_\omega(f),
      $$
    where $r$ and $\delta r$ are reals.
    So,
    the expression (\ref{eq:diamondsuit2}) of \ref{Value-of-the-moment-maps-for-manifolds} becomes
    \begin{eqnarray*}
      \Psi_\omega(p)(F)_r(\delta r) & = & \int_0^1 \omega_{p(t)}(\dot p(t), \xi(p(t)) \ dt \times \delta r \\
      & = & \int_0^1 \omega_{p(t)}(\dot p(t), \grad_\omega(f)(p(t)) \ dt \times \delta r\\
      & = & \int_0^1 df\bigg({d p(t) \over dt} \bigg) dt \times \delta r\\
      & = & [f(p(1)) - f(p(0))] \times \delta r
    \end{eqnarray*}
    That is,
    $\Psi_\omega(p)(F) = [f(m_1) -f(m_0)] \times dt$.
  \end{proof}
  
  \article{Moment maps for symplectic manifolds}
  \label{Moment-maps-for-symplectic-manifolds}
  Let $M$ be a connected Hausdorff manifold and $\omega$ be a closed 2-form defined on $M$.
  The form $\omega$ is non-degenerated,
  that is symplectic,
  if and only if
  \begin{enumerate}
    \item The manifold $M$ is an homogeneous space of $\Diff(M,\omega)$.
    \item Any one of its universal moment maps $\mu_\omega : M \to \cG^*_\omega/\Gamma_\omega$ is injective.
  \end{enumerate}
  Note that,
  if one of the universal moment maps $\mu_\omega$ is injective so are every ones.
  Note also that,
  if $\omega$ is symplectic,
  then the image of the moment map,
  $\cO_\omega = \mu_\omega(M) \in \cG^*_\omega/\Gamma_\omega$,
  is a $(\Gamma_\omega,\theta_\omega)$-coadjoint orbit of $\Diff(M,\omega)$.
  And,
  $\mu_\omega$ identifies $M$ to $\cO_\omega$,
  where $\cO_\omega$ is equipped with the quotient diffeology of $\Diff(M,\omega)$.
  In other words,
  every symplectic manifold is a coadjoint orbit.
  
  {\sc Remark} --- Let us consider the example $M = \RR^2$ and $\omega = (x^2 + y^2)\ dx \wedge dy$.
  This form is non degenerate on $\RR^2 - \{0\}$,
  but degenerates at the point $(0,0)$.
  Thus,
  $(0,0)$ is an orbit of the group $\Diff(X,\omega)$,
  and actually $\RR^2 -\{0\}$ is the other orbit.
  Since $\RR^2$ is contractible the holonomy $\Gamma_\omega$ is trivial and the universal moment map $\mu_\omega$ defined by $\mu_\omega(0,0) = 0_{\cG_\omega^*}$ is equivariant.
  Now,
  $\mu_\omega$ is injective,
  and $\omega$ is not symplectic.
  So,
  the hypothesis of transitivity of $\Diff(M,\omega)$ on $M$ is not superfluous is this proposition.
  
  \begin{proof}
    Let us assume first that $\omega$ is nondegenerate,
    that is symplectic.
    So,
    the group $\Diff(M,\omega)$ is transitive on $M$ \cite{Boo69}.
    Moreover,
    for every $m \in M$,
    the orbit map $\hat m : \varphi \mapsto \varphi(m)$ is a subduction \cite{Don84}.
    So,
    the image of moment moment map $\mu_\omega$ is one orbit $\cO_\omega$ of the affine coadjoint action of $G_\omega$ on $\cG^*_\omega/\Gamma_\omega$,
    associated to the cocycle $\theta_\omega$.
    Thus,
    for the orbit $\cO_\omega$ equipped with the quotient diffeology of $G_\omega$,
    the moment map $\mu_\omega$ is a subduction.
    
    Now,
    let $m_0$ and $m_1$ two points of $M$ such that $\mu_\omega(m_0) = \mu_\omega(m_1)$,
    that is $\psi_\omega(m_0,m_1) = \mu_\omega(m_1) - \mu_\omega(m_0) = 0$.
    Let $p \in \Paths(M)$ such that $p(0)= m_0$ and $p(1) = m_1$.
    Thus,
    $\psi_\omega(m_0,m_1) = 0$ is equivalent to $\Psi_\omega(p) = \Psi_\omega(\ell)$,
    where $\ell$ is some loop of $M$,
    we can choose $\ell(0)=\ell(1)=m_0$.
    Now,
    let us assume that $m_0 \neq m_1$.
    Since $M$ is Hausdorff there exists a smooth real function $f \in \Cinfty(M, \RR)$,
    with compact support,
    such that $f(m_0) = 0$ and $f(m_1) = 1$.
    Let us denote by $\xi$ the symplectic gradient field associated to $f$ and by $F$ the exponential of $\xi$.
    Thanks to \ref{The-paths-moment-maps-for-symplectic-manifolds},
    on one hand we have $\Psi(p)(F) = [f(m_1) - f(m_0)] dt = dt$,
    and on the other hand $\Psi_\omega(\ell)(F) = [f(m_0) - f(m_0)] dt = 0$.
    But $dt \neq 0$,
    thus $\psi_\omega(m_0,m_1) \neq 0$,
    and the moment map $\mu_\omega$ is injective.
    Therefore,
    $\mu_\omega$ is an injective subduction on $\cO_\omega$,
    that is a diffeomorphism.
    
    Conversely,
    let us assume that $M$ is an homogeneous space of $\Diff(M,\omega)$ and $\mu_\omega$ is injective.
    Let us notice first that,
    since $\Diff(M,\omega)$ is transitive,
    the rank of $\omega$ is constant.
    In other words,
    $\dim \ker \omega = \const$.
    Now,
    let us assume that $\omega$ is degenerated,
    that is $\dim(\ker \omega) \geq 1$.
    Since $m \mapsto \ker \omega_m$ is a smooth foliation,
    for any point $m$ of $M$ there exists a smooth path $p$ of $M$ such that $p(0) = m$ and for $t$ belonging to a small interval around $0 \in \RR$,
    $\dot p (t) \neq 0$ and $\dot p(t) \in \ker \omega_{p(t)}$ for all $t$ in this interval.
    So,
    we can re-parametrize the path $p$ and assume now that $p$ is defined on the whole $\RR$ and satisfies $p(0) = m$,
    $p(1) = m'$ with $m \neq m'$,
    and $\dot p(t) \in \ker \omega_{p(t)}$ for all $t$.
    Now,
    since $\dot p(t) \in \ker \omega_{p(t)}$ for all $t$,
    using the expression (\ref{eq:diamondsuit2}) given in \ref{Value-of-the-moment-maps-for-manifolds},
    we get $\Psi_\omega(p) = 0_{\cG^*_\omega}$ and thus $\mu_\omega(m) = \mu_\omega(m')$.
    But $m \neq m'$ and we have assumed that $\mu_\omega$ is injective.
    So the kernel of $\omega$ is reduced to $\{0\}$,
    $\omega$ is nondegenerate,
    that is symplectic.
    
    Let us finish by proving the remark.
    That is,
    the universal moment map $\mu_\omega$ of $\omega = (x^2 + y^2)\ dx \wedge dy$ is injective.
    First of all $\mu_\omega(0,0) = 0_{\cG^*}$.
    Now if $z=(x,y)$ and $z' =(x',y')$ are two different points of $\RR^2$ and different from $(0,0)$,
    there is a smooth function with compact support contained in a small ball not containing $(0,0)$ nor $z$ and such that $f(z')=1$.
    So the 1-parameter group generated by $\grad_\omega(f)$ belongs to $\Diff(\RR^2,\omega)$,
    and then a similar argument as the one of the proof above shows that $\mu_\omega(z) \neq \mu_\omega(z')$.
    Now it remains to prove that if $z \neq (0,0)$,
    $\mu_\omega(z) \neq 0_{\cG^*}$.
    Let us consider $p(t) = tz$ and $F(r)$ be the positive rotation of angle $2\pi r$,
    where $r \in \RR$.
    The application of the formula (\ref{eq:diamondsuit2}) of \ref{Value-of-the-moment-maps-for-manifolds},
    computed at the point $r=0$ and applied to the vector $\delta r =1$ gives $(2\pi/3) (x^2+y^2)^2$ which is not zero.
    So,
    the moment map $\mu_\omega$ is injective.
  \end{proof}
  
  \article{Restriction to hamiltonian diffeomorphisms}
  \label{Restriction-to-hamiltonian-diffeomorphisms}
  Let $(M,\omega)$ be a connected Hausdorff symplectic manifold.
  Let $\Ham(M,\omega)$ be the group of hamiltonian diffeomorphisms,
  and let $\cH^*_\omega$ be the space of its momenta.
  Let $\mu_\omega^\star : M \to \cH^*_\omega$ be any moment map associated to the action of $\Ham(M,\omega)$,
  and let $\theta_\omega^\star$ be the associated Souriau cocycle.
  So,
  $\mu_\omega^\star$ is injective,
  and identifies $M$ to a $\theta_\omega^\star$-coadjoint orbit of $\Ham(M,\omega)$ in $\cH_\omega$.
  
  \begin{proof}
    It is known also that the group $\Ham(M,\omega)$ acts transitively on $M$ \cite{Boo69}.
    With respect to that group,
    and by construction,
    the holonomy is trivial:
    the associated paths moment map $\Psi_\omega^\star$ and the moment maps $\mu_\omega^\star$ take their values in the space $\cH^*_\omega$.
    Let $j : \Ham(M,\omega) \to \Diff(M,\omega)$ be the inclusion,
    so the universal holonomy $\Gamma_\omega$ is in the kernel of $j^*$,
    and we get a natural mapping $j^*_{\Gamma_\omega} : \cG^*_\omega/\Gamma_\omega \to \cH^*_\omega$.
    Now,
    the paths moment maps satisfy $\Psi_\omega^\star = j^*_{\Gamma_\omega} \circ \Psi_\omega$,
    and $\mu_\omega^\star = j^*_{\Gamma_\omega} \circ \mu_\omega$,
    see \ref{Universal-moment-maps}.
    Then,
    since the \ref{The-paths-moment-maps-for-symplectic-manifolds} involves only plots of $\Ham(X,\omega)$,
    the first part of the proof of \ref{Moment-maps-for-symplectic-manifolds} applies {\em mutatis mutandis} to the hamiltonian case and we deduce that the moment maps $\mu_\omega^\star$ are injective and identify $M$ with some $\theta_\omega^\star$-coadjoint orbits of $\Ham(M,\omega)$.
  \end{proof}
  
  \article{Hamiltonian diffeomorphisms of symplectic manifolds}
  \label{Hamiltonian-diffeomorphisms-on-symplectic-manifolds}
  Let $(M,\omega)$ be a con\-nected Hausdorff symplectic manifold.
  According to Banayaga,
  a diffeomorphism $f$ is said to be {\em hamiltonian\/} if it can be connected to the identity $\id_M$ by a smooth path $t \mapsto f_t$ in $\Diff(M, \omega)$ such that
  $$
    \omega(\dot f_t, \cdot) = d\phi_t \qmbox{with} \dot f_t(x) = {d \over ds}\bigg\{f_s \circ f_t^{-1}(x)\bigg\}_{s=t},
    $$
  where $(t,x) \mapsto \phi_t(x)$ is a smooth real function,
  see \cite{Ban78}.
  If,
  according to this definition,
  $f$ is hamiltonian then it is an element of $\Ham(M, \omega)$,
  as defined in \ref{The-group-of-hamiltonian-diffeomorphisms}.
  Conversely,
  any element $f$ of $\Ham(M, \omega)$ satisfies the condition above.
  So,
  the definition of hamiltonian diffeomorphisms given in \ref{The-group-of-hamiltonian-diffeomorphisms} is a faithful generalization of the classical definition for symplectic manifolds.
  Note that the technical requirement of compacity of the original definition ({\em op cit}) doesn't play any role in this characterization of hamiltonian diffeomorphisms.
  
  \begin{proof}
    This proposition is a direct consequence of the general statement given in \ref{Time-dependent-hamiltonian} and the following comparison between the above 1-parameter family of vector fields $\dot f_t$ and the family $F_t$ of the \ref{Time-dependent-hamiltonian}.
    
    Since $f_{t'} \circ f_t^{-1} = f_t \circ (f_t^{-1} \circ f_{t'}) \circ f_t^{-1}$,
    the vector fields $\dot f_t$ and $F_t$ are conjugated by $f_t$,
    precisely:
    $$
      \dot f_t = (f_t)_*(F_t) \qmbox{or} \dot f_t(x) = \D(f_t)(f_t^{-1}(x))(F_t(f_t^{-1}(x))).
      $$
    This implies in particular that if the vector field $\dot f_t$ satisfies Banyaga's condition for the function $\phi_t$ then the vector field $F_t$ satisfies Banyaga's condition for the function $\Phi_t = -\phi_t \circ f_t$,
    and conversely.
    That is:
    $$
      \omega(\dot f_t, \cdot) = d\phi_t \quad \Leftrightarrow \quad \omega(F_t, \cdot) = - d\Phi_t \qmbox{with} \Phi_t = - \phi_t \circ f_t.
      $$
    Indeed,
    let $x \in M$,
    $x' = f_t(x)$,
    $\delta x \in \T_x M$,
    and $\delta x' = \D(f_t)(x)(\delta x)$,
    we have:
    \begin{eqnarray*}
      \omega_{x'} (\dot f_t(x'), \delta x') & =& [d\phi_t]_{x'}(\delta x') \\
      \omega_{f_t(x)}(\dot f_t (f_t(x)), \D(f_t)(x)(\delta x)) & =& [d\phi_t]_{f_t(x)}(\D(f_t)(x)(\delta x)) \\
      \omega_{f_t(x)}(\D(f_t)(x)(F_t(x)), \D(f_t)(x)(\delta x)) & =& [f_t^*(d\phi_t)]_x(\delta x) \\
      {[f_t^*(\omega)]}_x(F_t(x), \delta x) &=& d[f_t^*(\phi_t)]_x(\delta x) \\
      \omega_x(F_t(x),\delta x) & = & d[\phi_t \circ f_t]_x(\delta x).
    \end{eqnarray*}
    Thus,
    we get $\Phi_t = - \phi_t \circ f_t$.
  \end{proof}
  
  %%%%%%%%%%%%%%%%%%%%%%%%%%%%%%%%%%%%%%%%%%%%%%%%%%%%%%%%%%
  \section{The homogeneous case}
  %%%%%%%%%%%%%%%%%%%%%%%%%%%%%%%%%%%%%%%%%%%%%%%%%%%%%%%%%%
  
  As it is suggested by \ref{Moment-maps-for-symplectic-manifolds},
  the case of an homogeneous action of a diffeological group $G$ on a space $X$,
  preserving a closed 2-form $\omega$,
  deserves a special attention.
  
  \article{The homogeneous case}
  \label{The-homogeneous-case}
  Let $X$ be a connected diffeological space equipped with a closed 2-form $\omega$.
  Let $\rho$ be a smooth action of a diffeological group $G$ on $X$,
  preserving $\omega$.
  Let us assume that $X$ is homogeneous for this action,
  see \ref{Smooth-actions-of-a-diffeological-group}.
  Let $\Gamma$ be the holonomy of the action $\rho$,
  let $\mu$ be a moment,
  and let $\theta$ be the cocycle associated to $\mu$.
  Let $x_0$ be any point of $X$,
  and let $\mu_0 = \mu(x_0)$.
  Let $\Stab_{\Ad_*^{\Gamma,\theta}}(\mu_0)$ be the stabilizer of $\mu_0$ for the affine coadjoint action of $G$ on $\cG^*\!/\Gamma$.
  Thanks to the equivariance of the moment map $\mu$,
  with respect to the affine coadjoint action of $G$ on $\cG^*\!/\Gamma$,
  $\mu \circ \rho(g) = \Ad_*^{\Gamma,\theta}(g) \circ \mu$,
  the image $\cO = \mu(X)$ is a $(\Gamma,\theta)$-orbit of $G$.
  Let us equip $\cO$ with the quotient diffeology of $G$,
  such that $\cO \simeq G/\Stab_{\Ad_*^{\Gamma,\theta}}(\mu_0)$.
  So,
  the orbit map $\hat x_0 : G \to X$ is a principal fibration with structure group $\Stab_\rho(x_0)$,
  the orbit map $\hat \mu_0 : G \to \cO$ is a principal fibration with structure group $\Stab_{\Ad_*^{\Gamma,\theta}}(\mu_0)$,
  and $\Stab_\rho(x_0) \subset \Stab_{\Ad_*^{\Gamma,\theta}}(\mu_0)$.
  So,
  the moment map $\mu : X \to \cO$ is a fibration with fiber,
  the homogeneous space $\Stab_{\Ad_*^{\Gamma,\theta}}(\mu_0)/\Stab_\rho(x_0)$.
  \[
  \begin{tikzcd}[column sep=large, row sep=large, every label/.append style = {font = \small}]
    G \arrow[r, "\hat x_0"] \arrow[d, "\hat \mu_0"'] & X \arrow[dl, "\mu"] \\
    \cO
  \end{tikzcd}
  \quad \quad \quad
  \begin{tikzcd}[column sep=large, row sep=large, every label/.append style = {font = \small}]
    G \arrow[r, "\Stab_\rho(x_0)"] \arrow[d, "\Stab_{\Ad_*^{\Gamma,\theta}}(\mu_0)"'] & X \arrow[dl, "\Stab_{\Ad_*^{\Gamma,\theta}}(\mu_0)/\Stab_\rho(x_0)"] \\
    \cO
  \end{tikzcd}
  \]
  {\sc Note} --- The moment maps $\mu$ are defined up to a constant,
  but the {\em characteristics} of $\mu$,
  that is the subspaces defined by $\mu(x) = \const$,
  are not.
  They are the solutions of the equation $\psi(x_0,x) = 0$,
  where $\const = \mu(x_0)$ and $\psi$ is the 2-points moment map.
  
  \begin{proof}
    This is just an application of standard diffeological relations.
  \end{proof}
  
  \article{Symplectic homogeneous diffeological spaces}
  \label{Symplectic-homogeneous-diffeological-spaces}
  Let $X$ be a connected diffeological space and $\omega$ be a closed 2-form defined on $X$.
  
  {\sc Definition} We say that $(X,\omega)$ is an {\em homogeneous symplectic space} if it is homogeneous under the action of $\Diff(X,\omega)$ and if a universal moment map $\mu_\omega$ is a covering onto its image.
  
  The homogeneous situation where the moment maps $\mu_\omega$ are not coverings onto their images can be regarded as the {\em homogeneous pre-symplectic} case.
  
  Now,
  let $G$ be some diffeological group,
  and let $\rho$ be a smooth action of $G$ on $X$,
  preserving $\omega$.
  So,
  if the action $\rho$ of $G$ on $X$ is homogeneous,
  then $X$ is an homogeneous space of $\Diff(X,\omega)$.
  And,
  if a moment map $\mu : X \to \cG^*\!/\Gamma$ is a covering onto its image,
  then any universal moment map $\mu_\omega : X \to \cG^*_\omega/\Gamma_\omega$ is a covering onto its image.
  
  Thus,
  to check that an homogeneous pair $(X,\omega)$ is symplectic it is sufficient to find a smooth homogeneous smooth action of some diffeological group $G$ for which one moment map is a covering onto its image.
  
  \begin{proof}
    To be homogeneous under the action of $G$ means that,
    for some point (and thus for any point) $x \in X$,
    the orbit map $\hat x : G \to X$,
    defined by $\hat x (g) = \rho(g)(x)$,
    is a subduction.
    So,
    $\hat x$ is surjective and,
    for any plot $P : U \to X$,
    for any $r_0 \in U$,
    there exists a superset $V$ of $r_0$ and a plot $Q : V \to G$ such that $P \restriction V = \hat x \circ Q$.
    That is,
    $P (r) = \rho(Q(r))(x)$ for all $r \in V$.
    Since $\rho$ is smooth,
    $\bar Q = \rho \circ Q$ is a plot of $\Diff(X,\omega)$,
    and $P \restriction V = \hat x \circ \bar Q$.
    Since,
    $\hat x : \Diff(X,\omega) \to X$ is surjective,
    it is a subduction and $X$ is an homogeneous space of $\Diff(X,\omega)$.
    
    Now,
    let us remark that,
    since the moment maps differ just by a constant,
    if a moment map $\mu$ is a covering onto its image $\cO$ equipped with the quotient diffeology of $G$,
    then every other moment map $\mu' = \mu + \const$ is a covering onto its image $\cO' = \cO + \const$.
    So,
    let $x_0$ be a point of $X$,
    and let $\mu(x) = \psi(x_0,x)$,
    where $\psi$ is the 2-points moment map.
    Let $\mu_\omega = \psi_\omega(x_0,x)$.
    According to \ref{Universal-moment-maps},
    $\mu = \rho^*_{\Gamma_\omega} \circ \mu_\omega$.
    Let $\cO = \mu(X)$ and $\cO_\omega = \mu_\omega(X)$,
    equipped with the quotient diffeologies of $G$ and $G_\omega = \Diff(X,\omega)$.
    So,
    $\cO = \rho^*_{\Gamma_\omega}(\cO_\omega)$.
    Let $m \in \cO$ and $m_\omega \in \cO_\omega$ such that $\rho^*_{\Gamma_\omega}(m_\omega) = m$.
    So,
    $\mu_\omega^{-1}(m_\omega) = \{x \in X \mid \mu_\omega(x) = m_\omega \} \subset \mu^{-1}(m) = \{ x \in X \mid \mu(x) = \rho^*_{\Gamma_\omega}(\mu_\omega(x)) = m \}$.
    Thus,
    if $\mu^{-1}(m)$ is discrete,
    a fortiori $\mu_\omega^{-1}(m_\omega) \subset \mu^{-1}(m)$.
    Thus,
    if $\mu$ is a fibration onto its image,
    then $\mu_\omega$ is a fibration onto its image too.
    And of course if $\mu$ is injective,
    a fortiori $\mu_\omega$.
  \end{proof}
  
  %%%%%%%%%%%%%%%%%%%%%%%%%%%%%%%%%%%%%%%%%%%%%%%%%%%%%%%%%%
  \section{Examples of moment maps in diffeology}
  \label{Examples-of-moment-maps-in-diffeology}
  %%%%%%%%%%%%%%%%%%%%%%%%%%%%%%%%%%%%%%%%%%%%%%%%%%%%%%%%%%
  
  This short list of examples shows how the theory of moment map in diffeology can be applied to the folklore of infinite dimensional situations,
  but also to the less familiar cases of singular spaces.
  
  \article{The moments of imprimitivity}
  \label{The-moments-of-imprimitivity}
  Let $X$ be a diffeological space.
  Let us remind,
  and make some preliminary remarks on,
  the construction of the {\em cotangent bundle} and the definition of the Liouville form \cite{Piz05}.
  Let $\Omega^1(X)$ denotes the vector space of $1$-form of $X$,
  equipped with the functional diffeology.
  The mapping $\Taut$,
  which associates to each $n$-plot $Q \times \P$ of the product $X \times \Omega^1(X)$ the 1-form
  $$
    \Taut(P \times Q) : r \mapsto P (r)(Q)_r
    $$
  of $\dom(Q \times P)$,
  is a 1-form of $X \times \Omega^1(X)$.
  We call it the {\em tautological form}.
  
  Now,
  let us consider the $\Value$ equivalence relation.
  Let $\alpha$ and $\alpha'$ be two 1-forms of $X$,
  let $x$ be a point of $X$.
  We say that $\alpha$ and $\alpha'$ {\em have the same value} at the point $x$,
  and we denote $\Value(\alpha)(x) = \Value(\alpha')(x)$,
  if and only if,
  for every plot $Q$ of $X$ centered\footnote{We say that a plot $Q$ is {\em centered\ } at $x$ if and only if $0 \in \dom(Q)$ and $Q(0) = x$.} at $x$ ,
  $\alpha(Q)_0 = \alpha'(Q)_0$.
  Then,
  the {\em cotangent bundle} of $X$ is defined as the quotient $X \times \Omega^1(X)$ by the relation $\Value$,
  and denoted\footnote{Note that, as well as for the notation $\cG^*$ of the space of momenta of a diffeological group, the star in $\T^*X$ do not rely to any kind of duality a priori.} by $\T^*X$,
  $$
    \T^*X = X \times \Omega^1(X) / \Value.
    $$
  This notion of value,
  for smooth forms on numerical domains,
  coincides with the ordinary definition.
  So,
  when there will be no risk of confusion\footnote{This notation $\alpha(x)$ has not to be mixed up with the notation $\alpha(Q)$ for the value of $\alpha$ in the plot $Q$. But the different nature of $x$: a point of $X$, and $Q$: a plot of $X$, makes the difference.},
  we shall denote simply by $\alpha(x)$ the value of $\alpha$ at the point $x$,
  that is $\alpha(x) = \Value(\alpha)(x)$.
  
  Let $\pr : X \times \Omega^1(X) \to \T^*X$ be the canonical projection.
  So,
  there exists a 1-form on $\T^*X$,
  denoted by $\Liouv$ and called the {\em Liouville form} such that
  $$
    \Taut = \pr^*(\Liouv) \qmbox{or} \Liouv = \pr_*(\Taut), \quad \Liouv \in \Omega^1(\T^*X).
    $$
  The characteristic property of the Liouville form is the following.
  Let $\alpha$ be a 1-form of $X$,
  let $\bar \alpha$ be the section of the canonical projection $\pi : \T^*X \to X$ defined by $\bar \alpha : x \mapsto \Value(\alpha)(x)$,
  so $\alpha = \bar \alpha^*(\Liouv)$.
  Note also that,
  the group $\Diff(X)$ acts naturally on the product $X \times \Omega^1(X)$ by $\bar \varphi(x,\alpha) = (\varphi(x), \varphi_*(\alpha))$,
  where $\varphi$ is a diffeomorphism of $X$.
  So,
  the tautological form is invariant by this action.
  Moreover,
  this action is compatible with the relation $\Value$,
  and the group $\Diff(X)$ has a natural projected action on $\T^*X$.
  By equivariance,
  the Liouville form is invariant by this action.
  Note that,
  the moment map for the action of $\Diff(X)$ on $(\T^*X, d\Liouv)$ is given by the general construction of \ref{The-exact-case}.
  This can be compared to Donato's construction for manifolds in \cite{Don88}.
  
  Now,
  let us introduce the additive diffeological group of smooth functions $\Cinfty(X,\RR)$,
  acting smoothly on $X \times \Omega^1(X)$ by,
  $$
    \bar f : (x, \alpha) \mapsto (x , \alpha + df),
    $$
  for all $f \in \Cinfty(X,\RR)$.
  This action projects naturally on the cotangent $\T^*X$ into an action,
  denoted by the same way,
  $$
    \bar f : (x,a) \mapsto (x, a + df(x)),
    $$
  for all $(x,a) \in \T^*X$.
  So,
  \begin{enumerate}
    \item For all $f \in \Cinfty(X,\R)$, the variance of the tautological form and the Liouville form are given by,
    $$
      \bar f ^*(\Taut) = \Taut + \pr_1^*(df) \qmbox{and} \bar f^*(\Liouv) = \Liouv + \pi^*(df).
      $$
    So, the exterior differentials $d\Taut$ and $\omega = d\Liouv$ are invariant by the action of $\Cinfty(X,\RR)$.
    \item Let $p$ be a path of $\T^*X$, connecting $(x_0,a_0) = p(0)$ to $(x_1,a_1) = p(1)$. So, the paths moment map $\Psi$ and the 2-points moment map $\psi$, with respect to the 2-form $\omega = d \Liouv$, are given by
    $$
      \Psi(p) = \psi((x_0,a_0),(x_1,a_1)) = d[f \mapsto f(x_0)] - d[f \mapsto f(x_1)].
      $$
    \item For every $x \in X$, the real function $[f \mapsto f(x)]$ is smooth. We call it the {\em Dirac function} of the point $x$, and we denote it by $\delta_x$.
    $$
      \delta_x = [f \mapsto f(x)] \in \Cinfty(\Cinfty(X,\RR),\RR).
      $$
    The differential $d \delta_x = d[f \mapsto f(x)]$ is an invariant 1-form\footnote{This differential has nothing to do with the derivative of the Dirac distributions in the sense of De Rham's currents.} of the additive group $\Cinfty(X,\RR)$. Every moment map of the action of $\Cinfty(X,\R)$ on $\T^*X$ is cohomologous to the invariant moment map
    $$
      \mu : (x,a) \mapsto - d \delta_x.
      $$
    Note that, the moment $\mu$ is constant on the fibers $\T_x^*X = \pi^{-1}(x)$. And, if the real smooth functions separate\footnote{That is, $f(x) = f(x')$ for all smooth real function $f$ if and only if $x=x'$.} the points of $X$, the image of the moment map $\mu$ is the space $X$, identified with the space of Dirac's functions.
    \item The action of $\Cinfty(X,\RR)$ on $(\T^*X,\omega)$ is hamiltonian and exact. That is, $\Gamma = \{0\}$ and $\sigma = 0$.
  \end{enumerate}
  
  This example has been drawn to my attention by Fran\c{c}ois Ziegler.
  This moment appears informally in Ziegler's construction of a symplectic analogue for ``systems of imprimitivity'' in representation theory \cite{Zie96}.
  It is why the moment map $\mu$ will be called the {\em moment of imprimitivity}.
  The diffeological framework gives it so a full formal status.
  
  \begin{proof}
    First of all let us check the variance of $\Taut$ by the action of $\Cinfty(X,\RR)$.
    Let $f$ be a smooth real function defined on $X$,
    let $Q \times \P$ be a plot of $X \times \Omega^1(X)$.
    We have $\bar f^*(\Taut)(P \times Q)_r = \Taut(\bar f \circ (Q \times P))_r = (P (r) + df)(Q)_r = P (r)(Q)_r + df(Q)_r = \Taut(Q \times P)_r + df(\pr_1 \circ (Q \times P))_r = \Taut(Q \times P)_r + \pr_1^*(df)(Q \times P)_r$.
    So,
    $\bar f^*(\Taut) = \Taut + \pr_1^*(df)$.
    Now let us check that this action is compatible with the $\Value$ relation.
    Let $(x,\alpha)$ and $(x',\alpha')$ be two elements of $X \times \Omega^1(X)$ such that $\Value(\alpha)(x) = \Value(\alpha')(x')$.
    That is,
    $x = x'$ and for every plot $Q$ of $X$ centered at $x$,
    $\alpha(Q)_0 = \alpha'(Q)_0$.
    So,
    $(\alpha +df)(Q)_0 = (\alpha' + df)(Q)_0$ and $\Value(\alpha + df)(x) = \Value(\alpha)(x) + \Value(df)(x)$,
    or $(\alpha + df)(x) = \alpha(x) + df(x)$.
    Thus,
    the action of $\Cinfty(X,\RR)$ projects on $\T^*X$ as the action $\bar f : (x,a) \mapsto a + df(x)$.
    Now,
    since $\bar f^*(\Taut) = \Taut + \pr_1^*(df)$,
    clearly $\bar f^*(\Liouv) = \Liouv + \pi^*(df)$.
    Or,
    in another way,
    $\bar f^*(\Liouv) = \Liouv + dF(f)$ with $F \in \Cinfty(\Cinfty(X,\RR),\Cinfty(\T^*X,\RR))$ and $F(f) = \pi^*(f) =f \circ \pi$.
    
    Let us denote by $\R(x,a)$ the orbit map $f \mapsto a + df(x)$.
    Let $p$ be a path of $\T^*X$ such that $p(0) = (x_0,a_0)$ and $p(1) = (x_1,a_1)$.
    We get
    \begin{eqnarray*}
      \Psi(p) & = & \hat p^*(\cK d\Liouv) \\
      & = & \hat p^*(\but^*(\Liouv) - \source^*(\Liouv) - d \cK\Liouv) \\
      & = & (\but \circ \hat p)^*(\Liouv) - (\source \circ \hat p)^*(\Liouv) - d [(\cK\Liouv) \circ \hat p] \\
      & = & \R(x_1,a_1)^*(\Liouv) - \R(x_0,a_0)^*(\Liouv) - d [f \mapsto \cK\Liouv (\hat p(f))].
    \end{eqnarray*}
    Let us consider first the term $[f \mapsto \cK\Liouv (\hat p(f))]$.
    Let $p(t) =(x_t,a_t)$,
    so $\hat p(f) = [t \mapsto (x_t,a_t + df(x_t))]$.
    Thus,
    \begin{eqnarray*}
      \cK\Liouv(\hat p(f))) & = &\int_0^1 a_t[s \mapsto x_s]_{s=t} \ dt + \int_0^1 df[t \mapsto x_t] \ dt \\
      & = & \int_0^1 a_t[s \mapsto x_s]_{s=t} \ dt + f(x_1) - f(x_0).
    \end{eqnarray*}
    Thus,
    \begin{eqnarray*}
      d [f \mapsto \cK\Liouv (\hat p(f))] &=& d[f \mapsto \int_0^1 a_t[s \mapsto x_s]_{s=t} \ dt + f(x_1) - f(x_0)] \\
      & = & d[f \mapsto f(x_1) - f(x_0)].
    \end{eqnarray*}
    Let us compute now $\R(x,a)^*(\Liouv)$,
    for any $(x,a) \in \T^*X$.
    Let $\P: U \to \Cinfty(X,\RR)$ be a plot.
    We have
    \begin{eqnarray*}
      \R(x,a)^*(\Liouv)(P) & = & \Liouv (\R(x,a) \circ P) \\
      & = & \Liouv(r \mapsto \overline{P (r)}(x,a)) \\
      & = & \Liouv(r \mapsto a + d[P (r)](x)) \\
      & = & (a +d[P (r)](x))(r \mapsto x) \\
      & = & 0
    \end{eqnarray*}
    because the 1-form $a +d[P (r)](x)$ is evaluated on the constant plot $r \mapsto x$.
    And,
    every form evaluated to a constant plot vanishes.
    So,
    we get finally
    $$
      \Psi(p) = d[f \mapsto f(x_0)] - d[f \mapsto f(x_1)].
      $$
    Now,
    clearly $\Psi(\ell) = 0$ for every loop $\ell$ of $\T^*X$,
    and the action of $\Cinfty(X,\RR)$ is hamiltonian $\Gamma = \{0\}$.
    So,
    $\psi((x_0,a_0),(x_1,a_1)) = \mu(x_1,a_1) - \mu(x_0,a_0)$,
    with the moment map
    $$
      \mu : (x,a) \mapsto -d[f \mapsto f(x)] = - d \delta_x.
      $$
    Let us check now the invariance of the moment map $\mu$.
    Note that,
    for every $h \in \Cinfty(X,\RR)$,
    we have $\delta_x \circ \L(h) = [f \mapsto f(x) + h(x)]$.
    So,
    for every $h \in \Cinfty(X,\RR)$ we have $\hat h^*(\mu)(x,a) = \hat h^*(-d \delta_x) = -d(\delta_x \circ \L(h)) = - d [f \mapsto f(x) + h(x)] = - d [f \mapsto f(x)] = - d \delta_x = \mu(x,a)$.
    Hence,
    $\mu$ is invariant.
    The 2-points moment map $\psi$ is exact.
    Souriau's class of the action of $\Cinfty(X,\RR)$ on $\T^*X$ vanishes.
  \end{proof}
  
  \article{On the intersection 2-form of a surface I}
  \label{On-the-intersection-form-of-a-surface-I}
  Let $\Sigma$ be a closed surface oriented by a 2-form $\Surf$,
  chosen once and for all.
  Let us consider $\Omega^1(\Sigma)$,
  the infinite dimensional vector space of 1-forms of $\Sigma$,
  equipped with the functional diffeology.
  Let us consider the antisymmetric bilinear map defined on $\Omega^1(\Sigma)$ by
  $$
    (\alpha,\beta) \mapsto \int_\Sigma \alpha\wedge \beta,
    $$
  for all $\alpha$, $\beta$ in $\Omega^1(\Sigma)$.
  Since the wedge-product $\alpha \wedge\beta$ is a 2-form of $\Sigma$,
  there exists a real smooth function $\varphi \in \Cinfty(\Sigma,\RR)$ such that $\alpha \wedge \beta = \varphi \times \Surf$.
  So,
  by definition,
  $\int_\Sigma \alpha \wedge \beta = \int_\Sigma \varphi \times \Surf$.
  
  1) To the above bilinear form is naturally associated a well defined differential 2-form $\omega$ of $\Omega^1(X)$.
  For every $n$-plot $P : U \to X$,
  for all $r \in U$, $\delta r$ and $\delta' r$ in $\RR^n$,
  $$
    \omega(P)_r(\delta r,\delta' r) = \int_\Sigma {\partial P (r) \over \partial r}(\delta r)\wedge {\partial P (r) \over \partial r}(\delta' r)
    $$
  
  2) The 2-form $\omega$ is the differential of the 1-form $\lambda$ defined on $\Omega^1(\Sigma)$ by,
  $$
    \lambda(P)_r(\delta r) = \undemi \int_\Sigma P (r)\wedge {\partial P (r) \over \partial r}(\delta r) \qmbox{and} \omega = d\lambda.
    $$
  3) Let us consider now the the additive group $(\Cinfty(\Sigma,\RR),+)$ of smooth real functions of $\Sigma$.
  And,
  let us define the following action of $\Cinfty(\Sigma,\RR)$ on $\Omega^1(\Sigma)$.
  $$
    \mbox{For all $f \in \Cinfty(\Sigma,\RR)$}, \quad f \mapsto \bar f = [\alpha \mapsto \alpha + df].
    $$
  So,
  the additive group $\Cinfty(\Sigma,\RR)$ acts by automorphisms on the pair $(\Omega^1(\Sigma),\omega)$.
  $$
    \mbox{For all $f$ in $\Cinfty(\Sigma,\RR)$}, \quad f^*(\omega) = \omega.
    $$
  Note that the kernel of the action $f \mapsto \bar f$ is the subgroup of constant maps.
  And,
  the image of $\Cinfty(\Sigma,\RR)$ is just the group $\B^1_{\D\R}(\Sigma)$ of exact 1-forms of $\Sigma$.
  
  4) Let $p \in \Paths(\Omega^1(\Sigma))$ be a path connecting $\alpha_0$ to $\alpha_1$.
  The paths moment map $\Psi(p)$ is given by
  $$
    \Psi(p) = \bigg(\hat \alpha_1^*(\lambda) + d\bigg[f \mapsto \undemi \int_\Sigma f \times d\alpha_1\bigg]\bigg) - \bigg(\hat \alpha_0^*(\lambda) + d \bigg[f \mapsto \undemi \int_\Sigma f \times d\alpha_0 \bigg] \bigg).
    $$
  On this expression,
  we check immediately that the 2-points moment map is just given by $\psi(\alpha_0,\alpha_1) = \Psi(p)$,
  for any path $p$ connecting $\alpha_0$ to $\alpha_1$.
  Note that,
  since $\Omega^1(\Sigma)$ is contractible the holonomy of the action of $\Cinfty(\Sigma,\RR)$ vanishes,
  $\Gamma = \{0\}$,
  the action of $\Cinfty(\Sigma,\RR)$ is hamiltonian.
  
  5) The moment maps of this action of $\Cinfty(\Sigma,\RR)$ on $\Omega^1(\Sigma)$ are,
  up to a constant,
  equal to
  $$
    \mu : \alpha \mapsto d \bigg[ f \mapsto \int_\Sigma f \times d\alpha\bigg].
    $$
  Moreover,
  the moment map $\mu$ is equivariant.
  That is,
  invariant,
  since the group $\Cinfty(\Sigma,\RR)$ is abelian.
  $$
    \mbox{For all $f \in \Cinfty(\Sigma,\RR)$}, \quad \mu \circ \bar f = \mu.
    $$
  So,
  the action of $\Cinfty(\Sigma,\RR)$ on $\Omega^1(\Sigma)$ is exact and hamiltonian.
  
  {\sc Note} --- The moment map $\mu(\alpha)$ is fully characterized by $d\alpha$.
  This is why we find in the mathematical literature on the subject that,
  the moment map for this action is the exterior derivative (or curvature, depending on the authors) $\alpha \mapsto d\alpha$.
  But,
  as we see again on this example,
  diffeology gives to this sketchy assertion a precise meaning.
  
  Let us remark also that,
  the moment map $\mu$ is linear,
  for all $t$, $s$ reals and all $\alpha$ and $\beta$ in $\Omega^1(\Sigma)$,
  $\mu(t\,\alpha+s\,\beta) = t\,\mu(\alpha)+s\,\mu(\beta)$.
  And,
  the kernel of $\mu$ is the subspace of closed 1-forms,
  $$
    \ker(\mu) = \Z^1_{\D\R}(\Sigma) = \left\{ \alpha \in \Omega^1(\Sigma) \ \vert \ d\alpha = 0 \right\}
    $$
  If we consider the orbit of the zero form $0 \in \Omega^1(\Sigma)$ by $\Cinfty(\Sigma,\RR)$,
  this is just the subspace $\B^1(\Sigma,\RR)$,
  which is included in $\ker(\mu) = \Z^1_{\D\R}(\Sigma)$.
  The quotient $\ker(\mu)/ \Cinfty(\Sigma,\RR)$ is just $ \Z^1_{\D\R}(\Sigma) / \B^1_{\D\R}(\Sigma)= H^1_{\D\R}(\Sigma)$,
  and the 2-form $\omega \restriction \ker(\mu)$ is just the pullback of the usual intersection form on $H^1_{\D\R}(\Sigma)$.
  I will discuss,
  in a future work,
  the notion of ``symplectic reduction'' in diffeology.
  
  \begin{proof}
    1) Let us check that $\omega$ defines a differential 1-form on $\Omega^1(\Sigma)$.
    Note that,
    for any $r \in U = \dom(P)$,
    $P (r)$ is a section of the ordinary cotangent bundle $\T^*\Sigma$.
    That is,
    $P (r) = [x \mapsto P (r)(x)] \in \Cinfty(\Sigma,\T^*\Sigma)$,
    where $P (r)(x) \in \T_x^*(\Sigma)$.
    So,
    $$
      {\partial P (r) \over \partial r}(\delta r) = [x \mapsto {\partial P (r)(x) \over \partial r}(\delta r)] \quad \mbox{and} \quad {\partial P (r) (x) \over \partial r}(\delta r) \in {\T}^{*}_{x}(\Sigma)
      $$
    where $\partial P (r)(x) / \partial r$ denotes the tangent linear map $\D(r \mapsto P (r)(x)(r)$.
    And,
    the formula giving $\omega$ is well defined.
    Now,
    $\omega(P)_r$ is clearly antisymmetric and depends smoothly on $r$.
    So,
    $\omega(P)$ is a smooth 2-form of ${U}$.
    Let us check that $P \mapsto \omega(P)$ defines a 2-form on $\Omega^1(\Sigma)$.
    That is,
    satisfies the compatibility condition $\omega(P \circ F) = F^*(\omega(P))$,
    for all $F \in \Cinfty(V, U)$,
    where $V$ is a numerical domain.
    Let $s \in {V}$,
    $\delta s$ and $\delta' s$ two tangent vectors at $s$ at ${V}$,
    let $r = F(s)$:
    \begin{eqnarray*}
      \omega(P \circ F)_s(\delta s,\delta' s) & = & \int_\Sigma {\partial P \circ F(s) \over \partial s}(\delta s)\wedge {\partial P \circ F(s) \over \partial s}(\delta' s) \\
      & = & \int_\Sigma {\partial P (r) \over \partial r}{\partial F(s) \over \partial s}(\delta s)\wedge {\partial P (r) \over \partial r}{\partial F(s) \over \partial s}(\delta' s) \\
      & = & \omega(P)_{F(s)}(\D F_s(\delta s),\D F_s(\delta' s)) \\
      & = & {F}^*(\omega(P))_s(\delta s,\delta' s)
    \end{eqnarray*}
    Thus $\omega(P \circ F) = F^*(\omega(P))$,
    and $\omega$ is a well defined 2-form on $\Omega^1(\Sigma)$.
    
    2) First of all,
    the proof that the map $P \mapsto \lambda(P)$ is a well defined differential 1-form of $\Omega^1(\Sigma)$ is analogous to the proof of the first item.
    Now,
    let us remind that $\omega = d\lambda$ is and only if $d(\lambda(P)) = \omega(P)$ for all plot $P$ of $\Omega^1(\Sigma)$.
    Let us apply the usual formula of differentiation of 1-form on numerical domain,
    $$d\epsilon_r(\delta r,\delta' r) = \delta(\epsilon_r(\delta' r)) - \delta'(\epsilon_r(\delta r))$$
    where $\delta$ and $\delta'$ are to commuting variations.
    For the sake of simplicity let us denote
    $$
      \alpha = P (r), \quad \delta\alpha = {\partial P (r) \over \partial r}(\delta r), \quad \delta'\alpha = {\partial P (r) \over \partial r}(\delta' r).
      $$
    So,
    \begin{eqnarray*}
      d(\lambda(P))_r(\delta r,\delta' r) & = & \undemi \bigg[\delta \int_\Sigma \alpha \wedge \delta' \alpha - \delta' \int_\Sigma \alpha \wedge \delta \alpha \bigg] \\
      & = & \undemi \bigg[\int_\Sigma \delta \alpha \wedge \delta' \alpha + \alpha \wedge \delta \delta' \alpha - \int_\Sigma \delta' \alpha \wedge \delta \alpha + \alpha \wedge \delta' \delta \alpha \bigg].
    \end{eqnarray*}
    but,
    $\delta \delta' \alpha = \delta' \delta \alpha$.
    So,
    \begin{eqnarray*}
      d(\lambda(P))_r(\delta r,\delta' r) & = & \undemi \bigg[\int_\Sigma \delta \alpha \wedge \delta' \alpha - \int_\Sigma \delta' \alpha \wedge \delta\alpha \bigg] \\
      & = & \undemi \bigg[\int_\Sigma \delta \alpha \wedge \delta' \alpha + \int_\Sigma \delta \alpha \wedge \delta' \alpha \bigg] \\
      & = & \int_\Sigma \delta \alpha \wedge \delta' \alpha \\
      & = & \omega_r(\delta r,\delta' r).
    \end{eqnarray*}
    
    3) Let us compute the pullback of $\lambda$ by the action of $f \in \Cinfty(\Sigma,\RR)$.
    Let $P : U \to \Omega^1(\Sigma)$ be a $n$-plot,
    let $r \in U$ and $\delta r \in \RR^n$.
    \begin{eqnarray*}
      \bar f^*(\lambda)(P)_r(\delta r) & = & \lambda( \bar f \circ P)_r(\delta r) \\
      & = & \lambda(r \mapsto P (r) + df)_r (\delta r) \\
      & = & \undemi \int_\Sigma (P (r) + df) \wedge {\partial (P (r) +\alpha) \over \partial r}(\delta r)\\ & = & \undemi \int_\Sigma P (r) \wedge {\partial P (r)\over \partial r}(\delta r) + \undemi \int_\Sigma df \wedge {\partial P (r) \over \partial r}(\delta r) \\
      & = & \lambda(P)_r(\delta r) + {\partial \over \partial r} \bigg\{\undemi \int_\Sigma df \wedge P (r)\bigg\}(\delta r) \\
      & = & \lambda(P)_r(\delta r) - {\partial \over \partial r}\bigg\{ \undemi \int_\Sigma f \times d(P (r))\bigg\}(\delta r) \\
    \end{eqnarray*}
    So,
    for every $f \in \Cinfty(\Sigma,\RR)$,
    let us define the map $\varphi(f) : \Omega^1(\Sigma) \to \RR$ by,
    $$
      \varphi(f) : \alpha \mapsto \undemi \int_\Sigma f \times d\alpha.
      $$
    So,
    $$
      d(\varphi(f))(P)_r(\delta r) = {\partial \over \partial r} \bigg\{ \undemi \int_\Sigma f \times d(P (r)) \bigg\}(\delta r).
      $$
    Thus,
    $$
      \bar f^*(\lambda)(P)_r(\delta r) = \lambda(P)_r(\delta r) - (d\varphi(f))(P)_r(\delta r).
      $$
    That is,
    $$
      \bar f^*(\lambda) = \lambda - d(\varphi(f)).
      $$
    Therefore,
    differential $\omega = d\lambda$ is invariant by the action of $\Cinfty(\Sigma,\RR)$.
    
    4) Let $p$ be a path of $\Omega^1(\Sigma)$ connecting $\alpha_0$ to $\alpha_1$.
    By definition $\Psi(p) = \hat p^*(\cK\omega)$.
    Applying the property of the chain-homotopy operator $d \circ \cK + \cK \circ d = \but^* - \source^*$ to $\omega = d\lambda$,
    we get
    \begin{eqnarray*}
      \Psi(p) & = & \hat p^*(\cK d\lambda) \\
      & = & \hat p^*(\but^*(\lambda) - \source^*(\lambda) - d (\cK \lambda)) \\
      & = & (\but \circ \hat p)^*(\lambda) - (\source \circ \hat p)^*(\lambda) - d [(\cK \lambda) \circ \hat p] \\
      & = & \hat \alpha_1^*(\lambda) - \hat \alpha_0^*(\lambda) - d[f \mapsto \cK\lambda(\hat p(f)) ]
    \end{eqnarray*}
    But,
    $\cK\lambda(\hat p(f)) = \cK\lambda(\bar f \circ p) = \int_{\bar f \circ p} \lambda = \int_p \bar f^*(\lambda)$,
    and since $\bar f^*(\lambda) = \lambda - d(\varphi(f))$ we have $\cK\lambda(\hat p(f)) = \int_p \lambda - \int_p d(\varphi(f)) = \int_p \lambda - \varphi(f)(\alpha_1) + \varphi(f)(\alpha_0)$.
    Therefore,
    \begin{eqnarray*}
      \Psi(p) & = & \hat \alpha_1^*(\lambda) - \hat \alpha_0^*(\lambda) - d[f \mapsto - \varphi(f)(\alpha_1) + \varphi(f)(\alpha_0)] \\
      & = & \hat \alpha_1^*(\lambda) - \hat \alpha_0^*(\lambda) + d\bigg[ f \mapsto \undemi \int_\Sigma f \times d\alpha_1 - \undemi \int_\Sigma f \times d \alpha_0 \bigg]
    \end{eqnarray*}
    And,
    finally we get the paths moment map $\Psi$ given by
    $$
      \Psi(p) = \bigg(\hat \alpha_1^*(\lambda) + d \bigg[f \mapsto \undemi \int_\Sigma f \times d\alpha_1 \bigg]\bigg) - \bigg(\hat \alpha_0^*(\lambda) + d \bigg[f \mapsto \undemi \int_\Sigma f \times d \alpha_0 \bigg]\bigg)
      $$
    For the the 2-points moment map $\psi$,
    we have clearly $ \psi(\alpha_0,\alpha_1) = \Psi(p)$ for any path connecting $\alpha_0$ to $\alpha_1$.
    
    5) The 1-point moment maps are given by $\mu(\alpha) = \psi(\alpha_0,\alpha)$ for any origin $\alpha_0$.
    Let us choose $\alpha_0 = 0$.
    So,
    $$
      \mu(\alpha) = \hat \alpha^*(\lambda) + d \bigg[f \mapsto \undemi \int_\Sigma f \times d\alpha \bigg] - \source^*(\lambda).
      $$
    But $\source^*(\alpha)$ is not necessarily zero.
    Let us compute generally $\hat \alpha^*(\lambda)$.
    Let $P : U \to \Omega^1(\Sigma)$ be a $n$-plot.
    We have,
    $\hat \alpha^*(\lambda)(P) = \lambda(\hat \alpha \circ P) = \lambda(r \mapsto \hat \alpha(P (r)) = \lambda(r \mapsto \alpha + d(P (r)))$.
    But,
    \begin{eqnarray*}
      \lambda(r \mapsto \alpha + d(P (r))) & = & \undemi \int_\Sigma (\alpha + P (r)) \wedge {\partial \over \partial r}(\alpha + d(P (r))) \\
      & = & \undemi \int_\Sigma (\alpha + P (r)) \wedge {\partial d(P (r)) \over \partial r} \\
      & = & \undemi \int_\Sigma \alpha \wedge {\partial d(P (r)) \over \partial r} + \undemi \int_\Sigma P (r) \wedge {\partial d(P (r)) \over \partial r}.
    \end{eqnarray*}
    So,
    $$
      (\hat \alpha^*(\lambda) - \source^*(\lambda))(P) = \undemi \int_\Sigma \alpha \wedge {\partial d(P (r)) \over \partial r}.
      $$
    Therefore,
    \begin{eqnarray*}
      \mu(\alpha)(P)_r & = & (\hat \alpha^*(\lambda) - \source^*(\lambda))(P)_r + d \bigg[f \mapsto \undemi \int_\Sigma f \times d\alpha\bigg](P)_r \\
      & = & \undemi \int_\Sigma \alpha \wedge {\partial d(P (r)) \over \partial r} + {\partial \over \partial r} \bigg\{ \undemi \int_\Sigma P (r) \times d \alpha \bigg\} \\
      & = & \undemi {\partial \over \partial r} \bigg\{ \int_\Sigma \alpha \wedge d(P (r)) + P (r) \times d\alpha \bigg\} \\
      & = & {\partial \over \partial r} \bigg\{ \int_\Sigma P (r) \times d\alpha \bigg\}.
    \end{eqnarray*}
    So,
    we get finally,
    $$
      \mu(\alpha) = d \bigg[ f \mapsto \int_\Sigma f \times d\alpha \bigg].
      $$
    Now,
    let us express the variance of $\mu$.
    Let $f \in \Cinfty(\Sigma,\RR)$,
    and let $F(\alpha)$ be the real function $F(\alpha) : f \mapsto \int_\Sigma f \times d\alpha$,
    such that $\mu(\alpha) = dF(\alpha)$.
    We have,
    $\mu(\bar f(\alpha)) = \mu (\alpha + df) = dF(\alpha + df)$.
    But,
    for every $h \in \Cinfty(\Sigma,\RR)$,
    $F(\alpha + df)(h) = \int_\Sigma h \times d(\alpha + df) = \int_\Sigma h \times d\alpha = F(\alpha)(h)$.
    So,
    for all $f \in \Cinfty(\Sigma,\RR)$,
    we have $\mu \circ \hat f = \mu$.
    The moment map $\mu$ is invariant by the group $\Cinfty(\Sigma,\RR)$.
    Souriau's class vanishes.
    Thus,
    the action of $\Cinfty(\Sigma,\RR)$ is exact and hamiltonian.
    
    Let us compute finally the kernel of the moment map $\mu$.
    We have:
    $\mu(\alpha) = 0$ if and only if $dF(\alpha) = 0$.
    But since $\Cinfty(\Sigma,\RR)$ is connected (actually contractible as a diffeological vector space) $dF(\alpha) = 0$ if and only if $F(\alpha) = \const = F(\alpha)(0) = 0$.
    But $F(\alpha) = 0$ if and only if,
    for all $f \in \Cinfty(\Sigma,\RR)$,
    $\int_\Sigma f \times d\alpha = 0$.
    That is,
    if and only if $d\alpha = 0$.
  \end{proof}
  
  \article{On the intersection 2-form of a surface II}
  \label{On-the-intersection-form-of-a-surface-II}
  We continue with the example of \ref{On-the-intersection-form-of-a-surface-I},
  using the same notations.
  Let us introduce the group $G$ of positive diffeomorphisms of $(\Sigma,\Surf)$.
  That is,
  $$
    G = \bigg\{ g \in \Diff(\Sigma) \ \bigg| \ {g^*(\Surf) \over \Surf} >0 \bigg\}.
    $$
  The group $G$ acts by pushforward on $\Omega^1(\Sigma)$.
  For all $g \in G$,
  for all $\alpha \in \Omega^1(\Sigma)$,
  $g_*(\alpha) \in \Omega^1(\Sigma)$,
  and for all pair $g$, $g'$ of elements of $G$,
  $(g \circ g')_* = g_* \circ g'_*$.
  And,
  this action is smooth.
  Now,
  
  \begin{enumerate}
    \item The pushforward action of $G$ on $\Omega^1(\Sigma)$ preserves the 1-form $\lambda$, and thus the 2-form $\omega$. For all $g \in G$, $(g_*)^*(\lambda) = \lambda$, and $(g_*)^*(\omega) = \omega$. So, the action of $G$ is exact, $\sigma = 0$, and hamiltonian, $\Gamma = \{0\}$.
    \item The moment maps are, up to a constant, equal to the moment $\mu$, given by
    $$
      \mu(\alpha)(P)_r(\delta r) = \undemi \int_\Sigma \alpha \wedge P (r)^*\bigg({\partial P (r)_*(\alpha) \over \partial r}(\delta r)\bigg),
      $$
    for all $\alpha \in \Omega^1(\Sigma)$, for all $n$-plot $P$, where $r \in \dom(P)$ and $\delta r \in \RR^n$. In particular, applied to any 1-plot $F$ centered at the identity $\id_G$, that is $F(0) = \id_G$, we get the special expression
    $$
      \mu(\alpha)(F)_0(1) = - \undemi \int_\Sigma \alpha \wedge \DLie_F(\alpha) = - \int_\Sigma i_F(\alpha) \times d\alpha,
      $$
    where $\DLie_F(\alpha)$ is the Lie derivative of $\alpha$ along $F$, and $i_F(\alpha)$ the contraction of $\alpha$ by $F$.
  \end{enumerate}
  So,
  we find again,
  through the diffeological formalism of the moment map,
  what is asserted informally in the literature.
  The vague assertion ``the moment map of the group of diffeomorphism is the Lie derivative'' makes here sense.
  
  \begin{proof}
    1) Let us compute the pullback of $\lambda$ by the action of $g \in G$,
    that is $(g_*)^*(\lambda)$.
    Let $P : U \to \Omega^1(\Sigma)$ be a $n$-plot,
    let $r \in U$,
    and $\delta r \in \RR^n$.
    We have,
    \begin{eqnarray*} (g_*)^*(\lambda)(P)_r(\delta r) & = & \lambda(g_* \circ P)_r(\delta r) \\
      & = & \undemi \int_\Sigma g_*(P (r)) \wedge {\partial g_*(P (r)) \over \partial r}(\delta r)\\
      & = & \undemi \int_\Sigma g_*(P (r)) \wedge g_*\bigg({\partial P (r) \over \partial r}(\delta r)\bigg) \\
      & = & \undemi \int_\Sigma g_*\bigg(P (r)\wedge {\partial P (r) \over \partial r}(\delta r)\bigg) \\
      & = & \undemi \int_{g^*(\Sigma)} P (r)\wedge {\partial P (r) \over \partial r}(\delta r) \\
      & = & \undemi \int_\Sigma P (r)\wedge {\partial P (r) \over \partial r}(\delta r) \\
      & = & \lambda (P)_r(\delta r)
    \end{eqnarray*}
    Thus,
    $\lambda$ is invariant by $G$,
    and so do $\omega = d\lambda$.
    
    2) Since the 1-form $\lambda$ is invariant by the action of $G$,
    we can use directly the results of the exact case detailed in \ref{The-exact-case}.
    Thus,
    the moment maps are,
    up to a constant,
    equal to $\mu : \alpha \mapsto \hat \alpha^*(\lambda)$.
    So,
    let $P : U \to G$ be a $n$-plot,
    let $r \in U$ and $\delta r \in \RR^n$.
    We have,
    \begin{eqnarray*}
      \mu(\alpha)(P)_r(\delta r) & = & \alpha^*(\lambda)(P)_r(\delta r) \\
      & = & \lambda(\hat \alpha \circ P)_r(\delta r) \\
      & = & \lambda(r \mapsto P (r)_*(\alpha))_r(\delta r) \\
      & = & \undemi \int_\Sigma P (r)_*(\alpha) \wedge {\partial P (r)_*(\alpha) \over \partial r}(\delta r) \\
      & = & \undemi \int_\Sigma \alpha \wedge P (r)^*\bigg({\partial P (r)_*(\alpha) \over \partial r}(\delta r)\bigg).
    \end{eqnarray*}
    Now,
    let $P = F$ be a $1$-plot centered at the identity,
    $F(0) = \id_G$.
    Let us change the variable $r$ for the variable $t$.
    The previous expression,
    computed at $t = 0$ and applied to the vector $\delta t = 1$ gives immediately
    \begin{eqnarray*}
      \mu(\alpha)(F)_0(1) & = & \undemi \int_\Sigma \alpha \wedge \left.{\partial F(t)_*(\alpha) \over \partial t}\right|_{t=0}.
    \end{eqnarray*}
    But,
    by definition of the Lie derivative,
    we have
    $$ \bigg\{{\partial F(t)_*(\alpha) \over \partial t} \bigg\}_{t = 0} = \bigg\{ {\partial (F(t)^{-1})^*(\alpha) \over \partial t}\bigg\}_{t = 0} = - \DLie_F(\alpha).
      $$
    So,
    we get the first expression of the moment map $\mu$ applied to $F$
    $$
      \mu(\alpha)(F)_0(1) = - \undemi \int_\Sigma \alpha\wedge\DLie_F(\alpha).
      $$
    Now,
    on a surface $\alpha \wedge d\alpha = 0$,
    and $i_F(\alpha \wedge d\alpha) = i_F(\alpha) \times d\alpha - \alpha \wedge i_F(d\alpha)$.
    So,
    $i_F(\alpha) \times d\alpha = \alpha \wedge i_F(d\alpha)$.
    Then,
    using the Cartan-Lie formula $\DLie_F(\alpha) = i_F(d\alpha) + d(i_F(\alpha))$,
    we get
    \begin{eqnarray*}
      \int_\Sigma\alpha\wedge\DLie_F(\alpha) & = & \int_\Sigma\alpha\wedge[i_F(d\alpha) + d(i_F(\alpha))] \\
      & = & \int_\Sigma i_F(\alpha)d\alpha + \int_\Sigma \alpha \wedge d(i_F(\alpha)) \\
      & = & \int_\Sigma i_F(\alpha)d\alpha + \int_\Sigma i_F(\alpha)d\alpha - \int_\Sigma d[\alpha \wedge i_F(\alpha)] \\
      & = & 2 \int_\Sigma i_F(\alpha)d\alpha
    \end{eqnarray*}
    And finally,
    we have the second expression for the moment map:
    $$ \mu(\alpha)(F)_0(1) = - \int_\Sigma i_F(\alpha) \times d\alpha,
      $$
    for any $1$-plot of the group of positive diffeomorphisms of the surface $\Sigma$,
    centered at the identity.
  \end{proof}
  
  \article{On the intersection 2-form of a surface III}
  \label{On-the-intersection-form-of-a-surface-III}
  We continue again with the example of \ref{On-the-intersection-form-of-a-surface-I},
  using the same notations.
  Let us consider the space $\Omega^1(\Sigma)$ as an additive group acting onto itself by translations.
  Let us denote by $\et_\beta$ the translation $\et_\beta : \alpha \mapsto \alpha + \beta$,
  where $\alpha$ and $\beta$ belong to $\Omega^1(\Sigma)$.
  \begin{enumerate}
    \item The 2-form $\omega$ is invariant by translation. That is, $\et_\alpha^*(\omega) = \omega$ for all $\alpha \in \Omega^1(\Sigma)$. This action of $\Omega^1(\Sigma)$ onto itself is hamiltonian but not exact.
    \item The moment maps of the additive action of $\Omega^1(\Sigma)$ onto itself are equal, up to a constant to
    $$
      \mu : \alpha \mapsto d\left[\beta \mapsto \int_\Sigma \alpha \wedge \beta \right].
      $$
    In other words, $\mu(\alpha) = d[\omega(\alpha)]$, where $\omega$ is regarded as the smooth linear function $\omega(\alpha) : \beta \mapsto \omega(\alpha,\beta)$, defined on $\Omega^1(\Sigma)$. Moreover, the moment map $\mu$ is linear and injective.
    \item The moment map $\mu$ is its own Souriau cocycle, $\theta = \mu$. The moment map $\mu$ identifies $\Omega^1(\Sigma)$ with the $\theta$-affine coadjoint orbit of $0 \in \Omega^1(\Sigma)^*$. Be aware that $\Omega^1(\Sigma)^*$ denotes the space of invariant 1-forms of the abelian group $\Omega^1(\Sigma)$, and not its algebraic dual.
  \end{enumerate}
  {\sc Note} --- This situation is analogous to what happens for finite dimension symplectic vector spaces.
  The 2-form $\omega$ can be regarded as a real 2-cocycle of the additive group $\Omega^1(\Sigma)$.
  This cocycle build up a central extension by $\RR$,
  $$
    (\alpha,t) \cdot (\alpha',t') = \bigg(\alpha + \alpha', t+t' + \int_\Sigma \alpha \wedge \alpha'\bigg)
    $$
  for all $(\alpha,t)$ and $(\alpha',t')$ in $\Omega^1(\Sigma) \times \RR$.
  This central extension acts on $\Omega^1(\Sigma)$,
  preserving $\omega$.
  This action is hamiltonian,
  but now exact.
  The lack of equivariance,
  characterized by Souriau's class,
  has been absorbed in the extension.
  This group could be named as the {\em Heisenberg group} of the oriented surface $(\Sigma,\Surf)$.
  
  Note also that,
  according to \ref{Symplectic-homogeneous-diffeological-spaces},
  the space $\Omega^1(\Sigma)$ equipped with the 2-form $\omega$ is an homogeneous symplectic space.
  Thus,
  we have a first simple example of infinite dimensional {\em symplectic diffeological space},
  avoiding any discussion on the ``kernel'' of $\omega$.
  
  \begin{proof}
    Let us compute the pullback of $\lambda$ by a translation.
    Let $P : U \to X$ be a $n$-plot,
    let $r \in U$,
    and $\delta r \in \RR^n$.
    We have,
    \begin{eqnarray*}
      \et_\alpha^*(\lambda)(P)_r(\delta r)
      & = & \lambda(\et_\alpha \circ P)_r(\delta r) \\
      & = & \lambda[r \mapsto P (r) + \alpha]_r(\delta r) \\
      & = & \undemi \int_\Sigma (P (r)+\alpha) \wedge {\partial (P (r) +\alpha) \over \partial r}(\delta r)\\ & = & \undemi \int_\Sigma P (r) \wedge {\partial P (r) \over \partial r}(\delta r) + \undemi \int_\Sigma \alpha \wedge {\partial P (r) \over \partial r}(\delta r) \\
      & = & \lambda(P)_r(\delta r) + d\bigg[\beta \mapsto \undemi \int_\Sigma \alpha \wedge \beta \bigg](P)_r(\delta r)
    \end{eqnarray*}
    So,
    let us define,
    for all $\alpha \in \Omega^1(\Sigma)$,
    the smooth real function $F(\alpha)$ by
    $$
      F(\alpha) : \beta \mapsto \undemi \int_\Sigma \alpha \wedge \beta.
      $$
    Such that
    $$ \et_\alpha^*(\lambda) = \lambda + d(F(\alpha)) \qmbox{and} \et_\alpha^*(\omega) = \omega.
      $$
    Then,
    $\Omega^1(\Sigma)$,
    as an additive group,
    acts on itself by automorphisms.
    Let us compute the moment maps.
    Let $p$ be a path of $\Omega^1(\Sigma)$,
    connecting $\alpha_0$ to $\alpha_1$.
    We have
    \begin{eqnarray*}
      \Psi(p) & = & \hat \alpha_1^*(\lambda) - \hat \alpha_0^*(\lambda) - d\bigg[\beta \mapsto \int_p d (F(\beta))\bigg] \\
      & = & \hat \alpha_1^*(\lambda) - \hat \alpha_0^*(\lambda) - d[\beta \mapsto F(\beta)(\alpha_1) - F(\beta)(\alpha_0)] \\
      & = & \{\alpha_1^*(\lambda) - d[\beta \mapsto F(\beta)(\alpha_1)]\} - \{\alpha_0^*(\lambda) - d[\beta \mapsto F(\beta)(\alpha_0)]\} \\
      & = & \{\hat \alpha_1^*(\lambda) + d(F(\alpha_1))\} - \{\hat \alpha_0^*(\lambda) + d(F(\alpha_0))\}.
    \end{eqnarray*}
    So,
    the 2-points moment map \ref{Definition-of-the-2-points-moment-map} is given by $\psi(\alpha_0,\alpha_1) = \Psi(p)$.
    Now,
    the moment maps are,
    up to a constant equal to
    $$
      \mu(\alpha) = \psi(0,\alpha) = \hat \alpha_1^*(\lambda) + d(F(\alpha)) - \hat 0^*(\lambda).
      $$
    But,
    for any plot $\P: U \to \Omega^1(\Sigma)$,
    we have
    \begin{eqnarray*}
      \hat\alpha^*(\lambda) (P) - \hat 0^*(\lambda) (P) & = & \lambda(\hat \alpha \circ P) - \lambda(\hat 0 \circ P) \\
      & = & \lambda (r \mapsto P (r) +\alpha) - \lambda (r \mapsto P (r)) \\
      & = & d\bigg[\beta \mapsto \undemi \int_\Sigma \alpha \wedge \beta \bigg](P) \\
      & = & d(F(\alpha))(P).
    \end{eqnarray*}
    Thus,
    $\hat \alpha^*(\lambda) (P) - \hat 0^*(\lambda) = d(F(\alpha))$ and the moment map $\mu$ is finally given by
    $$
      \mu(\alpha) = 2 d(F(\alpha)) = d \bigg[\beta \mapsto \int_\Sigma \alpha \wedge \beta \bigg].
      $$
    The moment map $\mu$ is not equivariant,
    and Souriau's cocycle $\theta$ is given by,
    $$
      \mu(\et_\alpha^*(\beta)) = \mu(\alpha + \beta) = \mu(\beta) + \theta(\alpha) \qmbox{with} \theta(\alpha) = \mu(\alpha).
      $$
    So,
    the moment map $\mu$ is clearly smooth and linear.
    Let $\alpha \in \ker(\mu)$,
    $\mu(\alpha) = 0$ if and only if $d(F(\alpha)) = 0$,
    that is if and only if $F(\alpha) = \const = F(\alpha)(0) = 0$.
    But $F(\alpha)(\beta) = 0$ for any $\beta \in \Omega^1(\Sigma)$,
    hence $\alpha = 0$.
    Therefore,
    the moment map $\mu$ is injective.
  \end{proof}
  
  \article{On symplectic irrational tori}
  \label{On-symplectic-irrational-tori}
  Let us consider the numerical space $\RR^n$,
  for some integer $n$.
  For all $u \in \RR^n$,
  let us denote by $\et_u$ the translation by $u$.
  That is,
  $\et_u : x \mapsto x + u$.
  Let $\omega$ be a 2-form of $\RR^n$ invariant by translations.
  That is,
  for all $u \in \RR^n$,
  $\et_u^*(\omega) = \omega$.
  Thus,
  $\omega$ is a constant bilinear 2-form,
  necessarily closed,
  $d\omega = 0$.
  Let us consider the moment maps associated to the translations $(\RR^n,+)$.
  Since $\RR^n$ is simply connected,
  the holonomy vanishes,
  $\Gamma = \{0\}$.
  Let $p$ be a path of $\RR^n$ connecting $x = p(0)$ to $y=p(1)$,
  the paths moment map $\Psi(p)$,
  and the 2-points moment map $\psi(p)$ are given by
  $$
    \Psi(p) = \psi(x,y) = \omega(y-x),
    $$
  where $\omega(u)$ is regarded as the linear 1-form $\omega(u) : v \mapsto \omega(u,v)$.
  So,
  the moment maps are,
  up to constant,
  equal to the linear map
  $$
    \mu : x \mapsto \omega (x).
    $$
  And therefore,
  Souriau's cocycle $\theta$ associated to $\mu$ is equal to $\mu$.
  For all $u \in \RR^n$,
  $$
    \theta(u) = \mu(u) = \omega(u).
    $$
  Let us consider now a discrete diffeological subgroup $K \subset \RR^n$.
  Let us denote by $Q$ the quotient $Q = \RR^n/K$ and by $\pi : \RR^n \to Q$ the canonical projection.
  Let us continue to denote by $\et_u$ the translation on $Q$,
  by $u \in \RR^n$.
  That is $\et_u(q) = \pi(x+u)$ for any $x$ such that $q = \pi(x)$.
  Now,
  since $\omega$ is invariant by translations,
  $\omega$ is invariant by $K$,
  and since $K$ is discrete,
  $\omega$ projects on $Q$ as a $\RR^n$-invariant closed 2-form denoted by $\omega_Q$.
  That is,
  $$
    \omega_Q = \pi_*(\omega) \qmbox{or} \omega = \pi^*(\omega_Q).
    $$
  Note that,
  the translation by any vector $u$ of $\RR^n$ on $Q$ is still an automorphism of $\omega_Q$,
  that is $\et_u^*(\omega_Q) = \omega_Q$.
  \begin{enumerate}
    \item The holonomy $\Gamma_Q$ of the action of $(\RR^n,+)$ on $(Q,\omega_Q)$ is the image of the subgroup $K$ by $\mu$.
    $$
      \Gamma_Q = \mu(K), \quad \Gamma_Q \subset \RR^{n*}.
      $$
    Thus, if $\omega \neq 0$ and if $K$ is not reduce to $\{0\}$, then the action of $(\RR^n,+)$ on $(Q,\omega_Q)$ is not hamiltonian and not exact.
    \item The moment map $\mu : \RR^n \to \RR^{n*}$ projects on a moment $\mu_Q$ such that the following diagram commutes.
    \[
    \begin{tikzcd}
      \RR^n \arrow[r, "\mu"] \arrow[d, "\pi"'] & \RR^{n*} \arrow[d, "\pr"] \\
      Q = \RR^n/K \arrow[r, "\mu_Q"'] & \RR^{n*}/\mu(K)
    \end{tikzcd}
    \]
    That is, for all $q \in Q$, $\mu_Q(q) = \pr(\omega(x))$ for any $x$ such that $q = \pi(x)$. Souriau's cocycle $\theta_Q$ associated to $\mu_Q$, for all $u \in \RR^n$, is given by
    $$
      \theta_Q(u) = \mu_Q(\pi(u)).
      $$
    So, if we consider the space $Q$ as an additive group acting on itself by translations, then the moment map $\mu_Q$, of this action, coincide with its Souriau cocycle $\theta_Q$.
    \item The map $\mu$ is a fibration onto its image whose fiber is the kernel of $\mu$. That is $\Values(\mu) \simeq \RR^n/\E$, $\E = \ker(\mu)$. And, the map $\mu_Q$ is a fibration onto its image $\mu(\RR^n)/\mu(K)$ whose fiber is $\ker(\mu_Q) = \E/(K \cap \E)$. If $\omega: \RR^n \to \RR^{n*}$ is injective (which implies that $n$ is even) then the moment map $\mu_Q$ is a diffeomorphism which identifies $Q$ with its image $\RR^{n*}\!/\mu(K)$.
  \end{enumerate}
  {\sc Note 1} --- Regarded as a group $Q = \RR^n\!/{K}$ acts onto itself by projection of the translations of $\RR^n$.
  Since the pullback by $\pi: \RR^n \to Q$ is an isomorphism from $\cQ^*$ to $\RR^{n*}$ ($\RR^n$ is the universal covering of $Q$),
  the moment maps computed above give the moment maps associated to this action.
  
  {\sc Note 2} --- This construction applies to the torus $\T^2 = \RR^2/\ZZ^2$.
  The action of $(\RR^2,+)$,
  is obviously not hamiltonian,
  but the moment map $\mu_{\T^2}$ is well defined.
  And,
  $\mu_{\T^2}$ identifies $\T^2$ with the quotient of $\RR^{2*}$ --- the $(\Gamma_Q,\theta_Q)$-coadjoint orbit of the point $0$ --- by the holonomy $\Gamma_Q = \omega(\ZZ^2) \subset \RR^{2*}$.
  In the meaning we gave above of the notion of coadjoint orbit,
  the torus $\T^2$,
  equipped with the standard symplectic form $\omega$,
  is a coadjoint orbit of $\RR^2$,
  or even a coadjoint orbit of itself.
  This is a special case of the the \ref{Moment-maps-for-symplectic-manifolds} discussion.
  
  {\sc Note 3} --- All this construction above can be also applied to situations which are regarded as more singular that the simple quotient of $\RR^n$ by a lattice.
  It can by applied as well to the product of any irrational tori.
  An ($n$-dimensional) irrational torus $\T_K$ is the quotient of $\RR^n$ by any generating discrete strict subgroup $K$ of $\RR^n$.
  See for example \cite{IL90} for an analysis of 1-dimensional irrational tori.
  For example,
  we can consider the product of two 1-dimensional irrational torus $Q = {\T}_{H} \times {\T}_{K}$,
  quotient of $\RR^2 = \RR\times \RR$ by the discrete subgroup $\alpha_{H}(\ZZ^p) \times \alpha_{K}(\ZZ^q)$,
  where $\alpha_H : \RR^p \to \RR$ and $\alpha_K : \RR^q \to \RR$ are two linear 1-forms.
  In this case,
  the moment map $\mu_Q$ will also identify ${\T}_{H} \times {\T}_{K}$ with the quotient of $\RR^{2*}$ --- $(\Gamma_Q,\theta_Q)$-coadjoint orbit of $0$ --- by $\Gamma_Q = \omega(\alpha_{H}(\ZZ^p) \times \alpha_{K}(\ZZ^q))$.
  This is the simplest example of {\em totally irrational symplectic space},
  and {\em totally irrational coadjoint orbit}.
  Note that,
  these cases escape completely to the usual analysis,
  but also to the analysis in terms of Sikorski's or Fr\"olicher's spaces.
  
  \begin{proof}
    First of all,
    the fact that there exists a closed 2-form $\omega_Q$ on $\RR/K$ such that $\pi^*(\omega_Q) =\omega$ is an application of the criterion of pushing forward forms,
    in the special case of a covering \cite{Piz05}.
    Now,
    the computation of the moment map of a linear antisymmetric form $\omega$ on $\RR^n$ is well know,
    and independently of the method gives the same result $\mu(x) = \omega(x)$.
    The additive constant is fixed here by the condition $\mu(0)=0$.
    But,
    the value of the paths moment map $\Psi(p)$ can be found as well by the method described above,
    applying the particular expression
    $$
      \cK\omega_p(\delta p) = \int_0^1 \omega_{p(t)}(\dot p(t),\delta p(t)) dt \qmbox{with} \dot p(t) ={d p(t) \over dt}.
      $$
    of the chain-homotopy operator for manifold.
    Where $p$ is a path and $\delta p$ is a ``variation'' of $p$.
    So,
    since the result depends only on the ends of the path,
    let us choose,
    for any points $x$ and $y$ in $\RR^n$,
    the connecting path $p : t \mapsto x + t(y-x)$.
    Let us remind that $\Psi(p) = \hat p^*(\cK\omega)$.
    Let $u$ and $\delta u$ in $\RR^n$.
    Note that $\hat p_*(\et_u) = \et_u \circ p = [t \mapsto p(t) + u]$.
    So,
    \begin{eqnarray*}
      \Psi(p)_u(\delta u) & = & \hat p^*(\cK\omega)_u(\delta u) \\
      & = & ({\cK}\omega)_{\et_u \circ p} (\delta(\et_u \circ p)), \qmbox{with} \delta p = 0 \\
      & = & \int_0^1 \omega(\dot p(t), \delta u) \ d\, t \\ & = & \omega(y-x,\delta u)
    \end{eqnarray*}
    So $\Psi(p) = \psi(x,y) = \omega(y-x)= \omega(y) - \omega(x)$.
    And,
    $\mu : x \mapsto \omega(x)$,
    for all $x$ in $\RR^n$.
    
    Now,
    let us consider $\omega_Q$.
    Since $\RR^n$ is the universal covering of $Q$,
    every loop $\ell \in \Loops(Q,0)$ can be lifted into a path $p$ of $\RR^n$ starting at $0$ and ending in $K$.
    In other words,
    $$
      \Gamma = \left\{ \Psi (\ell) \mid \ell \in \Loops(Q) \right\} = \left\{ \Psi (t \mapsto tk) \mid k \in K \right\} = \omega (K)
      $$
    The other propositions are then a direct application of the functoriality of the moment map described in \ref{Pushing-forward- moment-maps},
    and standard analysis on quotients and fibrations.
  \end{proof}
  
  \article{The corner orbifold}
  \label{The-corner-orbifold}
  Let us consider the quotient $\cQ$ of $\RR^2$ by the action of the finite subgroup $K \simeq \{\pm1\}^2$,
  embedded in $GL(2,\RR)$ by
  $$
    K = \bigg\{ \mymatrix{\varepsilon & 0 \cr 0 & \varepsilon'} \bigg| \ \varepsilon, \varepsilon' \in \{\pm1\} \bigg\}.
    $$
  The space $\cQ = \RR^2/K$ is an orbifold,
  according to \cite{IKZ05}.
  It is diffeomorphic to the quarter space $[0,\infty[ \times [0,\infty[ \subset \RR^2$,
  equipped with the pushforward of the standard diffeology of $\RR^2$ by the map $\pi : \RR^2 \to [0,\infty[ \times [0,\infty[$,
  defined by,
  $$
    \pi(x,y) = (x^2,y^2) \qmbox{and} \cQ \simeq \pi_*(\RR^2).
    $$
  
  \begin{figure}[t]
    \centerline{\includegraphics{Figures/fig-corner-orbifold.pdf}}
    \caption{The corner orbifold $\cQ$}\label{fig-corner-orbifold}
  \end{figure}
  
  So the letter $\cQ$ will denote indifferently the quotient $\RR^2/K$ or the quarter space $\pi_*(\RR^2)$.
  And the meaning of the letter $\pi$ follows.
  Now,
  let us remark that,
  the decomposition of $\cQ$ in terms of point's structure is given by,
  $$
    \Str(0,0) = \{\pm1\}^2, \quad \Str(x,0) = \Str(0,y) = \{\pm1\} \qmbox{and} \Str(x,y) = \{1\},
    $$
  where $x$ and $y$ are positive real numbers.
  So,
  since the structure of a point is preserved by diffeomorphisms \cite{IKZ05},
  there are at least three orbits of $\Diff(\cQ)$,
  the point $0_\cQ=(0,0)$,
  the regular stratum $\dot \cQ = ]0,\infty[^2$ and the union of the two axes,
  $ox$ and $oy$.
  So,
  in particular any diffeomorphism of $\cQ$ preserves the origin $0_\cQ$.
  Actually,
  these are exactly the orbits of $\Diff(\cQ)$.
  Let us remark that,
  $\dim(\cQ) = 2$ \cite{Piz06-b}.
  So,
  every 2-form is closed.
  Now,
  
  1) Every 2-form of $\cQ$ is proportional to the 2-form $\omega$ defined on $\cQ$ by
  $$
    \pi^*(\omega) : \mymatrix{x \cr y} \mapsto 4 xy \times dx \wedge dy.
    $$
  That is,
  for any other 2-form $\omega'$ there exists a smooth function $\phi \in \Cinfty(\cQ,\RR)$ such that $\omega' = \phi \times \omega$.
  
  2) The space $(\cQ,\omega)$ is hamiltonian $\Gamma_\omega = \{0\}$.
  And,
  the action of $G_\omega$ is exact,
  that is $\sigma_\omega = 0$.
  In particular,
  the universal moment map $\mu_\omega$ defined by $\mu_\omega(0_\cQ) = 0$,
  is equivariant.
  
  3) The universal equivariant moment map $\mu_\omega$ vanishes on the singular strata $\{0\}$,
  $ox$ and $oy$,
  and is injective on the regular stratum $\dot \cQ$.
  So,
  the image $\mu_\omega(\cQ)$ is diffeomorphic to an open disc with a point attached on the boundary.
  
  \begin{proof}
    1) Let $\omega'$ be a 2-form on $\cQ$ and let $\tilde \omega'$ be its pullback by $\pi$,
    $\tilde \omega' = \pi^*(\omega')$.
    So,
    there exists a smooth real function $F$ such that $\tilde \omega' = F \times dx \wedge dy$.
    But,
    since $\pi \circ k = \pi$,
    for all $k \in K$ we get $\varepsilon \varepsilon' F(\varepsilon x, \varepsilon' y) = F(x,y)$,
    for all $(x,y) \in \RR^2$ and all $\varepsilon$, $\varepsilon'$ in $\{\pm1\}$.
    Thus,
    $F(-x,y) = -F(x,y)$ and $F(x,-y) = -F(x,y)$.
    In particular,
    $F(0,y) = 0$ and $F(x,0) = 0$.
    Therefore,
    since $F$ is smooth,
    there exists $f \in \Cinfty(\RR^2,\RR)$ such that $F(x,y) = 4xyf(x,y)$,
    with $f(\varepsilon x, \varepsilon' y) = f(x,y)$.
    Therefore,
    $\tilde \omega' = f \times \tilde \omega$,
    with $\tilde \omega = 4xy \times dx \wedge dy$.
    Now $\tilde \omega = d(x^2) \wedge d(y^2)$,
    but $x\mapsto x^2$ and $y \mapsto y^2$ are invariant by $K$ so,
    they are the pullback by $\pi$ of some smooth real functions on $\cQ$.
    Thus,
    $d(x^2)$ and $d(y^2)$ are the pullback of 1-forms on $\cQ$,
    let us say $d(x^2) = \pi^*(ds)$ and $d(y^2) = \pi^*(dt)$,
    so $\tilde \omega = \pi^*(\omega)$,
    where $\omega = ds \wedge dt$ is a well defined 2-form on $\cQ$.
    Now,
    since $f(\epsilon x, \epsilon'y) = f(x,y)$ means just that $f$ is the pullback of a smooth real function $\phi$ on $\cQ$,
    it follows that any 2-form $\omega'$ on $\cQ$ is proportional to $\omega$,
    that is $\omega' = \phi \times \omega$,
    with $\phi \in \Cinfty(\cQ,\RR)$.
    
    2) The orbifold is contractible.
    The deformation retraction $(s,x,y) \mapsto (sx,sy)$ of $\RR^2$ to $\{(0,0)\}$ projects on a smooth deformation retraction of $\cQ$.
    So,
    there is no holonomy,
    $\Gamma = \{0\}$.
    Now,
    since the origin $0_\cQ$ is the only point with structure $\{\pm1\}$,
    every diffeomorphism of $\cQ$ preserves the origin $0_\cQ$.
    So,
    the 2-point moment map is exact,
    see the note 2 of art.~\ref{Souriau-cocycles},
    Souriau's cocycle vanishes,
    $\sigma_\omega = 0$.
    Let $q$ be any point of $\cQ$ and let $\mu_\omega(q) = \psi(0_\cQ,q)$.
    This is an equivariant moment map and $\mu_\omega(0_\cQ) = \psi(0_\cQ,0_\cQ) =0$.
    
    3) Let $q$ in $\cQ$,
    thus $\mu_\omega(q) = \Psi(p)$ for any path $p$ connecting $0_\cQ$ to $q$.
    Now,
    let $q$ belongs to a semi-axis $ox$ or $oy$,
    and let us choose $p = t \mapsto \lambda(t)q$,
    where $\lambda$ is a smashing function equal to $0$ on $]-\infty,0]$ and equal to $1$ on $[1,+\infty[$.
    Thus for all $t \in \RR$,
    $p(t)$ belongs to the same semi-axis than $q$.
    Thanks to the expression (\ref{eq:heartsuit}) of \ref{Evaluation-of-the-paths-moment-map},
    we have for any 1-plot $\phi$ of $\Diff(\cQ,\omega_\omega)$,
    centered at the identity,
    $$
      \Psi(p)(\phi)_0(1) =
      \int_0^1 \omega \left[ \mymatrix{s \cr r} \mapsto
      \phi(r) (\lambda(s + t)q) \right]_{\left({0 \atop
      0}\right)}\mymatrix{1 \cr 0} \mymatrix{0 \cr 1} dt,
      $$
    But,
    now $(s,r) \mapsto \phi(r) (\lambda(s + t)q)$ is a plot of the semi-axis,
    and thanks to the item 1,
    the form $\omega$ vanishes on the semi-axis.
    So,
    the integrand vanishes and $\Psi(p)(\phi)_0(1) = 0$.
    Now,
    since 1-forms are characterized by 1-plots and since momenta are characterized by centered plots,
    $\mu_\omega(q) = 0$ for all $q \in \cQ$ belonging to any semi-axis.
    
    On the other hand,
    let $q$ and $q'$ be two points of the regular stratum $\dot\cQ$.
    Since $\pi \restriction \{ (x,y) \mid x>0 \ \& \ y>0 \}$ is a diffeomorphism,
    and since $\tilde \omega \restriction \{ (x,y) \mid x>0 \ \& \ y>0 \}$ is symplectic there exists always a symplectomorphism $\phi$ with compact support $\cS \subset \{ (x,y) \mid x>0 \ \& \ y>0 \}$ which exchange $q$ and $q'$.
    So,
    the image of this diffeomorphism on $\dot\cQ$ can be extended by the identity on the whole $\cQ$.
    Therefore,
    the automorphisms of $\omega$ are transitive on the regular stratum.
  \end{proof}
  
  \article{The cone orbifold}
  \label{The-cone-orbifold}
  Let $\cQ_m$ be the quotient of the smooth complex plane $\CC$ by the action of the cyclic subgroup
  $$\Z_m \simeq \{\zeta \in \CC \mid \zeta^m = 1\} \qmbox{with} m>1.
    $$
  The space $\cQ_m$ is an orbifold,
  according to \cite{IKZ05}.
  We identify $\cQ_m$ to the complex plane $\CC$,
  equipped with the pushforward of the standard diffeology by the map $\pi_m : z \mapsto z^m$.
  That is,
  a plot of $\cQ_m$ is any parametrization $P$ of $\CC$ which writes locally $P (r) = \phi(r)^m$,
  where $\phi$ is a smooth parametrization of $\CC$.
  
  \begin{figure}[t]
    \centerline{\includegraphics{Figures/fig-cone-orbifold.pdf}}
    \caption{The cone orbifold $\cQ_3$}\label{fig-cone-orbifold}
  \end{figure}
  
  Let us remark first that the decomposition of $\cQ_m$,
  in terms of structure group,
  is given by
  $$
    \Str(0) = \Z_m, \qmbox{and} \Str(z) = \{1\} \qmbox{if} z \neq 0.
    $$
  And secondly that there is two orbits of $\Diff(\cQ_m)$,
  the point $0$ and the regular stratum $\dot \cQ_m = \CC - \{0\}$.
  In particular any diffeomorphism of $\cQ_m$ preserves the origin $0$.
  It is not difficult to check that $\dim(\cQ_m) = 2$ \cite{Piz06-b},
  so every 2-form on $\cQ_m$ is closed.
  Now,
  
  1) Every 2-form of $\cQ_m$ is proportional to the 2-form $\omega$ uniquely defined by
  $$
    \pi_m^*(\omega) : z \mapsto dx \wedge dy \qmbox{with} z = x +iy.
    $$
  That is,
  for any other 2-form $\omega'$ there exists a smooth function $f \in \Cinfty(\cQ_m,\RR)$ such that $\omega' = f \times \omega$.
  
  2) The space $(\cQ,\omega)$ is hamiltonian $\Gamma_\omega = \{0\}$.
  And,
  the action of $G_\omega$ is exact,
  that is $\sigma_\omega = 0$.
  In particular,
  the universal moment map $\mu_\omega$ defined by $\mu_\omega(0) = 0$,
  is equivariant.
  
  3) The universal moment map $\mu_\omega$ is injective.
  Its image is the reunion of two coadjoint orbits,
  the point $0 \in \cG^*_\omega$,
  value of the origin of $\cQ_m$,
  and the image of the regular stratum $\dot \cQ_m$.
  
  \begin{proof}
    Let us first prove that the usual surface form $\Surf = dx \wedge dy$ is the pullback of a 2-form $\omega$ defined on $\cQ_m$.
    We shall apply the standard criterion and prove that for any two plots $\phi_1$ and $\phi_2$ of $\CC$ such that $\pi_m \circ \phi_1 = \pi_m \circ \phi_2$ we have $\Surf(\phi_1) = \Surf(\phi_2)$.
    That is,
    $\phi_1(r)^m = \phi_2(r)^m$ implies $\Surf(\phi_1) = \Surf(\phi_2)$.
    First of all let us recall that,
    since we are dealing with 2-forms,
    is is sufficient to consider 2-plots.
    So,
    let the $\phi_i$ be defined on some numerical domain $U \subset \RR^2$.
    Let $r_0 \in U$,
    we split the problem into 2 cases.
    
    1) $\phi_1(r_0) \neq 0$ --- Thus $\phi_2(r_0) \neq 0$,
    there exists a open disk $\B$ centered at $r_0$ on which the $\phi_i$ do not vanishes.
    Thus,
    the map $r \mapsto \zeta(r) = \phi_2(r)/\phi_1(r)$ defined on $\B$ is smooth with values in $\Z_m$.
    But,
    since $\Z_m$ is discrete there exists $\zeta \in \Z_m$ such that $\phi_2(r) = \zeta \times \phi_1(r)$ on $\B$.
    Now,
    $\Surf$ is invariant by $U(1) \supset \Z_m$.
    Therefore $\Surf(\phi_1) = \Surf(\phi_2)$ on $\B$.
    
    2) $\phi_1(r_0) = 0$ --- Thus, $\phi_2(r_0) = 0$.
    Now,
    we have $\Surf(\phi_i) = \det(\D(\phi_i)) \times \Surf$,
    where $\D(\phi_i)$ denotes the tangent map of $\phi_i$.
    We split this case into two sub-cases:
    
    2.a) $\D(\phi_1)_{r_0}$ is non-degenerate --- Thus, thanks to the implicit function theorem,
    there exists a small open disk $\B$ around $r_0$ where $\phi_1$ is a local diffeomorphisms onto its image.
    Since $\phi_1(r)^m = \phi_2(r)^m$,
    the common zero $r_0$ of both $\phi_1$ and $\phi_2$ is isolated.
    Thus,
    the map $r \mapsto \zeta(r) = \phi_2(r)/\phi_1(r)$ defined on $\B - \{r_0\}$ is smooth,
    and for the same reason than in the first case,
    $\zeta$ is constant.
    So,
    $\phi_2(r) = \zeta \times \phi_1(r)$ on $\B - \{r_0\}$.
    But,
    since $\phi_i(r_0) = 0$,
    this equality extends on $\B$.
    Therefore $\Surf(\phi_1) = \Surf(\phi_2)$ on $\B$.
    
    2.b) $\D(\phi_1)_{r_0}$ is degenerate --- Let $u$ be in the kernel of $\D(\phi_1)_{r_0}$.
    We have $\phi_1(r_0 + su)^m = \phi_2(r_0 + su)^m$ for enough small real $s$.
    Then,
    differentiating this equality $m$ times with respect to $s$,
    for $s=0$ we get $ 0 = \D(\phi_1)_{r_0}(u)^m = \D(\phi_2)_{r_0}(u)^m$.
    Therefore,
    $\D(\phi_2)_{r_0}$ is also degenerate at $r_0$ and thus $0 = \Surf(\phi_1)_{r_0} = \Surf(\phi_2)_{r_0}$.
    
    So,
    we have proved that for any $r \in U$,
    $\Surf(\phi_1)_r = \Surf(\phi_2)_r$.
    Therefore,
    there exists a 2-form $\omega$ on $\cQ_m$ such that $\pi_m^*(\omega) = \Surf$,
    and this form $\omega$ is completely defined by its pullback.
    Now,
    since the pullback by $\pi_m$ of any other 2-form $\omega'$ on $\cQ_m$ is proportional to $\Surf$,
    the form $\omega'$ is proportional to $\omega$.
    
    Now,
    for the same reasons than in \ref{The-corner-orbifold} the universal holonomy $\Gamma_\omega$ and Souriau's class $\sigma_\omega$ vanish,
    and the universal moment map $\mu_\omega$ defined by $\mu_\omega(0) = 0_{\cG^*}$ is equivariant.
    Moreover,
    the regular stratum $\dot \cQ$ is just a symplectic manifold for the restriction of $\omega$.
    Any symplectomorphism with compact support which doesn't contain $0$ can be extended to an automorphism of $(\cQ,\omega)$.
    Thus,
    since the compactly supported symplectomorphisms of a connected symplectic manifold are transitive,
    the regular stratum $\dot \cQ$ is an orbit of $\Diff(\cQ,\omega)$.
    Therefore,
    the moment map $\mu_\omega$ maps $\cQ$ onto two orbits,
    $\{0_{\cG^*}\}$ and $\mu_\omega(\dot \cQ)$.
  \end{proof}
  
  \subsection{The infinite projective space}
  \label{The-infinite-projective-space}
  
  This example of the symplectic structure of the infinite projective space is extracted from \cite{Piz06-a},
  everything not proved here can be found there.
  Let $\mathcal{H}$ be the Hilbert space of the square summable complex series.
  \[
  \mathcal{H} = \left\{ Z = (Z_i)_{i=1}^\infty \mid \sum_{i=1}^n \bar{Z}_i Z_i < \infty \right\}.
  \]
  Where the dot denotes the hermitian product.
  The space $\mathcal{H}$ is equipped with the \emph{fine structure} of complex diffeological vector space.
  That is,
  its diffeology is generated by the linear injections from $\mathbf{C}^n$ to $\mathcal{H}$,
  or if we prefer,
  let $P : U \to \mathcal{H}$ be a plot,
  then for every $r_0 \in U$,
  there exists an integer $n$,
  an open superset $V \subset U$ of $r_0$,
  a finite family $\mathcal{F} = \{(\lambda_a,Z_a)\}_{a \in A}$,
  where the $Z_a \in \mathcal{H}$,
  and the $\lambda_a \in C^\infty(V,\mathbf{C}^n)$ such that $P \restriction V : r \mapsto \sum_{a \in A} \lambda_a(r) \times Z_a$.
  Such a family $\{(\lambda_a,Z_a)\}_{a \in A}$ is called a \emph{local family} of $P$ at the point $r_0$.
  We defined the symbol $dZ$ which associates to every local family $\mathcal{F} = \{(\lambda_a,Z_a)\}_{a \in A}$ defined on the domain $V$,
  the complex valued 1-form of $V$
  \[
  dZ(\mathcal{F}) : r \mapsto \sum_{a \in A} d\lambda_a(r) Z_a.
  \]
  For every $\lambda_a = x_a +i y_a$,
  where $x_a$ and $y_a$ are real smooth parametrizations,
  $d\lambda_a = dx_a + i dy_a$.
  Now,
  there exists on $\mathcal{H}$ a 1-form $\alpha$ defined by
  \[
  \alpha = \frac{1}{2i} [\bar{Z} \cdot dZ - d\bar{Z} \cdot Z].
  \]
  
  \begin{enumerate}
    \item As an additive group $(\mathcal{H},+)$ acts on itself,
    preserving $d\alpha$.
    Let $Z \in \mathcal{H}$ and let $T_Z$ be the translation by $Z$,
    then $T^*_Z(d\alpha) = d\alpha$.
    This action is hamiltonian but not exact.
    Let $\mu$ be the moment map of the translations $(\mathcal{H},+)$,
    defined by $\mu(0_\mathcal{H}) = 0$.
    So
    \[
    \mu(Z) = 2d[w(Z)] \quad\text{with}\quad w(\zeta) : Z \mapsto \frac{1}{2i} [ \bar{\zeta} \cdot Z - \bar{Z} \cdot \zeta ] \in C^\infty(\mathcal{H},\mathbf{R}).
    \]
    The moment map $\mu$ is injective and $(\mathcal{H},d\alpha)$ is an homogeneous symplectic space.
    
    \item Let $\mathrm{U}(\mathcal{H})$ be the group of unitary transformations of $\mathcal{H}$,
    equipped with the functional diffeology.
    The group $\mathrm{U}(\mathcal{H})$ acts on $\mathcal{H}$ preserving $\alpha$.
    The action of $\mathrm{U}(\mathcal{H})$ on $(\mathcal{H},d\alpha)$ is exact and hamiltonian.
    Let $P : U \to \mathrm{U}(\mathcal{H})$ be a $n$-plot.
    The value of the moment map $\mu$ of the action of $\mathrm{U}(\mathcal{H})$ on $(\mathcal{H},d\alpha)$,
    evaluated on $P$ is given by
    \[
    \mu(Z)(P)_r(\delta r) = \frac{1}{2i} \left[ \overline{P(r)(Z)} \cdot \frac{\partial P(r)(Z)}{\partial r}(\delta r) - \overline{\frac{\partial P(r)(Z)}{\partial r}(\delta r)} \cdot P(r)(Z) \right],
    \]
    where,
    $r \in U$,
    $\delta r \in \mathbf{R}^n$ and:
    \[
    \text{If} \quad P(r)(Z) =_{\mathrm{loc}} \sum_{\alpha \in A} \lambda_\alpha(r) Z_\alpha,
    \quad\text{then}\quad
    \frac{\partial P(r)(Z)}{\partial r}(\delta r) =_{\mathrm{loc}} \sum_{\alpha \in A} \frac{\partial \lambda_\alpha(r)}{\partial r}(\delta r) Z_\alpha.
    \]
    
    \item The unit sphere $\mathcal{S} \subset \mathcal{H}$ is an homogeneous space of $\mathrm{U}(\mathcal{H})$.
    The fibers of the equivariant moment map $\mu$ of the action of $\mathrm{U}(\mathcal{H})$ on $(\mathcal{S},d\alpha \restriction \mathcal{S})$ are the fibers of the infinite Hopf fibration $\pi : \mathcal{S} \to \mathcal{P} = \mathcal{S}/\mathrm{S}^1$,
    where $\mathrm{S}^1 \in \mathbf{C}$ acts multiplicatively on $\mathcal{S}$.
    There exists a symplectic form $\omega$ on $\mathcal{P}$,
    such that $\pi^*(\omega) = d\alpha \restriction \mathcal{S}$.
    The equivariant moment map of the induced action of $\mathrm{U}(\mathcal{H})$ on $\mathcal{P}$ is injective.
    So,
    the infinite projective space $\mathcal{P}$,
    equipped with the Fubini-Study form,
    is an homogeneous symplectic space and can be regarded as a coadjoint orbit of $\mathrm{U}(\mathcal{H})$.
  \end{enumerate}
  
  \begin{proof}
    Many of what is asserted here has been proved in \cite{Piz06-a}.
    So,
    we shall just check what is not in this paper.
    
    1) Since $\mathcal{H}$ is contractible,
    there is no holonomy.
    Now,
    let $\zeta \in \mathcal{H}$ and $T_\zeta$ be the translation $T_\zeta(Z) = Z + \zeta$.
    A direct computation shows that,
    $T_\zeta^*(\alpha) = \alpha + d[w(\zeta)]$.
    Thus,
    $d\alpha$ is invariant by translation $T_\zeta^*(d\alpha) = d\alpha$.
    Now,
    let $p$ be any path connecting $0_\mathcal{H}$ to $Z$,
    we have $\mu(Z) = \Psi(p) = \hat p^*K(d\alpha) = \hat Z^*(\alpha) - \hat 0_\mathcal{H}^*(\alpha) - d[K\alpha \circ \hat p]$.
    But,
    on one hand we have $\hat Z = T_Z$,
    thus $\hat Z^*(\alpha) - \hat 0_\mathcal{H}^*(\alpha) = T_Z^*(\alpha) - \mathbf{1}_\mathcal{H}^*(\alpha) = \alpha + d[w(Z)] - \alpha = d[w(Z)]$.
    And,
    on the other hand we have,
    $\hat p (\zeta) = T_\zeta \circ p$,
    and thus $K\alpha \circ \hat p = \int_{T_\zeta \circ p} \alpha = \int_p T_\zeta^*(\alpha) = \int_p \alpha + \int_p d[w(\zeta)] = \int_p \alpha + w(\zeta)(Z)$,
    since $w(\zeta)(0_\mathcal{H}) = 0$.
    So,
    $\mu(Z) = d[w(Z)] - d[\zeta \mapsto w(\zeta)(Z)]$.
    But,
    $w(\zeta)(Z) = - w(Z)(\zeta)$ so $\mu(Z) = d[w(Z)] - d[\zeta \mapsto - w(Z)(\zeta)] = 2d[w(Z)]$.
    Now,
    let $Z$ be in the kernel of $\mu$,
    so $w(Z) = \mathrm{const} = w(0_\mathcal{H}) = 0$.
    But $w(Z)(Z') = 0$ for all $Z' \in \mathcal{H}$ if and only if $Z = 0_\mathcal{H}$,
    we have just to decompose $Z$ into real and imaginary parts and use the fact that the hermitian norm on $\mathcal{H}$ is not degenerated.
    Therefore,
    $\mu$ is injective.
    
    2) Since the 1-form $\alpha$ is invariant by $\mathrm{U}(\mathcal{H})$,
    this statement is a direct application of the exact case.
  \end{proof}
  
  
  \subsection{The Virasoro coadjoint orbits}
  \label{The-Virasoro-coadjoint-orbit}
  
  Let $\Imm(\mathbf{S}^1,\mathbf{R}^2)$ be the space of all the immersions of the circle $\mathbf{S}^1 = \mathbf{R}/2\pi\mathbf{Z}$ into $\mathbf{R}^2$,
  equipped with the functional diffeology.
  For every $n$-plot $P : U \to \Imm(\mathbf{S}^1,\mathbf{R}^2)$ let us defined the 1-form $\alpha(P)$ on $U$ by
  \[
  \alpha(P)_r(\delta r) = \int_0^{2\pi} \frac{1}{\lVert P(r)'(t)\rVert^2} \left\langle P(r)''(t) \mid \frac{\partial P(r)'(t)}{\partial r}(\delta r) \right\rangle \, dt.
  \]
  for every $r \in U$ and $\delta r \in \mathbf{R}^n$.
  Where the prime denotes the derivative with respect to the parameter $t$,
  and the bracket $\langle \cdot \mid \cdot \rangle$ denotes the ordinary scalar product of the vector space $\mathbf{R}^2$.
  
  \begin{enumerate}
    \item As defined above,
    $\alpha$ is a 1-form of $\Imm(\mathbf{S}^1,\mathbf{R}^2)$.
  \end{enumerate}
  
  Let us consider now the group $\Diff_+(\mathbf{S}^1)$ of positive diffeomorphisms of the circle,
  and its action on $\Imm(\mathbf{S}^1,\mathbf{R}^2)$ by re-parametrization.
  For every $\varphi \in \Diff_+(\mathbf{S}^1)$,
  for every $x \in \Imm(\mathbf{S}^1,\mathbf{R}^1)$,
  let us denote by $\bar{\varphi} (x)$ the pushforward of $x$ by $\varphi$,
  \[
  \bar{\varphi}(x) = \varphi_*(x) = x \circ \varphi^{-1}.
  \]
  And,
  let $F : \Diff_+(\mathbf{S}^1) \to C^\infty(\Imm(\mathbf{S}^1,\mathbf{R}^2),\mathbf{R})$ be the map defined,
  for all $\varphi \in \Diff_+(\mathbf{S}^1)$,
  by
  \[
  F(\varphi) : x \mapsto \int_0^{2\pi} \log \lVert x'(t)\rVert \, d\log(\varphi'(t))
  \]
  
  \begin{enumerate}
    \setcounter{enumi}{1}
    \item The map $F$ is smooth and for every $\varphi \in \Diff(\mathbf{S}^1)$,
    \[
    \bar{\varphi}^*(\alpha) = \alpha - d[F(\varphi)].
    \]
    So,
    the 2-form $\omega = d\alpha$,
    defined on $\Imm(\mathbf{S}^1,\mathbf{R}^2)$,
    is closed and invariant by the action of $\Diff(\mathbf{S}^1)$.
    Moreover,
    the action of $\Diff(\mathbf{S}^1)$ is hamiltonian.
    
    \item Let $x_0 : \mathrm{class}(t) \mapsto (\cos(t),\sin(t))$ be the \emph{standard immersion} from $\mathbf{S}^1 = \mathbf{R}/2\pi\mathbf{Z}$ to $\mathbf{R}^2$.
    The moment maps for $\omega$,
    of $\Diff_+(\mathbf{S}^1)$ on the connected component of $x_0 \in \Imm(\mathbf{S}^1,\mathbf{R}^2)$,
    are translated by a constant from
    \[
    \mu(x)(r \mapsto \varphi)_r(\delta r) = \int_0^{2\pi} \left\{ \frac{ \lVert x''(u)\rVert^2 }{ \lVert x'(u)\rVert^2} - \frac{d^2}{du^2} \log \lVert x'(u)\rVert^2 \right\} \delta u \, du.
    \]
    Where $r \mapsto \varphi$ is any plot of $\Diff_+(\mathbf{S}^1)$ defined on some $n$-domain $U$,
    $r$ is a point of $U$,
    $\delta r \in \mathbf{R}^n$,
    $u = \varphi^{-1}(t)$,
    and $\delta u = D(r \mapsto u)(r)(\delta r)$.
    
    \item With the same conventions as in item 3,
    Souriau's cocycles of the $\Diff_+(\mathbf{S}^1)$ action on $\Imm(\mathbf{S}^1,\mathbf{R}^2)$ are cohomologous to $\theta$ defined by,
    \[
    \theta(g)(r \mapsto \varphi)_r(\delta r) = \int_0^{2\pi} \frac{3 \gamma''(u)^2 -2 \gamma'''(u) \gamma'(u)}{\gamma'(u)^2} \, \delta u \, du,
    \]
    where $g \in \Diff_+(\mathbf{S}^1)$ and $\gamma = g^{-1}$.
    We recognize the integrand of the right hand side as the so-called Schwartzian derivative of $\gamma$.
    
    \item Let $\beta$ be the function for all $g$ and $h$ in $\Diff_+(\mathbf{S}^1)$ by
    \[
    \beta(g,h) = \int_0^{2\pi} \log(g \circ h)'(t) \, d\log h'(t).
    \]
    So,
    for all $g$ and $h$ in $\Diff_+(\mathbf{S}^1)$ we have
    \[
    F(g \circ g') = F(g)\circ \bar{g}' + F(g') - \beta(g,g').
    \]
    This function $\beta$ is known as \emph{Bott's cocycle} \cite{Bot78}.
    The central extension of $\Diff_+(\mathbf{S}^1)$ by $\beta$ is the so-called Virasoro group.
    Its action on $\Imm(\mathbf{S}^1,\mathbf{R}^2)$,
    through $\Diff_+(\mathbf{S}^1)$,
    is still hamiltonian,
    but now exact.
    This is a well known construction which will be not more developed here.
  \end{enumerate}
  
  This example which has been built on purpose \cite{Igl95},
  gathers the main ingredients found in the literature on the construction of Virasoro's group.
  I regard this example as a nice illustration of the whole theory.
  
  \begin{proof}
    The proof is actually a long and tedious series of computations.
    To make it as clear as possible,
    we shall split the computations in a few steps.
    
    \emph{The 1-form $\alpha$} --- We prove first that $\alpha$ is a well defined 1-form on $\Imm(\mathbf{S}^1,\mathbf{R}^2)$.
    Let $F : U \to U$ be a smooth $m$-parametrization.
    We have,
    for all $s \in V$ and all $\delta s \in \mathbf{R}^m$,
    \[
    \alpha(P \circ F)(s)(\delta s) = \int_0^{2\pi} \frac{1}{\lVert (P \circ F)(s)'(t)\rVert^2} \left\langle(P \circ F)(s)''(t) \mid \frac{\partial (P \circ F)(s)'(t)}{\partial s}(\delta s)\right\rangle \, dt
    \]
    That is,
    \[
    \alpha(P \circ F)(s)(\delta s) = \int_0^{2\pi} \frac{1}{\lVert P(F(s))'(t)\rVert^2} \left\langle P(F(s))''(t) \mid \frac{\partial P(F(s))'(t)}{\partial s}(\delta s) \right\rangle \, dt.
    \]
    Let us denote by $r$ the point $F(s)$.
    We get,
    \begin{align*}
      \alpha(P \circ F)(s) (\delta s) & = \int_0^{2\pi} \frac{1}{\lVert P(r)'(t)\rVert^2} \left\langle P(r)''(t) \mid \frac{\partial P(r)'(t)}{\partial r} \left( \frac{\partial F(s)}{\partial s} (\delta s) \right) \right\rangle \, dt \\
      & = \alpha(P)_{r = F(s)} \left(\frac{\partial F(s)}{\partial s}(\delta s) \right) \\
      & = F^*(\alpha(P))_s(\delta s).
    \end{align*}
    So,
    $\alpha(P \circ F) = F^*(\alpha(P))$,
    and $\alpha$ satisfies the differential form axiom.
    
    Let us consider now the action of $\Diff_+(\mathbf{S}^1)$ on $\Imm(\mathbf{S}^1,\mathbf{R}^2)$.
    This action is obviously smooth from the very definition of the functional diffeology of $\Diff_+({\mathbf{S}}^1)$.
    Let us denote $\varphi^{-1}$ by $\phi$ such that
    \[
    \bar{\varphi}^*(\alpha)(P) = \alpha(\bar{\varphi} \circ P) = \alpha[r \mapsto P(r) \circ \varphi^{-1}] = \alpha[r \mapsto P(r) \circ \phi].
    \]
    Note that $\Diff_+({\mathbf{S}}^1)$ acts on \emph{speed} and \emph{acceleration} of any immersion $x$,
    by
    \begin{equation}
      \label{eq:heartsuit}
      \begin{aligned}
        (x \circ \phi)'(t) & = x'(\phi(t))\cdot \phi'(t) \\
        (x \circ \phi)''(t) & = x''(\phi(t))\cdot \phi'(t)^2 + x'(\phi(t)) \cdot\phi''(t).
      \end{aligned}
    \end{equation}
    Let us denote by $Q$ the plot $\bar{\varphi} \circ P$,
    that is $Q = [r \mapsto P(r) \circ \phi]$.
    Such that,
    \[
    \alpha(\bar{\varphi} \circ P)_r(\delta r) = \int_0^{2\pi} \frac{1}{\lVert Q(r)'(t)\rVert^2} \left\langle Q(r)''(t) \mid \frac{\partial Q(r)'(t)}{\partial r} (\delta r)\right\rangle \, dt
    \]
    for all $r \in U$ and all $\delta r \in \mathbf{R}^n$.
    But,
    from \eqref{eq:heartsuit},
    \begin{align*}
      {Q}(r)'(t) & = (P(r) \circ \phi)'(t) = P(r)'(\phi(t))\cdot \phi'(t) \\
      {Q}(r)''(t) & = (P(r) \circ \phi)''(t) = P(r)''(\phi(t))\cdot \phi'(t)^2 + P(r)'(\phi(t))\cdot \phi''(t)
    \end{align*}
    So,
    $\alpha(\bar{\varphi} \circ P)_r(\delta r)$ is equal to the sum $A + B$ of the two following integrals,
    related to the decomposition of $Q(r)''(t)$,
    \[
    A = \int_0^{2\pi} \frac{1}{\lVert P(r)'(\phi(t)) \cdot \phi'(t)\rVert^2} \left\langle P(r)''(\phi(t))\cdot \phi'(t)^2 \mid \frac{\partial P(r)'(\phi(t)) \cdot \phi'(t)}{\partial r}(\delta r) \right\rangle \, dt,
    \]
    \[
    B = \int_0^{2\pi} \frac{1}{\lVert P(r)'(\phi(t)) \cdot \phi'(t)\rVert^2} \left\langle P(r)'(\phi(t)) \cdot \phi''(t) \mid \frac{\partial P(r)'(\phi(t)) \cdot \phi'(t)}{\partial r}(\delta r) \right\rangle \, dt.
    \]
    The first integral is equal to
    \[
    A = \int_0^{2\pi} \frac{1}{\lVert P(r)'(\phi(t))\rVert^2} \left\langle P(r)''(\phi(t)) \mid \frac{\partial P(r)'(\phi(t))}{\partial r}(\delta r) \right\rangle \, \phi'(t) dt.
    \]
    And,
    since $\varphi$,
    and thus $\phi$,
    is a positive diffeomorphism,
    after the change of variable $t \mapsto \phi(t)$,
    we get
    \[
    A = \alpha(P)_r(\delta r).
    \]
    The second integral is given by
    \[
    B = \int_0^{2\pi} \frac{1}{\lVert P(r)'(\phi(t))\rVert^2} \left\langle P(r)'(\phi(t)) \mid \frac{\partial P(r)'(\phi(t))}{\partial r}(\delta r) \right\rangle \, \frac{\phi''(t)}{\phi'(t)} \, dt
    \]
    Let us denote for short,
    \[
    x = P(r), \quad x' = P(r)', \quad\text{and}\quad \delta x' = \left[t \mapsto \frac{\partial P'(r)(t)}{\partial r}(\delta r) \right],
    \]
    such that the last integral writes
    \[
    B = \int_0^{2\pi} \frac{1}{\lVert x'(\phi(t))\rVert^2} \langle x'(\phi(t)) \mid \delta x'(\phi(t)) \rangle \, \frac{\phi''(t)}{\phi'(t)} dt.
    \]
    Let us remind that,
    for any variation $\delta$
    \[
    \delta \lVert v\rVert = \frac{1}{\lVert v\rVert} \langle v \mid \delta v \rangle \quad \Rightarrow \quad \delta \log\lVert v\rVert = \frac{1}{\lVert v\rVert} \delta \lVert v\rVert = \frac{1}{\lVert v\rVert^2} \langle v \mid \delta v \rangle.
    \]
    So,
    the integrand in the last expression of $B$ writes,
    \[
    \frac{1}{\lVert x'(\phi(t))\rVert^2} \langle x'(\phi(t)) \mid \delta x'(\phi(t)) \rangle = \delta \log \lVert x'(\phi(t))\rVert.
    \]
    Thus,
    the term $B$ becomes
    \begin{align*}
      {B} & = \int_0^{2\pi} \delta \log \lVert x'(\phi(t))\rVert \, d\log(\phi'(t)) \\
      & = \delta \int_0^{2\pi} \log \lVert x'(\phi(t))\rVert \, d\log(\phi'(t)) \\
      & = \delta \int_0^{2\pi} \log \lVert x'(\varphi^{-1}(t))\rVert \, d\log((\varphi^{-1})'(t))
    \end{align*}
    Let us make the change of variable $s = \varphi^{-1}(t)$,
    we get,
    \begin{align*}
      {B} & = + \ \delta \int_0^{2\pi} \log \lVert x'(s)\rVert \, d\log[(\varphi^{-1})'(\varphi(s))] \\
      & = - \ \delta \int_0^{2\pi} \log \lVert x'(s)\rVert \, d\log(\varphi'(s)) \\
      & = - \ \frac{\partial}{\partial r}\left\{ \int_0^{2\pi} \log \lVert P(r)'(s)\rVert \, d \log(\varphi'(s)) \right\} (\delta r) \\
      & = - \ \frac{\partial}{\partial r} \left\{ F(\varphi)(P(r)) \right\} (\delta r) \\
      & = - \ d[F(\varphi)](P)_r(\delta r).
    \end{align*}
    Coming back to $\alpha(\bar{\varphi} \circ P)_r(\delta r)$ we get finally,
    \[
    \alpha(\bar{\varphi} \circ P)_r(\delta r) = \alpha(P)_r(\delta r) - d[F(\varphi)](P)_r(\delta r) \quad \text{that is} \quad \bar{\varphi}^*(\alpha) = \alpha - d[F(\varphi)].
    \]
    Thus,
    the exterior differential $\omega = d\alpha$ is invariant by the action of $\Diff_+({\mathbf{S}}^1)$.
    And since the difference $\bar{\varphi}^*(\alpha) - \alpha$ is exact,
    this action is hamiltonian.
    
    \emph{The 2-point moment map} --- Now,
    let us compute the 2-points moment maps $\psi$ of the action of $\Diff_+(\mathbf{S}^1)$ on $(\Imm(\mathbf{S}^1,\mathbf{R}^2),\omega)$.
    Let $p$ be a path connecting two immersions $x_0$ and $x_1$.
    We have $\Psi(p) = \hat{p}^*(K\omega) = \hat{p}^*(K d\alpha) = \hat{p}^*(\hat{1}^*(\alpha) - \hat{0}^*(\alpha) - d(K\alpha)) = \hat{x}_1^*(\alpha) - \hat{x}_0^*(\alpha) - d(K\alpha \circ \hat{p})$.
    But,
    for all $\varphi \in \Diff_+(\mathbf{S}^1)$,
    \[
    K\alpha \circ \hat{p}(\varphi) = \int_{\bar{\varphi} (p)} \alpha = \int_p \bar{\varphi}^*(\alpha) = \int_p\alpha - \int_p dF(\varphi) = \int_p\alpha - F(\varphi)(x_1) + F(\varphi)(x_0).
    \]
    So,
    we get finally
    \[
    \Psi(p) = \psi(x_0,x_1) = \{\hat{x}_1^*(\alpha) + d[\varphi \mapsto F(\varphi)(x_1)]\} - \{\hat{x}_0^*(\alpha) + d[\varphi \mapsto F(\varphi)(x_0)]\}.
    \]
    But notice that,
    $\hat{x}^*(\alpha) + d[\varphi \mapsto F(\varphi)(x)$ is not a momentum of $\Diff_+(\mathbf{S}^1)$.
    
    \emph{The 1-point moment maps} --- Let us compute the moment map $\psi(x_0,x)$.
    Let
    \[
    m = \{\hat{x}^*(\alpha) + d[\varphi \mapsto F(\varphi)(x)]\}(r \mapsto \varphi)_r(\delta r).
    \]
    And,
    let us denote for short
    \begin{align*}
      {A} &= \hat{x}^*(\alpha)(r \mapsto \varphi)_r(\delta r) \\
      {B} &= d[\varphi \mapsto F(\varphi)(x)](r \mapsto \varphi)_r(\delta r) = \frac{\partial F(\varphi)(x)}{\partial r} \delta r.
    \end{align*}
    We shall use the notation $m_0$,
    $A_0$ and $B_0$ for the immersion $x_0$.
    Thus,
    \[
    \psi(x_0,x)(r \mapsto \varphi)_r(\delta r) = m - m_0 = A + B - A_0 - B_0.
    \]
    We have,
    $ \hat{x}^*(\alpha)(r \mapsto \varphi) = \alpha(\hat{x} \circ [r \mapsto \varphi]) = \alpha(r \mapsto x \circ \varphi^{-1})$.
    Let $\phi = \varphi^{-1}$,
    so
    \[
    A = \int_0^{2\pi} \frac{1}{\lVert (x \circ \phi)'(t)\rVert^2} \left\langle (x \circ \phi)''(t) \mid \frac{\partial (x \circ \phi)'(t)}{\partial r}(\delta r) \right\rangle.
    \]
    Let us introduce now,
    \[
    u = \phi(t), \quad u'=\phi(t) \quad\text{and}\quad u'' = \phi''(t).
    \]
    So,
    the decomposition given by \eqref{eq:heartsuit},
    writes
    \[
    (x \circ \phi)'(t) = x'(u) \cdot u' \quad\text{and}\quad (x \circ \phi)''(t) = x''(u) \cdot u'^2 + x'(u) \cdot u''.
    \]
    Then,
    we shall use the prefix $\delta$ for every variation associated to $\delta r$,
    that is $\delta \star = D(r \mapsto \star)(r)(\delta r)$.
    So,
    \[
    \frac{\partial (x \circ \phi)'(t)}{\partial r}(\delta r) = \delta[x'(u) \cdot u'] = x''(u) \cdot \delta u \cdot u' + x'(u) \cdot \delta u'.
    \]
    Thus,
    \begin{align*}
      {A} &= \int_0^{2\pi} \frac{1}{\lVert x'(u)\rVert^2 u'^2} \langle x''(u) u'^2 + x'(u) u'' \mid x''(u) u' \delta u + x'(u) \delta u' \rangle \, dt \\
      &= \int_0^{2\pi} \frac{\lVert x''(u)\rVert^2}{\lVert x'(u)\rVert^2} \ \delta u \ u' dt + \int_0^{2\pi} \frac{\langle x'(u), x''(u)\rangle}{\lVert x'(u)\rVert^2} \left[ \delta u' + \frac{u''}{u'} \delta u \right] \, dt + \int_0^{2\pi} \frac{u''}{u'} \delta u' \, dt
    \end{align*}
    Now,
    \[
    B = \frac{\partial F(\varphi)(x)}{\partial r} \delta r = - \frac{\partial \bar{F}(\phi)(x)}{\partial r} \delta r = - \delta [\bar{F}(\phi)(x)],
    \]
    with
    \[
    \bar{F}(\phi)(x) = \int_0^{2\pi} \log \lVert x'(\phi(t))\rVert \, d \log \phi'(t) = \int_0^{2\pi} \log\lVert x'(u)\rVert \, d\log(u').
    \]
    So,
    after the variation with respect to $\delta r$ and an integration by part,
    we get
    \begin{align*}
      {B} &= - \int_0^{2\pi} \frac{\langle x'(u), x''(u)\rangle}{\lVert x'(u)\rVert^2} \delta u \frac{u''}{u'} \, dt - \int_0^{2\pi} \log \lVert x'(u)\rVert \ \delta d \log(u') \\
      &= - \int_0^{2\pi} \frac{\langle x'(u), x''(u)\rangle}{\lVert x'(u)\rVert^2} \delta u \frac{u''}{u'} \, dt + \int_0^{2\pi} \frac{\langle x'(u), x''(u)\rangle}{\lVert x'(u)\rVert^2} u' \ \delta \log(u') \, dt \\
      &= - \int_0^{2\pi} \frac{\langle x'(u), x''(u)\rangle}{\lVert x'(u)\rVert^2} \delta u \frac{u''}{u'} \, dt + \int_0^{2\pi} \frac{\langle x'(u), x''(u)\rangle}{\lVert x'(u)\rVert^2} \ \delta u' \, dt
    \end{align*}
    Therefore,
    grouping the terms and integrating again by part,
    we get
    \begin{align*}
      {A} + B &= \int_0^{2\pi} \frac{\lVert x''(u)\rVert^2}{\lVert x'(u)\rVert^2} \delta u \, du + 2 \int_0^{2\pi} \frac{\langle x'(u), x''(u)\rangle}{\lVert x'(u)\rVert^2} \delta u' \, dt + \int_0^{2\pi} \frac{u''}{u'} \delta u' \, dt \\
      &= \int_0^{2\pi} \frac{\lVert x''(u)\rVert^2}{\lVert x'(u)\rVert^2} \delta u \, du - 2 \int_0^{2\pi} \frac{d^2}{du^2}\log\lVert x'(u)\rVert \delta u \, du + \int_0^{2\pi} \frac{u''}{u'} \ \delta u' \, dt \\
      &= \int_0^{2\pi} \left\{\frac{\lVert x''(u)\rVert^2}{\lVert x'(u)\rVert^2} - \frac{d^2}{du^2}\log\lVert x'(u)\rVert^2 \right\} \delta u \, du + \int_0^{2\pi} \frac{u''}{u'} \ \delta u' \, dt
    \end{align*}
    Now,
    since $\lVert x_0'(t)\rVert = 1$ we get the value of the 2-point moment map,
    \[
    \psi(x_0,x)(r \mapsto \varphi)_r(\delta r) = \int_0^{2\pi} \left\{\frac{\lVert x''(u)\rVert^2}{\lVert x'(u)\rVert^2} - \frac{d^2}{du^2}\log\lVert x'(u)\rVert^2 \right\} \delta u \, du - \int_0^{2\pi} \delta u \, du.
    \]
    The second term of the right hand side of the equality is a constant momentum of $\Diff_+(\mathbf{S}^1)$,
    so it can be avoided.
    And,
    every moment map is,
    up to a constant,
    equal to the moment $\mu$ announced.
    
    \emph{Souriau's cocycles} --- Souriau's cocycle associated to immersion $x_0$ is defined by $\theta(g) = \psi(x_0,\bar{g}(x_0))$,
    see art.~\ref{Souriau-cocycles}.
    So,
    we have to replace,
    in the expression of $\psi$ above,
    $x$ by $\bar{g} (x_0) = x_0 \circ g^{-1}$,
    that is $x= x_0 \circ \gamma$.
    So,
    $\theta(g)(r \mapsto \varphi)_r(\delta r) = \psi(x_0,x_0 \circ \gamma)$.
    So,
    note first that
    \[
    (x_0 \circ \gamma)'(u) = x_0'(\gamma(u)) \gamma'(u) \quad\text{and}\quad (x_0 \circ \gamma)''(u) = x_0''(\gamma(u)) \gamma'(u)^2 + x_0'(u) \gamma''(u).
    \]
    And,
    let us remind that $\lVert x_0'\rVert = \lVert x_0''\rVert = 1$ and $\langle x_0' \mid x_0'' \rangle = 0$.
    We get,
    \[
    \lVert x'(u)\rVert^2 = \gamma'(u)^2 \quad\text{and}\quad \lVert x''(u)\rVert^2 = \gamma'(u)^4 + \gamma''(u)^2.
    \]
    This gives
    \[
    \frac{\lVert x''(u)\rVert^2}{\lVert x'(u)\rVert^2} = \gamma'(u)^2 + \frac{\gamma''(u)^2}{\gamma'(u)^2} \quad\text{and}\quad \frac{d^2}{du^2}\log\lVert x'(u)\rVert^2 = 2\frac{\gamma'''(u) \gamma'(u) - \gamma''(u)^2}{\gamma''(u)^2}.
    \]
    Thus,
    \begin{align*}
      \theta(g)(r \mapsto \varphi)_r(\delta r) &= \int_0^{2\pi} \frac{ 3 \gamma''(u)^2 - 2 \gamma'''(u) \gamma'(u)}{\gamma'(u)^2} \ \delta u \ du \\
      &+ \int_0^{2\pi} \gamma'(u)^2 \ \delta u \ du - \int_0^{2\pi} \delta u \ du.
    \end{align*}
    But,
    after a change of variable $u \mapsto v = \gamma(u)$,
    we get
    \[
    \int_0^{2\pi} \gamma'(u)^2 \ \delta u \ du = \int_0^{2\pi} (\delta u \gamma'(u)) \ \gamma'(u) du = \int_0^{2\pi} \delta v \ dv.
    \]
    So the two last terms cancel each other,
    and we get the value announced for Souriau's cocycle $\theta$.
    
    \emph{Bott's cocycle} --- The real function $F(g \circ h) - F(g) \circ \bar{h} - F(h)$ is constant since $X$ is connected,
    and its differential is equal to $(\bar{g} \circ \bar{h})^*(\alpha) - \bar{h}^*(\bar{g}^*(\alpha))$,
    that is $0$.
    Now,
    to explicit $\beta(g,g') = F(g)\circ \bar{g}' + F(g') - \beta(g,g') - F(g \circ g')$,
    it is sufficient to compute the right hand member on the standard immersion $x_0$,
    for which the speed norm is equal to 1,
    and thus $\log \lVert x'(t)\rVert = 0$ for all real $t$.
    So we get,
    \begin{align*}
      \beta(g,h) &= {F}(g)(x_0 \circ h^{-1}) - F(h)(x_0) - F(g \circ h)(x_0) \\
      &= + \int_0^{2\pi} \log\lVert (x_0 \circ h^{-1})'(t)\rVert \ d \log g'(t) \\
      &= + \int_0^{2\pi} \log (h^{-1})'(t) \ d\log g'(t) \\
      &= - \int_0^{2\pi} \log h'(h^{-1}(t)) \ d\log g'(t) \\
      &= - \int_0^{2\pi} \log h'(s) \ d\log g'(h(s)) \\
      &= + \int_0^{2\pi} \log(g \circ h)'(t) \ d\log h'(t)
    \end{align*}
    And this is the standard expression of Bott's cocycle.
  \end{proof}
  
  
  \begin{thebibliography}{CDM88}
    
    \bibitem[Ban78]{Ban78}
    Augustin Banyaga.
    \newblock \emph{Sur la structure du groupe des diff\'eomorphismes qui pr\'eservent une forme symplectique}.
    \newblock Comment. Math. Helv., volume 53, pages 174--227, 1978.
    
    \bibitem[Boo69]{Boo69}
    William M. Boothby.
    \newblock \emph{Transitivity of the automorphisms of certain geometric structures}.
    \newblock Trans. Amer. Math. Soc., volume 137, pages 93--100, 1969.
    
    \bibitem[Bot78]{Bot78}
    Raoul Bott.
    \newblock \emph{On some formulas for the characteristic classes of group actions, differential topology, foliations and Gelfand-Fuchs cohomology}.
    \newblock In \emph{Proceed. Rio de Janeiro, 1976}, volume 652 of \emph{Springer Lectures Notes}. Springer Verlag, 1978.
    
    \bibitem[Che77]{Che77}
    Kuo Tsai Chen.
    \newblock \emph{Iterated path integral}.
    \newblock Bull. of Am. Math. Soc., volume 83, number 5, pages 831--879, 1977.
    
    \bibitem[CDM88]{CDM88}
    M. Condevaux, P. Dazord and P. Molino.
    \newblock \emph{G\'eom\'etrie du moment}.
    \newblock Travaux du S\'eminaire Sud-Rhodanien de G\'eom\'etrie, I, Publ. D\'ep. Math. Nouvelle S\'er. B 88-1, Univ. Claude-Bernard, pages 131--160, Lyon, 1988.
    
    \bibitem[Dnl99]{Dnl99}
    Simon K. Donaldson.
    \newblock \emph{Moment maps and diffeomorphisms}.
    \newblock Asian Journal of Math, vol. 3, pages 1--16, 1999.
    
    \bibitem[Don84]{Don84}
    Paul Donato.
    \newblock \emph{Rev\^etement et groupe fondamental des espaces diff\'e\-rentiels homog\`enes}.
    \newblock Th\`ese de doctorat d'\'etat, Universit\'e de Provence, Marseille, 1984.
    
    \bibitem[Don88]{Don88}
    Paul Donato.
    \newblock \emph{G\'eom\'etrie des orbites coadjointes des groupes de diff\'eomorphismes}.
    \newblock In \emph{Lect. Notes In Maths}, vol. 1416, pages 84--104, 1988.
    
    \bibitem[Igl85]{Igl85}
    Patrick Iglesias.
    \newblock \emph{Fibr\'es diff\'eologiques et homotopie}.
    \newblock Th\`ese de doctorat d'\'e\-tat, Universit\'e de Provence, Marseille, 1985.
    \newblock \url{http://math.huji.ac.il/~piz/documents/These.pdf}
    
    \bibitem[IKZ05]{IKZ05}
    Patrick Iglesias, Yael Karshon and Moshe Zadka.
    \newblock \emph{Orbifolds as diffeologies}.
    \newblock 2005.
    \newblock \url{http://arxiv.org/abs/math.DG/0501093}
    
    \bibitem[IL90]{IL90}
    Patrick Iglesias \& Gilles Lachaud.
    \newblock \emph{Espaces diff\'erentiables singuliers et corps de nombres alg\'ebriques}.
    \newblock Ann. Inst. Fourier, Grenoble, volume 40, number 1, pages 723--737, 1990.
    
    \bibitem[Igl95]{Igl95}
    Patrick Iglesias.
    \newblock \emph{La trilogie du moment}.
    \newblock Ann. Inst. Fourier, 45, 1995.
    
    \bibitem[Piz05]{Piz05}
    Patrick Iglesias-Zemmour.
    \newblock \emph{Diffeology}.
    \newblock eprint, 2005--07.
    \newblock \url{http://math.huji.ac.il/~piz/diffeology/}
    
    \bibitem[Piz06-a]{Piz06-a}
    Patrick Iglesias-Zemmour.
    \newblock \emph{Diffeology of the Infinite Hopf Fibration}.
    \newblock eprint, 2006.
    \newblock \url{http://math.huji.ac.il/~piz/documents/DIHF.pdf}
    
    \bibitem[Piz06-b]{Piz06-b}
    Patrick Iglesias-Zemmour.
    \newblock \emph{Dimension in diffeology}.
    \newblock eprint, 2006.
    \newblock \url{http://math.huji.ac.il/~piz/documents/DID.pdf}
    
    \bibitem[Piz07-a]{Piz07-a}
    Patrick Iglesias-Zemmour.
    \newblock \emph{Variations of integrals in diffeology}.
    \newblock eprint, 2007.
    \newblock \url{http://math.huji.ac.il/~piz/documents/VOIID.pdf}
    
    \bibitem[Piz07-c]{Piz07-c}
    Patrick Iglesias-Zemmour.
    \newblock \emph{Every symplectic manifold is a coadjoint orbit}.
    \newblock eprint, 2007.
    \newblock \url{http://math.huji.ac.il/~piz/documents/ESMIACO.pdf}
    
    \bibitem[Kir74]{Kir74}
    Alexandre A. Kirillov.
    \newblock \emph{Elements de la th\'eorie des repr\'esentations}.
    \newblock Editions MIR, Moscou, 1974.
    
    \bibitem[Kos70]{Kos70}
    Bertram Kostant.
    \newblock \emph{Orbits and quantization theory}.
    \newblock In Congr\`es international des math\'ematiciens 1970-1971.
    
    \bibitem[Omo86]{Omo86}
    Stephen Malvern Omohundro.
    \newblock \emph{Geometric Perturbation Theory in Physics}.
    \newblock World Scientific, 1986.
    
    \bibitem[Sou70]{Sou70}
    Jean-Marie Souriau.
    \newblock \emph{Structure des syst\`emes dynamiques}.
    \newblock Dunod, Paris, 1970.
    
    \bibitem[Sou81]{Sou81}
    Jean-Marie Souriau.
    \newblock \emph{Groupes diff\'erentiels}.
    \newblock Lecture notes in mathematics, Springer Verlag, New-York, volume 836, pages 91--128, 1981.
    
    \bibitem[Sou84]{Sou84}
    Jean-Marie Souriau.
    \newblock \emph{Groupes diff\'erentiels et physique math\'ematique}.
    \newblock Lecture Notes in Physics, Springer Verlag, Berlin -- Heidelberg, volume 201, pages 511--513, 1984.
    
    \bibitem[Zie96]{Zie96}
    Fran\c cois Ziegler.
    \newblock \emph{Th\'eorie de Mackey symplectique, in M\'ethode des or\-bi\-tes et repr\'esentations quantiques}.
    \newblock Th\`ese de doctorat d'Uni\-ver\-sit\'e, Universit\'e de Provence, Marseille, 1996.
    
  \end{thebibliography}
  
  %%%%%%%%%%%%%%%%%%%%%%%%%%%%%%%%%%%%%%%%%%%%%%%%%%%%
  % Fin du document
  %%%%%%%%%%%%%%%%%%%%%%%%%%%%%%%%%%%%%%%%%%%%%%%%%%%%
\end{document}
