%%%%%%%%%%%%%%%%%%%%%%%%%%%%%%%%%%%%%%%%%%%%%%%%%%%%%%%%%%
%%
%%  Example of Singular Reduction in Symplectic Diffeology
%%  Patrick Iglesias-Zemmour
%%  Archived: 2025
%%
%%%%%%%%%%%%%%%%%%%%%%%%%%%%%%%%%%%%%%%%%%%%%%%%%%%%%%%%%%

\documentclass[11pt,reqno,letterpaper]{amsart}

%--------------------------------------------------------------------------
% Packages
%--------------------------------------------------------------------------
\usepackage[utf8]{inputenc}
\usepackage[T1]{fontenc}
\usepackage{amssymb}
\usepackage{amscd}
\usepackage{tikz-cd}
\usetikzlibrary{calc}
\usepackage[hidelinks]{hyperref}

%--------------------------------------------------------------------------
% Page Layout & Typography
%--------------------------------------------------------------------------
\parindent 0mm
\parskip .5ex plus 2pt
\linespread{1.1}

%--------------------------------------------------------------------------
% Theorem Styles
%--------------------------------------------------------------------------
\newtheoremstyle{article}% ⟨name⟩
{7pt}%  ⟨Space above⟩
{7pt}%  ⟨Space below⟩
{}%     ⟨Body font⟩
{0pt}%  ⟨Indent amount⟩
{\bf}%  ⟨Theorem head font⟩
{.\ }%  ⟨Punctuation after theorem head⟩
{0pt}%  ⟨Space after theorem head⟩
{}%     ⟨Theorem head spec⟩

\theoremstyle{article}
\newtheorem{article}{}

\newcommand{\artlabel}[1]{\textbf{\textsc{#1}}}
\newcommand{\art}[1]{(\textsection\ref{#1})}
\newcommand{\alinea}[1]{\vskip-\lastskip\vspace{.25\baselineskip plus 2pt minus 2pt} \noindent\textsc{#1}}

%--------------------------------------------------------------------------
% Macros (Integrated from EOSRISD-macro.tex)
%--------------------------------------------------------------------------

% Upright letters
\def \F{\mathrm F}
\def \G{\mathrm G}
\def \K{\mathrm K}
\def \M{\mathrm M}
\def \N{\mathrm N}
\def \Q{\mathrm Q}
\def \T{\mathrm T}
\def \U{\mathrm U}
\def \V{\mathrm V}
\def \X{\mathrm X}
\def \Z{\mathrm Z}

% Sets
\newcommand{\RR}{\mathbf{R}}
\newcommand{\CC}{\mathbf{C}}
\newcommand{\ZZ}{\mathbf{Z}}
\newcommand{\NN}{\mathbf{N}}
\newcommand{\PP}{\mathbf{P}}
\newcommand{\QQ}{\mathbf{Q}}

% Calligraphic
\newcommand{\cB}{\mathcal{B}}
\newcommand{\cE}{\mathcal{E}}
\newcommand{\cG}{\mathcal{G}}
\newcommand{\cK}{\mathcal{K}}
\newcommand{\cO}{\mathcal{O}}
\newcommand{\cQ}{\mathcal{Q}}
\newcommand{\cS}{\mathcal{S}}
\newcommand{\cT}{\mathcal{T}}

% Operators
\DeclareMathOperator{\Ad}{Ad}
\DeclareMathOperator{\Cinfty}{\mathcal{C}^\infty}
\DeclareMathOperator{\Diff}{Diff}
\DeclareMathOperator{\dom}{dom}
\DeclareMathOperator{\GL}{GL}
\DeclareMathOperator{\id}{\mathbf 1}
\DeclareMathOperator{\Loops}{Loops}
\DeclareMathOperator{\Paths}{Paths}
\DeclareMathOperator{\Tr}{Tr}
\DeclareMathOperator{\Surf}{Surf}
\DeclareMathOperator{\Maps}{Maps}
\DeclareMathOperator{\pr}{pr}

% Symbols
\newcommand{\0}{\hat 0}
\newcommand{\1}{\hat 1}
\newcommand{\CHK}{\mathcal{K}}
\newcommand{\class}{\operatorname{class}}
\newcommand{\dt}{\mathop{dt}\nolimits}
\newcommand{\eends}{\operatorname{ends}}
\newcommand{\modulus}[1]{\vert #1 \vert}
\newcommand{\oline}[1]{\overline{#1}}
\newcommand{\qmbox}[1]{\quad\mbox{#1}\quad}
\newcommand{\QPS}{{\mathrm{QP}}_{\alpha}^\infty}
\newcommand{\Torus}{\mathbb{T}}

\usepackage[T1]{fontenc}

\usepackage[cal=scr,uppercase = upright,greekfamily = didot,greeklowercase = upright,utopia]{mathdesign}

\linespread{1.1}


%--------------------------------------------------------------------------
% Title Data
%--------------------------------------------------------------------------

\begin{document}

\title[Example of Singular Reduction in Symplectic Diffeology]{Example of Singular Reduction \\ in Symplectic Diffeology}

\author{Patrick Iglesias-Zemmour}
\email{piz@math.huji.ac.il}
\urladdr{http://math.huji.ac.il/~piz/}

\date{2016 (Published) / 2025 (Archived)}

\maketitle

\begin{abstract}
We present an example of symplectic reduction
in diffeology where the space involved is infinite
dimensional and the reduction is singular.
This example is a mix of two cases that are not handled by ordinary symplectic geometry.
We show that, in this infinite dimensional example,
the singularities are distributed in such a way that
the symplectic form, restricted to a generic level of the moment map,
passes to the reduced space.
\end{abstract}

%%%%%%%%%%%%%%%%%%%%%%%%%%%%%%%%%%%%%%%%%%%%%%%%%%%%%%%%%%
%%
%% MARK: Introduction
%%
%%%%%%%%%%%%%%%%%%%%%%%%%%%%%%%%%%%%%%%%%%%%%%%%%%%%%%%%%%
\section*{Introduction}

We equip the space of periodic functions from $\RR$ to $\CC$ with a homogeneous symplectic structure.
Then,
we push this structure forward on the space of Fourier coefficients.
We give an autonomous description of the diffeology we obtain that way.
Next,
we equip the infinite product of toruses with a {\em tempered diffeology},
introduced for the purpose of the example.
Equipped with this diffeology,
the infinite torus acts smoothly by automorphisms on the symplectic space of Fourier coefficients.
We compute the moment map of the action of the infinite torus.
Given a sequence of rationally independent real numbers we induce the real line in the infinite torus
as an \textit{irrational solenoid} and consider the induced action on the space of Fourier coefficients.
This action is not free and generates infinitely many singular orbits.
We consider then a generic level subspace of the moment map of the solenoid
(they are all isomorphic to the infinite sphere) and
we reduce it by the action of the solenoid.
We call that quotient an \textit{infinite quasi-projective space}.
We show eventually that, in spite of the presence of singular orbits, the restriction
of the symplectic form passes to the quotient and equip every infinite quasi-projective space with
a \emph{parasymplectic structure},
that is,
a closed $2$-form.

This is a very informative example. It shows that, not only is diffeology a very versatile and flexible
approach of differential geometry,
but it offers moreover the right framework to deal with singularities, even in infinite dimensional situation
where all the difficulties are met.
For the basic constructions and the vocabulary of diffeology used here we recommend the lecture of the book \cite{PIZ13}.

\textbf{A few words on Symplectic Diffeology.} For the sake of simplicity I decided to call \textit{parasymplectic space}
any diffeological space equipped with a closed $2$-form.
In diffeology there may be a few different ways to understand the word \emph{symplectic}%
\footnote{For example, a symplectic space could also be defined as a parasymplectic space generated by symplectic plots,
as it has been suggested once by Yael Karshon.
That would include,
for example,
the cone orbifold $\cQ_m= \CC/\ZZ_m$,
which is excluded by the definition of this article.}.
The definition I present here uses the two main tools at our disposal:
the group of automorphisms of the parasymplectic structure and the universal moment map%
\footnote{The \emph{moment maps} relative to a closed $2$-form are described in \cite{PIZ10} or in chapter 9 of \cite{PIZ13}.
The \emph{universal moment map} is the moment map for the group of all the automorphisms of the $2$-form.}.
We recall that the moment maps are defined for any parasymplectic space and do not need any particular conditions
on the parasymplectic form. Actually what we really need is the pseudogroup of local automorphisms.
Then we can distinguish four cases for a parasymplectic space:

\begin{enumerate}

\item[($+\ +$)] The local automorphisms are transitive and the universal moment map has discrete prevalues%
\footnote{We call \emph{prevalues} of a function, the preimages of its values.}.
For manifolds, that is a characterisation of symplectic forms%
\footnote{For manifold the universal moment map is actually injective.}.
In this case, we shall say that the diffeological space, equipped with such a parasymplectic form, is \textbf{symplectic}%
\footnote{A stronger case is when the group of automorphisms is transitive,
in this case the space is symplectic homogeneous.}.

\item[($+\ -$)] The local automorphisms are transitive but the the prevalues of the moment map are not discrete.
For manifolds, that is a characterization of presymplectic forms.
We shall say in this case that the parasymplectic form is \textbf{presymplectic}.

\item[($-\ +$)] The local automorphisms are not transitive but the moment map has discrete prevalues.
That is the case of some classic pictures like cone orbifolds for examples;
or forms, symplectic everywhere except in a few points, where it vanishes.
We have no special vocabulary yet for this kind of space.

\item[($-\ -$)] The local automorphisms are not transitive and the prevalues of the moment map are not discrete.
That is the general case of parasymplectic spaces.
In this case,
as well as the previous one,
we can still analyse the space in terms of orbits of the pseudogroup of automorphisms and the various level of the universal moment map in each orbit.

\end{enumerate}

\bigskip\noindent\textbf{Thanks} to Elisa Prato who invited me to Firenze for discussing the subject,
and also to Fiama Bataglia.
The discussion I had with them led me to develop this example as an infinite dimensional version of one of their cases,
see \cite{EP01}.
It would be certainly interesting to develop more seriously the diffeological version of their general constructions.
Thanks also to the anonymous referee for his careful reading of the manuscript,
for his advices and corrections that contributed to improve it.


%%%%%%%%%%%%%%%%%%%%%%%%%%%%%%%%%%%%%%%%%%%%%%%%%%%%%%%%%%
%%
%% MARK: Review on the Moment Maps of a Parasymplectic Form
%%
%%%%%%%%%%%%%%%%%%%%%%%%%%%%%%%%%%%%%%%%%%%%%%%%%%%%%%%%%%
\section*{Review on the Moment Maps of a Parasymplectic Form}

Let $(\X,\omega)$ be a parasymplectic space.
Let $\G$ be a diffeological group,
with a smooth action $g \mapsto g_\X$ on $\X$,
preserving $\omega$,
that is,
$g_\X^*(\omega)=\omega$ for all $g\in \G$.
We denote by $\cG^*$ the space of \emph{momenta} of $\G$,
defined as the left-invariant differential $1$-forms on $\G$,
$$
\cG^* = \{\varepsilon \in \Omega^1(\G) \mid \mathrm{L}(g)^*(\varepsilon) = \varepsilon, \mbox{ for all } g \in \G \}.
$$
To understand the essential nature of the moment map,
which is a map from $\X$ to $\cG^*$,
it is good to consider the simplest case,
and use it then as a guide to extend this simple construction to the general case.

\textbf{The Simplest Case}. Consider the case where $\X$ is a manifold,
and $\G$ is a Lie group.
Let us assume that $\omega$ is exact
$\omega = d\alpha$,
and that $\alpha$ is also invariant by $\G$.
Regarding $\omega$,
the \emph{moment map}%
\footnote{Precisely, one moment map, since they are defined up to a constant.}
of the action of $\G$ on $\X$ is the map
$$
\mu : \X \to \cG^*
\quad
\mbox{ defined by }
\quad
\mu(x) = \hat x^*(\alpha),
$$
where $\hat x : \G \to \X$ is the \emph{orbit map} $\hat x (g)= g_\X(x)$.

As we can see,
there is no obstacle,
in this simple situation,
to generalize, \emph{mutatis mutandis},
the moment map to a diffeological group acting by automorphisms on a diffeological parasymplectic space%
\footnote{Actually,
we shall be in this simple situation further,
with the action of the infinite torus on the space of rapidly decreasing sequences.}.
But,
as we know,
not all closed $2$-forms are exact,
and even if they are exact,
they do not necessarily have an invariant primitive.
We shall see now,
how we can generally come to a situation,
so close to the simple case above,
that modulo some minor subtleties we can build a good moment map in all cases.

\textbf{The General Case}. We consider a connected parasymplectic diffeological space $(\X,\omega)$,
and a diffeological group $\G$ acting on $\X$ and preserving $\omega$.
Let $\cK$ be the Chain-Homotopy Operator, defined in \cite[6.83]{PIZ13}.
The differential $1$-form $\cK\omega$,
defined on $\Paths(\X)$,
is related to $\omega$ by $d[\cK\omega] = (\1^* - \0^*)(\omega)$,
and $\cK\omega$ is invariant by $\G$.
Considering $\bar \omega = (\1^* - \0^*)(\omega)$ and $\bar\alpha = \cK\omega$,
we are in the simple case:
$\bar\omega = d\bar\alpha$ and $\bar\alpha$ invariant by $\G$.
We can apply the construction above and define then the \emph{Moment Map of Paths} by
$$
\Psi : \Paths(\X) \to \cG^*
\quad \mbox{with} \quad
\Psi(\gamma) = \hat \gamma^*(\cK \omega),
$$
and $\hat \gamma : \G \to \Paths(\X)$ is the orbit map $\hat \gamma(g)= g_\X \circ \gamma$ of a path $\gamma$.
The moment of paths is additive with respect to the concatenation,
$$
\Psi(\gamma \vee \gamma') = \Psi(\gamma) + \Psi(\gamma').
$$
This paths moment map $\Psi$ is equivariant by $\G$,
acting by composition on $\Paths(\X)$,
and by coadjoint action on $\cG^*$.
Next,
defining the \emph{Holonomy} of the action of $\G$ on $\X$ by
$$
\Gamma = \{\Psi(\ell) \mid \ell \in \Loops(\X) \} \subset \cG^*,
$$
the \emph{Two-Points Moment Map} is defined by pushing $\Psi$ forward on $\X\times\X$,
$$
\psi(x,x') = \class(\Psi(\gamma)) \in \cG^*/\Gamma,
$$
where $\gamma$ is a path connecting $x$ to $x'$,
and where $\class$ denotes the projection from $\cG^*$ onto its quotient $\cG^*/\Gamma$.
The holonomy $\Gamma$ is the obstruction for the action of $\G$ to be \emph{Hamiltonian}.
The additivity of $\Psi$ becomes the Chasles' cocycle condition
$$
\psi(x,x') + \psi(x',x'') = \psi(x,x'').
$$
The group $\Gamma$ is invariant by the coadjoint action.
Thus,
the coadjoint action passes to the quotient $\cG^*/\Gamma$,
and $\psi$ is equivariant.
Because $\X$ is connected,
there exists always a map
$$
\mu : \X \to \cG^*/\Gamma
\quad \mbox{such that} \quad
\psi(x,x') = \mu(x') - \mu(x).
$$
Up to an arbitrary constant,
$\mu$ is given by $\mu(x)=\psi(x_0,x)$,
where $x_0$ is a chosen point in $\X$.
But this map is \textit{a priori\/} no longer equivariant.
Its variance introduce a $1$-cocycle $\theta$ of $\G$ with values in $\cG^*/\Gamma$ such that
$$
\mu(g(x))= \Ad_*(g)(\mu(x)) + \theta(g),
$$
with $\theta(g) = \psi(x_0,g(x_0))$, modulo a coboundary due to the constant in the choice of $\mu$.
This construction extends to the category \{Diffeology\},
the moment map for manifolds introduced by Souriau in \cite{Sou70}.
The remarkable point is that none of the constructions brought up above involves differential equations,
and there is no need of considering a putative Lie algebra either.
That is a very important point.
The momenta appear as invariant $1$-forms on the group,
naturally,
without intermediary,
and the moment map as a map in the space of momenta.

Note that the group of automorphisms $\G_\omega = \Diff(\X,\omega)$ is a legitimate diffeological group.
The above constructions apply and give rise to universal objects:
\emph{universal momenta} $\cG_\omega^*$,
\emph{universal path moment map} $\Psi_\omega$,
\emph{universal holonomy} $\Gamma_\omega$,
\emph{universal two-points moment map} $\psi_\omega$,
\emph{universal moment maps} $\mu_\omega$,
\emph{universal Souriau's cocycles} $\theta_\omega$.
A \emph{parasymplectic action} of a diffeological group $\G$ is a smooth morphism $h : \G \to \G_\omega$,
and the objects,
associated with $\G$,
introduced by the above moment maps constructions,
are naturally subordinate to their universal counterparts.

%%%%%%%%%%%%%%%%%%%%%%%%%%%%%%%%%%%%%%%%%%%%%%%%%%%%%%%%%%
%%
%% MARK: Diffeology on Fourier Coefficients of Periodic Functions
%%
%%%%%%%%%%%%%%%%%%%%%%%%%%%%%%%%%%%%%%%%%%%%%%%%%%%%%%%%%%
\section*{Diffeology on Fourier Coefficients of Periodic Functions}

We denote by $\Cinfty_\mathrm{per}(\RR,\CC)$ the space of 1-periodic functions defined on $\RR$ with values in $\CC$,
that is,
$$
\Cinfty_\mathrm{per}(\RR,\CC) = \{f \in \Cinfty(\RR,\CC) \mid f(x+1) = f(x) \}.
$$
We equip this space with the functional diffeology. Let us recall that a plot for
this diffeology is a parametrization $\P : \U \to \Cinfty_\mathrm{per}(\RR,\CC)$ such that the
map $(r,t) \mapsto \P(r)(t)$, from $\U \times \RR$ to $\CC$, is smooth.

Let $f \in \Cinfty_\mathrm{per}(\RR,\CC)$, we denote by $(f_n)_{n\in \ZZ}$ its sequence of Fourier coefficients,
that is,
$$
f_n = \int_0^1 f(x) e^{-2i\pi n x} \, dx, \mbox{ for all } n \in \ZZ.
$$
We recall that the Fourier series converges to $f$, uniformly on $[0,1]$
\cite[\textsection 2 Thm.~1]{Vil68}.
We denote by
$$
f(x)=\sum_{n \in \ZZ} f_n\,e^{2i\pi n x} \qmbox{the limit} f(x) = \lim_{\N \to \infty} \sum_{n=-\N}^{+\N} f_n\,e^{2i\pi n x}.
$$
Let $j$ be the map from $\Cinfty_\mathrm{per}(\RR,\CC)$ to the set of series of complex numbers
$\Maps(\ZZ,\CC)$, that associates a smooth periodic function with its Fourier coefficients,
$$
j : \Cinfty_\mathrm{per}(\RR,\CC) \to \Maps(\ZZ,\CC) \qmbox{with} j(f) = (f_n)_{n\in \ZZ}.
$$
That map $j$ is injective, and we denote by $\cE$ its image, that is,
the subspace of $\Maps(\ZZ,\CC)$ made of the Fourier coefficients of
smooth periodic functions,
$$
\cE = \left\{ (f_n)_{n\in\ZZ} = j(f) \mid f \in \Cinfty_\mathrm{per}(\RR,\CC)\right\}.
$$
We know that the subspace $\cE$ is made exactly of all rapidly decreasing sequences of complex numbers (\textit{op. cit.}),
$$
(f_n)_{n\in \ZZ} \in \cE \qmbox{iff, for all $p \in \NN$,} n^p f_n \xrightarrow[\modulus{n} \to \infty]{} 0.
$$

\begin{article}\artlabel{Functional Diffeology on Fourier Coefficients.}\
A parametrizations $\P : r \mapsto (f_n(r))_{n\in\ZZ}$ in $\cE$ is a plot,
for the pushforward of the functional diffeology on $\Cinfty_\mathrm{per}(\RR,\CC)$,
if and only if:
\begin{enumerate}
    \item The functions $f_n : \dom(\P) \to \CC$ are smooth.
    \item For all closed ball $\oline{\cB}\subset \dom(\P)$, for every $k \in \NN$, for all $p \in \NN$, there exists
    a positive number $\M_{k,p}$ such that, for all integer $n\neq 0$,
    \begin{equation}\tag{$\diamondsuit$}
    \left| {\partial^k f_n(r) \over \partial r^k} \right| \leq {\M_{k,p} \over \modulus{n}^p} \quad \mbox{for all $r\in \cB$}.
    \end{equation}
\end{enumerate}

\alinea{Note~1.}~In other words, the parametrization $r \mapsto (f_n(r))_{n\in \ZZ}$
is a plot of this diffeology if the functions $f_n$ are smooth and their derivatives are uniformly rapidly decreasing,
what we denote also by
$$
n^p {\partial^k f_n(r) \over \partial r^k} \xrightarrow[\modulus{n} \to \infty]{} 0.
$$
\alinea{Note~2.}~By compactness, it is enough that, for every point $r_0\in\dom(\P)$,
there exists a ball $\cB'$ centered at $r_0$ such that ($\diamondsuit$)
holds to ensure that ($\diamondsuit$) holds on  every closed ball $\oline{\cB}\subset\dom(\P)$.
\end{article}

\begin{proof}
Let us check, first of all, that the condition $(\diamondsuit)$ defines a diffeology on the space $\cE$ of
rapidly decreasing sequences of complex numbers.

\alinea {Covering axiom}~--- If $f_n(r) = f_n$ is constant in $r$, for all $n$,
then the condition $(\diamondsuit)$ is trivially satisfied for $k>0$, and for $k=0$ it means that
the series is rapidly decreasing, what it is.

\alinea {Locality axiom}~--- According to Note~2, $(\diamondsuit)$ is a local condition.

\alinea {Smooth compatibility axiom}~--- Let $\P : (r \mapsto f_n(r))_{n\in\ZZ}$
satisfying $(\diamondsuit)$ and $\F : s \mapsto r$
be a smooth parametrization in the domain of $\P$. We have, for all $k>0$,
$$
{\partial^k f_n(s) \over \partial s^k} = \sum_{\ell = 1}^{k}
{\partial^\ell f_n(r) \over \partial r^\ell} \cdot
\Q_{k,\ell}\left({\partial r \over \partial s},\ldots,{\partial^k r \over \partial s^k}\right),
$$
where the $\Q_{k,\ell}$ are polynomials.
Now, since $s \mapsto r$ is smooth,
the partial derivatives are bounded on every ball,
and because the $\Q_{k,\ell}$ are polynomials in the partial derivatives,
as a function of $s$,
they are bounded operators on every ball.
Let then $s_0\in \dom(\F)$, $r_0=\F(s_0)$ and $\cB$ be a ball
centered at $r_0$ such that the condition $(\diamondsuit)$ is satisfied.
Let $\cB' \subset \F^{-1}(\cB)$ be a ball centered at $s_0$, we have for all $s \in \cB'$,
\begin{eqnarray*}
    \left|{\partial^k f_n(s) \over \partial s^k}\right| & \leq & \sum_{\ell = 1}^{k}
    \left| {\partial^\ell f_n(r) \over \partial r^\ell} \right|
    \left| \vphantom{{\partial^\ell f_n(s) \over \partial s^k}}  \Q_{k,\ell}\left({\partial r \over \partial s},\ldots,{\partial^k r \over \partial s^k}\right)\right|\\
    & \leq & \sum_{\ell = 1}^{k} {\M_{\ell,p} \over \modulus{n}^p} \, \mathop{\rm sup}_{s \in \cB'} \left| \Q_{k,\ell}\left({\partial r \over \partial s},\ldots,{\partial^k r \over \partial s^k}\right)\right|,
\end{eqnarray*}
where the $\M_{\ell,p}$ are the constants of the inequality $(\diamondsuit)$ for the ball $\cB$.
Let then
$$
m_{k,\ell} = \mathop{\rm sup}_{s \in \cB'} \left| \Q_{k,\ell}\left({\partial r \over \partial s},\ldots,{\partial^k r \over \partial s^k}\right)\right|
\qmbox{and} \M'_{k,p} = \sum_{\ell = 1}^{k}  m_{k,\ell} \, \M_{\ell,p} \, ,
$$
we get, for all $s\in \cB'$,
$$
\left|{\partial^k f_n(s) \over \partial s^k}\right|
\leq {\M'_{k,p} \over \modulus{n}^p}.
$$
Hence, thanks to Note~2, $\P \circ \F$ still satisfies the condition $(\diamondsuit)$.
Therefore, this condition defines a diffeology on the set $\cE$ of
rapidly decreasing sequences of complex numbers.

Let us show now that the diffeology defined on $\cE$ by ($\diamondsuit$) is the pushforward by $j$
of the functional diffeology of $\Cinfty_\mathrm{per}(\RR,\CC)$.

\alinea{A.---}~Let us prove, first of all, that $j$ is smooth, where $\Cinfty_\mathrm{per}(\RR,\CC)$ is
equipped with the functional diffeology and $\cE$ with the diffeology defined by ($\diamondsuit$).
Let $\P : r \to f_r$ be a plot of $\Cinfty_\mathrm{per}(\RR,\CC)$ defined on some real domain $\U$,
the composite $j \circ \P$ writes
$$
j\circ \P : r \mapsto (f_n(r))_{n\in\ZZ} \qmbox{with} f_n(r) = \int_0^1 f_r(x) e^{-2i\pi n x}\, dx.
$$
Clearly, since $(r,x) \mapsto f_r(x)$ is smooth, the $f_n : r \mapsto f_n(r)$ are smooth, and we have
$$
{\partial^k f_n(r) \over \partial r^k} = \int_0^1 {\partial^k f_r(x) \over \partial r^k} e^{-2i\pi n x}\,dx.
$$
For all nonzero integer $n$, after $p$ integrations by parts, we get
$$
{\partial^k f_n(r) \over \partial r^k} =  {1 \over (2i\pi n)^{p}} \int_0^1 {\partial^{p} \over \partial x^{p}}
\left( {\partial^k f_r(x) \over \partial r^k}\right)\, e^{-2i\pi n x}\,dx.
$$
Therefore, defining for every ball $\cB\subset\dom(\P)$ the number
$$
\M_{k,p} = {1 \over (2\pi)^p} \mathop{\rm sup}_{r \in \cB} \mathop{\rm sup}_{x \in [0,1]}
\left| {\partial^{p} \over \partial x^{p}} {\partial^k  \over \partial r^k}f_r(x) \right|,
$$
we have
$$
\left| {\partial^k f_n(r) \over \partial r^k} \right| \leq {\M_{k,p} \over \modulus{n}^p}\quad \mbox{for all $r\in \cB$.}
$$
Hence, $j \circ \P$ is a plot of $\cE$. Therefore $j$ is smooth.

\alinea{B.---}~Conversely, remember that $j$ is injective, then
let us show that $j^{-1}: \cE \to \Cinfty_\mathrm{per}(\RR,\CC)$ is smooth.
Let $\P: r \mapsto (f_n(r))_{n\in\ZZ}$ be a parametrization in $\cE$ satisfying the condition $(\diamondsuit)$, and let
$$
j^{-1}\circ \P : r \mapsto [x \mapsto f_r(x)].
$$
The parametrization $r \mapsto f_r$ is given by
$$
f_r(x) = \lim_{\N \to \infty} \S_\N(r,x) \qmbox{with} \S_\N(r,x) = \sum_{n=-\N}^{+\N} f_n(r)e^{2i\pi n x}.
$$
The $\S_\N$ are smooth for all $\N$, we want to show that the limit $(r,x) \mapsto f_r(x)$ is smooth too.
For all $k,p,\N \in \NN$, let us introduce the series of partial derivatives
$$
\S_\N^{k,p}(r,x) = {\partial^p \over \partial x^p}{\partial^k \over \partial r^k}\S_\N(r,x) =
\sum_{n=-\N}^{+\N} (2i\pi n)^p {\partial^k f_n(r) \over \partial r^k} e^{2i\pi n x}.
$$
The functions $\S_\N^{k,p}$ are smooth, with respect to the pair $(r,x)$, for all $\N,k,p$.
Let us denote by $\modulus{\S}_\N^{k,p}(r,x)$ the series of moduli of the terms of $\S_\N^{k,p}$
$$
\modulus{\S}_\N^{k,p}(r,x) = \sum_{n=-\N}^{+\N} \modulus{2\pi n}^p \left|{\partial^k f_n(r) \over \partial r^k}\right|.
$$
Then, let $\oline{\cB} \subset \dom(\P)$ be some closed ball.
Thanks to the hypothesis, applying (2) to $p+2$, we have, for all $r\in \cB$,
\begin{eqnarray*}
    \sum_{n=-\N}^{+\N} \modulus{2\pi n}^p \left| {\partial^k f_n(r) \over \partial r^k} \right| & \leq &
    c + 2\sum_{n=1}^\N (2\pi)^p {\M_{k,p+2} \over n^2},
\end{eqnarray*}
where $c$, corresponding to $n=0$, is some constant. Hence, for all $r\in\cB$, $x\in[0,1]$ and $\N\in\NN$
$$
\modulus{\S}_\N^{k,p}(r,x) \leq \K
$$
where $\K= c + 2(2\pi)^p \M_{k,p+2}(\pi^2/6)$. Next, since the series $\modulus{\S}_\N^{k,p}(r,x)$
is increasing and upper bounded, it
is convergent, and therefore the series $\S_\N^{k,p}(r,x)$ is also convergent. Moreover, according to what preceded:
the sequence of smooth functions $\S_\N^{k,p}$
converges uniformly on $\cB \times [0,1]$ to some function $\S_\infty^{k,p}$ when $\N \to \infty$.
Now, according to a classical result of differential calculus,
see for example (an obvious improvement of) \cite[Thm. 3.10]{Don00},
the map $(r,x) \mapsto f_r(x) = \lim_{\N\to\infty}\S_N(r,x)$ is smooth,
and the $\S_\infty^{k,p}$ are the partial derivatives
$$
\S_\infty^{k,p}(r,x) = {\partial^p \over \partial x^p}{\partial^k \over \partial r^k}f_r(x).
$$
Thus, the parametrization $r \mapsto [x \mapsto f_r(x)]$ is a plot of $\Cinfty_\mathrm{per}(\RR,\CC)$
such that $j(r\mapsto f_r) = (f_n(r))_{n\in\ZZ}$.
Therefore, $j$ is a diffeomorphism from $\Cinfty_\mathrm{per}(\RR,\CC)$, equipped with the functional diffeology, to
$\cE$, equipped with the diffeology defined by $(\diamondsuit)$. In other words, the diffeology defined on $\cE$ by $(\diamondsuit)$
is the pushforward of the functional diffeology on $\Cinfty_\mathrm{per}(\RR,\CC)$.
\end{proof}

%%%%%%%%%%%%%%%%%%%%%%%%%%%%%%%%%%%%%%%%%%%%%%%%%%%%%%%%%%
%%
%% MARK: The Infinite Torus
%%
%%%%%%%%%%%%%%%%%%%%%%%%%%%%%%%%%%%%%%%%%%%%%%%%%%%%%%%%%%
\section*{The Infinite Torus }

We denote by $\Torus^\infty$ the group of infinite sequences $\Z = (z_n)_{n\in \ZZ}$, with $z_n \in  \U(1)$, that is,
$$
\Torus^\infty = \prod_{n \in \ZZ} \U(1).
$$
We consider the natural $\CC$-linear action of $\Torus^\infty$ on $\cE$,
$$
(z_n)_{n \in \ZZ} \cdot (\Z_n)_{n \in \ZZ} = (z_n \Z_n)_{n \in \ZZ}.
$$
Indeed, multiplying by a number of modulus 1 transforms a rapidly decreasing sequence of complex numbers into another,
and every element $z=(z_n)_{n \in \ZZ} \in \Torus^\infty$ is invertible,
$$
(z_n)_{n \in \ZZ}^{-1} = (\bar z_n)_{n \in \ZZ},
$$
where $\bar z$ denotes the conjugate.
Moreover, for every plot $r \mapsto (\Z_n(r))_{n\in \ZZ}$ in $\cE$, for all $p \in \NN$,
$$
\left|{\partial^p z_n\Z_n(r) \over \partial r^p}\right|
= \left|{\partial^p \Z_n(r) \over \partial r^p}\right|.
$$
Obviously, the same holds for the inverse.
Hence, the action of $(z_n)_{n \in \ZZ}$ is smooth as well as its inverse, and then $(z_n)_{n \in \ZZ}$
acts on $\cE$ by diffeomorphism. Thus, we got a monomorphism
$$
\eta : \Torus^\infty \to \GL^\infty(\cE) = \GL(\cE) \cap \Diff(\cE).
$$

\begin{article}\textbf{The Tempered Diffeology on the Infinite Torus.}\
We shall say that a parametrization $\zeta : r \mapsto (z_n(r))_{n\in \ZZ}$ in $\Torus^\infty$ is {\em tempered}
if the $z_n$ are smooth and if for every $k \in \NN$,
for every $r_0$ in the domain of the parametrization,
there exist a closed ball $\oline{\cB}\subset\dom(\zeta)$ centered at $r_0$, a polynomial $\P_k$ and
an integer $\N$ such that
\begin{equation}
    \renewcommand{\theequation}{$\heartsuit$}
    \forall r \in \cB, \forall n > \N, \quad \left|{\partial^k z_n(r) \over \partial r^k}\right| \leq \P_k(n).
\end{equation}
The tempered parametrizations form a group diffeology on $\Torus^\infty$.
Equipped with the tempered diffeology, the action of the group $\Torus^\infty$
on $\cE$ is smooth. Moreover,
for all $\N\in \NN$, the injection $\iota_\N : \Torus^\N \to \Torus^\infty$ defined
as follows is an induction:
$$
\iota_\N(\zeta) = \Z \qmbox{with}  \left\{
\begin{array}{ll}
    \Z_n = \zeta_n & \mbox{if $n \in \{1,\ldots,\N\}$}, \\
    \Z_n = 1  & \mbox{otherwise}.
\end{array}
\right.
$$
\end{article}

\begin{proof}
Let us show first that the condition $(\heartsuit)$ defines a diffeology, actually a sub-diffeology of the
product diffeology on the infinite product of toruses $\T^\infty = \prod_{n\in \ZZ} \Torus$.

\alinea {Covering axiom}~--- If $\zeta = (z_n)_{n\in \ZZ}$ is constant,
then $\modulus{\partial^k z_n(r)/\partial r^k}$ is equal to $1$ for $k = 0$ and equal to $0$ for $k>0$.
Thus we chose $\P_0(n)=1$ and $\P_k(n)=0$ for $k\neq 0$, and the
condition $(\heartsuit)$ is satisfied.

\alinea {Locality axiom}~--- The condition $(\heartsuit)$ is local in the variable $r$, by construction.

\alinea {Smooth compatibility axiom}~--- Let $\zeta : (r \mapsto z_n(r))_{n\in \ZZ}$
satisfying $(\heartsuit)$ and $\F : s \mapsto r$
be a smooth parametrization in the domain of $\zeta$. We have, for all $k>0$,
$$
{\partial^k z_n(s) \over \partial s^k} = \sum_{\ell = 1}^{k}
{\partial^\ell z_n(r) \over \partial r^\ell} \cdot
\Q_{k,\ell}\left({\partial r \over \partial s},\ldots,{\partial^k r \over \partial s^k}\right),
$$
where the $\Q_{k,\ell}$ are polynomials. Therefore,
$$
\left|{\partial^k z_n(s) \over \partial s^k}\right| \leq \sum_{\ell = 1}^{k}
\left|{\partial^\ell z_n(r) \over \partial r^\ell}\right| \,
\left| \vphantom{{\partial^\ell z_n(r) \over \partial r^\ell}} \Q_{k,\ell}\left({\partial r \over \partial s},\ldots,{\partial^k r \over \partial s^k}\right)\right|.
$$
Since the function $s\mapsto r$ is smooth, the partial derivatives are bounded on every ball,
and these polynomials are locally absolutely bounded operators.
Let
$$
\M_{k,\ell} = \sup_{r \in \cB}
\left|\Q_{k,\ell}\left({\partial r \over \partial s},\ldots,{\partial^k r \over \partial s^k}\right)\right|.
$$
Next, let $\N_\ell$ satisfying ($\heartsuit$) for $k=\ell$, and $\N' = \sup_{\ell = 1\ldots k} \N_\ell$, then
for all $n>\N'$,
$$
\left|{\partial^k z_n(s) \over \partial s^k}\right| \leq \sum_{\ell = 1}^{k}
\M_{k,\ell}\left|{\partial^\ell z_n(r) \over \partial r^\ell}\right| \leq \sum_{\ell = 1}^{k}
\M_{k,\ell}\P_\ell(n).
$$
Hence, the partial derivatives of the $z_n$ with respect to $r$ are still upper bounded by polynomials,
and the axiom of compatibility is satisfied.

Therefore, the condition $(\heartsuit)$ defines a diffeology on the set $\Torus^\infty$.
Let us check now that $(\heartsuit)$ defines a group diffeology.
First of all, let $z : r \mapsto (z_n(r))_{n\in\ZZ}$ and $z' : r \mapsto (z'_n(r))_{n\in\ZZ}$ be two plots of $\Torus^\infty$,
the derivatives of the $n$-th term $r \mapsto z_n(r)z'_n(r)$, of the product $zz'$, writes
$$
{\partial^k z_n(r)z'_n(r) \over \partial r^k} = \sum_{\ell = 0}^{k}
{k \choose \ell}
{\partial^{k-\ell} z_n(r) \over \partial r^{k-\ell}} \cdot
{\partial^{\ell} z'_n(r) \over \partial r^{\ell}}.
$$
Hence,
\begin{eqnarray*}
    \left|{\partial^k z_n(r)z'_n(r) \over \partial r^k} \right|
    & \leq & \sum_{\ell = 0}^{k} {k \choose \ell} \left|{\partial^{k-\ell} z_n(r) \over \partial r^{k-\ell}}\right| \,
    \left|{\partial^{\ell} z'_n(r) \over \partial r^{\ell}}\right| \\
    & \leq & \sum_{\ell = 0}^{k} {k \choose \ell} \P_{k-\ell}(n) \, \P'_\ell(n).
\end{eqnarray*}
Thus, the partial derivatives of the product $z_nz'_n$ are still upper bounded by a polynomial.
For the inverse mapping $z^{-1} = (\bar z_n)_{n\in\ZZ}$, we have,
since $\bar z_n(r)z_n(r)=1$,
$$
{\partial^k z_n(r)\bar z_n(r) \over \partial r^k} = 0,
$$
that is,
$$
z_n {\partial^{k} \bar z_n(r) \over \partial r^{k}} +
\sum_{\ell = 0}^{k-1}
{k \choose \ell}
{\partial^{k-\ell} z_n(r) \over \partial r^{k-\ell}} \cdot
{\partial^{\ell} \bar z_n(r) \over \partial r^{\ell}}= 0.
$$
Therefore,
$$
{\partial^{k} \bar z_n(r) \over \partial r^{k}} = -{1 \over z_n(r)} \sum_{\ell = 0}^{k-1}
{k \choose \ell}
{\partial^{k-\ell} z_n(r) \over \partial r^{k-\ell}} \cdot
{\partial^{\ell} \bar z_n(r) \over \partial r^{\ell}}.
$$
Thus, if the partial derivatives $\partial^\ell \bar z_n(r)/\partial r^\ell$
are absolutely bounded by a polynomial for all $\ell<k$, then the $k$-th derivative
$\partial^k \bar z_n(r)/\partial r^k$ is also absolutely bounded by a polynomial.
And since $\modulus{\partial \bar z_n(r)/ \partial r} = \modulus{\partial z_n(r) / \partial r}$,
by recursion, $\partial^k \bar z_n(r)/\partial r^k$ is upper bounded by a polynomial.
Therefore, $\Torus^\infty$, equipped with the tempered diffeology is a diffeological group.

For the last sentence, the injection $\iota_\N : \Torus^\N \to \Torus^\infty$ is clearly smooth since,
for all plots $\zeta$ in $\Torus^\N$,
the derivatives of the components $\Z_n$ of the composite $\iota_\N \circ \zeta : r \mapsto (\Z_n(r))_{n\in\ZZ}$ vanish for $n$ outside $\{1,\ldots,\N\}$,
hence
$$
n^p\,{\partial^k \Z_n(r) \over \partial r^k} = 0 \qmbox{if $\modulus{n} > \N$.}
$$
Conversely if
$\zeta : r \mapsto (\Z_n(r))_{n\in\ZZ}$ is a plot of $\Torus^\infty$ with values in $\iota_\N(\Torus^\N)$,
then the components $r \mapsto \Z_n(r)$ are smooth,
by definition of the diffeology on $\Torus^\infty$,
and hence $\iota_\N^{-1}\circ \zeta$ is smooth.
Therefore, $\iota_\N$ is an induction.
\end{proof}

\begin{article}\textbf{Induced Solenoids}~---
A sequence $\alpha = (\alpha_n)_{n\in\ZZ}$ of positive numbers, is independent over $\QQ$,
if for every finitely supported sequence $(q_n)_{n\in\ZZ}$ of rational numbers $q_n \in \QQ$,
$$
\sum_{n\in\ZZ} q_n \alpha_n = 0 \quad \Rightarrow \quad q_n = 0 \qmbox{for all} n.
$$
In the following we will consider such sequences with $\modulus{\alpha_n} \leq 1$.
Then,
the map
$$
\iota : \RR \mapsto \Torus^\infty, \qmbox{defined by} \iota(t) = \bigg(e^{2i\pi\alpha_n t}\bigg)_{n\in \ZZ},
$$
is obviously injective,
but it is moreover an induction,
that is,
a diffeomorphism onto its image equipped with the subset diffeology.
We call the image $\iota(\RR) \subset \T^\infty$,
an \textit{irrational solenoid}.
\end{article}

\begin{proof}
Let us show first that such irrational sequences $\alpha = (\alpha_n)_{n\in\ZZ}$ exist.
Consider a transcendental number $\alpha$.
We define
$$
\alpha_0 = 1 \qmbox{and} \alpha_m = \alpha^m - \left[{\alpha^m}\right] \quad \mbox{for $m\neq0$},
$$
where the bracket denotes the integral part.
That defines $\alpha_n$ for all $n\in \ZZ$. Now, for a finitely supported sequence of rational numbers $q_n$,
we have
\begin{eqnarray*}
    \sum_{n\in \ZZ} q_n \alpha_n & = & q_0 + \sum_{n \in \ZZ \atop n\neq 0}
    q_n\left(\alpha^n -
    [\alpha^n]\right) \\
    &=& q_0 - \sum_{n \in \ZZ \atop n\neq 0} q_n [\alpha^n] + \sum_{n \in \ZZ \atop n\neq 0} q_n\alpha^n,
\end{eqnarray*}
with only a finite part of this sum is not zero. Then, if $\sum_{n\in\ZZ} q_n \alpha_n =0$ we multiply
both sides by $\alpha^\ell$,
with $\ell$ big enough to get an algebraic equation in $\alpha$,
with all powers positive.
But we assumed $\alpha$ transcendental,
then all the coefficients are $0$,
what implies $q_n= 0$, for all $n\in \ZZ$.

Next, we need to check first of all that $\iota$ is smooth.
The successive derivatives are simply
$$
{\partial^k e^{2i\pi\alpha_n t} \over \partial t^k} = (2i\pi\alpha_n)^k e^{2i\pi\alpha_n t}.
$$
Since $0<\alpha_n \leq 1$, for all $n \in \ZZ$, we have
$$
\left|{\partial^k e^{2i\pi\alpha_n t} \over \partial t^k}\right| \leq (2\pi)^k.
$$
Hence,
the derivatives are absolutely upper bounded by (constant) polynomials,
$\P_k(n) = (2\pi)^k$,
and therefore $\iota$ is a plot in $\Torus^\infty$.
Conversely,
consider a plot in $\Torus^\infty$
with values in $\iota(\RR)$,
the composite with the projection on the first two components gives a plot in $\Torus^2$ with values in the solenoid $(\exp(2i\pi t),\exp(2i\pi\alpha t))_{t\in \RR}$.
But the injection
$t \mapsto (\exp(2i\pi t),\exp(2i\pi\alpha t))$ being an induction \cite[Exercise 31]{PIZ13},
it follows that $\iota$ is also an induction.
\end{proof}

%%%%%%%%%%%%%%%%%%%%%%%%%%%%%%%%%%%%%%%%%%%%%%%%%%%%%%%%%%
%%
%% MARK: Symplectic Structure on the Space of Fourier Coefficients
%%
%%%%%%%%%%%%%%%%%%%%%%%%%%%%%%%%%%%%%%%%%%%%%%%%%%%%%%%%%%
\section*{Symplectic Structure on the Space of Fourier Coefficients}

We recall that a differential $k$-form $\varepsilon$ on the diffeological space $\cE$ is a map that associates,
with every plot $\P$ in $\cE$,
a $k$-form $\varepsilon(\P)$ on $\dom(\P)$ such that,
for all smooth parametrization $\F$ in $\dom(\P)$, one has
$$
\varepsilon(\P\circ \F) = \F^*(\varepsilon(\P)).
$$

\begin{article}\textbf{Symplectic Structure on $\Cinfty_\mathrm{per}(\RR,\CC)$}~---
Let $\Surf$ be the standard symplectic form on $\CC$, defined by
$$
\Surf_z(\delta z, \delta'z) = {1\over 2i}\left[\delta\bar z \, \delta'z - \delta'\bar z \, \delta z\right],
$$
where $z,\delta z,\delta'z \in \CC$.
For all $x\in \RR$, let
$$
\hat x : \Cinfty_\mathrm{per}(\RR,\CC) \to \CC \qmbox{with} \hat x(f) = f(x)
$$
be the {\em evaluation map}.
Because $\hat x$ is smooth, the pullback ${\hat x}^*(\Surf)/\pi$ is a $2$-form on $\Cinfty_\mathrm{per}(\RR,\CC)$,
and it is closed because $\Surf$ is closed.
We denote by $\omega$ its mean value
$$
\omega = {1\over \pi}\int_0^1 \hat x^*(\Surf) \, dx.
$$
Evaluated on a plot $\P: r \mapsto f_r$ in $\Cinfty_\mathrm{per}(\RR,\CC)$, $\omega$ is given by
$$
\omega(\P)_r(\delta r, \delta'r) = {1 \over 2i\pi}\int_0^1 \left\{ {\partial \oline{f_r(x)} \over \partial r}(\delta r)
{\partial f_r(x) \over \partial r}(\delta'r)\right.
- \left. {\partial \oline{f_r(x)} \over \partial r}(\delta'r)
{\partial f_r(x) \over \partial r}(\delta r)\right\} dx.
$$
Since integration is a smooth operation,
$\omega$ is still a differential closed $2$-form.
Now,
$\omega$ is invariant by translation $\Tr_g : f \mapsto f+g$,
and $\Cinfty_\mathrm{per}(\RR,\CC)$ is an homogeneous space of itself.
The moment map of this action is given,
up to a constant, by:
$$
\mu(f) = {1 \over 2i\pi} d \bigg[g \mapsto \int_0^1 \bar f g - \bar g f\bigg].
$$
An easy check shows that $\mu$ is injective.
Thus $\Cinfty_\mathrm{per}(\RR,\CC)$ equiped with $\omega$ is a {\em diffeological symplectic space},
in a strong sense.
Indeed, $\Cinfty_\mathrm{per}(\RR,\CC)$ acts transitively on itself by translation.
This action preserves $\omega$ and the moment map for this action is injective.
That falls under the definition we gave for being symplectic in diffeology,
as the case $(+\ +)$ of the Introduction,
and considering \cite[\textsection 9.19, Note~3]{PIZ13}.
By transfer,
that applies to the pushforward of $\omega$ on $\cE$.

\end{article}

\begin{proof}
The $2$-form $\omega$ is actually exact,
we choose the primitive $\varepsilon$ defined by
$$
\varepsilon(\P)_r(\delta r) = {1 \over 2i\pi} \int_0^1
\oline{f_r(x)} \, {\partial f_r(x) \over \partial r}(\delta r) \, dx.
$$
We use the notation of the Memoir on the Moment Maps in Diffeology \cite{PIZ10}.
Now,
since the space $\Cinfty_\mathrm{per}(\RR,\CC)$ is contractible,
we can compute the moment map through the paths moment map
applied, for every $f$, to the path $\gamma_f:t \mapsto tf$. That is,
$$
\mu(f)= \Psi(\gamma_f) = \hat\gamma_f^* (\CHK\omega),
$$
where $\CHK$ is Chain-Homotopy Operator. But $\omega = d\varepsilon$ and we recall that
$$
d\CHK + \CHK d = \1^* - \0^*.
$$
Therefore
\begin{eqnarray*}
    \mu(f) & = & \hat\gamma_f^* (\CHK (d \varepsilon))\\
    & = & \hat\gamma_f^* \left[\1^*(\varepsilon) - \0^*(\varepsilon) - d(\CHK\varepsilon)\right] \\
    & = & (\1\circ \hat\gamma_f)^*(\varepsilon) -(\0\circ \hat\gamma_f)^*(\varepsilon) -
    d \left[ \hat\gamma_f^* (\CHK\varepsilon) \right]
\end{eqnarray*}
But $\hat \gamma_f(g) = [t\mapsto T_g(tf)= tf + g]$, thus $\1\circ \hat\gamma_f(g)= f+g$,
that is,
$\1\circ \hat\gamma_f=T_f$.
Then,
\begin{eqnarray*}
    T_f^*(\varepsilon)(r\mapsto g_r)_r(\delta r) &= & \varepsilon(r \mapsto g_r + f)_r(\delta r) \\
    &=& {1\over 2i\pi}\int_0^1 \left(\oline{g_r(x)} + \oline{f(x)}\right){\partial g_r \over \partial r}(\delta r) \, dx \\
    &=& \varepsilon(r\mapsto g_r)_r(\delta r) \\
    &+& d\bigg[g \mapsto {1\over 2i\pi} \int_0^1 \oline{f(x)}g(x) \, dx\bigg](r \mapsto g_r)_r(\delta r),
\end{eqnarray*}
that is,
$$
(\1\circ \hat\gamma_f)^*(\varepsilon) = \varepsilon +  d\bigg[g \mapsto {1\over 2i\pi} \int_0^1 \oline{f(x)}g(x) \, dx\bigg].
$$
On the other hand, $\0\circ \hat\gamma_f(g)= g$, that is,
$\0\circ \hat\gamma_f=\id$ and $(\0\circ \hat\gamma_f)^*(\varepsilon) = \varepsilon$.
Next, since $\varepsilon$ is a $1$-form,
$\CHK\varepsilon$ is a function on $\Paths(\Cinfty_\mathrm{per}(\RR,\CC))$, precisely
$\CHK\varepsilon(\gamma)= \int_\gamma \varepsilon$.
Thus $\hat \gamma_f^*(\CHK\varepsilon)= (\CHK\varepsilon)\circ \hat \gamma_f$.
But $(\CHK\varepsilon)\circ \hat \gamma_f(g)= \CHK\varepsilon(t \mapsto tf+g) = \int_0^1\varepsilon(t\mapsto tf+g)_t(1) dt$,
that is,
\begin{eqnarray*}
    (\CHK\varepsilon)\circ \hat \gamma_f(g) &=& \int_0^1\varepsilon(t\mapsto tf+g)_t(1) \,dt \\
    &=& {1\over 2i\pi} \int_0^1 \left\{ \int_0^1 (t\oline{f(x)} +\oline{g(x)}) f(x)\, dx \right\} \,dt \\
    &=& {1\over 2i\pi}{1\over 2} \int_0^1 \modulus{f(x)}^2 dx + {1\over 2i\pi} \int_0^1 \oline{g(x)} f(x) \,dx
\end{eqnarray*}
Therefore,
$$
d \left[(\CHK\varepsilon)\circ \hat \gamma_f\right] = d\bigg[g\mapsto {1\over 2i\pi}\int_0^1 \oline{g(x)} f(x) \,dx\bigg].
$$
Combining everything into the expression of $\mu(f)$ we get what was announced.
\end{proof}

\begin{article}\textbf{Action of the Infinite Torus and Moment Maps}~---
We consider the $2$-form $\omega$ defined above pushed on the space of Fourier coefficients $\cE$.
Evaluated on a plot $\P:r\mapsto (f_n(r))_{n\in\ZZ}$,
$$
\omega(\P)_r(\delta r, \delta'r) =
{1 \over 2i\pi}\sum_{n\in\ZZ} {\partial \oline{f_n(r)} \over \partial r}(\delta r)
{\partial f_n(r) \over \partial r}(\delta'r) - {\partial \oline{f_n(r)} \over \partial r}(\delta'r)
{\partial f_n(r) \over \partial r}(\delta r).
$$
The ordinary action  $(z_n)_{n\in\ZZ}\cdot(\Z_n)_{n\in\ZZ} = (z_n\Z_n)_{n\in\ZZ}$,
of the infinite torus $\Torus^\infty$ on $\cE$,
is Hamiltonian and exact.%
\footnote{Being exact means having an equivariant solution to the equation in $\mu$, $\varphi(x,x')=\mu(x')-\mu(x)$ \cite[Chapter 9]{PIZ13}}
The Moment Map $\mu$ is given by
$$
\mu(\Z) = {1\over 2i\pi} \sum_{n\in\ZZ} \modulus{\Z_n}^2 \, \pi_n^*(\theta) + \sigma,
$$
where $\pi_n : \Torus^\infty \to \U(1)$ is the $n$-th projection $\pi_n(\Z) = \Z_n$,
$\theta$ is the canonical invariant $1$-form on $\U(1)$,
and $\sigma$ is some constant momentum of $\Torus^\infty$ (a constant invariant $1$-form).

Let $(\alpha_n)_{n\in\ZZ}$ be a sequence of numbers independent over $\QQ$.
We denote by $\iota : \RR \to \Torus^\infty$ the induction of $\RR$ in $\T^\infty$,
$\iota(t) = (e^{2i\pi \alpha_n t})_{n\in\ZZ}$,
and by $\underline{t}(\Z_n)_{n\in\ZZ} = (e^{2i\pi \alpha_n t} \Z_n)_{n\in\ZZ}$,
the induced action of $\RR$ on $\cE$.
Its moment map is given by
$$
\nu(\Z) = h(\Z) \, dt \qmbox{with} h(\Z) = \sum_{n\in\ZZ} \alpha_n \modulus{\Z_n}^2 + c,
$$
where $c$ is some constant.
\end{article}

\begin{proof}
The primitive $\varepsilon$ of $\omega$, that is,
$$
\varepsilon(\P)_r(\delta r)  = {1 \over 2i\pi}\sum_{n\in\ZZ} \oline{\Z_n(r)} \, {\partial \Z_n(r) \over \partial r}(\delta r),
$$
where $r \mapsto (\Z_n(r))_{n\in \ZZ}$ is a plot of $\cE$,
is obviously invariant under the action of $\Torus^\infty$.
Therefore, the action is Hamiltonian and equivariant \cite[\textsection 9.11]{PIZ13}.
And, the moment map is, up to a constant, the pullback of $\varepsilon$ by the orbit map
$\hat \Z: \T^\infty \to \cE$, $\hat \Z(z)=z\Z$, where $z=(z_n)_{n\in \ZZ}$ and $\Z=(\Z_n)_{n\in \ZZ}$.
Therefore, $\mu(\Z)= \hat \Z^*(\varepsilon) + \sigma$.
Now, let $\theta$ be the canonical $1$-form on $\U(1)$ defined by
$$
\theta(\zeta)_r(\delta r) = \bar\zeta(r){\partial \zeta(r) \over \partial r}(\delta r),
$$
for all plots $\zeta$ in $\U(1)$. Let $\pi_n : \Torus^\infty \to \U(1)$ be the $n$-th projection $\Z \mapsto \Z_n$, for all
$\Z= (\Z_n)_{n\in\ZZ} \in \cE$. Let $\zeta : r \mapsto(\zeta_n(r))_{n\in\ZZ}$ be a plot in $\Torus^\infty$,
the moment map $\mu$ is then explicitely given, up to the constant $\sigma$, by:
\begin{eqnarray*}
    \mu(\Z)(\zeta)_r(\delta r) & = & \hat\Z^*(\varepsilon)(\zeta)_r(\delta r)\\
    & = & \varepsilon(\hat \Z \circ \zeta)_r(\delta r) \\
    & = &  {1 \over 2i\pi}\sum_{n\in\ZZ} \oline{\Z_n \zeta_n(r)}\,{\partial \Z_n  \zeta_n(r) \over \partial r}(\delta r) \\
    & = &  {1 \over 2i\pi}\sum_{n\in\ZZ} \modulus{\Z_n}^2 \, \bar \zeta_n(r)\,{\partial \zeta_n(r) \over \partial r}(\delta r) \\
    & = &  {1 \over 2i\pi}\sum_{n\in\ZZ} \modulus{\Z_n}^2 \, \pi_n^*(\theta)(\zeta)_r(\delta r).
\end{eqnarray*}
Therefore, the general moment map is written:
$$
\mu(\Z) =  {1 \over 2i\pi}\sum_{n\in\ZZ} \modulus{\Z_n}^2 \, \pi_n^*(\theta) + \sigma.
$$
Remark that,
for all $n\in\ZZ$,
for all ball $\cB \subset \dom(\zeta)$,
there exists a smooth function $\tau_n$,
defined on $\cB$,
such that $\zeta_n(r) = \exp(2i\pi\, \tau_n(r))$.
Then,
modulo a constant,
the moment map is also given by
$$
\mu(\Z)(\zeta)_r(\delta r) = \sum_{n\in\ZZ} \modulus{\Z_n}^2 \, {\partial \tau_n(r) \over \partial r}(\delta r).
$$
Note also that, since $\zeta$ is a plot in $\Torus^\infty$ for the tempered diffeology,
the norm of the derivatives $\partial \zeta_n(r) / \partial r$, that is,
the norm of $\partial \tau_n(r) / \partial r$,
are upper bounded by a polynomial in $n$ what insures the convergence of the series defining the moment map just above.

Now, consider the induction $\iota$.
It induces a projection $\iota^* : \cT^{\infty*} \to \RR^*$,
where $\cT^{\infty*}$ is the space of momenta of $\Torus^\infty$.
The moment maps with respect to the group $\RR$ are then the composites $\nu = \iota^* \circ \mu$, that is,
$$
\nu = \iota^*\bigg\{  {1 \over 2i\pi}\sum_{n\in\ZZ} \modulus{\Z_n}^2 \, \pi_n^*(\theta) + \sigma \bigg\}
={1 \over 2i\pi}\sum_{n\in\ZZ} \modulus{\Z_n}^2 \, (\pi_n\circ\iota)^*(\theta) + \iota^*(\sigma).
$$
But $\pi_n\circ \iota : t \mapsto \exp(2i\pi \alpha_n t)$ then $(\pi_n\circ\iota)^*(\theta) = 2i\pi \alpha_n\, dt$.
Thus, $\nu(\Z) = h(\Z) \, dt$ with $h(\Z) = \sum_{n\in\ZZ} \alpha_n \modulus{\Z_n}^2 + c$,
where $\iota^*(\sigma) = c\, dt$,
$c \in \RR$.
\end{proof}

%%%%%%%%%%%%%%%%%%%%%%%%%%%%%%%%%%%%%%%%%%%%%%%%%%%%%%%%%%
%%
%% MARK: Quasiprojective action of the Real Line and Reduction
%%
%%%%%%%%%%%%%%%%%%%%%%%%%%%%%%%%%%%%%%%%%%%%%%%%%%%%%%%%%%
\section*{Quasiprojective Action of the Real Line and Reduction}

Let $\alpha=(\alpha_n)_{n\in\ZZ}$ be a sequence of positive numbers, independent over $\QQ$.
Let $\S^\infty_\alpha$ be the unit level%
\footnote{The choice of the constant $c$ in the moment map $\nu$ being irrelevant for the following,
we choose $c=0$.} of the moment map $\nu$ of the action of $\RR$,
$$
\S^\infty_\alpha = \bigg\{ \Z = (\Z_n)_{n\in\ZZ} \in \cE \ \bigg\vert\ \sum_{n\in\ZZ} \alpha_n \modulus{\Z_n}^2 = 1 \bigg\}.
$$
Let $\QPS$ be the quotient of $\S^\infty_\alpha$ under the action of $\RR$, and $\pr$ be the projection,
$$
\pr : \S^\infty_\alpha \to \QPS \qmbox{and} \QPS= \S^\infty_\alpha/\RR.
$$
We say that the quotient space is an \textit{Infinite Quasiprojective Space}.

\smallskip\noindent\textsc{Note.}\
Consider a general level $\sum_{n\in\ZZ} \alpha_n \modulus{\Z_n}^2 - c$,
with $c>0$.
By the change $\Z_n \mapsto \sqrt{\alpha_n}\Z_n/\sqrt{c}$,
the subspace $\S^\infty_\alpha$ is mapped into $\S^\infty \subset \cE$,
the unit sphere in $\cE$.
Now,
if all $\alpha_n$ would be equal to $1$,
then the action of $\RR$ would give the action of $\S^1$ and $\QPS$ would be diffeomorphic to the infinite projective space $\CC\PP^\infty=\S^\infty/\S^1$,
and the projection $\pr : \S^\infty_\alpha \to \QPS$ would be the infinite Hopf fibration%
\footnote{Same set but with another diffeology as in \cite[\textsection 4.11]{PIZ13}.}.
And that justifies our choice of vocabulary.

\begin{article}\textbf{The Orbits of the Quasiprojective Action of $\RR$}~---
The analysis of the action of $\RR$ on $\S^\infty_\alpha$ in terms of orbits is given by the following properties.
\begin{enumerate}

    \item Let $\Z \in \S^\infty_\alpha$.
    If there exist $\Z_n \neq 0$ and $\Z_m \neq 0$,
    then the stabilizer of $\Z$ is $\{0\}$ and the orbit of $\Z$ by $\RR$,
    equipped with the subset diffeology,
    is diffeomorphic to $\RR$.
    They are the {\em principal orbits}.

    \item The {\em singular orbits}, that is, the non principal orbits, are the
    subspaces
    $$
    \S^1_n = \{ \Z \in \S^\infty_\alpha \mid \Z_m = 0 \mbox{ if } m \neq n \}, \qmbox{with} n\in \ZZ.
    $$
    Each singular orbit, equipped with the subset diffeology, is diffeomorphic to the circle $\S^1$.

    \item The singular locus, that is, the union
    $
    \cS = \bigcup_{n\in\ZZ} \S^1_n \subset \S^\infty_\alpha,
    $
    equipped with the subset diffeology, is actually the diffeological sum%
    \footnote{See  \cite[1.39]{PIZ13}.}
    $$
    \cS = \coprod_{n\in\ZZ} \S^1_n, \qmbox{and} \dim(\cS) = 1.
    $$

    \item The regular locus, that is, the subset $\S^\infty_\alpha - \cS$, is D-open%
    \footnote{That is, open for the D-topology \cite[2.8]{PIZ13}.}.

\end{enumerate}

\end{article}

\begin{proof}
For the first item, let $\Z\in \S^\infty_\alpha$ with $\Z_n \neq 0$ and  $\Z_m \neq 0$, the map
$$
t \mapsto (e^{2i\pi \alpha_n t}z_n, e^{2i\pi \alpha_m t}z_m) \qmbox{with}
z_n={\Z_n \over \modulus{\Z_n}} \qmbox{and} z_m={\Z_m \over \modulus{\Z_m}}
$$
is an induction from $\RR$ into $\Torus^2$, because $\alpha_n$ and $\alpha_m$ are independent over $\QQ$,
see \cite[Exercise 31]{PIZ13}.
It follows from there that the orbit map $t \mapsto \underline{t}(\Z)$ is an induction.

For the second item, there exists $n$ such that for all $m \neq n$, $\Z_m = 0$
but $\Z_n\neq 0$, since $\sum_{n\in\ZZ} \alpha_n \modulus{\Z_n}^2>0$. The orbit map is a covering
onto the circle $\S^1_n$ induced in $\S^\infty_\alpha$.

For the third item, let $\P : \U \to \cS$ be a plot. For every $n \in \ZZ$ let
$\cO_n = (\pi_n\circ \P)^{-1}(\CC-\{0\})$, where $\pi_n : \S^\infty_\alpha \to \CC$ is the projection
$\pi_n((\Z_m)_{m\in \ZZ}) = \Z_n$.
Since $\pi_n \circ \P$ is smooth,
thus continuous, every $\cO_n\subset \U$ is open. Moreover, let $n\neq m$,
assume $r \in \cO_n\cap\cO_m$, that is, $\Z_n(r)\neq 0$ and $\Z_m(r)\neq 0$,
but $\P$ takes its values in the union of the $\S^1_m$,
$m \in \ZZ$,
hence $\Z_n(r)\neq 0$ implies $\Z_m(r) = 0$ for all $m\neq n$ thus $\cO_n\cap\cO_m = \varnothing$.
Therefore,
$$
\U = \bigcup_{n\in \ZZ} \cO_n \qmbox{and} \cO_n \cap \cO_m = \varnothing \mbox{ for all } n\neq m.
$$
That means that the $\cO_n$ make an open partition of $\U$.
Then, each $\cO_n$ is an union of connected components of $\U$.
Thus, $\P$ takes locally its values in the $\S^1_n$,
$n\in \ZZ$, that means that $\cS$ is the diffeological sum of the circles $\S^1_n$, $n\in \ZZ$ (\textit{op. cit.}).

For the fourth item, let $\P:\U \to \S^\infty_\alpha$ be a plot, $\P(r)=(\Z_n(r))_{n \in \ZZ}$.
For all $r_0 \in \P^{-1}(\S^\infty_\alpha - \cS)$ there exist
at least two different indices $n$ and $m$ such that $\Z_n(r_0)\neq 0$ and $\Z_m(r_0)\neq 0$. Since $\Z_n$ and
$\Z_m$ are smooth there exists an open neighborhood $\V$ of $r_0$ such that $\Z_n(r)\neq 0$ and $\Z_m(r)\neq 0$,
for all $r \in \V$, that is, $\V \subset \P^{-1}(\S^\infty_\alpha - \cS)$. Thus, $\P^{-1}(\S^\infty_\alpha - \cS)$
is a union of open domains, it is then an open domain, and consequently $\S^\infty_\alpha - \cS$ is D-open.
\end{proof}

\begin{article}\textbf{Symplectic Reduction on the Quasiprojective Space}~---
Independent of the various types of fibers of the projection $\pr : \S^\infty_\alpha \to \QPS$,
and the singularities induced on the quotient space $\QPS$,
there exists a differential closed $2$-form $\varpi$ on $\QPS$ such that
$$
\omega \restriction \S^\infty_\alpha = \pr^*(\varpi).
$$

\noindent\textsc{Note.}\ The group $\T^\infty$ being abelian, it still acts on the reduced space $\QPS$,
its moment on this space is the projection of the moment $\mu$ of its action on $\S^\infty_\alpha$
\cite[\textsection 9.13]{PIZ13}.
\end{article}

\begin{proof}
We shall apply the general criterion for a differential form to be the pullback of another one \cite[\textsection 6.38]{PIZ13}.
Let $\P : \U \to \S^\infty_\alpha$ and $\P' : \U \to \S^\infty_\alpha$ be two plots
$$
\begin{array}{ccc}
    \begin{array}{c}
        \begin{tikzpicture}[node distance=1.75cm, auto]
            
            \pgfmathsetmacro{\shift}{0.3ex}
            
            \node (A) {$\U$};
            \node (B) [right of=A] {$\S^\infty_\alpha$};
            \node (C) [below of=B] {$\QPS$};
            
            \draw[transform canvas={yshift=0.5ex},->] (A) -- (B) node[above,midway] {$\P$};
            \draw[transform canvas={yshift=-0.5ex},->](A) -- (B) node[below,midway] {$\P'$};
            \draw[->](B) to node {$\pr$}(C);
            
        \end{tikzpicture}
    \end{array}
    & \mbox{ such that } & \pr \circ \P = \pr \circ \P'.
\end{array}
$$

We want to check if, in these conditions, $\omega(\P) = \omega(\P')$.
That would insure the existence of $\varpi$,
a (necessarily closed) $2$-form on $\QPS$ such that $\omega = \pr^*(\varpi)$ \cite[\textsection 6.38]{PIZ13}.
We consider first of all what happens on the open subset
$$\U_0 = \P^{-1}(\S^\infty_\alpha - \cS).$$
Since $\pr \circ \P = \pr \circ \P'$,
$\P^{-1}(\S^\infty_\alpha - \cS) = \P'^{-1}(\S^\infty_\alpha - \cS) = \U_0$.
Now,
the restrictions of $\P$ and $\P'$ on $\U_0$ take their values in the subset of $\S^\infty_\alpha$ made of principal orbits of $\RR$,
for which the stabilizer of the action of $\RR$ is $\{0\}$.
Thus,
for each $r \in \U_0$ there is a unique $\tau(r) \in \RR$ such that,
for all $n$,
$\Z'_n(r) = e^{2i\pi \alpha_n \tau(r)}\Z_n(r)$.
The function $\tau$ is smooth.
Indeed,
for all $r_0 \in \U_0$,
there exists $n \in \ZZ$ such that $\Z_n(r_0)\neq 0$.
Then there exists a neighborhood of $r_0$ where $\Z_n(r)\neq 0$.
On this neighborhood $\Z'_n(r)\neq 0$,
and $e^{2i\pi \alpha_n \tau(r)} = \Z'_n(r)/\Z_n(r)$.
But $r \mapsto \Z'_n(r)$ and $r \mapsto \Z_n(r)$ are smooth,
thus $r \mapsto e^{2i\pi \alpha_n \tau(r)}$ is smooth,
and therefore so is $\tau$.

Now,
$\omega = d\varepsilon$, and
\begin{eqnarray*}
    \varepsilon(\P')_r(\delta r) &=& {1 \over 2i\pi} \sum_{n\in\ZZ} \oline{\Z_n'(r)}{\partial \Z'_n(r) \over \partial r}(\delta r) \\
    & = & {1 \over 2i\pi} \sum_{n\in\ZZ} \oline{\Z_n(r)}{\partial \Z_n(r) \over \partial r}(\delta r) \\
    & + & \bigg(\sum_{n\in\ZZ} \alpha_n \oline{\Z_n(r)}\, \Z_n(r)\bigg) {\partial \tau(r) \over \partial r}(\delta r) \\
    & = & \varepsilon(\P)_r(\delta r) +  \tau^*(dt)_r(\delta r).
\end{eqnarray*}
Therefore, $[\omega(\P') - \omega(\P)] \restriction \U_0 = 0$.
Thus, by continuity, $[\omega(\P') - \omega(\P)] \restriction \oline\U_0 = 0$, where $\oline\U_0$ is the closure of $\U_0$.
It remains to check what happens on the complementary subset $\V = \U-\oline\U_0$.
The subset $\V$ is open,
thus $\P\restriction \V$ and  $\P'\restriction \V$ are two plots of $\S^\infty_\alpha$ but with values in the subset of singular orbits $\cS$.
Since $\cS$ has dimension $1$ and $\omega$ is a $2$-form,
$\omega(\P\restriction \V) = \omega(\P'\restriction \V) = 0$, see \cite[\textsection 6.39 Note]{PIZ13},
In conclusion $\omega(\P') = \omega(\P)$ everywhere on $\U$.
That proves that there exists a $2$-form $\varpi$ on $\QPS=\S^\infty_\alpha/\RR$ such that $\pr^*(\varpi) = \omega$.
\end{proof}

%%%%%%%%%%%%%%%%%%%%%%%%%%%%%%%%%%%%%%%%%%%%%%%%%%%%%%%%%%
%%
%% MARK: Bibliography
%%
%%%%%%%%%%%%%%%%%%%%%%%%%%%%%%%%%%%%%%%%%%%%%%%%%%%%%%%%%%

\begin{thebibliography}{Don00}
    
    \bibitem[Don00]{Don00}
    {Paul Donato,}
    {\textit{Calcul diff\'erentiel pour la licence}.}
    {Dunod, Paris (2000).}
    
    \bibitem[PIZ10]{PIZ10}
    {Patrick Iglesias-Zemmour,}
    {\textit{Moment Maps in Diffeology}.}
    {Memoir of the AMS, vol. 192.}
    {Am. Math. Soc., Providence RI, (2010).}
    
    \bibitem[PIZ13]{PIZ13}
    {Patrick Iglesias-Zemmour,}
    {\textit{Diffeology}.}
    {Mathematical Surveys and Mo\-no\-graphs, vol. 185.}
    {Am. Math. Soc., Providence RI, (2013).}
    
    \bibitem[EP01]{EP01}
    {Elisa Prato,}
    {\textit{Simple Non--Rational Convex Polytopes via Symplectic Geometry}.}
    {Topology, \textbf{40} (2001), pp. 961--975.}
    
    \bibitem[Sou70]{Sou70}
    {Jean-Marie Souriau.}
    {\em Structure des syst\`emes dynamiques,}
    {Dunod, Paris 1970.}
    
    \bibitem[Vil68]{Vil68}
    {Naum Vilenkin,}
    {\em Special Functions and the Theory of Group Representations.}
    {Translations of Mathematical Mo\-no\-graphs.}
    {Am. Math. Soc., Providence RI (1968).}
    
\end{thebibliography}

\end{document}
