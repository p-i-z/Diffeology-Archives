\documentclass[10pt,letterpaper,reqno]{amsart}

%%%%%%%%%%%%%%%%%%%%%%%%%%%%%%%%%%%%%%%%%%%%%%%%%%%%%%%%%%
%%
%% MARK: MACROS File
%%
%%%%%%%%%%%%%%%%%%%%%%%%%%%%%%%%%%%%%%%%%%%%%%%%%%%%%%%%%%

\usepackage{amsmath}
\usepackage{amssymb}
\usepackage{amsthm}
\usepackage[hidelinks]{hyperref}
\usepackage{url}

% For diagrams
\usepackage{tikz-cd}
\usetikzlibrary{calc}
\tikzcdset{arrow style=tikz, diagrams={>={Straight Barb[scale=0.8]}}}

\usepackage[margin=10pt,font=small,labelfont=bf, labelformat=empty, labelsep=endash]{caption}

% Basic style
\parindent 0mm
\parskip .5ex plus 2pt

% Macros based on your 'Diffeology' book source conventions
\newcommand{\RR}{\mathbf{R}}
\newcommand{\Cinfty}{C^\infty}
\newcommand{\Dom}{\mathrm{Dom}}
\newcommand{\cU}{\mathcal{U}}
\newcommand{\cV}{\mathcal{V}}
\newcommand{\cW}{\mathcal{W}}
\newcommand{\cP}{\mathcal{P}}

\DeclareMathOperator{\D}{D} % Differential operator
\DeclareMathOperator{\SA}{SA}
\DeclareMathOperator{\surf}{Surf}
\DeclareMathOperator{\vol}{Vol}
\DeclareMathOperator{\Diff}{Diff}


% Theorem Environments
\newtheorem*{theorem}{Theorem} % Unnumbered theorem
\newtheorem*{definition}{Definition}
\newtheorem*{remark}{Remark}

% Font style
\usepackage{microtype}
\usepackage[cal=scr,uppercase = upright,greekfamily = didot,greeklowercase = upright,expert,utopia]{mathdesign}
\usepackage[supspaced=.04em]{superiors}
\linespread{1.1}

\allowdisplaybreaks

%%%%%%%%%%%%%%%%%%%%%%%%%%%%%%%%%%%%%%%%%%%%%%%%%%%%%%%%%%
%%
%% MARK: Title and front data
%%
%%%%%%%%%%%%%%%%%%%%%%%%%%%%%%%%%%%%%%%%%%%%%%%%%%%%%%%%%%

\title[Is the brain a 1-dimensional diffeological space?]{Is the brain a 1-dimensional diffeological space?\\
\small The Geometry of the Two-Thirds Power Law}

\author{Patrick Iglesias-Zemmour}

\thanks{I thank the Hebrew University of Jerusalem for its continuous academic support.}

\address{Einstein Institute of Mathematics, The Hebrew University of Jerusalem, Campus Givat Ram, $9190401$ Israel}
\email{piz@math.huji.ac.il}

\date{\today}

\subjclass[2020]{Primary 58A40; Secondary 53A15, 92C20}
\keywords{Diffeology, Wire Diffeology, Two-Thirds Power Law, Equi-affine Geometry, Motor Control}

%%%%%%%%%%%%%%%%%%%%%%%%%%%%%%%%%%%%%%%%%%%%%%%%%%%%%%%%%%
%%
%% MARK: Begin Document
%%
%%%%%%%%%%%%%%%%%%%%%%%%%%%%%%%%%%%%%%%%%%%%%%%%%%%%%%%%%%

\begin{document}
  
  %%%%%%%%%%%%%%%%%%%%%%%%%%%%%%%%%%%%%%%%%%%%%%%%%%%%%%%%%%%%%%%%%%%%%%%%%%%%%%%
  %%
  %% MARK: MACROS DraftWarning
  %%
  %%%%%%%%%%%%%%%%%%%%%%%%%%%%%%%%%%%%%%%%%%%%%%%%%%%%%%%%%%%%%%%%%%%%%%%%%%%%%%%
  
  %\tikz[overlay,remember picture]{\node at ($(current page.west)+(1.5,0)$) [rotate=90] {\Huge\textcolor{red}{Draft Version -- \pdfcreationdate}}}%
  
  %%%%%%%%%%%%%%%%%%%%%%%%%%%%%%%%%%%%%%%%%%%%%%%%%%%%%%%%%%%%%%%%%%%%%%%%%%%%%%%
  %%
  %% MARK: Abstract
  %%
  %%%%%%%%%%%%%%%%%%%%%%%%%%%%%%%%%%%%%%%%%%%%%%%%%%%%%%%%%%%%%%%%%%%%%%%%%%%%%%%
  
  \begin{abstract}
    Experimental neuroscience has long established that human motor control,
    specifically in hand movements,
    adheres to the ``Two-Thirds Power Law,''
    relating velocity to curvature ($v \propto \kappa^{-1/3}$).
    Geometrically,
    this implies that trajectories are geodesics of an equi-affine metric.
    However,
    in the standard framework of differential geometry,
    this metric is not intrinsic because its definition relies on acceleration,
    which is not a tensorial quantity under the full group of diffeomorphisms of $\RR^2$.
    In this note,
    we propose a change in perspective:
    we model the brain's internal geometry not as the standard plane $\RR^2$,
    but as the \textbf{Wire Plane}
    ---~$\RR^2$ equipped with the \emph{wire diffeology} generated by smooth curves.
    We prove that the equi-affine metric is founded upon a covariant 3-tensor which,
    in this specific diffeological context,
    becomes a natural and intrinsic object defined on the plots of the space.
    This mathematical result suggests a neuro-geometric hypothesis:
    the brain perceives and plans space through 1-dimensional paths rather than 2-dimensional charts.
  \end{abstract}
  
  %%%%%%%%%%%%%%%%%%%%%%%%%%%%%%%%%%%%%%%%%%%%%%%%%%%%%%%%%%%%%%%%%%%%%%%%%%%%%%%
  %%
  %% MARK: Make Title
  %%
  %%%%%%%%%%%%%%%%%%%%%%%%%%%%%%%%%%%%%%%%%%%%%%%%%%%%%%%%%%%%%%%%%%%%%%%%%%%%%%%
  
  \maketitle
  
  %%%%%%%%%%%%%%%%%%%%%%%%%%%%%%%%%%%%%%%%%%%%%%%%%%%%%%%%%%
  %%
  %% MARK: Introduction
  %%
  %%%%%%%%%%%%%%%%%%%%%%%%%%%%%%%%%%%%%%%%%%%%%%%%%%%%%%%%%%
  \section{Introduction}
  
  Since the seminal discovery by Lacquaniti, Terzuolo and Viviani \cite{LTV83},
  it has been recognized that voluntary human movements
  ---~such as drawing or writing~---
  are not executed with uniform velocity.
  Instead,
  there is a strong coupling between the tangential velocity $v(t)$ and the radius of curvature $R(t)$ of the trajectory,
  governed by the phenomenological \textbf{Two-Thirds Power Law}:
  \begin{equation}
    v(t) = C \cdot R(t)^{1/3} \quad \text{or} \quad v(t) = C \cdot \kappa(t)^{-1/3},
  \end{equation}
  where $C$ is a constant specific to the trajectory segment.
  This robust biological invariant implies that the brain does not plan movements using the standard Euclidean metric of the ambient space.
  Instead,
  the motion appears to minimize an energy functional associated with \emph{equi-affine geometry} \cite{PS97},
  and trajectories often converge toward parabolic primitives, which are the geodesics of this geometry \cite{FH07, Pol09}.
  
  From a differential geometric standpoint,
  this phenomenon introduces a subtle difficulty.
  The ``equi-affine metric'' required to formalize this law involves the second derivative of the path (acceleration).
  In the classical theory of smooth manifolds,
  acceleration is not a tensor due to its law of transformation under arbitrary coordinate changes.
  This means that on a standard manifold,
  the value of acceleration depends on the chosen coordinate system,
  preventing it from representing an intrinsic physical property.
  Consequently,
  the tensor defining this metric is not intrinsic to the standard smooth structure of $\RR^2$.
  To treat it as intrinsic,
  one is usually forced to restrict the symmetry group to the special affine group,
  effectively fixing a preferred chart.
  
  Is the brain constrained to a single coordinate chart, or is the smooth structure of the space itself different?
  
  In this note,
  we propose that the difficulty vanishes if we adjust the underlying domain of definition.
  We suggest that the brain's internal representation of space is not the standard manifold $\RR^2$,
  but rather the \textbf{Wire Plane},
  denoted by $\cW$.
  That is,
  the set $\RR^2$ equipped with the \emph{wire diffeology},
  generated solely by smooth curves ($1$-plots).%
  \footnote{The standard diffeology on $\RR^2$ (its standard smooth structure) is generated by all smooth plots defined on open sets of $\RR^n$, for all $n$.
  The wire diffeology is generated by the specific subset of plots that factor locally through a curve ($\RR \to \RR^2$).
  While both generating families are infinite, the latter is much more restricted.
  Note that the diffeological space is the set of points $\RR^2$ \emph{structured by} these plots;
  it is distinct from the infinite-dimensional functional space of the plots themselves.}
  This wire plane has dimension $1$,
  unlike the standard $\RR^2$ which has dimension $2$.
  Recall that the dimension in diffeology is defined as a minimax on the plots of the generating families \cite[\textsection 1.78]{PIZ13}.
  In our case,
  since the diffeology is generated by $1$-plots and is not discrete,
  the dimension is $1$,
  not $2$,
  not infinity.
  This is the minimal dimension of the \emph{probes} needed to explore the space.
  For the ordinary smooth structure of the plane,
  which probes the space with 2-dimensional plots,
  the dimension is 2.
  
  We show that the tensor defining the equi-affine arc length
  ---~which fails to be well-defined for the standard manifold structure due to the involvement of second derivatives~---
  descends to a well-defined,
  intrinsic covariant 3-tensor on the Wire Plane.
  Physically,
  this occurs because the specific antisymmetric nature of the area form acts as a filter:
  it annihilates the non-tensorial components of the coordinate transformation (which are collinear to velocity) when restricted to 1-dimensional generating families.
  
  This result leads us to the titular question: \emph{Is the brain a 1-dimensional diffeological space?}
  If the motor system operates on intrinsic geometric (tensorial) principles,
  then the Two-Thirds Power Law implies that its internal smooth structure \emph{is} 1-dimensional:
  it apprehends the external world (the plane) and execute its tasks (movements) through the wire structure;
  the motor cortex functioning effectively as a virtual 1-dimensional machine.
  
  \textbf{Note.} The mathematical core of this result formalizes and expands upon an example initially presented as Exercise 97 in \cite{PIZ13}.
  
  %%%%%%%%%%%%%%%%%%%%%%%%%%%%%%%%%%%%%%%%%%%%%%%%%%%%%%%%%%
  %%
  %% MARK: The Wire Plane and the Equi-Affine Tensor
  %%
  %%%%%%%%%%%%%%%%%%%%%%%%%%%%%%%%%%%%%%%%%%%%%%%%%%%%%%%%%%
  \section{The Wire Plane and the Equi-Affine Tensor}
  \label{sec:wire-plane}
  
  In standard differential geometry,
  tensor fields are defined by their transformation rules under the full (pseudo-)group of local diffeomorphisms.
  The \emph{Two-Thirds Power Law} suggests the existence of a geometric structure derived from the volume form $\surf$ on $\RR^2$ (the determinant) and the acceleration of curves.
  However,
  acceleration is not a tensor under arbitrary smooth coordinate changes of $\RR^2$,
  which prevents the associated metric from being intrinsic to the standard manifold structure.
  
  To resolve this,
  we change the diffeological structure of the underlying space.
  We consider the \textbf{wire plane} $\cW$,
  that is,
  $\RR^2$ equipped with the \emph{wire diffeology},
  generated by the set of all smooth curves (or $1$-plots) $\Cinfty(\RR,\RR^2)$ \cite[\textsection 1.10]{PIZ13}.
  
\begin{figure}[htbp]
  \centering
  \includegraphics[width=0.75\textwidth]{The-wire.png} % Replace with your filename
  \caption{\textbf{The Mode of Access: A Visualization of the Wire Structure.}
  An artistic representation of the \textbf{Wire Diffeology} on the plane.
  It is crucial to distinguish the set from the structure: the space comprises the entire continuum of points in the plane (the background),
  but its geometry is defined strictly by how these points are \textbf{accessed}.
  The drawing illustrates the generating family of plots: a dense arborescence of smooth curves where branches peel away tangentially.
  While you still need two numbers to mark a point on the plane,
  the ``legal'' motions are constrained to these 1-dimensional channels.
  For the motor cortex,
  this implies that while any point in the plane can be reached (topological density),
  it can only be approached through a specific 1-dimensional trajectory (diffeological dimension 1),
  giving rise to the intrinsic equi-affine geometry. \hfill{(Image AI generated)}}
  \label{fig:wire-structure}
\end{figure}

  In this context,
  local plots are no longer charts valued in open sets of $\RR^2$,
  but factor locally through curves.
  Specifically,
  for any plot $\cP : \cU \to \RR^2$ in the wire diffeology (where $\cU$ is an open domain of some $\RR^n$),
  and for any $r_0 \in \cU$,
  there exists an open neighborhood $\cV$ of $r_0$,
  a smooth parametrization $q \in \Cinfty(\cV,\RR)$,
  and a smooth curve $\gamma \in \Cinfty(\RR, \RR^2)$,
  such that:
  \begin{equation}
    \cP \restriction \cV = \gamma \circ q.
  \end{equation}
  
  We now define the candidate tensor field $\alpha$.
  Let $\gamma$ be a $1$-plot of $\RR^2$,
  let $\dot \gamma$ and $\ddot \gamma$ be the first and second derivatives of $\gamma$,
  and let $\surf$ be the canonical volume form on $\RR^2$,
  defined by $\surf(\mathbf{u}, \mathbf{v}) = \det[\mathbf{u} \ \mathbf{v}]$.
  
  We define the \textbf{Equi-affine Tensor} $\alpha(\gamma)$ on $\Dom(\gamma)$ by:
  \begin{equation}
    \alpha(\gamma)_t(\delta t, \delta't, \delta''t) =
    \surf\big(\dot\gamma(t)(\delta t), \, \ddot\gamma(t)(\delta't)(\delta''t)\big),
  \end{equation}
  where $\delta t, \delta't, \delta''t \in T_t(\RR) \simeq \RR$.
  
  \begin{theorem}
    The field $\alpha$ defines,
    through the generating family of $1$-plots,
    a unique, well-defined differential covariant 3-tensor on $\cW$.
  \end{theorem}
  
  \begin{figure}[htbp]
    \centering
    \begin{tikzcd}[row sep=large, column sep=large]
      & \RR \arrow[dr, "\gamma"] & \\
      \cU \arrow[ur, "q"] \arrow[dr, "q'"'] \arrow[rr, "\cP" description] & & \cW \\
      & \RR \arrow[ur, "\gamma'"'] &
    \end{tikzcd}
    \caption{\textbf{Diagram: Independence of Parametrization.}
    The diagram illustrates that the measurement of the tensor does not depend on which curve ($\gamma$ or $\gamma'$) is used to represent the plot.
    This compatibility condition is a requirement for the geometry to be intrinsic.}
    \label{fig:descent}
  \end{figure}
  
  \begin{proof}
    First,
    note that $\alpha(\gamma)$ is trilinear and smooth with respect to its arguments.
    Thus,
    $\alpha(\gamma)$ is a smooth covariant $3$-tensor on $\Dom(\gamma)$.
    To prove it descends to the wire diffeology,
    we must apply the descent criterion for differential tensors defined on generating families \cite[\textsection 6.41]{PIZ13}.
    
    We must verify that for any two local decompositions of a plot $\cP$,
    say $\cP = \gamma \circ q$ and $\cP = \gamma' \circ q'$ (as illustrated in Figure \ref{fig:descent}),
    the pullback tensors satisfy the compatibility condition:
    $$
      q^*(\alpha(\gamma)) = q'^*(\alpha(\gamma')).
      $$
    Let $r \in \cU$ be a point in the domain of the plot,
    and let $\delta r, \delta'r, \delta''r \in \RR^n$ be tangent vectors.
    The pullback is given by:
    $$
      q^*(\alpha(\gamma))_r(\delta r,\delta'r,\delta''r) =
      \alpha(\gamma)_{t}(\delta t,\delta't,\delta''t),
      $$
    with $t = q(r)$ and $\delta t = \D(q)(r)(\delta r)$,
    and \emph{mutatis mutandis} for $\delta't$ and $\delta''t$.
    
    Since $\gamma \circ q = \gamma' \circ q'$,
    differentiating once yields the collinearity of the tangent vectors:
    \begin{equation}
      \label{eq:diamond}
      \dot\gamma(t) \delta t = \dot\gamma'(t') \delta t'.
    \end{equation}
    Differentiating a second time,
    we apply the chain rule for second derivatives:
    $$
      \D^2(\gamma \circ q)(r)(\delta'r)(\delta''r) =
      \ddot\gamma(t)(\delta't)(\delta''t) + \dot\gamma(t) \D^2(q)(r)(\delta'r)(\delta''r).
      $$
    Equating this with the expansion for $\gamma' \circ q'$,
    we have:
    \begin{multline*}
      \ddot\gamma(t)(\delta't)(\delta''t) + \dot\gamma(t) \D^2(q)(r)(\delta'r)(\delta''r) \\
      =
      \ddot\gamma'(t') (\delta't')(\delta''t') + \dot\gamma'(t') \D^2(q')(r)(\delta'r)(\delta''r).
    \end{multline*}
    We now substitute these expressions into the definition of $\alpha$.
    We calculate the value of the tensor for the primed decomposition:
    \begin{align*}
      \surf\big(\dot\gamma'(t')\delta t', \, \ddot\gamma'(t')(\delta't')(\delta''t')\big)
      & =
      \surf\bigg(
      \dot\gamma(t)\delta t, \
      \ddot\gamma(t)(\delta't)(\delta''t) \\
      &  + \dot\gamma(t) \D^2(q)(r)(\delta'r)(\delta''r) \\
      &  - \dot\gamma'(t') \D^2(q')(r)(\delta'r)(\delta''r)
      \bigg).
    \end{align*}
    By linearity of $\surf$ in the second argument,
    this splits into the main term plus terms involving the second derivatives of the parametrizations.
    Specifically, the extra terms are of the form:
    $$
      \surf\big(\dot\gamma(t)\delta t, \, \dot\gamma(t) \cdot \D^2(q)(r)(\delta'r)(\delta''r) \big)
      $$
    and
    $$
      \surf\big(\dot\gamma(t)\delta t, \, -\dot\gamma'(t') \cdot \D^2(q')(r)(\delta'r)(\delta''r) \big).
      $$
    However,
    because of condition (\ref{eq:diamond}),
    the vectors $\dot\gamma(t)$ and $\dot\gamma'(t')$ are collinear.
    Since the volume form $\surf$ is alternating,
    $\surf(\mathbf{u}, \mathbf{v}) = 0$ whenever $\mathbf{u}$ and $\mathbf{v}$ are collinear.
    
    Therefore,
    these extra terms vanish identically.
    We are left with:
    $$
      \surf\big(\dot\gamma(t)(\delta t), \, \ddot\gamma(t)(\delta't)(\delta''t)\big) =
      \surf\big(\dot\gamma'(t')(\delta t'), \, \ddot\gamma'(t')(\delta't')(\delta''t')\big).
      $$
    This proves that $q^*(\alpha(\gamma)) = q'^*(\alpha(\gamma'))$.
    Consequently,
    $\alpha$ descends to a unique differential covariant 3-tensor on the wire plane.
  \end{proof}
  
  The quantity $\sqrt[3]{|\alpha|}$ is known in the literature as the \emph{equi-affine arc length element}.
  Our theorem justifies this historical name by proving that its generator,
  the \textbf{Equi-affine Tensor} $\alpha$,
  is an intrinsic object of the wire plane's general smooth structure.
  
  Furthermore,
  this tensor has a profound dynamic interpretation.
  In the context of mechanics,
  the tensor $\alpha$ is zero for an affine inertial motion (a straight line) and becomes non-zero when a "turning force" is applied.
  It can thus be seen as a measure of the dynamics of deviation from inertia.%
  \footnote{This perspective,
  which frames dynamics as the measure of deviations from the inertial motions inside a mechanics,
  is developed in detail in the author's forthcoming book, \emph{The Geometry of Motion}.}
  
  From a diffeological perspective, $\sqrt[3]{|\alpha|}$ constitutes a \textbf{1-density} on the wire plane.
  Indeed, under a reparametrization $t = \phi(u)$,
  the tensor $\alpha$ scales by $(\phi')^3$,
  and consequently $\sqrt[3]{|\alpha|}$ scales by $|\phi'|$,
  which is the characteristic transformation rule for a density.
  Thus,
  while this object is ``exotic'' in the standard smooth world, it is a \emph{native} measurable quantity in the wire world.
  
  \textbf{Remark.} (Generalization to higher dimensions)
  The construction presented here admits a natural generalization to any dimension $n \ge 2$.
  Consider the wire space $\cW_{\!n} = \RR^n$ equipped with the wire diffeology.
  Let $\vol_n$ be the standard volume form on $\RR^n$.
  We can define a tensor $\alpha_n$ on any curve $\gamma$ by:
  $$
    \alpha_n(\gamma)_t = \vol_n\big(\dot{\gamma}(t), \ddot{\gamma}(t), \dots, \gamma^{(n)}(t)\big) \cdot (dt)^{N},
    $$
  where $N = n(n+1)/2$.
  Following the same logic as the Theorem, the terms involving lower-order derivatives in the chain rule expansion vanish due to the alternating property of $\vol_n$.
  Thus, $\alpha_n$ descends to a well-defined intrinsic covariant tensor of degree $N$ on $\cW_{\!n}$.
  
  This defines the intrinsic geometry of the \emph{Special Affine Group} $\SA(n, \RR)$ on the wire space.
  For $n=3$, the invariant involves the third derivative (the ``jerk'').
  This provides a geometric context for the \emph{Minimum Jerk} models \cite{FH85}.
  While these models were originally formulated for planar movements,
  the presence of the third derivative as a fundamental intrinsic observable in the 3-dimensional wire structure suggests a deep link between the minimization of jerk and the intrinsic geometry of $\cW_3$.
  For the classical theory of these invariants,
  see Blaschke \cite{Bla23}.
  
  %%%%%%%%%%%%%%%%%%%%%%%%%%%%%%%%%%%%%%%%%%%%%%%%%%%%%%%%%%
  %%
  %% MARK: Discussion and Conclusion
  %%
  %%%%%%%%%%%%%%%%%%%%%%%%%%%%%%%%%%%%%%%%%%%%%%%%%%%%%%%%%%
  \section{Discussion and Conclusion}
  
  Our approach in this note has been that of a geometer recognizing a familiar structure in a foreign land.
  The mathematical result established in the Theorem offers a resolution to a long-standing paradox,
  and its implications are best understood through the lens of mathematical parsimony, or Occam's Razor.
  
  The standard approach to the non-tensorial nature of acceleration is to \emph{add} a structure:
  an affine connection.
  The diffeological approach is simpler: it \emph{removes} an unnecessary assumption
  ---~the Manifold Hypothesis,
  which dictates that a space must be probed by 2-dimensional patches.
  By restricting the probes to 1-dimensional curves,
  the obstruction vanishes without adding any new entities.
  The problem was an artifact of an overly rigid framework.
  
  We remark that the special affine group, $\SA(2,\RR)$, remains the group of automorphisms of the geometric structure $(\cW,\alpha)$.%
  \footnote{Actually, the group of diffeomorphisms of the wire plane is,
  as a set,
  that of the standard $\RR^2$: $\Diff(\cW) = \Diff(\RR^2)$.
  Only its internal structure (the functional diffeology) differs.}
  The affine symmetry thus emerges as a property of the intrinsic tensor $\alpha$ within the full smooth structure of the wire plane,
  which no longer requires an \emph{a priori} restriction to a rigid affine geometry to be treated as a tensor.
  
  We acknowledge that the pure Two-Thirds Power Law is an idealization.
  The biological reality is more complex,
  with a limited domain of validity (the law breaks at inflection points) and variations in the exponent suggesting a mixture of geometries \cite{BFBF09}.
  Our aim here is not to provide a comprehensive biological model for these complexities.
  Rather,
  it is to point out that the idealized law,
  in its purest form,
  has a natural and intrinsic home in the Wire Plane.
  Indeed,
  the Wire Diffeology is the unique framework that can host both the Euclidean and Affine metrics intrinsically,
  making it the necessary stage for any "mixed geometry" model.
  
  This brings us back to the provocative question posed in the title.
  The claim that the brain is "1-dimensional" is,
  of course,
  a mathematical farce when taken anatomically.
  However,
  it is a functional truth if we accept that the geometry of a system is defined by its natural probes.
  This is possible because diffeology decouples the number of coordinates needed to mark a point from the dimension of the smooth structure.
  The Wire Plane is still the set $\RR^2$,
  but its \emph{diffeological dimension} is 1.
  This flexibility seems perfectly suited to a system that operates in a 2D world but whose communication channels (axons) are 1D.
  
  In conclusion,
  the Two-Thirds Power Law finds its most natural geometric expression as an intrinsic property of the Wire Diffeology.
  Its associated invariant is best understood not as an "equi-affine" object,
  but as the native \textbf{Dynamic Arc Length} of the wire plane.
  This single,
  concrete example suggests that the flexibility of diffeology may offer a valuable new tool for modeling the functional geometries of the brain.
  
  %%%%%%%%%%%%%%%%%%%%%%%%%%%%%%%%%%%%%%%%%%%%%%%%%%%%%%%%%%
  %%
  %% MARK: Acknowledgements
  %%
  %%%%%%%%%%%%%%%%%%%%%%%%%%%%%%%%%%%%%%%%%%%%%%%%%%%%%%%%%%
  
  \section*{Acknowledgements}
  I thank the Hebrew University of Jerusalem for its continuous academic support.
  I am deeply indebted to Daniel Bennequin and Tamar Flash.
  It is through them that I became acquainted with this subject, and our discussions were the starting point of my reflection on the equi-affine arc length.
  I am also grateful to the AI assistant Gemini (Google) for its assistance in the linguistic revision and formatting of this note.
  Data sharing is not applicable to this article as no new data were created or analyzed in this study.
  
  %%%%%%%%%%%%%%%%%%%%%%%%%%%%%%%%%%%%%%%%%%%%%%%%%%%%%%%%%%
  %%
  %% MARK: Bibliography
  %%
  %%%%%%%%%%%%%%%%%%%%%%%%%%%%%%%%%%%%%%%%%%%%%%%%%%%%%%%%%%
  \begin{thebibliography}{99}
    
    \bibitem{Bla23}
    \textbf{W. Blaschke},
    \textit{Vorlesungen über Differentialgeometrie II: Affine Differentialgeometrie},
    Springer, Berlin, 1923.
    
    \bibitem{Bom67}
    \textbf{J. Boman},
    \textit{Differentiability of a function and of its compositions with functions of one variable}.
    {Mathematica Scandinavica},
    \textbf{20} (1967),
    p.~{249--268}.
    
    \bibitem{LTV83}
    \textbf{F. Lacquaniti, C. Terzuolo and P. Viviani},
    \textit{The law relating the kinematic and figural aspects of drawing movements},
    Acta Psychologica,
    \textbf{54}(1-3) (1983), p.~115--130.
    
    \bibitem{FH85}
    \textbf{T. Flash and N. Hogan},
    \textit{The coordination of arm movements: an experimentally confirmed mathematical model},
    Journal of Neuroscience,
    \textbf{5}(7) (1985), p.~1688--1703.
    
    \bibitem{PS97}
    \textbf{F.E. Pollick and G. Sapiro},
    \textit{Constant affine velocity predicts the 1/3 power law of planar motion perception and generation},
    Vision Research,
    \textbf{37}(3) (1997), p.~347--353.
    
    \bibitem{FH07}
    \textbf{T. Flash and A. Handzel},
    \textit{Affine differential geometry analysis of human arm movements},
    Biological Cybernetics,
    \textbf{96} (2007), p.~577--601.
    
    \bibitem{Pol09}
    \textbf{F. Polyakov, E. Stark, R. Drori, M. Abeles and T. Flash},
    \textit{Parabolic movement primitives and cortical states: merging optimality with geometric invariance},
    Biological Cybernetics,
    \textbf{100}(2) (2009), p.~159--184.```
    
    \bibitem{BFBF09}
    \textbf{D. Bennequin, R. Fuchs, A. Berthoz and T. Flash},
    \textit{Movement timing and invariance arise from several geometries},
    PLoS Computational Biology,
    \textbf{5}(7) (2009), e1000426.
    
    \bibitem{PIZ13}
    \textbf{P. Iglesias-Zemmour},
    \textit{Diffeology},
    Mathematical Surveys and Monographs,
    \textbf{185},
    AMS, Providence, RI, 2013.
    
  \end{thebibliography}
  
\end{document}
