%%%%%%%%%%%%%%%%%%%%%%%%%%%%%%%%%%%%%%%%%%%%%%%%%%%%%%%%%%
%%
%%  Differential Forms On Manifolds With Boundary and Corners
%%  Serap Gürer & Patrick Iglesias-Zemmour
%%  Archived: 2025
%%
%%%%%%%%%%%%%%%%%%%%%%%%%%%%%%%%%%%%%%%%%%%%%%%%%%%%%%%%%%

\documentclass[11pt,reqno,letterpaper,twoside]{amsart}

%--------------------------------------------------------------------------
% Packages
%--------------------------------------------------------------------------
\usepackage[utf8]{inputenc}
\usepackage[T1]{fontenc}
\usepackage{amssymb}
\usepackage{amscd}
\usepackage{tikz-cd}
\usetikzlibrary{calc}
\usepackage[hidelinks]{hyperref}

%--------------------------------------------------------------------------
% Page Layout & Typography
%--------------------------------------------------------------------------
\parindent 0mm
\parskip .5ex plus 2pt
\linespread{1.1}

%--------------------------------------------------------------------------
% Theorem Styles
%--------------------------------------------------------------------------
\newtheoremstyle{article}% ⟨name⟩
{7pt}%  ⟨Space above⟩
{7pt}%  ⟨Space below⟩
{}%     ⟨Body font⟩
{0pt}%  ⟨Indent amount⟩
{\bf}%  ⟨Theorem head font⟩
{.\ }%  ⟨Punctuation after theorem head⟩
{0pt}%  ⟨Space after theorem head⟩
{}%     ⟨Theorem head spec⟩

\theoremstyle{article}
\newtheorem{article}{}

\def\artlabel[#1]{\textbf{\textsc{#1}}}
\newcommand{\art}[1]{(\textsection\ref{#1})}
\renewenvironment{proof}{\noindent \textit{Proof.}} {\nolinebreak\hfill $\square$}

%--------------------------------------------------------------------------
% Macros
%--------------------------------------------------------------------------

% Sets
\newcommand{\RR}{\mathbf{R}}

% Calligraphic
\newcommand{\cA}{{\mathcal A}}
\newcommand{\cD}{{\mathcal D}}
\newcommand{\cG}{{\mathcal G}}
\newcommand{\cO}{{\mathcal O}}
\newcommand{\cQ}{{\mathcal Q}}
\newcommand{\cU}{{\mathcal U}}

% Roman Letters (Used in text)
\def \A{\mathrm A}
\def \D{\mathrm D}
\def \F{\mathrm F}
\def \G{\mathrm G}
\def \H{\mathrm H} % Half-space
\def \J{\mathrm J}
\def \K{\mathrm K} % Corner
\def \M{\mathrm M}
\def \N{\mathrm N}
\def \O{\mathrm O}
\def \P{\mathrm P}
\def \Q{\mathrm Q}
\def \S{\mathrm S}
\def \T{\mathrm T}
\def \U{\mathrm U}
\def \V{\mathrm V}
\def \X{\mathrm X}
\def \Y{\mathrm Y}
\def \Z{\mathrm Z}

% Operators
\DeclareMathOperator{\Cinfty}{\mathrm{C}^\infty}
\DeclareMathOperator{\Diff}{Diff}
\DeclareMathOperator{\dom}{dom}
\DeclareMathOperator{\pr}{pr}
\DeclareMathOperator{\SO}{SO}
\DeclareMathOperator{\sq}{sq}
\DeclareMathOperator{\Strata}{Strat}
\DeclareMathOperator{\val}{val}
\DeclareMathOperator{\rk}{rank}

% Symbols
\newcommand{\id}{\mathbf 1}
\newcommand{\loc}{\mathrm{loc}}
\newcommand{\norm}[1]{||#1||}
\newcommand{\supp}{\mathrm{supp}}
\newcommand{\vect}[1]{\begin{pmatrix}#1\end{pmatrix}}
\let \epsilon=\varepsilon

%--------------------------------------------------------------------------
% Title Data
%--------------------------------------------------------------------------

\begin{document}

\title[Differential Forms On Manifolds With Boundary and Corners]{Differential Forms On Manifolds \\ With Boundary and Corners}

\author{Serap Gürer}
\address{Serap Gürer --- Galatasaray University, Ortaköy, Çırağan Cd. No:36, 34349 Beşiktaş/İstanbul, Turkey.}
\email{sgurer@gsu.edu.tr}

\author{Patrick Iglesias-Zemmour}
\address{Patrick Iglesias-Zemmour --- CNRS, France \& The Hebrew University of Jerusalem, Israel.}
\email{piz@math.huji.ac.il}

\thanks{This research is partially supported by Tübitak, Career Grant No. 115F410,
by Galatasaray University Research Fund Grant No. 19.504.003 and by the Labex Archimède, Aix-Marseille Université.}
\thanks{The authors thank the «~Institut d'Études Politiques d'Aix en Provence~» for its hospitality,
at Espace Seguin,
in July 2016 and 2017.}

\date{2019 (Published) / 2025 (Archived)}

\keywords{Diffeology, Manifolds with Corners, Differential Forms.}
\subjclass{58A35, 58A10}

\maketitle

\begin{abstract}
We identify the category of manifolds with boundary and corners with a subcategory of the category \{Diffeology\}.
Then,
we show that any differential form on a manifold with boundary or corners,
embedded in a smooth manifold,
extends to a differential form on an open neighbourhood.
As an example of application,
we discuss the structure of $\SO(2)^n$ invariant closed 2-forms on $\RR^{2n}$.
\end{abstract}

%%%%%%%%%%%%%%%%%%%%%%%%%%%%%%%%%%%%%%%%%%%%%%%%%%%%%%%%%%
%%
%% MARK: § Introduction
%%
%%%%%%%%%%%%%%%%%%%%%%%%%%%%%%%%%%%%%%%%%%%%%%%%%%%%%%%%%%
\section*{Introduction}

The $n$-corner $\K^n \subset \RR^n$ is defined by $x_i\geq 0$, $i=1,\dots,n$, it is equipped with the subset diffeology.
The definition of manifolds with corners follows naturally:

\textsc{Definition.} --- \textit{A $n$-manifold with corners is a diffeological space which is everywhere locally diffeomorphic to the corner $\K^n$.}

The introduction of manifolds with corners goes back to J. Cerf and A. Douady in \cite{Cer61,Dou62},
and has been since adapted or refined by many authors,
for example \cite{ADLH73,GP74,Lee06,Joy10} and more.

In \art{Manifolds-with-Corners-as-Diffeological-Spaces} we show that the classical approach and the diffeological definition lead to the same category,
which was mandatory.
This result extends the special case of manifolds with boundary addressed in \cite[\textsection 4.16]{PIZ13}.

Considering next differential forms defined on corners,
we prove in \art{Differential-Forms-On-Corners} the following extension lemma:

\textsc{Lemma.} \textit{Every differential form on the corner $\K^n$ extends to a smooth form on an open neighbourhood of $\K^n$ in $\RR^n$.}

From which we deduce the main theorem:

\textsc{Theorem.} \textit{Every differential form on a manifold with corners,
embedded in a smooth manifold,
extends smoothly to an open neighbourhood.}

We could have said that every differential form defined on a manifold with corners extends smoothly to an open neighbourhood,
but only in the sense that a manifold with corners can always be embedded in itself as a «pièce à coins»%
\footnote{Definition according to Douady and Herault :
«Let $\V$ be a manifold (without boundary) and let $\X$ be a closed set in $\V$.
We say that $\X$ is a ``piece with corners'' (\emph{pièce à coins}) of $\V$ if,
for all $x_0 \in \X$,
there exists a neighbourhood $\U$ of $x_0$ in $\V$ and functions $u_1,\dots, u_k$ of class $\Cinfty$ on $\U$ such that $d_{x_0}u_1,\dots, d_{x_0}u_k$  are linearly independent and that $\X \cap \U$ is the set of $x \in \U$ such that $u_1(x)\geq 0,\dots,u_k(x)\geq 0$.»};
according to Douady et al. in \emph{Arrondissement des Variétés à Coins} \cite[Proposition 3.1]{ADLH73}.
%
Let us notice also that the theorem above applies in particular to differential forms on manifolds with boundary,
since they are a particular case of manifolds with corners.

In \art{Other-Corners} we give a variation of this theorem for other corners:
powers of various \emph{half-lines} $\Delta_k = \RR^k/\O(k)$.
And,
in \art{An-Application} we give an application of this analysis to the decomposition of $\SO(2)^n$ invariant closed $2$-forms on $\RR^{2n}$.

\textsc{Note.}~--- For more details on diffeology,
the reader is referred to the textbook \emph{Diffeology} \cite{PIZ13}.

\textsc{Thanks.}~--- The authors gratefully thank the two referees of this paper for their comments and suggestions,
which definitely help to improve its quality.

%%%%%%%%%%%%%%%%%%%%%%%%%%%%%%%%%%%%%%%%%%%%%%%%%%%%%%%%%%
%%
%% MARK: § Diffeology Of Manifolds With Corners
%%
%%%%%%%%%%%%%%%%%%%%%%%%%%%%%%%%%%%%%%%%%%%%%%%%%%%%%%%%%%
\section*{Diffeology Of Manifolds With Corners}

We first recall some definitions in diffeology that we shall use in the following.

%%%%%%%%%%%%%%%%%%%%%%%%%%%%%%%%%%%%%%%%%%%%%%%%%%%%%%%%%%%%%%%%%%%%%%%
%% MARK: article: Diffeology And Diffeological Spaces
%%%%%%%%%%%%%%%%%%%%%%%%%%%%%%%%%%%%%%%%%%%%%%%%%%%%%%%%%%%%%%%%%%%%%%%
\begin{article}\artlabel[Diffeology And Diffeological Spaces.]
\label{Diffeology-And-Diffeological-Spaces}
A \emph{diffeology} on a set $\X$ is the choice of a set $\cD$ of parametrizations in $\X$ which satisfies the following axioms.

\begin{enumerate}
    \item \textsc{Covering}\,: $\cD$ contains the constant parametrizations.
    \item \textsc{Locality}\,: Let $\P$ be a parametrization in $\X$.
    If for all $r \in \dom(\P)$ there is an open neighbourhood $\V$ of $r$ such that $\P \restriction \V \in \cD$,
    then $\P \in \cD$.
    \item \textsc{Smooth compatibility}\,: For all $\P \in \cD$,
    for all $\F \in \Cinfty(\V,\dom(\P))$,
    where $\V$ is a Euclidean domain,
    $\P \circ \F \in \cD$.
\end{enumerate}

We recall that a parametrization is a map defined on an open subset of an Euclidean space.
A set $\X$ equipped with a diffeology is a \emph{diffeological space}.
The elements of $\cD$ are then called plots of the diffeological space.

\textsc{Smooth Maps.} --- \textit{
A map $f \colon \X \to \X'$ is said to be \emph{smooth} if for any plot $\P$ in $\X$,
$f \circ \P$ is a plot in $\X'$.
If $f$ is smooth,
bijective,
and if its inverse $f^{-1}$ is smooth,
then $f$ is said to be a \emph{diffeomorphism}.
}

Diffeological spaces and smooth maps constitute the category \{Diffeology\} whose diffeomorphisms are isomorphisms.

\textsc{Subset Diffeology, Subspaces.} --- \textit{Let $\A$ be a subset of a diffeological space $\X$.
The plots in $\X$ which take their values in $\A$ are a diffeology called \emph{subset diffeology}.
Equipped with this diffeology,
$\A$ is said to be a \emph{subspace} of $\X$.}

\textsc{Local Smooth Maps.} --- \textit{
The finest topology on $\X$ such that the plots are continuous is called \emph{D-Topology}.
A map $f \colon \A \to \X'$,
where $\A$ is a subset of $\X$,
is said to be \emph{local smooth} if $\A$ is a D-open subset of $\X$,
and $f$ is smooth for the subset diffeology.
We denote by $\Cinfty_\loc(\X,\X')$ the set of local smooth maps from $\X$ to $\X'$.
}

Actually $f$ is local smooth if and only if,
for all plots $\P$ in $\X$,
the composite $f \circ \P \colon \P^{-1}(\A) \to \X'$ is a plot.
That implies in particular that $\A$ is D-open,
by definition of the D-topology.

\textsc{Local Diffeomorphisms.} --- \textit{
We say that $f \colon \A \to \X'$ is a \emph{local diffeomorphism} if it is an injective local smooth map and $f^{-1}$,
defined on $f(\A)$,
is also a local smooth map.
We denote by $\Diff_\loc(\X,\X')$ the set of local diffeomorphisms from $\X$ to $\X'$.
A smooth map or a local smooth map is said to be \emph{étale} if it is a local diffeomorphism at each point.}
\end{article}

%%%%%%%%%%%%%%%%%%%%%%%%%%%%%%%%%%%%%%%%%%%%%%%%%%%%%%%%%%%%%%%%%%%%%%%
%% MARK: article: Corners as Diffeological Spaces
%%%%%%%%%%%%%%%%%%%%%%%%%%%%%%%%%%%%%%%%%%%%%%%%%%%%%%%%%%%%%%%%%%%%%%%
\begin{article}\artlabel[Corners as Diffeological Spaces.]
\label{Corners-as-Diffeological-Spaces}
We denote by $\K^n$ the \emph{corner}
$$
\K^n = \{ (x_i)_{i=1}^n \in \RR^n \mid x_i \geq 0,\ i=1,\dots,n\},
$$
and we  equip it with the subset diffeology%
\footnote{The corner $\K^n$ is the diffeological $n$-power of the half-line $\K = \mathopen[0,\infty\mathclose[ \subset \RR$,
equipped with the subset diffeology.}.
A plot in $\K^n$ is just a smooth parametrization to $\RR^n$ but taking its values in $\K^n$.

The corner $\K^n$ is \emph{embedded} in $\RR^n$,
and closed.
That is,
the D-topology of the induction $\K^n \subset \RR^n$ coincides with the induced topology%
\footnote{The standard topology of $\RR^n$ is the D-topology of its standard smooth structure.}
of $\RR^n$.

The natural filtration of $\K^n$,
$
\X_0 = \{0\}\subset\X_1\subset \dots \subset \X_n=\K^n,
$
is defined by
$$
\X_j = \{(x_i)_{i=1}^n \in \K^n \mid \text{there exist } i_1<\dots<i_{n-j} \text{ such that } x_{i_\ell} = 0 \}.
$$

The subset $\S_j = \X_j - \X_{j-1}$ is the set of points in $\RR^n$ which have $j$ coordinates positive,
and the others null:
$$
\S_j = \bigg\{
(x_i)_{i=1}^n \in \RR^n \, \bigg| \,
\begin{array}{l}
\text{There exist } i_1<\dots<i_j \text{ such that } x_{i_\ell} > 0, \\
\text{and } x_m = 0 \text{ for all } m \notin \{i_1,\dots,i_j \}
\end{array}
\bigg\}.
$$
The \emph{layer} $\S_j$ is D-open in $\X_j$,
for $j \geq 1$.
It is the sum of $ n \choose j$ connected components called \emph{strata},
indexed by a string of $j$ ones and $n-j$ zeros.

\textsc{Theorem 1.} --- \textit{A map $f \colon \K^n \to \RR^k$ is smooth for the subset diffeology if and only if it is the restriction of a smooth map on some open neighbourhood of $\K^n$.}

This statement means technically that,
if for all smooth parametrizations $\P \colon \U \to \RR^n$ such that $\P(\U) \subset \K^n$, $f \circ \P \in \Cinfty(\U,\RR^k)$,
then there exists an open neighbourhood $\cO$ of $\K^n$ and $\F \in \Cinfty(\cO,\RR^k)$ such that $f = \F \restriction \K^n$.

\textsc{Theorem 2.} --- \textit{Let $f\in \Diff_\loc(\K^n)$.
Then,
$f$ respects the natural stratification,
\textit{i.e.} if $x\in \S_j$,
then $f(x)\in \S_j$.
Moreover,
$f$ is the restriction of an étale map defined on an open neighbourhood of its domain of definition.}
\end{article}

\begin{proof}
Theorem 1 was already proven in \cite[\textsection 4.16]{PIZ13}.
%
For Theorem 2,
let $\P$ be a parametrization in an Euclidean domain.
We denote by $\rk(\P)_x$ the rank of $\P$ at the point $x$,
that is,
the dimension of the image of the tangent linear map $\D(\P)(x)$.

\textsc{Lemma.} --- \textit{Let $\P \colon \U \rightarrow \K^n$ be a plot.
If $\P(r) \in \S_j$,
then $\rk(\P)_r \leq j$.}

$\blacktriangleleft$ \textit{Proof.} Let $\P \colon \U \rightarrow \K^n$ be a plot and assume that $\P(r) \in \S_j$.
Then,
$\P(r)=(\P_1(r),\P_2(r),\dots,\P_n(r))$ where there exist exactly $i_1 < \dots < i_{n-j}$ indices such that  $\P_{i_k}(r) = 0$.
Since $\P_{i_k}(r') \geq 0$ for all $r' \in \U$ and $\P_{i_k}(r) = 0$,
then $\D(\P_{i_k})(r) = 0$.
That is,
$\rk(\P)_r \leq j$. \nolinebreak \hfill $\blacktriangleright$

Now,
let us come back to the local diffeomorphism $f$,
and assume first that $f$ is defined on all $\K^n$.
Let $x \in \S_j$ and $x' = f(x) \in \S_k$ and $k \neq j$.
We can choose $k > j$.
There exists a smooth map $\F$ defined on an open neighbourhood $\cO \supset \K^n$,
such that $f$ and $\F$ coincide on $\K^n$,
$f = \F \restriction \K^n$.
And also,
there exists a smooth map $\G$ defined on an open neighbourhood $\cO' \supset \K^n$,
such that $f^{-1}$ and $\G$ coincide on $\K^n$,
$f^{-1} = \G \restriction \K^n$.
The restriction of $\G$ on $\S_k$ is a plot in $\K^n$,
and $\G \restriction \S_k \colon x' \mapsto x \in \S_j$.
By the lemma,
$\rk(\G \restriction \S_k)_{x'} \leq j$.
But $\G \restriction \S_k = \G \circ j_k$,
where $j_k: \S_k \hookrightarrow \K^n$ is identified with a plot.
And we know that $(\F \circ \G \restriction \S_k)(t)= \F \circ \G \circ j_k(t)=\F \circ \G(j_k(t))$.
But $j_k$ takes values in $\partial \K^n$ (the boundary of $\K^n$).
Now,
since $f$ is a homeomorphism of $K^n$ for the $\D$-topology, it maps the boundary into the boundary,
and $\G$ and $f^{-1}$ coincide on the boundary.
So we have $\F\circ \G(j_k(t)) = \F \circ f^{-1}(j_k(t))$.
As well,
$\F$ and $f$ coincide on the boundary,
and $\F \circ \G(j_k(t))=f \circ f^{-1}(j_k(t))=j_k(t)$.
Thus,
$\rk(\F \circ \G \restriction \S_k)_{x'} = \rk(j_k)_{x'} = k \leq \rk (\G \restriction \S_k)_{x'}\leq j$.
But,
we assumed that $k>j$ which is a contradiction,
and $k=j$.

Now,
consider the smooth parametrization $\G \circ \F \colon \cU \to \RR^n$,
with $\cU = \F^{-1}(\cO')$.
Then,
$\cU \supset \K^n$ and $\G \circ \F \restriction \K^n = \id_{\K^n}$.
Hence,
for all $x \in \mathring{\K}^n$,
$\D(\G \circ \F)(x) = \D(f^{-1} \circ f)(x) = \D(\id_{\mathring{\K}^n}) = \id_{\RR^n}$,
and by continuity,
for all $x \in \K^n$,
$\D(\G \circ \F)(x) =\id_{\RR^n}$.
Therefore,
for all $x \in \K^n$,
$\rk(\F)_x$ is maximum and equal to $n$.
Hence,
$\F$ is étale on $\K^n$.
And obviously,
the same for $\G$.
The situation,
where $f$ is only local,
is similar.
\end{proof}

%%%%%%%%%%%%%%%%%%%%%%%%%%%%%%%%%%%%%%%%%%%%%%%%%%%%%%%%%%%%%%%%%%%%%%%
%% MARK: article: Manifolds with Corners as Diffeological Spaces
%%%%%%%%%%%%%%%%%%%%%%%%%%%%%%%%%%%%%%%%%%%%%%%%%%%%%%%%%%%%%%%%%%%%%%%
\begin{article}\artlabel[Manifolds with Corners as Diffeological Spaces.]
\label{Manifolds-with-Corners-as-Diffeological-Spaces}
The concept of manifolds with corners goes back to Cerf \cite[Chap. 1 \textsection 1.2]{Cer61},
and Douady \cite[\textsection 4]{Dou62} (as \emph{variétés à bords anguleux}).
Over time the various descriptions of manifolds with boundary or corners evolved to a commonly accepted definition,
see for example Lee in \cite[pp. 251-252]{Lee06} from which we extract the following definition%
\footnote{See also Joyce in \cite[Chap. 2]{Joy10}.}.

\textsc{Classical Definition.} --- \textit{Let $\M$ be a paracompact Hausdorff topological space.
A $n$-chart with corners for $\M$ is a pair $(\U,\phi)$,
where $\U$ is an open subset of $\K^n$,
and $\phi$ is a homeomorphism from $\U$ to an open subset of $\M$.
Two charts with corners $(\U,\phi)$ and $(\V,\psi)$ are said to be smoothly compatible if the composite map
$\psi^{-1} \circ \phi \colon \phi^{-1}(\psi(\V)) \to \psi^{-1}(\phi(\U))$ is a diffeomorphism in the
sense that it admits a smooth extension to an open set in $\RR^n$.
An $n$-atlas with corners for $\M$ is a pairwise compatible family of n-charts with corners covering $\M$.
A maximal atlas is an atlas which is not a proper subset of any other atlas.
An $n$-manifold with corners is a paracompact Hausdorff topological space $\M$ equipped with a maximal $n$-atlas with corners.}

Then,
from the diffeology point of view:

\textsc{Diffeological Definition.} --- \textit{A $n$-manifold with corners is a diffeological space $\X$ which is everywhere locally diffeomorphic to the corner $\K^n$.}

The \{Manifolds with Corners\} form a subcategory of \{Diffeology\}.
The smooth maps between manifolds with corners are just the smooth maps between diffeological spaces.
Theorems 1 and 2 of the previous subsections ensure that morphisms between manifolds with corners preserve their natural stratifications.

\textsc{Theorem.} --- \textit{Let $(\M,\cA)$ be a $n$-manifold with corners according to the classical framework,
$\cA$ being the maximal atlas of $\M$.
The diffeology $\cD$ on $\M$ generated%
\footnote{See \cite[\textsection 1.66]{PIZ13} for the definition of \emph{generating family}.}
by the charts $\F \in \cA$ is a diffeology of manifold with corners for which $\Diff_\loc(\K^n,\M) = \cA$.
The D-topology of $(\M,\cD)$ coincides with the given topology of $(\M,\cA)$.
We shall denote $\Phi \colon (\M,\cA) \mapsto (\M,\cD)$ this association.
Conversely,
let $(\M,\cD)$ be a diffeological $n$-manifold with corners.
Equip $\M$ with its D-topology,
then $\cA = \Diff_\loc(\K^n,\M)$ is a maximal atlas equipping $\M$ with a usual structure of manifold with corners.
Let $\Psi \colon (\M,\cD) \mapsto (\M,\cA)$ be this association.
Then $\Phi$ and $\Psi$ are inverse of each other.}

Hence,
the classical category of manifolds with corners,
defined using the heuristic of what smoothness means for maps on closed subsets of Euclidean domains,
fits fully and faithfully in the category \{Diffeology\}.
That shows,
by the way,
that it is unnecessary to adopt the axiomatics where plots are defined on convex subsets,
as in Chen's approach \cite{Che77},
to include manifolds with corners in the new category.
Specifying the local model to be a corner equipped with its subset diffeology is,
indeed,
sufficient.

\textsc{Note 1.} --- As in the case of ordinary manifolds,
the category \{Manifolds with corners\} is closed for products and sums but is not closed for the other usual set theoretical operations.
As members of the category \{Diffeology\},
manifolds with corners inherit of all the diffeological constructions:
fiber bundles,
homotopy,
differential calculus,
homology,
cohomology,
etc.

\textsc{Note 2.} --- Since,
for a corner,
coordinates are smooths,
manifolds with corners are \emph{reflexive diffeological spaces} \cite[Exercise 79]{PIZ13}.
Also,
the set of smooth real maps from a manifold with corner is a \emph{differential structure} in the sense of Sikorski \cite{Sik72}.
Indeed,
real smooth maps from $\K^n$ to $\RR^m$ are still the restrictions of smooth maps on open neighbourhoods of $\K^n$.

\textsc{Note 3.} --- The diffeology framework gives a different perspective on the definition of strata of a $n$-manifold with corners $\M$.
We can regard it as the \emph{active point of view}.
Indeed,
instead of defining the strata through local chart
---~the \emph{passive point of view}---~
following Theorem 2 of \art{Corners-as-Diffeological-Spaces},
one can define the different strata of $\M$ as the connected components of the orbits of the pseudogroup of local diffeomorphisms $\Diff_\loc(\M)$.
That is,
$$
\Strata(\M) = \{\cO_i \in \pi_0(\cO) \mid \cO \in \M/\Diff_\loc(\M)\}.
$$
Moreover,
$\Strata(\M)$ does not capture only,
as a set,
the decomposition of $\M$ in strata,
but its quotient diffeology can be regarded as the \emph{transversal smooth structure} of the stratification,
see \cite[\textsection 1.42]{PIZ13}.
Note also that the \emph{regular part} of $\M$,
that is,
the principal orbit of $\Diff_\loc(\M)$,
is the union of strata of dimension $n$.
It is a regular $n$-submanifold and an open dense subset of $\M$ as it must be.
Actually,
$\M$ has a (geometrical) structure of locally conelike stratified space \cite{GIZ18}.
\end{article}

\begin{proof}
Let us consider a manifold with corners $(\M,\cA)$,
according to the classical definition.
The finest diffeology $\cD$ making the charts $\F \in \cA$ smooth is the set of parametrizations $\P \colon \U \to \M$ that satisfy the following:
there is a covering of $\U$ by a family of open sets $\U_i$,
and for each index $i$ a chart $\F_i \in \cA$ and a smooth maps $\Q_i \colon \U_i \to \K^n$ such that $\P \restriction \U_i = \F_i \circ \Q_i$.
We write $\P = \sup \F_i \circ \Q_i$.

Now, the charts $\F \in \cA$ are smooth,
by construction,
and injective.
Their domains are open for the induced topology of $\K^n$,
which is also the D-topology of $\K^n$,
according to above.

Let us show now that the topology of $\M$ and its D-topology coincide.
Let first $\cO \subset \M$ be an open subset of $\M$.
Let $\P$ be a plot of $\M$,
then $\P = \sup_i \F_i \circ \Q_i$ for some family of indices,
with the $\F_i$ in $\cA$ and the $\Q_i$ smooth parametrizations in $\K^n$.
Then,
$\P^{-1}(\cO) = (\sup \F_i \circ \Q_i)^{-1}(\cO) = \cup_i \Q_i^{-1}(\F_i^{-1}(\cO))$.
And since the $\F_i$ and the $\Q_i$ are continuous,
$\P^{-1}(\cO)$ is open.
Thus,
$\cO$ is open for the D-topology.
Conversely,
let $\cO$ be open for the D-topology.
For all $x \in \cO$,
there exists $\F_x \in \cA$ such that $x \in \val(\F_x)$.
Since $\F_x$ is a plot for $\cD$,
$\F_x^{-1}(\cO)$ is open in $\K^n$,
and since $\F_x$ is a local homeomorphism from $\K^n$ to $\M$,
$\F_x \restriction \F_x^{-1}(\cO)$ is still a local homeomophism.
Then,
$\val(\F_x \restriction \F_x^{-1}(\cO))$ is open in $\M$.
But $\val(\F_x \restriction \F_x^{-1}(\cO)) = \cO \cap \val(\F_x)$,
thus $\cO = \cup_x \cO \cap \val(\F_x)$ is a union of open subsets,
then open in $\M$.
Therefore the topologies coincide.

Let us prove now that $\Diff_\loc(\K^n,\M) = \cA$.
Let $\Phi \in \Diff_\loc(\K^n,\M)$.
Since the two topologies coincide,
we know already that $\Phi$ is a local homeomorphism from $\K^n$ to $\M$.
Now let $\F \in \cA$,
thus $\F^{-1} \circ \Phi = \F^{-1} \circ (\sup \F_i \circ \Q_i)$,
where $\Phi = \sup \F_i \circ \Q_i$,
as previously.
Hence,
$\F^{-1} \circ \Phi = \sup (\F^{-1} \circ \F_i) \circ \Q_i$.
But the $\F^{-1} \circ \F_i$ and the $\Q_i$ are smooth,
and moreover local diffeomorphisms,
thus $\F^{-1} \circ \Phi$ is a local diffeomorphism,
and then also $\Phi^{-1} \circ \F$.
Hence,
since $\cA$ is maximal,
$\Diff_\loc(\K^n,\M) \subset \cA$.
Next let $\F \in \cA$.
We know already that $\F$ is smooth,
and a local homeomorphism for both topologies.
Let us show that $\F^{-1} \colon \val(\F) \to \K^n$ is smooth.
Let $\P$ be a plot in $\val(\F) \subset \M$,
then $\P = \sup \F_i \circ \Q_i$.
Hence,
$\F^{-1} \circ \P = \sup (\F^{-1} \circ \F_i) \circ \Q_i$.
Thus,
$\F^{-1}$ is smooth and $\F$ is a local diffeomorphism.
Therefore,
$\cA \subset \Diff_\loc(\K^n,\M)$,
and then $(\M,\cD)$ is a diffeological manifold with corners such that $\Diff_\loc(\K^n,\M) = \cA$.

The proof of the converse,
from $(\M,\cD)$ to $(\M,\cA)$ with $\cA = \Diff_\loc(\K^n,\M)$,
is of the same vein. We leave it as an exercise for the reader.
\end{proof}

%%%%%%%%%%%%%%%%%%%%%%%%%%%%%%%%%%%%%%%%%%%%%%%%%%%%%%%%%%
%%
%% MARK: § Extension of Differential Forms on Corners
%%
%%%%%%%%%%%%%%%%%%%%%%%%%%%%%%%%%%%%%%%%%%%%%%%%%%%%%%%%%%
\section*{Extension of Differential Forms on Manifolds with Corners}

In this section,
we prove that any differential form on a manifold with corners is the restriction of a differential form on an open neighbourhood,
after having been pushed inside itself as a submanifold with corner,
thanks to a construction due to Douady \emph{et al.} described in \cite[Proposition 3.1]{ADLH73}.

%%%%%%%%%%%%%%%%%%%%%%%%%%%%%%%%%%%%%%%%%%%%%%%%%%%%%%%%%%%%%%%%%%%%%%%
%% MARK: article: The Square Function Lemma
%%%%%%%%%%%%%%%%%%%%%%%%%%%%%%%%%%%%%%%%%%%%%%%%%%%%%%%%%%%%%%%%%%%%%%%
\begin{article}\artlabel[The Square Function Lemma.]
\label{The-Square-Function-Lemma}
Let $\sq \colon \RR^n \to \K^n$ be the smooth parametrization:
$$
\sq(x_1,\dots,x_n) = (x_1^2,\dots,x_n^2).
$$
Then $\sq^* \colon \Omega^k(\K^n) \to \Omega^k(\RR^n)$ is injective.
That is,
for all $\alpha \in \Omega^k(\K^n)$,
if $\sq^*(\alpha) = 0$,
then $\alpha = 0$.
\end{article}

\begin{proof}
Note that each component of $\X_j - \X_{j-1}$ is diffeomorphic to $\RR^j$.
Thus,
if $\sq^*(\alpha) = 0$,
then $\alpha \restriction (\X_j - \X_{j-1}) = 0$,
because $\sq \restriction \sq^{-1}(\X_j - \X_{j-1})$ is a $2$-fold covering over $\X_j - \X_{j-1}$.
Hence,
for all plots $\Q$ in $\X_j - \X_{j-1}$,
$\alpha(\Q)=0$.
%
Let then,
for some $j\geq 1$,
$\P_j \colon \U_j \to \X_j$ be a plot.
According to \art{Corners-as-Diffeological-Spaces},
the subset $\cO_j = \P_j^{-1}(\X_j - \X_{j-1})$ is open,
and $\alpha(\P_j \restriction \cO_j) = \alpha(\P_j) \restriction \cO_j = 0$.
By continuity,
$\alpha(\P_j) \restriction \overline{\cO}_j = 0$,
where $\overline{\cO}_j$ is the closure of $\cO_j$.
Let then $\U_{j-1} = \U_j - \overline{\cO}_j$ and $\P_{j-1} = \P_j \restriction \U_{j-1}$.
Then,
$\U_{j-1}$ is open and $\P_{j-1} \colon \U_{j-1} \to \X_{j-1}$ is a plot.
%
This construction gives a descending recursion,
starting with any plot $\P \colon \U \to \K^n$,
by initializing $\P_n = \P$,
$\U_n = \U$ and $\X_n = \K^n$.
One has $\P_j = \P \restriction \U_j$,
$\U_{j-1} \subset \U_j$,
the recursion ends with a plot $\P_0$ with values in $\X_0 = \{0\}$,
and $\alpha(\P_0) = 0$ since $\P_0$ is constant.
%
Therefore $\alpha = 0$.
\end{proof}

%%%%%%%%%%%%%%%%%%%%%%%%%%%%%%%%%%%%%%%%%%%%%%%%%%%%%%%%%%%%%%%%%%%%%%%
%% MARK: article: Differential Forms On Corners
%%%%%%%%%%%%%%%%%%%%%%%%%%%%%%%%%%%%%%%%%%%%%%%%%%%%%%%%%%%%%%%%%%%%%%%
\begin{article}\artlabel[Differential Forms On Corners.]
\label{Differential-Forms-On-Corners}
The extension of smooth real functions on corners \art{Corners-as-Diffeological-Spaces},
is a particular case of the following lemma on differential forms of any degree.

\textsc{Lemma.} \textit{Any differential $k$-form on the corner $\K^n$,
equipped with the subset diffeology of\/ $\RR^n$,
is the restriction of a smooth differential $k$-form defined on some open neighbourhood of the corner.
}

In other words,
the pullback $:j^* \colon \Omega^k(\RR^n) \to \Omega^k(\K^n)$ is surjective,
where $j$  denotes the inclusion from $\K^n$ to $\RR^n$.

As a corollary of the previous lemma we get:

\textsc{Theorem.} \textit{Let $\M$ be a Hausdorff paracompact submanifold with corners,
embedded in a smooth manifold $\M'$.
Then,
any form $\omega \in \Omega^k(\M)$ can be smoothly extended to an open neighbourhood of $\M$ in $\M'$.}

\textsc{Note 1.} Thanks to \cite[Proposition 3.1]{ADLH73}, every manifold with corners can be embedded in itself.
Thus,
in the previous theorem,
to be a submanifold with corners of a smooth manifold is not a restriction.

\textsc{Note 2.} The theorem applies obviously to manifold with boundary,
since they are a particular case of manifolds with corners.

\textsc{Note 3.} There is a subtlety about the value of a form \cite[\textsection 6.40]{PIZ13} at the boundary of a manifold with corners.
For example on the half-line $\mathopen[0,\infty\mathclose[ \subset \RR$,
the restriction $\alpha$ of the constant $1$-form $a\, d\!x$ is a differential form on the half line.
Its value at $0$ is zero,
since it vanishes when precomposed by a plot and evaluated at zero,
that is,
$\alpha(\gamma)_0 = 0$ for all paths $\gamma$ in $\mathopen[0,\infty\mathclose[$ such that $\gamma(0)=0$.
But,
of course,
the value of the extension $a\,d\!x$ of $\alpha$ on $\RR$,
at the point $0$,
is the number $a$,
as usual.
So,
we have to be careful when using the notion of values when it comes to differential forms in diffeology.
\end{article}

\begin{proof}
Let us prove the lemma.
Let $\omega \in \Omega^k(\K^n)$ and $\mathring\K^n = \{ (x_i)_{i=1}^n \mid x_i > 0, i=1,\dots,n \}$.
One has
$$
\omega \restriction \mathring\K^n = \sum_{i_1<\dots<i_k} a_{i_1 \dots i_k}(x_1, \dots, x_n)\ dx_{i_1} \wedge \dots \wedge dx_{i_k},
$$
with $i_j = 1,\dots,n$ and $a_{i_1 \dots i_k} \in \Cinfty(\mathring\K^n,\RR)$.
Recall that $\sq \colon (x_i)_{i=1}^n \mapsto (x_i^2)_{i=1}^n$,
then
$$
\sq^*(\omega) = \sum_{i_1<\dots<i_k} \A_{i_1 \dots i_k}(x_1,\dots, x_n)\ dx_{i_1} \wedge \dots \wedge dx_{i_k},
$$
where $\A_{i_1 \dots i_k} \in \Cinfty(\RR^n,\RR)$.
Let $\varepsilon_j : (x_1,\dots, x_j,\dots,x_n) \mapsto (x_1,\dots, -x_j,\dots,x_n)$,
then $ \sq \circ \varepsilon_j = \sq$ and
$(\sq \circ \varepsilon_j)^*(\omega) = \varepsilon_j^*(\sq^*(\omega))$,
that is,
$\sq^*(\omega) = \varepsilon_j^*(\sq^*(\omega))$.
Hence,
\begin{eqnarray*}
    \varepsilon_j^*(\sq^*(\omega)) & =& \sum_{i_1<\dots<i_k \atop i_\ell \neq j} \A_{i_1 \dots i_k}(x_1,\dots, -x_j,\dots, x_n)\ dx_{i_1} \wedge \dots \wedge dx_{i_k} \\
    & - & \sum_{i_1 < \dots \leq j \leq \dots < i_k} \A_{i_1\dots j \dots i_k}(x_1,\dots, -x_j,\dots, x_n)\ dx_{i_1} \wedge \dots \wedge dx_j \wedge \dots \wedge dx_{i_k}.
\end{eqnarray*}
Then,
\begin{eqnarray*}
    \A_{i_1 \dots i_k \atop i_\ell \neq j}(x_1,\dots, -x_j,\dots, x_n) & = & \A_{i_1 \dots i_k}(x_1,\dots, x_j,\dots, x_n), \\
    \A_{i_1 \dots j \dots i_k}(x_1,\dots, -x_j,\dots, x_n) & = & - \A_{i_1\dots j \dots i_k}(x_1,\dots, x_j,\dots, x_n).
\end{eqnarray*}
Hence,
$$
\A_{i_1 \dots j \dots i_k}(x_1,\dots, x_j=0,\dots, x_n) = 0.
$$
Thus,
$$
\quad \A_{i_1\dots j \dots i_k}(x_1,\dots, x_j,\dots, x_n) = 2 x_j \underline{\A}_{i_1\dots j \dots i_k}(x_1,\dots, x_j,\dots, x_n),
$$
with $\underline{\A}_{i_1\dots j \dots i_k} \in \Cinfty(\RR^n,\RR)$.
By induction,
there are real smooth functions $\hat\A_{i_1\dots i_k}$ defined on $\RR^n$ such that
$$
\quad \A_{i_1 \dots i_k}(x_1,\dots, x_n) = 2^k x_{i_1}\dots x_{i_k} \hat\A_{i_1\dots i_k}(x_1,\dots, x_n).
$$
Now,
$$
\sq^*(\omega\restriction \mathring \K^n) = \sq^*(\omega) \restriction \{x_i \neq 0\}
$$
implies
\begin{eqnarray*}
    & & \sum_{i_1<\dots<i_k} 2^k x_{i_1}\dots x_{i_k} a_{i_1 \dots i_k}(x^2_1,\dots, x^2_n)\ dx_{i_1} \wedge \dots \wedge dx_{i_k} \\
    & = &
    \sum_{i_1<\dots<i_k} 2^k x_{i_1}\dots x_{i_k} \hat\A_{i_1 \dots i_k}(x_1,\dots, x_n) \ dx_{i_1} \wedge \dots \wedge dx_{i_k}.
\end{eqnarray*}
Hence,
$$
\hat\A_{i_1 \dots i_k}(x_1,\dots, x_n) = a_{i_1 \dots i_k}(x^2_1,\dots, x^2_n) \quad \text{for } x_i \neq 0, i = 1,\dots,n.
$$
Thus $(x_1,\dots, x_n) \mapsto \hat\A_{i_1 \dots i_k}(x_1,\dots, x_n)$,
which belongs to $\Cinfty(\RR^n,\RR)$,
is even in each variable.
Therefore,
according to Whitney's theorem on even smooth functions \cite[Theorem 1 \& Remark p.160]{Whi43},
there exist
$$
\underline{a}_{i_1 \dots i_k} \in \Cinfty(\RR^n,\RR),
$$
such that
$$
\hat\A_{i_1 \dots i_k}(x_1,\dots, x_n) = \underline{a}_{i_1 \dots i_k}(x_1^2,\dots, x_n^2).
$$
One deduces:
$$
\underline{a}_{i_1 \dots i_k}(x_1,\dots, x_n) = a_{i_1 \dots i_k}(x_1,\dots, x_n), \text{ for all } (x_1,\dots, x_n) \in \mathring{\K}^n.
$$
Then,
defining the $k$-form $\underline\omega$ on $\RR^n$ by
$$
\underline\omega = \sum_{i_1<\dots<i_k} \underline{a}_{i_1 \dots i_k}(x_1, \dots, x_n)\ dx_{i_1} \wedge \dots \wedge dx_{i_k},
$$
one has already
$$
\underline\omega \restriction \mathring\K^n = \omega \restriction \mathring\K^n.
$$
Let us show that $\underline\omega \restriction \K^n = \omega$.
That is,
let us check that for all plot $\P \colon \U \to \RR^n$,
$\P^*(\underline\omega)=\omega(\P)$.
Actually,
one has
$$
\sq^*(\omega) = \sq^*(\underline\omega \restriction \K^n).
$$
Indeed:
\begin{eqnarray*}
    \sq^*(\omega) & = & \sum_{i_1\dots i_k} \A_{i_1\dots i_k}(x_1,\dots,x_n)\ dx_{i_1}\wedge\dots\wedge dx_{i_k} \\
    & = & \sum_{i_1\dots i_k} 2^k x_{i_1} \dots x_{i_k} \hat\A_{i_1\dots i_k}(x_1,\dots,x_n)\ dx_{i_1}\wedge\dots\wedge dx_{i_k} \\
    &=& \sum_{i_1\dots i_k} 2^k x_{i_1} \dots x_{i_k} \underline{a}_{i_1\dots i_k}(x_1^2,\dots,x_n^2)\ dx_{i_1}\wedge\dots\wedge dx_{i_k}.
\end{eqnarray*}
And,
on the other hand:
\begin{eqnarray*}
    \sq^*(\underline\omega\restriction \K^n) = \sum_{i_1\dots i_k} 2^k x_{i_1} \dots x_{i_k} \underline{a}_{i_1\dots i_k}(x_1^2,\dots,x_n^2)\ dx_{i_1}\wedge\dots\wedge dx_{i_k}.
\end{eqnarray*}
Thus,
$\sq^*(\omega - \underline\omega \restriction \K^n) = 0$.
Therefore,
according to the lemma \art{The-Square-Function-Lemma},
$\omega - \underline\omega \restriction \K^n = 0$.
And then,
$\omega$ is the restriction on $\K^n$ of the smooth $k$-form $\underline\omega$ on $\RR^n$.

For the theorem:
thanks to the lemma,
we consider a locally finite open cover $\{\U_i\}$ of the boundary of $\M \subset \M'$ where the form $\omega$ has been extended smoothly on each $\U_i$ by $\omega_i$.
Then,
there exists a smooth partition of unity $\lambda_j$,
subordinate to a subcover $\{V_j\}$.
Let $\omega_j = \omega_i \restriction \V_j$,
for some index $i$ such that $\V_j \subset \U_i$.
The smooth form $\bar\omega = \sum_j \lambda_j \omega_j$,
defined on the neighbourhood $\cup_j \V_j$ of the boundary of $\M$,
is an extension of $\omega$ on $(\cup_j \V_j) \cap \M $.
It defines this way an extension of $\omega$ on an open neighbourhood of $\M$ in $\M'$.
\end{proof}

%%%%%%%%%%%%%%%%%%%%%%%%%%%%%%%%%%%%%%%%%%%%%%%%%%%%%%%%%%%%%%%%%%%%%%%
%% MARK: article: Other Corners
%%%%%%%%%%%%%%%%%%%%%%%%%%%%%%%%%%%%%%%%%%%%%%%%%%%%%%%%%%%%%%%%%%%%%%%
\begin{article}\artlabel[Other Corners.]
\label{Other-Corners}
The \emph{half-line} $\Delta_k = \RR^k/\O(k)$ is identified to the interval $\mathopen[0,\infty\mathclose[$,
equipped with the pushforward of the smooth diffeology of $\RR^k$ by the projection $\nu_k \colon \X \mapsto \norm{\X}^2$,
see \cite{PIZ07}.
Then,
with each half-line we can associate a \emph{$n$-corner} $\Delta_k^n$.
But note that,
according to definition \art{Manifolds-with-Corners-as-Diffeological-Spaces},
none of these corners is a manifold with corners.
Then,
let $\J_k^n : \Delta_k^n \to \RR^n$ be the natural injection.

\textsc{Proposition.} --- \textit{The pullback ${J_k^n}^* \colon \Omega^*(\RR^n) \to \Omega^*(\Delta_k^n)$ is surjective.
That is,
every differential form $\epsilon$ on $\Delta_k^n$ is the pullback by $\J_k^n$ of some smooth form $\alpha \in \Omega^*(\RR^n)$.}

\textsc{Note.} Let $j : \K^n \to \RR^n$ be the natural injection,
and let $j_k^n \colon \Delta_k^n \to \K^n$ be the injection such that $\J_k^n = j \circ j_k^n$.
Then,
thanks to the previous proposition,
${j_k^n}^*$ is a bijection.
\end{article}

\begin{proof}
First,
we notice that the map $\sq \colon \RR^n \to \Delta_k^n$ is still smooth.
Then,
we check that $\J_k^n$ is an embedding,
that is,
the pullback of the topology of $\RR^n$ is the D-topology of $\Delta_k^n$.
We conclude that the lemma \art{The-Square-Function-Lemma} is still true,
replacing $\K^n$ by $\Delta_k^n$ and $\mathring{\K}^n$ by $\mathring{\Delta}_k^n$.
Next the proof of \art{Differential-Forms-On-Corners} applies \emph{mutatis mutandis} to $\Delta_k^n$,
considering ${\J_k^n}^*(\omega)$ instead of $\omega \restriction \K^n$.
\end{proof}

%%%%%%%%%%%%%%%%%%%%%%%%%%%%%%%%%%%%%%%%%%%%%%%%%%%%%%%%%%%%%%%%%%%%%%%
%% MARK: article: An Application
%%%%%%%%%%%%%%%%%%%%%%%%%%%%%%%%%%%%%%%%%%%%%%%%%%%%%%%%%%%%%%%%%%%%%%%
\begin{article}\artlabel[An Application.]
\label{An-Application}
We can apply the theorems above to describe the closed $2$-forms on manifolds,
invariant with respect to some action of a Lie group.
%As it has been shown in particular in the classification of $\SO(3)$-symplectic manifolds \cite{Igl84,Igl91},
Roughly speaking,
a closed $2$-form $\omega$ on a manifold $\M$,
invariant by the Hamiltonian action of a compact group%
\footnote{There could a diffeological generalisation possible here to non compact group.}
$\G$,
is characterized by its \emph{moment map} $\mu \colon \M \to \cG^*$,
and for each given moment map,
a closed $2$-form $\varepsilon \in \Z^2(\M/\G)$.
Let us be more precise:
the space of $\G$-invariant closed $2$-forms $\Z^2(\M)^\G$ is a vector space,
the space of $\G$-equivariant maps from $\M$ to $\cG^*$ is also a vector space,
and the map associating its moment map%
\footnote{The manifold $\M$ is supposed to be connected.
To have a uniqueness of the moment maps we decide to fix their value to $0$ at some base point $m_0 \in \M$,
for example.}
$\mu$ with each invariant closed $2$-form $\omega$ is linear.
What we claim is that the kernel of this map is exactly $\Z^2(\M/\G)$,
where $\M/\G$ is equipped with the quotient diffeology.
Denoting by ${\mathcal E\!q}_{\!\bullet}(\M,\cG^*) \subset {\mathcal E\!q}(\M,\cG^*)$ the space of moment maps of $\G$-invariant closed $2$-forms on $\M$,
as a subset of smooth equivariant maps,
one has this exact sequence of smooth linear maps:
$$
0 \xrightarrow[]{} \Z^2(\M/\G) \xrightarrow[]{} \Z^2(\M)^\G \xrightarrow[]{} {\mathcal E\!q}_{\!\bullet}(\M,\cG^*) \xrightarrow[]{} 0.
$$
An equivariant map is easy to conceive,
but a differential form on the space of orbits,
which is generally not a manifold,
is more problematic.
This is where the above theorem can help:
$\M/\G$ is a diffeological space which,
sometimes,
is not far from being a manifold with boundary or corners,
as the following example shows.

Consider for example $\M = \RR^{2n}$,
and the space of closed $2$-forms invariant by $\SO(2)^n$.
The quotient space $\cQ^n = \RR^{2n}/\SO(2)^n$ is equivalent to the \emph{other corner} $\Delta_2^n$,
with $\Delta_2 = \RR^2/\O(2)$.
Let $\omega$ and $\omega'$ be two closed $2$-forms,
sharing the same moment map $\mu \colon \RR^{2n} \to \RR^n$.
Then,
according to the previous sections,
there exists a closed $2$-form $\epsilon$ defined on an open neighbourhood of $\K^n \subset \RR^n$ (which could be $\RR^n$),
such that $\omega' = \omega + \pi^*(\epsilon)$,
with $\pi \colon (z_1,\dots,z_n) \mapsto (||z_1||^2,\dots,||z_n||^2)$,
$z_i \in \RR^2$.
Moreover, since $\K^n$ is contractible,
$\epsilon = d\alpha$,
where $\alpha$ is a smooth extension of a $1$-form on $\K^n$.
Therefore,
$\omega' = \omega + d(\pi^*(\alpha))$.
\end{article}

%%%%%%%%%%%%%%%%%%%%%%%%%%%%%%%%%%%%%%%%%%%%%%%%%%%%%%%%%%
%%
%% MARK: § Bibliographie
%%
%%%%%%%%%%%%%%%%%%%%%%%%%%%%%%%%%%%%%%%%%%%%%%%%%%%%%%%%%%

\begin{thebibliography}{ADLH73}
    
    \bibitem[Cer61]{Cer61}
    Jean Cerf.
    \newblock {\em Topologie de certains espaces de plongements}.
    \newblock {Bulletin de la SMF}, (89):227--280, 1961.
    
    \bibitem[Che77]{Che77}
    Kuo-Tsai Chen.
    \newblock {\em Iterated path integrals}.
    \newblock {Bulletin of AMS}, 83(5):831--879, 1977.
    
    \bibitem[Dou62]{Dou62}
    Adrien Douady.
    \newblock {\em Vari{\'e}t{\'e}s {\`a} bord anguleux et voisinages tubulaires}.
    \newblock S{\'e}minaire Henri Cartan, (14):1--11, 1961--1962.
    
    \bibitem[ADLH73]{ADLH73}
    Adrien Douady and Letizia H{\'e}rault.
    \newblock {\em Arrondissement des vari{\'e}t{\'e}s {\`a} coins --- appendice {\`a} "corners and arithmetic groups"}.
    \newblock {Commentarii Mathematici Helvetici}, (48):484--491, 1973.
    
    \bibitem[GP74]{GP74}
    Victor Guillemin and Alan Pollack.
    \newblock {\em Differential Topology}.
    \newblock Prentice Hall, 1974.
    
    \bibitem[GIZ18]{GIZ18}
    Serap Gürer and Patrick Iglesias-Zemmour.
    \newblock {\em Differential Forms on Stratified Spaces II}.
    \newblock Bulletin of the Australian Mathematical Society, 98(2):319-330, 2019.
    
    \bibitem[PIZ07]{PIZ07}
    Patrick Iglesias-Zemmour.
    \newblock {\em Dimension in diffeology.}
    \newblock {Indagationes Mathematicae, (18)4:555--560, 2007.}
    
    \bibitem[PIZ13]{PIZ13}
    \newblock{Patrick Iglesias-Zemmour.}
    \newblock{\em Diffeology.}
    \newblock{Mathematical Surveys and Monographs. The American Mathematical Society, (185), USA R.I. 2013.}
    
    \bibitem[Joy10]{Joy10}
    \newblock{Dominic Joyce.}
    \newblock{\em On Manifolds with Corners.}
    Advanced Lectures in Mathematics series (21):225--258, International Press, Boston, 2012.
    
    \bibitem[Lee06]{Lee06}
    John M. Lee.
    \newblock {\em Introduction to Smooth Manifolds}.
    \newblock Graduate Texts in Mathematics. Springer Verlag, New York, 2006.
    
    \bibitem[Sik72]{Sik72}
    Roman Sikorski.
    \newblock \emph{Differential Modules}.
    \newblock Colloquium Mathematicum, (24):49--79, 1972.
    
    \bibitem[Whi43]{Whi43}
    Hassler Whitney.
    \newblock{\em Differentiable even functions.}
    \newblock{Duke Math. Journal, pp. 159--160, vol. 10, 1943.}
    
\end{thebibliography}


%%%%%%%%%%%%%%%%%%%%%%%%%%%%%%%%%%%%%%%%%%%%%%%%%%%%
% Fin du document
%%%%%%%%%%%%%%%%%%%%%%%%%%%%%%%%%%%%%%%%%%%%%%%%%%%%
\end{document}
