%%%%%%%%%%%%%%%%%%%%%%%%%%%%%%%%%%%%%%%%%%%%%%%%%%%%%%%%%%
%%
%%  La trilogie du moment (1995) - Modernized Version
%%
%%  Original Author: Patrick Iglesias-Zemmour
%%  Transmuted for the Diffeology Archives: December 2025
%%
%%%%%%%%%%%%%%%%%%%%%%%%%%%%%%%%%%%%%%%%%%%%%%%%%%%%%%%%%%

\documentclass[12pt,reqno,letterpaper,twoside]{amsart}

%%%%%%%%%%%%%%%%%%%%%%%%%%%%%%%%%%%%%%%%%%%%%%%%%%%%%%%%%%
%% MARK: - Packages
%%%%%%%%%%%%%%%%%%%%%%%%%%%%%%%%%%%%%%%%%%%%%%%%%%%%%%%%%%

\usepackage[french]{babel}
\usepackage[T1]{fontenc}
\usepackage[utf8]{inputenc}
\usepackage{microtype}

% --- Math Packages (Must load these BEFORE mathdesign) ---
\usepackage{amsmath}
\usepackage{amssymb}
\usepackage{amsthm}

% --- Fonts (Must load AFTER amssymb to avoid \circledS conflict) ---
\usepackage[cal=scr,frenchmath,uppercase=upright,greekfamily=didot,greeklowercase=upright,utopia]{mathdesign}

% --- Other Tools ---
\usepackage{pifont} % For \pencil symbol
\usepackage{graphicx}
\usepackage{epstopdf}
\usepackage{tikz-cd}
\usetikzlibrary{calc}
\tikzcdset{arrow style=tikz, diagrams={>={Straight Barb[scale=0.8]}}}
\usepackage[hidelinks]{hyperref}
\usepackage[margin=10pt,font=small,labelfont=bf,labelsep=endash]{caption}

%%%%%%%%%%%%%%%%%%%%%%%%%%%%%%%%%%%%%%%%%%%%%%%%%%%%%%%%%%
%% MARK: - Layout & Styling
%%%%%%%%%%%%%%%%%%%%%%%%%%%%%%%%%%%%%%%%%%%%%%%%%%%%%%%%%%

\linespread{1.1}
\textwidth 14cm
\oddsidemargin 0.76cm
\evensidemargin 0.76cm
\topmargin -0.04cm
\headheight 1cm
\headsep 0.75cm
\textheight 21cm
\footskip 1.25cm
\allowdisplaybreaks

\parindent 0mm
\parskip .5ex plus 2pt


% --- Theorem Styles ---
\newtheoremstyle{piz-style}
  {7pt}{7pt}{\itshape}{0pt}{\bfseries}{.}{6pt}{}

\theoremstyle{piz-style}
\newtheorem{theorem}{Théorème}[section]
\newtheorem{lemma}[theorem]{Lemme}
\newtheorem{proposition}[theorem]{Proposition}
\newtheorem{corollary}[theorem]{Corollaire}

\theoremstyle{definition}
\newtheorem{definition}[theorem]{Définition}
\newtheorem{example}[theorem]{Exemple}
\newtheorem{note}[theorem]{Remarque}

% --- Custom Proof Environment ---
\renewenvironment{proof}{\noindent\textsc{Démonstration.}}{\nolinebreak\hfill $\square$}

%%%%%%%%%%%%%%%%%%%%%%%%%%%%%%%%%%%%%%%%%%%%%%%%%%%%%%%%%%
%% MARK: - Mathematical Macros (Extracted & Standardized)
%%%%%%%%%%%%%%%%%%%%%%%%%%%%%%%%%%%%%%%%%%%%%%%%%%%%%%%%%%

% --- Number Sets ---
\newcommand{\RR}{\mathbf{R}}
\newcommand{\ZZ}{\mathbf{Z}}
\newcommand{\CC}{\mathbf{C}}
\newcommand{\NN}{\mathbf{N}}
\newcommand{\TT}{\mathbf{T}}

% --- Calligraphic letters ---
\newcommand{\cG}{\mathcal{G}}
\newcommand{\cT}{\mathcal{T}}

% --- Operators (Standard) ---
\DeclareMathOperator{\Aut}{Aut}
\DeclareMathOperator{\Diff}{Diff}
\DeclareMathOperator{\Ext}{Ext}
\DeclareMathOperator{\Ham}{Ham}
\DeclareMathOperator{\Hom}{Hom}
\DeclareMathOperator{\id}{\mathbf{1}}
\DeclareMathOperator{\im}{Im}

% --- Operators (Paper Specific) ---
\DeclareMathOperator{\Arc}{Arc}
\DeclareMathOperator{\cl}{cl}     % Class/Projection map
\DeclareMathOperator{\Is}{Is}     % Isotopies
\DeclareMathOperator{\Lac}{Lac}   % Loops (Lacets)
\DeclareMathOperator{\Vect}{Vect} % Vector fields
\DeclareMathOperator{\ad}{ad}
\DeclareMathOperator{\vol}{vol}
\DeclareMathOperator{\source}{s}  % Source map (noted 's' in text)
\DeclareMathOperator{\but}{b}     % But (Target) map (noted 'b' in text)

% --- Geometry & Diffeology ---
\newcommand{\Cinfty}{C^\infty}
\newcommand{\DLie}{\pounds}       % Lie Derivative
\newcommand{\cyl}{{\mathrm{cyl}}}

% --- Helper Macros ---
\newcommand{\xo}{x_\circ}
\renewcommand{\d}{\,d}            % Thin space for differential 'd'
\newcommand{\mor}[1]{\mathrel{\xrightarrow{#1}}}
\newcommand{\scal}[2]{\langle #1,#2\rangle}
\newcommand{\norm}[1]{\left\Vert #1 \right\Vert}
\newcommand{\cf}{{\em cf.\/}}
\newcommand{\ie}{{\em i.e.\/}}
\newcommand{\Imm}{\operatorname{Imm}}
\newcommand{\cad}{c'est-\`a-dire }
\newcommand{\tq}{\mid}
\newcommand{\cst}{{\mathrm{cst.}}}
\newcommand{\vect}[1]{\begin{pmatrix}#1\end{pmatrix}}

%%%%%%%%%%%%%%%%%%%%%%%%%%%%%%%%%%%%%%%%%%%%%%%%%%%%%%%%%%
%% MARK: - Title Information
%%%%%%%%%%%%%%%%%%%%%%%%%%%%%%%%%%%%%%%%%%%%%%%%%%%%%%%%%%

\title{La trilogie du moment}
\author{Patrick Iglesias-Zemmour}
\date{1995}

\begin{document}

\begin{abstract}
\noindent
A toute deux-forme ferm\'ee,
sur une vari\'et\'e connexe,
on associe une famille d'extensions centrales du groupe de ses automorphismes par son tore des p\'eriodes.
On discute quelques propri\'et\'es de cette construction.
\end{abstract}

\maketitle

%%%%%%%%%%%%%%%%%%%%%%%%%%%%%%%%%%%%%%%%%%%%%%%%%%%%%%%%%%
%% MARK: - Content
%%%%%%%%%%%%%%%%%%%%%%%%%%%%%%%%%%%%%%%%%%%%%%%%%%%%%%%%%%

\section*{Introduction}

\noindent
On consid\`ere une deux-forme ferm\'ee $\omega$ sur vari\'et\'e diff\'erentiable connexe $X$.
On appelle {\em tore des p\'eriodes} de la forme $\omega$,
le quotient $T_\omega$ de $\RR$ par son {\em groupe des p\'eriodes} $P_\omega$.
On appelle {\em fibr\'e d'int\'egration} de la forme $\omega$,
tout fibr\'e principal de base $X$ et de groupe structural $T_\omega$,
poss\'edant une connexion de courbure $\omega$.
On montre que l'ensemble des fibr\'es d'int\'egration de la forme $\omega$ est class\'ee,
\`a \'equivalence de fibr\'e principal pr\`es,
par le premier groupe d'extension de $H_1(X,\ZZ)$ \`a coefficients dans $P_\omega$,
c'est \`a dire $\Ext(H_1(X,\ZZ), P_\omega)$.
En particulier,
si le groupe $H_1(X,\ZZ)$ est sans torsion et si $X$ est compacte,
il y a unicit\'e du fibr\'e d'int\'egration.

D\`es que le tore des p\'eriodes $T_\omega$ n'est plus un groupe de Lie,
c'est \`a dire d\`es que $\omega$ n'est plus enti\`ere\footnote{On dit qu'une forme ferm\'ee est enti\`ere si son groupe des p\'eriodes est soit nul soit isomorphe \`a $\ZZ$.},
il est n\'ecessaire d'\'elargir la cat\'egorie des vari\'et\'es diff\'erentiables \`a celle des {\em espaces diff\'erentiables} (voir annexe \ref{AnnED}).
Dans ces conditions,
le fibr\'e d'int\'egration $Y$ n'est plus une vari\'et\'e,
mais les notions d'applications diff\'erentiables,
fibr\'es,
connexions et autres objets g\'eom\'etriques sont bien d\'efinis dans cette cat\'egorie \cite{Iglesias2},
et c'est dans ce sens qu'ils sont utilis\'es ici.

Chaque fibr\'e d'int\'egration $\pi : Y\to X$ peut \^etre muni d'une famille de connexions in\'equivalentes de courbure $\omega$ index\'ee par $H^1(X,\RR)$.
On appelle {\em structure d'int\'egration} de la forme $\omega$ tout couple $(Y,\lambda)$ o\`u $\pi : Y\to X$ est un fibr\'e d'int\'egration et $\lambda$ un forme de connexion de courbure $\omega$.
L'ensemble des structures d'int\'egrations de la forme $\omega$ est donc class\'ee par $H^1(X,T_\omega)$.

Soit $\Diff(Y,\lambda)$ le groupe des automorphismes de la structure d'int\'egration $(Y,\lambda)$,
c'est \`a dire le groupe des diff\'eomorphismes de $Y$,
\'equivariants sous l'action de $T_\omega$,
et pr\'eservant la forme de connexion $\lambda$.
Il s'envoie,
par homomorphisme,
dans le groupe $\Diff(X,\omega)$ des diff\'eomorphismes de $X$ qui pr\'eservent la forme $\omega$.
Son image $\Ham(Y,\lambda)\subset \Diff(X,\omega)$ est appel\'e groupe des {\em diff\'eomorphismes hamiltoniens} de $\omega$,
associ\'es \`a $(Y,\lambda)$.
Son noyau est r\'eduit au tore des p\'eriodes $T_\omega$.
Le groupe $\Diff(Y,\lambda)$ est une extension centrale de $\Ham(Y,\lambda)$ par $T_\omega$.

Lorsque $H_1(X,\ZZ)=0$,
on a \'egalit\'e des composantes connexes $\Ham^\circ(Y,\lambda) = \Diff^\circ(X,\omega)$.
L'extension centrale de $\Diff^\circ(X,\omega)$ par $\Diff^\circ(Y,\lambda)$ est repr\'esent\'ee par un cocycle de groupe $K_\omega\in H^2(\Diff^\circ(X,\omega),T_\omega)$ dont on donne une interpr\'etation g\'eom\'etrique simple,
en termes d'aire de triangle,
qui justifie le nom qu'on lui a donn\'e de {\em cocycle triangulaire}.

Le moment $J$ du groupe $\Ham(Y,\lambda)$ est une application de $X$ dans le dual $\Vect^*(X,\omega)$ de l'alg\`ebre de Lie $\Vect(X,\omega)$ des champs de vecteurs qui pr\'eservent la $2$-forme $\omega$.
Son {\em d\'efaut d'\'equivariance},
sous l'action de $\Ham(Y,\lambda)$,
est un $1$-cocycle $\Theta$ de $\Ham(Y,\lambda)$ dans $\Vect^*(X,\omega)$.
Il est reli\'e au cocycle triangulaire:
c'est l'image de $K_\omega$ par un certain {\em morphisme d\'eriv\'ee}.

L'aspect infinit\'esimal de cette construction fait intervenir un $2$-cocycle d'alg\`ebre de Lie $k_\omega\in H^2(\Vect(X,\omega),\RR)$ obtenu par d\'erivation du cocycle triangulaire $K_\omega$.
Ces trois objets $K_\omega$,
$\Theta$ et $k_\omega$ constituent ce qu'on a envie d'appeler la {\em trilogie du moment}.

On illustre cette construction par quelques exemples.
On montre en particulier comment on peut retrouver le cocycle de Bott-Thurston \cite{Bott1},
qui d\'efinit la seule extension centrale non triviale de $\Diff^\circ (S^1)$ par $\RR$ \cite{GelfandFuchs},
en consid\'erant l'espace des immersions du cercle $S^1$ dans $\RR^2$.
Dans ce cas,
le d\'efaut d'\'equivariance du moment est la d\'eriv\'ee Schwarzienne des diff\'eomorphismes du cercle \cite{Kirillov4},
et le niveau infinit\'esimal fait naturellement appara\^{\i}tre le cocycle de Gelfand-Fuchs \cite{GelfandFuchs}.

Tout cela g\'en\'eralise,
au cas d'une $2$-forme ferm\'ee quelconque $\omega$,
une construction relativement bien connue lorsque $\omega$ est enti\`ere.
Cette construction,
utilis\'ee notamment en quantification g\'eom\'etrique \cite{Souriau5} et dans la m\'ethode des orbites \cite{Kirillov1},
est \`a l'origine de ce travail.
Mais soulignons toutefois que le fibr\'e d'int\'egration est ici unique d\`es que $H^1(X,\ZZ)=0$,
ce qui n'est pas le cas en quantification g\'eom\'etrique.
C'est parce que nous utilisons ici le tore des p\'eriodes $T_\omega$,
bien d\'efini comme le quotient $\RR/P_\omega$ et non un groupe donn\'e \`a l'avance (en l'occurence $S^1$) d\'efini \`a isomorphisme pr\`es.

Il semble naturel que cette construction trouve sa place en cohomologie \'equivariante,
nous n'avons pas d\'evelopp\'e cet aspect.

Au moment d'envoyer la deuxi\`eme version de cet article \`a la revue,
j'ai eu connaissance du livre de Jean-Luc Brylinski \cite{Brylinski2} dans lequel certaines constructions analogues sont pr\'esent\'ees.

\section{Int\'egration d'une 2-forme ferm\'ee}

\noindent
Soit $\omega$ une $2$-forme ferm\'ee sur une vari\'et\'e diff\'erentiable connexe $X$.
Rappelons que le {\em groupe des p\'eriodes} de la forme $\omega$ est le sous-groupe additif de $\RR$ d\'efini par:
%
\begin{equation}
P_\omega = \left\{\int_\sigma \omega \mid \sigma \in H_2(X,\ZZ) \right\}.
\end{equation}

\begin{definition}
On appellera {\em tore des p\'eriodes} de la forme $\omega$ le quotient $T_\omega$ de $\RR$ par son groupe des p\'eriodes $P_\omega$.
On notera $\cl_\omega$ la projection de $\RR$ sur le tore $T_\omega$:
%
\begin{equation}
0\longrightarrow P_\omega \longrightarrow \RR \mor{\cl_\omega} T_\omega \longrightarrow 1.
\end{equation}
\end{definition}

Lorsque la forme $\omega$ est enti\`ere,
$T_\omega$ est un groupe de Lie ($\RR$ ou $S^1$,
suivant que $\omega$ est exacte ou non),
sinon il sera muni de sa structure de groupe diff\'erentiable (voir annexe \ref{AnnED}).
On notera additivement sa loi de groupe,
m\^eme si lorsqu'il est \'egal \`a $S^1$,
l'usage veut qu'elle soit not\'ee multiplicativement.
Rappelons qu'une forme ferm\'ee est dite enti\`ere lorsque son groupe des p\'eriodes est discret,
\ie trivial ou isomorphe \`a $\ZZ$.

Soit $\theta$ la $1$-forme diff\'erentielle (de Maurer-Cartan) d\'efinie sur $T_\omega$ par:
%
\begin{equation}
\cl_\omega ^*\theta = dt.
\end{equation}
%
On notera $\Arc(X,\xo)$ l'espace des arcs diff\'erentiables de $X$ point\'es en $\xo\in X$,
et $\Lac(X,\xo)$ le sous-espace de ses lacets.
Ces espaces sont munis de leurs structures diff\'erentiables standards.

Deux arcs $\gamma$ et $\gamma'$,
sont {\em homologues} si leur diff\'erence borde une $2$-cha\^{\i}ne singuli\`ere,
on notera :
%
\begin{equation}
\gamma \sim \gamma' \quad \Leftrightarrow \quad \exists \sigma \quad \partial \sigma = \gamma' - \gamma.
\end{equation}
%
Le quotient des arcs point\'es par la relation d'homologie sera not\'e $\hat X$,
c'est le {\em rev\^etement d'homologie} de $X$.
Son groupe structural,
quotient de $\Lac(X,\xo)$ par la relation d'homologie,
est isomorphe au groupe $H_1(X,\ZZ)$.

\begin{proposition}\label{prop1}
Soit $\omega$ une $2$-forme ferm\'ee sur une vari\'et\'e diff\'erentiable connexe $X$,
telle que $H_1(X,\ZZ) =0$.
Il existe un fibr\'e principal $\pi : Y \to X$,
de base $X$,
de groupe structural $T_\omega$,
muni d'une forme de connexion $\lambda$ de courbure $\omega$.
La structure $(Y,\lambda)$ est unique \`a \'equivalence pr\'es.
\end{proposition}

\begin{proof}
D\'emontrons d'abord l'existence.
Relevons la relation d'\'equivalence,
d\'efinie plus haut sur les arcs,
au produit $\Arc(X,\xo)\times T_\omega$:
%
\begin{equation}\label{defrelequiv}
(\gamma,z) \sim (\gamma',z') \quad \Leftrightarrow \quad \gamma \sim \gamma' \mbox{ et } z' = z +  \cl_\omega \int_\sigma \omega \quad \mbox{o\`u} \quad \partial \sigma =\gamma'-\gamma.
\end{equation}
%
Il est facile de v\'erifier que c'est bien une relation d'\'equivalence.
Soit $Y$ le quotient de $\Arc(X,\xo)\times T_\omega$ par cette relation d'\'equivalence.
Puisque $H_1(X,\ZZ)=0$,
$\hat X=X$ et $Y$ est fibr\'e principalement sur $X$,
de groupe structural $T_\omega$.
Soit $K$ l'op\'erateur de cha\^{\i}ne-homotopie tel qu'il est d\'efini dans l'annexe \ref{AnnOCH},
et $\alpha$ la $1$-forme d\'efinie sur le produit $\Arc(X,\xo)\times T_\omega$ par:
%
\begin{equation}\label{defalpha}
\alpha = K\omega \oplus \theta.
\end{equation}
%
C'est \'evidemment une forme de connexion pour l'action naturelle de $T_\omega$,
de courbure $\but^*\omega$,
puisque $K\omega$ est une primitive de $\but^*\omega$,
o\`u $\but$ d\'esigne l'application {\em but} de $\Arc(X,\xo)$ sur $X$ qui \`a $\gamma$ associe $\gamma(1)$.
En appliquant la proposition \ref{propformquot},
le passage de la forme $\alpha$ au quotient $Y$ est assur\'e par le lemme suivant:

\begin{lemma}
Soit $p$ la projection de $\Arc(X,\xo)\times T_\omega$ sur son quotient $Y$.
Si deux param\'etrages diff\'erentiables ({\em plaques}) $P$ et $P'$ de $\Arc(X,\xo)\times T_\omega$ v\'erifient $p\circ P=p\circ P'$,
alors $P^*\alpha = P'^*\alpha$.
\end{lemma}

\begin{proof}
Soit $U$ le domaine des plaques $P$ et $P'$ et $r\in U$,
notons $P(r)=(P_X(r),P_T(r))$ o\`u $P_X$ est une plaque de $\Arc(X,\xo)$ et $P_T$ une plaque de $T_\omega$.
On ne perd rien en g\'en\'eralit\'e en supposant que $P_T$ se rel\`eve globalement sur $\RR$ en une plaque $Q\in \Cinfty(U,\RR)$ (\ie $P_T=\cl_\omega\circ Q$).
On a donc $P^*(K\omega \oplus \theta) = P_X^*(K\omega) + P_T^*\theta$,
\cad $P_X^*(K\omega) + dQ$.
Notons $\gamma =P(r)$ et $\delta \gamma = D(P)(r)(\delta r)$ o\`u $\delta r$ est un vecteur tangent \`a $r\in U$,
par d\'efinition (\cf annexe \ref{AnnOCH}):
$$P_X^*(K\omega)_r(\delta r)= \int_0^1\omega_{\gamma(t)}(\dot\gamma(t),\delta \gamma(t))\ dt.$$

D'autre part,
$p\circ P=p\circ P'$ implique l'existence,
pour tout $r\in U$ d'une $2$-cha\^{\i}ne $\sigma_r$ telle que $\partial \sigma_r= P'(r)-P(r)$.
On peut v\'erifier,
en restreignant si n\'ecessaire le domaine $U$,
que $\sigma_r$ peut \^etre choisie diff\'erentiablement.
On a alors,
$$\cl_\omega(Q'(r)) = \cl_\omega(Q(r)) + \cl_\omega \int_{\sigma_r}\omega,$$
\cad $Q'(r)=Q(r) + \int_{\sigma_r}\omega +l(r)$,
o\`u $l:U\to P_\omega$ est diff\'erentiable.
Comme $P\omega$ est {\em diff\'erentiablement discret}:
toute application diff\'erentiable de $U$ dans $\RR$ \`a valeur dans $P_\omega$ est localement constante,
on d\'eduit que $l$ est constante et donc que :
$$dQ'(r)=dQ(r)+d\left[\int_{\sigma_r} \omega\right].$$
En utilisant le th\'eor\`eme de Stokes (voir annexe \ref{AnnStokes}),
qui peut s'\'ecrire aussi
$$d\left[\int_{\sigma} \omega\right] = \int_{\sigma} d\omega (\cdot) + \int_{\partial \sigma} \omega(\cdot),$$
et l'expression pr\'ec\'edente de $P_X^*(K\omega)$ on d\'eduit finalement :
$$dQ'-dQ= -P_X^*(K\omega) + {P'_X}^*(K\omega) \quad \Rightarrow \quad P^*[K\omega \oplus \theta] = {P'}^*[K\omega \oplus \theta].$$
C'est ce qu'il fallait d\'emontrer.
\end{proof}

D\'emontrons l'unicit\'e de cette construction.
Soit $(Y',\lambda')$ une autre structure r\'epondant aux hypoth\`eses.
L'image r\'eciproque de $Y$ par $\but : \Arc(X,\xo) \to X$ est triviale,
car $\Arc(X,\xo)$ est contractile et muni d'une connexion (cette propri\'et\'e remplace la paracompacit\'e dans le cas des vari\'et\'es).
L'image de la section nulle au dessus de $\Arc(X,\xo)$ d\'efinit une application $\psi$ de $\Arc(X,\xo)$ dans $Y$.
On v\'erifie alors que la projection $(\gamma, z)\mapsto z\psi(\gamma)$ r\'ealise le quotient de $\Arc(X,\xo)\times T_\omega$ par la relation d'\'equivalence d\'efinie plus haut et donc que $Y$ et $Y'$ sont \'equivalents en tant que fibr\'es principaux.
Il est toujours possible,
d'autre part,
de choisir un isomorphisme entre $\but^*(Y')$ et le produit $\Arc(X,\xo)\times T_\omega$ de telle sorte que l'image r\'eciproque de $\lambda'$ co\"{\i}ncide avec la forme $\alpha$.
\end{proof}

\begin{note}
Le relev\'e \`a $\Arc(X,\xo)$ de la relation d'\'equivalence sur les arcs d\'efinit un {\em cocycle} qu'on pourrait appeler le {\em cocycle d'arcs} associ\'e \`a $\omega$.
C'est la fonction $f_\omega$ d\'efinie sur les couples d'arcs homologues par:
%
\begin{equation}\label{cocarc}
f_\omega(\gamma,\gamma') = \cl_\omega \int_\sigma \omega \quad \mbox{o\`u} \quad \partial \sigma =\gamma'-\gamma.
\end{equation}
%
L'int\'egrale d\'epend de la cha\^{\i}ne bordant $\gamma'-\gamma$,
mais pas sa classe dans $T_\omega$.
Le choix de cette terminologie r\'esulte de l'identit\'e suivante,
v\'erifi\'ee pour tout triplet d'arcs homologues:
%
\begin{equation}
f_\omega(\gamma,\gamma') +  f_\omega(\gamma',\gamma'') + f_\omega(\gamma'',\gamma) =0.
\end{equation}
%
La fonction $f_\omega$ est un {\em cobord} s'il existe $\mu: \Arc(X,\xo) \to T_\omega$ diff\'erentiable,
telle que $f_\omega(\gamma,\gamma') = \mu(\gamma')-\mu(\gamma)$.
Ce qui se produit lorsque $\omega =d\alpha$,
auquel cas:
$\mu (\gamma) = \int_\gamma \alpha$.
\end{note}

Nous utiliserons le lemme pr\'ec\'edent pour d\'emontrer le th\'eor\`eme plus g\'en\'eral suivant:

\begin{theorem}
Pour toute $2$-forme ferm\'ee $\omega$ d\'efinie sur une vari\'et\'e diff\'erentiable connexe $X$,
il existe un fibr\'e principal $\pi : Y \to X$ de groupe structural $T_\omega$ muni d'une forme de connexion $\lambda $ de courbure $\omega$.
Un tel fibr\'e sera appel\'e {\em fibr\'e d'int\'egration} de la $2$-forme $\omega$.
Les fibr\'es d'int\'egration de la $2$-forme $\omega$ sont class\'es,
\`a \'equivalence de fibr\'e principal pr\`es,
par le premier groupe d'extension $\Ext(H_1(X,\ZZ),P_\omega)$.
\end{theorem}

\begin{proof}
Comme dans la d\'emonstration pr\'ec\'edente nous consid\'e\-rons la relation d'\'equivalence d\'efinie par (\ref{defrelequiv}),
sur le produit $\Arc(X,\xo)\times T_\omega$,
ainsi que la $1$-forme $\alpha$ (d\'efinition  \ref{defalpha}).
L'espace quotient $\hat Y$ est alors fibr\'e sur le rev\^etement d'homologie $\hat X$ de $X$,
de groupe $T_\omega$.
Il est muni d'une connexion $\hat \lambda$ de courbure  $\hat \omega$,
image r\'eciproque de $\omega$ par la projection naturelle $\hat X \to X$.
En vertu de la proposition pr\'ec\'edente la structure $(\hat Y, \hat \lambda)$,
qui int\`egre $\hat \omega$,
est unique \`a \'equivalence pr\`es.

Consid\'erons une section $s$ de $H_1(X,\ZZ)$ dans l'espace des lacets $\Lac(X,\xo)$.
Soit $\phi$ le $2$-cocycle de groupe d\'efini sur $H_1(X,\ZZ)$,
\`a valeurs dans $T_\omega$ :
%
\begin{equation}
\phi (h,h') = f_\omega (s(h+ h'), s(h)+ s(h')).
\end{equation}
%
Ce cocycle \'etant sym\'etrique,
il d\'efinit une extension ab\'elienne $\Gamma$ de $H_1(X,\ZZ)$ par $T_\omega$.
Cette extension agit naturellement sur $\Arc(X,\xo)\times T_\omega$ par :
%
\begin{equation}
(h,\tau)[\gamma,z] = [s(h)+ \gamma, z+ \tau],
\end{equation}
%
o\`u $l+ \gamma$ d\'esigne la juxtaposition du lacet $l$ avec l'arc $\gamma$,
et les crochets:
les classes d'\'equivalences.
Mais le groupe $T_\omega$ \'etant divisible,
l'extension est triviale (voir par exemple \cite{KargMerz}) et le groupe $\Gamma$ est isomorphe au produit direct de $H_1(X,\ZZ)$ par $T_\omega$.
Tout isomorphisme permet de d\'efinir une action de $H_1(X,\ZZ)$ sur $\hat Y$,
dont le quotient $Y = \hat Y/H_1(X,\ZZ)$ est un fibr\'e principal de groupe structural $T_\omega$ sur $X$.
On v\'erifie alors que la forme de connexion $\hat \lambda$,
\'etant invariante par ces actions de $H_1(X,\ZZ)$,
passe sur $Y$ en une forme de connexion $\lambda$ de courbure $\omega$.
Le fibr\'e $Y\to X$,
muni de la forme de connexion $\lambda$ est construit par quotients successifs:
\[
\begin{tikzcd}
\Arc(X,\xo)\times T_\omega \arrow[r] \arrow[d] & \hat Y \arrow[r] \arrow[d] & Y \arrow[d] \\
\Arc(X,\xo) \arrow[r] & \hat X \arrow[r] & X
\end{tikzcd}
\]
Deux tels isomorphismes entre $\Gamma$ et $H^1(X,\ZZ)$ ne diff\`erent que d'un \'el\'ement de $\Hom(H_1(X,\ZZ), T_\omega)$ c'est \`a dire d'un \'el\'ement de $H^1(X,T_\omega)$.
Gr\^ace \`a l'unicit\'e de la structure $(\hat Y, \hat \lambda)$ nous avons trouv\'e de cette fa\c{c}on,
toutes les structures qui int\`egrent $\omega$.

Soit alors $\pi : Y \to X$ et $\pi' : Y' \to X$ deux de ces fibr\'es obtenus par quotient de $\hat Y$ \`a partir des actions $\rho$ et $\rho'$ de $H_1(X,\ZZ)$.
Ces actions diff\`erent d'un \'el\'ement $r\in \Hom(H_1(X,\ZZ), T_\omega)$.
Il y a donc une surjection naturelle $\sigma$ de l'espace des classes de fibr\'es d'int\'egration de $\omega$ sur $\Hom(H_1(X,\ZZ), T_\omega)$.
On peut facilement v\'erifier,
en prenant les images r\'eciproques de $Y$ et $Y'$ par la projection naturelle de $\hat X$ sur $X$,
qu'ils sont \'equivalent si et seulement si il existe une application $\zeta : \hat X \to T_\omega$ telle que $r(k) = \zeta (k\hat x) - \zeta (\hat x)$,
pour tout $k\in H_1(X,\ZZ)$.
La fonction $\zeta$ d\'efinit alors une $1$-forme ferm\'ee $\varepsilon$ sur $X$,
par $\zeta^*\theta = \pi^*\varepsilon$,
qui d\'efinit \`a son tour (par int\'egration sur le rev\^etement d'homologie) une fonction $F$ de $\hat X$ dans $\RR$,
telle que $\zeta^*\theta =dF$,
on en d\'eduit $\zeta = \cl_\omega \circ F$.
Autre\-ment dit:
le noyau de la surjection $\sigma$ est le sous groupe des homomorphisme de $H_1(X,\ZZ)$ dans $T_\omega$ povenant d'un homomorphisme de $H_1(X,\ZZ)$ dans $\RR$.
Comme $\RR$ est divisible,
on conclut,
en utilisant la suite exacte du foncteur $\Ext$ \cite{MacLane2},
que le co-noyau de la fl\`eche naturelle $\Hom(H_1(X,\ZZ),\RR) \to \Hom(H_1(X,\ZZ),T_\omega)$ est exactement le premier groupe d'extension de $H_1(X,\ZZ)$ dans le groupe des p\'eriodes $P_\omega$,
c'est \`a dire:
$\Ext(H_1(X,\ZZ),P_\omega)$.
\end{proof}

\begin{note}\label{remsanstorsion}
Si le groupe d'homologie $H_1(X,\ZZ)$ est sans torsion et si $X$ est compacte,
Il y a unicit\'e du fibr\'e d'int\'egration de la $2$-forme $\omega$.
En effet tout homomorphisme de $H_1(X,\ZZ)$ dans $T_\omega$ est alors la projection d'un homomorphisme \`a valeurs r\'eelles.
\end{note}

\begin{definition}
On appellera {\em structure d'int\'egration} de la $2$-forme ferm\'ee $\omega$,
tout couple $(Y,\lambda)$ o\`u $Y$ est un fibr\'e d'int\'egration de $\omega$ et $\lambda$ une connexion de courbure $\omega$.
\end{definition}

\begin{note}
Etant donn\'e un fibr\'e d'int\'egration $\pi : Y \to X$,
les formes de connexions in\'equivalentes sur $Y$ de courbure $\omega$ sont class\'es par $H^1(X,\RR)$.
L'ensemble des structures d'int\'egrations est donc class\'ee par $H^1(X,T_\omega)$.
Ce que nous dit le th\'eor\`eme pr\'ec\'edent est que cette classification se scinde gr\^ace \`a la suite exacte:
%
\begin{eqnarray}
0 & \to & \Hom(H_1(X,\ZZ),P_\omega) \to \Hom(H_1(X,\ZZ),\RR) \to \nonumber \\
    & \to &\Hom(H_1(X,\ZZ),T_\omega) \to \Ext(H_1(X,\ZZ),P_\omega) \to 0
\end{eqnarray}
%
Le groupe $\Ext(H_1(X,\ZZ),P_\omega)$ classe les fibr\'es d'int\'egration,
et le groupe dual $\Hom(H_1(X,\ZZ), \RR)$,
c'est \`a dire $H^1(X,\RR)$,
classe les formes de connexions.
\end{note}

\begin{note}
Parce que les cocycles d'arcs de deux $2$-formes ferm\'ees cohomologues sont cohomologues,
leurs fibr\'es d'int\'egration sont \'equivalents.
Autre\-ment dit,
la classe de cohomologie de la forme $\omega$ est,
en un certain sens,
la premi\`ere «classe de Chern» de ces fibr\'es d'int\'egration.
\end{note}

\begin{note}
On peut \^etre surpris que des fibr\'es d'int\'egration existent pour toute forme ferm\'ee,
m\^eme multiples les unes des autres par un r\'eel non entier.
Mais,
c'est un exercice de montrer que pour $\omega = s\omega_0$,
o\`u $s\in \RR$,
le fibr\'e d'int\'egration de $\omega_0$ est \'equivalent comme fibr\'e \`a celui de $\omega$,
mais attention :
les groupe $T_\omega$ et $T_{\omega _0}$ ne sont pas identiques,
seulement isomorphes.
La diff\'erence avec la situation classique des $2$-formes enti\`eres est qu'il n'y a pas d'identification possible,
{\em a priori},
des tores des p\'eriodes pour les $2$-formes ferm\'ees quelconques.
Une telle identification,
dans le cas entier,
consiste \`a fixer \`a l'avance la longueur de la p\'eriode.
\end{note}

\section{Extensions centrales des diff\'eo\-morphis\-mes ha\-mil\-to\-niens}

\noindent
Soit $(Y,\lambda)$ une structure d'int\'egration de la $2$-forme ferm\'ee $\omega$ d\'efinie sur $X$.
On note $\Diff(Y,\lambda)$ le groupe des automorphismes du couple $(Y,\lambda)$ :
%
\begin{equation}
\Diff(Y,\lambda)  =  \{\psi \in \Diff(Y)\mid \psi^*\lambda =\lambda \mbox{ et } \forall z\in T_\omega : \psi\cdot z = z\cdot\psi \}.
\end{equation}
%
Le groupe $\Diff(Y,\lambda)$ s'envoie par homomorphisme dans le groupe $\Diff(X,\omega)$ des diff\'eomorphismes de $X$ qui pr\'eservent la $2$-forme $\omega$.
On notera $\Ham(Y,\lambda)$ son image,
et ses \'el\'ements seront appel\'es {\em diff\'eomorphismes hamiltoniens} de la structure $(Y,\lambda)$.
Le groupe $\Diff(Y,\lambda)$ contient \'evidemment le groupe $T_\omega$ comme sous groupe central,
plus pr\'ecisemment:

\begin{proposition}
Le noyau de l'homomorphisme naturel de $\Diff(Y,\lambda)$ dans $\Diff(X,\omega)$ est r\'eduit au tore des p\'eriodes $T_\omega$.
\end{proposition}
%
\begin{proof}
Soit $\psi$ un automorphisme de $(Y,\lambda)$ se projetant sur l'identit\'e de $X$,
il existe une application diff\'erentiable $\zeta :Y\to T_\omega$ telle que $\psi(y) = \zeta(y)\cdot y$.
Par \'equivariance de $\psi$ on d\'eduit que $\zeta$ ne d\'epend que de $x=\pi(y)$ et la condition $\psi^*\lambda =\lambda$ implique $\zeta_X^*\theta = 0$ c'est \`a dire $\zeta_X = \cst$,
o\`u $\zeta =\zeta _X\circ \pi$.
\end{proof}

Ainsi,
chaque structure d'int\'egration $(Y,\lambda)$ donne lieu \`a une extension centrale,
par le tore des p\'eriodes $T_\omega$,
de son sous-groupe des diff\'eomorphismes hamiltoniens.
%
\begin{equation}
0 \to T_\omega \to \Diff(Y,\lambda) \to \Ham(Y,\lambda) \to 1.
\end{equation}
%
\begin{definition}
Lorsque la structure d'int\'egration est unique,
ce qui est le cas si $H_1(X,\ZZ)=0$,
cette extension centrale sera appel\'ee {\em extension centrale r\'eguli\`ere} du groupe des automorphismes de la $2$-forme $\omega$.
\end{definition}

\section{Le cocycle triangulaire}

\noindent
A partir de ce paragraphe,
nous supposerons pour simplifier que $H_1(X,\ZZ)=0$,
nous avons vu que dans ce cas la structure d'int\'egration $(Y,\lambda)$ est unique.
De plus,
nous nous restreindrons aux composantes connexes des groupes $\Diff(Y,\lambda)$,
$\Diff(X,\omega)$ et $\Ham(X,\omega)$.

\begin{note}
Notons d\'ej\`a que dans ce cas:
$\Ham^\circ(X,\omega)= \Diff^\circ(X,\omega)$.
En effet,
soit $\phi \in \Diff^\circ(X,\omega)$ et $\Phi$ une isotopie de $\phi$,
d\'esignons par $c_\Phi$ la trajectoire sous l'action de $\Phi$ du point $\xo$:
%
\begin{equation}
c_\Phi : t \mapsto \Phi_t(\xo).
\end{equation}
%
Alors $\phi$ se rel\`eve en un diff\'eomorphisme $\hat \phi$ de $Y$:
%
\begin{equation}
\hat \phi[\gamma,z]= [c_\Phi + \phi \circ \gamma, z].
\end{equation}
%
Il est facile de v\'erifier que c'est un automorphisme de $(Y,\lambda)$.
\end{note}

Nous allons,
maintenant,
d\'ecrire la construction g\'eom\'etrique d'un cocycle $K_\omega$ de l'extension centrale r\'eguli\`ere de $\Diff^\circ(X,\omega)$.
Pour cela,
nous commencerons par construire une certaine extension centrale du groupe des isotopies de $\Diff(X,\omega)$ qui donnera,
par quotient,
l'extension recherch\'ee.
Soit:
%
\begin{equation}
\Is(X,\omega) = \Arc(\Diff(X,\omega), \id_X),
\end{equation}
%
nous noterons $*$ sa loi de groupe.
Elle est d\'efinie,
pour tout couple d'isotopies $(\Phi,\Psi)$ de $\Diff(X,\omega)$,
par:
%
\begin{equation}
\Phi *\Psi = [t\mapsto \Phi_t\circ \Psi_t],
\end{equation}
%
A $\Phi$ et $\Psi$ associons le $2$-simplex $\sigma(\Phi,\Psi)$ de $X$ d\'efinie par :
%
\begin{equation}
\sigma(\Phi,\Psi) : (t,s) \mapsto \Phi_t\circ \Psi_s(\xo) \quad \mbox{avec} \quad 0\leq s\leq t\leq 1,
\end{equation}
%
dont le bord est donn\'e par :
%
\begin{equation}\label{bord1}
\partial \sigma (\Phi ,\Psi ) = c_\Phi +\phi \circ c_\Psi -c_{\Phi *\Psi},
\end{equation}
%
o\`u $\phi =\Phi _1$ (idem pour $\psi =\Psi _1$).

\begin{proposition}\label{propcocycle}
L'application $\tilde K_\omega$ de $\Is(X,\omega)\times \Is(X,\omega)$ dans $T_\omega$,
d\'efinie par:
%
\begin{equation}
\tilde K_\omega (\Phi ,\Psi ) = \cl_\omega \int_{\sigma (\Phi ,\Psi )}\omega,
\end{equation}
%
est un $2$-cocycle de groupe.
L'extension centrale de $\Is(X,\omega)$ par $T_\omega$ qui lui est associ\'ee par l'op\'eration:
%
\begin{equation}
(\Phi,\tau)*(\Psi,\mu) = (\Phi *\Psi , \tau +\mu -\tilde K_\omega (\Phi ,\Psi))
\end{equation}
sera not\'ee $\Is(X,\omega)\times_{\tilde K_\omega }T_\omega$.
\end{proposition}

\begin{proof}
Soit $\Phi$,
$\Psi$ et $\Xi$ trois isotopies de $\Diff(X,\omega)$,
il faut v\'erifier que :
%
$$
\tilde K_\omega(\Phi*\Psi,\Xi) + \tilde K_\omega(\Phi,\Psi) = \tilde K_\omega(\Phi,\Psi*\Xi) + \tilde K_\omega(\Psi,\Xi).
$$
%
Ce qui est \'equivalent,
en posant $$\sigma(\Phi,\Psi,\Xi) = \sigma(\Phi*\Psi,\Xi) + \sigma(\Phi,\Psi)-\sigma(\Phi,\Psi*\Xi) - \sigma(\Psi,\Xi),$$ \`a:
%
$$
\cl_\omega \int_{\sigma(\Phi,\Psi,\Xi)} = 0
$$
%
Or,
on peut v\'erifier facilement que $\sigma(\Phi,\Psi,\Xi)$ et $\phi \circ \sigma (\Psi,\Xi)-\sigma(\Psi,\Xi)$ sont homologues,
il suffit donc de remarquer que:
%
$$
\int_{\phi \circ \sigma(\Psi,\Xi)}\omega = \int_{\sigma (\Psi,\Xi)}\omega,
$$
%
ce qui est imm\'ediat puisque $\phi^*\omega =\omega$.
\end{proof}

Le groupe $\Is(X,\omega)$ agit sur $\Arc(X)$ de fa\c{c}on naturelle,
pour toute isotopie $\Phi$ et tout arc $\gamma$ on notera:
 %
\begin{equation}
\Phi *\gamma  : t\mapsto \Phi_t(\gamma(t)).
\end{equation}
%
Evidemment,
si $\gamma \in \Arc(X,\xo)$ alors $\Phi*\gamma\in \Arc(X,\xo)$.
Soit $\sigma(\Phi,\gamma)$ le $2$-simplex de $X$ d\'efinie par:
%
\begin{equation}
\sigma(\Phi,\gamma) : (t,s) \mapsto \Phi_t(\gamma(s)) \quad \mbox{avec} \quad 0\leq s\leq t\leq 1,
\end{equation}
%
dont le bord est donn\'e par:
%
\begin{equation}
\partial \sigma(\Phi,\gamma) =  c_\Phi + \phi\circ\gamma - \Phi*\gamma.
\end{equation}
%
L'action de $\Is(X,\omega)$ sur $\Arc(X,\xo)$ s'\'etend en une action de $\Is(X,\omega)\times_{\tilde K_\omega }T_\omega$ sur le produit $\Arc(X,\xo)\times T_\omega$:

\begin{proposition}
L'application qui \`a tout couple $(\Phi,\tau)\in \Is(X,\omega)\times_{\tilde K_\omega} T_\omega$ et \`a tout couple $(\gamma,z)\in \Arc(X,\xo)\times T_\omega$ associe :
%
\begin{equation}
(\Phi,\tau)(\gamma,z)=\Big(\Phi*\gamma,z+\tau-\cl_\omega\int_{\sigma(\Phi,\gamma)}\omega\Big),
\end{equation}
%
est une action de groupe qui pr\'eserve la forme $\alpha = K\omega \oplus \theta$.
\end{proposition}

\begin{proof}
Soit $(\Phi,\tau)$ et $(\Psi,\mu)$ deux \'el\'ements de $\Is(X,\omega)\times_{\tilde K_\omega} T_\omega$.
Soit $(\gamma,z)$ un \'el\'ement de $\Arc(X,\xo)\times T_\omega$.
On a:
%
\begin{eqnarray*}
(\Phi,\tau)[(\Psi,\mu)(\gamma,z)] & = & (\Phi,\tau)(\Psi*\gamma, z+\mu - \scal(\Psi,\gamma)\\
 & = & \Big((\Phi *\Psi)*\gamma , z + \tau +\mu -\scal(\Phi,\Psi *\gamma) - \scal(\Psi,\gamma)\Big),
\end{eqnarray*}
%
o\`u on a not\'e pour simplifier $\scal(\Psi,\gamma)$ \`a la place de $\cl_\omega\int_{\sigma(\Psi,\gamma)}\omega$;
d'autre part:
%
\begin{eqnarray*}
[(\Phi,\tau)*(\Psi,\mu)](\gamma,z) & = & \Big(\Phi *\Psi, \tau +\mu - \tilde K_\omega(\Phi ,\Psi )\Big)(\gamma,z)\\
& = &  \Big((\Phi *\Psi)*\gamma ,  z + \tau +\mu - \tilde K_\omega(\Phi,\Psi) - \scal(\Phi*\Psi,\gamma)\Big).
\end{eqnarray*}
%
Il faut donc montrer que :
%
\begin{equation}\label{cidessus}
\tilde K_\omega(\Phi,\Psi) = \cl_\omega\int_{\sigma(\Phi,\Psi*\gamma)}\omega - \cl_\omega\int_{\sigma(\Phi*\Psi,\gamma)}\omega  + \cl_\omega\int_{\sigma(\Psi,\gamma)}\omega.
\end{equation}
%
Ce qui revient \`a v\'erifier que le deuxi\`eme membre de cette \'egalit\'e est ind\'ependant de $\gamma$,
puisque  $\sigma(\Phi,\Psi) = \sigma(\Phi,\Psi*\hat\xo)$,
o\`u $\hat\xo$ est l'arc constant $[t\mapsto \xo]$.
Or,
un calcul simple montre que les simplexes $\sigma(\Phi,\Psi*\gamma) + \sigma(\Psi,\gamma) - \sigma(\Phi*\Psi,\gamma)$ et  $\sigma(\Phi,\Psi)$ sont homologues,
donc l'\'egalit\'e a lieu.

Montrons maintenant que la forme $\alpha$ est pr\'eserv\'ee,
\cad  pour tout $(\Phi,\tau)\in \Is(X,\omega)\times_{\tilde K_\omega} T_\omega$:
$(\Phi,\tau)^*\alpha =\alpha$.
Soit $(\gamma,z)\in \Arc(X,\xo)$,
nous noterons $(\gamma^*,z^*)=(\Phi,\tau)(\gamma,z)$,
soit $(\delta \gamma, \delta z)$ un vecteur tangent\footnote{Une justification pr\'ecise de cette terminologie et du calcul qui suit est possible mais prendrait trop de place dans cet expos\'e sans \'eclairer davantage les id\'ees.} au point $(\gamma,z)$ et $(\delta \gamma^*,\delta z^*)$ son image par  $(\Phi,\tau)$.
Il faut montrer que $\alpha_{(\gamma,z)}(\delta \gamma,\delta z)= \alpha_{(\gamma^*,z^*)}(\delta \gamma^*,\delta z^*)$,
or
%
\begin{eqnarray*}
\alpha_{(\gamma^*,z^*)}(\delta\gamma^*,\delta z^*) & = & K\omega(\delta \gamma^*) + \theta(\delta z^*) \\
& = & \int_0^1\omega_{\gamma^*(t)}(\dot\gamma^*(t),\delta \gamma^*(t))\ dt + \theta(\delta z) - \delta \int_{\sigma(\Phi,\gamma)}\omega.
\end{eqnarray*}
%
En introduisant la famille de champs de vecteurs $\xi_t$ d\'efinie sur $X$ par:
%
$$
\xi_t (x) = [D(\Phi_t)(x)]^{-1}\bigg({\partial \Phi_t(x) \over \partial t}\bigg),
$$
%
on peut \'ecrire:
%
$$
K\omega(\delta \gamma^*) = K\omega (\delta \gamma) + \int_0^1\omega (\xi_t(\gamma (t)),\delta \gamma (t))\ dt.
$$
%
D'autre part,
en utilisant la formule de Stokes,
telle qu'elle est exprim\'ee par (\ref{varint}) dans l'annexe \ref{AnnStokes},
on peut v\'erifier que:
%
$$
\delta \int_{\sigma(\Phi,\gamma)}\omega = \int_0^1\omega (\xi_t(\gamma (t)),\delta \gamma (t))\ dt,
$$
%
d'o\`u on d\'eduit le r\'esultat.
\end{proof}

\begin{proposition}
L'action de $\Is(X,\omega)\times_{\tilde K_\omega} T_\omega$ sur $\Arc(X,\xo)\times T_\omega$ se projette en une action sur $Y$ dont le noyau est le sous-groupe :
%
\begin{equation}\label{defGamma}
\Gamma = \left\{(\Phi,\tau)\in \Is(X,\omega)\times_{\tilde K_\omega} T_\omega \tq \phi = \id_X \mbox{ et } \tau  = \cl_\omega \int_{\sigma_\Phi}\omega \right\},
\end{equation}
%
o\`u $\phi = \Phi_1$ et $\sigma_\Phi$ est une $2$-cha\^{\i}ne quelconque bordant $c_\Phi$ \ie $\partial \sigma_\Phi = c_\Phi$.
\end{proposition}

\begin{proof}
Il faut montrer que si $(\gamma,z)\sim (\gamma',z')$ au sens de la relation (\ref{defrelequiv}) alors $(\Phi,\tau)(\gamma',z')\sim (\Phi,\tau)(\gamma,z)$.
Puisque $\gamma(1)=\gamma'(1)$ il existe une $2$-cha\^{\i}ne $\sigma_{\gamma,\gamma'}$ bordant $\gamma'-\gamma$ telle que $z'=z+\cl_\omega \int_{\sigma_{\gamma,\gamma'}}\omega$.
Il faut donc v\'erifier que:
%
$$
\left(\Phi *\gamma' ,z+ \cl_\omega \int_{\sigma_{\gamma,\gamma'}}\omega + \tau -\cl_\omega \int_{\sigma(\Phi,\gamma')}\omega \right) \sim \left(\Phi *\gamma ,z+\tau -\cl_\omega \int_{\sigma(\Phi,\gamma)}\omega \right).
$$
%
Autrement dit,
il faut montrer que pour toute cha\^{\i}ne $\sigma_{\Phi*\gamma,\Phi*\gamma'}$ bordant $\Phi*\gamma'-\Phi*\gamma$ on a :
%
$$
\cl_\omega \int_{\sigma_{\Phi*\gamma,\Phi*\gamma'}}\omega = \cl_\omega \int_{\sigma_{\gamma,\gamma'}}\omega -\cl_\omega \int_{\sigma(\Phi,\gamma')}\omega + \cl_\omega \int_{\sigma(\Phi,\gamma)}.
$$
%
Puisque $\phi^*\omega =\omega$ ($\phi = \Phi_1$),
il existe une cha\^{\i}ne $\sigma_{\phi\circ \gamma,\phi \circ \gamma'}$ bordant  $\phi\circ \gamma'-\phi \circ \gamma$ telle que
%
$$\cl_\omega \int_{\sigma_{\phi\circ \gamma,\phi \circ \gamma'}} \omega = \cl_\omega \int_{\sigma_{\gamma,\gamma'}}\omega.$$
%
Un calcul \'el\'ementaire permet alors de v\'erifier que les cha\^{\i}nes $\sigma_{\Phi*\gamma,\Phi*\gamma'}$ et $\sigma(\Phi,\gamma') -  \sigma(\Phi,\gamma') + \sigma(\Phi,\gamma)$ sont homologues,
ce qui \'etablit le r\'esultat.
L'action de $\Is(X,\omega)\times_{\tilde K_\omega} T_\omega$ sur $\Arc(X,\xo)\times T_\omega$ passe donc au quotient $Y= [\Arc(X,\xo)\times T_\omega]/\sim$.
Le calcul du noyau est une formalit\'e.
\end{proof}

\begin{corollary}
L'extension centrale de $\Is(X,\omega)$ par $T_\omega$ d\'efinie par le cocycle $\tilde K_\omega$ se factorise sur l'extension centrale r\'eguli\`ere de $\Diff(X,\omega)$ :
%
\begin{equation}
\begin{tikzcd}
0 \arrow[r] & T_\omega \arrow[r] \arrow[d, equal] & \Is(X,\omega)\times_{\tilde K_\omega}T_\omega \arrow[r] \arrow[d, "\Gamma"] & \Is(X,\omega) \arrow[r] \arrow[d, "\but"] & 1 \\
0 \arrow[r] & T_\omega \arrow[r] & \Aut(Y,\lambda) \arrow[r] & \Diff^\circ(X,\omega) \arrow[r] & 1
\end{tikzcd}
\end{equation}
%
Le groupe $\Aut(Y,\lambda)$ est isomorphe au quotient de $\Is(X,\omega)\times_{\tilde K_\omega} T_\omega$ par son sous-groupe $\Gamma$ (d\'efinition \ref{defGamma}).
\end{corollary}

\begin{proof}
Il suffit de v\'erifier que la projection de $\Is(X,\omega)\times_{\tilde K_\omega} T_\omega$ sur $\Aut(Y,\lambda)$ est surjective,
ce qui est imm\'ediat.
\end{proof}

La construction du cocycle $\tilde K_\omega$ nous donne ainsi la possibilit\'e de d\'efinir un nouveau cocycle $K_\omega$ caract\'eristique de l'extension centrale r\'eguli\`ere de $\Diff^\circ (X,\omega)$.
Il suffit de choisir une section du morphisme $\but: \Is(X,\omega)\to \Diff^\circ(X,\omega)$,
c'est \`a dire d'associer \`a tout diff\'eomorphisme $\phi\in \Diff^\circ(X,\omega)$ une isotopie $\Phi$ telle que $\Phi_1=\phi$.
Tout autre choix d'isotopie conduit \`a un cocycle $K'_\omega$ cohomologue \`a $K_\omega$.
On peut le v\'erifier directement en consid\'erant l'application qui \`a $\phi\in \Diff^\circ(X,\omega)$ associe la classe de l'aire d'un cycle bordant le cycle $c'_\phi - c_\phi$.
Ces cocycles que nous appellerons {\em cocycles triangulaires} sont illustr\'es par la figure \ref{fig1}.

\begin{note}
Il n'y a aucune raison {\em a priori} pour qu'un des cocycle $K_\omega$ soit diff\'erentiable cela signifierait que la fibration principale $\Aut^\circ(\xi)\to \Diff^\circ(X,\omega)$ est diff\'erentiablement triviale.
\end{note}

\begin{note}\label{nulcoc}
Si le groupe $\Diff^\circ(X,\omega)$ a un point fixe,
que l'on peut toujours choisir comme point base,
l'extension centrale r\'eguli\`ere est triviale.
En effet,
le cocycle $K_\omega$ est le cobord de l'application $\phi\mapsto \cl_\omega \int_\sigma \omega$,
o\`u $\sigma$ est une une cha\^{\i}ne bordant $c_\phi$.
\end{note}
%
Nous verrons plus loin des exemples d'extensions r\'eguli\`eres non triviales.

\section{Cocycle triangulaire et moment}

\noindent
Le cocycle triangulaire de l'extension r\'eguli\`ere de $\Diff^\circ(X,\omega)$ est \'etroitement li\'e au moment de l'action de ce groupe sur $X$ et de son d\'efaut d'\'equivariance.
Soit $\Vect(X,\omega)$ l'espace des champs de vecteurs sur $X$ qui pr\'eservent $\omega$:
%
\begin{equation}
\Vect(X,\omega) =\{ \xi \in \Vect(X) \mid  \ \DLie_\xi \omega =0\}.
\end{equation}
%
D'apr\`es la formule de Cartan,
la $1$-forme $i_\xi \circ \omega$ est ferm\'ee,
puisque par hypoth\`ese $H_1(X,\ZZ)=0$,
cette forme est exacte.
Il existe donc une fonction diff\'erentiable r\'eelle $J_\xi$ telle que :
%
\begin{equation}
i_\xi \circ \omega = - dJ_\xi.
\end{equation}
%
La fonction $J_\xi$ n'est d\'efinie qu'\`a une constante pr\`es,
nous la fixerons en exigeant que $J_\xi (\xo)=0$.
Il existe alors une application $J$ de $X$ dans le dual $\Vect^*(X,\omega)$,
appel\'ee {\em application moment},
telle que $J_\xi (x)=J(x)(\xi)$,
d\'efinie par:
%
\begin{equation}
J  :   X\to \Vect^*(X,\omega)\quad \mbox{ et } \quad
J(x) = [\xi \mapsto - \int_{\xo}^x i_\xi \circ \omega],
\end{equation}
%
o\`u l'int\'egrale est calcul\'ee sur un arc quelconque de $\xo$ \`a $x$.
Le groupe $\Diff^\circ (X,\omega)$ agit naturellement sur $X$,
et par son action coadjointe,
sur $\Vect^*(X,\omega)$.
Soit $\phi\in \Diff^\circ (X,\omega)$,
nous noterons respectivement:
%
\begin{equation}
\ad(\phi): \xi \mapsto \phi_*\xi \quad \mbox{et} \quad \ad^*(\phi) : \mu \mapsto \mu \circ \ad(\phi ^{-1}),
 \end{equation}
%
son action adjointe sur $\Vect(X,\omega)$,
et son action coadjointe sur $\Vect^*(X,\omega)$.
La variance de $J$,
sous l'action de $\phi$,
est alors donn\'ee par:
%
\begin{equation}\label{defmom}
J\circ \phi = \ad^*(\phi)\circ J + \Theta (\phi ) \quad \mbox{avec} \quad   \Theta (\phi ) : \xi \mapsto - \int_{\xo}^{\phi (\xo)}i_\xi \circ \omega.
\end{equation}

\begin{definition}
L'application $\Theta$ est un $1$-cocycle du groupe $\Diff^\circ(X,\omega)$ dans $\Vect^*(X,\omega)$ pour l'action coadjointe,
il sera appel\'e le {\em d\'efaut d'\'equivariance\footnote{C'est \`a J.-M. Souriau que l'on doit les notions de moment et de d\'efaut d'\'equivariance que nous venons d'introduire. La pr\'esentation que nous en donnons est l\'eg\`erement diff\'erente pour satisfaire les besoins de cet article.}} du moment $J$:
%
\begin{equation}
\Theta : \Diff^\circ(X,\omega) \to \Vect^*(X,\omega), \quad   \Theta(\phi \circ \psi) = \ad^*(\phi)\Big(\Theta(\psi)\Big) + \Theta(\phi ).
\end{equation}
\end{definition}

Le d\'efaut d\'equivariance $\Theta$ et le cocycle triangulaire $K_\omega$ sont reli\'es par la construction suivante.
Soit $\xi\in \Vect(X,\omega)$ un champ de vecteurs com\-plet,
nous noterons $t\mapsto \exp(t\xi)\in \Diff(X,\omega)$ le groupe \`a un param\`etre de $\Diff(X,\omega)$ associ\'e.
Soit $E(t\xi)$ l'isotopie de $\exp(t\xi)$ d\'efinie par:
%
\begin{equation}
E(t\xi) : s \mapsto \exp(st\xi).
\end{equation}

\begin{proposition}
Soit $\phi \in \Diff^\circ(X,\omega)$ et $\xi \in \Vect(X,\omega)$ un champ de vec\-teurs complet,
alors:
%
\begin{equation}\label{valcocThetatilde}
\Theta(\phi )(\xi) = {\partial \over \partial t}\left\{\tilde K_\omega (\Phi,E(t\phi^{-1}_*\xi)) - \tilde K_\omega(E(t\xi),\Phi) \right\}_{t=0},
\end{equation}
%
o\`u $\Phi$ est une isotopie quelconque de $\phi$.
\end{proposition}

\begin{proof}
Soit $\Phi$ une isotopie de $\phi$,
consid\'erons la famille de $2$-cha\^{\i}nes rectangulaires suivante:
%
\begin{equation}
\sigma_t : (s,s') \mapsto \exp(st\xi)\Big(\Phi_{s'}(\xo)\Big).
\end{equation}
%
Un calcul \'el\'ementaire permet de v\'erifier que :
%
\begin{equation}
{\partial \over \partial t}\left\{\int_{\sigma_t}\omega\right\}_{t=0} = -\Theta(\phi)(\xi).
\end{equation}
%
Le bord de $\sigma_t$ est donn\'e par:
%
$$
\partial \sigma_t = c_\Phi + \phi \circ c_{E(t\phi^{-1}_*\xi)} - \exp(t\xi)\circ c_\Phi - c_{E(t\xi)}.
$$
%
En divisant ce rectangle en deux triangles le long de la diagonale $s=s'$ ce qui conduit \`a ajouter et retrancher $c_{\Phi*E(t\phi^{-1}_*\xi)}$ du bord (voir figure \ref{fig3}),
on obtient:
%
\begin{equation}
\cl_\omega \int_{\sigma_t}\omega = \tilde K_\omega(E(t\xi),\Phi) - \tilde K_\omega (\Phi,E(t\phi^{-1}_*\xi)),
\end{equation}
%
d'o\`u le r\'esultat.
\end{proof}

\begin{note}
Le groupe $\Is(X,\omega)$ agit par l'interm\'ediaire de $\Diff^\circ(X,\omega)$.
La formule ci-dessus est en r\'ealit\'e l'expression du d\'efaut d'\'equivariance du moment de $\Is(X,\omega)$ qui ne d\'epend de l'isotopie $\Phi$ que par son extr\'emit\'e,
et se projette donc sur le d\'efaut $\Theta$.
Soit $\phi \in \Diff^\circ(X,\omega)$ et $\xi \in \Vect(X,\omega)$ un champ complet,
il est possible de choisir une section de la projection $\but : \Is(X,\omega)\to \Diff^\circ(X,\omega)$ de telle sorte que l'application $t\mapsto K_\omega(\phi,\exp(t\xi))$ soit diff\'erentiable.
Il suffit de choisir une isotopie quelconque $\Phi$ de $\phi$ et d'\'etendre \`a $\Diff^\circ(X,\omega)$ tout entier la section d\'efinie sur $\{\phi\}\times \{\exp(t\xi)\}_{t\in \RR}$ par $(\phi, \exp(t\xi))\mapsto (\Phi, E(t\xi))$.
Dans ce cas le d\'efaut d'\'equivariance $\Theta$ est donn\'e par la formule:
%
\begin{equation}\label{valcocTheta}
\Theta(\phi)(\xi) = {\partial \over \partial t}\left\{K_\omega(\phi,\exp(t\phi^{-1}_*\xi)) - K_\omega(\exp(t\xi),\Phi) \right\}_{t=0},
\end{equation}
%
c'est comme \c{c}a qu'il faut interpr\'eter la formule (\ref{valcocThetatilde}).
\end{note}

Soit $k_\omega$ le $2$-cocycle d\'eriv\'e de $\Theta$ (annexe \ref{morder}),
c'est un cocycle de l'alg\`ebre de Lie $\Vect_\omega(X)$ \`a valeurs r\'eelles.
Par un calcul analogue \`a celui de $\Theta$,
on peut montrer que:
%
\begin{eqnarray*}
k_\omega(\xi,\xi') & = & {\partial^2 \over \partial t\partial s}\Big\{\tilde K_\omega (E(t\xi), E(s\xi')) - \tilde K_\omega (E(t\xi'), E(s\xi)) \Big\}_{s=t=0} \\ \nonumber
 & = & {\partial^2 \over \partial t\partial s}\Big\{K_\omega (\exp(t\xi), \exp(s\xi')) - K_\omega (\exp(t\xi'), \exp(s\xi)) \Big\}_{s=t=0},
\end{eqnarray*}
%
o\`u $\xi$ et $\xi'$ sont deux champs de vecteurs complets dans $\Vect(X,\omega)$,
et $K_\omega$ est choisi de telle sorte que $(t,s)\mapsto K_\omega (\exp(t\xi'), \exp(s\xi))$ soit diff\'erentiable.
Le $2$-cocycle $k_\omega$ d\'efinit une extension centrale de $\Vect(X,\omega)$ par $\RR$,
isomorphe \`a l'alg\`ebre de Lie de l'extension r\'eguli\`ere,
nous l'appellerons {\em extension centrale r\'eguli\`ere infinit\'esimale}.
Un calcul direct montre que :
%
\begin{equation}
k_\omega (\xi ,\xi') = \omega (\xi(\xo),\xi'(\xo)).
\end{equation}

\begin{note}
Si un groupe de Lie $G$ agit par automorphismes sur la forme $\omega$,
$\Vect^*(X,\omega)$ se projette sur le dual $\cG^*$ de l'alg\`ebre de Lie $\cG$ de $G$,
le moment de $G$ est alors l'image de $J$ par cette projection,
de m\^eme en ce qui concerne son d\'efaut d'\'equivariance.
\end{note}

\begin{note}
Le moment $J$ est constant sur le feuilletage caract\'eristique de la forme $\omega$.
C'est un invariant int\'egral de $\omega$ (th\'eor\`eme de N\oe ther).
\end{note}

\begin{note}
Le groupe $\Diff(Y,\lambda)$ agit aussi sur $X$ simplement par projection sur $\Diff^\circ(X, \omega)$.
Le moment $\tilde J$ de cette action:
$\tilde J(\xi,\epsilon) = J(\xi)$,
o\`u $(\xi,\epsilon)\in \Vect(X,\omega)\times_k \RR$,
est alors \'equivariant sous l'action de $\Diff^\circ(Y,\lambda)$.
\end{note}

\begin{note}
La classe de cohomologie de $\Theta$ ne d\'epend que de $\omega$:
si le moment est translat\'e d'une constante,
$\Theta$ est translat\'e d'un cobord.
C'est pour cette raison que seule la classe de cohomologie $\Theta$ n'a de sens intrins\`eque (il en va de m\^eme pour $K_\omega)$.
Cette classe de cohomologie qui traduit la non \'equivariance du moment sous l'action de $\Diff^\circ(\omega,X)$ peut s'annuler dans de nombreux cas.
Par exemple si $\Diff^\circ(\omega,X)$ a un point fixe,
ou encore si $\omega$ est symplectique et $X$ compacte,
dans ce cas on consid\`ere la moyenne du moment $J$ pour le volume d\'efini par la forme $\omega$.
\end{note}

\begin{note}
Si $H_1(X,\ZZ)\neq 0$ les alg\`ebres de Lie de $\Ham_\lambda(X,\omega)$ sont identiques et \'egales \`a $\Vect(X,\omega)$.
Il est vraisemblable que les diff\'erents cocycles d'alg\`ebres de Lie $k_\omega$,
associ\'es \`a chaque choix de structure d'int\'egration (index\'ees par $H^1(X,T_\omega)$),
conduisent \`a des extensions centrales in\'equivalentes de $\Vect(X,\omega)$ par $\RR$.
Par exemple,
pour une $2$-forme sur $\TT^2$,
on obtiendrait une famille d'extensions in\'equivalentes index\'ee par $\Hom(\ZZ^2,S^1) \simeq S^1\times S^1$,
le fibr\'e d'int\'egration \'etant unique puisque $T^2$ et compacte et $H^1(T^2,\ZZ)=\ZZ^2$ est sans torsion) (voir remarque \ref{remsanstorsion}).
\end{note}

\section{Exemples}

\noindent
Commen\c{c}ons par quelques pr\'eliminaires dans le cas d'une forme exacte:
$\omega = d\alpha$.
Le fibr\'e $Y$ est le produit direct $X\times \RR$ et $\lambda =\alpha + dt$.
Tout diff\'eomorphisme $\phi$ qui preserve $\omega$ v\'erifie $d[\phi^*\alpha - \alpha]=0$.
S'il existe une application $F:\Diff(X,\omega) \to \Cinfty(X,\RR)$ telle que :
%
\begin{equation}\label{eqF}
\phi^*\alpha  = \alpha + dF(\phi),
\end{equation}
%
alors elle v\'erifie:
%
\begin{equation}
d[F(\phi \circ \psi) - F(\phi)\circ \psi - F(\psi)]=0,
\end{equation}
et le cocycle de l'extension centrale r\'eguli\`ere est donn\'e par:
%
\begin{equation}
K_\omega(\phi,\psi) = F(\phi\circ \psi) - F(\phi)\circ \psi - F(\psi).
\end{equation}
%
On peut toujours normaliser $F$ en exigeant $F(\phi)(\xo)=0$ pour tout $\phi$,
auquel cas :
%
\begin{equation}\label{simpcoc}
K_\omega(\phi,\psi) = F(\phi)(\psi(\xo)).
\end{equation}
%
Formule que l'on obtient,
\`a un cobord pr\`es,
en appliquant la d\'efinition du cocycle triangulaire.
En d\'efinissant $j: X\to\Vect^*(X,\omega)$,
par:
%
\begin{equation}\label{defj}
j(x) : \eta \mapsto {\partial \over \partial t}\Big\{F(e^t\eta)(x)\Big\}_{t=0},
\end{equation}
%
on obtient l'expression suivante du moment:
%
\begin{equation}\label{formmomj}
J = i_\eta \circ \alpha -j\cdot \eta.
\end{equation}
%
Nous appliquerons ce petit formulaire dans le deux exemples suivants.

\begin{example}
On consid\`ere $X=\RR^2$,
muni de sa structure symplectique ordinaire $\omega =d\alpha$.
Son extension centrale r\'eguli\`ere n'est pas triviale.
En effet supposons que $K_\omega$ soit un cobord,
on peut trouver $F$ tel que $F(\phi \circ \psi') = F(\phi)\circ \psi'+F(\psi')$.
Ceci doit \^etre vrai,
en particulier,
pour les translations  $T_u : x \mapsto x+u$.
En utilisant l'identit\'e $\alpha_x(u) = \omega(x,u)$,
pour tout $x$ et $u$ dans $\RR^2$,
on d\'eduit $d[F(T_u)-i_u\circ \omega]=0$,
c'est \`a dire $F(T_u)(x) = \omega(u,x) + c(u)$,
o\`u $c:\RR^2\to \RR$.
Et par cons\'equent:
$\omega(u,u') = c(u+u') - c(u) - c(u')$,
ce qui est absurde puisque $\omega$ est antisym\'etrique.
De fa\c{c}on g\'en\'erale si on trouve un sous-groupe de $\Diff^\circ(X,\omega )$ pour lequel la restriction de l'extension r\'eguli\`ere n'est pas triviale,
l'extension elle-m\^eme n'est pas triviale.

On d\'efinit,
pour $X=\RR^2$,
un autre repr\'esentant de la classe de l'extension r\'eguli\`ere.
Soit $\mu :\Diff^\circ(\RR^2,\omega) \to \RR$,
l'application d\'efinie par:
%
\begin{equation}
\mu : \phi \mapsto \int_{c_\phi}^{t_x} K \omega,
\end{equation}
%
o\`u $x=\phi(0)$ et $t_x$ est le segment qui joint $0$ \`a $x$,
soit $\Delta \mu$ son cobord.
Le cocycle $H$,
d\'efini par:
%
\begin{equation}
H = K_\omega -\Delta \mu  = f_\omega(t_x+\phi \circ t_y,t_z),
\end{equation}
%
illustr\'e par la figure \ref{fig4},
repr\'esente encore la classe de l'extension r\'eguli\`ere,
c'est l'aire du simplex bord\'e par le segment $t_x$,
$\phi\circ t_y$ et le segment $t_z$ :

\begin{proposition}
Le cocycle $H$ prolonge \`a $\Diff^\circ(X,\omega)$ le {\em cocycle de Heisenberg} d\'efini sur le sous-groupe des translation de $\RR^2$.
\end{proposition}

Remarquons encore que la diff\'erence de $H$ avec le triangle lin\'eaire de sommets $(o,x,z)$ mesure en quelque sorte,
le d\'efaut de lin\'earit\'e du symplectomorphisme $\phi$.
\end{example}

\begin{example}
Consid\'erons l'espace $\Imm(S^1,\RR^2)$ des immersions du cercle $S^1$ dans $\RR^2$.
Il est point\'e par le plongement ordinaire $\iota : S^1 \hookrightarrow \RR^2$.
Nous le munissons de la $1$-forme suivante:
%
\begin{equation}
\alpha = \int {1 \over \norm{\gamma'}^2} \scal(\gamma'',\, d \gamma'),
\end{equation}
%
o\`u $\gamma'$ d\'esigne la d\'eriv\'ee par rapport au param\`etre de l'immersion,
et l'int\'egration se fait sur le cercle $S^1$.
Soit $\omega = d\alpha$.
On v\'erifie que $d[\phi^*\alpha -\alpha] = dF(\phi)$,
pour tout diff\'eomorphisme $\phi \in \Diff^\circ(S^1)$,
avec:
%
\begin{equation}
F(\phi) : \gamma  \mapsto \int \log\norm{(\gamma \circ \phi)'}\  d\log\phi'.
\end{equation}
%
La $2$-forme $\omega$ est donc invariante par $\Diff^\circ(S^1)$,
en voici une expression:
%
\begin{equation}
\omega = \int {1\over \norm{\gamma'}^2} \langle d\gamma'' - d[\log\norm{\gamma'}^2] \gamma'' \wedge d\gamma' \rangle,
\end{equation}
%
o\`u la notation $\langle \cdot \wedge \cdot\rangle$ d\'esigne l'op\'eration $(u,v)\mapsto \scal(u,v) -\scal(v,u)$.
En remarquant que $F(\phi)(\iota)=0$,
l'expression du cocycle $K_\omega$ se d\'eduit imm\'ediatement de la formule \ref{simpcoc}:
%
\begin{equation}
K_\omega (\phi,\psi) =\int \log(\phi \circ \psi )' \ d\log\psi'
\end{equation}
%
On reconnait la formule de Bott-Thurston \cite{Bott1} du cocycle de groupe qui d\'ecrit la seule extension centrale non triviale du groupe $\Diff^\circ(S^1)$ par $\RR$.
En utilisant l'application $j$,
d\'efinie par la formule \ref{defj},
qui a pour valeur dans cet exemple:
%
\begin{equation}
j(\gamma) : \eta \mapsto \int (\log\norm{\gamma'})''\ \eta,
\end{equation}
%
on obtient imm\'ediatement l'expression du moment:
%
\begin{equation}
J(\gamma) : \eta  \mapsto \int \left[ {\norm{\gamma''}^2  \over \norm{\gamma'}^2} - \Big(\log\norm{\gamma'}^2\Big)'' \right] \eta.
\end{equation}
%
Pour calculer $\Theta$,
il suffit d'appliquer la formule \ref{defmom} \`a l'immersion $\iota$ pour lesquels la vitesse et l'acc\'el\'eration sont de modules constants \'egaux \`a $1$.
En remarquant que $\int \phi_*\eta  = \int \eta (\phi')^2$,
il vient imm\'ediatement:
%
\begin{equation}
\Theta (\phi) : \eta \mapsto \int {3\phi''^2 - 2\phi'''\phi' \over \phi'^2}\ \eta
\end{equation}
%
On reconnait l'expression de la d\'eriv\'ee Schwarzienne dont la classe de cohomologie repr\'esente le d\'efaut d'\'equivariance du moment $J$ de cette $2$-forme $\omega$.
Il est facile ensuite de calculer le cocycle infinit\'esimal,
que l'on reconnait \'evidemment comme le cocycle de Gelfand-Fuchs \cite{GelfandFuchs}:
%
\begin{equation}
k_\omega(\xi,\eta) = \int \xi''\eta'-\eta''\xi'.
\end{equation}
%
Ainsi ces trois objets:
{\em cocycle de Bott-Thurston---d\'eriv\'ee Schwarzienne---co\-cycle de Gelfand-Fuchs},
se trouvent reli\'es \`a travers cet exemple,
dans cette trilogie des $2$-formes ferm\'ees:
{\em extension centrale r\'eguli\`ere---d\'efaut d'\'equiva\-rian\-ce du moment---extension centrale infinit\'esimale}.
\end{example}

\begin{note}
La construction pr\'ec\'edente doit pouvoir s'\'etendre sans difficult\'e \`a l'espace des lacets d'une vari\'et\'e riemannienne quelconque,
en rempla\c{c}ant l'acc\'el\'eration par l'acc\'el\'eration g\'eod\'esique et les d\'eriv\'ees par les d\'eriv\'ees covariantes.
\end{note}

\begin{example}
Consid\'erons l'espace $L(S^3)$ des lacets non point\'es de $S^3$ et soit $\source$ l'application qui \`a tout lacet $\gamma$ associe son point base:
$\source(\gamma) = \gamma(1)$.
Soit $K$ l'op\'erateur cha\^{\i}ne-homotopie et $\vol$ la forme volume sur $S^3$ normalis\'ee,
\cad $\int_{S^3}\vol=1$.
La $2$-forme $\omega = K\vol\mid L(S^3)$ est ferm\'ee puisque $K\vol= \but^*\vol-\source^*\vol$ sur $\Arc(S^3)$,
et $\but=\source$ sur $L(S^3)$.
En vertu de l'expression (\ref{formuleK}),
on a:
%
\begin{equation}
\omega(\delta_1\gamma,\delta_2\gamma) = \int_\gamma \vol(\dot \gamma,\delta_1\gamma,\delta_2\gamma),
\end{equation}
%
pour tout couple $(\delta_1 \gamma,\delta_2\gamma)$ de vecteurs tangents au point $\gamma \in L(S^3)$.
La forme $\omega$ est enti\`ere,
plus pr\'ecisemment $P_\omega =\ZZ$.
En effet,
en utilisant la suite exacte d'homotopie de la fibration $\source: L(S^3)\to S^3$ il est possible de montrer que $\pi_0(\Omega (S^3),N) = \pi_1(\Omega (S^3),N) = 0$ et $\pi_2(\Omega (S^3),N)= H_2(L(S^3),\ZZ) = \ZZ$,
o\`u $N$ est le p\^ole nord de $S^3$.
On peut choisir comme g\'en\'erateur $\epsilon$ du $H_2$ le lacet obtenu dans $\Cinfty(S^2,S^3)$ par l'intersection de $S^3$ avec un hyperplan (dans $\RR^4$) situ\'e au p\^ole nord et pivotant autour d'un plan fixe.
On a bien $\int_\epsilon \omega = \int_{S^3}\vol = 1$.
En appliquant la proposition \ref{prop1} on en d\'eduit l'existence d'un fibr\'e principal $Y$ au dessus de $L(S^3)$,
de groupe $S^1= \RR/\ZZ$,
muni d'une connexion $\lambda$ de courbure $\omega$.

Tout arc de $L(S^3)$ point\'e en l'arc constant $\hat N : t \mapsto N$ est identifi\'e \`a une application du cylindre $\cyl=[0,1]\times S^1$ dans $S^3$ dont l'image de la base $\{0\}\times S^1$ est le p\^ole nord.
La forme $\alpha =K\omega \oplus \theta$ qui d\'efinit $\lambda$ par passage au quotient est alors donn\'ee par:
%
\begin{equation}
\alpha(\delta \gamma,\delta z) = \int_\cyl\gamma^*[\vol(\delta \gamma)] + {\delta z\over iz}.
\end{equation}
%
Ce fibr\'e (ou plut\^ot le fibr\'e en droite complexe $\CC$ qui lui est associ\'e) a \'et\'e introduit par J.-L. Brylinski \cite{Brylinski1}.
La construction que nous venons de pr\'esenter fait l'\'economie de la th\'eorie des gerbes.
C'est Serge Tabashnikov qui a sugg\'er\'e cette application.
\end{example}

\appendix

\section{Espaces diff\'erentiables}\label{AnnED}

\noindent
Nous utilisons dans cet article une terminologie emprunt\'ee \`a la th\'eorie des espaces diff\'erentiables dont on trouvera diff\'erentes variantes dans \cite{Chen1} \cite{Haefliger1} \cite{Souriau3} \cite{Iglesias2}.
Nous en rappelons quelques \'el\'ements:

Un param\'etrage d'un ensemble $X$ est une application $P$ d'un ouvert $U$ d'un espace num\'erique quelconque $\RR^n$ dans $X$.
Un {\em espace diff\'erentiable} est un ensemble $X$ pour lequel on a choisi un ensemble de param\'etrages dits diff\'erentiables,
qui seront appel\'ees {\em plaques},
et qui v\'erifient les propri\'et\'es suivantes:

\begin{enumerate}
\item Les param\'etrages constants sont des plaques,
\item le plus petit prolongement commun d'une famille compatible de plaques est encore une plaque,
\item le compos\'e d'une plaque par un param\'etrage $\Cinfty$ de sa source est encore une plaque.
\end{enumerate}

Les vari\'et\'es sont \'evidemment des espaces diff\'erentiables.
Mais les quotients de vari\'et\'es sont aussi des espaces diff\'erentiables:
une plaque du quotient est un param\'etrage qui admet un relev\'e diff\'erentiable local au voisinage de tout point.
C'est muni de cette structure quotient que le tore des p\'eriodes $T_\omega$ est consid\'er\'e.

Une application $F$ d'un espace diff\'erentiable $X$ dans un autre $Y$ est dite {\em diff\'erentiable} si le compos\'e d'une plaque de $X$ par $F$ est une plaque de $Y$.
Les diff\'eomorphismes de $X$ \`a $Y$ sont \'evidemment les applications diff\'erentiables bijectives dont l'inverse est aussi diff\'erentiables.

L'espace $\Cinfty(X,Y)$ des applications diff\'erentiables de $X$ dans $Y$ est muni naturellement d'une structure diff\'erentiable appel\'ee {\em structure fonctionnelle}.
Les plaques de cette structure sont les param\'etrages $r\mapsto f$ telles que l'application $(r,x)\mapsto f(x)$ soit diff\'erentiable.
C'est de cette structure diff\'erentiable dont est muni l'espace $\Arc(X)$.

On sait d\'efinir sur ces espaces la notion de fibr\'es diff\'erentiables et la th\'eorie de l'homotopie qui l'accompagne \cite{Iglesias2}.
La relation d'homotopie est d\'efinie \`a partir des familles \`a un param\`etre d'arcs diff\'erentiables,
ind\'ependamment du choix d'une topologie {\em a priori\/}\footnote{Il existe toutefois, sur les espaces diff\'erentiables, une topologie remarquable (appel\'ee {\em D-topologie}), la plus fine telle que les plaques soient continues.}.
Une projection d'espaces diff\'erentiables $\pi  : Y\to X$ est une {\em fibration} si,
pour toute plaque $P: U \to X$,
l'image r\'eciproque $P^*(Y)\to U$ de la projection $P$ est localement triviale.
On montre que tout espace diff\'erentiable $X$ poss\`ede un rev\^etement universel,
simplement connexe,
unique \`a \'equivalence pr\`es,
de groupe structural $\pi_1(X)$.
Par exemple,
la projection $\cl_\omega: \RR\to T_\omega$ r\'ealise le rev\^etement universel de $T_\omega$.

La notion de groupe diff\'erentiable est imm\'ediate:
un groupe diff\'erentiable est un groupe $G$ muni d'une structure diff\'erentiable compatible avec la multiplication et l'inversion.
L'{\em alg\`ebre de Lie}  $\cG$ de $G$ est d\'efini comme l'espace des homomorphismes diff\'erentiables de $\RR$ dans $G$:
$\cG = \Hom^\infty(\RR,G)$.
Dans le cas du tore $T_\omega$ les homomorphismes diff\'erentiables de $\RR$ dans $G$ se rel\`event au rev\^etement universel $\cl_\omega : \RR\to T_\omega$ (th\'eor\`eme de monodromie des espaces diff\'erentiables \cite{Iglesias2}) en des homomorphismes de $\RR$ dans $\RR$ et donc $\cT_\omega \simeq \RR$.

Une $p$-forme diff\'erentielle $\omega$ sur un espace diff\'erentiable $X$ est un proc\'ed\'e,
qui \`a toute plaque $P$ de $X$ associe une $p$-forme not\'ee $\omega(P)$ ou encore  $P^*\omega$,
d\'efinie sur le domaine $U$ de $P$,
telle que pour tout param\'etrage $\Cinfty$,
$F$ de $U$:
$$(P\circ F)^*\omega = F^*(P^*\omega).$$
La d\'eriv\'ee ext\'erieure de $\omega$ est alors d\'efinie par $$P^*[d\omega] = d[P^*\omega].$$
Cette d\'efinition donne lieu \`a un calcul diff\'erentiel ext\'erieur sur $X$ qui co\"{\i}ncide avec le calcul ext\'erieur ordinaire lorsque $X$ est une vari\'et\'e.
L'image r\'eciproque $f^*\omega$ d'une $p$-forme $\omega$ sur $X$ par une application diff\'erentiable $f:Y\to X$ est d\'efinie par:
$P^*[f^*\omega] = [f\circ P]^*\omega$,
o\`u $P$ est une plaque de $Y$.

\begin{proposition}\label{propformquot}
Soit $X$ un espace diff\'erentiable et $\pi :X\to Y$ une surjection,
$Y$ est muni de sa structure quotient.
Pour qu'une forme diff\'erentielle $\alpha$ d\'efinie sur $X$ passe au quotient,
\cad pour qu'il existe une forme $\beta$ sur $Y$ telle que $\alpha =\pi^*\beta$,
il faut et il suffit que pour tout couple de plaques $(P,P')$ de $X$ telles que $\pi \circ P=\pi \circ P'$ on ait:
$\alpha(P)=\alpha(P')$.
\end{proposition}

\begin{proof}
Par d\'efinition de la structure quotient,
toute plaque $Q$ de $Y$ s'\'ecrit:
$Q= \sup \pi \circ P_i$,
o\`u les $P_i$ sont des plaques de $X$ et $\sup$ d\'esigne le plus petit prolongement commun.
La forme $\beta$ est d\'efinie par:
$\beta(Q) = \sup \alpha(P_i)$.
En effet,
soit $U_i$ le domaine de d\'efinition de $P_i$,
restreintes \`a $U_{ij}=U_i\cap U_j$ les plaques $P_i$ et $P_j$ v\'erifient $\pi \circ P_i=\pi \circ P_j$,
et donc:
$\alpha(P_i\mid U_{ij})=\alpha(P_j\mid U_{ij})$.
Les formes $\alpha(P_i)$ sont compatibles,
elles ont donc un plus petit prolongement commun $\sup \alpha(P_i)$.
Par un raisonnement analogue on montre que $\beta(Q)$ ne d\'epend pas du choix des relev\'es $P_i$ de $Q$.
Les autres v\'erifications sont imm\'ediates.
\end{proof}

Il existe sur les espaces diff\'erentiable une formule de Cartan pour la d\'eriv\'ee de Lie des formes diff\'erentielles.
Soit $h$ un groupe \`a un param\`etre de diff\'eomorphismes de $X$:
$h\in \Hom^\infty(\RR,\Diff(X))$,
la d\'eriv\'ee de Lie d'une $p$-forme $\omega$ par $h$ est d\'efinie par:
%
\begin{equation}
\DLie_h\omega  = {\partial \over \partial t} \Big\{h(t)^*\omega\Big\}_{t=0}.
\end{equation}
%
Le contract\'e $i_h(\omega)$ de la $p$-forme $\omega$,
par le groupe \`a un param\`etre $h$,
est donn\'e par la construction suivante,
soit $P:U \to X$ une plaque de $X$ et $h\cdot P$ la plaque d\'efinie sur $\RR\times U$ par:
%
\begin{equation}\label{defhP}
h\cdot P: (t,r) \to h(t)\circ P(r).
\end{equation}
%
La $(p-1)$-forme $i_h(\omega),$ \'evalu\'ee sur la plaque $P$,
est donn\'ee par la formule :
%
\begin{equation}
P^*[i_h(\omega)] = i_{\partial /\partial t}[(h\cdot P)^*\omega]_{\{0\}\times U}
\end{equation}
%
Cette formule \'etend la construction ordinaire sur les vari\'et\'es,
elle pr\'eserve la formule de Cartan:

\begin{proposition}
Soit $X$ un espace diff\'erentiable,
$\omega$ une $p$-forme sur $X$ et $h$ un groupe \`a un param\`etre de diff\'eomorphismes de $X$.
On a l'identit\'e:
%
\begin{equation}\label{formCartan}
\DLie_h\omega = d[i_h(\omega)] + i_h[d\omega].
\end{equation}
\end{proposition}

\begin{proof}
Soit $P$ une plaque de $X$ d\'efinie sur un domaine $U$,
par d\'efinition :
$$[\DLie_h\omega](P) = {\partial \over \partial t}\bigg\{[h(t)^*\omega](P)\bigg\}_{t=0} = {\partial \over \partial t}\bigg\{[h(t)\circ P]^*\omega\bigg\}_{t=0}.$$
Soit $j : U \mapsto \{0\}\times U$ d\'efinie par $j(r)=(0,r)$,
on a:
$$[i_h(\omega)](P)=i_{\partial/\partial t}[\omega(h\cdot P)]_{\{0\}\times U},$$
o\`u $h\cdot P$ est d\'efinie par (\ref{defhP}),
\cad $[i_h(\omega)](P)=j^*[i_{\partial /\partial t}(\omega(h\cdot P))]$.
On peut donc \'ecrire:
%
\begin{eqnarray*}
[d(i_h\omega )](P) + [i_h(d\omega)](P) & = & d\{j^*(i_{\partial /\partial t}[\omega(h\cdot P)])\} + j^*\{i_{\partial /\partial t}(d[\omega(h\cdot P)])\} \\
& = & j^*\{ d\circ i_{\partial /\partial t}[\omega(h\cdot P)] + i_{\partial /\partial t}\circ d[\omega(h\cdot P)]\} \\
& = & j^* \{ \DLie_{\partial /\partial t}[\omega(h\cdot P)]\} \\
& = & j^* \left\{ {\partial \over \partial t} \big[h(t)^*\omega(h\cdot P)\big]_{t=0} \right\} \\
& = & {\partial \over \partial t}\Big\{[h(t)\circ h\cdot P \circ j]^*\omega\Big\}_{t=0},
\end{eqnarray*}
%
or $h(t)\circ h\cdot P\circ j(r) = h(t)(P(r))$ c'est-\`a-dire $h(t)\circ h\cdot P\circ j = h(t)\circ P$,
on en d\'eduit donc:
%
\begin{eqnarray*}
[d(i_h\omega )](P) + [i_h(d\omega)](P) & = &  {\partial \over \partial t}\Big\{[h(t)\circ P]^*\omega\Big\}_{t=0} \\
& = & {\partial \over \partial t}\Big\{h(t) ^*[\omega(P)]\Big\}_{t=0},
\end{eqnarray*}
%
\cad $$ [d(i_h\omega )](P) + [i_h(d\omega)] = {\partial \over \partial t}\Big\{h(t) ^*\omega \Big\}_{t=0},$$
c'est ce qu'il fallait d\'emontrer.
\end{proof}

\section{Op\'erateur de cha\^{\i}ne-homotopie}\label{AnnOCH}

\noindent
Soit $X$ une vari\'et\'e diff\'erentiable connexe,
on d\'esigne par $\Arc(X)$ son espace des arcs,
c'est \`a dire l'espace des applications diff\'erentiables de $\RR$ \`a valeurs dans $X$:
%
\begin{equation}
\Arc(X) = \Cinfty(\RR,X),
\end{equation}
%
Les applications {\em source} et {\em but},
not\'ees $\source$ et $\but$,
associent respectivement \`a tout arc $\gamma$ les points $\gamma(0)$ et $\gamma(1)$.
Le choix de $\RR$ plut\^ot que l'intervalle $[0,1]$ est technique,
pour assurer la diff\'erentiabilit\'e de la juxtaposition des arcs et l'action du groupe des translations.

L'espace des arcs de $X$ est muni de sa structure fonctionnelle d'espace diff\'erentiable (voir annexe \ref{AnnED}).

A toute $p$-forme $\omega$ de $X$ on associe la $p$-forme sur $\Arc(X)$ d\'efinie par int\'egration de $\omega$ le long des arcs,
ce que nous noterons:
%
\begin{equation}\label{formomtild}
\tilde \omega_\gamma  = \int_\gamma  \omega,
\end{equation}
%
L'expression de $\tilde \omega$ dans toute plaque (param\'etrage diff\'erentiable) $P$  de l'espace $\Arc(X)$,
est donn\'ee par:
%
\begin{equation}
P^*\tilde \omega  = \int_0^1P_t^*\omega \d t,
\end{equation}
%
o\`u $P_t$ est la plaque de $X$ d\'efinie par $P_t(r)=P(r)(t)$.
L'application lin\'eaire $\Phi : \omega \mapsto \tilde \omega$ ainsi d\'efinie,
est un morphisme du complexe diff\'erentiel de De Rham de $X$ dans celui de $\Arc(X)$:
%
\begin{equation}
\Phi : \Omega^*(X)  \rightarrow \Omega^*(\Arc(X)) \quad \mbox{et} \quad  d\circ \Phi  = \Phi \circ d.
\end{equation}
%
Les translations de $\RR$ agissent comme un groupe \`a un param\`etre $\eta$ de diff\'eo\-morphismes de $\Arc(X)$,
pour tout $\gamma \in \Arc(X)$ et tout $a\in \RR$:
%
\begin{equation}
\eta(a): \gamma \mapsto \gamma \circ T_a \quad \mbox{o\`u} \quad T_a : t\mapsto t+a.
\end{equation}

\begin{proposition}
La d\'eriv\'ee de Lie de la $p$-forme $\tilde\omega$ par $\eta$ v\'erifie:
%
\begin{equation}
\DLie_\eta \tilde\omega = \but^*\omega -\source^*\omega.
\end{equation}
\end{proposition}

\begin{proof}
En effet,
pour toute plaque $P$ de $\Arc(X)$:
%
$$[\DLie_\eta\tilde\omega](P) = {\partial \over \partial t}\bigg\{[\eta(t)^*\tilde\omega](P)\bigg\}_{t=0} = {\partial \over \partial t}\bigg\{(\eta(t)^*\circ P)^*\tilde\omega\bigg\}_{t=0}.$$
%
Mais $\eta(t)\circ P : r\mapsto P(r)\circ T_t$ et donc:
%
$$(\eta(t)\circ P)^*\tilde \omega = \int_0^1(\eta(t)\circ P)_s^*\omega\ ds = \int_0^1[r\mapsto P(r)(t+s)]^*\omega\ ds.$$
%
En posant $u=t+s$ on obtient
%
$$(\eta(t)\circ P)^*\tilde \omega = \int_t^{1+t}[r\mapsto P(r)(u)]^*\omega\ du,$$
%
ce qui donne en d\'erivant:
%
$$[\DLie_\eta\tilde\omega](P) = [r\mapsto P(r)(1)]^*\omega - [r\mapsto P(r)(0)]^*\omega = [\but^*\omega-\source^*\omega](P).$$
%
C'est ce qu'il fallait d\'emontrer.
\end{proof}

En appliquant alors la formule de Cartan (\ref{formCartan}) on a:
$$d[i_\eta\Phi (\omega)]+i_\eta \circ d[\Phi (\omega)] = \but^*\omega - \source^*\omega.$$
En posant alors:
%
\begin{equation}\label{defOH}
K = i_\eta \circ \Phi , \quad K:\Omega ^*(X) \to \Omega ^{*-1}(\Arc(X,\xo)),
\end{equation}
%
on obtient l'identit\'e:
%
\begin{equation}\label{ophom}
K\circ d+d\circ K = \but^*-\source^*.
\end{equation}

\begin{definition}
Le morphisme $K$ ci-dessus est appel\'e {\em op\'erateur de cha\^{\i}ne-homoto\-pie}.
\end{definition}

Lorsque $X$ est une vari\'et\'e,
on peut donner une expression de l'op\'erateur $K$,
en termes d'int\'egrales de chemins.
Soit $P$ une plaque de $\Arc(X)$ d\'efinie sur un domaine $U$,
soit $r\in U$ et $\delta r$ un vecteur tangent en $r$.
Notons $$\gamma =P(r) \quad \mbox{et} \quad \delta \gamma : t\mapsto D[\gamma(t)](r)(\delta r),$$
alors $K\omega$,
calcul\'ee au point $\gamma$ et appliqu\'ee \`a $p-1$ {\em variations} $\gamma_i$ associ\'ees \`a des vecteurs $\delta r_i$ est donn\'ee par:
%
\begin{equation}\label{formuleK}
K\omega_\gamma(\delta \gamma_1,\ldots,\delta \gamma_{p-1}) = \int_0^1\omega_{\gamma(t)}(\dot\gamma(t),\delta \gamma_1,\ldots,\delta \gamma_{p-1})\ dt.
\end{equation}

\section{La formule de Stokes pour les espaces dif\-f\'e\-rentiables}\label{AnnStokes}

\noindent
Nous allons montrer,
dans cette annexe,
comment on peut g\'en\'eraliser  la formule de Stokes aux espace diff\'erentiables.
Nous choisirons pour cela d'utiliser l'homologie cubique.

Soit $X$ un espace diff\'erentiable,
nous appellerons {\em $p$-cube} de $X$ toute applications diff\'erentiable $c$ d\'efinie sur un voisinage $U$ de $I^p$ \`a valeurs dans $X$,
avec $I=[0,1]$ (c'est en particulier une $p$-plaque de $X$).
L'homologie cubique est d\'efinie de fa\c{c}on ordinaire,
en consid\'erant les diff\'erents groupes ab\'elien libres $C_p(X)$ engendr\'es par les $p$-cubes,
$p\in \NN$.
Les \'el\'ements de $C_p(X)$ sont les {\em $p$-cha\^{\i}nes cubiques} de $X$.

Soit $\epsilon\in \{0,1\}$,
on notera $j_i^\epsilon$ l'applications de $\RR^{p-1}$ dans $\RR^p$ d\'efinies par:
 %
\begin{equation}
j_i^\epsilon : (t_1,\ldots,t_{p-1}) \to (t_1,\ldots,t_{i-1},\epsilon,t_{i+1},\ldots,t_{p-1}).
\end{equation}
%
A chaque indice $i=1,\ldots p$ correspond les deux faces du cube $c$ :
%
\begin{equation}
c_i^\epsilon = c\circ j_i^\epsilon , \quad i=0,1.
\end{equation}
%
Ce qui permet de d\'efinir le bord $\partial c$ d'un $p$-cube $c$ par:
%
\begin{equation}
\partial c =\sum_{\epsilon =0}^1(-1)^\epsilon \sum_{i=1}^p (-1)^p c_i^\epsilon.
\end{equation}
%
L'op\'erateur $\partial$ est ensuite prolong\'e par lin\'earit\'e sur $C_p(X)$ tout entier,
il est \`a valeurs dans $C_{p-1}$ ($p\geq 1$).

L'int\'egrale d'une $p$-forme $\omega\in \Omega^p(X)$ sur une $p$-cha\^{\i}ne $\sigma\in C_p(X)$ est obtenue en prolongeant par lin\'earit\'e l'int\'egrale de $\omega$ sur les $p$-cubes:
%
\begin{equation}
\sigma = \sum_{c\in C_p(X)} n_{c_p} c \quad \Rightarrow \quad \int_\sigma \omega = \sum_{c\in C_p(X)} n_{c_p} \int_c\omega,
\end{equation}
%
o\`u les entiers $n_c$ sont tous nuls sauf un nombre fini d'entre eux,
avec:
%
\begin{equation}
\int_c \omega = \int_{I^p} c^*\omega.
\end{equation}
%
Consid\'erons maintenant une famille diff\'erentiable de $p$-cubes $c_\alpha$,
\cad une application diff\'erentiable d\'efinie sur $]-\epsilon,\epsilon[\times I^p$ \`a valeurs dans $X$.
L'application $\alpha\mapsto \int_{c_\alpha}\omega_\alpha$ est alors diff\'erentiable,
nous noterons :
%
\begin{equation}
\delta \int_c\omega = {\partial \over \partial \alpha} \left\{ \int_{c_\alpha}\omega\right\}_{\alpha =0}
\end{equation}
%
la variation premi\`ere de l'int\'egrale de $\omega$ sur $c$ associ\'ee \`a la famille $c_\alpha$.

Soit $\bar c : ]-\epsilon,\epsilon[\times U \to X $ et $\bar\omega$,
les $(p+1)$-plaque et $p$-forme d\'efinies  respectivement par:
%
\begin{equation}
\bar c = (\alpha,t) \mapsto c_\alpha(t) \quad \mbox{et} \quad \bar \omega = \omega(\bar c).
\end{equation}
%
Soit $t_0=\alpha$ et $e_0$ le vecteur de base $\partial/\partial \alpha = (1,0,\ldots,0)$ de $\RR\times \RR^p$,
nous introduiraons pour la comodit\'e des expressions finales les quantit\'es suivantes:
%
\begin{equation}
\int_c d\omega(\delta c) = \int_{\{0\}\times I^p}[d\bar\omega](e_0) \quad \mbox{et}\quad \int_{\partial c}\omega (\delta c) = \sum_{\epsilon =0}^1 (-1)^{\epsilon}\sum_{i=0}^p\int_{I^{p-1}} {j_i^\epsilon}^*\bar \omega(e_0),
\end{equation}
%
o\`u la notation $\bar \omega(e_0)$ d\'esigne le contract\'e de $\bar\omega$ par le vecteur $e_0$ (idem pour $d\bar\omega$).
Il serait posible de d\'efinir formellement $\delta c$,
disons simplement que cela repr\'esente la variation infinit\'esimale du cube $c$ associ\'ee \`a la famille $c_\alpha$.

Le lecteur pourra v\'erifier que,
compte tenu de ces d\'efinitions,
la d\'emonstration de la proposition suivante se ram\`ene \`a un calcul diff\'erentiel ordinaire sur $\RR^p$.

\begin{proposition}
Soit $X$ un espace diff\'erentiable,
$\omega$ une $p$-forme sur $X$ et $c$ un $p$-cube de $X$.
Soit $c_\alpha$ une famille diff\'erentiable de $p$-cubes telle que $c_0=c$.
La variation premi\`ere de l'int\'egrale de $\omega$ sur $c$ associ\'ee \`a la famille $c_\alpha$ est donn\'ee par la formule:
%
\begin{equation}\label{varint1}
\delta \int_c\omega = \int_c d\omega(\delta c) + \int_{\partial c}\omega (\delta c).
\end{equation}
\end{proposition}

On obtient en particulier comme corollaire la formule de Stokes \'etendue aux espaces diff\'erentiables.

\begin{corollary}
Soit $\omega$ une $(p-1)$-forme d\'efinie sur un espace diff\'erentiable $X$,
et $\sigma$ une $p$-cha\^{\i}ne de $X$,
on a:
%
\begin{equation}
\int_\sigma d\omega = \int_{\partial \sigma} \omega.
\end{equation}
\end{corollary}

\begin{proof}
En effet,
il suffit de le d\'emontrer pour un $p$-cube $c$.
On a $\delta \int_c d\omega = \int_{\partial c}d\omega (\delta c)$ et aussi $\delta \int_{\partial c} \omega = \int_{\partial c}d\omega (\delta c)$,
car $dd\omega =0$ et $\partial \partial c=0$.
On d\'eduit que $\delta[\int_c d\omega - \int_{\partial c}\omega]= 0$,
et donc que  $\int_c d\omega - \int_{\partial c}\omega$ est constante sur l'espace des $p$-cubes,
comme l'espace des $p$-cubes est contractile il suffit d'\'evaluer cette diff\'erence sur le $p$-cube constant,
ce qui donne le r\'esultat.
\end{proof}

On peut \'etendre \'evidemment cette proposition par lin\'earit\'e \`a toute $p$-cha\^{\i}ne cubique $\sigma$,
en d\'efinissant correctement les familles diff\'erentiables de $p$-cha\^{\i}nes.
On peut l'\'etendre aussi sans difficult\'e \`a des domaines poly\'edraux ou encore plus g\'en\'eraux.

La formule (\ref{varint1}) ci-dessus peut se g\'en\'eraliser imm\'ediatement au cas o\`u la forme $\omega$ est aussi variable.
Soit $\omega_\alpha$ une famille diff\'erentiable de $p$-formes telle $\omega_0=\omega$.
On dira que la famille de $p$-formes $\omega_\alpha$ est diff\'erentiable si pour toute plaque $P$ de $X$ l'application $(\alpha,r)\mapsto [\omega_\alpha(P)]_r$ est diff\'erentiable,
autrement dit,
si pour toute famille de $p$-vecteurs $(u_1,\ldots,u_p)$ dans $\RR^p$ l'application $(\alpha,r)\mapsto[\omega_\alpha(P)]_r(u_1,\ldots,u_p)$ est diff\'erentiable.
En notant $\delta \omega$ la $p$-forme d\'efinie par:
%
\begin{equation}
\delta \omega : P \mapsto {\partial \over \partial \alpha}\{\omega_\alpha (P)\}_{\alpha =0},
\end{equation}
%
on peut \'ecrire:
%
\begin{equation}\label{varint}
\delta \int_c\omega = \int_c d\omega(\delta c) + \int_{\partial c}\omega (\delta c) + \int_c \delta \omega.
\end{equation}
%

\begin{note}
La formule de Stokes permet de montrer en particulier com\-me dans le cas des vari\'et\'es,
l'invariance par homotopie de la cohomologie de De Rham.
Elle permet aussi,
sous sa forme \'etendue (\ref{varint}),
de donner une autre d\'emonstration de la formule de Cartan (annexe \ref{AnnED}).
Soit $h$ un groupe \`a un param\`etre de diff\'eomorphismes de $X$,
$\omega$ une $p$-forme sur $X$ et $c$ un $p$-cube.
Notons $c_t=h(t)\circ c$ et $\omega_t = h(t)_*\omega = h(-t)^*\omega$.
En appliquant (\ref{varint}) \`a ces familles \`a un param\`etre et en notant que $\int_{c_t}\omega_t= \int_c\omega$,
on obtient:
%
\begin{equation}
\int_c\DLie_h\omega = \int_c i_h[d\omega] + \int_c d[i_h\omega].
\end{equation}
%
Cette \'egalit\'e \'etant v\'erifi\'ee pour tout $p$-cube $c$ de $X$,
on en d\'eduit la formule de Cartan :
$\DLie_h\omega = i_h[d\omega] + d[i_h\omega]$.
\end{note}

\section{Les morphismes d\'eriv\'es}\label{morder}

\noindent
Consid\'erons un groupe de Lie $G$,
$\cG$ son alg\`ebre de Lie,
et $\cG^*$ le dual de $\cG$.
Nous noterons $\ad$ et $\ad^*$ les action adjointes et coadjointes de $G$,
d\'efinies respectivement sur $\cG$ et $\cG^*$.

Soit $K$ un deux cocycle de $G$ \`a valeur dans un tore diff\'erentiable $T_A$,
c'est \`a dire le quotient de $\RR$ par un sous-groupe ab\'elien $A$:
%
\begin{equation}
K(g,g'g'') +K(g',g'') = K(gg',g'') + K(g',g'').
\end{equation}
%
Soit $Z\mapsto g_*Z$,
l'action adjointe de $G$ sur $\cG$,
$g\in G$ et $Z\in \cG$.
Si $K$ est normalis\'e,
c'est \`a dire $K(\id_G,\phi)=K(\phi,\id_G) =0$,
une v\'erification \'el\'ementaire montre que l'application $\Theta$ d\'efinie sur $G$,
\`a valeur dans $\cG^*$,
par:
%
\begin{equation}
\Theta(g) : Z \mapsto {\partial \over \partial t}\Big\{ K(g,e^{tg^*Z}) -K(e^{tZ},g) \Big \}_{t=0},
\end{equation}
%
est un $1$-cocycle de $G$ dans $\cG^*$ pour l'action coadjointe,
c'est \`a dire :
%
\begin{equation}
\forall (g,g')\in G\times G : \quad \Theta(gg') = \ad^*(g)\circ \Theta(g') +\Theta(g).
\end{equation}
%
De plus,
l'application bilin\'eaire $k$ d\'efinie sur $\cG\times \cG$ par :
%
\begin{eqnarray*}
k (Z,Z') & = & D\Theta_{\id_G} (Z)(Z') \\ \nonumber
              & = & {\partial^2 \over \partial t\partial s}\Big\{ K(e^{tZ}, e^{sZ'}) - K(e^{tZ'}, e^{sZ}) \Big\}_{s=t=0},
\end{eqnarray*}
%
est un $2$-cocycle de la cohomologie altern\'ee de $\cG$ dans $\RR$.
Le couple d'application $K\mapsto \Theta \mapsto k$ d\'efinit une suite de morphismes en cohomologie :
%
\begin{equation}
H^2(G,A)\to H^1(G,\cG^*) \to H^2(\cG,\RR)
\end{equation}
%
que nous appellons {\em morphismes d\'eriv\'es}.

\begin{note}
Cette construction peut s'\'etendre \`a certain grou\-pes diff\'erentiables,
notamment aux groupes de diff\'eo\-morphismes de vari\'et\'es.
\end{note}

\clearpage % Force a page break and flush all pending figures

\clearpage % Force a new page

\clearpage

% --- Page 1 (Figures 1 & 2) ---
\begin{figure}[p]
  \centering
  % We create a box that takes up the WHOLE page height
  \begin{minipage}[c][\textheight][c]{\textwidth}
    \centering
        
    \includegraphics[width=0.5\textwidth]{figures/trilogiefig1.png}
    \caption{Le cocycle $K_\omega$}
    \label{fig1}

    \vspace{3cm}
    
    \includegraphics[width=0.5\textwidth]{figures/trilogiefig2.png}
    \caption{La condition de cocycle pour $K_\omega$}
    \label{fig2}

    \vfill % Space at the bottom
    
  \end{minipage}
\end{figure}

\clearpage

% --- Page 2 (Figures 3 & 4) ---
\begin{figure}[p]
  \centering
  \begin{minipage}[c][\textheight][c]{\textwidth}
    \centering
        
    \includegraphics[width=0.5\textwidth]{figures/trilogiefig3.png}
    \caption{Le d\'efaut d'\'equivariance $\Theta$}
    \label{fig3}

    \vspace{3cm}
    
    \includegraphics[width=0.5\textwidth]{figures/trilogiefig4.png}
    \caption{Le cocycle $H$}
    \label{fig4}

    \vfill
    
  \end{minipage}
\end{figure}

\clearpage

%•••• bibliographie

% Content from trilogie.bbl.txt is pasted here.
\begin{thebibliography}{McL71}
\newcommand{\noopsort}[1]{} \newcommand{\printfirst}[2]{#1}  \newcommand{\singleletter}[1]{#1} \newcommand{\switchargs}[2]{#2#1}  \newcommand {\CRAS}{C. R. Acad. Sc.} \newcommand{\JFA}{J. of Func. An.}  \newcommand{\adrCPT}{CPT-CNRS, Luminy, F.~Marseille}
\bibitem[Bot78]{Bott1}R.~Bott.\newblock On some formulas for the characteristic classes of group actions,  differential topology, foliations and gelfand-fuchs cohomology.\newblock In {\em Proceed. Rio de Janeiro, 1976}, volume 652 of {\em Springer  Lectures Notes}. Springer Verlag, 1978.
\bibitem[Bry90]{Brylinski1}J.-L. Brylinski.\newblock The kaehler geometry of the space of knots in a smooth threefold.\newblock preprint, Pensylvania State University, 1990.
\bibitem[Bry93]{Brylinski2}J.-L. Brylinski.\newblock {\em Loop spaces, characteristic classes and geometric quantization}.\newblock Birk\"auser, 1993.
\bibitem[Che77]{Chen1}K.~T. Chen.\newblock Iterated path integral.\newblock {\em Bull. of Am. Math. Soc.}, 83(5):831--879, 1977.
\bibitem[GF68]{GelfandFuchs}I.~Gelfand et D.~Fuchs.\newblock Cohomology of the lie algebra of vector fields on the circle.\newblock {\em Functional Analysis and Applications}, 2:342--343, 1968.
\bibitem[Hae72]{Haefliger1}A.~Haef{l}iger.\newblock Sur les classes caract\'eristiques des feuilletages.\newblock Publications s\'eminaires, S\'eminaire Bourbaki, juin 1972.
\bibitem[Igl85]{Iglesias2}P.~Iglesias.\newblock Fibr\'es diff\'eologiques et homotopie.\newblock Th\`ese de doctorat d'\'etat, Universit\'e de Provence, F.~Marseille,  1985.
\bibitem[Kir74]{Kirillov1}A.~A. Kirillov.\newblock {\em Elements de la th\'eorie des repr\'esentations}.\newblock MIR, Moscou, 1974.
\bibitem[Kir82]{Kirillov4}A.~A Kirillov.\newblock {\em Infinite dimensional Lie groups: their orbits, invariants and  representations.}, volume 970.\newblock Springer-Verlag, 1982.
\bibitem[KM85]{KargMerz}M.~Kargapalov et Iou. Merzliakov.\newblock {\em Elements de la th\'eorie des groupes}.\newblock MIR, Moscou, 1985.
\bibitem[McL71]{MacLane2}S.~McLane.\newblock {\em Homology}.\newblock Springer Verlag, New-York Heidelberg Berlin, 1967.
\bibitem[Sou70]{Souriau5}J.-M. Souriau.\newblock {\em Structure des syst\`emes dynamiques}.\newblock Dunod, Paris, 1970.
\bibitem[Sou84]{Souriau3}J.-M. Souriau.\newblock Groupes diff\'erentiels et physique math\'e\-ma\-tique.\newblock {\em Collection travaux en cours}, pages 75--79, 1984.
\end{thebibliography}

\end{document}
