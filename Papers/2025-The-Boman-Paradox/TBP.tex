\documentclass[10pt,letterpaper,reqno]{amsart}

%%%%%%%%%%%%%%%%%%%%%%%%%%%%%%%%%%%%%%%%%%%%%%%%%%%%%%%%%%
%%
%% MARK: MACROS File
%%
%%%%%%%%%%%%%%%%%%%%%%%%%%%%%%%%%%%%%%%%%%%%%%%%%%%%%%%%%%

\usepackage[hidelinks]{hyperref}
\usepackage{graphicx}
\usepackage[innercaption]{sidecap}

% manage captions
\usepackage[margin=10pt,font=small,labelfont=bf, labelformat=empty, labelsep=endash]{caption}

% Required for the Draft Warning and Figures
\usepackage{tikz}
\usetikzlibrary{calc, positioning, arrows.meta}

% Basic style
\parindent 0mm
\parskip .5ex plus 2pt

\def\ni{\noindent}

% Macros used in this document
\newcommand{\RR}{\mathbf{R}}
\newcommand{\NN}{\mathbf{N}}
\newcommand{\Cinfty}{C^\infty}
\newcommand{\cD}{\mathcal{D}}
\newcommand{\cF}{\mathcal{F}}
\newcommand{\cW}{\mathcal{W}}

\DeclareMathOperator{\id}{\mathbf{id}}
\DeclareMathOperator{\Diff}{Diff}
\DeclareMathOperator{\rank}{rank}
\DeclareMathOperator{\dom}{dom}
\DeclareMathOperator{\stab}{Stab}

\newcommand{\loc}{\mathrm{loc}}

% Theorem Environments
\newtheorem*{theorem*}{Theorem} % Unnumbered theorems
\newtheorem{theorem}{Theorem} % Numbered theorems
\newtheorem*{proposition}{Proposition}
\newtheorem*{definition}{Definition}
\newtheorem*{remark}{Remark}
\newtheorem*{example}{Example}

% Font style
\usepackage{microtype}
\usepackage[cal=scr,uppercase = upright,greekfamily = didot,greeklowercase = upright,expert,utopia]{mathdesign}
\usepackage[supspaced=.04em]{superiors}
\linespread{1.1}

\allowdisplaybreaks

%%%%%%%%%%%%%%%%%%%%%%%%%%%%%%%%%%%%%%%%%%%%%%%%%%%%%%%%%%
%%
%% MARK: The Title
%%
%%%%%%%%%%%%%%%%%%%%%%%%%%%%%%%%%%%%%%%%%%%%%%%%%%%%%%%%%%

\title[The Boman Paradox]{\textbf{The Boman Paradox}\\ \small When topology plus algebra do not define geometry}
\author{Patrick Iglesias-Zemmour}

\thanks{I thank the Hebrew University of Jerusalem for its continuous academic support.}

\address{Einstein Institute of Mathematics, The Hebrew University of Jerusalem, Campus Givat Ram, $9190401$ Israel}
\email{piz@math.huji.ac.il}

\date{\today}

\subjclass[2020]{Primary 58A40; Secondary 58D05, 57R50}
\keywords{Diffeology, Wire Diffeology, Boman's Theorem, Diffeomorphism Groups, Diffeological Dimension, Foundations of Geometry}

%%%%%%%%%%%%%%%%%%%%%%%%%%%%%%%%%%%%%%%%%%%%%%%%%%%%%%%%%%
%%
%% MARK: Begin Document
%%
%%%%%%%%%%%%%%%%%%%%%%%%%%%%%%%%%%%%%%%%%%%%%%%%%%%%%%%%%%

\begin{document}
  
  %%%%%%%%%%%%%%%%%%%%%%%%%%%%%%%%%%%%%%%%%%%%%%%%%%%%%%%%%%%%%%%%%%%%%%%%%%%%%%%
  %%
  %% MARK: MACROS DraftWarning
  %%
  %%%%%%%%%%%%%%%%%%%%%%%%%%%%%%%%%%%%%%%%%%%%%%%%%%%%%%%%%%%%%%%%%%%%%%%%%%%%%%%
  
%  \tikz[overlay,remember picture]{\node at ($(current page.west)+(1.5,0)$) [rotate=90] {\Huge\textcolor{red}{Draft Version -- \pdfcreationdate}}}%\n
  
  %%%%%%%%%%%%%%%%%%%%%%%%%%%%%%%%%%%%%%%%%%%%%%%%%%%%%%%%%%
  %%
  %% MARK: Make Title
  %%
  %%%%%%%%%%%%%%%%%%%%%%%%%%%%%%%%%%%%%%%%%%%%%%%%%%%%%%%%%%
  
  \maketitle
  
  %%%%%%%%%%%%%%%%%%%%%%%%%%%%%%%%%%%%%%%%%%%%%%%%%%%%%%%%%%
  %%
  %% MARK: Epigraph
  %%
  %%%%%%%%%%%%%%%%%%%%%%%%%%%%%%%%%%%%%%%%%%%%%%%%%%%%%%%%%%
  
  \thispagestyle{empty}
  \vspace{0.5cm}
  \begin{flushright}
    \begin{minipage}{0.6\textwidth}
      \itshape\small
      \raggedleft
      We shall not cease from exploration\\
      And the end of all our exploring\\
      Will be to arrive where we started\\
      And know the place for the first time.
      \par\vspace{1ex}
      \upshape --- T.S. Eliot, \textit{Little Gidding}
    \end{minipage}
  \end{flushright}
  \vspace{1cm}
  
  %%%%%%%%%%%%%%%%%%%%%%%%%%%%%%%%%%%%%%%%%%%%%%%%%%%%%%%%%%
  %%
  %% MARK: Abstract
  %%
  %%%%%%%%%%%%%%%%%%%%%%%%%%%%%%%%%%%%%%%%%%%%%%%%%%%%%%%%%%
  
  \begin{abstract}
    \noindent It is a commonly held view in the foundations of mathematics,
    descending from Klein's Erlangen Program,
    that a geometry is determined by a topological space and a group of symmetries acting upon it.
    In this note,
    we present a rigorous counter-example to this intuition:
    the \emph{Wire Plane} ($\cW$).
    
    \noindent We prove that $\cW$ shares the exact same topology as the standard Euclidean plane $\RR^2$ and that its group of diffeomorphisms is algebraically identical to the standard group $\Diff(\RR^2)$.
    Yet,
    these two spaces are geometrically distinct:
    the standard plane has dimension 2,
    while the Wire Plane has dimension 1.
    
    \noindent This phenomenon,
    which we call the \emph{Boman Paradox} in reference to the theorem that underlies the group isomorphism,
    is resolved by observing that the identity map between the groups is not a diffeomorphism.
    Furthermore,
    we show that the space cannot be reconstructed as the quotient of its group.
    This result demonstrates that the ``geometry'' is not contained in the abstract group,
    nor in the topology.
    Rather,
    the \emph{local structure precedes the global symmetry}:
    the space defines the group,
    but the group is insufficient to reconstruct the space.
  \end{abstract}
  
  %%%%%%%%%%%%%%%%%%%%%%%%%%%%%%%%%%%%%%%%%%%%%%%%%%%%%%%%%%
  %%
  %% MARK: Introduction
  %%
  %%%%%%%%%%%%%%%%%%%%%%%%%%%%%%%%%%%%%%%%%%%%%%%%%%%%%%%%%%
  
  \section{Introduction}
  
  What constitutes a geometry?
  Since Felix Klein's Erlangen Program (1872),
  the standard answer has been:
  a space $X$ equipped with a group of transformations $G$ acting transitively upon it.
  This viewpoint reached its zenith with Jean-Marie Souriau,
  who famously proposed that ``the group is the geometry'' \cite{Sou84}.
  
  Souriau's program was radical and rigorous.
  He argued that the internal algebraic structure of the group---specifically the classification of its subgroups and normalizers---should be sufficient to characterize all geometric invariants,
  effectively reducing geometry to group theory.
  For instance,
  in the context of Euclidean geometry,
  the distance is not a primary ingredient;
  it can be reconstructed as the class of pairs of points under the group action.
  In this view,
  the space itself is secondary,
  often reconstructible as a quotient of the group (a homogeneous space).
  Seeking to extend this perspective to differential geometry,
  Souriau treated infinite-dimensional groups of diffeomorphisms as genuine geometric objects,
  a foundational step from which the theory of \emph{Diffeology} emerged.
  
  However,
  by introducing this structure,
  Souriau unwillingly provided the tool that would undermine his own radical thesis.
  While the abstract group captures the symmetries,
  it fails to capture crucial geometric invariants---such as dimension---which belong strictly to the structure of the space.
  
  In this note,
  we show that the assumption that ``topology plus algebra equals geometry'' is false in the context of differential geometry.
  We exhibit two spaces,
  the standard Euclidean plane $\RR^2$ and the \emph{Wire Plane} $\cW$,
  which are indistinguishable through the lens of classical topology and abstract algebra,
  yet are fundamentally distinct geometric objects.
  
  The distinction relies on Diffeology.
  By defining smoothness via \emph{plots} (probes from Euclidean domains into the space) rather than charts,
  diffeology allows us to assign a smooth structure to the plane that is strictly finer than the standard one.
  This new structure,
  the wire diffeology,
  preserves the topology and the group of symmetries,
  but collapses the dimension from 2 to 1.
  
  This phenomenon,
  which we call the \emph{Boman Paradox} in reference to the theorem that underlies the group isomorphism,
  highlights a crucial epistemological point:
  geometry cannot be reduced to the sum of Topology and Algebra.
  It requires a third,
  distinct ingredient---the \emph{local smooth structure}---which determines the rigidity of the space and the nature of its symmetries.
  
  %%%%%%%%%%%%%%%%%%%%%%%%%%%%%%%%%%%%%%%%%%%%%%%%%%%%%%%%%%
  %%
  %% MARK: The Wire Plane
  %%
  %%%%%%%%%%%%%%%%%%%%%%%%%%%%%%%%%%%%%%%%%%%%%%%%%%%%%%%%%%
  
  \section{The Wire Plane}
  
  To construct the counter-example,
  we must leave the restricted world of manifolds and enter the realm of \emph{Diffeology}.
  In standard differential geometry,
  a space is defined by \emph{charts} that flatten it onto Euclidean space.
  In diffeology,
  we invert this perspective:
  a space is defined by \emph{plots}---probes that map Euclidean domains \emph{into} the space.%
  \footnote{See formal definition of diffeological space in \cite[\textsection 1.5]{PIZ13}.}
  
  Let us define our central object,
  the \emph{Wire Plane} $\cW$.
  The underlying set is the standard Cartesian plane $\RR^2$.
  However,
  we rigidly restrict which parametrizations count as ``smooth.''
  
  \begin{definition}[The Wire Diffeology]
    A parametrization $P: U \to \RR^2$ (where $U \subset \RR^n$ is open) is a \emph{plot} of the Wire Plane $\cW$ if and only if,
    locally around every point,
    $P$ factors through a smooth curve.
    That is,
    for every $u \in U$,
    there exists a neighborhood $V$,
    a smooth curve $\gamma: \RR \to \RR^2$,
    and a smooth function $q: V \to \RR$ such that $P|_V = \gamma \circ q$.
  \end{definition}
  
  \begin{figure}[htb]
    \centerline{\includegraphics[width=0.6\textwidth]{Figures/Wire-plot.pdf}}
    \vspace{0ex}
    \caption{\textbf{ Figure \ref{wire-plot}---The factorization property.} Every plot $P$ in $\cW$ is locally of the form $\gamma \circ q$, where $\gamma$ is a smooth curve and $q$ is a smooth real-valued function.}
    \label{wire-plot}
  \end{figure}
  
  This definition satisfies the axioms of a diffeology:
  it covers the set,
  it is local by definition,
  and it is stable under smooth composition.
  
  \textbf{Example.}
  The map $P(u,v) = (\sin(u^2+v^2), \cos(u^2+v^2))$ is a valid plot of $\cW$,
  as it factors through the circle $\gamma(t) = (\sin t, \cos t)$ via $q(u,v) = u^2+v^2$.
  Conversely,
  the identity map $\id(u,v) = (u,v)$ is \emph{not} a plot of $\cW$.
  It cannot be compressed into a 1-dimensional factor.
  
  This brings us to the first geometric divergence: \emph{Dimension}.
  In diffeology,
  dimension is defined by a minimax principle:
  it is the infimum of the dimensions of the domains required to generate the plots.%
  \footnote{See definition of diffeological dimension in \cite[\textsection 1.78]{PIZ13}.}
  Since every plot in $\cW$ is built from curves (domain dimension 1),
  and the space is not discrete (meaning generated by constant plots),%
  \footnote{A diffeological space is discrete if its plots are locally constant. See \cite[\textsection 1.20]{PIZ13}.}
  we immediately have:
  $$
    \dim(\cW) = 1.
    $$
  Intuitively,
  the ``wire'' structure forbids any 2-dimensional sliding motion.
  While the set of points is the plane,
  the \emph{mode of access} is strictly 1-dimensional.
  
  \textbf{Remark.}
  The flexibility of this framework has proven useful well beyond the bounds of differential geometry.
  For instance,
  the probabilist Boris Tsirelson employed diffeology as early as 2006 to handle the singular quotient structures arising from Brownian local minima \cite{Tsi06}.
  Interestingly,
  Tsirelson relied on an early ``samizdat'' version of the foundational text \cite{PIZ13} available online,
  recognizing the benefits of using plots to describe structures where standard topology failed.
  This highlights that diffeology is not merely a generalization for its own sake,
  but a tool for analyzing the ``wild'' spaces that naturally occur in mathematics.
  
  %%%%%%%%%%%%%%%%%%%%%%%%%%%%%%%%%%%%%%%%%%%%%%%%%%%%%%%%%%
  %%
  %% MARK: The Topological Identity
  %%
  %%%%%%%%%%%%%%%%%%%%%%%%%%%%%%%%%%%%%%%%%%%%%%%%%%%%%%%%%%
  
  \section{The Topological Identity}
  
  The first step in establishing the paradox is to show that the Wire Plane and the Standard Plane are indistinguishable from a topological perspective.
  Every diffeological space carries a natural topology,
  called the \emph{D-topology},
  which is the finest topology making all plots continuous.%
  \footnote{See definition of the D-topology in \cite[\textsection 2.8]{PIZ13}.}
  
  Let $\tau_{\text{std}}$ denote the standard Euclidean topology on $\RR^2$,
  and $\tau_{\cW}$ denote the D-topology of the Wire Plane.
  
  \begin{proposition}
    The D-topology of the Wire Plane coincides with the standard Euclidean topology:
    $$\tau_{\cW} = \tau_{\text{std}}.$$
  \end{proposition}
  
  \begin{proof}
    The proof proceeds by double inclusion.
    
    \textbf{1.} Inclusion $\tau_{\text{std}} \subseteq \tau_{\cW}$.
    The generating plots of $\cW$ are the standard smooth curves.
    Since every standard smooth curve is continuous with respect to the Euclidean topology $\tau_{\text{std}}$,
    the standard topology satisfies the condition of the D-topology.
    Thus,
    the D-topology $\tau_{\cW}$ must contain the standard topology $\tau_{\text{std}}$.
    
    \textbf{2.} Inclusion $\tau_{\cW} \subseteq \tau_{\text{std}}$.
    We proceed by contradiction.
    Assume there exists a set $U \in \tau_{\cW}$ that is not Euclidean open.
    This implies there exists a point $x \in U$ and a sequence of points $\{x_n\}$ in the complement $U^c$ such that $x_n \to x$ in the Euclidean norm.
    
    To derive a contradiction,
    we construct a plot of $\cW$ (a smooth curve $\gamma$) that passes through a subsequence of these points and converges to $x$.
    We use a standard ``smashing'' technique (see Figure \ref{fig:plateau}).
    
    Let $\lambda: \RR \to [0,1]$ be a smooth transition function such that $\lambda(t) = 0$ for $t \le 0$ and $\lambda(t) = 1$ for $t \ge 1$.
    Let $\{t_k\}$ be a sequence of times strictly decreasing to $0$.
    We define $\gamma$ piecewise on intervals $[t_{k+1}, t_k]$ to interpolate between a subsequence $\{x_{n_k}\}$.
    By scaling the transition function appropriately,
    we ensure that all derivatives vanish at $0$,
    making $\gamma$ a smooth curve in $\RR^2$.
    
    \begin{figure}[thb]
      \centerline{\includegraphics[width=0.33\textwidth]{Figures/Smashing-function.pdf}}
      \vspace{0ex}
      \caption{\textbf{Figure \ref{fig:plateau}---A smooth transition function.} The function $\lambda$ used to connect points smoothly.}
      \label{fig:plateau}
    \end{figure}
    
    Since $\gamma$ is a plot of $\cW$,
    it must be continuous in the D-topology.
    Thus,
    the preimage $V = \gamma^{-1}(U)$ must be open in $\RR$.
    However,
    $0 \in V$ (since $\gamma(0)=x \in U$),
    but the sequence $t_k$ converges to $0$ and lies entirely in the complement (since $\gamma(t_k) = x_{n_k} \notin U$).
    This contradicts the openness of $V$.
    Therefore,
    every D-open set must be Euclidean open.
  \end{proof}
  
  %%%%%%%%%%%%%%%%%%%%%%%%%%%%%%%%%%%%%%%%%%%%%%%%%%%%%%%%%%
  %%
  %% MARK: The Algebraic Identity
  %%
  %%%%%%%%%%%%%%%%%%%%%%%%%%%%%%%%%%%%%%%%%%%%%%%%%%%%%%%%%%
  
  \section{The Algebraic Identity}
  
  The second step is to show that the groups of symmetries of these two spaces are identical.
  
  In diffeology,
  a map is \emph{smooth} if it maps plots to plots,
  and a \emph{diffeomorphism} is a smooth bijection with a smooth inverse.%
  \footnote{See definition of smooth maps and diffeomorphisms in \cite[\textsection 1.14 and 1.17]{PIZ13}.}
  
  We now compare $\Diff(\cW)$ with the standard group $\Diff(\RR^2)$.
  The equality of these groups relies on a deep result in analysis known as \emph{Boman's Theorem} (1967) \cite{Bom67}.
  
  \begin{theorem*}[Boman]
    A function $f: \RR^n \to \RR^m$ is smooth (of class $C^\infty$) if and only if it maps smooth curves to smooth curves.
  \end{theorem*}
  
  \textbf{Intuition.}
  One might suspect that testing a function along straight lines (Gateaux differentiability) would be sufficient to detect smoothness.
  However,
  there exist functions that are smooth along every line passing through the origin but are not even continuous.
  Boman's theorem asserts that if we test along \emph{all} smooth curves,
  we recover full smoothness.
  This is the engine that drives the following proposition.
  
  \begin{proposition}
    The group of diffeomorphisms of the Wire Plane is identical as a set to the group of standard diffeomorphisms of $\RR^2$:
    $$
      \Diff(\cW) = \Diff(\RR^2).
      $$
  \end{proposition}
  
  \begin{proof}
    \textbf{1. Inclusion $\Diff(\RR^2) \subseteq \Diff(\cW)$}.
    Let $\phi$ be a standard diffeomorphism.
    Let $P$ be a plot in $\cW$.
    Locally,
    $P$ factors as $\gamma \circ q$,
    where $\gamma$ is a smooth curve.
    Then $\phi \circ P = \phi \circ \gamma \circ q$.
    Since $\phi$ is standard smooth,
    $\gamma' = \phi \circ \gamma$ is a smooth curve.
    Thus,
    $\phi \circ P$ factors through $\gamma'$,
    making it a plot of $\cW$.
    The same applies to $\phi^{-1}$.
    
    \textbf{2. Inclusion $\Diff(\cW) \subseteq \Diff(\RR^2)$}.
    Let $\phi \in \Diff(\cW)$.
    By definition,
    $\phi$ must map plots of $\cW$ to plots of $\cW$.
    Since smooth curves are plots of $\cW$,
    $\phi$ must map smooth curves to smooth curves.
    By Boman's Theorem,
    this implies that $\phi$ is smooth in the standard sense.
    Since $\phi^{-1}$ is also in $\Diff(\cW)$,
    it is also standard smooth.
    Thus,
    $\phi \in \Diff(\RR^2)$.
  \end{proof}
  
  \textbf{Remark on Filipkiewicz's Theorem.}
  The result above---that these two groups are identical as sets but the spaces are not diffeomorphic---stands in sharp contrast to the celebrated theorem of Filipkiewicz \cite{Fil82}.
  In the category of smooth manifolds,
  the algebraic structure of the group of diffeomorphisms determines the smooth structure of the manifold:
  if $\Diff(M) \cong \Diff(N)$ as groups,
  then $M \cong N$ as manifolds.
  The Boman Paradox demonstrates that this rigidity theorem fails in the broader context of diffeology.
  Here,
  the group isomorphism is the identity map,
  yet the spaces have different dimensions.
  
  %%%%%%%%%%%%%%%%%%%%%%%%%%%%%%%%%%%%%%%%%%%%%%%%%%%%%%%%%%
  %%
  %% MARK: The Geometric Schism
  %%
  %%%%%%%%%%%%%%%%%%%%%%%%%%%%%%%%%%%%%%%%%%%%%%%%%%%%%%%%%%
  
  \section{The Geometric Schism}
  
  Despite their topological and algebraic identity,
  the Standard Plane and the Wire Plane are geometrically distinct.
  This distinction manifests in two profound ways:
  their dimension and the smooth structure of their symmetry groups.
  
  We apply the minimax definition of dimension.
  For the Standard Plane $\RR^2$,
  the diffeology is generated by the single identity map $\id: \RR^2 \to \RR^2$.
  The family $\{\id\}$ has dimension 2.
  Since the identity has rank 2,
  it cannot be generated by plots of lower dimension.
  Thus,
  $\dim(\RR^2) = 2$.
  
  For the Wire Plane $\cW$,
  the diffeology is generated by the family of all smooth curves $\mathcal{C} = C^\infty(\RR, \RR^2)$.
  The domain of every plot in this family is 1-dimensional.
  Since the space is not discrete,
  the dimension is exactly 1.
  
  \begin{proposition}
    The diffeological dimensions differ:
    $$
      \dim(\RR^2) = 2 \quad \text{while} \quad \dim(\cW) = 1.
      $$
  \end{proposition}
  
  If we consider diffeology as an expression of geometry,
  two spaces of different dimensions cannot embody the same geometric object.
  This dimensional collapse is the first half of the paradox.
  
  %%%%%%%%%%%%%%%%%%%%%%%%%%%%%%%%%%%%%%%%%%%%%%%%%%%%%%%%%%
  %%
  %% MARK: The Non-Smoothness of the Isomorphism
  %%
  %%%%%%%%%%%%%%%%%%%%%%%%%%%%%%%%%%%%%%%%%%%%%%%%%%%%%%%%%%
  \section{The Non-Smoothness of the Isomorphism}
  
  The paradox is resolved by examining the \emph{diffeological} nature of the isomorphism between the groups.
  To do this,
  we must view the group of diffeomorphisms not just as a set,
  but as a diffeological space itself.
  We equip $\Diff(X)$ with the \emph{functional diffeology},
  where a parametrization in the group is smooth if and only if its action on the space is smooth.%
  \footnote{Formally, a parametrization $P: U \to \Diff(X)$ is a plot if the evaluation map $(u,x) \mapsto P(u)(x)$ is a smooth map $U \times X \to X$. See \cite[\textsection 1.57 and 1.61]{PIZ13}.}
  
  Let $j: \Diff(\RR^2) \to \Diff(\cW)$ be the identity map on the set of bijections.
  Is it a diffeomorphism?
  
  \begin{theorem}
    The map $j: \Diff(\RR^2) \to \Diff(\cW)$ is not smooth.
    Consequently,
    the groups are distinct as diffeological groups.
  \end{theorem}
  
  \begin{proof}
    Consider the path of translations $P: \RR \to \Diff(\RR^2)$ defined by $P(t)(x,y) = (x+t, y)$.
    This is a smooth path (a 1-plot) in the standard functional diffeology of $\Diff(\RR^2)$.
    
    To test if $P$ is a plot of $\Diff(\cW)$,
    we must check if the associated evaluation map $\text{ev}_P: \RR \times \cW \to \cW$ is smooth.
    Let us probe this map with a specific plot $Q: \RR^2 \to \RR \times \cW$ defined by $Q(u,v) = (u, (0,v))$.
    Note that $v \mapsto (0,v)$ is a valid wire plot (a vertical line).
    
    The composition $\Phi = \text{ev}_P \circ Q$ is:
    $$
      \Phi(u,v) = P(u)(0,v) = (u, v).
      $$
    This is the identity map on $\RR^2$.
    However,
    as we established in Section 2,
    the identity map is \emph{not} a plot of $\cW$ because it has rank 2 and cannot be factored through a curve.
    
    Hence,
    the evaluation map is not smooth.
    Therefore,
    the path $P$ is not a plot of $\Diff(\cW)$,
    and the inclusion $j$ is not smooth.
  \end{proof}
  
  \textbf{Remark.}
  The inverse map $j^{-1}: \Diff(\cW) \to \Diff(\RR^2)$ \emph{is} smooth.
  The diffeology of $\Diff(\cW)$ is strictly \emph{finer} than that of $\Diff(\RR^2)$.
  The group of the Wire Plane is more rigid;
  it accepts fewer smooth paths than the standard group.
  
  %%%%%%%%%%%%%%%%%%%%%%%%%%%%%%%%%%%%%%%%%%%%%%%%%%%%%%%%%%
  %%
  %% MARK: The Failure of the Quotient
  %%
  %%%%%%%%%%%%%%%%%%%%%%%%%%%%%%%%%%%%%%%%%%%%%%%%%%%%%%%%%%
  
  \section{The Failure of the Quotient}
  
  In classical Lie theory,
  a transitive action of a group $G$ on a space $X$ allows us to identify $X$ with the quotient $G/H$,
  where $H$ is the stabilizer of a point.%
    \footnote{When a group $G$ acts on a set $X$,
  the stabilizer (or isotropy) of a point $x$ is the subgroup of $G$ that fix $x$.
  Formally denoted by $\stab_G(x) = \{ h \in G \mid h(x) = x \}$.}
  In diffeology,
  this identification corresponds to the notion of a \emph{subduction}.%
  \footnote{A map $\pi: X \to Y$ is a subduction if the plots of $Y$ are exactly the local push-forwards of plots of $X$. See \cite[\textsection 1.46 and 1.50]{PIZ13} for subductions and quotients.}
  
  \begin{theorem}
    The orbit map $\hat{0}: \Diff(\cW) \to \cW$ defined by $\hat{0}(\phi) = \phi(0)$ is not a subduction.
    In other words,
    the Wire Plane is not diffeomorphic to the quotient of its symmetry group.
  \end{theorem}
  
  \begin{proof}
    For $\hat{0}$ to be a subduction,
    every plot of the codomain $\cW$ must locally lift to a plot of the domain $\Diff(\cW)$.
    Consider the simple plot $\gamma: \RR \to \cW$ defined by the horizontal axis: $\gamma(t) = (t, 0)$.
    We seek a plot $P: \RR \to \Diff(\cW)$ such that $\hat{0} \circ P = \gamma$,
    which means $P(t)(0) = (t, 0)$ for all $t$.
    
    For $P$ to be a plot of $\Diff(\cW)$,
    the evaluation map $\text{ev}_P: \RR \times \cW \to \cW$ must be smooth.
    Let us test this with the vertical wire $c(s) = (0, s)$.
    We define the map $\Phi: \RR^2 \to \cW$ by evaluating the path $P$ on the wire $c$:
    $$
      \Phi(t, s) = P(t)(c(s)).
      $$
    Let us compute the rank of this map at $(0,0)$.
    We may assume $P(0) = \id$.
    \begin{itemize}
      \item The partial derivative with respect to $s$ is determined by the curve $c$:
      $\partial_s \Phi(0,0) = (0, 1)$.
      \item The partial derivative with respect to $t$ is determined by the curve $\gamma$:
      $\partial_t \Phi(0,0) = (1, 0)$.
    \end{itemize}
    The Jacobian matrix is the identity.
    Thus,
    $\Phi$ has rank 2.
    
    However,
    every plot of $\cW$ must locally factor through a smooth curve,
    so its rank is at most 1.
    Therefore,
    $\Phi$ is not a plot of $\cW$.
    This contradiction implies that the path $P$ cannot exist in $\Diff(\cW)$.
    The curve $\gamma$ cannot be lifted.
  \end{proof}
  
  \begin{SCfigure}[50][htbp]
    \includegraphics[width=0.5\textwidth]{Figures/The-wire.pdf}
    \caption{\small\textbf{Figure~\ref{fig:The-wire} ---~The Forbidden Trans\-versa\-lity.}
    An artistic representation of the \emph{Wire Diffeology}.
    The space comprises the entire continuum of points in the plane (the background),
    but its geometry is defined strictly by how these points are \emph{accessed}.
    The drawing illustrates the generating family of plots: a dense arborescence of smooth curves where branches peel away tangentially.
    While you still need two numbers to mark a point on the plane,
    the ``legal'' motions are constrained to these 1-dimensional channels.
    Sweeping areas
    ---~the transverse motion~---
    is forbidden.
    \\ \null \hfill{(Image AI Gemini)}}
    \label{fig:The-wire}
  \end{SCfigure}
  
  \textbf{Remark: The Forbidden Transversality.}
  The proofs of Theorem 1 and Theorem 2 rely on the same construction.
  In both cases,
  we combine a horizontal motion from the group with a vertical wire from the space to generate a 2-dimensional map.
  Since the Wire Plane cannot support such maps,
  the construction fails.
  This reveals the fundamental rigidity of the Wire Plane:
  one can slide \emph{along} a wire,
  but one cannot slide \emph{the} wire itself.
  
  %%%%%%%%%%%%%%%%%%%%%%%%%%%%%%%%%%%%%%%%%%%%%%%%%%%%%%%%%%
  %%
  %% MARK: Conclusion
  %%
  %%%%%%%%%%%%%%%%%%%%%%%%%%%%%%%%%%%%%%%%%%%%%%%%%%%%%%%%%%
  
\section{Conclusion}
  
  The Boman Paradox serves as a cautionary tale for the foundations of geometry.
  It demonstrates that the classical invariants
  ---~topology and the abstract group of symmetries~---
  are insufficient to characterize a geometric structure.
  
  The Wire Plane $\cW$ and the Standard Plane $\RR^2$ are indistinguishable to the topologist and the algebraist.
  Yet,
  to the geometer equipped with the tools of diffeology,
  they are worlds apart.
  This distinction is captured by the \emph{diffeological group}:
  $\Diff(\cW)$ and $\Diff(\RR^2)$ are distinct as diffeological spaces,
  because the former forbids the transverse sliding of plots.
  
  This investigation is not a purely formal exercise.
  The Wire Plane $\cW$ first emerged as the natural geometric setting for the ``Two-Thirds Power Law'' of human motor control \cite{PIZ25}.
  In that model,
  the brain's internal representation of space is functionally 1-dimensional,
  apprehending the world through the paths it can trace.
  The Boman Paradox thus has a concrete physical interpretation:
  the symmetries of this neural geometry are insufficient to reconstruct the geometry itself.
  
  This result forces a re-evaluation of the hierarchy between Space and Group.
  Jean-Marie Souriau famously posited that ``the group is the geometry'' \cite{Sou84}.
  However,
  by introducing diffeology to define the structure of these infinite-dimensional groups,
  Souriau unwittingly changed the rules of the game.
  The Boman Paradox demonstrates that within this broader framework,
  the ``manifold property'' (homogeneity via subduction) is the exception,
  not the rule.
  The arrow of definition is irreversible:
  the space uniquely determines the group,
  but the group---even with its diffeological structure---cannot reconstruct the space.
  
  This result requires a refinement of the Kleinian ideal:
  a geometry is not determined by the abstract group,
  but by the pair formed by the diffeological space and its diffeological symmetry group,
  which is deduced from it but cannot be separated from it.
  
  %%%%%%%%%%%%%%%%%%%%%%%%%%%%%%%%%%%%%%%%%%%%%%%%%%%%%%%%%%
  %%
  %% MARK: Acknowledgements
  %%
  %%%%%%%%%%%%%%%%%%%%%%%%%%%%%%%%%%%%%%%%%%%%%%%%%%%%%%%%%%
  
  \section*{Acknowledgements}
  I thank the Hebrew University of Jerusalem for its continuous academic support.
  I am also grateful to the AI assistant Gemini (Google) for its assistance in the writing of this short article.
  Data sharing is not applicable to this article as no new data were created or analyzed in this study.
  
  %%%%%%%%%%%%%%%%%%%%%%%%%%%%%%%%%%%%%%%%%%%%%%%%%%%%%%%%%%
  %%
  %% MARK: Bibliography
  %%
  %%%%%%%%%%%%%%%%%%%%%%%%%%%%%%%%%%%%%%%%%%%%%%%%%%%%%%%%%%
  
  \begin{thebibliography}{99}
    
    \bibitem{Bom67}
    \textbf{J. Boman},
    \textit{Differentiability of a function and of its compositions with functions of one variable}.
    {Mathematica Scandinavica},
    \textbf{20} (1967),
    p.~{249--268}.
    
    \bibitem{Don84}
    \textbf{P. Donato},
    \textit{Revêtements de groupes différentiels},
    Thèse de doctorat d'État,
    Université de Provence, Marseille, 1984.
    
    \bibitem{Fil82}
    \textbf{R. P. Filipkiewicz},
    \textit{Isomorphisms between diffeomorphism groups},
    Ergodic Theory Dynam. Systems,
    \textbf{2} (1982), no. 2, p.~159--171.
    
    \bibitem{PIZdim}
    \textbf{P. Iglesias-Zemmour},
    \textit{Dimension in Diffeology},
    Preprint, The Hebrew University of Jerusalem (2006).
    
    \bibitem{PIZ13}
    \bysame %\textbf{P. Iglesias-Zemmour},
    \textit{Diffeology},
    Mathematical Surveys and Monographs,
    \textbf{185},
    AMS, Providence, RI, 2013.
    
    \bibitem{PIZ25}
    \bysame %\textbf{P. Iglesias-Zemmour},
    \textit{Is the brain a 1-dimensional diffeological space? The Geometry of the Two-Thirds Power Law},
    Preprint, The Hebrew University of Jerusalem  (2025). \\
    {\scriptsize\url{https://math.huji.ac.il/~piz/documents/ITBA1DDS.pdf}}.
    
    \bibitem{Sou84}
    \textbf{J.-M. Souriau},
    \textit{Les groupes comme universaux},
    Scientia,
    \textbf{119} (1984), p.~171--177.
    
    \bibitem{Tsi06}
    \textbf{B. Tsirelson},
    \textit{Brownian local minima, random dense countable sets and random equivalence classes},
    Electronic Journal of Probability,
    \textbf{11} (2006), p.~162--298.
    
  \end{thebibliography}
\end{document}
