%%%%%%%%%%%%%%%%%%%%%%%%%%%%%%%%%%%%%%%%%%%%%%%%%%%%%%%%%%
%%
%%  Orbifolds As Stratified Diffeologies
%%  Serap Gürer & Patrick Iglesias-Zemmour
%%  Archived: 2025
%%
%%%%%%%%%%%%%%%%%%%%%%%%%%%%%%%%%%%%%%%%%%%%%%%%%%%%%%%%%%

\documentclass[11pt,reqno,letterpaper,twoside]{amsart}

%--------------------------------------------------------------------------
% Packages
%--------------------------------------------------------------------------
\usepackage[utf8]{inputenc}
\usepackage[T1]{fontenc}
\usepackage{amssymb}
\usepackage{amscd}
\usepackage{graphicx}
\usepackage[normalem]{ulem} % For \sout (strikeout)
\usepackage[margin=10pt,font=small,labelfont=bf, labelformat=empty, labelsep=endash]{caption}
\usepackage[hidelinks]{hyperref}

% TikZ and Diagrams
\usepackage{tikz-cd}
\usetikzlibrary{calc}
\tikzcdset{
    arrow style=tikz,
    diagrams={>={Straight Barb[scale=0.8]}}
}
\tikzset{
    symbol/.style={
        draw=none,
        every to/.append style={
            edge node={node [sloped, allow upside down, auto=false]{$#1$}}
        }
    }
}

%--------------------------------------------------------------------------
% Page Layout & Typography
%--------------------------------------------------------------------------
\parindent 0mm
\parskip .5ex plus 2pt
\linespread{1.1}

%--------------------------------------------------------------------------
% Theorem Styles
%--------------------------------------------------------------------------
\newtheoremstyle{article}% ⟨name⟩
{7pt}%  ⟨Space above⟩
{7pt}%  ⟨Space below⟩
{}%     ⟨Body font⟩
{0pt}%  ⟨Indent amount⟩
{\bf}%  ⟨Theorem head font⟩
{.---\ }%  ⟨Punctuation after theorem head⟩
{0pt}%  ⟨Space after theorem head⟩
{}%     ⟨Theorem head spec⟩

\theoremstyle{article}
\newtheorem{article}{}

\def\artlabel[#1]{\textsc{\textsc{#1.}}}
\newcommand{\art}[1]{(\textsection~\ref{#1})}
\newcommand{\See}[1]{(See \textsection~\ref{#1})}
\def\qtext#1{\quad\text{#1}\quad}
\newcommand{\remend}{\nolinebreak\hfill $\blacktriangleright$}

% Environments
\newenvironment{definition}[1]%
{\noindent\textsc{Definition#1.} \it}%
{}

\newenvironment{proposition}[1]%
{\noindent\textsc{Proposition#1.} \it}%
{}

\newenvironment{theorem}[1]%
{\noindent\textsc{Theorem#1.} \it}%
{}

\renewenvironment{proof}%
{\noindent\textit{Proof.}}%
{\nolinebreak\hfill $\square$}

%--------------------------------------------------------------------------
% Macros
%--------------------------------------------------------------------------

% Sets
\newcommand{\RR}{\mathbf{R}}
\newcommand{\CC}{\mathbf{C}}
\newcommand{\QQ}{\mathbf{Q}}
\newcommand{\ZZ}{\mathbf{Z}}

% Calligraphic
\newcommand{\cA}{{\mathcal A}}
\newcommand{\cB}{{\mathcal B}}
\newcommand{\cD}{{\mathcal D}}
\newcommand{\cF}{{\mathcal F}}
\newcommand{\cG}{{\mathcal G}}
\newcommand{\cI}{{\mathcal I}}
\newcommand{\cK}{{\mathcal K}}
\newcommand{\cO}{{\mathcal O}}
\newcommand{\cQ}{{\mathcal Q}}
\newcommand{\cS}{{\mathcal S}}
\newcommand{\cT}{{\mathcal T}}
\newcommand{\cU}{{\mathcal U}}

% Roman Letters (Used in text)
\def \A{\mathrm A}
\def \D{\mathrm D}
\def \F{\mathrm F}
\def \G{\mathrm G}
\def \H{\mathrm H}
\def \K{\mathrm K}
\def \M{\mathrm M}
\def \N{\mathrm N}
\def \P{\mathrm P}
\def \Q{\mathrm Q}
\def \S{\mathrm S}
\def \T{\mathrm T}
\def \U{\mathrm U}
\def \V{\mathrm V}
\def \X{\mathrm X}

% Operators
\DeclareMathOperator{\Cinfty}{\mathrm{C}^\infty}
\DeclareMathOperator{\Cone}{Cone}
\DeclareMathOperator{\class}{class}
\DeclareMathOperator{\Diff}{Diff}
\DeclareMathOperator{\dom}{dom}
\DeclareMathOperator{\Geod}{Geod}
\DeclareMathOperator{\GL}{GL}
\DeclareMathOperator{\pr}{pr}
\DeclareMathOperator{\sq}{sq}
\DeclareMathOperator{\Stab}{St}
\DeclareMathOperator{\Str}{Str}

% Symbols
\newcommand{\id}{\mathbf 1}
\newcommand{\loc}{\mathrm{loc}}
\newcommand{\norm}[1]{||#1||}
\newcommand{\Sq}{\mathrm{Sq}}
\newcommand{\vect}[1]{\begin{pmatrix}#1\end{pmatrix}}
\newcommand{\scal}[2]{\langle #1,#2\rangle}
\let \epsilon=\varepsilon

%--------------------------------------------------------------------------
% Title Data
%--------------------------------------------------------------------------

\begin{document}

\title{Orbifolds As Stratified Diffeologies}

\author{Serap Gürer}
\address{Serap Gürer,
Galatasaray University,
Ortaköy, Çırağan Cd. No:36,
34349 Beşiktaş\,/\,İstanbul, Turkey.}
\email{sgurer@gsu.edu.tr}

\author{Patrick Iglesias-Zemmour}
\address{Patrick Iglesias-Zemmour,
Einstein Institute of Mathematics,
The Hebrew University of Jerusalem,
Edmond J. Safra Campus,
Givat Ram,
9190401 Jerusalem,
Israel.
}
\email{piz@math.huji.ac.il}
\urladdr{http://math.huji.ac.il/~piz}

\date{August 2022 --- January 2023}

\keywords{Orbifolds, Diffeology, Stratification.}
\subjclass{58A35, 58A10}

\maketitle

\begin{abstract}
We discuss general properties of stratified spaces in diffeology.
This leads to a formal framework for the theory of stratifications.
In particular, we consider the Klein stratification of diffeological orbifolds,
defined by the action of local diffeomorphisms.
We show that it is a standard stratification in the sense that the partition of the space into orbits of local diffeomorphisms is locally finite
(for orbifolds with locally finite atlases),
it satisfies the frontier condition and the orbits are locally closed manifolds.
\end{abstract}

%%%%%%%%%%%%%%%%%%%%%%%%%%%%%%%%%%%%%%%%%%%%%%%%%%%%%%%%%%
%%
%% MARK: Introduction
%%
%%%%%%%%%%%%%%%%%%%%%%%%%%%%%%%%%%%%%%%%%%%%%%%%%%%%%%%%%%
\section*{Introduction}

A recurrent problem in the theory of stratified spaces is the  competition of two structures:
the topology of the global space and the smooth manifold structure of the strata within.
There are a few proposals to reconcile the two,
meaning to exhibit a smooth structure,
in some sense,
that covers at the same time the ambient topology and the manifold structure of the strata.
See for example \cite{AFLT17,JW17,MCNM18} to cite a few.

Diffeology gives a global answer to this question,
thanks to the versatility of its objects.
By nature,
diffeology is not hostile to singularities which are naturally absorbed,
or encrypted,
in its definition.
A good illustration would be the teardrop diffeology defined on the $2$-sphere \cite[Fig. 2]{IZL18}.
Manifolds with boundary and corners \cite[\textsection 4.16]{PIZ13} \cite{GIZ19},
orbifolds \cite{IKZ10,IZL18},
quasifolds \cite{IZP21},
subspaces,
singular quotients even with infinite dimensions \cite{PIZ16-a}.
All these objects,
or situations,
are well described by diffeology without the help of extra structures or heuristic definitions.
Some of them,
like manifolds with corners,
come with an obvious stratification:
the vertices,
the edges,
the faces etc. that are clearly identified by the diffeology only.
This is the reason why the theory of stratification in diffeology is not of same nature than in traditional differential geometry.
In particuler,
there is no competition in diffeology between the global structure and the structure of the strata since they inherit naturally,
by induction,
their smooth structure from the diffeology of the ambient space \cite[\textsection 1.29,1.33]{PIZ13}.
This is especially the case for most of the examples of stratified spaces that are at the origin of the classical theory,
such as algebraic sets,
manifolds with boundaries or corners,
simplices etc.

The only question remaining is to define properly what a stratified space is,
to discuss its variants and to study its particular properties from the point of view of diffeology.

The first step would be to identify the natural extension to diffeology of the usual notion of topological stratified space.
It consists,
roughly speaking,
to replace the word ``topology'' by ``diffeology''.
That is what we will call the \emph{standard stratification}.
This is defined,
on a space $\X$,
as a partition into a set $\cS$ of subsets called strata,
that satisfies the three conditions of stratified topological spaces for the D-topology of $\X$ \cite[\textsection 2.8]{PIZ13},
%  \footnote{See \cite{PI85} and \cite[\textsection 2.8]{PIZ13}.}\n
which are:
The frontier condition,
to be locally finite and the strata to be locally closed submanifold for the induced diffeology.

It is clear however that these three conditions are not of the same nature,
the most important and fundamental property is certainly the \emph{frontier condition} which epitomizes the concept of stratification and which stipulates that the topological closure of a stratum is a union of strata.
The topology involved here is naturally the D-topology.
The other properties:
That the set of strata is locally finite and the strata are locally closed manifolds appear as secondary labels characterizing the type of stratification considered.
We can imagine a few others,
locally conical for example,
and so on.

On the other hand,
a natural stratification defined by the action of diffeomorphisms pre-exists on any diffeological spaces:
The decomposition into \emph{Klein strata},
introduced in \cite[\textsection 1.42]{PIZ13}.
This \emph{Klein stratification} satisfies the fundamental \emph{frontier condition},
and admits a few variants based on the same principle \cite{PIZ22-a}.
Hence,
every diffeological space is structurally stratified.
The action of the diffeomorphisms,
or local diffeomorphisms,
delivers a stratification internally hidden or embedded in the diffeology.
This stratification is irrelevant for manifolds%
\footnote{Manifolds are usually considered Hausdorff and second-countable.}
since connected manifolds are homogeneous under the action of diffeomorphisms,
local or not.
Klein stratification is reserved for diffeological spaces that are not manifolds,
as it reveals the essential singularities embedded in the diffeology,
if any.
We can think here again of manifolds with corners,
for example.

Of course,
for a general diffeological space the Klein strata may be not manifolds,
but the main property of stratifications is satisfied.
This is the reason why we introduce the notion of \emph{basic stratification} in diffeology as partitions that satisfy the frontier condition.
Indeed,
from the point of view of diffeology  it is not necessary to enforce the structure of the strata since they inherit naturally the induced diffeology,
that make them diffeological subspaces.
Whatever the strata is,
we stay inside the category \{Diffeology\}.
On the other hand,
in classical differential geometry it is important to require that the strata are manifolds since it is the only smooth structure known by the theory.
This constraint then vanishes in diffeology.

Thus,
in diffeology,
stratifications begin with the frontier condition which is already satisfied by its own geometry defined by the action of the group of diffeomorphisms,
or by local diffeomorphisms.
However,
this Klein stratification is the maximal case of stratifications defined by the smooth actions of diffeological groups.
Indeed since diffeomorphisms are homeomorphism,
for a smooth action of a diffeological group the closure of an orbit is a union of orbits.
This consideration leads us to the definition of \emph{geometric stratifications},
the ones defined by smooth action of diffeological groups.
On the other hand,
a stratification which does not satisfy the geometric condition will be called a \emph{formal stratification}.

A simple example of a geometric stratification which is not Klein,
is the partition of the real line $\RR$\,:
$\S_- = \{x <0\}$, %]-\infty,0[$,
$\S_0 = \{0\}$,
$\S_+ = \{x>0\}$. %]0, +\infty[$.
They are the orbit of the multiplication by a positive number.
However,
as topological stratification, it is equivalent to the Klein stratification of the subspace of $\RR^2$ union of:
$\S_- = \{(0,y) \mid y>0\}$,
$\S_0 = \{(0,0)\}$,
$\S_+ = \{(x,0) \mid x>0\}$.

Until now,
we introduced implicitely three labels:
The first one [B] for basic,
means we have a partition which is a basic stratification.
The second one [LF] for locally finite,
and [\sout{LF}] otherwise.
The third label [G] when the stratification is geometric.
We reserve the label [GK] to notify a Klein stratification.

Next,
we have to consider the nature of the strata.
To respect the traditional approach we will introduce the label [M] for manifold.
It will indicate that all the strata are submanifolds for the induced diffeology.
We shall use the label [\sout{M}] in the opposite case.
We give the example of the space of geodesic trajectories of the torus $\T^2$,
which is not a manifold,
and for which a countable infinite subset of Klein strata are diffeomorphic to a circle,
and are irrational tori otherwise.
That does not prevent the theory of stratifications from working properly,
and distinguish between strata.
Diffeology takes over,
with the action of local diffeomorphisms,
when topology is insufficient.

In the traditional theory of stratification,
particular attention is paid to whether the strata are locally closed or not.
It turns out that this condition is related to topological properties of the space $\cS$ of strata.
In every case,
the space of strata inherits the quotient diffeology of $\X$ which describes,
in some way,
the \emph{transverse smooth structure}.
The D-topology of $\cS$ is actually the quotient topology of the D-topology of $\X$ \cite[\textsection 2.12]{PIZ13}.
This D-topology is an Alexandroff topology \cite{PA37} for the pre-order $\S \preceq \S'$ if $\S \subset \overline \S'$,
the overline meaning the closure for the D-topology.
In every case,
the space of strata of a stratified diffeological space is a so-called PrOSet (Pre Ordered Set) for the quotient diffeology.
It also turns out that the quotient topology satisfies the $\T_0$ axiom of separation%
\footnote{That means that for every two points in $\cS$ there is a D-open neighborhood of one of them that does not contain the other one.}
if and only if the strata are locally closed,
and that is equivalent for the pre-order to be a partial order.
The space of strata is then a POSet (Partial Ordered Set),
see for example \cite{SY19}.
These considerations lead to introduce a new label [$\T_0$],
to signify that the strata are locally closed.
We denote the opposite case by [\sout{$\T_0$}].
The action of the irrational solenoid is an example of a geometric stratification for which the strata are manifolds and the space of strata is not $\T_0$.
It is actually an irrational torus where the preorder is not a partial order.

%  These labels [B], [LF], [\sout{LF}], [G], [GK], [M], [\sout{M}], [$\T_0$] and [\sout{$\T_0$}] are the main labels we need in that paper.\n
These labels:
%
\begin{itemize}
\item[{[B]}] For basic:
the closure of a stratum is a union of strata.
\item[{[LF]}] The stratification is locally finite (or not locally finite [\sout{LF}]).
\item[{[G]}] The strata are the orbits of a group,
or a pseudo group of diffeomorphisms.
\item[{[GK]}] The strata are the orbits of the group of diffeomorphisms,
or the pseudo group of local diffeomorphisms.
\item[{[M]}] All the strata are manifolds (or not [\sout{M}]).
\item[{[$\T_0$]}] The strata are locally closed:
the space of strata is $\T_0$ separated (or not [\sout{$\T_0$}]).
\end{itemize}
%
are the main ones we need in that article.
Obviously,
according to particular stratification we could study in the future we can imagine other labels,
as for example [C] for locally conical,
but this does not concern us at the moment.

Now, coming back to the standard stratification of diffeological spaces,
we could describe their family by the encoding [B]-[LF]-[M]-[$\T_0$].
Another example,
the stratification defined by a smooth action of a compact Lie group on a manifold will belong to the family [B]-[LF]-[G]-[M]-[$\T_0$].
The manifolds with corners fall in the family [B]-[LF]-[GK]-[M]-[$\T_0$],
and so on.

After having drawn the general picture of the theory of stratification in diffeology,
we focus in this paper on diffeological orbifolds.
Like every diffeological space,
orbifolds are stratified by the action of local diffeomorphisms.
We describe explicitely the structure of these strata and we prove the main thorem of this article:

\textsc{Theorem}{\it The stratification of diffeological orbifolds by local diffeomorphisms is standard:
The Klein strata are locally closed manifolds.
If the orbifold is locally finite in the sense that it admits a locally finite atlas,
then the stratification is locally finite.
The code for such orbifolds is then \textup{[B]-[LF]-[GK]-[M]-[$\T_0$]}.
}

\textsc{Acknowledgements.}~We sincerely thank the anonymous referee who by his remarks allowed us to correct and improve our work.
He pointed out in particular the case of singular Stefan foliations%
\footnote{Precisely \cite[Theorem 3]{PS74}.}
which indeed fall within the general framework of diffeological stratified spaces.
As for his remark on Whitney’s conditions,
we think that it is  associated with geometric stratifications for some families of local diffeomorphisms.
We reserve this question for a future work.

%%%%%%%%%%%%%%%%%%%%%%%%%%%%%%%%%%%%%%%%%%%%%%%%%%%%%%%%%%
%%
%% MARK: Diffeological Stratified Spaces
%%
%%%%%%%%%%%%%%%%%%%%%%%%%%%%%%%%%%%%%%%%%%%%%%%%%%%%%%%%%%
\section*{Diffeological Stratified Spaces}

Beginning with stratification in diffeology,
it is necessary to connect with the standard and commonly admitted definition,
for example \cite[Definition 1.1 and 1.8]{Klo07}.
It will consist,
roughly speaking,
to change the word ``topology'' by ``diffeology''.

\begin{article}
\artlabel[Standard Diffeological Stratifications]
\label{Standard-Diffeological-Stratifications}
Consider a diffeologiacal space $\X$,
we call \emph{standard stratification} of $\X$ any locally finite partition in \emph{strata} $\cS$ of $\X$,
such that:

\begin{enumerate}
    \item Each stratum is a manifold for the induced diffeology.
    \item Each stratum is locally closed for the D-topology.
    \item The strata satisfy the \emph{frontier condition}\/:
    for all $\S,\S' \in \cS$,
    $$
    \S \cap \overline\S' \neq \varnothing \quad \Rightarrow \quad \S \subset \overline\S'.
    $$
    In other words the closure of a stratum is a union of strata.
\end{enumerate}

We remind that the D-topology of a diffeological space is the finest topology that makes the plots continuous \cite[\textsection 1.2.3]{PI85} and \cite[\textsection 2.8]{PIZ13}.
A subset $\A \subset \X$ is open for the D-topology (or D-open) if $\P^{-1}(\A)$ is open for any plot $\P$ in $\X$.

We remind also that every subset of a diffeological space is naturally a diffeological subspace,
whose plots are the plots of the ambient space but taking their values in the subset [Ibid. \textsection 1.33].
Now,
a subspace of a diffeolgical space is a submanifold if its diffeology is a manifold diffeology,
that is,
generated by local diffeomorphisms with some Euclidean space [Ibid. \textsection 4.1].
\end{article}

\begin{article}
\artlabel[Basic Diffeological Stratifications]
\label{Basic-Diffeological-Stratifications}

The first condition for a partition $\cS$ of a space $\X$ to be a stratification is to satisfy the frontier condition,
that the closure of a stratum is a union of strata.
This is the very nature of the concept of stratification.
In the standard approach,
the ambient space is only a topological space,
but the strata are supposed to be equiped with a manifold structure and maybe the first question raised by the theory is how to manage the competition between the two structures.
In diffeology,
since every subset inherits a natural subset diffeology there is no need to enforce the structure of the strata,
they are always regarded as diffeological subspaces.
They can be or not manifolds but that is not anymore a necessity,
they stay in the category \{Diffeology\}.
Note that the traditional approach has no choice,
with respect to the smooth structure of the strata,
but to require them to be manifolds,
since this is the only smooth structure in classical differential geometry,
defined without additional or heuristic considerations.
This is the reason that justifies the following definition of \emph{basic diffeological stratification}\,:

\begin{definition}{}
    Let $\X$ be a diffeological space,
    we call \emph{basic stratification} of $\X$ any partition $\cS$ of $\X$ in strata such that the closure,
    for the D-topology,
    of a stratum is a union of strata.
\end{definition}

We can reword the condition of basic stratification in several ways,
starting with:
$$
\text{For all \ $\S, \S' \in \cS$}, \ \  \S \cap \bar \S' \neq \varnothing \ \Rightarrow \ \S \subset \bar \S'.
$$
Another way,
$$
\text{For all \ $\S \in \cS$, \ \ there exists \ $\Sigma \subset \cS$ \ such that:} \quad \overline \S = \bigcup_{\S' \in \Sigma} \S'.
$$
We will refer to partitions that satisfy the frontier condition above with the label [B],
for basic.

Now,
the set of strata $\cS$ has a natural quotient diffeology,
that is,
the pushforward of the diffeology \cite[\textsection 1.43]{PIZ13} of $\X$ by the projection
$$
\Str : \X \to \cS \ \ \text{with} \ \ \Str(x) = \S \ \ \text{iff} \ \ x \in \S.
$$
This diffeology represents the \emph{smooth tranversal structure} of the stratification,
in the sense that a plot $\P$ in $\cS$ is everywhere represented locally by some plot $\Q$ in $\X$ such that $\P \restriction \dom(\Q) = \Str \circ\, \Q$.
Moreover,
the D-topology of the space $\cS$,
equiped with the quotient diffeology,
is the quotient topology of the D-topology of $\X$ \cite[\textsection 2.12]{PIZ13}.
It will play a role later.
\end{article}

\begin{article}
\artlabel[Category of Diffeological Stratifications]
\label{Category-of-Diffeological-Stratifications}
A smooth map $f$ from a diffeological stratified space $(\X,\cS)$ to another $(\X',\cS')$ is said to be \emph{stratified} if it maps strata into strata.
That is,
if there exists a map $\phi$ from $\cS$ to $\cS'$ such that:
$$
\Str \circ f = \phi \circ \Str,
$$
where $\Str$ denote the projection from the space to the space of strata.
This is summarized by the commutative diagram:
$$
\begin{tikzcd}[column sep=1.3cm,row sep=1.3cm,every label/.append style = {font = \small}]
    \X \arrow[r,"f"] \arrow[d,swap, "\Str"] & \X' \arrow[d,"\Str"] \\
    \cS \arrow[r,"\phi"] & \cS'
\end{tikzcd}
$$
Note that since $\Str$ are subduction,
the map $\phi$ is automatically smooth.

Diffeological stratified spaces together with \emph{stratified smooth maps} as morphisms define the category of \{Stratified Diffeology\}.
The isomorphisms of this category are the \emph{stratified diffeomorphisms}.
\end{article}

\begin{figure}[t]
\includegraphics[width=.5\textwidth]{Figures/Frontier-condition.pdf}
\vspace{-.25\baselineskip}
\caption{The frontier condition.}
\label{Frontier-condition}
\end{figure}

\begin{article}
\artlabel[Geometric Stratifications]
\label{Geometric-Stratifications}
An important class of stratification is defined by the actions of groups on spaces,
Precisely:

\begin{proposition}{}
    Let $\G$ be a diffeological group acting smoothly on a diffeological space $\X$.
    That is,
    a smooth morphism $g \mapsto g_\X$ from $\G$ to $\Diff(\X)$,
    equipped with the functional diffeology.
    Then:
    
    The partition of $\X$ by the orbits of $\G$ is a basic stratification.
    We call such stratifications \emph{geometric}.
\end{proposition}

See \cite[\textsection 1.57, 7.1 and 7.4]{PIZ13} for the definitions involved in the proposition.
We shall use the label [G] to specify that a stratification is defined,
or recognized,
as a geometric stratification.
When a stratification is not geometric,
or when we don't know,
we just say that it is a \emph{formal stratification}.
\end{article}

\begin{proof}
Let $\cO_x$ and $\cO_y$ be two orbits of $\G$.
Assume that $x \in \overline\cO_y$ and let $x' \in \cO_x$.
Let $\U'$ be an D-open neighborhood of $x'$ and $g \in \G$ such that $g_\X(x) = x'$.
Let $\U = g_\X^{-1}(\U')$.
Since $g_\X \in \Diff(\X)$,
$\U$ is an open neighborhood of $x$ and since $x \in \overline\cO_y$ there exists $z \in \cO_y \cap \U$,
and $\cO_y = \cO_{z}$.
On Figure \ref{Frontier-condition},
we represent $z$ as $f(y)$ with $f \in \G$.
Hence,
$z' = g_\X(z) \in \U' \cap \cO_y$.
Thus,
$\cO_y \cap \U' \neq \varnothing$,
and $x' \in \overline\cO_y$.
Therefore,
the closure of any orbit is a union of orbits.
The partition of a diffeological space by a smooth action of a diffeological group is a basic stratification.
\end{proof}

\begin{article}
\artlabel[Klein Stratifications]
\label{Klein-Stratifications}
The \emph{Klein strata} of a diffeological space $\X$ have been introduced in \cite[\textsection 1.42]{PIZ13} as the orbits of the goup of diffeomorphisms $\Diff(\X)$.
The fact that they constitute a stratification was established and discussed in \cite{PIZ22-a},
as well as their variants. 
The standard example of this situation is the square $\Sq$ of Figure \ref{the-Square}.
The group $\Diff(\Sq)$ has three orbits:

\begin{enumerate}
    \item[1.] the 4-corners-orbit;
    \item[2.] the 4-edges-orbit;
    \item[3.] the interior-orbit.
\end{enumerate}

Note that if we prefer considering connected Klein strata,
orbits of the identity component of the group of diffeomorphisms,
they are nine in number and are independently the corners, edges and interior.

Now,
the notion of stratification goes hand in hand with that of singularity we will discuss later.
The idea of singularity is by definition local.
We consider then the local geometry of a diffeological space:
it is defined at each point by the germ of the diffeology there \cite[\textsection 2.19]{PIZ13}.
The local geometry at each point is preserved by the action of local diffeomorphisms,
which is no more a group but a so-called pseudo-group,
we denote it by $\Diff_\loc(\X)$.
Local diffeomorphisms can exchange only points with the same local geometry.
That leads to the following definition,
which is the variant of Klein strata we will use

\begin{definition}{} Let $\X$ be a diffeological space,
    We call (local) \emph{Klein strata} the orbits of its local diffeomorphisms.
    The set of Klein strata is called the \emph{Klein stratification} of $\X$.
\end{definition}

Therefore,
diffeological spaces share this important property:

\begin{proposition}{}
    Every diffeological space has a natural stratification embedded in its diffeology,
    revealed by the action of the local diffeomorphisms
\end{proposition}

We shall use the label [GK] to specify that a stratification is a Klein stratification,
or by the whole group of diffeomorphisms,
or by its identity component,
or by the action of local diffeomorphisms.

Note that for manifolds in particular,
this stratification is trivial since a $n$-manifold is everywhere locally diffeomorphic to $\RR^n$.
But that remark applies also to every homogeneous,
or locally homogeneous,
diffeological space,
for example the irrational tori.

\textsc{Remark.}
The frontier condition which is the essence of stratification is {\em a priori} a topological condition:
homeomorphisms preserve closure.
Stratification in diffeology inherits from topology,
through the D-topology which is a by-product of it,
but is not reduced to it as the example of the square shows.
It is homeomorphic to the disk which has only two orbits,
the boundary and the interior,
instead of three:
the vertices,
the edges and the interior.
Note however that,
since D-topology is a by-product of diffeology,
Klein stratification,
and more generally geometric stratifications,
can be defined without the help of D-topology since local diffeomorphisms are originally defined without it.
Geometric stratifications,
and Klein stratifications in particular,
are pure products of diffeology. %\remend
\end{article}

\begin{figure}[t]
\includegraphics[width=.475\textwidth]{Figures/Square.pdf}
\vspace{-0.5\baselineskip}
\caption{A diffeomorphism of the square.}
\label{the-Square}
\end{figure}

\begin{article}
\artlabel[Diffeological Stratifications as PrOSet and POSet]
\label{Diffeological-Stratifications-as-PrOSet-and-POSet}
Let $\cS$ be a stratification on a diffeological space $\X$,
and let $\cS$ equiped with the quotient diffeology.

Consider the following binary relation on $\cS$:
$$
\S \preceq \S' \quad \Leftrightarrow \quad \S \subset \bar\S'.
$$
It is clearly reflexive and transitive,
that is,
a pre-order on $\cS$.

It is a well known fact that the quotient on $\cS$ of the topology of the total space $\X$ is an Alexandrov topology.
Since the quotient of the D-topology of $\X$ is the D-topology of $\cS$ \cite[\textsection 2.12]{PIZ13},
one could say anachronistically that the diffeology of $\cS$ is an \emph{Alexandrov diffeology}.%
\footnote{If we need it,
an Alexandrov diffeology could be {\em a priori} a diffeology whose D-topology is an Alexandrov topology.
This definition could be amended in the future if necessary.}

The open subsets of the D-topology on $\cS$ are the upper sets for the relation $\preceq$,
that is,
any union of cones:
$$
\Cone(S) = \{ \S' \mid \S \preceq \S' \}
$$

\textsc{Remark.} In some case the Alexandrov D-topology of the space $\cS$ is trivial with the Alexandrov diffeology of $\cS$ non trivial.
This is the case of the geometric stratification of the $2$-torus by the irrational solenoid:
$$
\underline t_{\T^2} (z_1,z_2) = \big(z_1 e^{2i\pi t},z_2 e^{2i\pi \alpha t}\big),
$$
where $t \in \RR$,
$(z_1,z_2) \in \T^2$ and $\alpha \in \RR -\QQ$.
In this case every orbit of $\RR$ is in the closure of any other orbit,
since the closure of any orbit is the $2$-torus itself.
As we can notice,
in this case one  have for all $\S, \S' \in \cS$,
$\S \preceq \S'$ and $\S' \preceq \S$.
The pre-order $\preceq$ is not a partial order,
but the diffeology of $\cS = \T_\alpha$ is not trivial \cite{DI83}.
\remend

Now,
it happens that the D-topology of the space of strata satisfies an axiom of separation,
the minimal $\T_0$.
That is,
for any two point $\S, \S' \in \cS$ there a D-open neighborhood of one of them that does not contain the other one.
In this case:

\begin{theorem}{}\textup{(See \cite{SY19})}
    The D-topology of the space of strata satisfies the $\T_0$ axiom of separation if and only if the preorder $\preceq$ is a partial order,
    that is,
    antisymetric:
    $\S \preceq \S'$ and $\S' \preceq \S$ implies $\S = \S'$.
    This is equivalent to the fact that the strata are locally closed.
    \footnote{For every point of $\S$ there is a neighborhood $\V$ in $\X$ such that $\S \cap \V$ is closed in $\V$.}
\end{theorem}

If that is the case we say that $\cS$ is a POSet.
We shall use the label [$\T_0$] to specify that the space of strata is a POSet,
which is equivalent to the strata being locally closed or the preorder being a partial order,
and we will note [\sout{$\T_0$}] otherwise.

\end{article}

\begin{article}
\artlabel[Manifold or Not Manifold]
\label{Manifold-or-Not-Manifold}
Since diffeology includes so many various subcategories it is not surprising to have stratifications for which the strata are not manifolds,
at least not all of them.
A simple example would be the space of (oriented) geodesics of the $2$-torus,
let us denote it by $\X = \Geod(\T^2)$.
It is described in the blog post \cite{PIZ16-b}.
Without going into details,
let us say that
$$
\Geod(\RR^2) \simeq \S^1 \times \RR \ \text{ and } \ \Geod(\T^2) \simeq \Geod(\RR^2)/\ZZ^2,
$$
    with the following action of $\ZZ^2$\,:
$$
\underline k(u,\rho) = (u, \rho + u \cdot k),
$$
where $k = (m,n) \in \ZZ^2$,
$\underline k$ is the action of $k$ on $\T^2$,
$(u,\rho) \in \S^1 \times \RR$,
$u \cdot k$ is the scalar product of $u \in \S^1 \subset \RR^2$ by $k \in \ZZ^2 \subset \RR^2$.
A point in $\X$ will be denoted by $x=\class(u,\rho) = (u,\class_u(\rho))$.

Actually this kind of diffeological space,
locally diffeomorphic to the quotient of an Euclidean space by a countable subgroup of the affine group,
has been categorized recently as \emph{quasifolds},
their diffeological version \cite{IZP21}.

The fibers of the projection $\pr_1 : (u,\tau) \mapsto u$ from $\X$ to $\S^1$ are the rational or irrational tori $\T_u = \RR/[\cos(\theta) \ZZ + \sin(\theta) \ZZ]$,
with $u = (\cos(\theta),\sin(\theta))$,
according to the dependence over $\QQ$ of $\cos(\theta)$ and $\sin(\theta)$.
They are manifolds (circles) only when the line $\RR u$ is \emph{rational},
that is,
when there are two integers $m,n$ such that $m \cos(\theta) + n \sin(\theta) = 0$.
The space $\X$ is a \emph{group bundle}.%
\footnote{Not a diffeological fiber bundle in the sense of \cite{PI85} since the fibers are not diffeomorphic.}

It is proved in \cite{PIZ22-b} that the restriction to a fiber $\T_u$ of a diffeomorphism $f \in \Diff(\X)$ is a diffeomorphism to some fiber $\T_{u'}$.
Hence,
$f$ projects on a diffeomorphism of $\S^1$.
On the other hand,
we know that two torus $\T_u$ and $\T_{u'}$ are diffeomorphic if and only if $u$ and $u'$ are conjugate modulo $\GL(2,\ZZ)$ \cite{DI83}.
That is,
if there exists a matrix $\M \in \GL(2,\ZZ)$ such that
$$
\M u = \lambda u',
$$
with $\lambda = \pm \norm{\M u}$.
The map $f \mapsto \M$ is then a morphism from $\Diff(\X)$ to $\GL(2,\ZZ)$.
It happens that this morphism is surjective [ibid.].
That is,
if $u,u' \in \S^1$ are equivalent modulo $\GL(2,\ZZ)$ there exists a diffeomorphism $f$ of $\X$ such that $f(\T_u) = \T_{u'}$.

Let us show that quickly.
Assume that $u' = \M u / \norm{\M u}$,
with $\M \in \GL(2,\ZZ)$.
Consider the map $\F$ from $\S^1 \times \RR$ to itself defined by
$$
\F : (v,\rho) \mapsto (v',\rho'), \ \text{ with } \ v' = {\M v \over \norm{\M v}} \ \text{ and } \ \rho' = {\rho \over \norm{\M v}}.
$$
It is clearly a diffeomorphism.
Now,
let $k \in \ZZ^2$ and $(v'',\rho'') = \F(v,\rho + v \cdot k)$.
Thus
$$
v'' = {\M v \over \norm{\M v}} \ \text{ and } \
\rho'' = {\rho + v \cdot k \over \norm{\M v}}.
$$
But,
$$
\rho''  =  {\rho + v \cdot k \over \norm{\M v}}
    =  {\rho \over \norm{\M v}} + {v \cdot k \over \norm{\M v}}
    =  {\rho \over \norm{\M v}} + {\M v \over \norm{\M v}} \cdot (\M^{-1})^t k,
$$
where the supscript $t$ denotes the transposition.
Hence,
$$
v'' = v' \ \text{ and } \ \rho'' = \rho' + v' \cdot k' \ \text{ with } \ k' = (\M^{-1})^t k \in \ZZ^2.
$$
Therefore,
$$
\F\circ \underline{k} (v,\rho) = \underline{k'}(\F(v,\rho)).
$$
So,
there exists a smooth map $f$ from $\Geod(\T^2) = (\S^1 \times \RR) / \ZZ^2$ to itself such that $\class \circ \F = f \circ \class$.
This map $f$ is a diffeomorphism of $\Geod(\T^2)$ which projects on $\M$ acting on $\S^1$,
and satisfies by construction $f(\T_u) = \T_{u'}$.

$$
\begin{tikzcd}[column sep=1.3cm,row sep=1.3cm,every label/.append style = {font = \small}]
    \S^1 \times \RR \arrow[r,"\F"] \arrow[d,swap, "\class"] & \S^1 \times \RR \arrow[d,"\class"] \\
    \Geod(\T^2) \arrow[r,"f"] & \Geod(\T^2)
\end{tikzcd}
$$

Now,
the group $\RR$ acts on $\S^1 \times \RR$ additively on the second factor,
for all $s \in \RR$ and $(u,t) \in \S^1 \times \RR$ we denote $\underline{s}(u,t) = (u, t+s)$.
This action commutes with the action of $\ZZ^2$,
$\underline{s} \circ \underline{k} = \underline{k} \circ \underline{s}$.
It defines then an action of $\RR$ on $\Geod(\T^2)$ by $\underline{s}(u,\tau) = (u, \class_u(t+s))$ with $\tau = \class_u(t)$.
This action is transitive on the fibers $\T_u$.
Thus,
composing the action of $\M \in \GL(2,\ZZ)$ above with this action of $\RR$,
we conclude that the orbits of the group $\Diff(\X)$ are the subspaces
$$
\cO_{(u,\tau)} = \bigcup_{\M \in \GL(2,\ZZ)} \T_{\underline{\M}(u)},
\quad \text{with} \quad
\underline{\M}(u) = {\M u \over \norm{\M u}}.
$$
Thus,
except for the rational lines $\RR u$,
which constitute one orbit of $\GL(2,\ZZ)$,
the Klein strata are not manifolds.
The space of Klein strata identifies with the quotient:
$$
\X/\Diff(\X) \simeq \S^1/\GL(2,\ZZ).
$$
Therefore,
we shall use the label [M] to specify that the strata are all manifolds and [\sout{M}] otherwise.

\textsc{Remark.} The connected Klein strata,
orbits of the identity component of the group of diffeomorphisms,
are the fibers $\T_u$ of the projection $\pr_1 : \X \to \S^1$.
In that case the space of strata is $\S^1$ and the stratification is [$\T_0$].
\end{article}

\begin{article}
\artlabel[The Example of Manifolds with Corners]
\label{The-Example-of-Manifolds-with-Corners}

Manifolds with corners have been introduced in diffeology as spaces locally diffeomorphic to some corners $\K^n = [0,\infty[^n$,
equiped with the subset diffeology \cite[\textsection 4.16 Note]{PIZ13}.
It is proved there that smooth real functions on $\K^n$,
in the sense of diffeology,
are restrictions of smooth functions defined on an open neighborhood of $\K^n$ in $\RR^n$.
Later this result has been extended to any differential $k$-form \cite{GIZ19}.
In this paper,
the Theorem 2 states that the local diffeomorphims of the corner $\K^n$ can exchange only points of the same strata $\S$,
defined by the number of vanishing coordinates,
between $0$ and $n$.
Therefore,
the Klein strata of a manifold with corners are the union of the images of the strata of the corner $\K^n$,
by the charts of the manifold.
\end{article}

\begin{article}
\artlabel[Singularities of Diffeological Spaces]
\label{Singularities-of-Diffeological-Spaces}
Klein strata are associated with the idea of \emph{singularity}.
They discriminate the points of the diffeological space according to their geometry.
However,
the singularity is a relative concept,
a point is not singular by itself but relatively to others.
This means for the square,
for example,
that the corners are singular to the interior points,
as are the edges,
but they are not equivalent to each other.

The definition of singularity dwells precisely in the preorder associated with the local Klein stratification.
To be accurate we should distinguish between two cases:
when the precedence relation is a partial order and when it is not.

\begin{definition}{} Let $\X$ be a diffeologial space such that its Klein stratification,
    by local diffeomorphisms,
    defines a partial order.
    Then,
    a point $x$ is singular with respect to another point $x'$ is the strata $\S$ of $x$ precedes the strata $\S'$ of $x'$.
    That is,
    $\S \preceq \S'$.
\end{definition}

In this case the notion of singularity is totally handled by the D-topology.
When the precedence is not a partial order,
it is more complicated.
We could introduce an equivalence relation:
$\S \sim \S'$ when $\S \preceq \S'$ and $\S' \preceq \S$,
and declare that $x$ is singular with respect to $x'$ if $\S \preceq \S'$ with $\S \not\sim \S'$.
However,
if that would work for some geometric stratifications like the irrational solenoid,
for which all strata are diffeomorphic and equivalent to each other,
it is clearly not satisfactory for the Klein stratification of the space of geodesic trajectories of $\T^2$,
since they would be all equivalent to each other without being diffeomorphic. %\See{Manifold-or-Not-Manifold}.
This shows the limit of topology in this case,
and force to consider the notion of singularity more carefully.

By the way, note also that the irrational torus has no singularity,
which is not surprising since it is a group.
It is,
in this case,
another kind of singularity which resides in the global nature of the space,
and comes from the mismatch between its trivial topology and its non-trivial diffeology.
\end{article}


%%%%%%%%%%%%%%%%%%%%%%%%%%%%%%%%%%%%%%%%%%%%%%%%%%%%%%%%%%
%%
%% MARK: The Case of Orbifolds
%%
%%%%%%%%%%%%%%%%%%%%%%%%%%%%%%%%%%%%%%%%%%%%%%%%%%%%%%%%%%
\section*{The Case of Orbifolds}

\begin{article}
\artlabel[Diffeological Orbifolds]
\label{Diffeological Orbifolds}
An \emph{orbifold} is a diffeological space $\X$ which is everywhere locally diffeomorphic to a quotient $\RR^n/\Gamma$,
for some integer $n$ and where $\Gamma$ is a finite subgroup of $\GL(n,\RR)$,
which may change from one point to another \cite{IKZ10}.
A \emph{chart} of $\X$ is any local diffeomorphism $f$ from $\RR^n/\Gamma$ to $\X$.
A set $\cA$ of charts covering $\X$ is called an \emph{atlas}.
We assume in general the atlas to be locally finite,
that is,
every point $x \in \X$ is in the image of a finite number of charts.
We denote by $\class : \RR^n \to \RR^n/\Gamma$ the canonical projection,
and $\F = f \circ \class$.
The set
$$
\cF = \{ \F = f \circ \class \mid f \in \cA \}
$$
is a generating family of $\X$  called the \emph{strict generating family} associated with the atlas $\cA$.
For examples:

(A) Consider the map $\Phi_m : \CC \to \CC$ with $\Phi_m(z) = z^m$,
then $\CC$ equiped with the pushforward of the standard diffeology of $\CC$ by $\Phi_m$ is an orbifold diffeomorphic to $\cQ_m = \CC/\cU_m$,
with $\cU_m$ the group of $m$th-roots of unity.
We call it the \emph{cone orbifold}.

(B) Another simple example,
the \emph{corner orbifold} $\cK_2 = [\RR/\{\pm 1\}]^2$ is the image in $\RR^2$ of the map $\sq^2 : (x,y) \mapsto (x^2,y^2)$

(C) We have also the typical example of teardrop $\cT$ whose diffeology on $\S^2 \subset \RR^3 \simeq \CC \times \RR$ is illustrated by the Figure \ref{TearDrop}.
%
\begin{figure}[th]
    \includegraphics[width=1\textwidth]{Figures/TearDrop.pdf}
    \caption{The Teardrop as a diffeology on the sphere.}
    \label{TearDrop}
\end{figure}
%
Let $\N = (0,1)$ be the North pole.
The following set of parametrizations $\zeta$ defines an orbifold diffeology on $\S^2$ with a $\cU_m$-cyclic singular north pole.
Let $\U$ be an Euclidean domain,
$$
\zeta \colon \U \to \S^2 \quad \mbox{with} \quad \zeta(r) = \begin{pmatrix}z(r) \\ t(r) \end{pmatrix},
\quad \mbox{and} \quad |z(r)|^2 + t(r)^2 =1,
$$
such that, for all $r_0 \in \U$,

\begin{enumerate}
    \item if $\zeta(r_0) \neq \N$,
    then there exists a small ball $\cB$ centered at $r_0$ such that $\zeta \restriction \cB$ is smooth.
    \item If $\zeta(r_0)=\N$,
    then there exist a small ball $\cB$ centered at $r_0$ and a smooth parametrization $z$ in $\CC$ defined on $\cB$ such that,
    for all $r\in \cB$,
    $$
    \zeta(r) = {1 \over \sqrt{1 + |z(r)|^{2m}}}\begin{pmatrix} z(r)^m \\ 1 \end{pmatrix}.
    $$
\end{enumerate}

\end{article}

\begin{article}
\artlabel[Local Centered Charts]
\label{Local-Centered-Charts}
Let $\X$ be an orbifold and $x \in \X$.
Let $f \in \cA$ be a chart defined on some D-open subset $\U \subset \RR^n/\Gamma$ and let $\F = f \circ \class$.
Let $\tilde\U$ be the domain of $\F$,
$\tilde\U = \class^{-1}(\U) \subset \RR^n$.
Let $r \in \tilde\U$ such that $\F(r) = f(\class(r)) = x$.

\begin{proposition}{}
    There exists a $\Gamma$-invariant Euclidean scalar product on $\RR^n$ and for the associated metric,
    there exists a ball $\cB(r,\epsilon)$ centered in $r$ with radius $\epsilon$ such that:
    \begin{enumerate}
        \item For all $r' \in \cB(r,\epsilon)$,
        $\Stab_\Gamma(r') \subset \Gamma_r$,
        with $\Gamma_r = \Stab_\Gamma(r)$ the stabilizer of $r$.
        \item The map $j_r : \class_{\Gamma_r}(r') \mapsto \class_\Gamma(r')$ is a local diffeomorphism from $\cB(r,\epsilon)/\Gamma_r$ to $\tilde\U/\Gamma$.
        The composite $f \circ j_r$ is a local diffeomorphism from $\cB(r,\epsilon)/\Gamma_r$ to $\X$.
        We call it a \emph{centered chart} on $x$.
    \end{enumerate}
\end{proposition}
\end{article}

\begin{proof}
Since the group $\Gamma \subset \GL(n,\RR)$ is finite,
one considers the new Euclidean scalar product defined by:
$$
u \cdot v = {1 \over \# \Gamma} \sum_{\gamma \in \Gamma} \scal{\gamma u}{\gamma v},
$$
where the brackets denote the ordinary scalar product.
This scalar product is obviously $\Gamma$-invariant.
For this new metric,
let $\cB(r,\epsilon) \subset \tilde\U$ be a ball centered in $r$ with radius $\epsilon$:
$$
\text{For all $r' \in \cB(r,\epsilon)$, \ for all $\gamma \in \Stab_\Gamma(r)$}: \quad  \gamma(r') \in \cB(r,\epsilon).
$$
Indeed,
on the one hand:
$\norm{\gamma(r'-r)} = \norm{r'-r} < \epsilon$.
On the other hand,
$\norm{\gamma(r') - r} = \norm{\gamma(r'-r)}$,
since $\gamma \in \Stab_\Gamma(r)$.
Thus,
$\norm{\gamma(r') - r} < \epsilon$ and $\gamma(r') \in \cB(r,\epsilon)$.

Next,
there exists $\epsilon \in \RR$ such that:
$$
\gamma\big(\cB(r,\epsilon)\big) \cap \cB(r,\epsilon) = \varnothing
\quad \text{for all $\gamma \notin \Stab_\Gamma(r)$.}
$$
Indeed,
let $\gamma \notin \Stab_\Gamma(r)$,
and let $r_\gamma = \gamma(r)$.
Then,
$r_\gamma \neq r$,
and since $\Gamma$ is finite
---~so is the number of $\gamma$ that are not in $\Stab_\Gamma(r)$~---
there exits $\epsilon \in \RR$ such that the balls $\cB(r,\epsilon)$ and $\cB(r_\gamma,\epsilon) = \gamma(\cB(r,\epsilon))$ are disjoint,
for all $\gamma \notin \Stab_\Gamma(r)$.

Now, consider the map $j_r : \class_{\Gamma_r}(r') \mapsto \class_\Gamma(r')$,
defined on $\cB(r,\epsilon)/\Gamma_r$ to $\tilde\U/\Gamma$.
Explicitely:
$$
j_r : \{\gamma(r') \mid \gamma \in \Stab_\Gamma(r) \} \mapsto \{\gamma(r') \mid \gamma \in \Gamma \}.
$$
The map $j_r$ is injective:
$$
j_r^{-1} : \class_\Gamma(r') \mapsto \class_\Gamma(r') \cap \cB(r,\epsilon).
$$
Next,
the map $j_r$ is a smooth injection since the $\class$ projections are subductions.
$$
\begin{tikzcd}[column sep=1cm,row sep=1cm,every label/.append style = {font = \small}]
    %    \cB(r,\epsilon) \arrow[hookrightarrow]{r}{j} \arrow[d,swap, "\class_{\Gamma_r}"] & \tilde\U \arrow[d,"\class_\Gamma"] \\
    \cB(r,\epsilon) \arrow[r,symbol=\subset] \arrow[d,swap, "\class_{\Gamma_r}"] & \tilde\U \arrow[d,"\class_\Gamma"] \\
    \cB(r,\epsilon)/\Gamma_r \arrow[r,"j_r"] & \tilde\U/\Gamma
\end{tikzcd}
$$
Then,
let $s \mapsto \sigma_s$ be a plot in $\tilde\U/\Gamma$ but taking its values in $j_r(\cB(r,\epsilon))$.
Since $\class_\Gamma$ is a subduction,
there exists locally a smooth lifting $s \mapsto r'_s$ in $\tilde\U$ such that $\sigma_s = \class_\Gamma(r'_s)$.
We assume  that the lifting is defined on a small ball.
By connexity $r'_s$ belongs to $\cB(r,\epsilon)$ or one of the balls $\gamma\big(\cB(r,\epsilon)\big)$ with $\gamma \notin \Stab_\Gamma(r)$.
If that is the case,
then $s \mapsto \gamma^{-1}(r'_s)$ is a smooth local lifting with values in $\cB(r,\epsilon)$.
Hence,
$j_r^{-1}$ is smooth and $j_r$ is an induction.
An induction is a diffeomorphism onto its image.
Now,
since $\U$ is the quotient of $\tilde\U$ by a group of diffeomorphisms,
the projection $\class_\Gamma$ is not just a subduction but a local subduction and then a D-open map \cite[\textsection 2.18 and Exercise 60]{PIZ13}.
Thus,
$\class_\Gamma(\cB(r,\epsilon)) = j_r\big(\class_{\Gamma_r}(\cB(r,\epsilon))\big)$ is D-open.
Therefore,
$j_r$ is a local diffeomorphism [Ibid. \textsection 2.10] and $f \circ j_r$ is a chart of $\X$.
\end{proof}

\begin{article}
\artlabel[Klein Strata]
\label{Klein Strata}
We choose the variant definition of Klein strata of a diffeological space as orbits of local diffeomorphisms.
They satisfy the basic frontier condition of stratified spaces \cite[\textsection 1.42]{PIZ13} and \cite{PIZ22-a}.

Let $\X$ be an $n$-orbifold,
let $\cA$ be an atlas of $\X$ and we keep the definitions and notations introduced in the previous paragraph.
Let $x \in \X$,
let $f$ be a chart defined on some D-open subset of $\RR^n/\Gamma$,
let $\F = f \circ \class_\Gamma$ and $r \in \dom(\F)$ such that $\F(r) = x$.

\begin{definition}{}
    The stabilizer $\Stab_\Gamma(r)$ of $r$ in $\Gamma$ is called \emph{isotropy} of $x$ in the chart $f$.
\end{definition}

\begin{proposition}{}\textup{\cite[Lemma 20-21]{IKZ10}}
    Two different charts on $x$ give equivalent isotropies,
    conjugate by an element of $\GL(n,\RR)$.
\end{proposition}

Now we can characterize the Klein strata of the orbifold $\X$

\begin{proposition}{}
    Two points $x$ and $x'$ in $\X$ are on the same Klein stratum if and only if their isotropies are conjugate modulo $\GL(n,\RR)$.
    In other words,
    if $f$ and $f'$ are two charts of $\X$ on $x$ and $x'$,
    with notations introduced previously:
    $$
    \exists\M \in \GL(n,\RR)\ : \
    \forall \gamma \in \Stab_\Gamma(r), \
    \M \gamma \M^{-1} \in \Stab_{\Gamma'}(r'),
    $$
    which can be shortened to:
    $$
    \Gamma'_{r'} = \M\, \Gamma_r\, \M^{-1}.
    $$
\end{proposition}
\end{article}

\begin{proof}
Let $g$ be a local diffeomorphism,
defined on some D-open $\cO$,
such that $g(x) = x'$.
Let $f : \U \to \X$ be a chart of $\X$ and $\F = f \circ \class_\Gamma$,
with $\F(r) = x$ and $r \in \tilde\U = \class_\Gamma^{-1}(\U) = \dom(\F)$.
Let $\cB(r,\epsilon)$ be the domain of a local centered chart $f_r = f \circ j_r$,
as defined above.
Let $\F_r = f_r \circ \class_{\Gamma_r}$ with $\Gamma_r = \Stab_\Gamma(r)$.
Let $f'$ be a chart over $x'$,
$\F' = f' \circ \class_{\Gamma'}$,
$r' \in \dom(\F')$ with $\F'(r') = x'$.
There exists a local lifting $\psi$ defined on some open neighborhood of $r$ in $\cB(r,\epsilon)$ to $\tilde\U'$ such that
$$
\F' \circ \psi =_\loc g \circ \F_r.
$$
We can admit that $\dom(\psi) = \cB(r,\epsilon)$,
possibly for a smaller $\epsilon$.
The Lemma 20-21 of \cite{IKZ10} claims that this lifting $\psi$ is a local diffeomorphism.
$$
\begin{tikzcd}[column sep=1cm,row sep=1cm,every label/.append style = {font = \small}]
    \cB(r,\epsilon) \arrow[r,"\psi"] \arrow[d,swap, "\F_r"] & \tilde\U' \arrow[d,"\F'"] \\
    \X \supset \cO \arrow[r,"g"] & \X
\end{tikzcd}
$$
Thus,
for all $\gamma \in \Stab_\Gamma(r)$\ there exists $\gamma' \in \Gamma$ such that for all $s \in \cB(r,\epsilon)$\,:
$$
\psi(\gamma s) = \gamma' \psi(s),
\quad \text{that is} \quad
\gamma' = \psi \circ \gamma \circ \psi^{-1}
$$
{\it A priori} $\gamma'$ depends on $s$ and $\gamma$,
but since it depends smoothly on $s$ and  since $\Gamma$ is discrete,
it depends only on $\gamma$.
Thus,
$h : \gamma \mapsto \gamma'$ is an isomorphism from $\Stab_\Gamma(r)$ onto its image.
On the other hand $h(\gamma)(r') = \psi \circ \gamma \circ \psi^{-1}(r') = \psi \circ \gamma (r) = \psi(r) = r'$.
Therefore $h$ is an isomorphism from $\Stab_\Gamma(r)$ to $\Stab_\Gamma(r')$.

Next,
the derivative of the identity $\psi(\gamma s) = h(\gamma) \psi(s)$ for $s = r$ gives
$$
h(\gamma) = \M \gamma \M^{-1}
\quad \text{with} \quad
\M = \D(\psi)(r).
$$
Hence,
we get,
as claimed,
that $\Stab_\Gamma'(r') = \M\, \Stab_\Gamma(r)\, \M^{-1}$

Conversely,
assume that there is $\M \in \GL(n,\RR)$ such that $\Stab_\Gamma'(r') = \M\, \Stab_\Gamma(r)\, \M^{-1}$.
Define locally around $r$
$$
\psi : s \mapsto r' + \M(s-r).
$$
The map $\psi$ is clearly a diffeomorphism.
Let $\gamma \in \Stab_\Gamma(r)$,
then $\psi(\gamma s) = r' + \M(\gamma s-r) = r' + \M \gamma(s-r)$ since $\gamma r = r$.
On the other hand,
$h(\gamma)\psi(s) = \M \gamma \M^{-1}[r' + \M(s-r)] = r' + \M \gamma(s-r)$ since $\M \gamma \M^{-1}r' = r'$.
Hence,
$\psi(\gamma s) = h(\gamma)\psi(s)$ the map $\psi$ descends to the quotient $\X$ into a local diffeomorphism.
\end{proof}

\begin{article}
\artlabel[Klein Strata Are Locally Closed Manifolds]
\label{Klein Strata Are Locally Closed Manifolds}
Consider an orbifold $\X$.
Let $f : \U \to \X$ be a chart of $\X$ and $\F = f \circ \class_\Gamma$,
with $\F(r) = x$ and $r \in \tilde\U = \class_\Gamma^{-1}(\U) = \dom(\F)$.
Let $\cB(r,\epsilon)$ be the domain of a local centered chart $f_r = f \circ j_r$,
as defined above.
Let $\F_r = f_r \circ \class_{\Gamma_r}$ with $\Gamma_r = \Stab_\Gamma(r)$.

\begin{proposition}{}
    The projection $\F_r : \cB(r,\epsilon) \to \X$ is open and closed,
    for the D-topology.
\end{proposition}

As a corollary,
we get:

\begin{theorem}{}
    Let $\X$ be an orbifold.
    The Klein strata $\Str(x)$,
    for all $x \in \X$,
    are locally closed submanifolds of $\X$.
    If $\X$ admits a locally finite atlas,
    then the Klein stratification of the orbifold $\X$ satisfies the standard conditions,
    encoded by \textup{[B]-[LF]-[GK]-[M]-[$\T_0$]}.
\end{theorem}

We recall that the Klein strata,
here orbits of local diffeomorphisms,
are naturally equipped with the subset diffeology.
\end{article}

\begin{proof}
In the following,
open and closed refer to the D-topology.
We recall that the D-topology of a quotient is the quotient of the D-topology \cite[\textsection 2.12]{PIZ13}.

Since $f_r$ is a local diffeomorphism it is sufficient to check these properties on $\class_{\Gamma_r} : \cB(r,\epsilon) \to \cB(r,\epsilon)/\Gamma_r$.
As we have said previously,
since the subduction is a local subduction,
$\class_{\Gamma_r}$ is open [Ibid. \textsection 2.18 and Exercise 60]\,:
images of open subsets are open.

Now,
let $\A \subset \cB(r,\epsilon)$ be a closed subset.
Its image $\class_{\Gamma_r}(\A)$ is closed if its complement $\complement\class_{\Gamma_r}(\A)$ is open.
That is,
if its pullback $\class_{\Gamma_r}^{-1} \big(\complement\class_{\Gamma_r}(\A)\big)$ is open.
The pullback of the complement of the image $\class_{\Gamma_r}(\A)$ is equal to the ball $\cB(r,\epsilon)$ minus the orbit of $\A$,
that is,
$\cB(r,\epsilon) - \cup_{\gamma \in \Gamma_r}\gamma(\A)$.
Since $\A$ is closed and $\# \Gamma_r < \infty$,
$\cup_{\gamma \in \Gamma_r}\gamma(\A)$ is closed and $\cB(r,\epsilon) - \cup_{\gamma \in \Gamma_r}\gamma(\A)$ is open.
Hence,
$\class_{\Gamma_r}(\A)$ is closed.
The projection $\class_{\Gamma_r}$ is then closed,
and therefore,
open and closed.

Now,
$\X$ is locally equivalent to the quotient $\cB(r,\epsilon)/\Gamma_r$.
The structure of the strata sharing conjugate stabilizers under a smooth action of a compact group has been clarified:
\cite[Theorem 3.3]{Bre72} claims that the subset of points with conjugated stabilizers is a topological manifold,
locally closed.
Since the action of $\Gamma_r$ on $\cB(r,\epsilon)$ is smooth,
these manifods are smooth [Ibid. Proof].

That applies in particular to the strata
$$
\Str(r) = \{ r' \in \cB(r,\epsilon) \mid \Stab_{\Gamma}(r') = \Stab_\Gamma(r) \},
$$
passing through $r$ which projection on $\Str(x)$ by the chart $f_r$.
Since for all $r' \notin \Str(r)$,
$\Stab_{\Gamma_r}(r') \subsetneq \Stab_\Gamma(r)$,
the projection $\class_{\Gamma_r} \restriction \Str(r)$ is one-to-one.
Then,
since it is a bijective subduction it is a diffeomorphism from $\Str(r)$ to its image.
Hence,
since $\class_{\Gamma_r}$ is open and closed,
$f_r(\class_{\Gamma_r}\Str(r))$ is a locally closed submanifold.
\end{proof}

\begin{article}
\artlabel[Concluding Remarks]
\label{Concluding-Remarks}
In this paper we have presented a program for the theory of stratifications in diffeology.
Noting first that in diffeology,
this theory avoids by default the hiatus between the topology of the ambient space and smooth structure of the strata.
That is because the ambient space has its own smooth structure described by its diffeology,
which perfectly integrates the singularities inherent to the space.
This is for one aspect of the theory.
Another aspect is that any part of a diffeological space inherits the subset diffeology and there is no reason to coerce the structure of the strata a priori.
Each partition is admissible as long as it satisfies the frontier condition.
These two aspects make diffeology an excellent formal framework for developing a theory of stratification.

We then proposed to categorize the different identified properties,
satisfied by a stratification,
with a series of characteristic labels,
which allows us to improve the concept of stratified spaces by including many different cases separately,
or refining some of them.
Starting with the label [B] for basic,
which indicates that the frontier condition is fulfilled.
Then,
we have discussed these various labels [LF] [G] [GK] [M] [$\T_0$] and gave some examples.
We have identified the code of the standard stratification:
the minimum conditions for which a stratified diffeological space corresponds to what one expects from the usual theory of stratification:
[B]-[LF]-[M]-[$\T_0$].
To enrich the theory,
inclusion of further labels,
such as [C] for locally conical,
into this scheme would be possible.


The very nature of diffeology,
its capability to integrate the singularities inherent to its smooth structure,
is revealed by the very specific Klein stratification defined by the action of diffeomorphisms,
or local diffeomorphisms,
on the space itself.
Each diffeological space has its own natural geometric stratification,
embedded in its diffeology,
which reveals in particular its internal singularities.

As an application of this approach,
we have treated in detail the case of orbifolds.
The Klein strata of an orbifold are grouped by conjugate isotropies and are locally closed submanifolds.
If in addition the orbifold admits a locally finite atlas,
then the Klein stratification is locally finite,
which makes it the usual standard for geometric stratification.

\end{article}

%%%%%%%%%%%%%%%%%%%%%%%%%%%%%%%%%%%%%%%%%%%%%%%%%%%%%%%%%%
%%
%% MARK: Bibliography
%%
%%%%%%%%%%%%%%%%%%%%%%%%%%%%%%%%%%%%%%%%%%%%%%%%%%%%%%%%%%

\begin{thebibliography}{MCNM18}
    
    \bibitem[PA37]{PA37}
    \newblock{Pavel Alexandroff,}
    \newblock{\em Diskrete Räume},
    \newblock{Mat. Sb. (N.S.) 2(3): 501--519, 1937.}
    
    \bibitem[AFLT17]{AFLT17}
    \newblock{David Ayala, John Francis and Hiro Lee Tanaka,}
    \newblock{\em Local structures on stratified spaces},
    \newblock{Adv. Math. 307: 903--1028, 2017.}
    
    \bibitem[Bre72]{Bre72}
    Glen~E. Bredon.
    \newblock {\em Introduction to compact transformation groups}.
    \newblock {Academic Press, New-York, 1972.}
    
    \bibitem[MCNM18]{MCNM18}
    \newblock{Marius Crainic and Jo\~ao Nuno Mestre},
    \newblock {\em Orbispaces as differentiable stratified spaces},
    \newblock{Lett. Math. Phys. 108(3): 805--859, 2018.}
    
    \bibitem[DI83]{DI83}
    \newblock{Paul Donato and Patrick Iglesias},
    \newblock{\em Exemple de groupes différentiels : flots irrationnels sur le tore},
    \newblock{Preprint, CPT-83/P.1524, 1983.}
    \newblock{Published in C. R. Acad. Sci. 301: 127--130, 1985}
    
    \bibitem[GIZ19]{GIZ19}
    \newblock{Serap G\"urer and Patrick Iglesias-Zemmour},
    \newblock {\em Differential forms on manifolds with boundary and corners.}
    \newblock{Indagationes Mathematicae 30: 920–-929, 2019},
    
    \bibitem[PI85]{PI85}
    \newblock{Patrick Iglesias-Zemmour},
    \newblock {\em Fibrations diff\'eologiques et homotopie},
    \newblock{State doctorate dissertation, Universit\'e de Provence, Marseille},
    \newblock{1985}.
    
    \bibitem[PIZ13]{PIZ13}
    \bysame
    %\newblock{Patrick Iglesias-Zemmour.}
    \newblock{\em Diffeology.}
    \newblock{Mathematical Surveys and Monographs. The American Mathematical Society, (185), USA R.I. 2013.}
    
    \bibitem[PIZ16-a]{PIZ16-a}
    \bysame
    %\newblock{Patrick Iglesias-Zemmour.}
    \newblock{\em Example of singular reduction in symplectic diffeology.}
    \newblock{Proc. Amer. Math. Soc. 144(3): 1309–-1324, 2016.}
    
    \bibitem[PIZ16-b]{PIZ16-b}
    \bysame
    %\newblock{Patrick Iglesias-Zemmour.}
    \newblock{\em The Geodesics of the $2$-Torus.}
    \newblock{Blog post July 2016}.\\
    \verb|http://math.huji.ac.il/~piz/documents/DBlog-Rmk-TGOT2T.pdf|
    
    \bibitem[PIZ22-a]{PIZ22-a}
    \bysame
    %\newblock{Patrick Iglesias-Zemmour.}
    \newblock{\em Klein Stratification of Diffeological Spaces.}
    \newblock{Blog post February 2022}.\\
    \verb|http://math.huji.ac.il/~piz/documents/DBlog-Rmk-KSODS.pdf|
    
    \bibitem[PIZ22-b]{PIZ22-b}
    \bysame
    %\newblock{Patrick Iglesias-Zemmour.}
    \newblock{\em Diffeomorphisms of $\Geod(\T^2)$.}
    \newblock{Blog post February 2022}.\\
    \verb|http://math.huji.ac.il/~piz/documents/DBlog-Rmk-DOGT2.pdf|
    
    \bibitem[IZL18]{IZL18}
    Patrick Iglesias-Zemmour and Jean-Pierre Laffineur,
    \newblock{\em Noncommutative geometry and diffeology: The case of orbifolds}.
    \newblock{J. Noncommut. Geom. 12(4): 1551--1572, 2018.}
    
    \bibitem[IKZ10]{IKZ10}
    Patrick Iglesias, Yael Karshon and Moshe Zadka.
    \newblock{\em Orbifolds as Diffeology.}
    \newblock{Transactions of the American Mathematical Society 362(6): 2811--2831, 2010.}
    
    \bibitem[IZP21]{IZP21}
    Patrick Iglesias-Zemmour and Elisa Prato,
    \newblock{\em Quasifolds, diffeology and noncommutative geometry}.
    \newblock{J. Noncommut. Geom. 15(2): 735--759, 2021}.
    
    \bibitem[Klo07]{Klo07}
    \newblock{Benoît Kloeckner.}
    \newblock\emph{Quelques Notions d'Espaces Stratifiés.}
    \newblock{Séminaire de Théorie Spectrale et Géométrie 26: 13--28, Grenoble, 2007/08.}
    
    \bibitem[PS74]{PS74}
    \newblock{Peter Stefan.}
    \newblock{\em Accessible sets, orbits, and foliations with singularities.}
    \newblock{Proc. London Math. Soc. 3(29): 699--713, 1974.}
    
    \bibitem[JW17]{JW17}
    \newblock{Jordan Watts.}
    \newblock{\em The Differential Structure of an Orbifold.}
    \newblock{Rocky Mountain Journal of Mathematics 47(1): 289--327, 2017.}
    
    \bibitem[SY19]{SY19}
    \newblock{Shoji Yokura.}
    \newblock{\em Decomposition Spaces and POSet-Stratified Spaces.}
    \newblock{Tbilisi Math. J. 13(2): 101--127, 2020.}
    \newblock{\scriptsize\verb!https://arxiv.org/abs/1912.00339v1!}
    
\end{thebibliography}


%%%%%%%%%%%%%%%%%%%%%%%%%%%%%%%%%%%%%%%%%%%%%%%%%%%%
% Fin du document
%%%%%%%%%%%%%%%%%%%%%%%%%%%%%%%%%%%%%%%%%%%%%%%%%%%%
\end{document}
