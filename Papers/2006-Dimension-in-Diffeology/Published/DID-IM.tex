


%%%%%%%%%%%%%%%%%%%%%%%%%%%%%%%%%%%%%%%%%%%%%%%%%%%%%%%%%%
%%
%%  PROJECT: Dimension in Diffeology (Published Version)
%%  FILENAME: DID-Published.tex
%%
%%  Original Author: Patrick Iglesias-Zemmour
%%  Published in the Indangationes 2007
%%  Modernized for the Diffeology Archives in December 2025.
%%
%%  This is a self-contained document.
%%
%%%%%%%%%%%%%%%%%%%%%%%%%%%%%%%%%%%%%%%%%%%%%%%%%%%%%%%%%%

\documentclass[11pt,reqno]{amsart}

%%====================================================================
% MARK: - Preamble (Self-Contained)
%%====================================================================

% --- Core Packages ---
\usepackage{amsmath}
\usepackage{amssymb}
\usepackage{amsthm}
\usepackage[hidelinks]{hyperref}
\usepackage{graphicx}
\usepackage{microtype}
\usepackage{hyperref}

% --- Fonts ---
\usepackage[cal=scr,uppercase=upright,greeklowercase=upright,utopia]{mathdesign}

% --- Layout ---
\parindent 0mm
\parskip 0.5em plus 1pt
\allowdisplaybreaks

% --- Figures and Diagrams ---
\usepackage{tikz-cd}
\usetikzlibrary{calc}
\tikzcdset{arrow style=tikz, diagrams={>={Straight Barb[scale=0.8]}}}

% --- Theorem Environments ---
\theoremstyle{plain}
\newtheorem{article}{}[section]
\renewenvironment{proof}{\noindent{\sc Proof --}}{\nolinebreak\hfill$\blacksquare$}

% --- Minimal Set of Required Macros ---

% Standard Sets
\newcommand{\RR}{\mathbf{R}}
\newcommand{\CC}{\mathbf{C}}
\newcommand{\NN}{\mathbf{N}}
\newcommand{\KK}{\mathbf{K}}
\newcommand{\QQ}{\mathbf{Q}}
\newcommand{\ZZ}{\mathbf{Z}}

% Calligraphic Letters
\newcommand{\cD}{\mathcal{D}}
\newcommand{\cF}{\mathcal{F}}
\newcommand{\cI}{\mathcal{I}}
\newcommand{\cO}{\mathcal{F}}
\newcommand{\cX}{\mathcal{X}}

% Mathematical Operators
\DeclareMathOperator{\Cinfty}{\mathcal{C}^\infty}
\DeclareMathOperator{\Diff}{Diff}
\DeclareMathOperator{\Domains}{Domains}
\DeclareMathOperator{\dimension}{dim}
\DeclareMathOperator{\dom}{dom}
\DeclareMathOperator{\rank}{rank}
\DeclareMathOperator{\D}{D}
\DeclareMathOperator{\Param}{Param}
\DeclareMathOperator{\Gen}{Gen}

% Helper Macros
\newcommand{\art}[1]{(art.~\ref{#1})}
\newcommand{\fig}[1]{(fig.~\ref{#1})}
\newcommand{\id}{\mathbf{1}}
\newcommand{\qtext}[1]{\quad\text{#1}\quad}
\newcommand{\set}[1]{\left\{ #1 \right\}}
\newcommand{\interval}[1]{\left[ #1 \right[}
\newcommand{\catDiff}{\{Diffeology\}}
\newcommand{\artplus}[2]{(art.~\ref{#1}, #2)}

%%%%%%%%%%%%%%%%%%%%%%%%%%%%%%%%%%%%%%%%%%%%%%%%%%%%%%%%%%
%%
%% MARK: Document Information
%%
%%%%%%%%%%%%%%%%%%%%%%%%%%%%%%%%%%%%%%%%%%%%%%%%%%%%%%%%%%

\begin{document}

\title{Dimension in Diffeology}
\author{Patrick Iglesias-Zemmour}
\thanks{CNRS, France \& The Hebrew University of Jerusalem, Israel.}
\date{\today}

\maketitle

%%%%%%%%%%%%%%%%%%%%%%%%%%%%%%%%%%%%%%%%%%%%%%%%%%%%%%%%%%
%%
%% MARK: Abstract
%%
%%%%%%%%%%%%%%%%%%%%%%%%%%%%%%%%%%%%%%%%%%%%%%%%%%%%%%%%%%

\begin{abstract}
\noindent We define the \emph{dimension function} for diffeological spaces,
a simple but new invariant.
We show then how it can be applied to prove that,
for two different integers $m$ and $n$ the quotient spaces $\RR^m\!/\mathrm{O}(m)$ and $\RR^n\!/\mathrm{O}(n)$ are not diffeomorphic,
and not diffeomorphic to the half-line $[0,\infty) \subset \RR$.
\end{abstract}

%%%%%%%%%%%%%%%%%%%%%%%%%%%%%%%%%%%%%%%%%%%%%%%%%%%%%%%%%%
%%
%% MARK: Introduction
%%
%%%%%%%%%%%%%%%%%%%%%%%%%%%%%%%%%%%%%%%%%%%%%%%%%%%%%%%%%%

\section*{Introduction}
\addcontentsline{toc}{part}{Introduction}

The notion of \emph{dimension} in diffeology,
which we introduce in Section \ref{Dimension-of-a-diffeology},
gives a quick and easy answer to the question:
\emph{For two different integers $n$ and $m$,
are the diffeological spaces $\Delta_n = \RR^n\!/\mathrm{O}(n)$ and $\Delta_m = \RR^m\!/\mathrm{O}(m)$ diffeomorphic?\/}
In Section \ref{Examples-of-the-half-lines},
we show that since $\dim(\Delta_n) =n$ and since the dimension is a diffeological invariant,
the answer is \emph{No, they are not}.
This method simplifies a partial result,
obtained in a much more complicated way in \cite{Igl85},
stating that $\Delta_1$ and $\Delta_2$ are not diffeomorphic.
The half-line $\Delta_\infty = [0, \infty[ \subset \RR$ is a similar example for which $\dim(\Delta_\infty) = \infty$.
Hence,
$\Delta_m$ is not diffeomorphic to the half-line $\Delta_\infty$ for any integer $m$.
Dimension appears to be a simple but a powerful diffeological invariant.
Hopefully,
the diffeological dimension coincides with the usual definition when the diffeology space is a manifold.
That is,
when the diffeology is generated by local diffeomorphisms with $\RR^n$,
for some integer $n$.
For more details,
the reader who is not familiar with diffeology can look at \cite{PIZ05}.

\bigskip
\textbf{Thanks} I am pleased to thank Prof. Hans Duistermaat for his advices concerning the writing of this paper.
It is also a pleasure to thank the Hebrew University of Jerusalem Israel,
for its warm hospitality.

%%%%%%%%%%%%%%%%%%%%%%%%%%%%%%%%%%%%%%%%%%%%%%%%%%%%%%%%%%
%%
%% MARK: Diffeologies and diffeological spaces
%%
%%%%%%%%%%%%%%%%%%%%%%%%%%%%%%%%%%%%%%%%%%%%%%%%%%%%%%%%%%
\section{Diffeologies and diffeological spaces}
\label{Diffeologies-and-diffeological-spaces}

\begin{article}[Parametrizations of a set]
\label{Parametrizations-of-a-set}
Let $X$ be a set,
we call \emph{parametrization in $X$} any map defined on any open subset of any space $\RR^n$ for any integer $n$,
with values in $X$.
The set of all the parametrizations in $X$ will be denoted by $\Param(X)$.
For any parametrization $P : U \to X$,
the \emph{numerical domain} $U$ is called the \emph{domain} of $P$ and is denoted by $\dom(P)$.
If $U$ is a subset of $\RR^n$ we say that $P$ is a \emph{$n$-parametrization},
the integer $n$ will be called the \emph{dimension of the parametrization $P$},
and we shall denote $\dim(P) = n$.
\end{article}

\begin{article}[Diffeology and diffeological spaces]
\label{Diffeology-and-diffeological-spaces}
Let $X$ be a set.
A \emph{diffeology on $X$} is a set $\cD$ of parametrizations in $X$,
that is ${\cD} \subset \Param(X)$,
such that
\begin{enumerate}
\item[D1.] \emph{Covering}\hspace{1em} Every point of $X$ is contained in the range of some $P\in \cD$.
\item[D2.] \emph{Locality}\hspace{1em} If $P \in\Param(X)$ and there exist $P_i \in {\cD}$,
$i\in \cI$,
such that the $\dom(P_i)$,
$i\in \cI$ form an open covering of $\dom(P)$ and $P_i = P_j$ on $\dom(P_i) \cap \dom(P_j)$ for every $i, j\in \cI$,
then $P\in {\cD}$.
\item[D3.] \emph{Smooth compatibility}\hspace{1em} If $P \in \cD$ and $F$ is a $\Cinfty$ mapping from a open subset $V$ of $\RR^m$ to $\dom(P)$,
then $P \circ F \in {\cD}$.
\end{enumerate}
The first axiom can be replaced by the original ``constant parametrizations belong to $\cD$''.
The second axiom clearly means that to be an element of $\cD$ is a local condition.
Note that the third axiom implies in particular that the restriction of any element of $\cD$ to an open subset of its domain still belongs to $\cD$.
Equipped with a diffeology $\cD$,
$X$ is a \emph{diffeological space}\footnote{If $X$ denotes a diffeological space, the elements of its diffeology are usually called the \emph{plots} of $X$.}.
Note that the definition of a diffeology does not assume any pre-existing structure on the underlying set.
\end{article}

\begin{article}[Smooth maps and diffeomorphisms]
\label{Differentiable-maps-and-diffeomorphisms}
Let $X$ and $X'$ be two sets equipped with the diffeologies $\cD$ and $\cD'$ respectively.
A map $F : X \to Y$ is said to be \emph{smooth} if for each $P \in \cD$ we have $F \circ P \in \cD'$.
The set of smooth maps from $X$ to $Y$ is denoted by $\Cinfty(X,Y)$.
A bijective map $F : X \to Y$ is said to be a \emph{diffeomorphism} if both $F$ and $F^{-1}$ are smooth.
The set of diffeomorphisms of $X$ is a group denoted by ${\rm Diff}(X)$.
Diffeological spaces are the objects of the category \{Diffeology\} whose morphisms are \emph{smooth maps},
and isomorphisms are \emph{diffeomorphisms}.
This category is stable by set theoretic operations.
In particular,
let $\sim$ be an equivalence relation on $X$,
let $Q = X/\!\sim$ and $\pi :X \to Q$ be the projection.
There exists a natural \emph{quotient diffeology} on $Q$,
for which $\pi$ is smooth,
defined by the parametrizations which can be lifted locally along $\pi$ by elements of $\cD$.
As well,
there exists on every subset $A \subset X$ a natural \emph{subset diffeology},
for which the inclusion is smooth,
defined by the elements of $\cD$ which take their values in $A$.
\end{article}

\begin{article}[Generating families]
\label{Generating-families}
Let $X$ be a set,
and let $\cF\subset \Param(X)$.
There exists a smallest diffeology $\cD$ containing $\cF$.
We call it the \emph{diffeology generated} by $\cF$ and we denote $\cD = \langle \cF \rangle$.
A parametrization $P : U \to X$ belongs to $\langle \cF \rangle$ if and only if for any point $r$ of $U$ there exists an open subset $V \subset U$ containing $r$ such that either $P \restriction V$ is a constant parametrization,
or there exists $F : W \to X$ element of $\cF$,
and a smooth mapping $Q\in \Cinfty(V,W)$ such that $P\restriction V = F \circ Q$.
Note that,
for example,
the empty family generates the discrete diffeology.
\end{article}

\begin{article}[Standard diffeology on numerical domains]
\label{Standard-diffeology-of-domains}
This is the very basic example of diffeological space.
Any open set $U$ of any $\RR^n$ has a natural \emph{smooth diffeology} defined as the set of all smooth parametrizations of $U$.
Now,
for any numerical domain $U$ equipped with the smooth diffeology,
for any set $X$ equipped with a diffeology $\cD$ we have $\Cinfty(U,X) = \{\P \in \cD \mid \dom(\P) = U\}$.
And obviously,
the singleton $\{\id_U\}$ is a generating family of $U$.
\end{article}

%%%%%%%%%%%%%%%%%%%%%%%%%%%%%%%%%%%%%%%%%%%%%%%%%%%%%%%%%%
%%
%% MARK: Dimension of a diffeology
%%
%%%%%%%%%%%%%%%%%%%%%%%%%%%%%%%%%%%%%%%%%%%%%%%%%%%%%%%%%%
\section{Dimension of diffeological spaces}
\label{Dimension-of-a-diffeology}

\begin{article}[The dimension function of a diffeological space]
\label{The-dimension-map}
Let $X$ be a set.
We define the \emph{dimension} of any family $\cF$ of parametrizations of $X$ as
\[
    \dim(\cF) = \sup \{\dim(F) \mid F \in \cF \},
\]
where the dimension of a parametrization has been defined in Subsection \ref{Parametrizations-of-a-set}.
If for any $n \in \NN$ there exists $F \in \cF$ such that $\dim(F) = n$ then $\dim(\cF) = \infty$.
Let $X$ be a diffeological space,
and let $\cD$ be its diffeology.
Let $x \in X$,
we define the \emph{dimension of the diffeological space $X$ at the point $x$} as the infimum of the dimensions of the families of parametrizations generating the diffeology $\cD$ at the point $x$.
In other words,
\[
    \dim_x(X) = \inf \{ \dim(\cF) \mid \langle \cF \rangle = \cD_x \}.
\]
The \emph{dimension function} $x \mapsto \dim_x(X)$,
of the diffeological space $X$,
takes its values in $\NN \cup \{\infty\}$.
The \emph{global dimension} of $X$ can be defined as the supremum of the dimension map of $X$,
and we have
\[
    \dim(X) = \sup_{x \in X} \set{\dim_x(X)} = \inf\{ \dim(\cF) \mid \langle \cF \rangle = \cD \}.
\]
See \cite{PIZ05} for the proof of the second equality.
\end{article}

\begin{article}[The dimension map is a local invariant]
\label{The-dimension-map-is-a-finer-invariant}
Let $X$ and $X'$ be two diffeological spaces.
If $x \in X$ and $x' \in X'$ are two points related by a local (a fortiori global) diffeomorphism then $\dim_x(X) = \dim_{x'}(X')$.
\end{article}

\begin{article}[Dimensions of numerical domains]
\label{Dimensions-of-numerical-domains}
Let $U \subset \RR^n$ be an open set equipped with the smooth diffeology defined in Subsection \ref{Standard-diffeology-of-domains}.
We have,
$\dim(U) = n$.
And,
thanks to the proposition \ref{The-dimension-map-is-a-finer-invariant},
dimension for diffeological spaces coincides with the usual notion in the case of manifolds.
\end{article}

%%%%%%%%%%%%%%%%%%%%%%%%%%%%%%%%%%%%%%%%%%%%%%%%%%%%%%%%%%
%%
%% MARK: Examples of the half-lines
%%
%%%%%%%%%%%%%%%%%%%%%%%%%%%%%%%%%%%%%%%%%%%%%%%%%%%%%%%%%%
\section{Examples of the half-lines}
\label{Examples-of-the-half-lines}

\begin{article}[The half-lines $\Delta_n$]
\label{Dimension-of-Rn/On}
Let $\Delta_n = \RR^n/\mathrm{O}(n,\RR)$ equipped with the quotient diffeology,
$n \in \NN$.
So,
$\dim_0(\Delta_n) = n$,
and $\dim_x(\Delta_n) = 1$ if $x \neq 0$.
Thus,
$\dim(\Delta_n) = n$ and for $n \neq m$ the half-lines $\Delta_n$ and $\Delta_m$ are not diffeomorphic.
\end{article}

\begin{proof}
The case $n=0$ is trivial.
Let us assume $n>0$,
and let us denote by $\pi_n : \RR^n \to \Delta_n$ the projection from $\RR^n$ onto its quotient.
There is a natural bijection $f : \Delta_n \to [0,\infty[$ such that $f \circ \pi_n = \nu_n$,
where $\nu_n(x) = \lVert x\rVert^2$.
Now,
thanks to the uniqueness of quotients \cite{PIZ05},
we use $f$ to identify $\Delta_n$ with $[0,\infty[$ equipped with the diffeology $\cD_n$ generated by $\nu_n$.
The elements of $\cD_n$ consist of the parametrizations which locally can be lifted along $\nu_n$ by smooth parametrizations of $\RR^n$.
So,
since $\dim(\nu_n) = n$,
we get $\dim(\Delta_n) \leq n$.
Let us prove now that $\dim(\Delta_n) \geq n$.
Let us assume that $\nu_n$,
which is an element of $\cD_n$,
can be lifted locally at the point $0_n$,
along $P \in \cD_n$ with $\dim(P) = p < n$.
So,
there exists a smooth parametrization $\phi : V \to \dom(P)$ such that $P \circ \phi = \nu_n \restriction V$.
We can assume without loss of generality that $0_p \in \dom(P)$,
$P(0_p) = 0$ and $\phi(0_n) = 0_p$.
Now,
since $P$ is an element of $\cD_n$ there exists a smooth parametrization $\psi: W \to \RR^n$ such that $0_p \in W$ and $\nu_n \circ \psi = P \restriction W$.
Let $V' = \phi^{-1}(W)$,
and $F = \psi \circ \phi \restriction V'$,
we get $\nu_n \restriction V' = \nu_n \circ F$,
with $F \in \Cinfty(V', \RR^n)$,
$0_n \in V'$ and $F(0_n) = 0_n$,
that is $\lVert x\rVert^2 = \lVert F(x)\rVert^2$.
The second derivative of this identity computed at the point $0_n$ gives $\id_n = M^t M$ with $M = \D(F)(0_n)$.
But $M = A B$ with $A = \D(\psi)(0_p)$ and $B = \D(\phi)(0_n)$.
So,
$\id_n = B^t A^t A B$ which is impossible because $\rank(B) \leq p < n$.
Therefore,
$\dim(\Delta_n) = n$.
And,
since the dimension is a diffeological invariant,
$\Delta_n$ is not diffeomorphic to $\Delta_m$ for $n \neq m$.
\end{proof}

\begin{article}[The half-line $\Delta_\infty$]
\label{Dimension-of-the-half-line}
The dimension of a diffeological subspace $A \subset X$ can be less,
equal or even greater than the dimension of $X$.
The following example is a clear illustration of this phenomenon.
Let $\Delta_\infty = [0,\infty[ \subset \RR$ equipped with the subset diffeology.
So,
$\dim_0(\Delta_\infty) = \infty$ and $\dim_x(\Delta_\infty) = 1$ if $x \neq 0$.
Thus,
$\dim(\Delta_\infty) = \infty$,
and for any integer $m$,
$\Delta_\infty$ is not diffeomorphic to $\Delta_m$.
\end{article}

\begin{proof}
Let us assume that $\dim(\Delta_\infty) = N < \infty$.
For any integer $n$,
the map $\nu_n : \RR^n \to \Delta_\infty$,
defined by $\nu_n(x) = \lVert x\rVert^2$ belongs to $\cD_\infty$,
the subset diffeology on $[0,\infty[$.
Hence,
$\nu_n$ lifts locally at the point $0_n$ along some $P \in \cD_\infty$ where $\dim(P) =p \leq N$.
Now,
let us choose $n > N$.
So,
$P$ belongs to some $\Cinfty(U,\RR)$ with $\mathrm{Val}(P) \subset [0,\infty[$,
and there exists a smooth parametrization $\phi : V \to U$ such that $P \circ \phi = \nu_n \restriction V$.
We can assume,
without loss of generality,
that $0_p \in U$,
$\phi(0_n) = 0_p$,
and thus $P(0_p) = 0$.
Now,
the first derivative of $\nu_n$ at a point $x \in V' = \phi^{-1}(V)$ is given by $x = \D(P)(\phi(x)) \circ \D(\phi)(x)$.
Since $P$ is smooth and positive,
and since $P(0)=0$ we have $\D(P)(0_p) = 0$.
So,
the second derivative of $\nu_n$ computed at the point $0_n$ gives $\id_n = M^t H M$,
where $M = \D(\phi)(0)$ and $H = \D^2(P)(0)$.
But since $\rank(M) \leq p \leq N$ and $n > N$ this is impossible.
Therefore $\dim(\Delta_\infty) = \infty$.
\end{proof}

%%%%%%%%%%%%%%%%%%%%%%%%%%%%%%%%%%%%%%%%%%%%%%%%%%%%%%%%%%
%%
%% MARK: Other examples
%%
%%%%%%%%%%%%%%%%%%%%%%%%%%%%%%%%%%%%%%%%%%%%%%%%%%%%%%%%%%
\section{Some other examples}
\label{Some-other-examples}

\begin{article}[Dimension zero spaces are discrete]
\label{dimension-zero-spaces-are-discrete}
A diffeological space is said to be discrete if its diffeology is generated by the empty set.
A diffeological space has dimension zero if and only if it is discrete.
\end{article}

\begin{article}[Has the set $\{0,1\}$ dimension 1?]
\label{has-the-set-0-1-dimension-1}
Let us consider the set $\{0,1\}$ and $\pi: \RR \to \{0,1\}$ be the parametrization defined by:
$\pi(x) = 0$ if $x \in \QQ$,
and $\pi(x) = 1$ otherwise.
Let $\{0,1\}_{\pi}$ be the set $\{0,1\}$ equipped with the diffeology generated by $\pi$.
Thus $\dim(\{0,1\}_{\pi})=1$.
So,
a diffeological space made of a finite number of points may have a non zero dimension.
\end{article}

\begin{article}[Dimension of tori]
\label{Dimension-of-tori}
Let $\Gamma \subset \RR$ be a strict subgroup of $(\RR,+)$ and let $T_\Gamma$ be the quotient $\RR/\Gamma$.
So,
$\dim(T_\Gamma) = 1$.
This applies in particular to the circles $\RR/ a \ZZ$,
with length $a >0$,
or to \emph{irrational tori} \cite{DI85} when $\Gamma$ is generated by more than one generators,
rationally independent.
\end{article}

%%%%%%%%%%%%%%%%%%%%%%%%%%%%%%%%%%%%%%%%%%%%%%%%%%%%%%%%%%
%%
%% MARK: Bibliography
%%
%%%%%%%%%%%%%%%%%%%%%%%%%%%%%%%%%%%%%%%%%%%%%%%%%%%%%%%%%%

\begin{thebibliography}{XXXX}

\bibitem[Che77]{Che77}
Kuo Tsai Chen.
\newblock \emph{Iterated path integral}.
\newblock Bull. of Am. Math. Soc., 83(5):831--879, 1977.

\bibitem[Don84]{Don84}
Paul Donato.
\newblock \emph{Revêtement et groupe fondamental des espaces différentiels homogènes}.
\newblock Thèse de doctorat d'état, Université de Provence, Marseille, 1984.

\bibitem[DI85]{DI85}
Paul Donato \& Patrick Iglesias.
\newblock \emph{Exemple de groupes difféologiques : flots irrationnels sur le tore}.
\newblock Compte Rendu de l'Académie des Sciences, 301(4), 1985.

\bibitem[Igl85]{Igl85}
Patrick Iglesias.
\newblock \emph{Fibrés difféologiques et homotopie}.
\newblock Thèse de doctorat d'état, Université de Provence, Marseille, 1985.
\newblock \url{http://math.huji.ac.il/~piz/documents/TEPI.pdf}

\bibitem[Igl86]{Igl86}
Patrick Iglesias.
\newblock \emph{Difféologie d'espace singulier et petits diviseurs}.
\newblock Compte Rendu de l'Académie des Sciences, 302:519--522, 1986.

\bibitem[Igl87]{Igl87}
Patrick Iglesias.
\newblock \emph{Connexions et difféologie}.
\newblock In \emph{Aspects dynamiques et topologiques des groupes infinis de transformation de la mécanique}, volume 25, pages 61--78. Hermann, Paris, 1987.

\bibitem[Igl95]{Igl95}
Patrick Iglesias.
\newblock \emph{La trilogie du Moment}.
\newblock Ann. Inst. Fourier, 45, 1995.

\bibitem[IL90]{IL90}
Patrick Iglesias \& Gilles Lachaud.
\newblock \emph{Espaces différentiables singuliers et corps de nombres algébriques}.
\newblock Ann. Inst. Fourier, Grenoble, 40(1):723--737, 1990.

\bibitem[PIZ05a]{PIZ05a}
Patrick Iglesias-Zemmour.
\newblock \emph{Diffeology of the infinite Hopf fibration}.
\newblock In \emph{Geometry and Topology of Manifolds}.
\newblock Banach Center Publication, vol. 76. Institute of Mathematics, Polish Academy of Sciences, Warszawa, 2007.
\newblock Web based e-print, 2005: \url{http://math.huji.ac.il/~piz/documents/DIHF.pdf}

\bibitem[PIZ05]{PIZ05}
Patrick Iglesias-Zemmour.
\newblock \emph{Diffeology}.
\newblock Web based e-print, 2005: \url{http://math.huji.ac.il/~piz/documents/Diffeology.pdf}

\bibitem[IKZ05]{IKZ05}
Patrick Iglesias, Yael Karshon and Moshe Zadka.
\newblock \emph{Orbifolds as diffeologies}.
\newblock 2005. \url{http://arxiv.org/abs/math.DG/0501093}

\bibitem[Sou81]{Sou81}
Jean-Marie Souriau.
\newblock \emph{Groupes différentiels}.
\newblock In \emph{Lecture notes in mathematics}, volume 836, pages 91--128. Springer Verlag, New-York, 1981.

\bibitem[Sou84]{Sou84}
Jean-Marie Souriau.
\newblock \emph{Groupes différentiels et physique mathématique}.
\newblock In \emph{Collection travaux en cours}, pages 75--79. Hermann, Paris, 1984.

\bibitem[Whi43]{Whi43}
Hassler Whitney.
\newblock \emph{Differentiable even functions}.
\newblock Duke Math. J., 10:159--160, 1943.

\end{thebibliography}

\bigskip
\flushright{Patrick Iglesias-Zemmour \\ Einstein Institute \\
The Hebrew University of Jerusalem \\
Campus Givat Ram \\
Jerusalem 91904 \\
ISRAEL }
\flushright{\tt http://math.huji.ac.il/$\sim$piz/
\\ \tt piz@math.huji.ac.il}

\end{document}
