%%%%%%%%%%%%%%%%%%%%%%%%%%%%%%%%%%%%%%%%%%%%%%%%%%%%%%%%%%
%%
%%  PROJECT: Dimension in Diffeology
%%  FILENAME: DID.tex
%%
%%  Original Author: Patrick Iglesias-Zemmour
%%  First typeset June 2006
%%  Modernized for the Diffeology Archives in December 2025.
%%
%%  This is a self-contained document.
%%
%%%%%%%%%%%%%%%%%%%%%%%%%%%%%%%%%%%%%%%%%%%%%%%%%%%%%%%%%%

\documentclass[11pt,reqno]{amsart}

%%====================================================================
% MARK: - Preamble (Self-Contained)
%%====================================================================

% --- Core Packages ---
\usepackage{amsmath}
\usepackage{amssymb}
\usepackage{amsthm}
\usepackage[hidelinks]{hyperref}
\usepackage{graphicx}
\usepackage{microtype}
\usepackage{hyperref}

% --- Fonts ---
\usepackage[cal=scr,uppercase=upright,greeklowercase=upright,utopia]{mathdesign}

% --- Layout ---
\parindent 0mm
\parskip 0.5em plus 1pt
\allowdisplaybreaks

% --- Figures and Diagrams ---
\usepackage{tikz-cd}
\usetikzlibrary{calc}
\tikzcdset{arrow style=tikz, diagrams={>={Straight Barb[scale=0.8]}}}

% --- Theorem Environments ---
\theoremstyle{plain}
\newtheorem{article}{}[section]
\renewenvironment{proof}{\noindent{\sc Proof --}}{\nolinebreak\hfill$\blacksquare$}

% --- Minimal Set of Required Macros ---

% Standard Sets
\newcommand{\RR}{\mathbf{R}}
\newcommand{\CC}{\mathbf{C}}
\newcommand{\NN}{\mathbf{N}}
\newcommand{\KK}{\mathbf{K}}
\newcommand{\QQ}{\mathbf{Q}}
\newcommand{\ZZ}{\mathbf{Z}}

% Calligraphic Letters
\newcommand{\cD}{\mathcal{D}}
\newcommand{\cF}{\mathcal{F}}
\newcommand{\cI}{\mathcal{I}}
\newcommand{\cO}{\mathcal{F}}
\newcommand{\cX}{\mathcal{X}}

% Mathematical Operators
\DeclareMathOperator{\Cinfty}{\mathcal{C}^\infty}
\DeclareMathOperator{\Diff}{Diff}
\DeclareMathOperator{\Domains}{Domains}
\DeclareMathOperator{\dimension}{dim}
\DeclareMathOperator{\dom}{dom}
\DeclareMathOperator{\rank}{rank}
\DeclareMathOperator{\D}{D}
\DeclareMathOperator{\Param}{Param}
\DeclareMathOperator{\Gen}{Gen}

% Helper Macros
\newcommand{\art}[1]{(art.~\ref{#1})}
\newcommand{\fig}[1]{(fig.~\ref{#1})}
\newcommand{\id}{\mathbf{1}}
\newcommand{\qtext}[1]{\quad\text{#1}\quad}
\newcommand{\set}[1]{\left\{ #1 \right\}}
\newcommand{\interval}[1]{\left[ #1 \right[}
\newcommand{\catDiff}{\{Diffeology\}}
\newcommand{\artplus}[2]{(art.~\ref{#1}, #2)}

%%%%%%%%%%%%%%%%%%%%%%%%%%%%%%%%%%%%%%%%%%%%%%%%%%%%%%%%%%
%%
%% MARK: Document Information
%%
%%%%%%%%%%%%%%%%%%%%%%%%%%%%%%%%%%%%%%%%%%%%%%%%%%%%%%%%%%

\begin{document}
  
  \title{Dimension in Diffeology}
  \author{Patrick Iglesias-Zemmour}
  \thanks{CNRS, LATP Université de Provence, France \& The Hebrew University of Jerusalem, Israel.}
  \date{First typeset June 2006}
  
  \maketitle
  
  %%%%%%%%%%%%%%%%%%%%%%%%%%%%%%%%%%%%%%%%%%%%%%%%%%%%%%%%%%
  %%
  %% MARK: Abstract
  %%
  %%%%%%%%%%%%%%%%%%%%%%%%%%%%%%%%%%%%%%%%%%%%%%%%%%%%%%%%%%
  
  \begin{abstract}
    \noindent The notion of \emph{dimension} for diffeologies,
    introduced here,
    generalizes the dimension of manifolds.
    More appropriate to diffeology is the \emph{dimension map}.
    We give some elementary properties of this dimension under several diffeological constructions.
    We illustrate these definitions with the example of the quotient spaces $\Delta_n = \RR^n/\mathrm{O}(n)$ for which $\dimension(\Delta_n) = n$,
    due to the singularity at the origin.
    Then,
    we deduce that the dimension of the half-line $\Delta_\infty = [0,\infty[ \subset \RR$,
    equipped with the subset diffeology,
    is infinite at the origin.
    And we show how this diffeological invariant can be used,
    in particular,
    to completely avoid topological considerations in some diffeological questions.
  \end{abstract}
  
  %%%%%%%%%%%%%%%%%%%%%%%%%%%%%%%%%%%%%%%%%%%%%%%%%%%%%%%%%%
  %%
  %% MARK: Introduction
  %%
  %%%%%%%%%%%%%%%%%%%%%%%%%%%%%%%%%%%%%%%%%%%%%%%%%%%%%%%%%%
  
  \section*{Introduction}
  \addcontentsline{toc}{part}{Introduction}
  
  The introduction and the use of \emph{dimension} in diffeology \art{Dimension-of-a-diffeology} gives us a quick and easy answer to the question:
  ``Are the diffeological spaces $\Delta_n = \RR^n/\mathrm{O}(n)$ and $\Delta_m = \RR^m/\mathrm{O}(m)$ diffeomorphic?''.
  The answer is ``No'' ($n\neq m$) since $\dimension(\Delta_n) =n$ \art{Dimension-of-Rn/On},
  and since the dimension is a diffeological invariant.
  This method simplifies the partial result,
  obtained in a much more complicated way in \cite{Igl85},
  that $\Delta_1$ and $\Delta_2$ are not diffeomorphic.
  Another example:
  computing the dimension of the half-line $\Delta_\infty = [0, \infty[ \subset \RR$ we obtain:
  $\dimension(\Delta_\infty) = \infty$,
  due to the ``singularity'' at the origin \art{Dimension-of-the-half-line}.
  Hence,
  $\Delta_m$ is not diffeomorphic to the half-line $\Delta_\infty$ for any integer $m$.
  Dimension appears to be the simplest diffeological invariant introduced until now,
  and a useful one.
  
  Moreover dimension can,
  in some situations,
  avoids topological considerations for solving pure diffeological questions.
  For example,
  if we want to show that any diffeomorphism of the half line $\Delta_\infty$ preserves the origin it is possible to note that diffeomorphisms are homeomorphisms and then use the fact that any homeomorphism preserves the origin.
  But,
  we can just mention that the dimension \art{The-dimension-map} of $\Delta_\infty$ is $\infty$ at the origin and 1 elsewhere.
  Then,
  since the dimension map is a diffeological invariant \art{The-dimension-map-is-a-finer-invariant},
  the origin is preserved.
  This kind of consideration avoid topology where it is not needed,
  and make diffeology technics still more independent from topology.
  As an application,
  we characterize the diffeomorphisms of $\Delta_\infty$ \art{The-diffeomorphisms-of-the-half-line}.
  For all these reasons it seems that the notion of dimension in diffeology deserved to be emphasized,
  it is what this paper is intended to do.
  
  For the reader who still doesn't know:
  diffeology is a theory which generalizes ordinary differential geometry by including,
  in its scope,
  a huge catalog of objects going from quotient spaces,
  regarded by many people as ``bad spaces'',
  to infinite dimensional spaces,
  like spaces of smooth maps.
  Between these two extremes,
  we find manifolds,
  orbifolds and all the usual differential objects.
  
  The category $\set{\text{Diffeology}}$ is very stable by quotients,
  subsets,
  products,
  sums.
  Many examples and constructions can be found now to illustrate this theory in the web-based document \cite{PIZ05}.
  
  The notion of dimension of a diffeological space is rather simpe and natural,
  it involves the notion of generating family.
  A generating family of a diffeology $\cD$ is a set of parametrizations $\cF$ such that $\cD$ is the finest diffeology containing $\cF$.
  More precisely,
  $\cD$ is the intersection of all the diffeologies containing $\cF$.
  Then,
  we define the dimension of a generating family as the supremum of the dimensions of the domains of its elements.
  And the dimension of a diffeological space is defined as the infimum of the dimensions of its generating families.
  This dimension is an integer or is infinite.
  
  Because diffeological spaces are not always homogeneous nor locally transitive,
  they may have ``singularities'',
  this \emph{global dimension} needs to be refined into a \emph{dimension map}.
  That is,
  a map from the diffeological space to the integers extended by the infinite,
  which associates to each point the dimension of the space at this point.
  We show that the dimension of a diffeological space is the supremum of this dimension map.
  Naturally,
  the dimension map is an invariant of the category \{Diffeology\}.
  
  This diffeological dimension coincides with the usual definition when the diffeology space is a manifold.
  That is,
  when the diffeology is generated by local diffeomorphisms with $\RR^n$,
  for some integer $n$.
  
  \textbf{Thanks} I am pleased to thank Yael Karshon for a pertinent remark about the use of the germs of a diffeology :) And also Henrique Macias for having incited me to publish this paper.
  
  %%%%%%%%%%%%%%%%%%%%%%%%%%%%%%%%%%%%%%%%%%%%%%%%%%%%%%%%%%
  %%
  %% MARK: Vocabulary and notations
  %%
  %%%%%%%%%%%%%%%%%%%%%%%%%%%%%%%%%%%%%%%%%%%%%%%%%%%%%%%%%%
  \section{Vocabulary and notations}
  
  Let us introduce first some vocabulary and let us fix some basics definitions and notations,
  related to ordinary differential geometry.
  
  \begin{article}[Domains and parametrizations]
    \label{Domains-and-parametrizations}
    We call \emph{$n$-numerical domain} any subset of the vector space $\RR^n$,
    open for the standard topology,
    that is any union of open balls.
    We denote by $\Domains(\RR^n)$ the set of all $n$-numerical domains,
    $$
      \Domains(\RR^n) = \{ U \subset \RR^n \mid U \mbox{ is open } \}.
      $$
    We denote by $\Domains$ the set of all the $n$-numerical domains when $n$ runs over the integers,
    $$
      U \in \Domains \ \Leftrightarrow \ \exists n \in \NN, \ U \in \Domains(\RR^n).
      $$
    The elements of $\Domains$ are called \emph{numerical domains},
    or simply \emph{domains}.
    
    Let $X$ be any set,
    we call \emph{$n$-parametrization} of $X$ any map $P : U \to X$ where $U$ is a $n$-numerical domain.
    We denote by $\Param_n(X)$ the set of all the $n$-parametrizations of $X$.
    $$
      \Param_n(X) = \{ P : U \to X \mid U \in \Domains(\RR^n) \}.
      $$
    We denote by $\Param(X)$ the set of all the $n$-parametrizations of $X$,
    when $n$ runs over the integers,
    $$
      \Param(X) = \bigcup_{n \in \NN} \Param_n(X) = \{ P : U \to X \mid U \in \Domains \}.
      $$
    We denote by $\dom(P)$ the \emph{definition set} of a parametrization $P$,
    we call it the \emph{domain} of $P$.
    That is,
    if $P \in \Param(X)$ and $P : U \to X$,
    then $\dom(P) = U$.
    
    Let $X$ be any set,
    and $x$ be some point of $X$.
    A \emph{superset} of $x$ is any subset $V$ of $X$ containing $x$.
    If $X$ is a topological space,
    an \emph{open superset} of $x$ is any superset of $x$ which is open for the given topology of $X$.
    In particular,
    if $r$ is some point of a numerical space $\RR^n$,
    an \emph{open superset} of $r$ is any $n$-domain $V$ containing the point $r$.
  \end{article}
  
  \begin{article}[Smooth maps and tangent linear maps]
    \label{Smooth-maps-and-tangent-linear-maps}
    Let $U$ be any $n$-domain and $V$ be any $m$-domain.
    Let $f : U \to V$ be a \emph{smooth map},
    that is an infinitely differentiable map,
    $f \in \Cinfty(U,V)$.
    
    The \emph{tangent linear map},
    or \emph{differential},
    of $f$ will be denoted by $\D(f)$,
    it is a smooth map from $U$ to the space of linear maps $\mathrm{L}(\RR^n, \RR^m)$.
    Its value at the point $x \in U$ will be denoted by $\D(f)(x)$ or by $\D(f)_x$,
    it is the matrix made up with the partial derivatives:
    $$
      \D(f)(x) = \begin{pmatrix}
      \frac{\partial y_1}{\partial x_1} & \cdots & \frac{\partial y_1}{\partial x_{n}} \\
      \vdots & \ddots & \vdots \\
      \frac{\partial y_m}{\partial x_1}  & \cdots & \frac{\partial y_m}{\partial x_n}
      \end{pmatrix}
      \mbox{ where } f : x = \begin{pmatrix}x_1 \\ \vdots \\ x_n\end{pmatrix} \mapsto y = \begin{pmatrix}y_1 \\ \vdots \\ y_m\end{pmatrix}.
      $$
    Every partial derivative is a smooth map defined on $U$ with values in $\RR$.
  \end{article}
  
  %%%%%%%%%%%%%%%%%%%%%%%%%%%%%%%%%%%%%%%%%%%%%%%%%%%%%%%%%%
  %%
  %% MARK: Diffeologies and diffeological spaces
  %%
  %%%%%%%%%%%%%%%%%%%%%%%%%%%%%%%%%%%%%%%%%%%%%%%%%%%%%%%%%%
  \section{Diffeologies and diffeological spaces}
  
  We remind the basic definition of a diffeology,
  and the very basic example of the standard diffeology of domains.
  More about diffeology or Chen's differentiable space can be found in \cite{Che77},
  \cite{Sou81},
  \cite{Sou84},
  \cite{Don84},
  \cite{DI85},
  \cite{Igl85},
  \cite{Igl86},
  \cite{Igl87},
  \cite{IL90},
  \cite{PIZ05},
  \cite{PIZ05a},
  \cite{IKZ05}.
  
  \begin{article}[Diffeology and diffeological spaces]
    \label{Diffeology-and-diffeological-spaces}
    Let $X$ be any set.
    A \emph{diffeology} of $X$ is a subset $\cD$ of parametrizations of $X$ whose elements,
    called \emph{plots},
    satisfy the following axioms.
    
    \begin{enumerate}
      \item[D1.] \emph{Covering}\hspace{1em}Any constant parametrization is a plot:
      for any point $x$ of $X$ and for any integer $n$,
      the constant map $\mathbf{x} : \RR^n \to X$,
      defined by $\mathbf{x}(r) = x$ for all $r$ in $\RR^n$,
      is a plot.
      \item[D2.] \emph{Locality}\hspace{1em}For any parametrization $P : U \to X$,
      if $P$ is locally a plot at each point of $U$ then $P$ is a plot.
      This means that,
      if for any $r$ in $U$ there exists a superset $V$ of $r$ such that the restriction $P \restriction V$ is a plot,
      then $P$ is a plot.
      \item[D3.] \emph{Smooth compatibility}\hspace{1em}The composite of a plot with any smooth parametrization of its source is a plot:
      let $P : U \to X$ be a plot and let $F$ be a parametrization belonging to $\Cinfty(V,U)$,
      where $V$ is any numerical domain,
      then $P \circ F$ is a plot.
    \end{enumerate}
    A set equipped with a diffeology is called a \emph{diffeological space}.
    Note that we do not assume any structure on $X$,
    $X$ is simply an amorphous set.
    To refer to any element of $\cD$ we shall say equivalently:
    a ``plot for the diffeology $\cD$'',
    or a ``plot of the diffeological space $X$''.
    
    \begin{itemize}
      \item[$\clubsuit$]The set of all the plots for the diffeology $\cD$,
      defined on a numerical domain $U$,
      will be denoted by $\cD(U)$.
      A plot defined on an $n$-domain $U \subset \RR^n$ is called a $n$-plot.
      The set of all the $n$-plots will be denoted by $\cD_n$.
    \end{itemize}
    
    The first axiom implies that each point of $X$ is covered by a plot.
    The second axiom clearly means that to be a plot is a local condition.
    Note that ``$P \restriction V$ is a plot'' implies that $V$ is a domain \art{Domains-and-parametrizations}.
    And,
    the third axiom defines the \emph{class of differentiability} of the diffeology.
    
    Formally,
    a diffeological space is a pair of a set $X$ and a diffeology $\cD$ of $X$.
    But most of the time the diffeological space will be denoted by a single capital letter or a group of letters.
    Note however that a diffeology contains always its underlying set as the set of its 0-plots,
    that is $X \simeq \cD_0 = \cD(\RR^0) =\cD(\{0\})$.
  \end{article}
  
  \begin{article}[Standard diffeology of domains]
    \label{Standard-diffeology-of-domains}
    The set $\Cinfty_\cO$ of all smooth parametrizations of a numerical domain $\cO$ is a diffeology.
    That is,
    for any domain $U$,
    $\Cinfty_\cO(U) = \Cinfty(U,\cO)$.
    We shall call this diffeology the \emph{standard diffeology},
    or the \emph{smooth diffeology},
    of the numerical domain $\cO$.
  \end{article}
  
  %%%%%%%%%%%%%%%%%%%%%%%%%%%%%%%%%%%%%%%%%%%%%%%%%%%%%%%%%%
  %%
  %% MARK: Differentiable maps
  %%
  %%%%%%%%%%%%%%%%%%%%%%%%%%%%%%%%%%%%%%%%%%%%%%%%%%%%%%%%%%
  \section{Differentiable maps}
  
  Diffeological spaces are the objects of a category whose morphisms are \emph{differentiable maps},
  and isomorphisms are \emph{diffeomorphisms}.
  
  \begin{article}[Differentiable maps and diffeomorphisms]
    \label{Differentiable-maps-and-diffeomorphisms}
    Let $X$ and $Y$ be two diffeological spaces and $F : X \to Y$ be a map.
    The map $F$ is said to be \emph{differentiable} if for each plot $P$ of $X$,
    $F \circ P$ is a plot of $Y$.
    The set of differentiable maps from $X$ to $Y$ is denoted $\Cinfty(X,Y)$.
    A bijective map $F : X \to Y$ is said to be a \emph{diffeomorphism} if both $F$ and $F^{-1}$ are differentiable.
    The set of all diffeomorphisms of $X$ is a group denoted by ${\rm Diff}(X)$.
  \end{article}
  
  \begin{article}[The Diffeology Category]
    The composite of two differentiable maps is a differentiable map.
    Diffeological spaces,
    together with differentiable maps,
    define a category,
    denoted by \catDiff.
    The isomorphisms of the category are diffeomorphisms.
  \end{article}
  
  \begin{article}[Plots are the smooth parametrizations]
    \label{Plots-are-smooth}
    The set of differentiable maps from a numerical domain $U$ to a diffeological space $X$ is just the set of plots of $X$ defined on $U$.
    This is a direct consequence of the axiom D3.
    Hence,
    $\Cinfty(U,X) = \cD(U)$.
    If we add that,
    for any numerical domain $X = \cO$,
    $\cD(U)$ is the set of smooth parametrizations $\Cinfty(U,\cO)$ \art{Standard-diffeology-of-domains},
    that justifies a posteriori the use of the symbol $\Cinfty$ to denote the differentiable maps between diffeological spaces.
    For the same reason we may use indifferently the word \emph{smooth} or the word \emph{differentiable}.
  \end{article}
  
  \begin{article}[Comparing diffeologies]
    \label{Comparing-diffeologies}
    A great number of constructions in diffeology use the following relation on diffeologies.
    A diffeology $\cD$ of a set $X$ is said to be \emph{finer} than another $\cD'$ if $\cD$ is included in $\cD'$.
    \emph{Fineness} is a partial order defined on the set of diffeologies of any set $X$.
    If $X$ denotes a set equipped with a diffeology $\cD$ and $X'$ the same set equipped with the diffeology $\cD'$,
    we denote $X \preceq X'$ to mean that $\cD$ is finer than $\cD'$.
    We say indifferently that $\cD$ is \emph{finer} than $\cD'$ or $\cD'$ \emph{coarser} than $\cD$.
    Note that \emph{coarser} means more plots and \emph{finer} means fewer plots.
  \end{article}
  
  \begin{article}[Discrete and coarse diffeology]
    \label{Discrete-and-coarse-diffeology}
    There are a priori two universal diffeologies defined on any set $X$.
    \begin{itemize}
      \item[$\clubsuit$] The finest diffeology,
      finer than any other diffeology.
      This diffeology is called \emph{discrete diffeology}.
      The plots of the discrete diffeology are the locally constant parametrizations.
      \item[$\spadesuit$] The coarsest diffeology,
      containing any other diffeology.
      This diffeology is called the \emph{coarse diffeology}.
      The plots of the coarse diffeology are all the parametrizations of $X$,
      that is the whole set ${\rm Param}(X)$.
    \end{itemize}
    
    In these two cases,
    the three axioms of diffeology,
    covering,
    locality and smooth compatibility,
    are obviously satisfied.
    Therefore,
    any diffeology is somewhere between the discrete and the coarse diffeologies.
  \end{article}
  
  \begin{article}[Intersecting diffeologies]
    \label{Intersecting-diffeologies}
    Let $X$ be a set and $\cF$ be any family of diffeologies of $X$.
    The intersection
    $$
      \inf(\cF) = \bigcap_{\cD \in \cF} \cD
      $$
    is a diffeology.
    It is the coarsest diffeology contained in every element of $\cF$,
    the finest being the discrete diffeology.
    This proposition is used to prove that every family of diffeology has a \emph{supremum} and an \emph{infimum}.
    In other words,
    diffeologies form a \emph{lattice}.
  \end{article}
  
  %%%%%%%%%%%%%%%%%%%%%%%%%%%%%%%%%%%%%%%%%%%%%%%%%%%%%%%%%%
  %%
  %% MARK: Generating families
  %%
  %%%%%%%%%%%%%%%%%%%%%%%%%%%%%%%%%%%%%%%%%%%%%%%%%%%%%%%%%%
  \section{Generating families}
  
  Generating families are a main tool for defining diffeologies,
  starting with any set of parametrizations.
  They are defined by the following proposition.
  
  \begin{article}[Generating families]
    \label{Generating-families}
    Let $X$ be a set,
    let $\cF$ be any subset of parametrizations of $X$.
    There exists a finest diffeology containing $\cF$.
    This diffeology will be called the diffeology \emph{generated} by $\cF$ and denoted $\langle \cF \rangle$.
    This diffeology is the infimum \art{Intersecting-diffeologies} of all diffeologies containing $\cF$,
    that is the intersection of all the diffeologies containing $\cF$.
    Given a diffeological space $X$,
    a family $\cF$ generating the diffeology of $X$ is called a \emph{generating family} for $X$.
    The plots of the diffeology generated by $\cF$ are given by the following characterization
    \begin{itemize}
      \item[$\clubsuit$] A parametrization $P : U \to X$ is a plot of the diffeology generated by $\cF$ if and only if for any point $r$ of $U$ there exists a superset $V$ of $r$ such that either $P \restriction V$ is a constant parametrization,
      or there exists a parametrization $Q : W \to X$ element of $\cF$ and a smooth parametrization $F$ from $V$ to $W$ such that $P \restriction V = Q \circ F$.
    \end{itemize}
    In the second case,
    we say that the plot $P$ \emph{lifts locally along} $F$,
    or that $Q$ is a \emph{local lifting} of $P$ along $F$ \fig{fig-Local-lifting}.
    Note that the process of generating family is a \emph{projector},
    that is for any diffeology $\cD$ we have $\langle \cD \rangle = \cD$.
  \end{article}
  
  \begin{article}[Generated by the empty set]
    As a remark,
    note that for any set $X$ the empty family $\cF = \varnothing \subset {\rm Param}(X)$ generates the discrete diffeology.
  \end{article}
  
  \begin{figure}[ht]
    \centerline{\includegraphics[width=0.5\textwidth]{Figures/fig-local-lifting.pdf}}
    \caption{Local lifting of $P$ along $F$.}\label{fig-Local-lifting}
  \end{figure}
  
  %%%%%%%%%%%%%%%%%%%%%%%%%%%%%%%%%%%%%%%%%%%%%%%%%%%%%%%%%%
  %%
  %% MARK: Dimension of a diffeology
  %%
  %%%%%%%%%%%%%%%%%%%%%%%%%%%%%%%%%%%%%%%%%%%%%%%%%%%%%%%%%%
  \section{Dimension of a diffeology}
  \label{Dimension-of-a-diffeology}
  
  We define,
  in this paragraph,
  the \emph{global dimension} of a diffeology or a diffeological space.
  This dimension is a diffeological invariant.
  Further we will see a finer invariant,
  the dimension map.
  
  \begin{article}[Dimension of a parametrization]
    \label{Dimension-of-a-parametrization}
    Let $P$ be any $n$-parametrization of some set $X$,
    for some integer $n$.
    We say that $n$ is the \emph{dimension} of $P$,
    and we denote it by $\dimension(P)$.
    That is,
    $$
      \mbox{For all } P \in \Param(X), \ \dimension(P) = n \ \Leftrightarrow \ P \in \Param_n(X).
      $$
  \end{article}
  
  \begin{article}[Dimension of a family of parametrizations]
    \label{Dimension-of-a-family-of-parametrizations}
    Let $X$ be a set and let $\cF$ be any family of parametrizations of $X$.
    We define the \emph{dimension} of $\cF$ as the supremum of the dimensions of its elements \art{Domains-and-parametrizations},
    $$
      \dimension(\cF) = \sup \{ \dimension(F) \mid F \in \cF \}.
      $$
    Note that $\dimension(\cF)$ can be infinite if for any $n \in \NN$ there exists an element $F$ of $\cF$ such that $\dimension(F) = n$.
    In this case we denote $\dimension(\cF) = \infty$.
  \end{article}
  
  \begin{article}[Dimension of a diffeology]
    \label{Dimension-of-a-diffeology}
    Let $\cD$ be a diffeology.
    The \emph{dimension} of $\cD$ is defined as the infimum of the dimensions of its generating families:
    $$
      \dimension(\cD) = \inf \{ \dimension(\cF) \mid \langle \cF \rangle = \cD \}.
      $$
    Let $X$ be a diffeological space and $\cD$ be its diffeology,
    we define the dimension of $X$ as the dimension of $\cD$:
    $$
      \dimension(X) = \dimension(\cD) \in \NN \cup \{\infty\}.
      $$
  \end{article}
  
  \begin{article}[Dimensions of numerical domains]
    \label{Dimensions-of-numerical-domains}
    As we can expect,
    the diffeological dimension of a numerical domain $U \subset \RR^n$,
    equipped with the standard diffeology \art{Standard-diffeology-of-domains},
    is equal to $n$.
    That is,
    for all $U \in \Domains(\RR^n)$,
    $\dimension(U) = n$.
  \end{article}
  
  \begin{proof}
    Let us denote by $\mathcal{U}$ the domain $U$ equipped with the standard diffeology.
    Let $\id_U$ be the identity map of $U$.
    The singleton $\set{\id_U}$ is a generating family of $U$,
    therefore \art{Dimension-of-a-diffeology} $\dimension(\mathcal{U}) \leq \dimension \set{\id_U}$.
    But $\dimension\set{\id_U} = n$,
    hence $\dimension(\mathcal{U}) \leq n$.
    Now let us assume that $\dimension(\mathcal{U}) < n$.
    So,
    there exists a generating family $\cF$ for $\mathcal{U}$ such that $\dimension(\cF) < n$.
    Since the identity map $\id_U$ is a plot of $\mathcal{U}$,
    it lifts locally at every point along some element of $\cF$.
    Thus,
    for any $r \in U$ there exists a superset $V$ of $r$,
    a parametrization $F : W \to U$,
    element of $\cF$ (that is $F \in \Cinfty(W,U)$) and a smooth map $Q : V \to W$ such that $\id_U \restriction V = \id_V = F \circ Q$.
    But $\dimension(\cF) < n$ implies that $\dimension(F)= \dimension(W) <n$.
    Now,
    the rank of the linear tangent map $\D(F \circ Q)$ \art{Smooth-maps-and-tangent-linear-maps} is less or equal to $\dimension(W) <n$,
    but $\D(F \circ Q) = \D(\id_V) = \id_{\RR^n}$,
    thus $\rank(\D(F \circ Q)) = \rank(\id_{\RR^n}) = n$.
    Therefore,
    there is no generating family $\cF$ of $U$ with $\dimension(\cF) <n$,
    and $\dimension(U) = n$.
  \end{proof}
  
  \begin{article}[Dimension zero spaces are discrete]
    \label{dimension-zero-spaces-are-discrete}
    A diffeological space has dimension zero if and only if it is discrete.
  \end{article}
  
  \begin{proof}
    Let $X$ be a set equipped with the discrete diffeology.
    Any plot $P: U \to X$,
    is locally constant.
    So,
    for any $r \in U$,
    $P$ lifts locally,
    on some domain containing $r$,
    along the 0-plot $\mathbf{x} = [0 \mapsto x]$,
    where $x = P(r)$.
    Hence,
    the 0-plots form a generating family and $\dimension(X) = 0$.
    Conversely,
    let $X$ be a diffeological space such that $\dimension(X) = 0$.
    So,
    the 0-plots generate the diffeology of $X$.
    But,
    any plot lifting locally along a 0-plot is locally constant.
    Therefore,
    $X$ is discrete.
  \end{proof}
  
  \begin{article}[The dimension is a diffeological invariant]
    \label{The-dimension-is-a-diffeological-invariant}
    If two diffeological spaces are diffeomorphic,
    then they have the same dimension.
  \end{article}
  
  \begin{proof}
    Let $X$ and $X'$ be two diffeological spaces.
    Let $f : X \to X'$ be a diffeomorphism.
    Let $\cF$ be a generating family of $X$.
    The pushforward $\cF' = f_*(\cF)$ made up with the plots $f \circ F$,
    where $F \in \cF$,
    is clearly a generating family of $X'$.
    Conversely let $\cF'$ a generating family of $X'$,
    the pullback $f^*(\cF') = (f^{-1})_*(\cF')$ is a generating family of $X$.
    And these two operations are inverse each from the other.
    It is then clear that $\dimension(X) = \dimension(X')$.
  \end{proof}
  
  \begin{article}[Has the set $\{0,1\}$ dimension 1?]
    \label{has-the-set-0-1-dimension-1}
    Let's consider the set $\{0,1\}$.
    Let $\pi: \RR \to \{0,1\}$ be the parametrization defined by:
    $$
      \pi(x) = 0 \mbox{ if }x \in \QQ, \mbox{ and } \pi(x) = 1 \mbox{ otherwise.}
      $$
    Let $\{0,1\}_{\pi}$ be the set $\{0,1\}$ equipped with the diffeology generated by $\pi$ \art{Generating-families}.
    Since $\{\pi\}$ is a generating family,
    the dimension of $\{0,1\}_\pi$ is less than or equal to $1 = \dimension(\{\pi\})$.
    But,
    since the plot $\pi$ is not locally constant,
    by density of the rational (or irrational) numbers in $\RR$,
    the space $\{0,1\}_\pi$ is not discrete.
    Hence,
    $\dimension\{0,1\}_\pi\neq 0$ \art{dimension-zero-spaces-are-discrete},
    and finally $\dimension \{0,1\}_\pi = 1$.
    So,
    a finite set may have a non zero dimension,
    this example illustrates well the meaning of the dimension of a diffeological space.
    This can be compared with the topological situation for which a space consisting in a finite number of points can be equipped with the discrete or the coarse topology,
    for example.
  \end{article}
  
  \begin{article}[Example: Dimension of tori]
    \label{Dimension-of-tori}
    Let $\Gamma \subset \RR$ be any strict subgroup of $(\RR,+)$ and let $T_\Gamma$ be the quotient $\RR/\Gamma$,
    whose diffeology is generated by the projection $\pi : \RR \to \RR/\Gamma$.
    So
    $$
      \dimension(T_\Gamma) = 1.
      $$
    This applies in particular to the circles $\RR/a \ZZ$,
    with length $a >0$,
    or to \emph{irrational tori} when $\Gamma$ is generated by more than one generators,
    rationally independent.
  \end{article}
  
  \begin{proof}
    Since $\RR$ is a numerical domain,
    $\pi$ is a plot of the quotient,
    and $\cF = \set{\pi}$ is a generating family of $\RR/\Gamma$,
    and $\dimension(\cF) = 1$.
    Thus,
    as a direct consequence of the definition \art{Dimension-of-a-diffeology},
    $\dimension(\RR/\Gamma) \leq 1$.
    Now,
    if $\dimension(\RR/\Gamma) = 0$,
    then the diffeology of the quotient is generated by the constant parametrizations \art{dimension-zero-spaces-are-discrete}.
    Now,
    $\pi$ is not locally constant,
    therefore $\dimension(\RR/\Gamma) = 1$.
  \end{proof}
  
  %%%%%%%%%%%%%%%%%%%%%%%%%%%%%%%%%%%%%%%%%%%%%%%%%%%%%%%%%%
  %%
  %% MARK: The dimension map of a diffeological space
  %%
  %%%%%%%%%%%%%%%%%%%%%%%%%%%%%%%%%%%%%%%%%%%%%%%%%%%%%%%%%%
  \section{The dimension map of a diffeological space}
  \label{The-dimension-map-of-a-diffeological-space}
  
  Because diffeological spaces are not necessarily homogeneous,
  the global dimension of a diffeological space \art{Dimension-of-a-diffeology} is a too rough invariant.
  It is necessary to refine this definition and to introduce the dimension of a diffeological space at each of its points.
  
  \begin{article}[Pointed plots and germ of a diffeological space]
    \label{Pointed-plots-of-a-diffeological-space}
    Let $X$ be a diffeological space,
    let $x \in X$.
    Let $P : U \to X$ be a plot.
    We say that $P$ is \emph{pointed at $x$} if $0 \in U$ and $P(0) = x$.
    We will agree that the set of \emph{germs} of the pointed plots of $X$ at $x$ represents the \emph{germ of the diffeology} at this point,
    and we shall denote it by $\cD_x$.
  \end{article}
  
  \begin{article}[Local generating families]
    \label{Local-generating-families}
    Let $X$ be a diffeological space and let $x$ be any point of $X$.
    We shall call \emph{local generating family at $x$} any family $\cF$ of plots of $X$ such that:
    \begin{enumerate}
      \item Every element $P$ of $\cF$ is pointed at $x$,
      that is $0 \in \dom(P)$ and $P(0) =x$.
      \item For any plot $P : U \to X$ pointed at $x$,
      there exists a superset $V$ of $0 \in U$,
      a parametrization $F : W \to X$ belonging to $\cF$ and a smooth parametrization $Q : V \to W$ pointed at $0\in W$ such that $F \circ Q = P \restriction V$.
    \end{enumerate}
    We shall say also that $\cF$ \emph{generates the germ $\cD_x$ of the diffeology $\cD$ of $X$ at the point $x$} \art{Pointed-plots-of-a-diffeological-space}.
    And we denote,
    by analogy with \art{Generating-families},
    $\cD_x = \langle \cF \rangle$.
    
    Note that,
    for any $x$ in $X$,
    the set of local generating families at $x$ is not empty,
    since it contains at the set of all the plots pointed at $x$,
    and this set contains the constant parametrizations with value $x$ \art{Diffeology-and-diffeological-spaces}.
  \end{article}
  
  \begin{article}[Union of local generating families]
    \label{Union-of-local-generating_families}
    Let $X$ be a diffeological space.
    Let us choose,
    for every $x \in {X}$,
    a local generating family $\cF_x$ at $x$.
    The union $\cF$ of all these local generating families,
    $$
      \cF = \bigcup_{x \in {X}}\cF_x,
      $$
    is a generating family of the diffeology of $X$.
  \end{article}
  
  \begin{proof}
    Let $P: U \to X$ be a plot,
    let $r \in U$ and $x = P(r)$.
    Let $T_r$ be the translation $T_r(r') = r' + r$.
    Let $P'=P \circ T_r$ defined on $U' = T_r^{-1}(U)$.
    Since the translations are smooth,
    the parametrization $P'$ is a plot of $X$.
    Moreover $P'$ is pointed at $x$,
    $P'(0) = P \circ T_r (0) = P(r) = x$.
    By definition of a local generating family \art{Local-generating-families},
    there exists an element $F: W \to X$ of $\cF_x$,
    a superset $V'$ of $0 \in U'$ and a smooth parametrization $Q' : V' \to W$,
    pointed at $0$,
    such that $P' \restriction V' = F \circ Q'$.
    Thus,
    $P \circ T_r \restriction V' = F \circ Q'$,
    that is $P \restriction V = F \circ Q$,
    where $V = T_r(V')$ and $Q = Q' \circ T_r^{-1}$.
    Hence,
    $P$ lifts locally,
    at every point of its domain,
    along an element of $\cF$.
    Therefore,
    $\cF$ is a generating family \art{Generating-families} of the diffeology of $X$.
  \end{proof}
  
  \begin{article}[The dimension map]
    \label{The-dimension-map}
    Let $X$ be a diffeological space and let $x$ be a point of $X$.
    By analogy with the global dimension of $X$ \art{Dimension-of-a-diffeology},
    we define the \emph{dimension of $X$ at the point $x$} by:
    $$
      \dimension_x(X) = \inf \{ \dimension(\cF) \mid \langle \cF \rangle = \cD_x \}.
      $$
    The map $x \mapsto \dimension_x(X)$,
    with values in $\NN \cup \{\infty\}$,
    will be called the \emph{dimension map} of the space $X$.
  \end{article}
  
  \begin{article}[Global dimension and dimension map]
    \label{Global-and-pointwise-dimensions-of-a-diffeology}
    Let $X$ be a diffeological space.
    The global dimension of $X$ is the supremum of the dimension map of $X$:
    $$
      \dimension(X) = \sup_{x \in X} \set{\dimension_x(X)}.
      $$
  \end{article}
  
  \begin{proof}
    Let $\cD$ be the diffeology of $X$.
    Let us prove first that for every $x \in X$,
    $\dimension_x(X) \leq \dimension(X)$,
    which implies that $\sup_{x \in X} \dimension_x(X) \leq \dimension(X)$.
    For that we shall prove that for any $x \in X$ and any generating family $\cF$ of $\cD$,
    $\dimension_x(X) \leq \dimension(\cF)$.
    Then,
    since $\dimension(X) = \inf \{ \dimension(\cF) \mid \cF \in \cD \mbox{ and } \langle \cF \rangle = \cD \}$ we shall get,
    $\dimension_x(X) \leq \dimension(X)$.
    Now,
    let $\cF$ be a generating family of $\cD$.
    For any plot $P : U \to X$ pointed at $x$ let us choose an element $F$ of $\cF$ such that there exists a superset $V$ of $0 \in U$ and a smooth parametrization $Q : V \to {\rm def}(F)$ such that $F \circ Q = P \restriction V$.
    Then,
    let $r = Q(0)$ and $T_r$ be the translation $T_r(r') = r' + r$.
    Let $F' = F \circ T_r$,
    defined on $T_r^{-1}({\rm def}(F))$.
    So $F'(0) = x$,
    and $F'$ is a plot of $X$,
    pointed at $x$,
    such that $\dimension(F') = \dimension(F)$.
    Let $Q' = T_r^{-1} \circ Q$,
    so $Q'$ is smooth and $P \restriction V = F' \circ Q'$.
    Thus,
    the set $\cF'_x$ of all these plots $F'$ associated with the plots pointed at $x$ is a generating family of $\cD_x$,
    and for each of them $\dimension(F') = \dimension(F) \leq \dimension(\cF)$.
    Therefore $\dimension(\cF'_x) \leq \dimension(\cF)$.
    But,
    $\dimension_x(X) \leq \dimension(\cF'_x)$,
    hence $\dimension_x(X) \leq \dimension(\cF)$.
    And we conclude that $\dimension_x(X) \leq \dimension(X)$,
    for any $x \in X$,
    and $\sup_{x \in X} \dimension_x(X) \leq \dimension(X)$.
    
    Now,
    let us prove that $\dimension(X) \leq \sup_{x \in X} \dimension_x(X)$.
    Let us assume that $\sup_{x \in X} \dimension_x(X)$ is finite.
    Otherwise,
    according to the previous part we have $\sup_{x \in X} \dimension_x(X) \leq \dimension(X)$,
    and then $\dimension(X)$ is infinite and $\sup_{x \in X} \dimension_x(X) = \dimension(X)$.
    Now,
    for any $x \in X$,
    $\dimension_x(X)$ is finite.
    And since the sequence of the dimensions of the generating families of $\cD_x$ is lower bounded,
    there exists for any $x$ a generating family $\cF_x$ such that $\dimension_x(X) = \dimension(\cF_x)$.
    For every $x$ in $X$ let us choose one of them.
    Now,
    let us define $\cF_m$ as the union of all these chosen families.
    Thanks to \art{Union-of-local-generating_families},
    $\cF_m$ is a generating family of $\cD$.
    Hence,
    $\dimension(X) \leq \dimension(\cF_m)$.
    But $\dimension(\cF_m) = \sup_{F \in \cF_m} \dimension(F) = \sup_{x \in X} \sup_{F \in \cF_x} \dimension(F) = \sup_{x \in X} \dimension(\cF_x) = \sup_{x \in X} \dimension_x(X)$.
    Therefore $\dimension(X) \leq \sup_{x \in X} \dimension_x(X)$.
    And we can conclude,
    from the two parts above,
    that $\dimension(X) = \sup_{x \in X} \dimension_x(X)$.
  \end{proof}
  
  \begin{article}[Local differentiable maps and diffeomorphisms]
    \label{Local-differentiable-maps-and-local-diffeomorphisms}
    Let $X$ and $X'$ be two diffeological spaces.
    Let $f$ be a map from a subset $A$ of $X$,
    to $X'$.
    The map $f$ is said to be a \emph{local differentiable map} if for any plot $P$ of $X$,
    $f \circ P$ is a plot of $X'$.
    That is,
    $P^{-1}(A)$ is a domain,
    and either $P^{-1}(A)$ is empty or $f \circ P \restriction P^{-1}(A)$ is a plot of $X'$.
    
    $\clubsuit$ The composite of two local differentiable maps is a local differentiable map.
    
    The map $f$ is said to be a \emph{local diffeomorphism} if $f$ is injective,
    locally differentiable as well as its inverse $f^{-1}$,
    defined on $f(A)$.
    
    Let $x$ be a point of $X$ and $x'$ be a point of $X'$.
    A \emph{local diffeomorphism} from $x$ to $x'$ is a local diffeomorphism $f$ defined on some superset $A$ of $x$ such that $f(x) = x'$.
    We say that $X$,
    at the point $x$,
    is \emph{locally equivalent} to $X'$ at the point $x'$.
    
    $\spadesuit$ For $X = X'$,
    the local diffeomorphisms split the space $X$ into classes according to the relation:
    $x \sim x'$ if and only if there exists a local diffeomorphism $f$ mapping $x$ to $x'$.
    These classes are called the \emph{orbits of the local diffeomorphisms} of $X$.
  \end{article}
  
  \begin{proof}
    Let us prove the assertion $\clubsuit$.
    Let $X$,
    $X'$ and $X''$ be three diffeological spaces.
    Let $f$ be a locally differentiable map defined on $A$,
    and $f'$ be a locally differentiable map defined on $A'$.
    Let $f'' = f' \circ f$ defined on $A'' = f^{-1}(A')$.
    Assume $A''$ not empty.
    Let show that $f''$ is locally differentiable.
    Let $P : U \to X$ be a plot of $X$.
    The map $f$ is a locally differentiable,
    so $P^{-1}(A)$ is a domain and $f \circ P \restriction P^{-1}(A)$ is a plot of $X'$.
    Since $f'$ is locally differentiable,
    and $f \circ P \restriction P^{-1}(A)$ is a plot of $X'$,
    so $[f \circ P \restriction P^{-1}(A)]^{-1}(A') = (f \circ P)^{-1}(A')$ is a domain and $ f' \circ [f \circ P \restriction P^{-1}(A)] \restriction [(f \circ P)^{-1}(A')] = f' \circ f \circ P \restriction (f \circ P)^{-1}(A')$ is a plot of $X''$.
    But $(f \circ P)^{-1}(A') = P^{-1}(f^{-1}(A')) = P^{-1}(A'')$,
    thus $P^{-1}(A'')$ is a domain.
    And $f' \circ f \circ P \restriction (f \circ P)^{-1}(A') = f'' \circ P \restriction P^{-1}(A'')$,
    thus $f'' \circ P \restriction P^{-1}(A'')$ is a plot of $X''$.
    Therefore $f''$ is locally differentiable.
    
    Now,
    let us prove the assertion $\spadesuit$.
    To be related by a local diffeomorphism is clearly reflexive:
    the identity is a local diffeomorphism,
    and symmetric:
    if $f$ is a local diffeomorphism $f^{-1}$ is also a local diffeomorphism.
    We have just to check its transitivity.
    But this is a consequence of the transitivity of the local differentiability proved above and the transitivity of injectivity.
  \end{proof}
  
  \begin{article}[The dimension map is a local invariant]
    \label{The-dimension-map-is-a-finer-invariant}
    Let $X$ and $X'$ be two diffeological spaces.
    If $x \in X$ and $x' \in X'$ are two points related by a local diffeomorphism \art{Local-differentiable-maps-and-local-diffeomorphisms},
    then $\dimension_x(X) = \dimension_{x'}(X')$.
    In other words,
    local diffeomorphisms (a fortiori global) can exchange only points where the spaces have the same dimension.
    In particular,
    for $X = X'$,
    the dimension map is invariant under the local diffeomorphisms of $X$,
    that is constant on the orbits of local diffeomorphisms \art{Local-differentiable-maps-and-local-diffeomorphisms}.
  \end{article}
  
  \begin{article}[Transitive and locally transitive spaces]
    \label{Transitive-and-locally-transitive-spaces}
    A diffeological space $X$ is said to be \emph{transitive} if for any two points $x$ and $x'$ there exists a diffeomorphism $F$ \art{Differentiable-maps-and-diffeomorphisms} such that $F(x) = x'$.
    The space is said to be \emph{locally transitive} if for any two points $x$ and $x'$ there exists a local diffeomorphism $f$ \art{Local-differentiable-maps-and-local-diffeomorphisms} defined on a superset of $x$ such that $f(x) = x'$.
    If the space $X$ is transitive it is a fortiori locally transitive.
    
    As a direct consequence of \art{The-dimension-map-is-a-finer-invariant} :
    if a diffeological space $X$ is locally transitive the dimension map \art{The-dimension-map} is constant,
    a fortiori if the space $X$ is transitive.
  \end{article}
  
  %%%%%%%%%%%%%%%%%%%%%%%%%%%%%%%%%%%%%%%%%%%%%%%%%%%%%%%%%%
  %%
  %% MARK: Pushforwards of diffeologies, ...
  %%
  %%%%%%%%%%%%%%%%%%%%%%%%%%%%%%%%%%%%%%%%%%%%%%%%%%%%%%%%%%
  \section{Pushforwards of diffeologies, subductions and dimensions}
  
  It is a well known property that the category of diffeological spaces is stable by quotient.
  That is,
  any quotient of a diffeological space has a natural diffeology,
  called the quotient diffeology.
  We shall see that,
  as we can expect,
  the dimension of the quotient is less than or equal to the dimension of the total space.
  Let us remind first some needed diffeological constructions.
  
  \begin{article}[Pushforward of diffeologies]
    \label{Pushforward-of-diffeologies}
    Let $X$ be a diffeological space.
    Let $X'$ be a set and $f : X \to X'$ be a map.
    There exists a finest diffeology \art{Comparing-diffeologies} on $X'$ such that $f$ is differentiable.
    This diffeology is called the \emph{pushforward} (or \emph{image}) of the diffeology $\cD$ of $X$ and is denoted by $f_*(\cD)$.
    A parametrization $P : U \to X$ is an element of $f_*(\cD)$ if and only if for any $r \in U$ there exists a superset $V$ of $r$ such that either $P \restriction V$ is a constant parametrization or there exists a plot $Q : V \to X$ such that $P \restriction V = f \circ Q$.
    In other words,
    the diffeology $f_*(\cD)$ is generated \art{Generating-families} by the plots of the form $f \circ Q$ where $Q$ is a plot of $X$.
  \end{article}
  
  \begin{article}[Subductions]\label{Subductions}
    Let $X$ and $X'$ be two diffeological spaces.
    A map $f : X \to X'$ is called a \emph{subduction} if it is a surjection and if the image of the diffeology of $X$ \art{Pushforward-of-diffeologies} coincides with the diffeology of $X'$.
    In this case,
    a parametrization $P:U \to X'$ is a plot if and only if for any $r \in U$ there exists a superset $V$ of $r$ and a plot $Q : V \to X$ such that $P \restriction V = f \circ Q$.
  \end{article}
  
  \begin{article}[Quotients of diffeological spaces]
    \label{Quotients-of-diffeological-spaces}
    Let $X$ be a diffeological space and $\sim$ be any equivalence relation on $X$.
    The quotient space $X' = X/\!\!\sim$ carries a \emph{quotient diffeology},
    image of the diffeology of $X$ by the projection $\pi : X \to X'$.
    A parametrization $P : U \to X'$ is a plot of the quotient diffeology if and only if for any $r$ of $U$ there exists a superset $V$ of $r$ and a plot $P' : V \to X$ such that $P \restriction V = \pi \circ P'$.
    
    Note that injective subductions are diffeomorphisms,
    and the composite of two subductions is again a subduction.
    This makes subductions a subcategory of the diffeology category.
  \end{article}
  
  \begin{article}[Uniqueness of quotients]
    \label{Uniqueness-of-quotients}
    Let $X$,
    $X'$ and $X''$ be three diffeological spaces.
    Let $\pi' : X \to X'$ and $\pi'' : X \to X''$ be two subductions \art{Subductions}.
    If there exists a bijection $f : X' \to X''$ such that $f \circ \pi' = \pi''$,
    then $f$ is a diffeomorphism.
    In other words the quotient diffeological structure is unique.
  \end{article}
  
  \begin{proof}
    Let $P :U \to X'$ be a plot and $r \in U$.
    Since $\pi'$ is a subduction there exits a superset $V$ of $r$,
    a plot $Q : V \to X$ such that $\pi' \circ Q = P \restriction V$.
    Thus,
    $f \circ P \restriction V = f\circ \pi' \circ Q = \pi'' \circ Q$.
    But,
    $\pi''$ is differentiable,
    so $\pi'' \circ Q$ is a plot of $X''$.
    Hence,
    $f \circ P$ is locally a plot of $X''$ at every point of $U$,
    therefore it is a plot \artplus{Diffeology-and-diffeological-spaces}{D2}.
    Thus,
    the map $f$ is differentiable.
    Now,
    since $f^{-1} \circ \pi'' = \pi'$,
    the same holds for $f^{-1}$ and $f$ is a diffeomorphism.
  \end{proof}
  
  \begin{article}[Dimensions and quotients of diffeologies]
    \label{Dimensions-of-quotients}
    Let $X$ and $X'$ be two diffeological spaces,
    let $\pi: {X} \to {X}'$ be a subduction.
    So we have $\dimension(X') \leq \dimension(X)$.
    In other words,
    for any equivalence relation $\sim$ defined on ${X}$,
    $\dimension(X/\!\!\sim) \leq \dimension(X)$.
  \end{article}
  
  \begin{proof}
    Let $\cD$ and $\cD'$ be the diffeologies of $X$ and $X'$.
    Let $\Gen(\cD)$ and $\Gen(\cD')$ be the set of all the generating families of $\cD$ and $\cD'$.
    
    Let us prove first that for any $\cF \in \Gen(\cD)$,
    $\pi \circ \cF$ is a generating family of $\cD'$,
    that is $\pi \circ \Gen(\cD) \subset \Gen(\cD')$.
    Let $P: U \to X'$ be a plot and $r \in U$.
    Since $\pi$ is a subduction,
    there exists a superset $V$ of $r$ and a plot $Q: V \to X$ such that $\pi \circ Q = P \restriction V$.
    Since $\cF$ generates $\cD$,
    there exists a superset $W \subset V$,
    a plot $F \in \cF$ and a smooth parametrization $\phi: W \to \dom(F)$,
    such that $F \circ \phi = Q \restriction W$.
    Then,
    $P \restriction W = (\pi \circ F) \circ \phi$.
    Hence,
    any plot $P$ of $X'$ lifts locally along some element of $\pi \circ \cF$.
    Thus,
    $\pi \circ \cF$ is a generating family of $X'$.
    
    Now,
    for any plot $P: U \to X$,
    $\dimension(P) = \dimension(U) = \dimension \pi \circ P$.
    Hence,
    for any generating family $\cF \in \Gen(\cD)$,
    $\dimension(\cF) = \dimension(\pi \circ \cF)$.
    It follows:
    \begin{align*}
      \dimension(X') & = \inf_{\cF' \in \Gen(\cD') }\dimension(\cF') && \text{(by definition)} \\
      & \leq \inf_{\cF \in \Gen(\cD) }\dimension(\pi \circ \cF) && \text{(since } \pi \circ \Gen(\cD) \subset \Gen(\cD')) \\
      & \leq \inf_{\cF \in \Gen(\cD) }\dimension(\cF) && \text{(since } \dimension(\pi \circ \cF) = \dimension(\cF)) \\
      & \leq \dimension(X) && \text{(by definition)}
    \end{align*}
    Therefore $\dimension(X')\leq \dimension(X)$.
  \end{proof}
  
  \begin{article}[Pointed subduction and dimension]
    \label{Pointed-subduction-and-dimension}
    Let $X$ and $X'$ be two diffeological spaces,
    and $\pi : X \to X'$ be a subduction.
    Let $x$ be a point of $X$ and $x' = \pi(x)$.
    We shall say that $\pi$ is a \emph{subduction at the point $x$} if the the germ of the diffeology of $X'$ at the point $x'$ is the pushforward of the germ of the diffeology of $X$ at the point $x$.
    That is,
    If for any plot $P' : U' \to X'$ pointed at $x'$,
    there exists a superset $U$ of $0 \in U'$ a plot $P : U \to X$,
    pointed at $x$,
    such that $\pi \circ P = P' \restriction U$.
    So,
    we have $\dimension_{x'}(X') \leq \dimension_x(X)$.
  \end{article}
  
  \begin{proof}
    The same arguments developed in \art{Dimensions-of-quotients} apply here.
  \end{proof}
  
  \begin{article}[Example: The dimension of $\RR^n/ \mathrm{O}(n,\RR)$]
    \label{Dimension-of-Rn/On}
    Let $\Delta_n = \RR^n/\mathrm{O}(n,\RR)$,
    $n \in \NN$,
    equipped with the quotient diffeology \art{Quotients-of-diffeological-spaces}.
    Then,
    $$
      \dimension_0(\Delta_n) = n, \mbox{ and } \dimension_x(\Delta_n) = 1 \mbox{ if } x \neq 0.
      $$
    Hence,
    the global dimension of the quotient is $\dimension(\Delta_n) = n$.
  \end{article}
  
  \begin{proof}
    1) Let us denote by $\pi_n : \RR^n \to \Delta_n$ the projection from $\RR^n$ onto its quotient.
    Since,
    by the very definition of $\mathrm{O}(n, \RR)$,
    $\lVert x'\rVert = \lVert x\rVert$ if and only if $x' = A x$,
    with $A \in \mathrm{O}(n, \RR)$,
    there exists a bijection $f : \Delta_n \to [0, \infty[$ such that $f \circ \pi_n = \nu_n$,
    where $\nu_n(x) = \lVert x\rVert^2$.
    $$
      \begin{tikzcd}[column sep=large, row sep=large,every label/.append style = {font = \small}]
      \RR^n \arrow[dr,swap, "\nu_n"'] \arrow[d,swap, "\pi_n"] & \\
      \Delta_n \arrow[r,swap, "f"] & {[0, \infty[}
      \end{tikzcd}
      $$
    Now,
    thanks to the uniqueness of quotients \art{Uniqueness-of-quotients},
    $f$ is a diffeomorphism between $\Delta_n$ equipped with the quotient diffeology and $[0, \infty[$,
    equipped with the pushforward of the standard diffeology of $\RR^n$ by the map $\nu_n$.
    Now,
    let us denote by $\cD_n$ the pushforward of the standard diffeology of $\RR^n$ by $\nu_n$.
    The space $([0, \infty[, \cD_n)$ is a representation of $\Delta_n$ \art{Uniqueness-of-quotients}.
    
    2) Let us denote by $0_k$ the zero of $\RR^k$.
    Now,
    let us assume that the plot $\nu_n$ can be lifted at the point $0_n$ along a $p$-plot $P : U \to \Delta_n$,
    with $p < n$.
    Hence,
    there exists a smooth parametrization $\phi : V \to U$ such that $P \circ \phi = \nu_n \restriction V$.
    We can assume without loss of generality that $P(0_p) = 0$ and $\phi(0_n) = 0_p$.
    If it is not the case we compose $P$ with a translation mapping $\phi(0_n)$ to $0_p$.
    Now,
    since $P$ is a plot of $\Delta_n$ it can be lifted locally at the point $0_p$ along $\nu_n$.
    That is,
    there exists a smooth parametrization $\psi : W \to \RR^n$ such that $0_p \in W$ and $\nu_n \circ \psi = P \restriction W$.
    Let us denote $V' = \phi^{-1}(W)$,
    we have the following commutative diagram:
    $$
      \begin{tikzcd}[column sep=large, row sep=huge,every label/.append style = {font = \small}]
      {} & W \arrow[d,midway, "P \restriction W" description] \arrow[dr,swap, "\psi"']& {} \\
      V' \arrow[ur, "\phi \restriction V'"] \arrow[r, "\nu_n \restriction V'"'] & {[0, \infty[}  & \RR^n \arrow[l, "\nu_n"]
      \end{tikzcd}
      $$
    Now,
    denoting $F = \psi \circ \phi \restriction V'$,
    we get $\nu_n \restriction V' = \nu_n \circ F$,
    with $F \in \Cinfty(V', \RR^n)$,
    $0_n \in V'$ and $F(0_n) = 0_n$,
    that is:
    $$
      \lVert x\rVert^2 = \lVert F(x)\rVert^2.
      $$
    The derivative of this identity gives:
    $$
      x \cdot \delta x = F(x) \cdot \D(F)(x)(\delta x), \mbox{ for all } x \in V' \mbox{ and for all } \delta x \in \RR^n.
      $$
    The second derivative,
    computed at the point $0_n$,
    where $F$ vanishes,
    gives then:
    $$
      \id_n = \bar{M} M \quad\text{with}\quad M = \D(F)(0_n),
      $$
    and where $\bar{M}$ is the transposed matrix of $M$.
    But $\D(F)(0_n) = \D(\psi)(0_p) \circ \D(\phi)(0_n)$.
    Let us denote $A = \D(\psi)(0_p)$ and $B = \D(\phi)(0_n)$,
    $A \in \mathrm{L}(\RR^p, \RR^n)$ and $B \in \mathrm{L}(\RR^n, \RR^p)$.
    So $M = A B$ and the previous identity $\id_n = \bar{M} M$ becomes $\id_n = \bar{B} \bar{A} A B$.
    But,
    the rank of $B$ is less than or equal to $p$ which is,
    by hypothesis,
    strictly less than $n$ which would imply that the rank of $\id_n$ is strictly less than $n$.
    And this is not true since the rank of the $\id_n$ is $n$.
    Therefore,
    the plot $\nu_n$ cannot be lifted locally at the point $0_n$ by a $p$-plot of $\Delta_n$ with $p < n$.
    
    3) The diffeology of $\Delta_n$,
    represented by $([0, \infty[, \cD_n)$,
    is generated by $\nu_n$.
    Hence,
    $\cF = \set{\nu_n}$ is a generating family of $\Delta_n$.
    Therefore,
    by definition of the dimension of diffeological spaces \art{Dimension-of-a-diffeology},
    we have $\dimension(\Delta_n) \leq n$.
    Let us assume that $\dimension(\Delta_n) = p$ with $p < n$.
    So,
    since $\nu_n$ is a plot of $\Delta_n$ it can be lifted locally,
    at the point $0_n$,
    along an element $P'$ of some generating family $\cF'$ of $\Delta_n$ satisfying $\dimension(\cF') = p$.
    But,
    by definition of the dimension of generating families \art{Dimension-of-a-family-of-parametrizations},
    we get $\dimension(P') \leq p$,
    that is $\dimension(P') < n$.
    But this is not possible,
    thanks to the second point.
    Therefore,
    $\dimension(\Delta_n) = n$.
    Now,
    since the dimension is a diffeological invariant \art{The-dimension-is-a-diffeological-invariant},
    $\Delta_n = \RR^n / \mathrm{O}(n, \RR)$ is not diffeomorphic to $\Delta_m = \RR^m / \mathrm{O}(m, \RR)$,
    $n \neq m$.
  \end{proof}
  
  %%%%%%%%%%%%%%%%%%%%%%%%%%%%%%%%%%%%%%%%%%%%%%%%%%%%%%%%%%
  %%
  %% MARK: Pullbacks of diffeologies, ...
  %%
  %%%%%%%%%%%%%%%%%%%%%%%%%%%%%%%%%%%%%%%%%%%%%%%%%%%%%%%%%%
  \section{Pullbacks of diffeologies, inductions and dimensions}
  
  The category \catDiff\ is stable by subset operation.
  This stability is expressed by the following construction.
  
  \begin{article}[Pullbacks of diffeologies]
    \label{Pullbacks-of-diffeologies}
    Let $X$ be a set,
    and $Y$ be a diffeological space.
    Let $f : X \to Y$ be a map.
    There exists a coarsest diffeology on $X$ such that the map $f$ is differentiable.
    This diffeology is called the \emph{pullback diffeology}.
    A parametrization $P : U \to X$ is a plot of the pullback diffeology if and only if $f \circ P$ is a plot of $Y$.
    Let $\cD$ be the diffeology on $Y$,
    $f^*{(\cD)}$ will denote the pullback diffeology of $\cD$ by $f$.
    $$
      f^*{(\cD)} = \{ P : U \to X \in {\rm Param}(X) \mid f \circ P \in \Cinfty(U,Y) \}
      $$
  \end{article}
  
  \begin{article}[Composition of pullbacks]
    \label{Composites-of -pullbacks}
    Let $X$,
    $Y$ be two sets and $Z$ be a diffeological space.
    Let $f : X \to Y$ and $g : Y\to Z$ be two maps.
    Let $\cD$ be a diffeology on $Z$,
    then $f^*(g^*(\cD)) = (g\circ f)^*(\cD)$.
  \end{article}
  
  \begin{article}[Inductions]\label{Inductions}
    Let $X$ and $Y$ be two diffeological spaces.
    A map $f : X \to Y$ is called an \emph{induction} if $f$ is injective and if the pullback diffeology of $Y$ by $f$ coincides with the diffeology of $X$.
    That is,
    the plots of $X$ are the parametrizations $P$ of $X$ such that $f \circ P$ are plots of $Y$.
    The composite of two inductions is again an induction.
    Inductions make up a subcategory of the category \catDiff.
  \end{article}
  
  \begin{article}[Surjective inductions]
    \label{Surjective-inductions}
    Let $f : X \to Y$ be an injection,
    where $X$ and $Y$ are diffeological spaces.
    The map $f$ is an induction if and only if for any plot $P$ of $Y$,
    with values in $f(X)$,
    the map $f^{-1}\circ P$ is a plot of $X$.
    In particular,
    surjective inductions are diffeomorphisms.
  \end{article}
  
  \begin{article}[Subset diffeology and diffeological subspaces]
    \label{Subset-diffeology-and-diffeological-subspaces}
    Let $X$ be a diffeological space.
    Any subset $A \subset X$ carries a natural diffeology induced by the inclusion.
    Namely the pullback diffeology by the inclusion $j_{A} : A \hookrightarrow X$ \art{Inductions}.
    Equipped with this \emph{induced diffeology} the subset $A$ is called a \emph{subspace} of $X$.
    This diffeology is also called the \emph{subset diffeology}.
    The plots of the subset diffeology of $A$ are the plots of $X$ taking their values in $A$.
  \end{article}
  
  \begin{article}[Sums of diffeological spaces]
    \label{Sums-of-diffeological-spaces}
    Let $\cX$ be a family of diffeological spaces,
    there exists on the disjoint union $\coprod_{X \in \cX} X$ of the elements of $\cX$ a finest diffeology such that each injection $j_X : X \hookrightarrow \coprod_{X \in \cX} X$,
    $X \in \cX$,
    is differentiable.
    This diffeology is called the \emph{sum diffeology} of the family $\cX$.
    The plots of the sum diffeology are the parametrizations $P:U\to \coprod_{X \in \cX} X$ which are locally plots of elements of the family $\cF$.
    In other words,
    a parametrization $P:U\to \coprod_{X \in \cX} X$ is a plot of the sum diffeology if and only if there exists an open covering $\{U_i\}_{i \in \cI}$ of $U$ and for each $i \in \cI$ an element $X_i$ of the family $\cX$,
    such that $P\restriction U_i$ is a plot of $X_i$.
    For this diffeology,
    the injections $j_X$ are inductions.
    
    It is obvious,
    thanks to \art{Global-and-pointwise-dimensions-of-a-diffeology},
    that the dimension of a sum is the supremum of the dimensions of its components,
    $$
      \dimension \coprod_{X \in \cX} X = \sup_{X \in \cX} \dimension(X).
      $$
  \end{article}
  
  \begin{article}[Example : The dimension of the half-line]
    \label{Dimension-of-the-half-line}
    Unlike the situation of quotients \art{Dimensions-of-quotients},
    there is no simple relation between the dimension of a subspace $A$ and the dimension of the ambient space $X$.
    The dimension of $A \subset X$ can be less than,
    equal or even greater than the dimension of $X$.
    The following example is a clear illustration of that phenomenon.
    
    Let $\Delta_\infty = [0,\infty[ \subset \RR$ equipped with the subset diffeology.
    Then,
    $$
      \dimension_0(\Delta_\infty) = \infty, \mbox{ and } \dimension_x(\Delta_\infty) = 1 \mbox{ if } x \neq 0.
      $$
    Hence,
    the global dimension of the quotient is $\dimension(\Delta_\infty) = \infty$.
  \end{article}
  
  \begin{proof}
    First of all,
    let us remark that any map $\nu_n : \RR^n \to \Delta_\infty$,
    defined by $\nu_n(x) = \lVert x\rVert^2$,
    is a plot of $\Delta_\infty$.
    Indeed,
    it is a smooth parametrization of $\RR$ and it takes its values in $[0, \infty[$.
    Now,
    let us assume that $\dimension(\Delta_\infty) = N < \infty$.
    Hence for any integer $n$,
    the plot $\nu_n$ lifts locally at the point $0_n$ along some $p$-plot of $\Delta_\infty$,
    with $p \leq N$.
    Let us choose now $n > N$.
    So,
    there exists a smooth parametrization $f : U \to \RR$ such that $\mathrm{Val}(f) \subset [0, \infty[$,
    that is $f$ is a $p$-plot of $\Delta_\infty$.
    And,
    there exists a smooth parametrization $\phi : V \to U$ such that $f \circ \phi = \nu_n \restriction V$.
    $$
      \begin{tikzcd}[column sep=large, row sep=large,every label/.append style = {font = \small}]
      & U \arrow[dr, "f"] & \\
      V \arrow[ur, "\phi"] \arrow[rr, "\nu_n \restriction V"'] & & {[0, \infty[}
      \end{tikzcd}
      $$
    We can assume,
    without loss of generality,
    that $0_p \in U$,
    $\phi(0_n) = 0_p$,
    which implies $f(0_p) = 0$.
    Now,
    let us follow the method of \art{Dimension-of-Rn/On}.
    The first derivative of $\nu_n$ at a point $x \in V' = \phi^{-1}(V)$ is given by:
    $$
      x = \D(f)(\phi(x)) \circ \D(\phi)(x).
      $$
    Since $f$ is smooth,
    positive and $f(0)=0$ we have,
    in particular,
    $\D(f)(0_p) = 0$.
    Now,
    considering this property,
    the second derivative,
    computed at the point $0_n$,
    gives in matrix notation:
    $$
      \id_n = \bar{M} H M, \mbox{ where } M = \D(\phi)(0) \mbox{ and } H = \D^2(f)(0).
      $$
    The bar represents the transposition,
    and $H$ is a symmetric bilinear map,
    the hessian of $\phi$ at the point $0_n$.
    The matrix $M$ represents the tangent map of $f$ at $0_p$.
    Now,
    since we chose $n > N$ and assumed $p \leq N$ we have $p < n$.
    Thus the map $M$ has a non zero kernel and then $\bar{M} H M$ is degenerate,
    which is impossible since $\id_n$ is non degenerate.
    In other words,
    the rank of $\bar{M} H M$ is less than or equal to $p < n$,
    and the rank of $\id_n$ is $n$.
    Therefore the dimension of $\Delta_\infty$ is infinite.
  \end{proof}
  
  \begin{article}[The diffeomorphisms of the half line]
    \label{The-diffeomorphisms-of-the-half-line}
    We can illustrate the invariance of the dimension map \art{The-dimension-map} under diffeomorphisms \art{The-dimension-map-is-a-finer-invariant} by the following characterization of the diffeomorphisms of the half line $\Delta_\infty$ \art{Dimension-of-the-half-line}.
    
    A bijection $f : \interval{0,\infty} \to \interval{0,\infty}$ is a diffeomorphism,
    for the subset diffeology induced by $\RR$,
    if and only if:
    \begin{enumerate}
      \item The origin is fixed,
      $f(0) = 0$.
      \item The restriction of $f$ to the open half line is an increasing diffeomorphism of the open half-line,
      $f \restriction \left]0,\infty\right[ \in {\rm Diff}^+(\left]0,\infty\right[)$.
      \item The map $f$ is continuously indefinitely differentiable at the origin and its first derivative $f'(0)$ does not vanish.
    \end{enumerate}
    Moreover,
    these three conditions are equivalent to the following:
    $f$ is the restriction to $\interval{0, \infty}$ of a smooth parametrization $\tilde f$,
    defined on an open superset of $\interval{0, \infty}$ and such that:
    $\tilde f(0) =0$,
    $\tilde f$ is strictly increasing on $\interval{0, \infty}$ and $\tilde f'(0) > 0$.
  \end{article}
  
  \begin{proof}
    Let us prove first that any diffeomorphism $f$ of $\Delta_\infty$ satisfies the three points of the proposition.
    
    1. Since the dimension map is invariant under diffeomorphism \art{The-dimension-map-is-a-finer-invariant} and since the origin is the only point where the dimension is infinite \art{Dimension-of-the-half-line},
    $f$ fixes the origin.
    That is $f(0) = 0$.
    
    2. Since $f(0) = 0$ and $f$ is a bijection we have $f(\left]0, \infty \right[) = \left]0, \infty \right[$.
    Now,
    since the restriction of a diffeomorphism to any subset is a diffeomorphism of this subset onto its image,
    for the subset diffeology \cite{PIZ05},
    we have $f \restriction \left]0,\infty\right[ \in {\rm Diff}(\left]0,\infty\right[)$.
    The induced diffeology of the open interval being the standard diffeology \art{Standard-diffeology-of-domains}.
    Moreover,
    since $f(0) = 0$,
    $f$ restricted to $\left]0,\infty\right[$ is necessarily increasing.
    
    3. Since $f$ is differentiable,
    by the very definition of differentiability \art{Differentiable-maps-and-diffeomorphisms},
    for any smooth parametrization $P$ of the interval $\interval{0,\infty}$ the composite $f \circ P$ is smooth.
    In particular for $P = [t \mapsto t^2]$.
    Hence,
    the map $\varphi : t \mapsto f(t^2)$ defined on $\RR$ with values in $\interval{0, \infty}$ is smooth.
    Now,
    by theorem 1 of \cite{Whi43},
    since $\varphi$ is smooth,
    $f$ can be extended to an open superset of $\interval{0, \infty}$ by a smooth function.
    Hence,
    $f$ is continuously differentiable at the origin.
    Moreover since $f$ is a diffeomorphism of $\Delta_\infty$,
    what has been said for $f$ can be say for $f^{-1}$.
    And,
    since $(f^{-1})'(0) = 1/f'(0)$ and $f$ is increasing we have $f'(0) >0$.
    
    Conversely,
    it is obvious that if $f$ is the restriction on $\interval{0, \infty}$ of a smooth function $\tilde f$ such that:
    $\tilde f(0) =0$,
    $\tilde f$ is strictly increasing on $\interval{0, \infty}$ and $\tilde f'(0) > 0$ then $f$ is differentiable for the subset diffeology as well as its inverse.
    That is,
    $f$ is a diffeomorphism of $\Delta_\infty$.
  \end{proof}
  
  %%%%%%%%%%%%%%%%%%%%%%%%%%%%%%%%%%%%%%%%%%%%%%%%%%%%%%%%%%
  %%
  %% MARK: Locality, manifolds and dimension
  %%
  %%%%%%%%%%%%%%%%%%%%%%%%%%%%%%%%%%%%%%%%%%%%%%%%%%%%%%%%%%
  \section{Locality, manifolds and dimension}
  
  We remind here the diffeological definition of \emph{manifolds}.
  We check that the definition of dimension given in \art{Dimension-of-a-diffeology} coincides with the usual definition of the differential geometry.
  
  \begin{article}[Manifolds]
    \label{Manifolds}
    Let $X$ be a diffeological space,
    and let $\cD$ be its diffeology.
    If there exists an integer $n$ such that $X$ is locally diffeomorphic to $\RR^n$ in every point $x$ of $X$,
    then we say that $X$ is a \emph{manifold} modeled on $\RR^n$.
    We say also that $\cD$ is a \emph{manifold diffeology}.
    More precisely,
    for every point $x \in X$,
    there exists a $n$-domain $U$ and a local diffeomorphism $F : U \to X$ \art{Local-differentiable-maps-and-local-diffeomorphisms} such that for some $r \in U$,
    $F(r) = x$.
    Such local diffeomorphism is called a \emph{chart} of $X$.
    A collection of charts whose domains cover $X$ is called an atlas.
  \end{article}
  
  \begin{article}[Dimension of manifolds]
    \label{Dimension-of-manifolds}
    For any manifold $X$ modeled on $\RR^n$ \art{Dimension-of-a-diffeology} \art{Manifolds},
    we have $\dimension(X) = n$.
  \end{article}
  
  \begin{proof}
    Since $X$ is locally diffeomorphic to $\RR^n$ at every point,
    the group of local diffeomorphisms of $X$ is transitive on $X$.
    That is every point is equivalent to any other one \art{Local-differentiable-maps-and-local-diffeomorphisms},
    hence the dimension map is constant \art{The-dimension-map-is-a-finer-invariant},
    and equal to the dimension of $\RR^n$,
    but $\dimension(\RR^n) = n$ \art{Dimensions-of-numerical-domains}.
  \end{proof}
  
  \begin{article}[Diffeological manifolds]
    \label{Diffeological-manifolds}
    A more diffeological version of manifold follows the definition of diffeological vector spaces.
    Let us remind \cite{PIZ05a} that a diffeological vector space over a field $\KK = \RR$ or $\CC$ is a vector space $(E,+,\cdot)$ with $E$ equipped with a diffeology such that $(E,+)$ is a diffeological group,
    and the multiplication by a scalar is differentiable from $\KK \times E$ to $E$,
    $\KK$ being equipped with the standard diffeology.
    A \emph{diffeological manifold} $M$ modeled on $E$ is a diffeological space locally diffeomorphic at every point to $E$.
    Since $E$ is transitive by translations,
    which are diffeomorphisms,
    the dimension of $M$ is constant and equal to the dimension of $E$.
    The infinite sphere and the infinite projective space defined in \cite{PIZ05a} are two examples of such diffeological manifolds,
    of infinite dimension.
  \end{article}
  
  %%%%%%%%%%%%%%%%%%%%%%%%%%%%%%%%%%%%%%%%%%%%%%%%%%%%%%%%%%
  %%
  %% MARK: Bibliography
  %%
  %%%%%%%%%%%%%%%%%%%%%%%%%%%%%%%%%%%%%%%%%%%%%%%%%%%%%%%%%%
  
  \begin{thebibliography}{XXXX}
    
    \bibitem[Che77]{Che77}
    Kuo Tsai Chen.
    \newblock \emph{Iterated path integral}.
    \newblock Bull. of Am. Math. Soc., 83(5):831--879, 1977.
    
    \bibitem[Don84]{Don84}
    Paul Donato.
    \newblock \emph{Revêtement et groupe fondamental des espaces différentiels homogènes}.
    \newblock Thèse de doctorat d'état, Université de Provence, Marseille, 1984.
    
    \bibitem[DI85]{DI85}
    Paul Donato \& Patrick Iglesias.
    \newblock \emph{Exemple de groupes difféologiques : flots irrationnels sur le tore}.
    \newblock Compte Rendu de l'Académie des Sciences, 301(4), 1985.
    
    \bibitem[Igl85]{Igl85}
    Patrick Iglesias.
    \newblock \emph{Fibrés difféologiques et homotopie}.
    \newblock Thèse de doctorat d'état, Université de Provence, Marseille, 1985.
    \newblock \url{http://www.umpa.ens-lyon.fr/~iglesias/these/}
    
    \bibitem[Igl86]{Igl86}
    Patrick Iglesias.
    \newblock \emph{Difféologie d'espace singulier et petits diviseurs}.
    \newblock Compte Rendu de l'Académie des Sciences, 302:519--522, 1986.
    
    \bibitem[Igl87]{Igl87}
    Patrick Iglesias.
    \newblock \emph{Connexions et difféologie}.
    \newblock In \emph{Aspects dynamiques et topologiques des groupes infinis de transformation de la mécanique}, volume 25, pages 61--78. Hermann, Paris, 1987.
    
    \bibitem[Igl95]{Igl95}
    Patrick Iglesias.
    \newblock \emph{La trilogie du Moment}.
    \newblock Ann. Inst. Fourier, 45, 1995.
    
    \bibitem[IL90]{IL90}
    Patrick Iglesias \& Gilles Lachaud.
    \newblock \emph{Espaces différentiables singuliers et corps de nombres algébriques}.
    \newblock Ann. Inst. Fourier, Grenoble, 40(1):723--737, 1990.
    
    \bibitem[PIZ05a]{PIZ05a}
    Patrick Iglesias-Zemmour.
    \newblock \emph{Diffeology of the Infinite Hopf Fibration}.
    \newblock eprint, 2005.
    \newblock \url{http://www.umpa.ens-lyon.fr/~iglesias/}
    
    \bibitem[PIZ05]{PIZ05}
    Patrick Iglesias-Zemmour.
    \newblock \emph{Diffeology}.
    \newblock eprint, 2005.
    \newblock \url{http://www.umpa.ens-lyon.fr/~iglesias/diffeology/}
    
    \bibitem[IKZ05]{IKZ05}
    Patrick Iglesias, Yael Karshon and Moshe Zadka.
    \newblock \emph{Orbifolds as diffeologies}.
    \newblock 2005. \url{http://arxiv.org/abs/math.DG/0501093}
    
    \bibitem[Sou81]{Sou81}
    Jean-Marie Souriau.
    \newblock \emph{Groupes différentiels}.
    \newblock In \emph{Lecture notes in mathematics}, volume 836, pages 91--128. Springer Verlag, New-York, 1981.
    
    \bibitem[Sou84]{Sou84}
    Jean-Marie Souriau.
    \newblock \emph{Groupes différentiels et physique mathématique}.
    \newblock In \emph{Collection travaux en cours}, pages 75--79. Hermann, Paris, 1984.
    
    \bibitem[Whi43]{Whi43}
    Hassler Whitney.
    \newblock \emph{Differentiable even functions}.
    \newblock Duke Math. J., 10:159--160, 1943.
    
  \end{thebibliography}
  
  \bigskip
  \flushright{Patrick Iglesias-Zemmour \\ Einstein Institute \\
  The Hebrew University of Jerusalem \\
  Campus Givat Ram \\
  Jerusalem 91904 \\
  ISRAEL }
  \flushright{\tt http://math.huji.ac.il/$\sim$piz/
  \\ \tt piz@math.huji.ac.il}
  
\end{document}
