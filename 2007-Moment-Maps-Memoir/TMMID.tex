\documentclass[11pt]{amsart}

%%%%%%%%%%%%%%%%%%%%%%%%%%%%%%%%%%%%%%%%%%%%%%%%%%%%%%%%%%
%% Packages
%%%%%%%%%%%%%%%%%%%%%%%%%%%%%%%%%%%%%%%%%%%%%%%%%%%%%%%%%%
\usepackage[french,english]{babel} % Added english for amsart compatibility, french is default
\usepackage{amssymb} % For \amssquare and other symbols
\usepackage{eucal}   % For Euler script font, used by \euls
\usepackage{graphicx}
\usepackage{tikz-cd} % For modern commutative diagrams

%%%%%%%%%%%%%%%%%%%%%%%%%%%%%%%%%%%%%%%%%%%%%%%%%%%%%%%%%%
%% Page Layout (preserved from original)
%%%%%%%%%%%%%%%%%%%%%%%%%%%%%%%%%%%%%%%%%%%%%%%%%%%%%%%%%%
\oddsidemargin 10mm
\evensidemargin 10mm
\topmargin 5mm
\headheight 15mm
\headsep 10mm
\textwidth 137mm
\textheight 178mm
\footskip 10mm

\parindent 0mm
\parskip 0.5em plus 1pt

%%%%%%%%%%%%%%%%%%%%%%%%%%%%%%%%%%%%%%%%%%%%%%%%%%%%%%%%%%
%% Theorem Environments (modern amsthm setup)
%%%%%%%%%%%%%%%%%%%%%%%%%%%%%%%%%%%%%%%%%%%%%%%%%%%%%%%%%%
\theoremstyle{definition} % Sets theorem body to Roman font, matching original style
\newtheorem{article}{Article}[section]

%\renewenvironment{proof}\n
%{\noindent{\sc Proof --}}\n
%{\nolinebreak \hfill $\square$}\n

%%%%%%%%%%%%%%%%%%%%%%%%%%%%%%%%%%%%%%%%%%%%%%%%%%%%%%%%%%
% Custom Macro definitions (preserved from original)
%%%%%%%%%%%%%%%%%%%%%%%%%%%%%%%%%%%%%%%%%%%%%%%%%%%%%%%%%%

% Implement \eulr and \euls for compatibility with old macros
\newcommand{\eulr}{\mathrm}
\newcommand{\euls}{\mathcal}

% Redefine old guillemet macros to use modern babel commands
\def\<<{\og}
\def\>>{\fg}

\newcommand{\delete}[1]{}
\renewcommand{\over}{\above .2pt}
\def\qmbox#1{\quad\mbox{#1}\quad}

\def\to{\rightarrow}

\newcommand{\Param}{\eulr{Param}}
\newcommand{\dom}{\eulr{dom}}
\newcommand{\id}{{\bf 1}}
\newcommand{\pr}{{\eulr{pr}}}
\newcommand{\ev}{\eulr{ev}}
\newcommand{\ZDR}[1]{{\rm Z}_{\rm dR}^{\rm#1}}
\newcommand{\BDR}[1]{{\rm B}_{\rm dR}^{\rm#1}}
\newcommand{\HDR}{{\rm H}_{\rm dR}}
\newcommand{\Diff}{\eulr{Diff}}
\newcommand{\Hom}{\eulr{Hom}}
\newcommand{\Imm}{\eulr{Imm}}
\newcommand{\Stab}{\eulr{St}}
\newcommand{\Ab}{\eulr{Ab}}
\newcommand{\Ad}{\eulr{Ad}}
\newcommand{\norm}[1]{\Vert #1 \Vert}
\newcommand{\Cinfty}{{\eulr{C}}^\infty}
\newcommand{\modulus}[1]{\vert #1 \vert}
\newcommand{\Paths}{\eulr{Paths}}
\newcommand{\Loops}{\eulr{Loops}}
\newcommand{\comp}{\eulr{comp}}
\newcommand{\class}{\eulr{class}}
\newcommand{\bounds}{\eulr{ends}}
\newcommand{\eB}{\eulr{B}}
\newcommand{\eF}{\eulr{F}}
\newcommand{\eH}{\eulr{H}}
\newcommand{\eK}{\eulr{K}}
\newcommand{\eL}{\eulr{L}}
\newcommand{\eR}{\eulr{R}}
\newcommand{\ef}{\eulr{f}}
\newcommand{\eh}{\eulr{h}}
\newcommand{\ek}{\eulr{k}}
\newcommand{\eo}{\eulr{o}}
\newcommand{\et}{\eulr{t}}
\newcommand{\ew}{\eulr{w}}
\newcommand{\eZ}{\eulr{Z}}
\newcommand{\flip}{\eulr{flip}}
\newcommand{\Taut}{\eulr{Taut}}
\newcommand{\Liouv}{\eulr{Liouv}}
\newcommand{\Value}{\eulr{value}}
\newcommand{\Values}{\eulr{val}}
\newcommand{\Surf}{\eulr{Surf}}
\newcommand{\undemi}{{\raisebox{-2.5pt}{1} \over \raisebox{1pt}{2}}}
\newcommand{\unsurdeuxi}{{\raisebox{-2.5pt}{$1$} \over \raisebox{1pt}{$2i$}}}
\newcommand{\const}{\eulr{const}}
\newcommand{\but}{\hat{1}}
\newcommand{\source}{\hat{0}}
\newcommand{\DLie}{\pounds}
\newcommand{\GL}{{\rm GL}}
\newcommand{\Str}{\eulr{Str}}
\newcommand{\Ham}{\eulr{Ham}}
\newcommand{\grad}{\mathop{\rm grad}\nolimits}
\newcommand{\Gomega}{\G_\omega}
\newcommand{\idGomega}{\G_\omega^\circ}
\newcommand{\tidGomega}{\widetilde \G_\omega^\circ}
\newcommand{\speed}{\eulr{sp}}
\newcommand{\accl}{\eulr{ac}}
\def\scal<#1,#2>{\langle #1 | #2 \rangle}

\input{diagram.tex}

\newcommand{\cA}{{\euls A}} \newcommand{\cB}{{\euls B}} \newcommand{\cC}{{\euls C}}
\newcommand{\cD}{{\euls D}} \newcommand{\cE}{{\euls E}} \newcommand{\cF}{{\euls F}}
\newcommand{\cG}{{\euls G}} \newcommand{\cH}{{\euls H}} \newcommand{\cI}{{\euls I}}
\newcommand{\cJ}{{\euls J}} \newcommand{\cK}{{\euls K}} \newcommand{\cL}{{\euls L}}
\newcommand{\cM}{{\euls M}} \newcommand{\cN}{{\euls N}} \newcommand{\cO}{{\euls O}}
\newcommand{\cP}{{\euls P}} \newcommand{\cQ}{{\euls Q}} \newcommand{\cR}{{\euls R}}
\newcommand{\cS}{{\euls S}} \newcommand{\cT}{{\euls T}} \newcommand{\cU}{{\euls U}}
\newcommand{\cV}{{\euls V}} \newcommand{\cW}{{\euls W}} \newcommand{\cX}{{\euls X}}
\newcommand{\cY}{{\euls Y}} \newcommand{\cZ}{{\euls Z}}

\renewcommand{\AA}{{\bf A}} \newcommand{\BB}{{\bf B}} \newcommand{\CC}{{\bf C}}
\newcommand{\DD}{{\bf D}} \newcommand{\EE}{{\bf E}} \newcommand{\FF}{{\bf F}}
\newcommand{\GG}{{\bf G}} \newcommand{\HH}{{\bf H}} \newcommand{\II}{{\bf I}}
\newcommand{\JJ}{{\bf J}} \newcommand{\KK}{{\bf K}} \newcommand{\LL}{{\bf L}}
\newcommand{\MM}{{\bf M}} \newcommand{\NN}{{\bf N}} \newcommand{\OO}{{\bf O}}
\newcommand{\PP}{{\bf P}} \newcommand{\QQ}{{\bf Q}} \newcommand{\RR}{{\bf R}}
\renewcommand{\SS}{{\bf S}} \newcommand{\TT}{{\bf T}} \newcommand{\UU}{{\bf U}}
\newcommand{\VV}{{\bf V}} \newcommand{\WW}{{\bf W}} \newcommand{\XX}{{\bf X}}
\newcommand{\YY}{{\bf Y}} \newcommand{\ZZ}{{\bf Z}}

\newcommand{\ff}{{\bf f}} \newcommand{\mm}{{\bf m}} \newcommand{\ee}{{\bf e}}
\newcommand{\bmx}{\mbox{\boldmath $x$}}

\def \A{\ifmmode{{\rm A}}\fi} \def \B{\ifmmode{{\rm B}}\fi} \def \C{\ifmmode{{\rm C}}\fi}
\def \D{\ifmmode{{\rm D}}\fi} \def \E{\ifmmode{{\rm E}}\fi} \def \F{\ifmmode{{\rm F}}\fi}
\def \G{\ifmmode{{\rm G}}\fi} \def \H{\ifmmode{{\rm H}}\fi} \def \I{\ifmmode{{\rm I}}\fi}
\def \J{\ifmmode{{\rm J}}\fi} \def \K{\ifmmode{{\rm K}}\fi} \def \L{\ifmmode{{\rm L}}\fi}
\def \M{\ifmmode{{\rm M}}\fi} \def \N{\ifmmode{{\rm N}}\fi} \def \O{\ifmmode{{\rm O}}\fi}
\def \P{\ifmmode{{\rm P}}\fi} \def \Q{\ifmmode{{\rm Q}}\fi} \def \R{\ifmmode{{\rm R}}\fi}
\def \S{\ifmmode{{\rm S}}\fi} \def \T{\ifmmode{{\rm T}}\fi} \def \U{\ifmmode{{\rm U}}\fi}
\def \V{\ifmmode{{\rm V}}\fi} \def \W{\ifmmode{{\rm W}}\fi} \def \X{\ifmmode{{\rm X}}\fi}
\def \Y{\ifmmode{{\rm Y}}\fi} \def \Z{\ifmmode{{\rm Z}}\fi}

\newcommand{\art}[1]{Subsection \ref{#1}}
\newcommand{\chap}[1]{{(chap.~\ref{#1})}}
\newcommand{\sect}[1]{{(sec.~\ref{#1})}}
\newcommand{\fig}[1]{{(fig.~\ref{#1})}}
\newcommand{\eqnum}[1]{$#1$}
\newcommand{\arteq}[2]{{(art.~\ref{#1}, #2)}}

\def \longhookrightarrow{\lhook\joinrel\longrightarrow}
\newcommand{\rfl}[1]{{\buildrel{\displaystyle #1}\over{\hbox to 10mm{\rightarrowfill}}}}
\newcommand{\lfl}[1]{{\buildrel{#1}\over{\hbox to 10mm{\leftarrowfill}}}}
\newcommand{\rdfl}[1]{\mathrel{\mathop{\null\hbox to 10mm{\rightarrowfill}}\limits_{#1}}}
\newcommand{\ufl}[1]{\llap{$\scriptstyle #1$}\left\uparrow\vbox to 5mm{}\right .}
\newcommand{\dlfl}[1]{\llap{$\scriptstyle #1\;$}\left\downarrow\vbox to 5mm{} \right .}
\newcommand{\drfl}[1]{\left\downarrow\vbox to 5mm{}\rlap{$\;\scriptstyle #1$} \right .}
\newcommand{\ulfl}[1]{\llap{$\scriptstyle #1$}\left\uparrow\vbox to 5mm{}\right .}
\newcommand{\urfl}[1]{\left\uparrow\vbox to 5mm{}\rlap{$\scriptstyle #1$}\right .}

\newcommand{\mymatrix}[1]{\begin{pmatrix}#1\end{pmatrix}}

\allowdisplaybreaks

%%%%%%%%%%%%%%%%%%%%%%%%%%%%%%%%%%%%%%%%%%%%%%%%%%%%%%%%%%
% The document
%%%%%%%%%%%%%%%%%%%%%%%%%%%%%%%%%%%%%%%%%%%%%%%%%%%%%%%%%%

\begin{document}

\title{The Moment Maps in Diffeology}
\author{Patrick Iglesias-Zemmour}

\address{Laboratory of Analysis, Topology and Probability, CNRS, 39 F. Joliot-Curie, 13453 Marseille Cedex 13, France.}
\curraddr{The Hebrew University of Jerusalem, Einstein Institute, Campus Givat Ram, 91904 Jerusalem, Israel.}
\email{piz@math.huji.ac.il}

\date{Received by the editor October 4, 2007}

\keywords{Diffeology, Moment Map, Symplectic Geometry}
\subjclass[2000]{53C99, 53D30, 53D20}

\begin{abstract}
This memoir presents a generalization of the moment maps to the category \{Diffeology\}. This construction applies to every smooth action of any diffeological group $\G$ preserving a closed 2-form $\omega$, defined on some diffeological space $\X$. In particular, that reveals a universal construction, associated to the action of the whole group of automorphisms $\Diff(\X,\omega)$. By considering directly the space of momenta of any diffeological group $\G$, that is the space $\cG^*$ of left-invariant 1-forms on $\G$, this construction avoids any reference to Lie algebra or any notion of vector fields, or does not involve any functional analysis. These constructions of the various moment maps are illustrated by many examples, some of them originals and others suggested by the mathematical literature.
\end{abstract}

\maketitle
\tableofcontents

\section*{Thanks}
I am happy to thank the Hebrew University of Jerusalem Israel for its hospitality. The friendly and studious atmosphere I found here helped me to complete this work. I am glad to thank my friends with whom I discussed the matter developed in this memoir, Jean-Marie Souriau of course, but also Paul Donato, Yael Karshon and Fran\c{c}ois Ziegler. Also I would like to thank the referee who allowed me, by its remarks and questions, to enrich a part of this memoir.

\newpage

%************************************************
%***** Abstract 
%************************************************
\newpage \ \thispagestyle{empty} 
\pagenumbering{roman}
\setcounter{page}{6}
\thispagestyle{plain}

\begin{center} {Abstract}
\end{center}

\begin{quote}This memoir presents a generalization of
the  moment maps to the category
\{Diffeo\-logy\}. This construction applies to every
smooth action of any diffeological group $\G$ preserving a closed 2-form
$\omega$, defined on some
diffeological space $\X$. In particular, that reveals a 
universal construction, associated to the action of the
whole group of automorphisms $\Diff(\X,\omega)$. By
considering directly the space of momenta of any
diffeological group $\G$, that is the space  $\cG^*$  of
left-invariant 1-forms on $\G$, this construction avoids
any reference to Lie algebra or any notion of vector
fields, or does not involve any functional
analysis.  These constructions of the various moment maps
are illustrated by many examples, some of them originals
and others suggested by the mathematical literature.
\end{quote}

\vfill
\hrule height .5pt width 1in
Received by the editor October 4, 2007.

Key words: Diffeology, Moment Map,  Symplectic Geometry.

2000 Mathematics Subject Class: 53C99, 53D30, 53D20


%************************************************
%***** Thanks 
%************************************************
\newpage \thispagestyle{empty}
\newpage 
\pagenumbering{roman}
\setcounter{page}{7}
\thispagestyle{plain}

\begin{center} {Thanks}
\end{center}

 \begin{quote} 
 I am happy to thank the Hebrew University
of Jerusalem Israel for its hospitality. The
friendly and studious atmosphere I found here 
helped me to complete this work. I am glad to thank my
friends with whom I discussed the matter developed in
this memoir, Jean-Marie Souriau of course, but also Paul
Donato, Yael Karshon and Fran\c{c}ois Ziegler. Also I
would like to thank the referee who allowed me, by its
remarks and questions, to enrich a part of this memoir.
 \end{quote}

%************************************************
\newpage 
\pagenumbering{arabic}
\setcounter{page}{1}
\pagestyle{headings}
\section{Introduction}
%\addcontentsline{toc}{part}{Introduction}

The moment map has been introduced in the 1970's in 
Souriau's work about the structure of dynamical systems
\cite{Sou70}. It is the tool by excellence for dealing
with symmetries in symplectic, or pre-symplectic geometry.
But, in recent decades, the necessity appeared to extend
the notion of symplectic formalism and moment maps,
outside the usual framework of manifolds, to include
constructions in infinite dimension --- spaces
of connections of principal bundles, spaces of functions
etc. --- or to include singular spaces --- orbifolds,
singular symplectic reduction spaces etc..

In this paper, we shall use the category \{Diffeology\} as
the framework for such a generalization. We know already
that diffeology is suitable to describe, in a unique and
satisfactory way, manifolds or infinite dimensional
spaces, as well as singular quotients. But, if diffeology
excels with covariant objects, as differential forms,
it is more subtle when it is question of contravariant
objects like vector
fields, Lie algebra\footnote{Several authors, beginning
with Souriau, proposed some generalizations of Lie algebra
in diffeology. But, it does not seem to exist a unique
good choice. Such generalizations rely actually on the
kind of problem treated.}, kernel etc.. Thus, in order to
build a good diffeological theory of the moment map, and
to avoid useless debates, we need to get freed from
everything related to contravariant geometrical objects.
 
Actually, the
notion of moment map is not really an object of the
symplectic world, but relates more generally to the
category of space equipped with closed 2-forms. The
non-degeneracy condition is secondary and can
be skipped first from the data. This has been underlined
explicitly by Souriau in his symplectic formulation of
Noether's theorem, which involves pre-symplectic
manifolds. On symplectic manifolds, Noether's theorem is
empty. So, the moment map is just an object of the world
of differential closed form,  and there is no reason a
priori that it could not be extended to diffeology which
has a very well developed framework for De Rham's
calculus.

Now, in order to generalize the
moment map in diffeology, we need  to understand its
meaning in the simplest possible case. Let $\M$
be a  manifold equipped with a closed 2-form
$\omega$. And, let $\G$ be a Lie group acting smoothly on
$\M$ and preserving $\omega$. That is, $g_\M^*(\omega) =
\omega$ for all elements $g$ of $\G$, where $g_\M$ denotes
the action of $g$ on $\M$. Let us assume that $\omega$
is exact, $\omega = d \lambda$, and
moreover that $\lambda$ is also invariant by the action
of $\G$. So, for every point $m$ of $\M$, the pullback of
$\lambda$, by the orbit map $\hat m : g \mapsto g_\M(m)$
is a left-invariant 1-form of $\G$. That is, an element
of the dual of the Lie algebra $\cG^*$. The map, $\mu :
m \mapsto \hat m^*(\lambda)$ is exactly the moment map of
the action of $\G$ on the pair $(\M,\omega)$ (at least
one of the moment maps, since they are defined up to
constants). As we can see, this construction does not
involve really the Lie algebra of $\G$ but the space 
$\cG^*$ of left-invariant 1-forms on $\G$. Since this
space is well defined in diffeology,  we have just to
replace \og manifold\\fg by \og diffeological space\fg,and
\og Lie group\\fg by \og diffeological group\fg,and
everything works the same. So, let us
change the manifold $\M$ for a diffeological
space\footnote{The space $\X$ will be assumed to be
connected, as many results need this hypothesis.} $\X$,
and let $\G$ be some diffeological group. Let us continue
to denote the space of left-invariant 1-forms on $\G$ by
$\cG^*$, even if the star does not refer a priori to some
 duality, and let
us  call it simply the {\em space of momenta} of
the group $\G$. Note that the group $\G$ continues to
act on $\cG^*$ by pullback of its adjoint action $\Ad  :
(g,k) \mapsto gkg^{-1}$, so we don't lose the notions of
coadjoint action and coadjoint orbits.

 So, if we got the
good space of momenta, which is the space where the
moment maps are assumed to take their values, the problem
remains that not every $\G$-invariant closed 2-form is
exact. And moreover, even if such form is exact, there is
no reason, for some of its primitives to be
$\G$-invariant. We shall pass over this difficulty by
introducing an intermediary, on which we can realize the
simple case described above. This intermediary is the
space $\Paths(\X)$, of all the smooth paths of $\X$,
where the group $\G$ acts naturally by composition. And
since $\Paths(\X)$ carries a natural functional
diffeology, it is legitimate to consider its differential
forms, and this is what we do. By integrating $\omega$
along the paths, we get a differential 1-form defined on
$\Paths(\X)$, and invariant by the action of $\G$. The
exact tool used here is the chain-homotopy operator $\eK$
\cite{Piz05}. The 1-form $\Lambda = \eK\omega$, defined
on $\Paths(\X)$, is a $\G$-invariant primitive of the
2-form  $\Omega = (\but^* - \source^*)(\omega)$, where
$\but$ and $\source$ map every path of $\X$ to its ends.
Thus, thanks to the construction described above, we get
a moment map $\Psi$ for the 2-form $\Omega = d\Lambda$
and the action of $\G$ on $\Paths(\X)$. But, this {\em
paths moment map\/} $\Psi$ is not the one we are waiting
for. We need to push it down on $\X$, or moreover on $\X
\times \X$. Now, if we get this way a
{\em 2-points moment map\/} $\psi$ well defined on $\X
\times \X$, it doesn't take anymore its value in $\cG^*$,
as does $\Psi$, but in the quotient $\cG^*\!/\Gamma$,
where $\Gamma$ is the image by $\Psi$ of all the loops of
$\X$. Fortunately, $\Gamma = \Psi(\Loops(\X))$ is a
subgroup of $(\cG^*,+)$ and depends on the loops only
through their free homotopy classes. In other words,
$\Gamma$ is an homomorphic image of the fundamental group
$\pi_1(\X)$ of $\X$, or more precisely of its abelianized.
Well, it is not a big deal to have the moment map taking
its values in some quotient of the space of momenta, we
can live with that. Especially if the group $\Gamma$ is
invariant under the coadjoint action of $\G$, which is
actually the case\footnote{More precisely, the elements
of $\Gamma$ are not just elements of $\cG^*$ but are
moreover closed, and therefore invariant, each of them,
by the coadjoint action of $\G$.}. But, we are not
completely done. The usual moment map is not a 2-points
function, but a 1-point function. So, we have to extract
our usual moment maps from this 2-points function $\psi$.
This is quite easy, thanks to its very definition, the
moment map $\Psi$ satisfies an additive property for
juxtaposition of paths. And, the moment map $\psi$
inherits this property as a cocycle condition: for any
three point $x$, $x'$ and $x''$ of $\X$ we have
$\psi(x,x') + \psi(x',x'') = \psi(x,x'')$. Hence, for
$\X$   connected, there exists always a map $\mu$ such
that $\psi(x,x') = \mu(x') - \mu(x)$. And, any two such
maps differ just by a constant. So, we get finally our
wanted set of {\em moment maps\/} $\mu$, defined in the
diffeological framework. The only difference, with the
simplest case described above, is that the moment maps
take their values in some quotient of the space of
momenta, instead of the space of momenta itself. But,
this is in fact already the case in the classical theory.
It doesn't appear explicitly because people focus more on
hamiltonian actions than  just on symplectic actions.
Actually, the group $\Gamma$ represents the very
obstruction, for the action of $\G$ on $(\X,\omega)$, to
be {\em hamiltonian}. We shall call $\Gamma$, the {\em
holonomy} of the action of $\G$.

Now, let us come back to some properties of the
various moment maps introduced above. The paths moment
maps $\Psi$ and its projection $\psi$ are equivariant with
respect to the action of $\G$ on $\X$ and the coadjoint
action of $\G$ on $\cG^*$, or the projection of the
coadjoint action on $\cG^*\!/\Gamma$. But this is not
anymore the case for the moments maps $\mu$. The variance
of the maps $\mu$ reveals a family of cocycles $\theta$
from $\G$ to $\cG^*\!/\Gamma$ differing just by
coboundaries, and generalizing
{\em Souriau's cocycles\/} \cite{Sou70}. This
class of cocycles $\sigma$ belongs to the cohomology
group $\H^1(\G, \cG^*\!/\Gamma)$, and will be called {\em
Souriau's class} of the action of $\G$ of $(\X,
\omega)$. Souriau's class $\sigma$ is precisely the
obstruction for the 2-points moment map $\psi$ to be
exact, that is for some moment map $\mu$ to be
equivariant. Moreover,
in parallel with the classical situation, every
Souriau's cocycle $\theta$ defines a new action of $\G$ on
$\cG^*\!/\Gamma$, which we still call the affine
coadjoint action (associated to $\theta$). And, the image
of a moment maps $\mu$ is a collection of coadjoint
orbits for this action. We call these orbits, the
$(\Gamma,\theta)$-coadjoint orbits of $\G$. Two different
cocycles give two families of orbits translated by the
same constant.

Let us
remark that the holonomy group $\Gamma$ and Souriau's
class $\sigma$ appear clearly on a different
level of meaning, the first one is responsible of the non
hamiltonian character of the action of $\G$,
and the second characterizes the lack of equivariance of
the moment maps. 

Well, until now we didn't use all the
facilities offered by the diffeological framework. Since
we do not restrict ourselves to the category of Lie groups,
nothing prevents us to consider the group of all the
{\em automorphisms} of the pair $(\X,\omega)$. That is,
the group $\Diff(\X,\omega)$ of all the diffeomorphisms of
$\X$, preserving $\omega$. This group is a natural
diffeological group, acting smoothly on $\X$. Thus,
everything built above applies to  $\Diff(\X,\omega)$,
and every other action preserving $\omega$, of any
diffeological group, pass through
$\Diff(\X,\omega)$, and through the associated object of
the theory developed here. Therefore, considering the
whole group of automorphisms of the closed 2-form
$\omega$ of $\X$, we get a natural notion of universal
moment maps $\Psi_\omega$, $\psi_\omega$ and
$\mu_\omega$, universal holonomy $\Gamma_\omega$,
universal Souriau's cocycles $\theta_\omega$, and
universal Souriau's class $\sigma_\omega$. By the way,
this universal construction suggests a simple and new
characterization, for any diffeological space $\X$
equipped with a closed 2-form $\omega$, of the group of
{\em hamiltonian diffeomorphisms} $\Ham(\X,\omega)$, as
the largest connected subgroup of $\Diff(\X,\omega)$
whose holonomy vanishes. 

It is interesting to notice that, contrary to the
original constructions \cite{Sou70} and most of its 
generalizations, the theory described above is
essentially global, more or less algebraic,  do not refer to
any differential, or partial differential, equation and do
not involve any notion of vector field or functional
analysis techniques.

I give, at the end of
the memoir, several examples involving diffeological
groups which are not Lie groups, or involving
diffeological spaces which are not manifolds. We can see
how the general theory applies to the singular 
\og symplectic irrational tori\\fg for which topology is
irrelevant. These general constructions of 
moment maps are also applied to a few examples in infinite
dimension, and an example which mixes finite and infinite
dimensions. Finally, two examples of orbifolds are
also examined. These examples show without any doubt the
ability of this theory to treat correctly, in a unique
framework, avoiding heuristic arguments, the large variety
of situations we can find in the mathematical literature
today. For infinite dimensional (heuristic) examples, see
Donaldson's paper \cite{Dnl99}. By the way, I developed on
purpose some tedious computations, even if it is boring,
just to show diffeology at work. I mean, to show that
diffeology is not just a formalism, but a working
calculus method too.

 Considering the
classical case of a closed 2-form $\omega$ defined on a
manifold $\M$, we show in particular that $\omega$ is
non degenerate if and only if the group
$\Diff(\M,\omega)$ is transitive on $\M$ and if a
universal moment maps $\mu_\omega$ is injective. In other
words, symplectic manifolds are identified, by the
universal moment maps, to some coadjoint orbits (in our
general sense) of their group of symplectomorphisms. This
idea that \og every symplectic manifold is a coadjoint
orbit\\fg is not new, it is suggested by a well known
classification theorem for symplectic homogeneous Lie
group actions \cite{Kir74} \cite{Kos70} \cite{Sou70},
and  has been  stated already in a different context
\cite{Omo86}. What is new here is that diffeology make
this statement rigorous without the use of any 
functional analysis tools. 

In conclusion, beside the point that the construction
developed in this memoir is a first step in the
elaboration of the {\em symplectic diffeology program\/},
I would  emphasize the fact that, since \{Manifolds\} is
a full and faithful subcategory of \{Diffeology\}, all
the constructions developed here apply to manifolds and
give a faithful description of the classical theory of
moment maps. As we have seen, there is no mention, and no
use, of Lie algebra or vector fields in this exposition.
This reveal the fact that these objects are also
superfluous in the traditional approach, and can be
avoided. And, I would add, they should be avoided. No
just because then, they can be extended to larger
categories, but because the use of contravariant object
hide the deep fact that the theory of moment maps is a
pure covariant theory. For example, we know that since
coadjoint orbits of Lie groups are symplectic they are
even dimensional. This is often regarded as a miracle,
since it is not necessarily the case for adjoint orbits.
But if we think that Lie algebra have little to do with
the space of momenta of a Lie group, there is no more
miracle, just different behaviors for different objects,
which is unsurprising. Moreover I would add, but this can
appear as more or less subjective, that avoiding all this
va-et-vient between Lie algebra and dual of Lie algebra,
the diffeological approach of the moment maps is much more
simpler, and even deeper, than the classical approach.
Compare for example  Souriau's cocycle constructions in
the original \og Structure des syst\`emes
dynamiques\\fg \cite{Sou70} and in  this memoir. The only
crucial property used here is connectedness, that is the
existence of enough smooth paths connecting points in
spaces.

Now, this constructions, in particular the new
diffeological symplectic framework it suggests, come
together with a lot of new questions which have not be
answered here. And I hope I'll develop some of them in
future works.

\bigskip{\sc Note} --- Diffeology is a maximal
extension of the local category of smooth real domains.
It contains by the way, fully and faithfully, the
category of manifolds. Diffeology has been introduced by
J.-M Souriau at the beginning of the 1980s \cite{Sou81},
and it is a variant of the theory of  K.-T. Chen's {\em
differentiable spaces\/} introduced few years before
\cite{Che77}. Since then, the theory has been enhanced by
some authors. The reader is assumed to be familiar with
diffeology even if we remind some basics constructions in
the first Section. For an comprehensive report on
diffeology see \cite{Piz05}. 

%************************************************
\section{Few words about diffeology}

This is a reminder of the few diffeological notions we
will use in the following. More details about these
constructions, and proofs, can be found
in \cite{Piz05}.

\begin{article}[Domains and parametrizations] 
\label{Domains-and-parametrizations}
We call {\em numerical space} any power of the real
numbers $\RR$, and we call {\em numerical domain}, or
simply {\em domain\/}, any open set of any numerical
domain. If $\U$ is a domain of $\RR^n$, we say
that $\U$ is an {\em $n$-domain}.  Let $\X$ be a set, we
call {\em parametrization in $\X$} any map defined on some
numerical domain with values in $\X$. The set of all the
parametrizations in $\X$ is denoted by $\Param(\X)$.
For any parametrization $\P : \U \to \X$, the numerical
domain $\U$ is called the {\em domain} of $\P$ and is
denoted by $\dom(\P)$. If $\U$ is an $n$-domain we say
that $\P$ is a {\em $n$-parametrization\/}.
 \end{article}

\begin{article}[Diffeology and diffeological spaces]
\label{Diffeology-and-diffeological-spaces}
Let $\X$ be a set. A {\em diffeology on $\X$} is a set $\cD$
of parametrizations in $\X$, that is ${\cD} \subset
\Param(\X)$, such that
 \begin{enumerate}
 \item[D1.] {\em Covering}\hspace{1em} Every point of $\X$
is contained in the range of some $\P\in \cD$.
  \item[D2.]
{\em Locality}\hspace{1em} If $\P \in\Param(\X)$ and if
for any $r \in \dom(\P)$ there exists a domain $\V$ such
that $r \in \V \subset \dom(\P)$ and $\P \restriction \V
\in \cD$, then $\P\in {\cD}$.
 \item[D3.] {\em Smooth
compatibility}\hspace{1em}  If $\P \in \cD$ and $\F$ is a
$\C^{\infty}$ mapping from some domain $\V$ to
$\dom(\P)$, then $\P \circ \F \in {\cD}$.
 \end{enumerate} 
Equipped with a diffeology $\cD$, $\X$ is
a {\em diffeological space}. To make it short, the
elements of the diffeology are called the
{\em plots} of the diffeological space. So, the plots of
a diffeological space are the elements of its diffeology.
Note that the definition of a diffeology does not
assume any pre-existing structure on the underlying set. 
 \end{article}

\begin{article}[Smooth maps and diffeomorphisms]
\label{Differentiable-maps-and-diffeomorphisms}
 Let $\X$ and
$\X'$ be two sets equipped with the diffeologies $\cD$ and
$\cD'$ respectively. A map $\F : \X \to \Y$ is said
to be {\em smooth} if for each $\P \in \cD$ we have $\F \circ
\P \in \cD'$. The set of smooth maps from $\X$ to $\Y$
is denoted by $\cC^{\infty}(\X,\Y)$. A bijective map $\F : \X
\to \Y$ is said to be a {\em diffeomorphism} if both $\F$ and
$\F^{-1}$ are smooth. The set of diffeomorphisms of
$\X$ is a group denoted by ${\rm Diff}(\X)$. Diffeological
spaces are the objects of the category \{Diffeology\} whose
morphisms are {\em smooth maps}, and isomorphisms are
{\em diffeomorphisms}.
 \end{article}

\begin{article}[Quotients and subspaces]
\label{Quotient-and-subspaces}
 The category \{Diffeology\} is stable by
set theoretic operations. Products, sums of
diffeological spaces are naturally diffeological spaces,
but also quotient and subsets. Let $\sim$
be any equivalence relation on a diffeological space $\X$,
let $\Q = \X/\!\!\sim$ and $\pi : \X \to \Q$ be the
projection. There exists a natural {\em quotient
diffeology} on $\Q$, for which $\pi$ is smooth, defined
by the parametrizations which can be lifted locally
along  $\pi$ by elements of $\cD$. That is, a
parametrization $\P : \U \to \Q$ is a plot if and only if
for each $r \in \U$ there exists a domain $\V$ containing
$r$ and a plot $\phi : \V \to \X$ such that $\P
\restriction \V = \pi \circ \phi$. On the other hand,
there exists on every subset $\A \subset \X$ a natural
{\em subset diffeology}, for which the inclusion is
smooth, defined by the elements of $\cD$ which take their
values in $\A$. In the first case, the map $\pi : \X \to
\Q$ is a {\em subduction}, and in the second case the
injection $j_\A : \A \to \X$ is an {\em induction}.
 \end{article}

\begin{article}[Functional diffeology] 
\label{Functional-diffeology}
Let $\X$ and $\X'$ be two diffeological spaces. There
exists on $\Cinfty(\X,\X')$ a diffeology called the
{\em functional diffeology} whose plots are
parametrizations $\P$ such that $(r,x) \mapsto \P(r)(x)$,
defined on $\dom(\P) \times \X$ to $\X'$ is smooth. This
diffeology is the {\em coarsest\/} (e.g. largest)
diffeology such that the {\em evaluation map} $(f,x)
\mapsto f(x)$, from $\Cinfty(\X,\X') \times \X$ to $\X'$,
is smooth. In particular, the set of paths
$\Cinfty(\RR,\X)$, denoted by $\Paths(\X)$, is naturally
a diffeological space, equipped with the functional
diffeology.
 \end{article}

\begin{article}[Differential forms] 
\label{Differential-forms} Let $\X$ be a diffeological
space. A {\em differential $k$-form} on $\X$, for $k
\geq 0$, is a mapping  $\alpha$ which associates to each
plot $\P$ of $\X$ a {\em smooth $k$-form} on $\dom(\P)$.
That is, if $\P$ is an $n$-plot, $\alpha(\P)$ belongs to
$\Cinfty(\dom(\P), \Lambda^k(\RR^n))$. And satisfying the
following compatibility condition: for any plot $\P$ of
$\X$ and for any smooth parametrization $\F : \V \to
\dom(\P)$, $$\alpha(\P \circ \F) = \F^*(\alpha(\P)).$$
The space $\Omega^k(\X)$ of differential $k$-forms on
$\X$ is naturally a vector space. It carries also a
natural diffeology called again {\em functional
diffeology} for which the ordinary vectorial operations
are smooth. A parametrization $r \mapsto \alpha_r$ of
$\Omega^k(\X)$, defined on a domain $\U$, is a plot for
this functional diffeology if and only if for any
$n$-plot $\P : \V \to \X$, the parametrization $(r,s)
\mapsto \alpha_r(\P)_s$, defined on $\U \times \V$ with
values in $\Lambda^k(\RR^n)$, is smooth.

Note that, if it is necessary for a differential form to
check the compatibility condition on all the plots of the
space, two differential $k$-forms coincide if and only if
they coincide on the $k$-plots. In other words, the value
of a differential $k$-form is characterized by its values
on the $k$-plots.

The {\em
exterior differential} of a $k$-form $\alpha$ is the
differential $(k+1)$-form defined by $$d\alpha(\P) =
d(\alpha(\P)).$$ Let $f : \X \to \X'$ be a smooth map
between diffeological spaces, let $\alpha'$ be a
differential $k$-form on $\X'$, the pullback
$f^*(\alpha')$ is the differential $k$-form on $\X$
defined by $f^*(\alpha')(\P) = \alpha'(f \circ \P)$. The
exterior differential and the pullback are
linear and smooth operations. 

Let $\F : \cI \to \Diff(\X)$ be a 1-plot
defined on a open interval and centered at the identity
$\id_\X$, that is $0 \in \cI$ and  $\F(0) = \id_\X$. Let
$\alpha$ be a differential $k$-form on $\X$, with $k>0$.
The {\em contraction} $i_{\F}(\alpha)$ of $\alpha$ by
$\F$ is the $(k-1)$-differential form defined by
 $$i_{\F}(\alpha)({\P})_r(v_2, \ldots, v_{k}) = 
\alpha\bigg[\mymatrix{t \cr r} \mapsto
\F(t)(\P(r))\bigg]_{0 \choose r} \mymatrix{1 &
0 & \cdots & 0 \cr 0 &v_2 & \cdots & v_{k}},
 $$
 where  $\P$ is any plot of $\X$, 
$r \in \dom(\P)$, and $v_2,\ldots,v_{k}$  are any
$k-1$ vectors of $\RR^n$, $n$ being the dimension of the
plot $\P$. 

Let us continue with the 1-plot  $\F : \cI \to \Diff(\X)$
defined on $\cI$ and centered at $\id_\X$. Let
$\alpha$ be a differential $k$-form on $\X$, with $k
\geq 0$. There exists a differential $k$-form on $\X$,
called the {\em Lie derivative} of $\alpha$ by $\F$,
defined by
 $$
 \DLie_\F(\alpha)(\P)_r = 
\left.{\partial
\alpha(\F(t) \circ \P)_r \over \partial t}\right|_{t=0}
 $$
 for every $n$-plot $\P$ and every $r \in \dom(\P)$. Note
that $\alpha(\F(t) \circ \P)$ is just
$\F(t)^*(\alpha)(\P)$, and regarded as a function of $t$
is smooth from $\cI$ to $\Lambda^k(\RR^n)$, so the
derivative with respect to $t$ makes sense. Now, the so
called classical {\em Cartan formula} extends to
diffeology and we have, for any differential $k$ form
$\alpha$, with $k>0$,
 $$
 \DLie_\F(\alpha) = d[i_\F(\alpha)] + i_\F(d\alpha).
 $$

Let us fix now some  vocabulary we shall use
in the later paragraphs. We call {\em automorphism} of a
differential $k$-form $\alpha$ on $\X$ any diffeomorphism
$\varphi$ of $\X$ which preserves $\alpha$, that is
$\varphi^*(\alpha) = \alpha$. The set of all the
automorphisms of the form $\alpha$ is a group 
denoted by $\Diff(\X,\alpha)$,
 $$
 \Diff(\X,\alpha) =\{ \varphi \in \Diff(\X) \mid \varphi^*(\alpha) = \alpha \}.
 $$
  The group $\Diff(\X,\alpha)$ will be called {\em the
group of automorphisms of $\alpha$}, and any of its
subgroups will be called  {\em  a group of automorphisms
of  $\alpha$}.
  \end{article}

\begin{article}[Chain-Homotopy operator] 
\label{Chain-Homotopy-operator} 
Let $\X$ be a diffeological space. Let $\source$ and
$\but$ be the maps defined on $\Paths(\X)$ to $\X$ by
$$\source(p) = p(0) \qmbox{and} \but(p) = p(1).$$ There
exists a smooth linear operator $\eK$, called {\em
Chain-Homotopy operator} such that, for any integer $k>0$,
 $$
 \eK : \Omega^k(\X) \to \Omega^{k-1}(\Paths(\X))
\qmbox{and} \eK \circ d + d \circ \eK = \but^* -
\source^*.
 $$
 The value of the chain-homotopy operator $\eK$ on a
differential $k$-form $\alpha$ is given by the following
  formulas. For $k = 1$, $\eK\alpha$ is a real
function
 $$
 \eK(\alpha)(p) = \int_0^1 \alpha(p)_t(1) \ dt
\qmbox{with} \alpha \in \Omega^1(\X) \qmbox{and} p \in
\Paths(\X).
 $$
 For $k>1$, let $\P : \U
\to \Paths(\X)$ be a $n$-plot, let $r \in \U$ and
let $v_2,\ldots,v_{k}$ be $k-1$ vectors of
$\RR^n$, so
 $$
(\K\alpha)(\P)_r(v_2, \ldots,v_{k}) =
\int_0^1 \alpha \left[ \mymatrix{s \cr r} \mapsto \P(r)(s
+ t) \right]_{0 \choose r}\mymatrix{1 & 0 &
\cdots & 0 \cr 0 & v_2 & \cdots & v_{k}} \
dt.
 $$
 The chain-homotopy operator satisfies a
natural equivariance relation. Let $\X'$ be another
diffeological space and $f \in \Cinfty(\X,\X')$. Let
$f_* : \Paths(\X) \to \Paths(\X')$ be the natural map $f_*
: p \mapsto f \circ p$. Let $\eK_\X$ and $\eK_{\X'}$ be
the chain-homotopy operators associated to $\X$ and
$\X'$, so
 $$
 \eK_\X \circ f^* = (f_*)^* \circ \eK_{\X'}.
 $$ 
 In particular, if $\X = \X'$ and if  $f$ preserves a
differential $k$-form $\alpha$, that is $f^*(\alpha) =
\alpha$, then $f_*$ preserves the differential
$(k-1)$-form $\eK(\alpha)$, that is $(f_*)^*(\eK\alpha) =
\eK\alpha$.
 \end{article}

%************************************************
\section{Diffeological groups and momenta}

Diffeological groups have been first introduced as
\og groupes diff\'erentiels\\fg by Souriau in \cite{Sou81}
\cite{Sou84}. They are, with respect to  diffeological
spaces, what Lie groups are to manifolds. We remind here
their  definition. Then, we propose a diffeological
equivalent of the \og dual of the Lie algebra\\fg as the
space of invariant 1-forms on the group. We don't
consider any duality with a putative diffeological Lie
algebra. This is the simpler and the more natural way to
work with coadjoint action and coadjoint orbits in
diffeology. 

\begin{article}[Diffeological groups]
\label{Diffeological-groups}
Let $\G$ be a group
equipped with a diffeology $\cD$. We say that $\G$ is a
{\em diffeological group}, or $\cD$ is a {\em group
diffeology}, if and only if the multiplication as well as
the inversion are smooth. That is,
  $$
[(g,g')\mapsto gg'] \in \Cinfty(\G\times \G,\G) \quad \mbox{and}
\quad [g\mapsto g^{-1}] \in \Cinfty(\G, \G).
 $$
Note that if $\G$ is a standard manifold, this definition
is nothing but the definition of Lie groups. Note that
any subgroup of a diffeological group, equipped with the
subset diffeology, is a diffeological group. As well, the
quotient of any diffeological group by a normal subgroup
is a diffeological group for the quotient diffeology. We
denote by $\Hom^\infty(\G,\G')$ the space of smooth
homomorphisms from $\G$ to another diffeological group
$\G'$. 

 An important example of
diffeological group is the groups of
all the diffeomorphisms of a
diffeological space $\X$, equipped with the
{\em functional diffeology of group of
diffeomorphisms}. This diffeology is the coarsest group
diffeology on  $\Diff(\X)$ such that the evaluation map
$(f,x) \mapsto f(x)$ is smooth. A parametrization $\P :
\U \to \Diff(\X)$ is a plot if and only
if the maps $(r,x) \mapsto \P(r)(x)$ and $(r,x) \mapsto
\P(r)^{-1}(x)$ are smooth.
\end{article}

\delete{
\begin{article}[Group of commutators and abelianized]
\label{Group-of-commutators-and-abelianized}
Let $\G$ be a diffeological group. For any to elements
$g$, $g'$ of $\G$, we denote by $[g,g']$ the
{\em commutator} $gg'g^{-1}g'^{-1}$. We denote by $[\G,
\G]$ the {\em group of commutators} of $\G$. That is, the
subgroup of $\G$ generated by the commutators.
 $$
 [\G,\G] = \bigg\{ g = \prod_{i = 1}^k [g_i,g'_i] \
\bigg| \  k \in \NN \mbox{ and } g_i,g'_i \in \G \bigg\}.
 $$
 The group $[\G,\G]$ is a normal subgroup of $\G$, and the
quotient $\Ab(\G)$ of $\G$ by $[\G,\G]$ is the largest
abelian quotient of $\G$. By construction, the second
arrow $j$ of the following exact sequence is an induction
and the third $\pi_\A$ is a fibration.
 $$
 \{\id_\G\} \rfl{} [\G, \G] \rfl{j} \G
\rfl{\pi_\Ab} \Ab(\G) \rfl{} \{0\}.
 $$
 Any smooth homomorphism $h$ from $\G$ to an abelian
group $\B$ factorizes through the abelianized $\A$ of
$\G$. That is, for every $h \in \Hom^\infty(\G,\B)$, there
exists $k \in \Hom^\infty(\A,\B)$ such that $h = k \circ
\pi_\Ab$. \end{article}
}

\begin{article}[Covering diffeological groups]
\label{Covering-diffeological-groups}
Let $\hat \G$ and
$\G$ be two diffeological groups. We say that a subduction
$\pr : \hat \G \to \G$ is a {\em group covering} if and
only if $\pr$ is an homomorphism and the fiber $\K =
\pr^{-1}(\id_\G)$ is discrete\footnote{Let us remind
that {\em discrete} means that the plots (here the plots
for the subset diffeology) are locally constant.}. Let
$\G$ be a connected diffeological group. Its universal
covering $\tilde \G$ has a natural structure of
diffeological group such that the subduction $\pi :
\tilde \G \to \G$ is an homomorphism. The first homotopy
group $\pi_1(\G) = \ker(\pi)$ is a discrete invariant 
subgroup of $\tilde \G$, so $\pi$ is a group covering.
Any other connected covering $\pr : \hat \G \to \G$ is
the quotient of the universal covering by a subgroup $\K$
of $\pi_1(\G)$. If the subgroup $\K$ is normal then $\pr$
is a group covering.
 \end{article}

\begin{proof}
This property has been stated originally in \cite{Sou84}
\cite{Don84}, but let us remind the general
construction given in \cite{Igl85}. Let $\X$ be a
connected diffeological space, let $x_0$ be a point of
$\X$, chosen at the base point. Let $\Paths(\X,x_0)$ be
the space of paths starting at $x_0$. First of all, the
end map $\but : p \mapsto p(1)$, defined on 
$\Paths(\X,x_0)$ is a subduction. The quotient of
$\Paths(\X,x_0)$ by the fixed ends homotopy relation is
exactly the universal covering pointed by the constant map
$\hat x_0 : t \mapsto x_0$, over the pointed
space $(\X, x_0)$. The fiber over $x_0$ is the
homotopy group $\pi_1(\X, x_0)$. Now if $\X = \G$ we
choose the identity $\id_\G$ as base point. Thus, the
multiplication of paths $(p,p') \mapsto [t \mapsto p(t)
\cdot p'(t)]$ defines on $\tilde \G$ a group
multiplication such that the projection $\pi : \tilde \G
\to \G$, defined by $\pi(\class(p)) = \but(p)$, is an
homomorphism. The kernel of this morphism is clearly the
fiber over $\id_\G$, that is $\pi_1(\G)$. Now, the kernel
of an homomorphism is always an invariant subgroup. And,
since $\pi$ is a covering, $\pi^{-1}(\id_\G)$ is
discrete. This last points are general results of the
diffeological theory of homotopy 
\cite{Igl85}.
 \end{proof}

\begin{article}[Smooth actions of a diffeological group]
\label{Smooth-actions-of-a-diffeological-group}
Let $\G$ be a diffeological group. Let $\X$ be a
diffeological space. Let the group $\Diff(\X)$, of all
the  diffeomorphisms of $\X$, be equipped with the
functional diffeology of group of diffeomorphisms.
A {\em smooth action} of $\G$ on $\X$, or simply an {\em
action} of $\G$ on $\X$, is a smooth homomorphism $\rho$
from $\G$ to $\Diff(\X)$, that is $\rho \in
\Hom^\infty(\G, \Diff(\X))$. Let us fix or remind some
vocabulary used in the following.
 \begin{enumerate}
 \item We says that the action is {\em
effective} if $\ker(\rho) = \{\id_\G\}$. 
 \item The {\em orbits} of $\G$ are the subsets
$\rho(\G)(x) = \{ \rho(g)(x) \mid g \in \G \}$, where $x
\in \X$. 
 \item We call {\em orbit
maps} of a point $x \in \X$, the smooth map $\hat x : \G
\to \X$, defined by $\hat x : g \mapsto \rho(g)(x)$. 
 \item The {\em stabilizer} $\Stab_\rho(x)$ of
a point $x \in \X$ is the
subgroup of $\G$ defined by the equation $\hat x(g)
= x$, $g \in \G$. 
 \item We say that $\X$ is {\em homogeneous} for the
action $\rho$ of $\G$, or that $\X$ is an {\em homogeneous
space} of $\G$, for $\rho$, if and only if the orbit map
$\hat x$ of some point $x \in \X$
is a subduction, thus for every point. In this case, $\hat
x$ is a principal fibration  \cite{Igl85}
with structure group the stabilizer $\Stab_\rho(x)$. That
is $\X \simeq \G /\Stab_\rho(x)$, where $g' \sim g h$ with
$h \in \Stab_\rho(x)$.
 \end{enumerate}
 Let $\alpha$ be a differential $k$-form on $\X$.  We say
that $\G$ {\em acts by automorphisms} on $(\X, \alpha)$
if $\rho$ takes it values in $\Diff(\X,\alpha)$. That is,
if $\rho(\G)$ is a group of automorphisms of the
differential form $\alpha$. 
 \end{article}

\begin{article}[Covering smooth actions]
\label{Covering-smooth-actions}
Let $\X$ be a connected diffeological space. Let $\G$ be
a connected diffeological group. Let $\rho : \G \to
\Diff(\X)$ be a smooth action of $\G$ on $\X$. Thus,
$\rho$  takes its values in the identity component
$\Diff(\X)^\circ = \comp(\id_\X) \subset \Diff(\X)$. So,
there exists a unique smooth action $\tilde \rho$ of the
universal covering $\tilde \G$ of $\G$ on the
universal covering $\tilde \X$ of $\X$,
covering $\rho$.  
 $$ 
\begin{tikzcd}
    \ \widetilde \G \arrow[r, "\tilde \rho"] \arrow[d, "\pi_\G"'] & \widetilde{\Diff(\X)^\circ} \arrow[d, "\pi_{\Diff(\X)}"] \\
    \ \G \arrow[r, "\rho"] & \Diff(\X)^\circ
\end{tikzcd}
 $$
 \end{article}

\begin{proof}
The map $\rho \circ \pi$ is smooth and $\widetilde \G$ is
simply connected. So, thanks to the monodromy theorem
\cite{Igl85}, there exists a
unique lifting $\tilde \rho$ of $\rho \circ \pi$ mappings
the identity of $\tilde \G$ to the identity of
$\widetilde{ \Diff(\X)^\circ}$. Now, this lifting is an
homomorphism because its restriction on $\ker(\pi_\G)$
and its projection $\rho$ are both homomorphisms.
 \end{proof}

\begin{article}[Left, right and adjoint actions of a group
onto itself]
\label{Left-right-and-adjoint-actions-of-a-group-onto-itself}
Let $\G$ be a diffeological group. We
denote by $\eL(g)$ and $\eR(g)$ the {\em left} and
{\em right actions} of $\G$ onto itself. $$ \mbox{For all
} g \in \G, \quad  \left\{ 
\begin{array}{l}
 {\eL(g)} : g' \mapsto gg' \\ 
 {\eR(g)} : g' \mapsto g'g.
\end{array}
\right.
$$
Note that the \og right action\\fg is in fact an
anti-action. That is, $\eR(gg') = \eR(g') \circ \eR(g)$.
The {\em adjoint action} of $\G$ onto itself is denoted by
$\Ad$, and is defined by:
 $$ \mbox{For all } g \in
\G, \quad \Ad(g):k \mapsto gkg^{-1} = \eL(g) \circ
\eR(g^{-1})(k).
 $$ 
 The maps $\eL$ and $\Ad$ are smooth homomorphisms from
$\G$ to $\Diff(\G)$, equipped with the diffeology of
group of diffeomorphisms. The map $\eR$ is a
smooth anti-homomorphism from
$\G$ to $\Diff(\G)$.
 \end{article}

\begin{article}[Momenta of a diffeological group]
\label{Momenta-of-a-diffeological-group}
We call {\em left momentum} --- or simply {\em momentum}
--- of a diffeological group $\G$, any 1-form of $\G$,
invariant by the left action of $\G$ onto itself. We
denote by $\cG^*$ the {\em space of momenta} of $\G$. The
space of momenta of a diffeological group is naturally a
diffeological vector space, equipped with the functional
diffeology. So,
 $$ \cG^*
= \{ \alpha \in \Omega^1(\G) \mid \mbox{For all } g \in
\G, \ \eL(g)^*(\alpha) = \alpha \}.
 $$
Note that, in spite of what the notation $\cG^*$ suggests,
the space of momenta of a diffeological group is
not defined by some duality. This notation is chosen here
just to remind us the connection with the dual of the Lie
algebra in the case of Lie groups. 
 \end{article}

\begin{article}[Momenta and connectedness]
\label{Momenta-and-connectedness} 
Let $\G$ be a diffeological group. Let $\G^\circ$ be the
identity component of $\G$, that is  $\G^\circ =
\comp(\id_\G) \subset \G$. So, the pullback
$j^* : \cG^* \to
{\cG^\circ}^{\raisebox{-3.2pt}{\scriptsize *}}$ of the
injection $j : \G^\circ \to \G$ is an isomorphism. This
property is quite natural but needed to be checked up in
our context of diffeological groups. 

{\sc Note} --- Said differently, the space of momenta of a
connected diffeological group, or any of its extensions 
by a discrete group, coincide. In particular, the only
momentum of a discrete group is the zero momentum.
 \end{article}

\begin{proof}
Let us check first the injectivity. Let $\alpha \in \cG^*$
such that $j^*(\alpha) = 0$, and let $\P : \U \to \G$ be a
plot. Let $r_0 \in \U$ and let $\B \subset \U$ be a small
open ball centered at $r_0$. Let $g_0 = \P(r_0)$. Since
$\B$ is connected, since $\eL(g_0^{-1}) \circ \P(r_0) =
\id_\G$, and thanks to the smoothness of group
operations, the parametrization $\Q = [\eL(g_0^{-1})
\circ \P] \restriction \B$ is a plot of $\G^\circ$. So,
$\alpha(\Q) = 0$. But, $\alpha(\Q) = \alpha(\eL(g_0^{-1})
\circ (\P \restriction \B)) = \eL(g_0^{-1})^*(\alpha)(\P
\restriction \B) = \alpha(\P \restriction \B)$. Thus,
$\alpha(\P \restriction \B) = 0$. Since $\alpha$ vanishes
locally at each point of $\U$, $\alpha = 0$. And, $j^*$ is
injective. Now, let us prove the surjectivity. Let
$\alpha \in {\cG^\circ}^{\raisebox{-3.2pt}{\scriptsize
*}}$. For any component $\G_i$ of $\G$, let us choose an
element $g_i \in \G_i$, and the identity for the
identity component. Let $\P : \U \to \G$ be a plot, an let
us assume that $\U$ is connected. So, $\P(\U)$ is
contained in one connected component of $\G$, let us say
the component $\G_i$. Let us define then, $\bar
\alpha(\P) = \alpha(\eR(g_i^{-1}) \circ \P)$. Since
$\eR(g_i^{-1}) \circ \P (r) \in \G^\circ$ for all $r \in
\U$, this is well defined. Now, since any plot is the sum
of its restrictions on the components of its domain, the
map $\bar \alpha$ extends naturally to every plot of $\G$.
Now, let $\P : \U \to \G$ be a plot, let $\V$ be a
domain, and let $\F \in \Cinfty(\V,\U)$. Let $s_0 \in
\V$, let $\V_0$ be the  component of $s_0$ in $\V$, let
$r_0 = \F(s_0)$, and let $\U_0$ be the component of $r_0$
in $\U$. Let $\G_i$ be the component of $\P \circ \F(s_0)
= \P(r_0)$ in $\G$. We have, $\bar \alpha((\P \circ \F)
\restriction \V_0) = \bar \alpha((\P \restriction \U_0)
\circ (\F \restriction \V_0)) = \alpha(\eR(g_i^{-1})
\circ (\P \restriction \U_0) \circ (\F \restriction
\V_0)) = \alpha([\eR(g_i^{-1}) \circ (\P \restriction
\U_0)] \circ (\F \restriction \V_0)) = (\F \restriction
\V_0)^*[\alpha(\eR(g_i^{-1}) \circ (\P \restriction
\U_0)] = (\F \restriction
\V_0)^*[\bar \alpha(\P \restriction
\U_0)]$. So locally, $\bar \alpha (\F \circ \P) =_{\rm
loc} \F^*(\bar \alpha(\P))$. And if it is satisfied
locally, it is satisfied globally, thus $\bar \alpha (\F
\circ \P) = \F^*(\bar \alpha(\P))$. The map $\bar \alpha$
is a well defined differential 1-form on $\G$. Now, let us
check that $\bar \alpha$ is invariant by left
multiplication. Let $g \in \G$, let $\P : \U \to \G$ be a
plot, let $r_0 \in \U$, let $\U_0$ be the component of
$r_0$ in $\U$, let $\G_i$ be the component of $\P(r_0)$
in $\G$, so $\P(\U_0) \subset \G_i$. We have,
$\eL(g)^*(\bar \alpha(\P \restriction \U_0)) = \bar
\alpha(\eL(g) \circ (\P \restriction \U_0)) =
\alpha(\eR(g_i^{-1}) \circ \eL(g) \circ (\P
\restriction \U_0)) = \alpha( \eL(g) \circ \eR(g_i^{-1})
\circ (\P \restriction \U_0)) =
[\eL(g)^*(\alpha)] (\eR(g_i^{-1}) \circ (\P \restriction
\U_0)) = \alpha (\eR(g_i^{-1}) \circ (\P \restriction
\U_0)) = \bar \alpha (\P \restriction
\U_0)$. So locally, $\eL(g)^*(\bar \alpha)(\P) =_{\rm loc}
\bar \alpha(\P)$, and therefore globally. So,
$\eL(g)^*(\bar \alpha) = \bar \alpha$, thus $\bar \alpha$
is an element of $\cG^*$, which coincide with $\alpha$ on
$\G^\circ$.
 \end{proof}

\begin{article}[Momenta of coverings of diffeological
groups]
\label{Momenta-of-covering-of-diffeological-groups} 
Let $\G$ be a diffeological group, let $\pr :
\hat \G \to \G$ be some group covering, see
\art{Covering-diffeological-groups}. Let $\cG^*$ and
$\hat \cG^*$ be the spaces of momenta of $\G$ and
$\hat \G$. So, the pullback $\pr^* : \cG^* \to
\hat \cG^*$ is a smooth linear isomorphism.  
 \end{article}

\begin{proof}
Thanks to \art{Momenta-and-connectedness}, it is
sufficient to assume that $\hat \G$ and $\G$ are
connected. And thanks to
\art{Covering-diffeological-groups}, it is sufficient
to prove this for the universal covering $\pi : \tilde \G
\to \G$. Now, $\pi^*$ is obviously linear, let us show
that $\pi^*$ is surjective. Let $\tilde\alpha \in
\widetilde\cG^*$. The group $\G$ is isomorphic to
$\widetilde \G/\pi_1(\G)$, with respect to the left
action of $\pi_1(\G)$. That is $\tilde g \sim k \tilde
g$, for all $k \in \pi_1(\G)$. Now, let $\tilde \alpha
\in \widetilde\cG^*$, $\tilde \alpha$ is left invariant
by $\widetilde \G$, thus by $\pi_1(\G)$.  That is, for
all $k \in \pi_1(\G)$, $\eL(k)^*(\tilde \alpha) = \tilde
\alpha$. But, since $\pi_1(\G) = \ker(\pi)$ is discrete,
this is sufficient for the existence of a 1-form $\alpha$
on $\G$ such that $\tilde \alpha = \pi^*(\alpha)$.  Now,
let $\tilde g \in \widetilde \G$ and $g = \pi(\tilde g)$.
Since $\pi$ is an homomorphism, $\pi \circ \eL(\tilde g) =
\eL(g) \circ \pi$. So, on one hand we have $\eL(\tilde
g)^*(\tilde \alpha) = \eL(\tilde g)^*(\pi^*(\alpha)) =
(\pi \circ \eL(\tilde g))^*(\alpha) = (\eL(g) \circ
\pi)^*(\alpha) = \pi^*(\eL(g)^*(\alpha))$. And, on the
other hand, we have $\eL(\tilde g)^*(\tilde \alpha) =
\tilde \alpha = \pi^*(\alpha)$. Hence,
$\pi^*(\eL(g)^*(\alpha)) = \pi^*(\alpha)$. But, since
$\pi$ is a subduction, $\eL(g)^*(\alpha) = \alpha$. Thus,
$\alpha \in \cG^*$, and the map $\pi^*$ is surjective.
Now, let $\tilde \alpha$ and $\tilde \beta$ be such that
$\pi^*(\tilde \alpha) = \pi^*(\tilde \beta)$. But, since
$\pi$ is a subduction, $\tilde \alpha = \tilde \beta$.
Finally, $\pi^*$ is injective. Finally, since the
pullback is a smooth operation, $\pi^* :
\cG^* \to \widetilde \cG^*$ is a smooth linear
isomorphism.
 \end{proof}

\begin{article}[Linear coadjoint action and coadjoint
orbits] \label{Linear-coadjoint-action} Let $\G$ be a
diffeological group and let $\cG^*$ be the space of its
momenta. The pushforward $\Ad(g)_*(\alpha)$ of a momentum
$\alpha \in \cG^*$, by the adjoint action of any element
$g$ of $\G$, is again a momentum of $\G$, that is again a
left-invariant 1-form. This defines a linear smooth
action of $\G$ on $\cG^*$ called  {\em
coadjoint action}, and denoted by $\Ad_*$.
  $$ 
 \Ad_* : (g,\alpha) \mapsto \Ad(g)_*(\alpha) =
\Ad(g^{-1})^*(\alpha).
 $$
We check immediately that for all $g$, $g'$ in $\G$,
$\Ad_*(gg') = \Ad_*(g) \circ \Ad_*(g')$, and
that $\Ad_*(g)$ is linear. Note that, since
$\alpha$ is left-invariant, $\Ad_*(g)(\alpha) = 
\eR(g)^*(\alpha)$.

The orbit of $\alpha$ by $\G$ is by definition
a {\em coadjoint orbit\/} of $\G$, and
it will be denoted by
 $$
 \cO_\alpha \mbox{ or } \Ad_*(\G)(\alpha) =
\{ \Ad_*(g)(\alpha) \mid g\in \G \}. 
 $$
The orbit $\cO_\alpha$ can be regarded as a subset of
$\cG^*$, but also as the quotient of the group $\G$ by the
stabilizer of the moment $\alpha$,
 $$
\cO_\alpha \simeq \G/\Stab_{\G}(\alpha), \mbox{ with } 
\Stab_{\G}(\alpha) = \{ g \in \G \mid \Ad(g)_*(\alpha) =
\alpha \}.
 $$
 {\sc Note} --- The orbit $\cO_\alpha$ can be equipped
with the subset diffeology of the functional diffeology of
$\cG^*$, or with the quotient diffeology of $\G$. There is
no reason a priori that these two diffeologies coincide.
But it could be interesting however to understand in
which conditions they do.
 \end{article}

\begin{article}[Affine coadjoint actions and
$(\Gamma,\theta)$-coadjoint orbits] 
\label{Affine-coadjoint-actions-and-orbits} Let $\G$ be a
diffeological group, and $\cG^*$ be the space of its
momenta. Let $\Gamma \subset \cG^*$ be a subgroup of
$(\cG^*,+)$, invariant by the coadjoint action $\Ad_*$.
That is, for all $g \in \G$,
 $$
 \Ad_*(g)(\Gamma) \subset \Gamma.
 $$
 So, the coadjoint action of $\G$ on $\cG^*$ project to
the quotient $\cG^*\!/\Gamma$, regarded as an abelian group,
on a smooth action. Let us denote this action by
$\Ad_*^\Gamma$. For every $g \in \G$ and $\tau \in
\cG^*\!/\Gamma$,
 $$
 \Ad_*^\Gamma(g)(\tau) = \class(\Ad_*(g)(\mu))
\qmbox{with} \tau = \class(\mu) \in \cG^*\!/\Gamma.
 $$ 
 Now, let $\theta$ be a smooth map from $\G$
to the space $\cG^*\!/\Gamma$, such that for any pair $g$
and $g'$ of elements of $\G$,
 $$
 \theta(g g') = \Ad_*^\Gamma(g)(\theta(g')) + \theta(g).
 $$
 Such maps are formally known, in the literature as
twisted 1-cocycles of $\G$ with values in
$\cG^*\!/\Gamma$ \cite{Kir74}. We shall  call them
cocycles of $\G$, with values in $\cG^*\!/\Gamma$, or
simply $(\cG^*\!/\Gamma)$-cocycles. A cocycle $\theta$ is
a coboundary if and only if there exists a constant $c \in
\cG^*\!/\Gamma$, such that $\theta = \Delta c$, with
 $$
 \Delta c : g \mapsto \Ad_*^\Gamma(g)(c) -c.
 $$
 Cocycles modulo coboundaries define a cohomology group
denoted by $\H^1(\G,\cG^*\!/\Gamma)$. Every such cocycle
$\theta$ defines a new action of $\G$ on $\cG^*\!/\Gamma$ by
 $$
 \Ad^{\Gamma,\theta}_* : (g, \tau) \mapsto
\Ad_*^\Gamma(g)(\tau) + \theta(g).
 $$
The orbits for these actions will be called the
{\em $(\Gamma,\theta)$-coadjoint orbits} of $\G$. If
$\Gamma = \{0\}$ we shall call them simply
$\theta$-coadjoint orbits. If $\theta = 0$ we shall call
them simply $\Gamma$-coadjoint orbits. And, if $\Gamma =
\{0\}$ and $\theta = 0$ we find again the ordinary
coadjoint orbits defined in 
\art{Linear-coadjoint-action}.
 \end{article}

\begin{article}[Closed momenta of a diffeological group]
\label{Closed-momenta-of-a-diffeological-group}
Let $\G$ be a diffeological group, and let $\cG^*$ be its
space of momenta. Let us denote by $\eZ$ the subset of
closed momenta of $\G$,  and by $\eB$
the subset of exact momenta of $\G$. That is,
 $$
 \eZ = \Z^1_{\D\R}(\G) \cap \cG^* \qmbox{and} \eB =
\B^1_{\D\R}(\G) \cap \cG^*.
 $$
1) Let us assume that $\G$ is connected, and let $\tilde
\G$ be its universal covering. By factorization, the
chain-homotopy operator defines a canonical De Rham
isomorphism $\ek$, from the space of closed momenta
$\eZ$ to the vector space
$\Hom^\infty(\tilde \G,\RR)$. That is, for all $\zeta \in
\eZ$,
 $$
 \ek(\zeta) = [\tilde g \mapsto
\eK\zeta(p)], \qmbox{where} 
\eK\zeta(p) =
\int_{p} \zeta
\qmbox{and} \tilde g = \class(p).
 $$ 
 Here, we have denoted by $\class(p)$  the
fixed ends homotopy class of the path $p \in
\Paths(\G,\id_\G)$. The subspace of exact momenta
$\eB$ identifies, through the
isomorphism $\ek$, to the subspace $\Hom^\infty(\G,\RR)$.
 $$
 \eZ \simeq \Hom^\infty(\tilde \G,\RR)
\qmbox{and} 
\eB \simeq \Hom^\infty(\G,\RR).
 $$

2)  Let $\G$ be any diffeological group connected or
not. Let $\zeta \in \cG^*$, if $\zeta$ is closed then
$\zeta$ is $\Ad_*$ invariant.
 $$
 \mbox{For all } \zeta \in \cG^*, \ d\zeta = 0 \
\Rightarrow \ \Ad_*(g)(\zeta) = \zeta, \mbox{ for
all } g \in \G.
 $$
{\sc Note} --- Every homomorphism from a diffeological
 group $\G$ to an abelian group factorizes through the
{\em abelianized group} $\Ab(\G) = \G/[\G,\G]$, where
$[\G,\G]$ is the normal subgroup of the commutators of
$\G$.  So actually, $\eZ \simeq
\Hom^\infty(\Ab(\tilde \G),\RR)$ and $\eB \simeq
\Hom^\infty(\Ab(\G),\RR)$.
 \end{article}

\begin{proof} 1) Let $\pi : \tilde \G \to
\G$ be the universal covering defined in 
\art{Covering-diffeological-groups}. Since $\tilde
\G$ is simply connected, every closed 1-form is exact
\cite{Piz05}. Thus, for every $\zeta \in \eZ$, the
pullback $\pi^*(\zeta)$ is exact. So, let $\F$ be a
primitive of $\pi^*(\alpha)$, that is $d\F =
\pi^*(\alpha)$. We can even fix uniquely $\F$ by choosing
$\F(\id_{\tilde \G}) = 0$. Actually $\F$ is defined by
integrating the form $\zeta$ along the paths starting at
the identity, that is $\F = \ek(\zeta)$. Since $\alpha$ is
left-invariant and since the projection $\pi$ commutes
with the left actions, on $\G$ and $\tilde \G$,
$\pi^*(\alpha)$ is left invariant. So, for every $\tilde
g \in \tilde \G$, $d[\F \circ \eL(\tilde g)] = d\F$.
Since $\tilde \G$ is connected, for every $\tilde g$,
$\tilde g'$ in $\tilde \G$, $\F(\tilde g \tilde g') =
\F(\tilde g') + f(\tilde g)$. Where $f$ is a smooth real
function. But since $\F(\id_\G) = 0$, $f(\tilde g) =
\F(\tilde g)$, and $\F$ is a smooth homomorphism from
$\tilde \G$ to $\RR$. So, for every closed momentum
$\zeta \in \eZ$, there exists a unique homomorphism $\F
\in \Hom^\infty(\tilde \G, \RR)$ such that $\zeta =
\pi_*(d\F)$. The homomorphism $\ek$ is thus injective,
and it is obviously surjective. Now, if $\zeta$ is
exact, that is if $\zeta = df$, then $\F = \pi^*(f)$.
So, $\ek(\eB) = \pi^*(\Hom^\infty(\G,\RR)) \simeq
\Hom^\infty(\G,\RR)$.

2) Thanks to \art{Momenta-and-connectedness} we can
assume that $\G$ is connected. Now, for every $\tilde g$,
$\tilde g'$ in $\tilde \G$, $\F(\tilde g \tilde g' \tilde
g^{-1}) = \F(\tilde g')$. That is, $\F \circ \Ad(\tilde
g) = \Ad(\tilde g)^*(\F) = \F$, for all $\tilde g \in
\tilde \G$. So, $d[\Ad(\tilde g)^*(\F)] = d\F$, or
$\Ad^*(\tilde g)(\pi^*(\zeta)) = \pi^*(\zeta)$, or $(\pi
\circ \Ad(\tilde g))^*(\zeta) = \pi^*(\zeta)$. But $\pi
\circ \Ad(\tilde g) =  \Ad(g) \circ \pi$,  where $g =
\pi(\tilde g)$. So, $\pi^*(\Ad(g)^*(\zeta)) =
\pi^*(\zeta)$. And since $\pi$ is a subduction,
$\Ad(g)^*(\zeta) = \zeta$. That is,
$\Ad_*(g)(\zeta) = \zeta$.
 \end{proof}

\begin{article}[Equivalence between right and left
momenta] 
\label{Equivalence-between-right-and-left-momenta} Let
$\G$ be a diffeological group, and let $\cG^\star$ denote
the space of {\em right momenta} of the group $\G$. That
is, the space of 1-forms of $\G$, invariant by the right
multiplication.
 $$ \cG^\star
= \{ \alpha \in \Omega^1(\G) \mid \mbox{For all } g \in
\G, \ \eR(g)^*(\alpha) = \alpha \}.
 $$
There exists a natural linear isomorphism $\flip : \cG^*
\to \cG^\star$ equivariant with respect to the coadjoint
action. That is, the following diagram commutes.
$$
\begin{tikzcd}
    \ \cG^* \arrow[r, "\flip"] \arrow[d, "\Ad_*(g)"'] & \cG^\star \arrow[d, "\Ad_*(g)"] \\
   \ \cG^* \arrow[r, "\flip"'] & \cG^\star
\end{tikzcd}
 $$
In other words, there is no reason to prefer left or
right momenta of a diffeological group. The
particularization of left momenta comes because we are
dealing with actions of groups and not anti-actions.
\end{article} 

\begin{proof}
Let us denote by a dot the multiplication in $\G$. Let
$\alpha$ be any left $p$-momentum of $\G$. Let $\P : \U
\to \G$ be a $n$-plot. Let $\bar
\alpha(\P)$ be defined by
 $$ \bar \alpha(\P)(r) = \alpha\left[s \mapsto \P(s)
\cdot \P(r)^{-1}\right](s=r).
 $$
where $r$ belongs to $\U$. Let us show that $
\bar{\alpha}$ defines a $p$-form of $\G$. First
of all let us remark that $\bar \alpha(\P)$ is the
restriction of the 1-form $\alpha((s,r) \mapsto
\P(s)\cdot \P(r)^{-1})$ to the diagonal $s=r$. Thus,
$\bar \alpha(\P)$ is a smooth 1-form of $\U$. 

Now, let us prove that $\bar\alpha$ is a well
defined 1-form on $\G$, according
to the definition of differential forms in diffeology. let
$\F : \V \to \U$ be a smooth $m$-parametrization. Let $v$
be a point of $\V$, and $\delta v$ be a vector of
$\RR^m$. We have:  \begin{eqnarray*}
 \bar{\alpha}(\P \circ \F)_v(\delta v) 
 & = & \alpha\left[s \mapsto (\P \circ \F)(s) \cdot (\P
\circ \F)(v)^{-1}\right]_v(\delta v) \\  
 & = & \alpha\left[s \mapsto
\F(s) \mapsto (\P \circ \F)(s) \cdot (\P \circ
\F)(v)^{-1}\right]_v(\delta v) \\  
 & = & \alpha\left[s \mapsto r = \F(s) \mapsto \P(r)
\cdot \P(\F(v))^{-1}\right]_v(\delta v) \\  
 & = & \alpha\left(\left[r \mapsto \P(r) \cdot
\P(\F(v))^{-1}\right] \circ \F\right)_v(\delta v) \\ 
 & = & {\F}^*\left[\alpha\left(r \mapsto \P(r) \cdot
\P(\F(v))^{-1}\right)\right]_v(\delta v) \\
 & = & \alpha\left[r \mapsto \P(r) \cdot
\P(\F(v))^{-1}\right]_{\F(v)} (\D(\F)(v)(\delta v)) \\
 & = & \bar{\alpha}(\P)_{\F(v)}(\D(\F)(v)(\delta v))  \\
 & = & {\F}^*\left[\bar \alpha(\P)\right]_v(\delta v). 
 \end{eqnarray*}
Then, let us check that $\bar{\alpha}$ is
right-invariant, that is $\bar \alpha \in \cG^\star$. For
all $g \in \G$, we have:
 \begin{eqnarray*}
\eR(g)^*(\bar\alpha)(\P)_r(\delta r) 
 & = &
 \bar\alpha(\eR(g) \circ \P)_r(\delta r) \\
& = & \alpha\left[s \mapsto 
(\eR(g) \circ \P)(s) \cdot (\eR(g) \circ
\P)(r)^{-1}\right]_r(\delta r) \\
 & = & \alpha\left[s \mapsto
\P(s) \cdot g \cdot (\P(r) \cdot
g)^{-1}\right]_r(\delta r) \\
 & = & \alpha\left[s \mapsto \P(s) \cdot g \cdot g^{-1}
\cdot \P(r)^{-1}\right]_r(\delta r) \\
 & = & \alpha\left[s \mapsto \P(s) \cdot
\P(r)^{-1}\right]_r(\delta r) \\
 & = & \bar{\alpha}(\P)_r(\delta r)
 \end{eqnarray*}
So, we have defined a map $\flip : \alpha \mapsto \bar
\alpha$, from $\cG^*$ to $\cG^\star$. Let us prove now
that $\flip$ is bijective. Let $\beta = \bar \alpha$. Let
$\P : \U \to \G$ be a plot, and let us define $\bar
\beta$ by 
 $$\bar \beta(\P)(r) = \beta
[s \mapsto \P(r)^{-1} \cdot \P(s)](s=r),
 $$
 for all $r \in \U$. So, we have:
 \begin{eqnarray*}
\bar\beta(\P)(r) 
 & = & \beta \left[s \mapsto
\P(r)^{-1} \cdot \P(s)\right](s=r) \\
& = & \bar \alpha \left[s \mapsto
\P(r)^{-1} \cdot \P(s)\right](s=r) \\
& = & \alpha \left[s \mapsto
\P(r)^{-1} \cdot \P(s) \cdot \P(r)^{-1} \cdot \P(r)
\right](s=r) \\ 
& = & \alpha \left[s \mapsto
\P(r)^{-1} \cdot \P(s) \right](s=r)  \\
& = & \eL(\P(r)^{-1})^*(\alpha) \left[s \mapsto \P(s)
\right](s=r) \\
& = & \alpha(\P)(r). 
  \end{eqnarray*}
Hence, $\bar \beta = \alpha$. Thus, $\flip$ is bijective.
And, $\flip$ is clearly linear. Therefore, $\flip$ is a
linear isomorphism from $\cG^*$ to $\cG^\star$. It is easy
to check that it is a smooth isomorphism.

Finally, let us check that $\flip$ is equivariant under
the coadjoint action. Let $\alpha \in \cG^*$, let
$\P: \U \to \G$ be a plot and $r \in \U$. On one
hand we have,
 \begin{eqnarray*}
\flip[\Ad(g)^*(\alpha)](\P)_r &=&
\flip[\eR(g)^*(\alpha)](\P)_r  \\
 & = & \eR(g)^*(\alpha)[s \mapsto 
 \P(s) \cdot \P(r)^{-1}]_r\\
& = & \alpha(s \mapsto \P(s) \cdot \P(r)^{-1}
\cdot g)_r. \end{eqnarray*}
And, on the other hand: 
\begin{eqnarray*}
[\Ad(g)^*(\flip(\alpha))](\P)_r & = &
[\eL(g)_*(\flip(\alpha))](\P)_r \\
 & = &
\flip(\alpha)(\eL(g^{-1}) \circ \P)_r \\
& = & \alpha[s \mapsto (\eL(g^{-1}) \circ \P)(s) \cdot
(\eL(g^{-1}) \circ \P)(r))^{-1}]_r \\
& = & \alpha[s \mapsto g^{-1} \cdot \P(s) \cdot
\P(r)^{-1} \cdot g]_r \\ & = & \eL(g^{-1})^*(\alpha)[s
\mapsto \P(s) \cdot \P(r)^{-1} \cdot g]_r \\
& = & \alpha[s \mapsto \P(s) \cdot \P(r)^{-1} \cdot g]_r 
\end{eqnarray*}
Therefore, $\flip \circ \Ad(g)^* = \Ad(g)^* \circ \flip$
for all $g \in \G$. \end{proof}

%************************************************
\section{The paths moment map}

We shall now introduce the notion of moment
map step by step. The first step consists to define
the {\em paths moment map}. 

\begin{article}[Definition of the paths moment map] 
\label{Definition-of-the-paths-moment-map}
Let $\X$ be a diffeological space and $\omega$ be a
closed 2-form defined on $\X$. Let $\G$ be a diffeological
group and $\rho : \G \to \Diff(\X)$ be a smooth
action. Let us denote by the same letter the natural
action of $\G$ on $\Paths(\X)$, induced by the action 
$\rho$ of $\G$ on $\X$. That is, for all $g \in \G$, for
all $p \in \Paths(\X)$,
 $$
 \rho(g)(p) = \rho(g) \circ p = [t \mapsto \rho(g)(p(t))].
 $$
Let us assume now that the action $\rho$ of $\G$ on
$\X$ preserves $\omega$. That is, for all $g \in
\G$,
 $$
 \rho(g)^*(\omega) = \omega \qmbox{or} \rho \in
\Hom^\infty(\G, \Diff(\X,\omega)).
 $$
 Let $\eK$ be the chain-homotopy operator, so $\eK \omega$
is a 1-form of $\Paths(\X)$, and the action of $\G$ on
$\Paths(\X)$ preserves the 1-form $\eK \omega$. This is a
consequence of the variance of the chain-homotopy
operator, see \art{Chain-Homotopy-operator}. Thus, for
all $g \in \G$, 
 $$
   \rho(g)^*(\eK \omega) = \eK \omega.
 $$
 Now, let $p$ be any paths of $\X$, and let $\hat p : \G
\to \Paths(\X)$ be the orbit map. So, the
pullback $\hat p^*(\eK \omega)$ is a left-invariant 1-form
of $\G$, that is an element of $\cG^*$. The map 
 $$
 \Psi : \Paths(\X) \to
\cG^* \qmbox{defined by}  \Psi(p) = \hat p^*(\eK \omega),
 $$
 is smooth with respect to the functional diffeology,
$\Psi \in \Cinfty(\Paths(\X), \cG^*)$. The map $\Psi$
will be called the {\em paths moment map}.
 \end{article}

\begin{article}[Evaluation of the paths moment map]
\label{Evaluation-of-the-paths-moment-map} 
Let $\X$ be a diffeological space and $\omega$ be a
closed 2-form defined on $\X$. Let $\G$ be a diffeological
group and $\rho$ be a smooth action of $\G$ on $\X$,
preserving $\omega$. Let
$p$ be a path in $\X$. Thanks to the explicit expression
of the chain-homotopy operator given in
\art{Chain-Homotopy-operator}, we get the evaluation of
the momentum $\Psi(p)$ on any $n$-plot $\P$ of $\G$,
 \renewcommand{\theequation}{$\heartsuit$}
 \begin{equation}
 \Psi(p)(\P)_r(\delta r) =
\int_0^1 \omega \left[ \mymatrix{s \cr u} \mapsto
(\rho \circ \P)(u) (p(s + t)) \right]_{\left({s=0 \atop
u=r}\right)}\mymatrix{1 \cr 0} \mymatrix{0 \cr \delta r} dt,
 \end{equation}
for all $r$ in $\dom(\P)$ and all $\delta r$ in $\RR^n$.
Now, as a differential 1-form,  $\Psi(p)$
is characterized by its values on the $1$-plots
\cite{Piz05}. So, let $f : t \mapsto f_t$ be a $1$-plot of
$\G$ centered at the identity $\id_\G$, that is $f \in
\Paths(\G)$ and $f(0) = \id_{\G}$. For any $t \in \RR$,
let $\F_t$ be the path in $\Diff(\X,\omega)$ --- centered
at the identity $\id_\X$ --- defined by
 $$
 \F_t : s \mapsto \rho( f_t^{-1} \circ f_{t+s}).
 $$
 So, we have
 \renewcommand{\theequation}{$\clubsuit$}
 \begin{equation}
 \Psi(p)(f)_t(1) = - \int_p i_{\F_t}(\omega) = - \int_0^1
i_{\F_t}(\omega)(p)_s(1) ds,
 \end{equation}  
 where $i_{\F_t}(\omega) $ is the contraction of
$\omega$ by $\F_t$, see
\art{Differential-forms}.

But, as an invariant
1-form on $\G$ the moment $\Psi(p)$ is characterized by
its {\em value at the identity}, that is for $t=0$,
 \renewcommand{\theequation}{$\diamondsuit$}
 \begin{equation}
 \Psi(p)(f)_0(1) = - \int_p i_\F(\omega) = - \int_0^1
i_\F(\omega)(p)_t(1) \ dt \qmbox{with} \F
= \rho \circ f.
 \end{equation}

 {\sc Note} --- Let $f \in
\Hom^\infty(\RR,\G)$, so $\Psi(p)(f)$ is an invariant
$1$-form on $\RR$ whose coefficient is just $\int_p
i_\F(\omega)$. That is,
 $$
 \Psi(p)(f) = h_f(p) \times dt \qmbox{where} h_f(p) =
- \int_p i_\F(\omega).
 $$
 The smooth map $h_f : \Paths(\X) \to \RR$
is the {\em hamiltonian} of $f$, or the hamiltonian of the
1-parameter group $f(\RR)$. Note also that, the map $h :
\Hom^\infty(\RR,\G) \to \Cinfty(\Paths(\X),\RR)$, defined
above, is smooth.
 \end{article}

\begin{proof}
Let us prove $\heartsuit$. Let us remind that for every
$p \in \Paths(\X)$ and every $g \in \G$, $\hat p (g) =
\rho(g)(p) = [t \mapsto \rho(g)(p(t))]$. So, by definition
 \begin{eqnarray*}
\Psi(p)(\P)_r(\delta r) &=& \hat
p^*(\eK\omega)_r(\delta r) \\
&=& \eK\omega(\hat p \circ \P)_r(\delta r) \\
&=& \int_0^1 \omega \bigg[\mymatrix{s \cr r} \mapsto \hat
p \circ \P(r)(s+t) \bigg]_{\left({0  \atop r}\right)}
\mymatrix{1 \cr 0} \mymatrix{0 \cr \delta r} dt \\
&=& \int_0^1 \omega \bigg[\mymatrix{s \cr r} \mapsto (\rho
\circ \P)(r)(p(s+t)) \bigg]_{\left({0  \atop r}\right)}
\mymatrix{1 \cr 0} \mymatrix{0 \cr \delta r} dt.
 \end{eqnarray*}
Let us prove $\clubsuit$. Let us apply the general
formula $\heartsuit$ for $\P = f$. Introducing $u' = u-t$
and $s'' = s+s'$, using the compatibility property of
$\omega(\P \circ \Q) = \Q^*(\omega(\P))$ and the
$\rho(f_t)$ invariance of $\omega$, we
get
 \begin{eqnarray*}
 \Psi(p)(f)_t(1) &=&
\int_0^1 \omega \left[ \mymatrix{s \cr u} \mapsto
\rho(f_u)(p(s + s')) \right]_{\left({s=0 \atop
u=t}\right)}\mymatrix{1 \cr 0} \mymatrix{0 \cr 1}
ds' \\
&=& \int_0^1 \omega \left[ \mymatrix{s'' \cr u'} \mapsto
\rho(f_{t+u'})(p(s'')) \right]_{\left({s''=s' \atop
u'=0}\right)}\mymatrix{1 \cr 0} \mymatrix{0 \cr 1} ds' \\
&=&  \int_0^1 \omega \left[ \mymatrix{s'' \cr u'}
\mapsto \rho(f_t \circ f_t^{-1} \circ f_{t+ u'})(p(s''))
\right]_{\left({s''= s' \atop u'=0}\right)}\mymatrix{1 \cr
0} \mymatrix{0 \cr 1} ds' \\ &=& \int_0^1 \omega \left[
\mymatrix{s'' \cr u'} \mapsto \rho(f_t)
\bigg(\F_t(u')(p(s''))\bigg) \right]_{\left({s''=s' \atop
u'=0}\right)}\mymatrix{1 \cr 0} \mymatrix{0 \cr 1} ds' \\
&=& \int_0^1 \omega \left[ \mymatrix{s'' \cr u'} \mapsto
\F_t(u')(p(s'')) \right]_{\left({s''=s' \atop
u'=0}\right)}\mymatrix{1 \cr 0} \mymatrix{0 \cr 1} ds' \\
&=& \int_0^1 \omega \left[ \mymatrix{u' \cr s''} \mapsto
\F_t(u')(p(s'')) \right]_{\left({u'=0 \atop
s''=s'}\right)}\mymatrix{0 \cr 1} \mymatrix{1 \cr 0} ds' \\
&=& - \int_0^1 \omega \left[ \mymatrix{u' \cr s''} \mapsto
\F_t(u')(p(s'')) \right]_{\left({u'=0 \atop
s''=s'}\right)}\mymatrix{1 \cr 0} \mymatrix{0 \cr 1} ds' \\
&=& - \int_0^1 i_{\F_t}(\omega)(p)_{s'}(1) ds' \\
&=& - \int_p i_{\F_t}(\omega).
 \end{eqnarray*}
 Let us prove the Note. Let $f \in \Hom^\infty(\RR,\G)$.
By definition of differential forms and pullbacks,
$\Psi(p)(f) = f^*(\Psi(p))$, but since $f$ is an
homomorphism from $\RR$ to $\Diff(\X,\omega)$ and
$\Psi(p)$ is a left-invariant 1-form on
$\Diff(\X,\omega)$, $f^*(\Psi(p))$ is an invariant
1-form of $\RR$, so $\Psi(p)(f) = f^*(\Psi(p)) = a \times
dt$, for some real $a$.
 So, $\Psi(p)(f)_r = 
\Psi(p)(f)_0(1) \times dt = h_f(p) \times dt$, with 
$h_f(p) = \Psi(p)(f)_0(1) = -\int_p i_\F(\omega)$, and
$dt$ is the canonical 1-form on $\RR$.
 \end{proof}

\begin{article}[Variance of the paths moment map] 
\label{Variance-of-the-paths-moment-map} Let $\X$ be a
diffeological space and $\omega$ be a closed 2-form
defined on $\X$. Let $\G$ be a diffeological group and
$\rho$ be a smooth action of $\G$ on $\X$, preserving
$\omega.$ The paths moment map $\Psi$, defined in
\art{Definition-of-the-paths-moment-map}, is equivariant
under the action of $\G$. That is, for all $g \in \G$,
 $$
 \Psi \circ \rho(g)_* = \Ad(g)_* \circ \Psi.
 $$
 \end{article}

\begin{proof}
Let us denote here the
orbit map  $\hat p$ of every path $p \in \Paths(\X)$ by
$\eL(p)$. That is, $\eL(p)(g) = \rho(g)_*(p) = \rho(g)
\circ p$. So, $\Psi(\rho(g)_*(p)) = \Psi(\rho(g) \circ p)
= (\eL(\rho(g) \circ p)^*(\eK\omega)$. But, $\eL(\rho(g)
\circ p)(g') = \rho(g')(\rho(g) \circ p) = \rho(g'g)\circ
p = \eL(p)(g'g) = \eL(p)\circ \eR(g)(g')$. Thus,
$\eL(\rho(g) \circ p) = \eL(p)\circ \eR(g)$, and
$\Psi(\rho(g)_*(p)) = (\eL(p) \circ \eR(g))^*(\eK\omega) =
\eR(g)^*(\eL(p)^*(\eK(p)) =
\eR(g)^*(\Psi(p))$. But since $\Psi(p)$ is
left-invariant, $\eR(g)^*(\Psi(p)) =
\Ad(g)_*(\Psi(p))$, and $\Psi(\rho(g)_*(p)) =
\Ad(g)_*(\Psi(p))$.
 \end{proof}

\begin{article}[Additivity of the paths moment map] 
\label{Additivity-of-the-paths-moment-map} Let $\X$ be a
diffeological space and $\omega$ be a closed 2-form
defined on $\X$. Let $\G$ be a diffeological group and
$\rho$ be a smooth action of $\G$ on
$\X$, preserving $\omega$. The paths moment map $\Psi$,
defined in \art{Definition-of-the-paths-moment-map},
satisfies the following additive property: for any two
juxtaposable paths $p$ and $p'$ in $\X$,
 $$
 \Psi( p \vee p') = \Psi(p) + \Psi(p') \qmbox{and}
\Psi(\bar p) = - \Psi(p), \qmbox{with} \bar p (t) =
p(1-t).
 $$
\end{article}

\begin{proof}
This is a direct application of the expression given in
\art{Evaluation-of-the-paths-moment-map} $\diamondsuit$,
and of the additivity of the integral of differential
form on paths.
 \end{proof}

\begin{article}[Differential of the paths moment map] 
\label{Differential-of-a-paths-momentum} Let $\X$ be a
diffeological space and $\omega$ be a closed 2-form
defined on $\X$. Let $\G$ be a diffeological group and
$\rho$ be  a smooth action of $\G$ on $\X$, preserving
$\omega$. Let $p$ be a path in $\X$. So, the exterior
differential of the paths momentum $\Psi(p)$ is given by
 $$
 d(\Psi(p)) = \hat x_1^*(\omega) - \hat x_0^*(\omega),
 $$
 where $x_0=p(0)$ and $x_1 = p(1)$, and the $\hat x_i$
denote the orbit maps. \end{article}

\begin{proof}
This is a direct application of the main property of the
chain-homotopy operator, $d \circ \eK + \eK \circ d =
\but^* - \source^*$. Since
$d\omega = 0$, we have $d(\eK \omega) = \but^*(\omega) -
\source^*(\omega)$, composed with $\hat p^*$, we get $\hat
p^* \circ d(\eK \omega) = \hat p^* \circ \but^*(\omega)
- \hat p^* \circ \source^*(\omega)$. That is $d(\hat
p^*(\eK \omega)) = (\but \circ \hat p)^*(\omega) -
(\source \circ \hat p)^*(\omega)$. Thus, $d(\Psi(p)) =
\hat x_1^*(\omega) - \hat x_0^*(\omega)$.
 \end{proof}

\begin{article}[Homotopic invariance of the paths moment
map]
\label{Homotopic-invariance-of-the-paths-moment-map} 
 Let $\X$ be a diffeological space and $\omega$ be a
closed 2-form defined on $\X$. Let $\G$ be a diffeological
group and $\rho$ be a smooth action of $\G$ on $\X$,
preserving $\omega$. Let $p_0$ and $p_1$ be any two paths
in $\X$. If $p_0$ and $p_1$ are
 fixed ends homotopic, then $\Psi(p_0) = \Psi(p_1)$.
\end{article}

\begin{proof}
 Let $s \mapsto
p_s$ be a fixed ends homotopy connecting $p_0$ to $p_1$,
for example let $p_s(0)=x_0$ and $p_s(1) = x_1$, for all
$s$. Let $f$ be a 1-plot of $\G$ centered at the identity
$\id_\G$, that is $f(0) = \id_\G$, and let $\F = \rho
\circ f$. We use the fact that the moment of paths is
characterized by its value at the identity,
$\Psi(p_s)(f)_0(1) = -\displaystyle\int_{p_s}
i_\F(\omega)$, see
\art{Evaluation-of-the-paths-moment-map} $\diamondsuit$.
Let us differentiate this equality with respect to $s$,
 $$
 {\partial \over
\partial s}\bigg(\Psi(p_s)(f)_0(1)\bigg) = - \delta
\int_{p_s} i_\F (\omega), \qmbox{with} \delta = {\partial
\over \partial s}.
 $$
 The
variation of the integral of differential forms on
chains gives
 $$
 \delta \int_{p_s} i_\F (\omega) =
\int_0^1 d\,[i_\F(\omega)] ( \delta p_s ) +
\bigg[ i_\F(\omega) ( \delta
p_s)\bigg]_{\raisebox{1pt}{\scriptsize
0}}^{\raisebox{-2pt}{\scriptsize
 1}}.
 $$
See  \cite{Piz05}  for the definition of
$\delta p_s$ and for the proof of this formula in
diffeology. Since the homotopy $s \mapsto p_s$ is a
fixed end homotopy, $\delta p_s(0) = 0$ and $\delta p_s(1)
= 0$, thus the second summand of the right term
vanishes. Now, the Cartan formula writes
$\pounds_\F(\omega) =  d[i_\F(\omega)]+i_\F(d\omega)$, see
\art{Differential-forms}. But $\omega$ is invariant under
the action of $\G$, so $\pounds_\F(\omega) =0$, and since
$d\omega = 0$ we get $d[i_\F(\omega)] =
\pounds_\F(\omega) = 0$. So, $\delta {\displaystyle
\int_{p_s}} i_\F (\omega) = 0$ and $\Psi(p_0) = \Psi(p_s)
= \Psi(p_1)$, for all $s$. 
 \end{proof}

\begin{article}[The holonomy group] 
\label{The-holonomy-group} 
 Let $\X$ be a connected diffeological space, and let
$\omega$ be a closed 2-form defined on $\X$. Let $\G$ be a
diffeological group and $\rho$ be a smooth action
of $\G$ on $\X$, preserving $\omega$. Let
$\Psi$ be the paths moment map defined in 
\art{Definition-of-the-paths-moment-map}. We define  the
{\em holonomy} $\Gamma$ of the action $\rho$ as
 $$
 \Gamma = \{ \Psi(\ell) \mid \ell \in \Loops(\X) \}.
 $$
 \begin{enumerate}
 \item The holonomy $\Gamma$ is an
additive subgroup of the subspace of closed
momenta, $\Gamma \subset \eZ$
(see \art{Closed-momenta-of-a-diffeological-group}). That
is, for every elements $\gamma$ and $\gamma'$ of
$\Gamma$, 
 $$
 d\gamma = 0 \qmbox{and} \gamma - \gamma' \in \Gamma.
 $$
 \item The paths moment map $\Psi$, restricted to
$\Loops(\X)$, factorizes through an homomorphism
from $\pi_1(\X)$ to $\cG^*$. Thus, $\Gamma$ is an
homomorphic image of $\pi_1(\X)$, or its abelianized
$\Ab(\pi_1(\X))$.
 \item In
particular, every element $\gamma$ of $\Gamma$ is
invariant by the coadjoint action of $\G$ on $\cG^*$. For
all $g$ in $\G$,
 $$
 \Ad_*(g)(\gamma) = \gamma.
 $$
 \end{enumerate}
 The holonomy $\Gamma$ is the obstruction
for the action $\rho$
to be \og hamiltonian\fg. Precisely, the action of $\G$
on $\X$ will be said to be {\em hamiltonian} if and only
if $\Gamma = \{0\}$. Note that, if the group $\G$
has no $\Ad_*$-invariant 1-forms except 0, the action
$\rho$ is
necessarily hamiltonian, see 
\art{Closed-momenta-of-a-diffeological-group}.
\end{article}

\begin{proof}
We get immediately that $\gamma \in \Gamma$ is closed, by
application of the differential of a path momentum: for
all path $p \in \Paths(\X)$, $d(\Psi(p)) = \hat
x_1^*(\omega) -\hat x_0^*(\omega)$, where $x_0=p(0)$ and
$x_1=p(1)$, see
\art{Differential-of-a-paths-momentum}. So, for any loop $\ell$ of $\X$, $\ell(0) =
\ell(1)$ and $d(\Psi(\ell)) = 0$. Now, let $x_0$ be any
point of $\X$. Thanks to
\art{Homotopic-invariance-of-the-paths-moment-map}, for
every loop $\ell \in \Loops(\X,x_0)$, the momentum
$\Psi(\ell)$ depends on $\ell$ only through the its
homotopy class. So $\Gamma$ is the image of
$\pi_1(\X,x_0)$. And, thanks to the additive property
of $\Psi$, see \art{Additivity-of-the-paths-moment-map},
the map $\class(\ell) \mapsto \Psi(\ell)$ is an
homomorphism. Now, since $\X$ is connected, for every
other point $x_1$ of $\X$, there exists a path $c$
connecting $x_0$ to $x_1$, and let $\bar c = t \mapsto
c(1-t)$. Thanks to the additive property, $\Psi(\bar c
\vee \ell \vee c) = \Psi(\bar c) + \Psi(\ell) + \Psi(c) =
- \Psi(c) + \Psi(\ell) + \Psi(c) = \Psi(\ell)$. And,
since the map $\class(\ell) \mapsto \class(\bar c \vee
\ell \vee c)$ is a conjugation from $\pi_1(\X,x_0)$ to
$\pi_1(\X,x_1)$, $\Gamma$ is the same homomorphic image
of $\pi_1(\X,x)$, for every point $x \in \X$. So, we
proved the points 1 and 2, the third one is a direct
consequence of
\art{Closed-momenta-of-a-diffeological-group}.
 \end{proof}

%************************************************
\section{The 2-points moment map}

The definition of the paths moment map leads immediately
to the {\em $2$-points moment map}. The
2-points moment map satisfies a cocycle condition
inherited from the additive property of the paths
moment map. This
is the second step in our general construction.

\begin{article}[Definition of the 2-points moment map] 
\label{Definition-of-the-2-points-moment-map} 
 Let $\X$ be a connected diffeological space and
$\omega$ be a closed 2-form defined on $\X$. Let $\G$ be a
diffeological group and $\rho$ be a smooth action
of $\G$ on $\X$, preserving $\omega$. Let $\Psi$ be the
paths moment map and $\Gamma$ be
the holonomy of the action $\rho$, see  
\art{Definition-of-the-paths-moment-map} and
\art{The-holonomy-group}. So, there exists a smooth map
$\psi : \X \times \X \to \cG^* / \Gamma$ such that the
following diagram commutes.
 $$
\begin{tikzcd}
   \ \Paths(\X) \arrow[r, "\Psi"] \arrow[d, "\bounds"'] & \cG^* \arrow[d, "\pr"] \\
    \ \X \times \X \arrow[r, "\psi"'] & \cG^*\!/\Gamma
\end{tikzcd}
 $$
where $\pr$ is the canonical projection from $\cG^*$ onto
its quotient, and $\bounds = \source \times \but$,
that is $\bounds(p) =  (p(0),p(1))$. The map $\psi
\in \Cinfty(\X \times \X, \cG^*\!/\Gamma)$ will be called
the {\em 2-points moment map}.  \begin{enumerate}
 \item The 2-points moment map $\psi$ satisfies the
Chasles cocycle relation, for any three points $x$, $x'$,
$x''$ of $\X$,
 \begin{equation}
\renewcommand{\theequation}{$\heartsuit$}
  \psi(x,x'') = \psi(x,x') + \psi(x',x'') .
 \end{equation}
 \item The 2-points moment map $\psi$ is equivariant
under the action of $\G$. That is, for any $g \in \G$,
and any pair of points $x$ and $x'$ of $\X$,
 $$
 \psi(\rho(g)(x), \rho(g)(x')) =
\Ad_*^\Gamma(g)(\psi(x,x')).
 $$
\end{enumerate}
{\sc Note.} T. Ratiu and A. Weinstein have kindly
pointed out that Condevaux, Dazord and Molino \cite{CDM88}
proposed a similar construction in the case where
$\X$ is a manifold, $\G$ is a Lie group, and $\Gamma$ is
closed in $\cG^*$.
 \end{article}

\begin{proof}
By construction $\psi$ is defined by $\psi(x,x') =
\class_\Gamma(\Psi(p))$, where $p \in \Paths(\X)$, $x =
p(0)$, $x' = p(1)$, and $\class_\Gamma(\alpha)$ denotes
the class of $\alpha \in \cG^*$ in $\cG^*\!/\Gamma$. The
map $\psi$ is smooth simply by general properties of
subductions in diffeology. Now, the first point is a
direct consequence of the additive property of the paths
moment map, see \art{Additivity-of-the-paths-moment-map}.
The second point is a direct consequence of the
equivariance of the paths moment map of the $\Ad_*$
invariance of $\Gamma$, see
\art{Variance-of-the-paths-moment-map}, and of the
definition of the $\Ad_*^\Gamma$ action, see
\art{Affine-coadjoint-actions-and-orbits}. 
 \end{proof}

\delete{
\begin{article}[Moment maps and equivariant
cohomology] 
\label{Moment-maps-and-equivariant-cohomology} 
 Let $\X$ be a diffeological space, together with a
smooth action of a diffeological group $\G$, denoted by
$(g,x) \mapsto g_\X(x)$, with $(g,x) \in \G \times \X$.
Let $\A$ be an abelian group, together with an action of
$\G$ denoted by $(g,a) \mapsto g_\A(a)$, where $(g,a) \in
\G \times \A$. Let $p$ be a positive integer. A (smooth)
$p$-cochain of the {\em $\G$-equivariant cohomology} from
$\X$ with coefficients in $\A$ is any smooth map $f \in
\Cinfty(\X^{p+1},\A)$ equivariant under the action of
$\G$. That is
 $$
 f(g_X(x_0),\ldots,g_\X(x_p)) = g_\A(f(x_0,\ldots,x_p)),
 $$
 for all
$g \in \G$ and for all $(x_0,\ldots,x_p) \in
\X^{p+1}$. The {\em coboundary operator} $\Delta$
associates to every $p$ chain $f$ the $(p+1)$-chain
$\Delta f$ defined by
 $$
 \Delta(f)(x_0,\ldots,x_{p+1}) = \sum_{k=0}^{p+1} (-1)^k
f(x_0,\ldots,[x_k],\ldots,x_{p+1}),
 $$
where the bracket means that the argument is
omitted. Any cochain in the kernel of  $\Delta$
is a {\em cocycle}, and any
cochain in the image of $\Delta$ is a {\em
coboundary}, of the equivariant cohomology. The spaces of
$p$-cocycles and $p$-coboundaries are denoted
respectively by $\Z^p_\G(\X,\A)$ and $\B^p_\G(\X,\A)$.
And, we say that a cocycle $f$ is {\em exact\/} if it is
the coboundary of some cochain. Then, the $\G$-equivariant
cohomology group of degree $p$, from $\X$ to $\A$,
with respect to the given action of $\G$, is by definition
the quotient group $\H^p_\G(\X,\A) =
\Z^p_\G(\X,\A)/\B^p_\G(\X,\A)$. 

Let us remark that every $p$-cocycle $f$ from $\X$ to
$\A$ defines a map $\F$ from $\X$ to the group
$\Z^p_\G(\G,\A)$ of $p$-cocycle of $\G$ with coefficients
in $\A$ by
 $$
 \F(x) :
(g_0,\ldots,g_p) \mapsto f((g_0)_\X(x), \ldots,
(g_p)_\X(x))
 $$
 where $\G$ is regarded with its left
multiplication. Now, for any two points $x$ and $x'$ of
$\X$, the two cocycles $\F(x)$ and $\F(x')$ are
cohomologous. Thus, the map $f \mapsto \F$ from
$\H^p_\G(\X,\A)$ to $\H^p_\G(\G,\A)$ depends only on the
action of $\G$ on $\X$.


For example, a 1-cocycle $f$ is a smooth map from $\X
\times \X$ to $\A$ such that $\Delta f(x,x',x'') =
f(x',x'') - f(x,x'') + f(x,x') =0$, or $f(x,x'') =
f(x,x') + f(x',x'')$. A 1-coboundary is a 1-cocycle
$\Delta(m) : (x,x') \mapsto m(x') - m(x)$, with $m \in
\Cinfty(\X,\A)$, equivariant under the action of $\G$. 

\end{article}

\begin{proof}
Let $x$ and $x$ be two points of $\X$. Let $f$ be a
$p$-cocycle of the equivariant cohomology of $\X$ with
coefficients in $\A$. For all $(g_0,\ldots,g_p) \in
\G^{p+1}$ we have
 \end{proof}
}

%************************************************
\section{The moment maps}

From the construction of the paths moment map of
\art{Definition-of-the-paths-moment-map} and the 2-points
moment map of \art{Definition-of-the-2-points-moment-map}
we get the notion of (1-point) moment map. This is the
third step of our general construction, and the
generalization of the notion of moment map
coming from classical symplectic geometry.

\begin{article}[Definition of the moment maps] 
\label{Definition-of-the-moment-maps} 
 Let $\X$ be a connected diffeological space and let
$\omega$ be a closed 2-form defined on $\X$. Let $\G$ be a
diffeological group and $\rho$ be a smooth action
of $\G$ on $\X$, preserving $\omega$. Let $\psi$ be the
2-points moment map defined in 
\art{Definition-of-the-2-points-moment-map}. There
exists always a smooth map $\mu : \X \to \cG^*\!/\Gamma$,
called a {\em primitive} of $\psi$, such that, for any two
points $x$ and $x'$ of $\X$,
 $$
 \psi(x,x') = \mu(x') - \mu(x).
 $$
For every point $x_0 \in \X$, for every
constant $c \in \cG^*\!/\Gamma$, the map $\mu$ defined by
 $$
 \mu(x) = \psi(x_0,x) + c.
 $$
is a primitive of $\psi$. Every primitive $\mu$ of
$\psi$ is of this kind, and any two primitive $\mu$ and
$\mu'$ of $\psi$ differ only by a constant.

The 2-points moment map $\psi$ will be said to be
{\em exact} if there exists a primitive
$\mu$, {\em equivariant} by the action of $\G$. That is,
if there exists a primitive $\mu$ such that
 $$
 \mu \circ \rho(g) = \Ad_*^\Gamma(g) \circ \mu,
 $$
 for all $g \in \G$. The primitives $\mu$ of $\psi$,
equivariant or not, will be called the {\em moment
maps}\footnote{These maps should have been called the
{\em1-point moment maps\/}. But to conform with the usual
denomination we chose to call them simply {\em moment
maps}.}. 

{\sc Note} --- By the identity $\heartsuit$ of
\art{Definition-of-the-2-points-moment-map},
$\psi$ is a $1$-cocycle of the $\G$-equivariant
cohomology of $\X$ with coefficients in $\cG^*\!/\Gamma$,
twisted by the coadjoint action. Two cocycles $\psi$ and
$\psi'$ are cohomologous if and only if, there exists a
smooth equivariant map $\mu : \X \to \cG^*\!/\Gamma$,
such that $\psi'(x,x') = \psi(x,x') + \Delta \mu(x,x')$
where $\Delta \mu(x,x') = \mu(x') - \mu(x)$, $\Delta \mu$
is a coboundary. So, the 2-points moment map $\psi$
defines a class belonging to $\H^1_\G(\X,\cG^*\!/\Gamma)$
which depends only on the form $\omega$ and the action
$\rho$ of $\G$ on $\X$. If the moment map $\psi$ is exact,
that is if $\class(\psi) = 0$, we shall say that the
action $\rho$ of $\G$ on $\X$ is {\em exact}, with
respect to $\omega$. In this case, there
exists a point $x_0$ of $\X$ and a constant $c$ such that
$\mu : x \mapsto \psi(x_0,x) + c$ is an equivariant
primitive for $\psi$. 
 \end{article}

\begin{proof}
Let $x_0$ be a chosen point of $\X$. Since $\X$ is
connected, for any $x \in \X$ there exists always a path
$p \in \X$ such that $p(0) = x_0$ and $p(1) = x$. Thus,
defining $\mu(x) = \psi(x_0,x) = \class(\Psi(p))$, and
thanks to the cocycle properties of $\psi$, we have
$\psi(x,x') = \psi(x,x_0) + \psi(x_0,x') = \psi(x_0,x') -
\psi(x_0,x) = \mu(x') - \mu(x)$. Now, since $\psi$ is
smooth, $\mu$ is smooth. Therefore, the equation
$\psi(x',x) = \mu(x') - \mu(x)$ has always a solution in
$\mu$.

Now, let $\mu$ and
$\mu'$ be two primitives of $\psi$. For each pair
$x$, $x'$ of points of $\X$ we have $\mu'(x') - \mu'(x) =
\mu(x') - \mu(x)$. That is, $\mu'(x') - \mu(x') = \mu'(x)
- \mu(x)$. So, the map $x \mapsto \mu'(x) - \mu(x)$ is
constant. There exists $c \in \cG^*\!/\Gamma$ such that
$\mu'(x) - \mu(x) = c$, that is $\mu'(x) = \mu(x) + c$. 

Since, the maps $x \mapsto \psi(x_0,x)$, where
$x_0$ is a fixed point of $\X$, is a special solution of
the equation in $\mu$, $\psi(x',x) = \mu(x') -
\mu(x)$, any solution writes $\mu(x) = \psi(x_0,x) + c$
for some point $x_0 \in \X$ and some constant $c \in
\cG^*\!/\Gamma$. \end{proof}

\begin{article}[Souriau's cocycles] 
\label{Souriau-s-cocycles} 
 Let $\X$ be a connected diffeological space and
$\omega$ be a closed 2-form defined on $\X$. Let $\G$ be a
diffeological group and $\rho$ be a smooth action
of $\G$ on $\X$, preserving $\omega$. Let $\psi$ be the
2-points moment map defined in 
\art{Definition-of-the-2-points-moment-map} and let
$\mu$ be a primitive of $\psi$ as defined in
\art{Definition-of-the-moment-maps}. So there exists a
 map $\theta \in \Cinfty(\G,\cG^*\!/\Gamma)$ such that
 $$
 \mu(\rho(g)(x)) = \Ad_*^\Gamma(g)(\mu(x)) + \theta(g).
 $$
 The map $\theta$ is a $(\cG^*\!/\Gamma)$-cocycle, as
defined in  \art{Affine-coadjoint-actions-and-orbits}.
For all $g,g' \in \G$,
 $$
  \theta(gg') = \Ad_*^\Gamma(g)(\theta(g')) + \theta(g).
 $$
We shall call the cocycle $\theta$,
{\em Souriau's cocycle} of the moment $\mu$.
 \begin{enumerate}
 \item Two Souriau's cocycles $\theta$ and $\theta'$,
associated to two moment maps $\mu$ and $\mu'$ are
{\em cohomologous\/}. That is, they differ by a {\em
coboundary}
 $$
 \Delta c : g \mapsto \Ad_*^\Gamma(g)(c) -c, \qmbox{where}
c \in \cG^*\!/\Gamma.
 $$
 \item For the affine coadjoint action of $\G$ on
$\cG^*\!/\Gamma$ defined by $\theta$, see
\art{Affine-coadjoint-actions-and-orbits}, the moment
map $\mu$ is equivariant. For all $g \in \G$,
 $$
 \mu \circ \rho(g) = \Ad_*^{\Gamma,\theta}(g)
\circ \mu.
 $$
 \item For every cocycle $\theta$, associated to some
moment $\mu$, there exists always a point $x_0 \in \X$ and
a constant $c \in \cG^*\!/\Gamma$ such that, for all $g$
in $\G$
 $$
 \theta(g) = \psi(x_0,\rho(g)(x_0)) + \Delta c(g).
 $$
 \item The cohomology class $\sigma$ of $\theta$ belongs
to a cohomology group denoted by
$\H^1(\G,\cG^*\!/\Gamma)$. And, it depends only
on the cohomology class of the 2-points moment map $\psi$.
This class $\sigma$ will be called {\em Souriau's
cohomology class}. 
 \end{enumerate}
 {\sc Note 1} --- Let $x_0$ by some point of $\X$. The
2-moment map (1-cocycle) $\psi$ defines a 1-cocycle $f$
from $\G$ to $\cG^*\!/\Gamma$ by $f(g,g') =
\psi(\rho(g)(x_0),\rho(g')(x_0))$. The cocycle $f'$
associated to another point $x_0'$ will differ just by a
coboundary. So, Souriau's cocycle $\sigma$ represents just
the class of this pullback $f = \hat x_0^*(\psi)$ by the
orbit map $\hat x_0$, where $\hat
x_0^* : \H^1_\rho(\X,\cG^*\!/\Gamma)
\to \H^1(\G,\cG^*\!/\Gamma)$. And, by the way, depends
only of the restriction of $\omega$ on any one orbit of
$\G$ on $\X$. So, a good choice of the point $x_0$
can simplify sometimes the computation of $\sigma$.

 {\sc Note 2} --- The nature of the action
$\rho$ has strong consequences on Souriau's class. For
example, thanks to the third item, if the group $\G$ has
a fixed point $x_0$, that is $\rho(g)(x_0) = x_0$ for all
$g$ in $\G$, then Souriau's class vanishes. So, the
cocycle $\psi$ is exact, and there exists an equivariant
primitive $\mu$ of $\psi$.
 \end{article}

\begin{proof}
Thanks to \art{Definition-of-the-moment-maps}, every
moment map $\mu$ writes $\mu(x) = \psi(x_0,x) +c$, where
$x_0$ is some fixed point of $\X$ and $c \in
\cG^*\!/\Gamma$. So,  $\mu(\rho(g)(x)) -
\Ad_*^\Gamma(g)(\mu(x)) = \psi(x_0,\rho(g)(x))
+ c - \Ad_*^\Gamma(g)(\psi(x_0,x) + c)
= \psi(x_0,\rho(g)(x)) + c - \Ad_*^\Gamma(g)(\psi(x_0,x))
- \Ad_*^\Gamma(g)(c) =  \psi(x_0,\rho(g)(x))
- \psi(\rho(g)(x_0),\rho(g)(x)) - \Delta c(g)
= \psi(x_0,\rho(g)(x))
+ \psi(\rho(g)(x),\rho(g)(x_0)) - \Delta c(g)
= \psi(x_0,\rho(g)(x_0)) - \Delta c(g)$. Therefore,
$\mu(\rho(g)(x)) - \Ad_*^\Gamma(g)(\mu(x))$ is 
constant with respect to $x$. That proves the points
1) and 4). Now, the variance of $\theta$ with respect to
the multiplication of $\G$ is a classical result of
cohomology (see for example \cite{Kir74}). It is then
obvious that two moment maps $\mu$ and $\mu'$ differing
just by a constant, the associated cocycles $\theta$ and
$\theta'$ differ by a coboundary. The remaining items are
just the results of  elementary, or well known, algebraic
computations.
 \end{proof}

%************************************************
\section{The moment maps for exact 2-forms}

The special case where the closed 2-form is the
exterior differential of an invariant 1-form deserves a
special care, since it justifies the constructions
above, by analogy with the moment maps of classical
symplectic geometry.

\begin{article}[The exact case] 
\label{The-exact-case} 
 Let $\X$ be a connected diffeological space and let
$\omega$ be a closed 2-form defined on $\X$. Let $\G$ be
a diffeological group and $\rho$ be a smooth action
of $\G$ on $\X$, preserving $\omega$. Let us assume that
$\omega = d\alpha$ and that $\alpha$ is also invariant
under the action of $\G$, that is $\rho(g)^*(\alpha) =
\alpha$ for all $g$ in $\G$. Let $\Psi$ be the paths
moment map defined in
\art{Definition-of-the-paths-moment-map}, and $\psi$ be
the 2-points moment map defined in 
\art{Definition-of-the-2-points-moment-map}. So, for
every $p \in \Paths(\X)$
 $$
 \Psi(p) = \psi(x,x') = \hat x_1^*(\alpha) - \hat
x_0^*(\alpha),
 $$
where $x_1 = p(1)$ and $x_0 = x_0$. Moreover, the 2-points
moment map $\psi$ is exact, and every equivariant moment
map is
cohomologous to
 $$
 \mu : x \mapsto \hat x^*(\alpha).
 $$
 The action of $\G$ is hamiltonian, $\Gamma = \{0\}$
and exact $\sigma = 0$, see
\art{The-holonomy-group} and
\art{Souriau-s-cocycles}. So, this shows in particular the
coherence of the general constructions developed until
now. 
 \end{article}

\begin{proof} By definition of the paths moment map,
$\Psi(p) = \hat p^*(\eK \omega)$. So, $\Psi(p) = \hat
p^*(\eK(d\alpha))$. But, $\eK(d\alpha) + d(\eK \alpha) =
\but^*(\alpha) - \source^*(\alpha)$, thus $\eK(d\alpha) 
= \hat p^*[\but^*(\alpha) - \source^*(\alpha) -
d(\eK\alpha)]$. And, $\Psi(p) = (\but \circ \hat
p)^*(\alpha) - (\source \circ \hat p)^*(\alpha) - d[\hat
p^* (\eK(\alpha))]$. But, $\but \circ \hat p = \hat x_1$,
and $\source \circ \hat p = \hat x_0$. So $\Psi(p) = \hat
x_1^*(\alpha)-\hat x_0^*(\alpha) -d[\hat
p^*(\eK\alpha)]$. Now, $\eK\alpha$ is the real function
 $$
 \eK \alpha : p \mapsto \int_p \alpha.
 $$
 Since $\hat p^*(\eK\alpha) = 
\eK\alpha \circ \hat p$, we have for all $g \in \G$,
 $$
\eK\alpha(\hat p(g)) = \int_{\rho(g) \circ p} \alpha
=  \int_p \rho(g)^*(\alpha) = \int_p \alpha.
 $$ 
 So, the function $\hat p^*(\eK\alpha): \G \to \RR$ is
constant and equal to $\int_p \alpha$. So, $d [\hat
p^*(\eK\alpha)] = 0$, and $\Psi(p) = \hat
x_1^*(\alpha) - \hat x_0^*(\alpha)$. Thus,
$\Psi(p) = \psi(x_0,x_1)$ and $\Gamma = \{0\}$.

Now, the function 
$\mu: x \mapsto \hat x^*(\alpha)$ is clearly a primitive
of $\psi$. That is, $\psi(x_0,x_1) = \mu(x_1) - \mu(x_0)$.
But $\eR(\rho(g)(x)) = \hat x \circ \eR(g)$,
where $\eR(\rho(g)(x))$ denotes the orbit map of
$\rho(g)(x)$, with $g \in \G$. So,
$\mu(\rho(g)(x)) = (\hat x \circ
\eR(g))^*(\alpha) = \eR(g)^*(\hat x^*(\alpha))
= \eR(g)^*(\mu(x)) = \Ad_*(g)(\mu(x))$. Thus, $\mu$ is an
equivariant primitive of $\psi$. And, Souriau's
class $\sigma$ vanishes.
 \end{proof}

%************************************************
\section{Functoriality of the moment maps}

We inspect now, the behavior of the moment
maps and the various associated objects under natural
transformations. 

\begin{article}[Images of the moment maps by morphisms] 
\label{Images-of-the-moment-maps-by-morphisms} Let $\X$ be
a connected diffeological space and $\omega$ be a
closed 2-form defined on $\X$. Let $\G$ be a diffeological
group and $\rho$ be a smooth action of $\G$ on $\X$,
preserving $\omega$. Let $\G'$ be another
diffeological group, and let $h : \G' \to \G$ be a smooth
homomorphism. Let $\rho' = \rho \circ h$ be the induced
action of $\G'$ on $\X$. Let us remind
that the pullback $h^* : \cG^* \to \cG'^{*}$ is a linear
smooth map.
 \begin{enumerate}
 \item Let $\Psi : \Paths(\X) \to \cG$, and $\Psi' :
\Paths(\X) \to \cG'$ be the paths moment map with respect
to the actions of  $\G$ and $\G'$ on $\X$. So, $\Psi' =
h^* \circ \Psi$.
 \item Let $\Gamma$ and $\Gamma'$ be the holonomy groups
with respect to the actions of  $\G$ and $\G'$ on $\X$.
So, $\Gamma' = \ h^*(\Gamma)$.
 \item The linear map $h^*$ projects on a smooth
homomorphism $h^*_\Gamma : \cG/\Gamma \to \cG'^{*} /
\Gamma'$, such that the following diagram commutes.
 $$
\begin{tikzcd}
   \ \cG^* \arrow[r, "h^*"] \arrow[d, "\pr"'] & \cG'^* \arrow[d, "\pr'"] \\
    \ \cG^*\!/\Gamma \arrow[r, "h^*_\Gamma"'] & \cG'^*\!/\Gamma'
\end{tikzcd}
 $$
 \item Let $\psi$ and $\psi'$ be the 2-points moment maps
with respect to the actions of  $\G$ and $\G'$. So,
$\psi' = h^*_\Gamma \circ \psi$.
 \item Let $\mu$ be a moment map relative to the action
$\rho$ of $\G$. So $\mu' =
h^*_\Gamma \circ \mu$ is a moment map relative to the
action $\rho'$ of $\G'$.
 \item Let $\mu$ be a moment map relative to the action
$\rho$ of $\G$, and let $\mu'
= \mu \circ h^*_\Gamma$ be the associated moment map
relative to the action $\rho'$ of $\G'$. So, the
associated Souriau's cocycles satisfy $\theta' = 
h^*_\Gamma \circ \theta \circ h$, summarized by the
following commutative diagram.
 $$
\begin{tikzcd}
   \  \G \arrow[d, "\theta"'] & \G' \arrow[l, "h"'] \arrow[d, "\theta'"] \\
    \ \cG^*\!/\Gamma \arrow[r, "h^*_\Gamma"'] & \cG'^*\!/\Gamma'
\end{tikzcd}
 $$
Said differently, if $\theta$ is Souriau's
cocycle associated to a moment $\mu$ of the action
$\rho$ of $\G$, and $\mu'$ is a moment
of the action $\rho'$ of $\G'$, so $\theta'$ and
$h^*_\Gamma \circ \theta \circ h$ are cohomologous.
 \end{enumerate}
{\sc Note} --- Thanks to the identification between the
space of momenta of a diffeological group and any of its
extensions by a discrete group, stated in
\art{Momenta-and-connectedness}, the moment maps of the
action of a group or the moment map of the restriction of
this action to its identity component coincide. Said
differently, the moment maps doesn't say anything about
actions of discrete groups.
 \end{article}

\begin{proof}
To avoid confusion, let us denote by $\R(p)$ and $\R'(p)$
the orbit maps of $\G$ and $\G'$ of $p \in \Paths(\X)$.
That is, $\R(p)(g) = \rho(g) \circ p$ and $\R'(p)(g) =
\rho'(g) \circ p$. So, we have, $\R'(p)(g) = \rho'(g)
\circ p = \rho(h(g)) \circ p = (\R(p) \circ h)(g))$.
Thus, $\R'(p) = \R(p) \circ h$.

1. By definition of the paths moment map, we
have $\Psi'(p) = \R'(p)^*(\eK \omega) = (\R(p) \circ
h)^*(\eK \omega) = h^*(\R(p)^*(\eK \omega)) =
h^*(\Psi(p))$. Thus, $\Psi' = h^* \circ \Psi$.

2. Since $\Gamma' = \Psi'(\Loops(\X))$, and
thanks to item 1, we have $\Gamma' = h^*(\Gamma)$.

3. The map $h_\Gamma^*$ is defined by
$\class_\Gamma(\alpha) \mapsto
\class_{\Gamma'}(h^*(\alpha))$, for all $\alpha \in
\cG^*$. If $\beta = \alpha + \gamma$, with $\gamma \in
\Gamma$, then $h^*(\beta) = h^*(\alpha) + \gamma'$, with
$\gamma' = h^*(\gamma) \in \Gamma'$ (item 2). So,
$\class_{\Gamma'}(h^*(\beta))
=\class_{\Gamma'}(h^*(\alpha))$. And, $h_\Gamma^*$ is well
defined.  Thanks to the linearity of $h^*$, $h_\Gamma^*$
is clearly an homomorphism. And, for $\cG^*\!/\Gamma$ and
$\cG'^*/\Gamma'$ equipped with the quotient diffeologies,
$h_\Gamma^*$ is naturally smooth.

4. With to the notations above, $\psi$ and $\psi'$ are
defined by, $\pr \circ \Psi = \psi \circ \bounds$ and 
$\pr' \circ \Psi' = \psi' \circ \bounds$, where $\bounds(p) = \source \times \but (p) = 
(p(0),p(1))$, with $p \in \Paths(\X)$. So, by
item 1 and 3, we have $\pr' \circ h^* \circ \Psi =
h_\Gamma^* \circ \psi \circ \pr$. That is, $\pr' \circ
\Psi' = (h_\Gamma^* \circ \psi) \circ \pr$. So,
$h_\Gamma^* \circ \psi = \psi'$.

5. Let $\mu' = h_\Gamma^* \circ \mu$, and let $x,y \in
\X$. So, $\mu'(y) - \mu'(x) = h_\Gamma^* \circ \mu(y) -
 h_\Gamma^* \circ \mu(y) =  h_\Gamma^*(\mu(y)-\mu(x)) =
h_\Gamma^*\circ \psi(y,x) = \psi'(y,x)$. So, $\mu'$ is a
moment map for the action $\rho'$ of $\G$. 

6. According to \art{Souriau-s-cocycles}, there exists a
point $x_0 \in \X$ such that, for all $g' \in \G'$,
$\theta'(g') = \psi'(x_0,\rho'(g')(x_0))$. So, thanks to
the previous items we have, $\theta'(g') =
(h_\Gamma^* \circ \psi)(x_0,\rho( h (g'))(x_0)) =
h_\Gamma^*(\psi(x_0,\rho( h (g'))(x_0))) =
h_\Gamma^*(\theta(h(g'))) = (h_\Gamma^* \circ \theta \circ
h) (g')$. Thus, we get $\theta' = h_\Gamma^* \circ \theta
\circ h$
 \end{proof}

\begin{article}[Pushing forward  moment maps]
\label{Pushing-forward- moment-maps}
Let $\X$ and $\X'$ be two connected diffeological spaces.
Let $\omega$ and $\omega'$ be two closed 2-forms defined
respectively on $\X$ and $\X'$. Let $\G$ be a
diffeological group, let $\rho$ be a smooth action of
$\G$ on $\X$, preserving $\omega$, and let $\rho'$ be a
smooth action of the same group $\G$ on $\X'$, preserving
$\omega'$. Let $f: \X \to \X'$ be a smooth map such that
$\omega = f^*(\omega')$, and  
$f \circ \rho(g) = \rho'(g) \circ f$, for all $g \in \G$.
 \begin{enumerate}
 \item Let $f_* : \Paths(\X) \to \Paths(\X')$ defined by
$f_*(p) = f \circ p$. So, the paths moment maps $\Psi$
and $\Psi'$ relative to the action $\rho$ and $\rho'$ are
related by
 $$
 \Psi = \Psi' \circ f_*,
 $$
 and the associated holonomy groups $\Gamma$ and
$\Gamma'$ satisfy
 $$
 \Gamma =\{
\Psi'(f \circ \ell) \mid \ell \in \Loops(\X) \}
\subset\Gamma'.
 $$
 \item Let $\phi: \cG^*\!/\Gamma \to
\cG^*\!/\Gamma'$ be the projection induced by the
inclusion $\Gamma \subset\Gamma'$. Let $\psi$ and $\psi'$
be the 2-points moment maps relative to the actions
$\rho$ and $\rho'$. So, for all pairs of points
$x_1$, $x_2$ of $\X$,
 $$ \psi'(f(x_1),f(x_2)) = \phi(\psi(x_1,x_2)).
 $$
 \item For every moment map $\mu$ relative to the
action $\rho$, there exists a
moment map $\mu'$ relative to the action $\rho'$, such
that
 $$
 \mu' \circ f = \phi \circ \mu.
 $$ 
 \item Let $\theta$ and $\theta'$ be two Souriau's
cocycles relative to the actions $\rho$ and $\rho'$. So,
the map $\phi \circ \theta$ is a Souriau cocycle,
cohomologous to $\theta'$. Thus, the two Souriau's
classes $\sigma$ and $\sigma'$ satisfy $\sigma' =
\phi_*(\sigma)$. Where $\phi_*$ denotes the action of
$\phi$ on cohomology, $\phi_*(\class(\theta)) =
\class(\phi \circ \theta)$.
  \end{enumerate}
 \end{article}

\begin{proof} 
1. By definition $\Psi(p) =
\hat p^*(\eK\omega)$, that is
$ \Psi(p) =  \hat p^*(\eK(f^*(\omega')))$. And thanks to
the variance of the chain-homotopy operator $\eK \circ
f^* = 
 (f_*)^* \circ \eK'$, see \art{Chain-Homotopy-operator},
we have $\Psi(p) = \hat p^* \circ
(f_*)^* (\eK'\omega') =  (f_* \circ \hat
p)^*(\eK'\omega')$. But, for all $g \in \G$, $f_* \circ
\hat p(g) = f \circ \rho(g) \circ p = \rho'(g) \circ f
\circ p = \hat p'(g)$, where $p' = f \circ p$. So,
$\Psi(p) = \hat p'^*(\eK' \omega') = \Psi'(p') =
\Psi'(f_*(p))$. Therefore, $\Psi = \Psi' \circ f_*$. Now,
by definition of the holonomy groups, $\Gamma =
\Psi(\Loops(\X)) = \Psi'(f_*(\Loops(\X)))$, and since
$f_*(\Loops(\X)) \subset \Loops(\X')$, we get $\Gamma
\subset \Gamma'$.

2. Since $\Gamma \subset \Gamma'$, the map $\phi :
\class_\Gamma(\alpha) \mapsto \class_{\Gamma'}(\alpha)$,
from $\cG^*\!/\Gamma \to \cG^*\!/\Gamma'$, is well
defined. Now, let $x'_1 = f(x_1)$ and $x'_2 = f(x_2)$,
there exists $p \in \Paths(\X)$ connecting $x_1$
to $x_2$. So the path $f_*(p)$ connects $x'_1$ to $x'_2$.
Thus, by definition of $\psi'$,
$\psi'(x'_1,x'_2) = \class_{\Gamma'}(\Psi'(p')) =
\class_{\Gamma'}(\Psi'\circ f_*(p))$, and thanks to the
first item, $\class_{\Gamma'}(\Psi'(p')) =
\class_{\Gamma'}(\Psi(p)) = \phi(\class_\Gamma(\Psi(p)))$.
But $\class_\Gamma(\Psi(p)) = \psi(x_1,x_2)$. So,
$\psi'(x'_1,x'_2) = \phi(\psi(x_1,x_2))$, that is
$\psi'(f(x_1), f(x_2)) = \psi(x_1,x_2)$.

3. According to \art{Definition-of-the-moment-maps}, for
every moment map $\mu$ there exists a point $x_0 \in \X$
and a constant $c \in \cG^*\!/\Gamma$ such that $\mu(x) =
\psi(x_0,x) + c$ . Let us define $\mu'$ by $\mu'(x') =
\psi'(x'_0,x') + c'$, where $x'_0 = f(x_0)$ and $c' =
\phi(c)$. So, thanks to the item 2, $\psi'(f(x_0),f(x)) =
\phi(\psi(x_0,x))$, so $\mu'(f(x)) = \phi(\psi(x_0,x)) +
\phi(c) = \phi(\psi(x_0,x) + c) = \phi(\mu(x))$. Thus,
$\mu'$ satisfies $\mu' \circ f = \phi \circ \mu$.

4. Let $\theta$ be a Souriau cocycle for the action
$\rho$. According to  \art{Souriau-s-cocycles}, $\theta$
is cohomologous to $\vartheta : g \mapsto \psi(x_0,
\rho(g)(x))$, where $x_0$ is some point of $\X$. So,
let $x'_0 = f(x_0)$, and $\vartheta' : g \mapsto 
\psi'(x'_0, \rho'(g)(x'_0))$. Thus, $\vartheta'(g) =
\psi'(f(x_0), \rho'(g)(f(x_0))) = \psi'(f(x_0),
f(\rho(g)(x_0))) = \phi(\psi(x_0,\rho(g)(x_0))) = \phi
\circ \vartheta(g)$. Now since all Souriau's cocycles,
with respect to a given action of $\G$, are
cohomologous, the cocycle $\theta'$ is cohomologous to
$\vartheta'$, and then cohomologous to $\phi \circ
\vartheta$, and thus to $\phi \circ \theta$. Therefore,
$\sigma' = \class(\theta') = \class(\phi \circ \theta) =
\phi_*(\class(\theta)) = \phi_*(\sigma)$.
 \end{proof}

%************************************************
\section{The universal moment maps}

The theory of moment maps developed in the previous
paragraph applies in particular to the whole group of
automorphisms $\Diff(\X,\omega)$ of a closed 2-form
$\omega$ defined on a diffeological space $\X$. We will
describe, in this paragraph, the relationships between the
\og universal\\fg moment maps and associated objects
obtained by considering the whole group $\Diff(\X,\omega)$
and the equivalent objects associated to a smooth
action of some other group $\G$ on $\X$, preserving
$\omega$.

 \begin{article}[Universal moment maps]
  \label{Universal-moment-maps}
  Let $\X$ be
a connected diffeological space and let $\omega$ be a
closed 2-form defined on $\X$. Let us remind that the
group $\Diff(\X,\omega)$ of all the automorphisms of
$(\X,\omega)$ is equipped with the functional diffeology
of group of diffeomorphisms.
Let us denote also this group by $\G_\omega$. Every
constructions defined above, the moment space, the paths
moment map, the
holonomy group, the
2-points moment map, the
moment maps,
Souriau's cocycle and Souriau's class, apply for
$\G_\omega$. We shall distinguish these objects by the
index $\omega$. So, we denote by $\cG^*_\omega$ the
momenta space of $\G_\omega$, by $\Psi_\omega
: \Paths(\X) \to \cG^*_\omega$ the paths moment map, by
$\Gamma_\omega = \Psi_\omega(\Loops(\X))$ the holonomy
group, by $\psi_\omega$ the 2-points moment map, by
$\mu_\omega$ the moment maps, by $\theta_\omega$
Souriau's cocycles, and by $\sigma_\omega$ Souriau's
class. Since $\G_\omega$ and its action on $\X$ are
uniquely defined by $\omega$, these objects depend only
on the 2-form $\omega$.

Now, let $\G$ be a diffeological group
and $\rho$ be a smooth action of $\G$ on
$\X$, preserving $\omega$. That is, a smooth homomorphism
$\rho$ from $\G$ to $\G_\omega$. The values of
the various objects $\Psi$, $\Gamma$, $\psi$, $\mu$,
$\theta$, with respect to the action $\rho$ of $\G$ on
$\X$,  depend only on $\rho^*$, $\Psi_\omega$,
$\Gamma_\omega$, $\psi_\omega$, $\mu_\omega$, and
$\theta_\omega$, as described in 
\art{Images-of-the-moment-maps-by-morphisms}. And, we
have:
 $$
 \left\{ 
 \begin{array}{rcl}
 \Psi & = & \rho^* \circ \Psi_\omega \\
 \Gamma & = & \rho^*(\Gamma_\omega) \\
 \psi & = & \rho_{\Gamma_\omega}^* \circ \psi_\omega
 \end{array} 
 \right. 
 \quad \& \quad 
 \left\{ 
 \begin{array}{rcl}
 \mu & \simeq & \rho_{\Gamma_\omega}^* \circ \mu_\omega \\
 \theta & \simeq & \rho_{\Gamma_\omega}^* \circ
 \theta_\omega \circ \rho.
 \end{array} 
 \right.
 $$
 In this
sense the objects $\G_\omega$,
$\Gamma_\omega$, $\Psi_\omega$, $\Gamma_\omega$,
$\psi_\omega$, $\mu_\omega$, $\theta_\omega$ and
$\sigma_\omega$ are {\em universal}.  So, we shall call $\Psi_\omega$ the {\em
universal paths moment map},  $\Gamma_\omega$
the {\em universal holonomy},  $\psi_\omega$
the {\em universal 2-points moment map}, 
$\mu_\omega$ the {\em universal moment maps}, 
$\theta_\omega$ {\em universal Souriau's cocycles}, and
$\sigma_\omega$ {\em universal Souriau's class} of
$\omega$. 

Note that in particular, this gives us a notion of
{\em hamiltonian spaces}, those for which,
for one reason or another, the universal holonomy is
trivial $\Gamma_\omega = \{0\}$.
 \end{article}

\begin{article}[The group of hamiltonian
diffeomorphisms] 
\label{The-group-of-hamiltonian-diffeomorphisms} Let $\X$ be
a connected diffeological space equipped with a closed
2-form $\omega$. There exists a largest connected
subgroup $\Ham(\X,\omega) \subset \Diff(\X,\omega)$ whose
action is hamiltonian, that is whose holonomy vanishes.
The elements of $\Ham(\X,\omega)$ are called {\em
hamiltonian diffeomorphisms\/}. An
action $\rho$ of a diffeological group $\G$ on
$\X$ is hamiltonian if and only if,
restricted to the identity component of $\G$, $\rho$
takes its values in $\Ham(\X,\omega)$. 

The construction of $\Ham(\X,\omega)$ is
actually given as follows. Let us denote by $\G_\omega$
the group $\Diff(\X,\omega)$ and by $\idGomega$ its
identity component. Let  $\pi :
\tidGomega \to \idGomega$ be the universal covering. Since
the universal holonomy $\Gamma_\omega$ is made up of
closed momenta, according to
\art{Closed-momenta-of-a-diffeological-group} every
$\gamma \in \Gamma_\omega$ defines a unique homomorphism
$\ek(\gamma)$ from $\tidGomega$ to $\RR$ such that
$\pi^*(\gamma) = d[\ek(\gamma)]$. Let 
 $$
 \widehat \H_\omega = \bigcap_{\gamma \in
\Gamma_\omega} \ker(\ek(\gamma)),
 $$
 and let $\widehat \H_\omega^\circ$ be its
identity component. So,
 $$
 \Ham(\X,\omega) = \pi(\widehat \H_\omega^\circ).
 $$
{\sc Note 1} --- The map $\ef : \tidGomega
\to \Hom(\pi_1(\X),\RR)$ defined by $\ef(\tilde g) = [
\tau \mapsto \ek(\gamma)(\tilde g)]$, with $\tau =
\class(\ell)$ and  $\gamma = \Psi(\ell)$, is an
homomorphism. And,
$\widehat \H_\omega = \ker(\ef)$. In classical symplectic
geometry, the image $\eF = \Values(\ef)$ is called, by
some authors, the {\em group of flux} of $\omega$.

{\sc Note 2} --- Since to be hamiltonian for a
group of automorphisms depends only on its connected
component, see \art{Momenta-and-connectedness} and
\art{Momenta-of-covering-of-diffeological-groups}, any
extension $\H \subset \Diff(\X,\omega)$ of
$\Ham(\X,\omega)$, such that $\H / \Ham(\X,\omega)$ is
discrete\footnote{Where $\H$ and $\Ham(\X,\omega)$ are
equipped with the subset diffeology of the functional
diffeology of $\Diff(\X,\omega)$.}, is hamiltonian. In
particular $\pi(\widehat\H_\omega)$ is hamiltonian, or
if $\Gamma_\omega = \{0\}$ then $\Diff(\X,\omega)$ is
hamiltonian, and  $\Ham(\X,\omega)$ is the identity
component of $\Diff(\X,\omega)$.

{\sc Note 3} --- Let us choose a point $x_0$ in $\X$ and
let $\mu$ be the moment map with respect to the group
$\Ham(\X,\omega)$, defined by $\mu(x_0) = 0$. Let $f$ be 
a 1-parameter subgroup of $\Ham(\X,\omega)$. Applying
the note of \art{Evaluation-of-the-paths-moment-map}, we
get for all $x \in \X$ the expression of $\mu(x)$,
evaluated on $f$
 $$
 \mu(x)(f) = h_f(x) \times dt \qmbox{with} h_f(x) =
- \int_{x_0}^x i_f(\omega).
 $$
 The smooth function $h_f : \X \to \RR$ is the {\em
hamiltonian}  (vanishing
at $x_0$) of the 1-parameter subgroup $f$.
 \end{article}

\begin{proof}
Let us remark, first of all, that for every
$\gamma \in \Gamma_\omega$, $\pi^*(\gamma) \restriction
\widehat\H_\omega = 0$. Indeed, $\pi^*(\gamma)
\restriction \widehat \H_\omega = d[\ek(\gamma)]
\restriction \widehat \H_\omega =
d[\ek(\gamma)\restriction \widehat \H_\omega]$. But, by
the very definition of $\widehat \H_\omega$, 
$\ek(\gamma)\restriction \widehat \H_\omega = 0$, so 
$\pi^*(\gamma) \restriction \widehat \H_\omega = 0$.


a) Let us prove that the holonomy of $\Ham(\X,\omega)$ is
trivial. Let $\H_\omega = \pi(\widehat\H_\omega)$ and let
us denote by $j_{\H_\omega}$ the inclusion $\H_\omega
\subset \G_\omega$, by $j_{\widehat \H_\omega}$ the
inclusion $\widehat \H_\omega \subset \tidGomega$, and by
$\pi_{\H_\omega} : \widehat \H_\omega \to \H_\omega$ the
projection. So, $ j_{\H_\omega} \circ \pi_{\H_\omega} =
\pi \circ j_{\widehat \H_\omega}$. Let
$\Gamma_{\H_\omega}$ be the holonomy of $\H_\omega$, so
according to
\art{Images-of-the-moment-maps-by-morphisms},
$\Gamma_{\H_\omega} = j_{\H_\omega}^*(\Gamma_\omega)$.
Thus, for every $\bar \gamma \in \Gamma_{\H_\omega}$
there exists $\gamma \in \Gamma_\omega$ such that $\bar
\gamma = \gamma \restriction \H_\omega =
j_{\H_\omega}^*(\gamma)$. So, for all $\bar \gamma \in
\Gamma_{\H_\omega}$, $\pi_{\H_\omega}^* (\bar \gamma) =
\pi_{\H_\omega}^* (j_{\H_\omega}^*(\gamma)) =
(j_{\H_\omega} \circ \pi_{\H_\omega})^*(\gamma) = (\pi
\circ j_{\widehat \H_\omega})^*(\gamma) = j_{\widehat
\H_\omega}^*(\pi^*(\gamma)) = \pi^*(\gamma) \restriction
\widehat \H_\omega$. But, $\pi^*(\gamma) \restriction
\widehat \H_\omega = 0$, so $\pi_{\H_\omega}^*(\bar
\gamma) = 0$. And since $\pi_{\H_\omega}$ is a
subduction, $\bar \gamma = 0$. Therefore, the holonomy of
$\H_\omega$ vanishes, $\Gamma_{\H_\omega} = \{0\}$.

b) Let us prove now that every connected subgroup $\H
\subset \G_\omega$ whose action is hamiltonian is a
subgroup of $\Ham(\X,\omega)$. Let $\widehat \H =
\pi^{-1}(\H)$ and $\widehat\H^\circ$ be its identity
component. Let $j_\H$ be the inclusion $\H \subset
\G_\omega$, and  $j_{\widehat \H^\circ}$ be the
inclusion $\widehat \H^\circ \subset \tidGomega$. Let
$\pi_{\H} = \pi \restriction \widehat \H^\circ$. So,
$j_\H \circ \pi_\H = \pi \circ j_{\widehat \H^\circ}$. Let
$\Gamma_{\H}$ be the holonomy of $\H$. Since $\Gamma_{\H}
= j_{\H}^*(\Gamma_\omega)$ and $\Gamma_\H = \{0\}$, for
all $\gamma \in \Gamma_\omega$, $j_\H^*(\gamma) = 0$.
Thus, for all $\gamma \in \Gamma_\omega$,
$\pi_\H^*(j_\H^*(\gamma)) = 0$. But, $
\pi_\H^*(j_\H^*(\gamma)) = (j_\H \circ \pi_\H)^*(\gamma)
= (\pi \circ j_{\widehat \H^\circ})^*(\gamma) =
j_{\widehat \H^\circ}^*(\pi^*(\gamma)) = \pi^*(\gamma)
\restriction \widehat \H^\circ$. So, for all $\gamma \in
\Gamma_\omega$, $\pi^*(\gamma) \restriction \widehat
\H^\circ = 0$. But $\pi^*(\gamma) = d[\ek(\gamma)]$, hence
$d[\ek(\gamma) \restriction \widehat \H^\circ] = 0$. So,
since $\H^\circ$ is connected, $\ek(\gamma)$ is constant
on $\widehat \H^\circ$, and since $\ek(\gamma)$ is an
homomorphism to $\RR$, this constant is necessarily $0$.
Thus, $\widehat \H^\circ \subset \ker(\ek(\gamma))$, for
all $\gamma \in \Gamma_\omega$, that is $\widehat \H^\circ
\subset \widehat \H_\omega$. But, since $\H^\circ$ is
connected $\widehat \H^\circ
\subset \widehat \H_\omega^\circ \subset \H_\omega$ and
thus $\H = \pi(\widehat \H^\circ) \subset
\Ham(\X,\omega) = \pi(\widehat \H_\omega^\circ)$.
 \end{proof}

\begin{article}[Time-dependent hamiltonian] 
\label{Time-dependent-hamiltonian} 
Let $\X$ be a connected diffeological space and $\omega$
be a closed 2-form defined on $\X$. A diffeomorphism $f$
of $\X$ belongs to $\Ham(\X,\omega)$ if and only if:
 \begin{enumerate}
 \item There exists a smooth path $t \mapsto f_t$ in
$\Diff(\X,\omega)$ connecting the identity $\id_\M = f_0$
to $f = f_1$.
 \item There exists a smooth path $t \mapsto \Phi_t$
in $\Cinfty(\X,\RR)$ such that for all $t$,
 $$
 i_{\F_t}(\omega) = - d\Phi_t \qmbox{with} \F_t : s
\mapsto f_t^{-1} \circ f_{t+s}.
 $$
 \end{enumerate}
 According to the tradition of classical symplectic
geometry, the path $t \mapsto \Phi_t$ can be
called a {\em time-dependent
hamiltonian} of the 1-parameter family of hamiltonian
diffeomorphisms $t \mapsto f_t$.
 \end{article}

\begin{proof}
Let us assume first that $f$ satisfies the condition
above. That is, there exists a smooth path $t \mapsto
f_t$ in $\Diff(\X,\omega)$ such that $f_0 = \id_\M$, $f_1
= f$, and there exists a smooth path $t \mapsto \Phi_t$
in $\Cinfty(\X,\RR)$ such that $i_{\F_t}(\omega) = -
d\Phi_t$ for all $t$ where $\F_t : s \mapsto f_t^{-1}
\circ f_{t+s}$.  Let us remind that $\Ham(\X,
\omega) = \pi(\widehat \H_\omega^\circ)$, with $\widehat
\H_\omega^\circ$ the identity component of $\widehat
\H_\omega = \cap_{\gamma \in \Gamma_\omega}
\ker(\ek(\gamma))$, and let $\tilde f \in
\G_\omega^\circ$ be the homotopy class of the path $t
\mapsto f_t$, notations of
\art{The-group-of-hamiltonian-diffeomorphisms}. So, let
$\gamma \in \Gamma_\omega$, that is $\gamma =
\Psi_\omega(\ell)$ where $\ell$ is some loop in $\M$. By
definition, we have
 $$
 \ek(\gamma)(\tilde f) = \int_{[t \mapsto f_t]} \gamma
=  \int_{[t \mapsto f_t]} \Psi_\omega(\ell) = \int_0^1
\Psi_\omega(\ell)([t \mapsto f_t])_t(1) dt
 $$
 Now, thanks to \art{Evaluation-of-the-paths-moment-map}
$\clubsuit$, we have 
 $$
 \Psi_\omega(\ell)([t \mapsto f_t])_t(1) = - \int_\ell
i_{\F_t}(\omega) = \int_\ell d\Phi_t =
\int_{\partial \ell} \Phi_t = 0.
 $$
So, $\ek(\gamma)(\tilde f) = 0$ for all $\gamma
\in \Gamma_\omega$ and $\tilde f$ belongs to $\widehat
\H_\omega$ and more precisely in the identity component
of $\widehat \H_\omega$. Therefore $f \in
\Ham(\X,\omega)$.

Conversely, let $f \in \Ham(\M,
\omega)$. Since $\Ham(\M,\omega)$ is connected there
exists a path $t \mapsto f_t$ in $\Ham(\M, \omega)$
connecting $\id_\M$ to $f$. And, since the projection 
$\pi \restriction \widehat \H_\omega^\circ : \widehat
\H_\omega^\circ \to \Ham(\M,\omega)$ is a covering, there
exists a (unique) lifting $t \mapsto \tilde f_t$ of $t
\mapsto f$ in $\widehat \H_\omega^\circ$, along $\pi
\restriction \widehat \H_\omega^\circ$, such that $\tilde
f_0 = \id_{\widehat\H_\omega}$. 
 This lifting is actually given by $\tilde f_t =
\class(p_t)$, with $p_t : s \mapsto f_{st}$. So, for
all $t$, $\tilde f_t \in \widehat \H_\omega^\circ \subset
\widehat \H_\omega = \cap_{\gamma \in \Gamma_\omega}
\ker(\ek(\gamma))$. That is, for all $\gamma \in
\Gamma_\omega$, $\ek(\gamma)(\tilde f_t) = 0$, or in
other words, for all $\ell \in \Loops(\M)$,
$\ek(\Psi_\omega(\ell))(\tilde f_t) = 0$. But, 
 \begin{eqnarray*}
 \ek(\Psi_\omega(\ell))(\tilde
f_t) &=& \int_{p_t} \Psi_\omega(\ell) \\
&=& \int_0^1 \Psi_\omega(\ell)(s \mapsto f_{st})_s(1) ds
\\
&=& \int_0^1 \Psi_\omega(\ell)(s \mapsto st \mapsto
f_{st})_s(1) ds \\
&=& \int_0^1 [\Psi_\omega(\ell)(u \mapsto f_u)]_{u=st}
\bigg({d st \over ds}\bigg) ds  \\
&=& \int_0^t \Psi_\omega(\ell)(u \mapsto f_u)_u(1) du.
 \end{eqnarray*}
 So, 
 $$
 \ek(\Psi_\omega(\ell))(\tilde
f_t) = 0 \quad \Rightarrow \quad {1 \over t} \int_0^t
\Psi_\omega(\ell)(u \mapsto f_u)_u(1) du = 0, 
 $$
 and taking the limit for $t \to 0$ we get, 
 $$
 \ek(\Psi_\omega(\ell))(\tilde
f_t) = 0 \quad \Rightarrow \quad \Psi_\omega(\ell)(t
\mapsto f_t)_t(1) = 0.
 $$
 But, $\Psi_\omega(\ell)([t \mapsto f_t])_t(1) =
- \displaystyle\int_\ell i_{\F_t}(\omega)$, see
\art{Evaluation-of-the-paths-moment-map} $\clubsuit$.
So, for all $t$ and all $\ell \in \Loops(\X)$
 $$
  \int_\ell
i_{\F_t}(\omega) = 0.
 $$ 
 But $\F_t$ is a path in $\Diff(\X,\omega)$ centered at
the identity, so the Lie derivative of $\omega$ by $\F_t$
vanishes, and applying the Cartan formula given in
\art{Differential-forms}, we get
 $$
   \DLie_{\F_t} \omega =  0 \quad
\Rightarrow \quad d[i_{\F_t}(\omega)] + i_{\F_t}(d\omega)
= d[i_{\F_t}(\omega)] = 0.
 $$
 So, the 1-form $i_{\F_t}(\omega)$ is closed and its
integral on any loop $\ell$ of $\X$ vanishes, therefore
$i_{\F_t}(\omega)$ is exact \cite{Piz05}. Thus, for all
real $t$ there exists a real function $\Phi_t \in
\Cinfty(\X,\RR)$ such that  $i_{\F_t}(\omega) = -
d\Phi_t$. The fact that $t \mapsto \Phi_t$ is a smooth map
from $\RR$ to $\Cinfty(\X,\RR)$, for the
functional diffeology, is a consequence of the explicit
construction of the function $\Phi_t$ by integration
along the paths, see \cite{Piz05}.
 \end{proof}

%************************************************
\section{About Symplectic Manifolds}

The case of symplectic manifolds $(\M, \omega)$ deserves a
special care: any universal moment map $\mu_\omega$ is
injective and therefore identifies $\M$ with a coadjoint
orbit --- in the general sense given in
\art{Affine-coadjoint-actions-and-orbits} --- of
$\Diff(\M,\omega)$. 

\begin{article}[Value of the moment
maps for manifolds] 
\label{Value-of-the-moment-maps-for-manifolds} 
Let $\M$ be a connected manifold equipped with a closed
2-form $\omega$. In this context, the paths moment map
$\Psi_\omega$ takes a special expression. Let $p$ be a
path in $\M$, let $\F : \U \to \Diff(\M,\omega)$ be a
$n$-plot, we have
\begin{equation}
\renewcommand{\theequation}{$\diamondsuit$}
 \Psi_\omega(p)(\F)_r(\delta r) = \int_0^1
\omega_{p(t)}(\dot p(t), \delta p(t)) \ dt 
 \end{equation}
 for all $r \in \U$ and $\delta r \in \RR^n$, where
$\delta p$ is the lifting in the tangent space $\T\M$ of
the path $p$, defined by 
 \begin{equation}\renewcommand{\theequation}{$\heartsuit$}
 \delta p(t) =
[\D(\F(r))(p(t))]^{-1} {\partial \F(r)(p(t)) \over
\partial r} (\delta r).
 \end{equation}
  \end{article}

\begin{proof}
By definition, $\Psi(p)(\F) = \hat p^*(\eK \omega)(\F) =
\eK \omega (\hat p \circ \F)$. The explicit expression
of the operator $\eK$ given in
\art{Chain-Homotopy-operator}, applied to the plot $\hat p
\circ \F : r \mapsto \F(r) \circ p$ of $\Paths(\X)$, gives
 $$
 (\K\omega)(\hat p \circ
\F)_r(\delta r) = \int_0^1 \omega \left[ \mymatrix{s \cr
u} \mapsto (\hat p \circ \F)(u)(s + t) \right]_{s=0
\choose u=r}\mymatrix{1 \cr 0} \mymatrix{0 \cr
\delta r} dt.
 $$
 But $(\hat p \circ \F)(u)(s + t)
= \F(u)(p(s+t))$, let us denote temporarily by $\Phi_t$
the plot $(s,u) \mapsto \F(u)(p(s+t))$, so $\F(u)(p(s+t))$
writes $\Phi_t(s,u)$. Now, let us denote by $\cI$ the
integrand  of the
right term of this expression. We have,
 \begin{eqnarray*}
\cI & = & \omega \left[ \mymatrix{s \cr
u} \mapsto \Phi_t(s,u) \right]_{s=0
\choose u=r}\mymatrix{1 \cr 0} \mymatrix{0 \cr
\delta r}\\
 & = & \Phi_t^*(\omega)_{0
\choose r} \mymatrix{1 \cr 0} \mymatrix{0 \cr \delta
r} \\ & = & \omega_{\Phi_t{{0
\choose r}}} \left(\D(\Phi_t)_{0 \choose
r} \mymatrix{1 \cr
0},\D(\Phi_t)_{0 \choose
r} \mymatrix{0 \cr
\delta v}\right) \\
& =& \omega_{\F(r)(p(t))} \left({\partial
 \over \partial s} \bigg\{ \F(r)(p(s+t))\bigg\}_{s=0},
{\partial  \over \partial r} \bigg\{\F(r)(p(t))
\bigg\} (\delta r)\right).
 \end{eqnarray*}
 But, 
 $$
 {\partial
 \over \partial s} \bigg\{ \F(r)(p(s+t))\bigg\}_{s=0} =
\D(\F(r))(p(t))\bigg({\partial p(s+t)
 \over \partial s}\bigg|_{s=0}\bigg) =
\D(\F(r))(p(t))(\dot p(t)).
 $$
 So, using this last expression and the fact that $\F$ is
a plot of $\Diff(\M,\omega)$, that is for all $r$ in $\U$,
$\F(r)^*\omega = \omega$, we have
 \begin{eqnarray*}
 \omega \left[\mymatrix{s \cr u} \mapsto
\Phi_t(s,u)\right]_{s=0 \choose u=r} & = &
\omega_{\F(r)(p(t))} \bigg(\D(\F(r))(p(t))(\dot p(t)) ,
{\partial \F(r)(p(t)) \over \partial r} (\delta r)\bigg)
\\
 &=& \omega_{p(t)}
\bigg(\dot p(t) , [\D(\F(r))(p(t))]^{-1}{\partial
\F(r)(p(t)) \over \partial r} (\delta r)\bigg) \\
&=& \omega_{p(t)}
(\dot p(t) , \delta p(t)).
 \end{eqnarray*}
 Therefore, $\Psi_\omega(p)(\F)_r(\delta
r) = \eK \omega (\hat p \circ \F)_r(\delta r) =
\displaystyle{\int_0^1} \omega_{p(t)}(\dot p(t), \delta
p(t)) \ dt$. \end{proof}

 \begin{article}[The paths moment
maps for symplectic manifolds] 
\label{The-paths-moment-maps-for-symplectic-manifolds} 
Let $\M$ be a Hau\-sdorff manifold and $\omega$ be a
non degenerate closed 2-form defined on $\M$. Let $m_0$
and $m_1$ be two points of $\M$ connected by a path $p$.
Let $f \in \Cinfty(\M,\RR)$ with compact support. Let
$\F$ be the exponential of the symplectic
gradient\footnote{Let us remind that the symplectic
gradient is defined by $\omega(\grad_\omega(f),\cdot) =
-df$.} $\grad_\omega(f)$, $\F$ is a 1-plot of
$\Diff(\M,\omega)$, and precisely a 1-parameter
homomorphism. So, the universal paths moment map
$\Psi_\omega$, computed at the path $p$, evaluated to the
1-plot $\F$, is the constant 1-form of $\RR$,
 $$
 \Psi_\omega(p)(\F) = [f(m_1) -f(m_0)] \times dt
\qmbox{with} \F :t \mapsto e^{t \grad_\omega(f)},
 $$
 and $dt$  the standard 1-form of $\RR$. Note that we are
in the special case where $\F$ is actually a
1-parameter homomorphism of
$\Ham(\M,\omega) \subset \Diff(\M,\omega)$, and the
function $f$ is one {\em hamiltonian} of $\F$.
 \end{article}

\begin{proof}
Let us remark that, in our case, the lift
$\delta p$ defined by  $\heartsuit$ of
\art{Value-of-the-moment-maps-for-manifolds} writes
simply 
 $$
\delta p(t) = [\D(e^{r \xi})(p(t))]^{-1}
{\partial e^{r \xi}(p(t)) \over \partial r} (\delta r) =
\xi(p(t)) \times \delta r \qmbox{with} \xi =
\grad_\omega(f),
 $$
where $r$ and $\delta r$ are reals. So, the expression
$\diamondsuit$ of
\art{Value-of-the-moment-maps-for-manifolds} becomes
 \begin{eqnarray*}
 \Psi_\omega(p)(\F)_r(\delta r) & = & \int_0^1
\omega_{p(t)}(\dot p(t), \xi(p(t)) \ dt \times \delta r \\
 & = & \int_0^1
\omega_{p(t)}(\dot p(t), \grad_\omega(f)(p(t))
\ dt \times \delta r\\
 & = & \int_0^1 df\bigg({d p(t) \over dt} \bigg) dt
\times \delta r\\
 & = & [f(p(1)) - f(p(0))] \times \delta r
 \end{eqnarray*}
 That is, $\Psi_\omega(p)(\F) = [f(m_1)
-f(m_0)] \times dt$.
 \end{proof}

\begin{article}[Moment maps for symplectic manifolds] 
\label{Moment-maps-for-symplectic-manifolds} 
Let $\M$ be a connected Hausdorff
manifold and
$\omega$ be a closed 2-form defined on $\M$. The form
$\omega$ is non-degenerated, that is symplectic, if and
only if  
 \begin{enumerate} 
 \item The manifold $\M$ is an
homogeneous space of 
$\Diff(\M,\omega)$.
  \item Any one of its universal moment maps $\mu_\omega
: \M \to \cG^*_\omega/\Gamma_\omega$ is injective. 
 \end{enumerate}
 Note that, if one of the universal moment maps
$\mu_\omega$ is injective so are every ones. Note also
that, if $\omega$ is symplectic, then the image of the
moment map,  $\cO_\omega = \mu_\omega(\M) \in
\cG^*_\omega/\Gamma_\omega$, is a
$(\Gamma_\omega,\theta_\omega)$-coadjoint orbit of
$\Diff(\M,\omega)$. And, $\mu_\omega$ identifies $\M$ to
$\cO_\omega$, where $\cO_\omega$ is equipped with the
quotient diffeology of $\Diff(\M,\omega)$. In other
words, every symplectic manifold is a coadjoint orbit.

{\sc Remark} --- Let us consider the example $\M = \RR^2$
and $\omega = (x^2 + y^2)\ dx \wedge dy$. This form is
non degenerate on $\RR^2 - \{0\}$,
but degenerates  at the point $(0,0)$. Thus, $(0,0)$ is an
orbit of the group $\Diff(\X,\omega)$, and actually
$\RR^2 -\{0\}$ is the other orbit. Since $\RR^2$ is
contractible the holonomy $\Gamma_\omega$ is trivial and
the universal moment map $\mu_\omega$ defined by
$\mu_\omega(0,0) = 0_{\cG_\omega^*}$ is equivariant. Now,
$\mu_\omega$ is injective, and $\omega$ is not
symplectic. So, the hypothesis of
transitivity of $\Diff(\M,\omega)$ on $\M$
 is not superfluous is this proposition.
 \end{article}

\begin{proof}
Let us assume first that $\omega$ is nondegenerate, that
is symplectic. So, the group $\Diff(\M,\omega)$ is
transitive on $\M$ \cite{Boo69}. Moreover, for every $m
\in \M$, the orbit map $\hat m : \varphi \mapsto
\varphi(m)$ is a subduction \cite{Don84}. So, the image of moment
moment map $\mu_\omega$ is one orbit $\cO_\omega$ of the
affine coadjoint action of $\G_\omega$ on
$\cG^*_\omega/\Gamma_\omega$, associated to the cocycle
$\theta_\omega$. Thus, for the
orbit $\cO_\omega$ equipped with the quotient diffeology
of $\G_\omega$, the moment map $\mu_\omega$ is a
subduction.

 Now, let $m_0$ and $m_1$ two points of $\M$ such
that 
$\mu_\omega(m_0) = \mu_\omega(m_1)$, that is
$\psi_\omega(m_0,m_1) = \mu_\omega(m_1) -
\mu_\omega(m_0)   = 0$. Let $p \in
\Paths(\M)$ such that $p(0)= m_0$ and $p(1) = m_1$. Thus,
$\psi_\omega(m_0,m_1) = 0$ is equivalent to
$\Psi_\omega(p) = \Psi_\omega(\ell)$, where $\ell$ is
some loop of $\M$, we can choose $\ell(0)=\ell(1)=m_0$.
Now, let us assume that $m_0 \neq m_1$. Since $\M$ is
Hausdorff there exists a smooth real
function $f \in \Cinfty(\M, \RR)$, with compact support,
such that $f(m_0) = 0$ and $f(m_1) = 1$. Let us denote by
$\xi$ the symplectic gradient field associated to $f$ and
by $\F$ the exponential of $\xi$. Thanks to
\art{The-paths-moment-maps-for-symplectic-manifolds}, on
one hand we have $\Psi(p)(\F) = [f(m_1) - f(m_0)] dt =
dt$, and on the other hand $\Psi_\omega(\ell)(\F) =
[f(m_0) - f(m_0)] dt = 0$. But $dt \neq 0$, thus
$\psi_\omega(m_0,m_1) \neq 0$, and the moment map
$\mu_\omega$ is injective. Therefore, $\mu_\omega$ is an
injective subduction on $\cO_\omega$, that is a
diffeomorphism.

Conversely, let us assume that $\M$ is an
homogeneous space of $\Diff(\M,\omega)$ and
$\mu_\omega$ is injective. Let us notice first that, since
$\Diff(\M,\omega)$ is transitive, the rank of $\omega$ is
constant. In other words, $\dim \ker \omega = \const$.
Now, let us assume that $\omega$ is degenerated, that is
$\dim(\ker \omega) \geq 1$. Since $m \mapsto \ker
\omega_m$ is a smooth foliation, for any point $m$ of
$\M$ there exists a smooth path $p$ of $\M$ such that
$p(0) = m$ and for $t$ belonging to a small interval
around $0 \in \RR$, $\dot p (t) \neq 0$ and $\dot p(t)
\in \ker \omega_{p(t)}$ for all $t$ in this interval. So,
we can re-parametrize the path $p$ and assume now that $p$
is defined on the whole $\RR$ and satisfies  $p(0) = m$,
$p(1) = m'$ with $m \neq m'$, and $\dot p(t) \in \ker
\omega_{p(t)}$ for all $t$. Now,  since $\dot
p(t) \in \ker \omega_{p(t)}$ for all $t$, using the
expression $\diamondsuit$ given in
\art{Value-of-the-moment-maps-for-manifolds}, we get
$\Psi_\omega(p) = 0_{\cG^*_\omega}$ and thus
$\mu_\omega(m) = \mu_\omega(m')$. But $m \neq m'$ and we
have assumed that $\mu_\omega$ is injective. So the
kernel of $\omega$ is reduced to $\{0\}$, $\omega$ is
nondegenerate, that is symplectic.

Let us finish by proving the
remark. That is, the universal moment map $\mu_\omega$ of
$\omega = (x^2 + y^2)\ dx \wedge dy$ is injective. First
of all $\mu_\omega(0,0) = 0_{\cG^*}$. Now if $z=(x,y)$
and $z' =(x',y')$ are two different points of $\RR^2$ and
different from $(0,0)$, there is a smooth function with
compact support contained in a small ball not containing
$(0,0)$ nor $z$ and such that $f(z')=1$. So the
1-parameter group generated by $\grad_\omega(f)$ belongs
to $\Diff(\RR^2,\omega)$, and then a similar argument as
the one of the proof above shows that $\mu_\omega(z) \neq
\mu_\omega(z')$. Now it remains to prove that if $z \neq
(0,0)$, $\mu_\omega(z) \neq 0_{\cG^*}$. Let us consider
$p(t) = tz$ and $\F(r)$ be the positive rotation of angle
$2\pi r$, where $r \in \RR$. The application of the
formula $\diamondsuit$  of
\art{Value-of-the-moment-maps-for-manifolds}, computed at
the point $r=0$ and applied to the vector $\delta r =1$
gives $(2\pi/3) (x^2+y^2)^2$ which is not zero. So, the
moment map $\mu_\omega$ is injective.
 \end{proof}

\begin{article}[Restriction to hamiltonian
diffeomorphisms] 
\label{Restriction-to-hamiltonian-diffeomorphisms} Let
$(\M,\omega)$ be a connected Hausdorff symplectic
manifold. Let $\Ham(\M,\omega)$ be the group of
hamiltonian diffeomorphisms, and let $\cH^*_\omega$
be the space of its momenta. Let $\mu_\omega^\star : \M
\to \cH^*_\omega$ be any moment map associated to the
action of $\Ham(\M,\omega)$, and let
$\theta_\omega^\star$ be the associated Souriau cocycle.
So, $\mu_\omega^\star$ is injective, and identifies $\M$
to a $\theta_\omega^\star$-coadjoint orbit of
$\Ham(\M,\omega)$ in $\cH_\omega$.
 \end{article}

\begin{proof}
It is known also that the group
$\Ham(\M,\omega)$ acts transitively on $\M$ \cite{Boo69}.
With respect to that group, and by construction, the
holonomy is trivial: the associated paths moment map
$\Psi_\omega^\star$ and the moment maps $\mu_\omega^\star$
take their values in the space $\cH^*_\omega$. Let $j :
\Ham(\M,\omega) \to \Diff(\M,\omega)$ be the inclusion,
so the universal  holonomy $\Gamma_\omega$ is in the
kernel of $j^*$, and we get a natural mapping
$j^*_{\Gamma_\omega} : \cG^*_\omega/\Gamma_\omega \to
\cH^*_\omega$. Now, the paths moment maps satisfy
$\Psi_\omega^\star = j^*_{\Gamma_\omega} \circ
\Psi_\omega$, and $\mu_\omega^\star = j^*_{\Gamma_\omega}
\circ \mu_\omega$, see \art{Universal-moment-maps}.
Then, since the
\art{The-paths-moment-maps-for-symplectic-manifolds}
involves only plots of $\Ham(\X,\omega)$, the first part
of the proof of
\art{Moment-maps-for-symplectic-manifolds} applies
{\em mutatis mutandis} to the hamiltonian case and we
deduce that the moment maps $\mu_\omega^\star$ are
injective and identify $\M$ with some
$\theta_\omega^\star$-coadjoint orbits of
$\Ham(\M,\omega)$.
 \end{proof}


\begin{article}[Hamiltonian
diffeomorphisms of symplectic manifolds] 
\label{Hamiltonian-diffeomorphisms-on-symplectic-manifolds}
Let $(\M,\omega)$ be a con\-nected Hausdorff symplectic
manifold. According to Banayaga, a
diffeomorphism $f$ is said to be {\em hamiltonian\/} if
it can be connected to the identity $\id_\M$ by a smooth
path  $t \mapsto f_t$ in $\Diff(\M, \omega)$ such that
 $$
 \omega(\dot f_t, \cdot) = d\phi_t \qmbox{with} \dot
f_t(x) = {d \over ds}\bigg\{f_s
\circ f_t^{-1}(x)\bigg\}_{s=t},
 $$
where $(t,x) \mapsto \phi_t(x)$ is a smooth real function,
see \cite{Ban78}. If,
according to this definition, $f$ is hamiltonian then it
is an element of $\Ham(\M, \omega)$, as defined in
\art{The-group-of-hamiltonian-diffeomorphisms}.
Conversely, any element $f$ of $\Ham(\M, \omega)$
satisfies the condition above. So, the definition of
hamiltonian diffeomorphisms given in
\art{The-group-of-hamiltonian-diffeomorphisms} is a
faithful generalization of the classical definition for
symplectic manifolds. Note that the technical requirement
of compacity of the original definition ({\em
op cit}) doesn't play any role in this characterization of
hamiltonian diffeomorphisms.
 \end{article}

\begin{proof}
This proposition is a direct consequence of the
general statement given in
\art{Time-dependent-hamiltonian} and the
following comparison between the above 1-parameter family
of vector fields $\dot f_t$ and the family
$\F_t$ of the \art{Time-dependent-hamiltonian}.

 Since $f_{t'} \circ f_t^{-1} = f_t \circ (f_t^{-1}
\circ f_{t'}) \circ f_t^{-1}$, the vector fields $\dot
f_t$ and $\F_t$ are conjugated by $f_t$, precisely:
 $$
 \dot f_t = (f_t)_*(\F_t) \qmbox{or} \dot f_t(x) =
\D(f_t)(f_t^{-1}(x))(\F_t(f_t^{-1}(x))).
 $$
 This implies in particular that if the vector field
$\dot f_t$ satisfies Banyaga's condition for the function
$\phi_t$ then the vector field $\F_t$ satisfies Banyaga's
condition for the function $\Phi_t = -\phi_t \circ
f_t$, and conversely. That is:
 $$
  \omega(\dot f_t, \cdot) = d\phi_t \quad
\Leftrightarrow \quad \omega(\F_t, \cdot) = - d\Phi_t
\qmbox{with} \Phi_t = - \phi_t \circ f_t.
 $$
Indeed, let $x \in \M$, $x' = f_t(x)$, $\delta x \in \T_x
\M$, and $\delta x' = \D(f_t)(x)(\delta x)$, we have:
 \begin{eqnarray*}
 \omega_{x'} (\dot f_t(x'), \delta x') & =&
[d\phi_t]_{x'}(\delta x') \\
 \omega_{f_t(x)}(\dot f_t (f_t(x)), \D(f_t)(x)(\delta x)) 
& =& [d\phi_t]_{f_t(x)}(\D(f_t)(x)(\delta x))  \\
 \omega_{f_t(x)}(\D(f_t)(x)(\F_t(x)), \D(f_t)(x)(\delta
x))  & =&  [f_t^*(d\phi_t)]_x(\delta x) \\
{[f_t^*(\omega)]}_x(\F_t(x), \delta x) &=&
d[f_t^*(\phi_t)]_x(\delta x) \\
 \omega_x(\F_t(x),\delta x) &
= & d[\phi_t \circ f_t]_x(\delta x).
 \end{eqnarray*}
 Thus, we get $\Phi_t = - \phi_t \circ f_t$.  \end{proof}

%************************************************
\section{The homogeneous case}

As it is suggested by 
\art{Moment-maps-for-symplectic-manifolds}, the case of
an homogeneous action of a diffeological group
$\G$ on a space $\X$, preserving a closed 2-form $\omega$,
deserves a special attention. 

\begin{article}[The homogeneous case] 
\label{The-homogeneous-case} Let $\X$ be a connected
diffeological space equipped with a closed 2-form
$\omega$. Let $\rho$ be a smooth action of a
diffeological group $\G$ on $\X$, preserving $\omega$.
Let us assume that $\X$ is homogeneous for this action,
see \art{Smooth-actions-of-a-diffeological-group}. Let
$\Gamma$ be the holonomy of the action $\rho$,
let $\mu$ be a moment, and let $\theta$ be the cocycle
associated to $\mu$. Let $x_0$ be any point of $\X$,
and let $\mu_0 = \mu(x_0)$. Let
$\Stab_{\Ad_*^{\Gamma,\theta}}(\mu_0)$ be the stabilizer of
$\mu_0$ for the affine coadjoint action of $\G$ on
$\cG^*\!/\Gamma$. Thanks to the
equivariance of the moment map $\mu$, with respect to the
affine coadjoint action of $\G$ on $\cG^*\!/\Gamma$, $\mu
\circ \rho(g) = \Ad_*^{\Gamma,\theta}(g) \circ \mu$, the
image $\cO = \mu(\X)$ is a $(\Gamma,\theta)$-orbit of
$\G$. Let us equip $\cO$ with the quotient diffeology of
$\G$, such that $\cO \simeq
\G/\Stab_{\Ad_*^{\Gamma,\theta}}(\mu_0)$. So, the
orbit map $\hat x_0 : \G \to \X$ is a principal
fibration with structure group $\Stab_\rho(x_0)$, the
orbit map $\hat \mu_0 : \G \to \cO$ is a principal
fibration with structure group
$\Stab_{\Ad_*^{\Gamma,\theta}}(\mu_0)$, and
$\Stab_\rho(x_0) \subset
\Stab_{\Ad_*^{\Gamma,\theta}}(\mu_0)$. So, the moment map
$\mu : \X \to \cO$ is a fibration with fiber, the
homogeneous space
$\Stab_{\Ad_*^{\Gamma,\theta}}(\mu_0)/\Stab_\rho(x_0)$.
\delete{
Note that the fibration $\mu : \X \to \cO$ is a covering
if and only if $\Stab_\rho(x_0)$ is an union of connected
components of $\Ad_*^{\Gamma,\theta}(\mu_0)$.
}
 $$
\begin{tikzcd}
	{} &\ \G \arrow[dl, "\hat x_0"'] \arrow[dr, "\hat \mu_0"] & \\
	\ \X \arrow[rr, "\mu"'] & & \cO
\end{tikzcd}
\quad \quad \quad
\begin{tikzcd}
	{} & \ \G \arrow[dl, "\Stab_\rho(x_0)"'] \arrow[dr, "\Stab_{\Ad_*^{\Gamma,\theta}}(\mu_0)"] & \\
	\ \X \arrow[rr, "\Stab_{\Ad_*^{\Gamma,\theta}}(\mu_0)/\Stab_\rho(x_0)"'] & & \cO
\end{tikzcd}
 $$
{\sc Note} --- The moment maps $\mu$ are defined up to a
constant, but the {\em characteristics} of $\mu$, that is
the subspaces defined by $\mu(x) = \const$, are 
not. They are the solutions of the equation $\psi(x_0,x)
= 0$, where $\const = \mu(x_0)$ and $\psi$ is the
2-points moment map. 
 \end{article}

\begin{proof}
This is just an application of standard diffeological
relations. \end{proof}

 \begin{article}[Symplectic homogeneous diffeological
spaces]
 \label{Symplectic-homogeneous-diffeological-spaces}
Let $\X$ be a connected diffeological space and $\omega$
be a closed 2-form defined on $\X$.  

{\sc Definition} We say that $(\X,\omega)$ is an
{\em homogeneous symplectic space} if it is homogeneous
under the action of $\Diff(\X,\omega)$ and if a universal
moment map $\mu_\omega$ is a covering onto its image. 

The homogeneous situation where the moment maps
$\mu_\omega$ are not coverings onto their images can be
regarded as the {\em homogeneous pre-symplectic} case.

Now,
let $\G$ be some diffeological
group, and let $\rho$ be a smooth action of
$\G$ on $\X$, preserving $\omega$. So, if the action
$\rho$ of $\G$ on $\X$ is homogeneous, then $\X$ is an
homogeneous space of $\Diff(\X,\omega)$. And, if a moment
map $\mu : \X \to \cG^*\!/\Gamma$ is a covering onto its
image, then any universal moment map $\mu_\omega : \X \to
\cG^*_\omega/\Gamma_\omega$ is  a covering onto its
image.
 
Thus, to check that an homogeneous pair
$(\X,\omega)$ is symplectic it is sufficient to find a
smooth homogeneous smooth action of some
diffeological group $\G$ for which one moment map
is  a covering onto its image.
 \end{article}

\begin{proof}
To be homogeneous under the action of $\G$ means
that, for some point (and thus for any point) $x \in \X$,
the orbit map $\hat x : \G \to \X$, defined by $\hat x
(g) = \rho(g)(x)$, is a subduction. So, $\hat x$ is
surjective and, for any plot $\P : \U \to \X$, for any
$r_0 \in \U$, there exists a superset $\V$ of $r_0$ and a
plot $\Q : \V \to \G$ such that $\P \restriction \V =
\hat x \circ \Q$. That is, $\P(r) = \rho(\Q(r))(x)$ for
all $r \in \V$. Since $\rho$ is smooth, $\bar \Q = \rho
\circ \Q$ is a plot of $\Diff(\X,\omega)$, and $\P
\restriction \V = \hat x \circ \bar \Q$. Since, $\hat
x : \Diff(\X,\omega) \to \X$ is surjective, it is
a subduction and $\X$ is an homogeneous space of
$\Diff(\X,\omega)$. 

Now, let us remark that, since the moment maps differ just
by a constant, if a moment map $\mu$ is a covering onto
its image $\cO$ equipped with the quotient diffeology of
$\G$, then every other moment map $\mu' = \mu + \const$
is a covering onto its image $\cO' = \cO + \const$. So,
let $x_0$ be a point of $\X$, and let $\mu(x) =
\psi(x_0,x)$, where $\psi$ is the 2-points moment map.
Let $\mu_\omega = \psi_\omega(x_0,x)$. According to
\art{Universal-moment-maps}, $\mu =
\rho^*_{\Gamma_\omega} \circ \mu_\omega$. Let $\cO =
\mu(\X)$ and $\cO_\omega = \mu_\omega(\X)$, equipped with
the quotient diffeologies of $\G$ and $\G_\omega =
\Diff(\X,\omega)$. So, $\cO =
\rho^*_{\Gamma_\omega}(\cO_\omega)$. Let $m \in \cO$ and
$m_\omega \in \cO_\omega$ such that
$\rho^*_{\Gamma_\omega}(m_\omega) = m$. So,
$\mu_\omega^{-1}(m_\omega) = \{x \in \X \mid
\mu_\omega(x) = m_\omega \} \subset \mu^{-1}(m) = \{ x
\in \X \mid \mu(x) =
\rho^*_{\Gamma_\omega}(\mu_\omega(x)) = m \}$. Thus, if
$\mu^{-1}(m)$ is discrete, a fortiori
$\mu_\omega^{-1}(m_\omega) \subset \mu^{-1}(m)$. Thus, if
$\mu$ is a fibration onto its image, then $\mu_\omega$ is
a fibration onto its image too. And of course if $\mu$ is
injective, a fortiori $\mu_\omega$.
 \end{proof}

\delete{
\begin{article}[Covering symplectic spaces]
\label{Covering-symplectic-spaces}
Let $(\X,\omega)$ be a connected symplectic diffeological
space. Let $\pi : \hat \X \to \X$ be a covering. Let
$\hat \omega = \pi^*(\omega)$, so $(\hat \X, \hat
\omega)$ is a symplectic diffeological space.
\end{article}

\begin{proof}
\end{proof}
}

%************************************************
\section{Examples of moment maps in diffeology}
\label{Examples-of-moment-maps-in-diffeology}

This short list of examples shows how the
theory of moment map in diffeology can be
applied to the folklore of infinite dimensional
situations, but also to the less familiar cases of
singular spaces. 

\begin{article}[The moments of imprimitivity]
\label{The-moments-of-imprimitivity}
Let $\X$ be a diffeological space. Let us remind, and make
some preliminary remarks on, the construction of the {\em
cotangent bundle} and the definition of the Liouville
form \cite{Piz05}. Let $\Omega^1(\X)$ denotes the
vector space of $1$-form of $\X$, equipped with the
functional diffeology. The mapping $\Taut$, which
associates to each $n$-plot $\Q \times \P$ of the product
$\X \times \Omega^1(\X)$ the 1-form 
 $$
 \Taut(\P \times \Q) : r \mapsto 
\P(r)(\Q)_r
 $$
 of $\dom(\Q \times \P)$, is a 1-form of $\X
\times \Omega^1(\X)$. We call it
 the {\em tautological form}. 

Now, let us
consider the $\Value$ equivalence relation. Let $\alpha$
and $\alpha'$ be two 1-forms of $\X$, let $x$ be a point
of $\X$. We say that $\alpha$ and $\alpha'$ {\em have the
same value} at the point $x$, and we denote
$\Value(\alpha)(x) = \Value(\alpha')(x)$, if and only if,
for every plot $\Q$ of $\X$ centered\footnote{We say
that a plot $\Q$ is {\em centered\ }at $x$ if and only
if $0 \in \dom(\Q)$ and $\Q(0) = x$.} at $x$ ,
 $\alpha(\Q)_0 = \alpha'(\Q)_0$. Then,
the {\em cotangent bundle} of $\X$ is defined as the
quotient $\X \times \Omega^1(\X)$ by the relation
$\Value$, and denoted\footnote{Note that, as well as for
the notation $\cG^*$ of the space of momenta of a
diffeological group, the star in $\T^*\X$ do not rely to
any kind of duality a priori.} by $\T^*\X$,
 $$
 \T^*\X = \X \times \Omega^1(\X) / \Value.
 $$
 This notion of value, for smooth forms on
numerical domains, coincides with the ordinary
definition. So, when there will be no
risk of confusion\footnote{This
notation $\alpha(x)$  has not to be mixed up with the
notation  $\alpha(\Q)$ for the value of $\alpha$ in the
plot $\Q$. But the different nature of $x$: a point of
$\X$, and $\Q$: a plot of $\X$, makes the difference.},
we shall denote simply by $\alpha(x)$ the value of
$\alpha$ at the point $x$, that is $\alpha(x) =
\Value(\alpha)(x)$.

Let $\pr :
\X \times \Omega^1(\X) \to \T^*\X$ be the canonical
projection. So, there exists a 1-form on $\T^*\X$,
denoted by $\Liouv$ and called the {\em Liouville form}
such that 
 $$
 \Taut = \pr^*(\Liouv) \qmbox{or} \Liouv = \pr_*(\Taut),
\quad \Liouv \in \Omega^1(\T^*\X).
 $$ 
 The characteristic property of the Liouville form is the
following. Let $\alpha$ be a 1-form of $\X$, let $\bar
\alpha$ be the section of the canonical projection
$\pi : \T^*\X \to \X$ defined by $\bar \alpha : x
\mapsto \Value(\alpha)(x)$, so $\alpha = \bar
\alpha^*(\Liouv)$. Note also that, the group $\Diff(\X)$
acts naturally on the product $\X \times \Omega^1(\X)$ by
$\bar \varphi(x,\alpha) = (\varphi(x),
\varphi_*(\alpha))$, where $\varphi$ is a diffeomorphism
of $\X$. So, the tautological form is invariant by this
action. Moreover, this action is compatible with the
relation $\Value$, and the group $\Diff(\X)$ has a
natural projected action on $\T^*\X$. By equivariance,
the Liouville form is invariant by this action. Note
that, the moment map for the action of $\Diff(\X)$
on $(\T^*\X, d\Liouv)$ is given by the general
construction of \art{The-exact-case}. This
can be compared to Donato's construction for manifolds in
\cite{Don88}.

Now, let us introduce the additive diffeological group of
smooth functions $\Cinfty(\X,\RR)$, acting smoothly on
$\X \times \Omega^1(\X)$ by,
 $$
 \bar f : (x, \alpha) \mapsto (x , \alpha + df),
 $$
for all $f \in \Cinfty(\X,\RR)$. This action projects
naturally on the cotangent $\T^*\X$ into an action,
denoted  by the same way,
 $$
 \bar f : (x,a) \mapsto (x, a + df(x)),
 $$
 for all $(x,a) \in \T^*\X$. So,
 \begin{enumerate}
 \item For all $f \in \Cinfty(\X,\R)$, the variance of the
tautological form and the Liouville form are given by,
 $$
 \bar f ^*(\Taut) = \Taut + \pr_1^*(df) \qmbox{and} \bar
f^*(\Liouv) = \Liouv + \pi^*(df).
 $$ 
  So, the
exterior differentials $d\Taut$ and $\omega = d\Liouv$ are
invariant by the action of $\Cinfty(\X,\RR)$.
 \item Let $p$ be a path of $\T^*\X$, connecting
$(x_0,a_0) = p(0)$ to $(x_1,a_1) = p(1)$. So, the paths
moment map $\Psi$ and the
2-points moment map $\psi$, with respect
to the 2-form $\omega = d \Liouv$, are given by
 $$
 \Psi(p) = \psi((x_0,a_0),(x_1,a_1)) = d[f \mapsto
f(x_0)] - d[f \mapsto f(x_1)].
 $$ 
 \item For every $x \in \X$,  the real function
$[f \mapsto f(x)]$ is smooth. We call it the
{\em Dirac function} of the point $x$, and we denote
it by $\delta_x$.
 $$
 \delta_x = [f \mapsto f(x)] \in
\Cinfty(\Cinfty(\X,\RR),\RR).
 $$
 The differential
$d \delta_x = d[f \mapsto f(x)]$ is an invariant
1-form\footnote{This differential has nothing to do
with the derivative of the Dirac distributions in the
sense of De Rham's currents.} of the additive group
$\Cinfty(\X,\RR)$. Every moment map of the
action of $\Cinfty(\X,\R)$ on $\T^*\X$ is cohomologous to
the invariant moment map
 $$
 \mu : (x,a) \mapsto - d \delta_x.
 $$
Note that, the moment $\mu$ is constant on the fibers
$\T_x^*\X = \pi^{-1}(x)$. And, if the real smooth
functions separate\footnote{That is, $f(x) = f(x')$ for
all smooth real function $f$ if and only if $x=x'$.} the
points of $\X$,
the image of the moment map $\mu$ is the space $\X$,
identified with the space of Dirac's functions.
 \item The action of $\Cinfty(\X,\RR)$ on
$(\T^*\X,\omega)$ is hamiltonian and exact. That is,
$\Gamma = \{0\}$ and $\sigma = 0$. 
 \end{enumerate}

This example
has been drawn to my attention by Fran\c{c}ois Ziegler.
This moment appears informally in Ziegler's construction
of a symplectic analogue for \og systems of
imprimitivity\\fg in representation theory \cite{Zie96}.
It is why the moment map $\mu$ will be called the {\em
moment of imprimitivity}.  The diffeological framework
gives it so a full formal status.
 \end{article}

\begin{proof}
First of all let us check the variance of $\Taut$ by the
action of $\Cinfty(\X,\RR)$. Let $f$ be a smooth real
function defined on $\X$, let $\Q \times \P$ be a plot of
$\X \times \Omega^1(\X)$. We have $\bar f^*(\Taut)(\P
\times \Q)_r = \Taut(\bar f \circ (\Q \times \P))_r =
(\P(r) + df)(\Q)_r = \P(r)(\Q)_r + df(\Q)_r = \Taut(\Q
\times \P)_r + df(\pr_1 \circ (\Q \times \P))_r = \Taut(\Q
\times \P)_r + \pr_1^*(df)(\Q \times \P)_r$. So, $\bar
f^*(\Taut) = \Taut + \pr_1^*(df)$. Now let us check that
this action is compatible with the $\Value$ relation. Let
$(x,\alpha)$ and $(x',\alpha')$ be two elements of $\X
\times \Omega^1(\X)$ such that $\Value(\alpha)(x) =
\Value(\alpha')(x')$. That is, $x = x'$ and for every plot
$\Q$ of $\X$ centered at $x$, $\alpha(\Q)_0 =
\alpha'(\Q)_0$. So, $(\alpha +df)(\Q)_0 =
(\alpha' + df)(\Q)_0$ and $\Value(\alpha + df)(x) =
\Value(\alpha)(x) + \Value(df)(x)$,
or $(\alpha + df)(x) = \alpha(x) + df(x)$. Thus, the
action of $\Cinfty(\X,\RR)$ projects on $\T^*\X$ as the
action $\bar f : (x,a) \mapsto a + df(x)$. Now, since
$\bar f^*(\Taut) = \Taut + \pr_1^*(df)$, clearly  $\bar
f^*(\Liouv) = \Liouv + \pi^*(df)$. Or, in another way,
$\bar f^*(\Liouv) = \Liouv
 + d\F(f)$ with $\F \in
\Cinfty(\Cinfty(\X,\RR),\Cinfty(\T^*\X,\RR))$ and $\F(f) =
\pi^*(f) =f \circ \pi$.

Let us denote by $\eR(x,a)$ the orbit map $f \mapsto a +
df(x)$. Let $p$ be a path of $\T^*\X$ such that $p(0) =
(x_0,a_0)$ and $p(1) = (x_1,a_1)$. We get 
 \begin{eqnarray*}
 \Psi(p) & = & \hat p^*(\eK d\Liouv) \\
 & = & \hat p^*(\but^*(\Liouv) - \source^*(\Liouv) - d
\eK\Liouv) \\
 & = & (\but \circ \hat p)^*(\Liouv) - (\source  \circ
\hat p)^*(\Liouv) - d [(\eK\Liouv) \circ \hat p] \\
 & = &
\eR(x_1,a_1)^*(\Liouv) - \eR(x_0,a_0)^*(\Liouv) - d [f 
\mapsto \eK\Liouv (\hat p(f))].
 \end{eqnarray*}
 Let us consider first the term $[f \mapsto \eK\Liouv
(\hat p(f))]$. Let $p(t) =(x_t,a_t)$, so
$\hat p(f) = [t \mapsto (x_t,a_t + df(x_t))]$. Thus,
 \begin{eqnarray*}
 \eK\Liouv(\hat p(f))) & = &\int_0^1 a_t[s \mapsto
x_s]_{s=t} \ dt + \int_0^1 df[t \mapsto x_t] \ dt \\
& = & \int_0^1
a_t[s \mapsto x_s]_{s=t} \ dt + f(x_1) - f(x_0).
 \end{eqnarray*}
 Thus, 
 \begin{eqnarray*}
d [f 
\mapsto \eK\Liouv (\hat p(f))] &=& d[f \mapsto \int_0^1
a_t[s \mapsto x_s]_{s=t} \ dt + f(x_1) -
f(x_0)] \\
& = & d[f \mapsto f(x_1) -
f(x_0)].
 \end{eqnarray*}
 Let us compute now
$\eR(x,a)^*(\Liouv)$, for any $(x,a) \in \T^*\X$. Let
$\P: \U \to \Cinfty(\X,\RR)$ be a plot. We have
 \begin{eqnarray*}
 \eR(x,a)^*(\Liouv)(\P) & = & \Liouv (\eR(x,a) \circ
\P) \\
 & = &
\Liouv(r \mapsto \overline{\P(r)}(x,a)) \\
 & = & \Liouv(r \mapsto a + d[\P(r)](x)) \\
 & = & (a +d[\P(r)](x))(r \mapsto x) \\
 & = & 0
 \end{eqnarray*}
 because the 1-form $a +d[\P(r)](x)$ is evaluated on
the constant plot $r \mapsto x$. And, every form
evaluated to a constant plot vanishes. So, we get finally
 $$
 \Psi(p) =  d[f \mapsto
f(x_0)] - d[f \mapsto f(x_1)]. 
 $$
 Now, clearly $\Psi(\ell) = 0$ for every loop $\ell$ of
$\T^*\X$, and the action of $\Cinfty(\X,\RR)$ is
hamiltonian $\Gamma = \{0\}$. So,
$\psi((x_0,a_0),(x_1,a_1)) = \mu(x_1,a_1) -
\mu(x_0,a_0)$, with the moment map
 $$
 \mu : (x,a) \mapsto -d[f \mapsto f(x)] = - d \delta_x.
 $$
   Let us check now the
invariance of the moment map $\mu$. Note that, for every
$h \in \Cinfty(\X,\RR)$, we have $\delta_x \circ \eL(h) =
[f \mapsto f(x) + h(x)]$. So, for every $h \in
\Cinfty(\X,\RR)$ we have  $\hat h^*(\mu)(x,a) =
\hat h^*(-d \delta_x) = -d(\delta_x \circ
\eL(h)) = - d [f \mapsto f(x) + h(x)] 
= - d [f \mapsto f(x)] = - d \delta_x = \mu(x,a)$.
Hence, $\mu$ is
invariant. The 2-points moment map $\psi$ is exact. 
Souriau's class of the action of $\Cinfty(\X,\RR)$ on
$\T^*\X$ vanishes.  \end{proof}

\begin{article}[On the intersection 2-form of a surface I]
\label{On-the-intersection-form-of-a-surface-I}
Let $\Sigma$ be a closed surface oriented by a  2-form
$\Surf$, chosen once and for all.
Let us consider  $\Omega^1(\Sigma)$, the infinite
dimensional vector space  of 1-forms of $\Sigma$,
equipped with the functional diffeology. Let us consider
the antisymmetric bilinear map defined on
$\Omega^1(\Sigma)$ by
 $$
 (\alpha,\beta) \mapsto \int_\Sigma \alpha\wedge \beta,
 $$
  for all $\alpha$, $\beta$ in $\Omega^1(\Sigma)$. Since
the wedge-product $\alpha \wedge\beta$ is a 2-form of
$\Sigma$, there exists a real smooth function $\varphi \in
\Cinfty(\Sigma,\RR)$ such that $\alpha \wedge \beta =
\varphi \times \Surf$. So, by definition, $\int_\Sigma
\alpha \wedge \beta = \int_\Sigma \varphi \times \Surf$. 

1) To the above bilinear form is naturally
associated a well defined differential 2-form $\omega$ of
$\Omega^1(\X)$.  For every $n$-plot $\P : \U \to \X$,
for all  $r \in \U$, $\delta r$ and $\delta' r$ in
$\RR^n$,
 $$
\omega(\P)_r(\delta r,\delta' r) = \int_\Sigma
{\partial \P(r) \over \partial r}(\delta r)\wedge
{\partial \P(r) \over \partial r}(\delta' r)
 $$

2) The 2-form $\omega$ is the differential of the 1-form 
$\lambda$ defined on $\Omega^1(\Sigma)$ by,
  $$
 \lambda(\P)_r(\delta r) = \undemi \int_\Sigma
\P(r)\wedge {\partial \P(r) \over \partial r}(\delta
r) \qmbox{and} \omega = d\lambda.
 $$
3) Let us consider now the
the additive group $(\Cinfty(\Sigma,\RR),+)$ of smooth
real functions of $\Sigma$. And, let us define the
following action of $\Cinfty(\Sigma,\RR)$ on
$\Omega^1(\Sigma)$.
 $$
 \mbox{For all $f \in \Cinfty(\Sigma,\RR)$}, \quad f
\mapsto \bar f = [\alpha \mapsto  \alpha + df].
 $$
 So, the additive group $\Cinfty(\Sigma,\RR)$ acts by
automorphisms on the pair $(\Omega^1(\Sigma),\omega)$.
 $$
 \mbox{For all $f$ in $\Cinfty(\Sigma,\RR)$},
\quad f^*(\omega) = \omega.
 $$
 Note
that the kernel of the action $f \mapsto \bar f$ is the
subgroup of constant maps. And, the image of
$\Cinfty(\Sigma,\RR)$ is just the group
$\B^1_{\D\R}(\Sigma)$ of exact 1-forms of $\Sigma$.

4)  Let $p \in \Paths(\Omega^1(\Sigma))$ be a path
connecting $\alpha_0$ to $\alpha_1$. The paths moment map
 $\Psi(p)$
is given by
 $$
\Psi(p) = \bigg(\hat \alpha_1^*(\lambda)
+ d\bigg[f \mapsto \undemi \int_\Sigma f
\times d\alpha_1\bigg]\bigg) - \bigg(\hat
\alpha_0^*(\lambda) + d\bigg[f \mapsto
\undemi \int_\Sigma
f \times  d\alpha_0 \bigg] \bigg).
 $$
 On this expression, we check immediately that the
2-points moment map is just
given by $\psi(\alpha_0,\alpha_1) = \Psi(p)$, for any
path $p$ connecting $\alpha_0$ to $\alpha_1$.  Note that,
since $\Omega^1(\Sigma)$ is 
contractible the holonomy of the action of
$\Cinfty(\Sigma,\RR)$ vanishes,
$\Gamma = \{0\}$, the
action of $\Cinfty(\Sigma,\RR)$ is hamiltonian.

5) The moment maps
of this action of $\Cinfty(\Sigma,\RR)$ on
$\Omega^1(\Sigma)$ are, up to a constant, equal to 
 $$
 \mu : \alpha \mapsto d \bigg[ f \mapsto \int_\Sigma
f \times d\alpha\bigg].
 $$
Moreover, the moment map $\mu$ is equivariant. That is,
invariant, since the group $\Cinfty(\Sigma,\RR)$ is
abelian.
 $$
 \mbox{For all $f \in \Cinfty(\Sigma,\RR)$}, \quad \mu
\circ \bar f = \mu.
 $$
 So, the action of $\Cinfty(\Sigma,\RR)$ on
$\Omega^1(\Sigma)$ is exact and hamiltonian.

{\sc Note} --- The moment
map $\mu(\alpha)$ is fully characterized by $d\alpha$.
This is why we find in the mathematical
literature on the subject that, the moment map for this
action is the exterior derivative (or curvature,
depending on the authors) $\alpha \mapsto d\alpha$. But,
as we see again on this example, diffeology gives to this
sketchy assertion a precise meaning.

Let us remark also that, the moment map
$\mu$ is linear, for all $t$, $s$ reals and all $\alpha$
and $\beta$ in $\Omega^1(\Sigma)$,
$\mu(t\,\alpha+s\,\beta) = 
t\,\mu(\alpha)+s\,\mu(\beta)$. And, the kernel
of $\mu$ is the subspace of closed 1-forms,
 $$
 \ker(\mu) = \Z^1_{\D\R}(\Sigma) = \left\{ \alpha \in
\Omega^1(\Sigma) \ \vert \ d\alpha = 0 \right\}
 $$ 
If we consider the orbit of the zero form $0 \in
\Omega^1(\Sigma)$ by $\Cinfty(\Sigma,\RR)$, this is just
the subspace $\B^1(\Sigma,\RR)$, which is included in
$\ker(\mu) = \Z^1_{\D\R}(\Sigma)$. The quotient
$\ker(\mu)/ \Cinfty(\Sigma,\RR)$ is just
$ \Z^1_{\D\R}(\Sigma) / \B^1_{\D\R}(\Sigma)=
\H^1_{\D\R}(\Sigma)$, and the 2-form $\omega \restriction
\ker(\mu)$ is just the pullback of the usual
intersection form on $\H^1_{\D\R}(\Sigma)$. I will
discuss, in a future work, the notion of \og symplectic
reduction\\fg in diffeology.
 \end{article}

\begin{proof}
1) Let us check that $\omega$ defines a differential
1-form on $\Omega^1(\Sigma)$. Note
that, for any $r \in \U = \dom(\P)$, $\P(r)$ is a section
of the ordinary cotangent bundle $\T^*\Sigma$. That is,
$\P(r) = [x \mapsto \P(r)(x)] \in
\Cinfty(\Sigma,\T^*\Sigma)$, where $\P(r)(x) \in
\T_x^*(\Sigma)$. So,
 $$
 {\partial \P(r) \over \partial
r}(\delta r) = [x \mapsto  {\partial \P(r)(x) \over
\partial r}(\delta r)] \quad \mbox{and} \quad {\partial
\P(r) (x) \over \partial r}(\delta r) \in
{\T}^{*}_{x}(\Sigma)
 $$
 where $\partial \P(r)(x) / \partial r$ denotes the
tangent linear map $\D(r \mapsto \P(r)(x)(r)$. And,
the formula giving $\omega$ is well defined. Now,
$\omega(\P)_r$ is clearly antisymmetric and depends
smoothly on $r$. So, $\omega(\P)$ is a smooth 2-form
of ${\U}$. Let us check that $\P \mapsto \omega(\P)$
defines a 2-form on $\Omega^1(\Sigma)$. That is,
satisfies the compatibility condition $\omega(\P \circ
\F) = \F^*(\omega(\P))$, for all $\F \in \Cinfty(\V,
\U)$, where $\V$ is a numerical domain. Let $s \in {\V}$,
$\delta s$ and $\delta' s$ two tangent vectors at $s$ at
${\V}$, let $r = \F(s)$: 
 \begin{eqnarray*} 
 \omega(\P \circ \F)_s(\delta s,\delta' s) & = &
\int_\Sigma {\partial \P \circ \F(s) \over \partial
s}(\delta s)\wedge {\partial \P \circ \F(s) \over
\partial s}(\delta' s) \\ 
 & = & \int_\Sigma {\partial \P(r) \over \partial
r}{\partial \F(s) \over \partial s}(\delta s)\wedge
{\partial \P(r) \over \partial r}{\partial  \F(s)
\over \partial s}(\delta' s) \\
 & = & \omega(\P)_{\F(s)}(\D\F_s(\delta s),\D\F_s(\delta'
s)) \\
 & = & {\F}^*(\omega(\P))_s(\delta s,\delta' s)
 \end{eqnarray*}
 Thus $\omega(\P \circ \F) =
\F^*(\omega(\P))$, and $\omega$ is a well
defined 2-form on $\Omega^1(\Sigma)$. 

2) First of all, the proof that the map $\P \mapsto
\lambda(\P)$ is a well defined differential 1-form of
$\Omega^1(\Sigma)$ is analogous to the proof of the
first item. Now, let us remind that $\omega = d\lambda$ is
and only if $d(\lambda(\P)) = \omega(\P)$ for all plot
$\P$ of $\Omega^1(\Sigma)$. Let us apply the usual
formula of differentiation of 1-form on numerical domain,
$$d\epsilon_r(\delta r,\delta' r) =
\delta(\epsilon_r(\delta' r)) - \delta'(\epsilon_r(\delta
r))$$ where $\delta$ and $\delta'$ are to commuting
variations. For the sake of simplicity let us denote 
 $$
 \alpha = \P(r), \quad \delta\alpha = {\partial \P(r)
\over \partial r}(\delta r), \quad \delta'\alpha =
{\partial \P(r) \over \partial r}(\delta' r).
 $$
So, 
 \begin{eqnarray*}
 d(\lambda(\P))_r(\delta r,\delta' r) & = &
\undemi \bigg[\delta \int_\Sigma \alpha \wedge \delta'
\alpha - \delta' \int_\Sigma \alpha \wedge \delta \alpha
\bigg] \\
 & = & \undemi \bigg[\int_\Sigma \delta \alpha \wedge
\delta' \alpha + \alpha \wedge \delta \delta' \alpha -
\int_\Sigma \delta' \alpha \wedge \delta \alpha + \alpha
\wedge \delta' \delta \alpha \bigg].
 \end{eqnarray*} but, $\delta \delta' \alpha = \delta'
\delta \alpha$. So,
 \begin{eqnarray*}
 d(\lambda(\P))_r(\delta r,\delta' r) & = &
\undemi \bigg[\int_\Sigma \delta \alpha \wedge \delta'
\alpha
 - \int_\Sigma \delta' \alpha \wedge \delta\alpha \bigg]
\\
 & = & \undemi \bigg[\int_\Sigma \delta \alpha \wedge
\delta' \alpha
 + \int_\Sigma \delta \alpha \wedge \delta' \alpha \bigg]
\\
 & = & \int_\Sigma \delta \alpha \wedge \delta' \alpha \\
 & = & \omega_r(\delta r,\delta' r).
\end{eqnarray*}

3) Let us compute the pullback of $\lambda$ by the
action of $f \in \Cinfty(\Sigma,\RR)$. Let $\P : \U \to
\Omega^1(\Sigma)$ be a $n$-plot, let $r \in \U$ and
$\delta r \in \RR^n$.
 \begin{eqnarray*}
 \bar f^*(\lambda)(\P)_r(\delta r) & = & \lambda( \bar f
\circ \P)_r(\delta r) \\
 & = & \lambda(r \mapsto \P(r) + df)_r (\delta r) \\
 & = & \undemi \int_\Sigma (\P(r) + df) \wedge {\partial
 \P(r) \over \partial r}(\delta r) \\
 & = & \undemi \int_\Sigma \P(r) \wedge {\partial
\P(r)\over \partial r}(\delta r) + \undemi \int_\Sigma df
\wedge {\partial \P(r) \over \partial r}(\delta r) \\
 & = & \lambda(\P)_r(\delta r) + {\partial \over \partial
r} \bigg\{\undemi \int_\Sigma df \wedge
 \P(r)\bigg\}(\delta r) \\
 & = & \lambda(\P)_r(\delta r) - {\partial \over \partial
r}\bigg\{ \undemi \int_\Sigma f
\times d(\P(r))\bigg\}(\delta r) \\
 \end{eqnarray*}
 So, for every $f \in \Cinfty(\Sigma,\RR)$, let us
define the map $\varphi(f) : \Omega^1(\Sigma) \to \RR$ by,
  $$
 \varphi(f) : \alpha  \mapsto \undemi \int_\Sigma f \times
d\alpha.
 $$
So, 
 $$
 d(\varphi(f))(\P)_r(\delta r) = {\partial \over \partial
r} \bigg\{ \undemi \int_\Sigma f
\times d(\P(r)) \bigg\}(\delta r).
 $$
 Thus,
 $$ 
 \bar f^*(\lambda)(\P)_r(\delta r) =
\lambda(\P)_r(\delta r)  -
(d\varphi(f))(\P)_r(\delta r).
 $$
 That is,
 $$
 \bar f^*(\lambda) = \lambda - d(\varphi(f)).
 $$
 Therefore, differential $\omega = d\lambda$ is
invariant by the action of $\Cinfty(\Sigma,\RR)$. 

4) Let $p$ be a path of $\Omega^1(\Sigma)$ connecting
$\alpha_0$ to $\alpha_1$. By definition $\Psi(p) =
\hat p^*(\eK\omega)$.
Applying the property of the chain-homotopy
operator $d \circ \eK + \eK \circ d = \but^* -
\source^*$ to $\omega = d\lambda$, we
get
 \begin{eqnarray*}
\Psi(p) & = & \hat
p^*(\eK d\lambda) \\
 & = & \hat p^*(\but^*(\lambda) -
\source^*(\lambda) - d (\eK \lambda)) \\
 & = & (\but \circ
\hat p)^*(\lambda) - (\source \circ \hat
p)^*(\lambda) - d [(\eK \lambda) \circ \hat p] \\
 & = & \hat \alpha_1^*(\lambda) - \hat
\alpha_0^*(\lambda) - d[f \mapsto \eK\lambda(\hat p(f)) ]
 \end{eqnarray*}
But, $\eK\lambda(\hat p(f)) = \eK\lambda(\bar f \circ
p) = \int_{\bar f \circ p} \lambda = \int_p \bar
f^*(\lambda)$, and since $\bar f^*(\lambda) = \lambda -
d(\varphi(f))$ we have $\eK\lambda(\hat p(f)) = \int_p
 \lambda - \int_p d(\varphi(f)) = \int_p \lambda
- \varphi(f)(\alpha_1) + \varphi(f)(\alpha_0)$. Therefore,
 \begin{eqnarray*}
 \Psi(p) & = & \hat \alpha_1^*(\lambda) -
\hat \alpha_0^*(\lambda) - d[f \mapsto -
\varphi(f)(\alpha_1) + \varphi(f)(\alpha_0)] \\
 & = & \hat \alpha_1^*(\lambda) - \hat
\alpha_0^*(\lambda) + d\bigg[ f \mapsto \undemi
\int_\Sigma f \times d\alpha_1 - \undemi \int_\Sigma f
\times d \alpha_0 \bigg]
 \end{eqnarray*}
And, finally we get the paths moment map $\Psi$ given by
 $$
 \Psi(p) =
\bigg(\hat \alpha_1^*(\lambda) + d \bigg[f \mapsto
\undemi \int_\Sigma f \times  d\alpha_1 \bigg]\bigg) -
\bigg(\hat \alpha_0^*(\lambda) + d \bigg[f  \mapsto
\undemi \int_\Sigma f \times d \alpha_0 \bigg]\bigg)
 $$
For the the
2-points moment map $\psi$, we have
clearly $ \psi(\alpha_0,\alpha_1) = \Psi(p)$ for any path
connecting $\alpha_0$ to $\alpha_1$.

5)  The 1-point moment maps are given by $\mu(\alpha) =
\psi(\alpha_0,\alpha)$ for any origin $\alpha_0$. Let us
choose $\alpha_0 = 0$. So,
 $$
 \mu(\alpha) = \hat \alpha^*(\lambda) +
d \bigg[f \mapsto \undemi \int_\Sigma f \times d\alpha
\bigg] - \source^*(\lambda).
 $$
 But $\source^*(\alpha)$ is not necessarily zero. Let us
compute generally $\hat \alpha^*(\lambda)$. Let $\P : \U
\to \Omega^1(\Sigma)$ be a $n$-plot. We have,
$\hat \alpha^*(\lambda)(\P) =
\lambda(\hat \alpha \circ \P) =  \lambda(r \mapsto
\hat \alpha(\P(r)) = \lambda(r \mapsto 
\alpha + d(\P(r)))$. But,
 \begin{eqnarray*}
\lambda(r \mapsto \alpha + d(\P(r))) & = &
\undemi \int_\Sigma (\alpha + \P(r)) \wedge {\partial
\over \partial r}(\alpha + d(\P(r))) \\
 & = &  \undemi \int_\Sigma (\alpha + \P(r)) \wedge
{\partial d(\P(r)) \over \partial r} \\
 & = & \undemi \int_\Sigma \alpha \wedge {\partial
d(\P(r)) \over \partial r} + \undemi \int_\Sigma
\P(r) \wedge {\partial d(\P(r)) \over \partial r}. 
\end{eqnarray*}
 So,
$$
(\hat \alpha^*(\lambda) - \source^*(\lambda))(\P) = 
\undemi \int_\Sigma \alpha \wedge {\partial d(\P(r)) \over
\partial r}.
 $$
 Therefore, 
\begin{eqnarray*}
\mu(\alpha)(\P)_r & = & (\hat \alpha^*(\lambda) -
\source^*(\lambda))(\P)_r + d \bigg[f \mapsto \undemi
\int_\Sigma f \times d\alpha\bigg](\P)_r \\
 & = & \undemi \int_\Sigma \alpha \wedge {\partial
d(\P(r)) \over \partial r} + {\partial \over \partial
r} \bigg\{ \undemi \int_\Sigma \P(r) \times
d \alpha \bigg\} \\
 & = & \undemi {\partial \over
\partial r} \bigg\{ \int_\Sigma \alpha \wedge d(\P(r)) +
\P(r) \times d\alpha \bigg\} \\
 & = & {\partial \over
\partial r} \bigg\{ \int_\Sigma \P(r) \times d\alpha
\bigg\}.
 \end{eqnarray*}
So, we get finally, 
 $$
\mu(\alpha) = d \bigg[ f \mapsto \int_\Sigma f \times
d\alpha \bigg].
 $$
Now, let us express the variance of $\mu$. Let $f \in
\Cinfty(\Sigma,\RR)$, and let $\F(\alpha)$ be the real
function $\F(\alpha) : f \mapsto \int_\Sigma f \times
d\alpha$, such that $\mu(\alpha) = d\F(\alpha)$. We
have, $\mu(\bar f(\alpha)) =  \mu (\alpha + df) =
d\F(\alpha + df)$. But, for every $h \in
\Cinfty(\Sigma,\RR)$, $\F(\alpha + df)(h) = \int_\Sigma h
\times d(\alpha + df) = \int_\Sigma h \times d\alpha =
\F(\alpha)(h)$. So, for all $f \in
\Cinfty(\Sigma,\RR)$, we have $\mu \circ \hat f = \mu$.
The moment map $\mu$ is invariant by the group
$\Cinfty(\Sigma,\RR)$. Souriau's class vanishes. Thus,
the action of $\Cinfty(\Sigma,\RR)$ is exact and
hamiltonian.

Let us compute finally the kernel of the moment map
$\mu$. We have: $\mu(\alpha) = 0$ if and only if 
$d\F(\alpha) = 0$. But since $\Cinfty(\Sigma,\RR)$
is connected (actually contractible as a diffeological
vector space) $d\F(\alpha) = 0$ if and only if
$\F(\alpha) = \const = \F(\alpha)(0) = 0$. But
$\F(\alpha) = 0$ if and only if, for all
$f \in \Cinfty(\Sigma,\RR)$, $\int_\Sigma f
\times d\alpha = 0$. That is, if and only if $d\alpha =
0$.
 \end{proof}

\begin{article}[On the intersection 2-form of a surface
II] \label{On-the-intersection-form-of-a-surface-II}
We continue with the example of  
\art{On-the-intersection-form-of-a-surface-I}, using the
same notations. Let us introduce the
group $\G$ of positive diffeomorphisms of
$(\Sigma,\Surf)$. That is,
 $$
 \G = \bigg\{ g \in \Diff(\Sigma) \ \bigg| \ {g^*(\Surf)
\over \Surf} >0 \bigg\}.
 $$
 The group $\G$ acts by pushforward on
$\Omega^1(\Sigma)$. For all $g \in \G$, for all $\alpha
\in \Omega^1(\Sigma)$, $g_*(\alpha) \in \Omega^1(\Sigma)$,
and for all pair $g$, $g'$ of elements of $\G$, $(g \circ
g')_* = g_* \circ g'_*$. And, this action is smooth. Now,

 \begin{enumerate} 
 \item The pushforward action of $\G$ on
$\Omega^1(\Sigma)$ preserves the 1-form $\lambda$, and
thus the 2-form $\omega$. For all $g \in \G$,
$(g_*)^*(\lambda) = \lambda$, and $(g_*)^*(\omega) =
\omega$. So, the action of $\G$ is exact, $\sigma = 0$,
and hamiltonian, $\Gamma = \{0\}$.
 \item The moment maps are, up to a constant, equal to
the moment $\mu$, given by
 $$ 
 \mu(\alpha)(\P)_r(\delta r) = \undemi \int_\Sigma
\alpha \wedge \P(r)^*\bigg({\partial \P(r)_*(\alpha) \over
\partial r}(\delta r)\bigg),
 $$
for all $\alpha \in \Omega^1(\Sigma)$, for all $n$-plot
$\P$, where $r \in \dom(\P)$ and $\delta r \in
\RR^n$. In particular, applied to any 1-plot $\F$
centered at the identity $\id_\G$, that is $\F(0) =
\id_\G$, we get the special expression
 $$
\mu(\alpha)(\F)_0(1) = - \undemi \int_\Sigma
\alpha \wedge \DLie_\F(\alpha) = - \int_\Sigma
i_\F(\alpha) \times d\alpha,
 $$
where $\DLie_\F(\alpha)$ is the Lie derivative of
$\alpha$ along $\F$, and
$i_\F(\alpha)$  the contraction of $\alpha$ by
$\F$.
 \end{enumerate}
So, we find again, through the diffeological formalism
of the moment map, what is asserted informally in the
literature. The vague assertion \og the moment map of the
group of diffeomorphism is the Lie derivative\\fg makes
here sense.  \end{article}

\begin{proof}
1)  Let us compute the pullback of $\lambda$ by the
action of $g \in \G$, that is $(g_*)^*(\lambda)$. Let $\P
: \U \to \Omega^1(\Sigma)$ be a $n$-plot, let $r \in \U$,
and $\delta r \in \RR^n$. We have,
 \begin{eqnarray*}  (g_*)^*(\lambda)(\P)_r(\delta r) & = &
\lambda(g_* \circ \P)_r(\delta r) \\
 & = & \undemi \int_\Sigma g_*(\P(r)) \wedge {\partial
g_*(\P(r)) \over \partial r}(\delta r)\\
 & = & \undemi \int_\Sigma g_*(\P(r)) \wedge
g_*\bigg({\partial \P(r) \over \partial r}(\delta
r)\bigg) \\
 & = & \undemi  \int_\Sigma g_*\bigg(\P(r)\wedge
{\partial \P(r) \over \partial r}(\delta r)\bigg) \\
 & = & \undemi  \int_{g^*(\Sigma)} \P(r)\wedge {\partial
\P(r) \over \partial r}(\delta r) \\
 & = & \undemi \int_\Sigma \P(r)\wedge {\partial \P(r)
\over \partial r}(\delta r) \\
 & = & \lambda (\P)_r(\delta r)
 \end{eqnarray*}
Thus, $\lambda$ is invariant by $\G$, and so do $\omega =
d\lambda$.

2) Since the 1-form $\lambda$ is invariant by the
action of $\G$, we can use directly the
results of the exact case detailed in
\art{The-exact-case}. Thus, the moment maps are, up to a
constant, equal to $\mu : \alpha \mapsto \hat
\alpha^*(\lambda)$. So, let $\P : \U \to \G$ be a
$n$-plot, let $r \in \U$ and $\delta r \in \RR^n$. We
have,
 \begin{eqnarray*}
 \mu(\alpha)(\P)_r(\delta r) & = &
\alpha^*(\lambda)(\P)_r(\delta r) \\
 & = & \lambda(\hat \alpha \circ \P)_r(\delta r) \\
 & = & \lambda(r \mapsto \P(r)_*(\alpha))_r(\delta r) \\
 & = & \undemi \int_\Sigma \P(r)_*(\alpha) \wedge
{\partial \P(r)_*(\alpha) \over \partial r}(\delta r) \\
 & = & \undemi \int_\Sigma
\alpha \wedge \P(r)^*\bigg({\partial \P(r)_*(\alpha) \over
\partial r}(\delta r)\bigg).
 \end{eqnarray*}
 Now, let $\P = \F$ be a $1$-plot centered at the
identity, $\F(0) = \id_\G$. Let us change the variable $r$
for the variable $t$. The previous expression, computed
at $t = 0$ and applied to the vector $\delta t = 1$ gives
immediately
 \begin{eqnarray*}
 \mu(\alpha)(\F)_0(1) & = & \undemi \int_\Sigma
\alpha \wedge \left.{\partial \F(t)_*(\alpha) \over
\partial t}\right|_{t=0}.
 \end{eqnarray*}
 But, by definition of the Lie derivative, we have
  $$ \bigg\{{\partial \F(t)_*(\alpha) \over \partial
t} \bigg\}_{t = 0} = \bigg\{ {\partial
(\F(t)^{-1})^*(\alpha) \over \partial t}\bigg\}_{t = 0} =
- \DLie_\F(\alpha).
 $$ 
So, we get
the first expression of the moment map $\mu$ applied to
$\F$
  $$
\mu(\alpha)(\F)_0(1) = - \undemi \int_\Sigma
\alpha\wedge\DLie_\F(\alpha).
 $$
 Now, on a surface $\alpha \wedge d\alpha = 0$,
and $i_\F(\alpha \wedge d\alpha) =
i_\F(\alpha) \times d\alpha - \alpha \wedge
i_\F(d\alpha)$. So,  $i_\F(\alpha) \times d\alpha = 
\alpha \wedge i_\F(d\alpha)$. Then, using the
Cartan-Lie formula
$\DLie_\F(\alpha) = 
i_\F(d\alpha) + d(i_\F(\alpha))$, we
get
 \begin{eqnarray*}
\int_\Sigma\alpha\wedge\DLie_\F(\alpha) & = & 
\int_\Sigma\alpha\wedge[i_\F(d\alpha) +
d(i_\F(\alpha))] \\
 & = & \int_\Sigma i_\F(\alpha)d\alpha + \int_\Sigma
\alpha \wedge d(i_\F(\alpha)) \\
 & = & \int_\Sigma i_\F(\alpha)d\alpha + \int_\Sigma
i_\F(\alpha)d\alpha - \int_\Sigma d[\alpha \wedge
i_\F(\alpha)] \\
 & = & 2 \int_\Sigma i_\F(\alpha)d\alpha
 \end{eqnarray*}
And finally, we have the second expression for the moment
map:
 $$ \mu(\alpha)(\F)_0(1) = - \int_\Sigma
i_\F(\alpha) \times d\alpha,
  $$
for any $1$-plot of the group of positive
diffeomorphisms of the surface $\Sigma$, centered at the
identity. \end{proof}

\begin{article}[On the intersection 2-form of a surface
III] \label{On-the-intersection-form-of-a-surface-III}
We continue again with the example of 
\art{On-the-intersection-form-of-a-surface-I}, using the
same notations. Let us consider the space
$\Omega^1(\Sigma)$ as an additive group acting onto
itself by translations. Let us denote by $\et_\beta$ the
translation $\et_\beta : \alpha \mapsto \alpha + \beta$,
where $\alpha$ and $\beta$ belong to $\Omega^1(\Sigma)$.
 \begin{enumerate}
 \item The 2-form $\omega$ is invariant by translation.
That is, $\et_\alpha^*(\omega) = \omega$ for all $\alpha
\in \Omega^1(\Sigma)$. This action of
$\Omega^1(\Sigma)$ onto itself is hamiltonian but not
exact.
 \item The moment maps of the additive
action of $\Omega^1(\Sigma)$ onto itself are equal, up to
a constant to
 $$
 \mu : \alpha \mapsto d\left[\beta \mapsto \int_\Sigma
\alpha \wedge \beta \right].
 $$ 
In other words, $\mu(\alpha) = d[\omega(\alpha)]$, where
$\omega$ is regarded as the smooth linear function
$\omega(\alpha) : \beta \mapsto \omega(\alpha,\beta)$,
defined on $\Omega^1(\Sigma)$. Moreover, the moment map
$\mu$ is linear and injective.
 \item  The moment
map $\mu$ is its own Souriau cocycle, $\theta = \mu$. The
moment map $\mu$ identifies $\Omega^1(\Sigma)$ with the
$\theta$-affine coadjoint orbit of $0 \in
\Omega^1(\Sigma)^*$. Be aware that
$\Omega^1(\Sigma)^*$ denotes the space of invariant
1-forms of the abelian group $\Omega^1(\Sigma)$, and not
its algebraic dual. 
 \end{enumerate}
{\sc Note} --- This situation is analogous to what
happens for finite dimension symplectic vector spaces. The
2-form $\omega$ can be regarded as a real 2-cocycle of the
additive group $\Omega^1(\Sigma)$. This cocycle build up a
central extension by $\RR$,
 $$
 (\alpha,t) \cdot (\alpha',t') = \bigg(\alpha + \alpha',
t+t' + \int_\Sigma \alpha \wedge \alpha'\bigg)
 $$ 
 for all $(\alpha,t)$ and $(\alpha',t')$ in
$\Omega^1(\Sigma) \times \RR$. This central extension
acts on $\Omega^1(\Sigma)$, preserving $\omega$. This
action is hamiltonian, but now exact. The lack of
equivariance, characterized by Souriau's class, has
been absorbed in the extension. This group could be named
as the {\em Heisenberg group} of the oriented surface
$(\Sigma,\Surf)$.

Note also that, according to
\art{Symplectic-homogeneous-diffeological-spaces}, the
space $\Omega^1(\Sigma)$ equipped with the 2-form
$\omega$ is an homogeneous symplectic space. Thus, we have
a first simple example of infinite dimensional {\em
symplectic diffeological space}, avoiding any
discussion on the \og kernel\\fg of $\omega$.  
 \end{article}

\begin{proof}
Let us compute the pullback of $\lambda$ by a
translation. Let $\P : \U \to \X$ be a $n$-plot, let $r
\in \U$, and $\delta r \in \RR^n$. We have, 
 \begin{eqnarray*} 
 \et_\alpha^*(\lambda)(\P)_r(\delta r)
 & = & \lambda(\et_\alpha \circ \P)_r(\delta r) \\
 & = & \lambda[r \mapsto \P(r) + \alpha]_r(\delta r) \\
 & = & \undemi \int_\Sigma (\P(r)+\alpha) \wedge
{\partial (\P(r) +\alpha) \over \partial r}(\delta r)\\ &
= & \undemi \int_\Sigma \P(r) \wedge {\partial
 \P(r) \over \partial r}(\delta r) + \undemi \int_\Sigma
\alpha \wedge {\partial \P(r) \over \partial r}(\delta r)
\\
 & = & \lambda(\P)_r(\delta r) + d\bigg[\beta \mapsto 
\undemi \int_\Sigma \alpha \wedge \beta
\bigg](\P)_r(\delta r)
 \end{eqnarray*}
So, let us define, for all $\alpha \in \Omega^1(\Sigma)$,
the smooth real function $\F(\alpha)$ by
 $$ 
 \F(\alpha) : \beta \mapsto \undemi \int_\Sigma
\alpha \wedge \beta.
 $$
Such that
  $$ \et_\alpha^*(\lambda) = \lambda +
d(\F(\alpha)) \qmbox{and} \et_\alpha^*(\omega) = \omega.
 $$
Then, $\Omega^1(\Sigma)$, as an additive group, acts on
itself by automorphisms. Let us compute the moment maps.
Let $p$ be a path of $\Omega^1(\Sigma)$, connecting
$\alpha_0$ to $\alpha_1$. We have
 \begin{eqnarray*}
 \Psi(p) & = & \hat \alpha_1^*(\lambda) - \hat
\alpha_0^*(\lambda) - d\bigg[\beta \mapsto  \int_p d
(\F(\beta))\bigg] \\
 & = & \hat \alpha_1^*(\lambda) - \hat
\alpha_0^*(\lambda) - d[\beta \mapsto 
\F(\beta)(\alpha_1) - \F(\beta)(\alpha_0)] \\
 & = & \{\alpha_1^*(\lambda) - d[\beta \mapsto 
\F(\beta)(\alpha_1)]\} - \{\alpha_0^*(\lambda) - d[\beta
\mapsto  \F(\beta)(\alpha_0)]\} \\
 & = & \{\hat \alpha_1^*(\lambda) + d(\F(\alpha_1))\} -
\{\hat \alpha_0^*(\lambda) + d(\F(\alpha_0))\}.
 \end{eqnarray*}
 So, the 2-points moment map
\art{Definition-of-the-2-points-moment-map} is given
by $\psi(\alpha_0,\alpha_1) = \Psi(p)$. Now, the
moment maps are, up to a constant equal to 
 $$
 \mu(\alpha) = \psi(0,\alpha) =
\hat \alpha_1^*(\lambda) + d(\F(\alpha)) - \hat
0^*(\lambda).
 $$ 
 But, for any plot $\P: \U \to \Omega^1(\Sigma)$, we have
 \begin{eqnarray*}
 \hat\alpha^*(\lambda) (\P) -
\hat 0^*(\lambda) (\P) & = &
\lambda(\hat \alpha \circ \P) - \lambda(\hat 0 \circ
\P) \\
 & = & \lambda (r
\mapsto \P(r) +\alpha) - \lambda (r \mapsto \P(r)) \\
 & =
&  d\bigg[\beta \mapsto \undemi \int_\Sigma
\alpha \wedge \beta \bigg](\P) \\
 & = & d(\F(\alpha))(\P).
 \end{eqnarray*}
 Thus, $\hat \alpha^*(\lambda) (\P) -
\hat 0^*(\lambda) = d(\F(\alpha))$ and the
moment map $\mu$ is finally given by
 $$
\mu(\alpha) = 2 d(\F(\alpha)) = d \bigg[\beta \mapsto 
\int_\Sigma \alpha \wedge \beta \bigg].
 $$
 The moment map $\mu$ is not equivariant, and 
Souriau's cocycle $\theta$ is given by,
 $$
 \mu(\et_\alpha^*(\beta)) = \mu(\alpha + \beta) 
 = \mu(\beta) + \theta(\alpha)  \qmbox{with} 
\theta(\alpha) = \mu(\alpha).
 $$
So, the moment map $\mu$ is clearly smooth and
linear. Let $\alpha \in \ker(\mu)$,  $\mu(\alpha) = 0$ if
and only if $d(\F(\alpha)) = 0$,
that is if and only if $\F(\alpha) = \const =
\F(\alpha)(0) = 0$. Thus, $\F(\alpha)(\beta) = 0$ for any
$\beta \in \Omega^1(\Sigma)$, hence $\alpha =
0$. Therefore, the moment map $\mu$ is injective.
\end{proof}

\begin{article}[On symplectic irrational tori]
\label{On-symplectic-irrational-tori}
Let us consider the numerical space $\RR^n$, for some
integer $n$. For all $u \in \RR^n$, let us denote by
$\et_u$ the translation by $u$. That is, $\et_u : x
\mapsto x + u$. Let $\omega$ be a 2-form of $\RR^n$
invariant by translations. That is, for all $u \in \RR^n$,
$\et_u^*(\omega) = \omega$. Thus, $\omega$ is a constant
bilinear 2-form, necessarily closed, $d\omega = 0$. Let
us consider the moment maps associated to the
translations $(\RR^n,+)$. Since $\RR^n$ is simply
connected, the holonomy vanishes,
$\Gamma = \{0\}$. Let $p$ be a path of $\RR^n$
connecting  $x = p(0)$ to $y=p(1)$, the paths moment map
$\Psi(p)$,
and the 2-points moment map $\psi(p)$ are
given by
 $$
 \Psi(p) = \psi(x,y) =
\omega(y-x),
 $$
 where $\omega(u)$ is regarded as the linear 1-form
$\omega(u) : v \mapsto \omega(u,v)$. So, the moment maps
are, up to constant, equal to the linear map
 $$ 
 \mu : x \mapsto \omega (x).
 $$
 And therefore, Souriau's cocycle $\theta$ associated
to $\mu$ is equal to $\mu$. For all $u \in \RR^n$,
 $$
 \theta(u) = \mu(u) = \omega(u).
 $$
Let us consider
now a discrete diffeological subgroup $\K \subset
\RR^n$. Let us denote by $\Q$ the quotient $\Q
= \RR^n/\K$ and by $\pi : \RR^n \to \Q$ the canonical
projection. Let us continue to denote by $\et_u$ the
translation on $\Q$, by $u \in \RR^n$. That is
$\et_u(q) = \pi(x+u)$ for any $x$ such that $q =
\pi(x)$. Now, since $\omega$ is invariant by translations,
$\omega$ is invariant by $\K$, and since $\K$ is
discrete, $\omega$ projects on $\Q$ as a
$\RR^n$-invariant closed 2-form denoted by $\omega_\Q$.
That is, 
 $$
 \omega_\Q = \pi_*(\omega)
\qmbox{or} \omega = \pi^*(\omega_\Q).
 $$
Note that, the translation by any vector $u$ of $\RR^n$ on
$\Q$ is still an automorphism of $\omega_\Q$, that is
$\et_u^*(\omega_\Q) = \omega_\Q$. 
 \begin{enumerate}
\item The holonomy $\Gamma_\Q$ of the action of
$(\RR^n,+)$ on $(\Q,\omega_\Q)$ is the image of the
subgroup $\K$ by $\mu$. 
 $$
 \Gamma_\Q = \mu(\K), \quad \Gamma_\Q \subset
\RR^{n*}.
 $$
Thus, if $\omega \neq 0$ and if $\K$ is not reduce to
$\{0\}$, then the action of $(\RR^n,+)$ on
$(\Q,\omega_\Q)$ is not hamiltonian and not exact.
 \item The moment map $\mu : \RR^n \to \RR^{n*}$ projects
on a moment $\mu_\Q$ such that the following diagram
commutes.
 $$ 
\begin{tikzcd}
   \ \RR^n \arrow[r, "\mu"] \arrow[d, "\pi"'] & \RR^{n*} \arrow[d, "\pr"] \\
   \  \Q = \RR^n/\K \arrow[r, "\mu_\Q"'] & \RR^{n*}/\mu(\K)
\end{tikzcd}
 $$
That is, for all $q \in \Q$, $\mu_\Q(q) =
\pr(\omega(x))$ for any $x$ such that $q
= \pi(x)$. Souriau's cocycle $\theta_\Q$ associated
to $\mu_\Q$, for all $u \in \RR^n$, is given by
 $$
 \theta_\Q(u) = \mu_\Q(\pi(u)).
 $$
 So, if we consider the space $\Q$ as an additive group
acting on itself by translations, then the moment map
$\mu_\Q$, of this action, coincide with its Souriau
cocycle $\theta_\Q$.
 \item  The map $\mu$ is a fibration onto its image
whose fiber is the kernel of $\mu$. That is
$\Values(\mu) \simeq \RR^n/\E$, $\E = \ker(\mu)$. And, the
map $\mu_\Q$ is a fibration onto its image
$\mu(\RR^n)/\mu(\K)$ whose fiber is $\ker(\mu_\Q) =
\E/(\K \cap \E)$. If $\omega: \RR^n \to \RR^{n*}$ is
injective (which implies that $n$ is even) then the
moment map $\mu_\Q$ is a diffeomorphism which identifies
$\Q$ with its image $\RR^{n*}\!/\mu(\K)$. 
\end{enumerate}
{\sc Note 1} --- Regarded as a group $\Q = \RR^n\!/{\K}$
acts onto itself by projection of the translations of
$\RR^n$. Since the pullback by $\pi: \RR^n \to \Q$ is an
isomorphism from $\cQ^*$ to
 $\RR^{n*}$ ($\RR^n$ is the
universal covering of $\Q$), the moment maps computed
above give the moment maps associated to this
action. 

{\sc Note 2} --- This construction applies to the torus
$\T^2 = \RR^2/\ZZ^2$. The action of $(\RR^2,+)$, is
obviously not hamiltonian, but the moment map
$\mu_{\T^2}$ is well defined. And, $\mu_{\T^2}$
identifies $\T^2$ with the quotient of $\RR^{2*}$ --- the
$(\Gamma_\Q,\theta_\Q)$-coadjoint orbit of the
point $0$ --- by the holonomy $\Gamma_\Q = \omega(\ZZ^2)
\subset \RR^{2*}$. In the meaning we gave above of the
notion of coadjoint orbit, the torus $\T^2$, equipped
with the standard symplectic form $\omega$, is a
coadjoint orbit of $\RR^2$, or even a coadjoint orbit of
itself. This is a special case of the the
\art{Moment-maps-for-symplectic-manifolds} discussion.

{\sc Note 3} --- All this construction above can be also
applied to situations which are regarded as more singular
that the simple quotient of $\RR^n$ by a lattice. It can
by applied as well to the product of any irrational tori.
An ($n$-dimensional) irrational torus $\T_\K$ is the
quotient of $\RR^n$ by any generating discrete strict
subgroup $\K$ of $\RR^n$. See for example \cite{IL90} for
an analysis of 1-dimensional irrational tori. For
example, we can consider the product of two 1-dimensional
irrational torus $\Q = {\T}_{\H} \times {\T}_{\K}$,
quotient of $\RR^2 = \RR\times \RR$ by the discrete
subgroup $\alpha_{\H}(\ZZ^p) \times \alpha_{\K}(\ZZ^q)$,
where $\alpha_\H : \RR^p \to \RR$ and $\alpha_\K : \RR^q
\to \RR$ are two linear 1-forms. In
this case, the moment map $\mu_\Q$ will also identify
${\T}_{\H} \times {\T}_{\K}$ with the quotient of
$\RR^{2*}$ --- $(\Gamma_\Q,\theta_\Q)$-coadjoint orbit of
$0$ --- by $\Gamma_\Q = \omega(\alpha_{\H}(\ZZ^p) \times
\alpha_{\K}(\ZZ^q))$. This is the simplest example of
{\em totally irrational symplectic space}, and {\em
totally irrational coadjoint orbit}. Note that, these
cases escape completely to the usual analysis, but  also
to the analysis in terms of Sikorski's or Fr\"olicher's
spaces. 
 \end{article}

\begin{proof}
First of all, the fact that there exists a closed 2-form
$\omega_\Q$ on $\RR/\K$ such that
$\pi^*(\omega_\Q) =\omega$ is an application of the
criterion of pushing forward forms, in the special case
of a covering \cite{Piz05}. Now, the computation of the
moment map of a linear antisymmetric form $\omega$ on
$\RR^n$ is well know, and independently of the method
gives the same result $\mu(x) = \omega(x)$. The additive
constant is fixed here by the condition $\mu(0)=0$. But,
the value of the paths moment map $\Psi(p)$ can be found
as well by the method described above, applying the
particular expression 
  $$
 \eK\omega_p(\delta p) = \int_0^1
\omega_{p(t)}(\dot p(t),\delta
p(t)) dt \qmbox{with} \dot p(t) ={d p(t) \over dt}.
 $$
of the chain-homotopy operator for manifold. Where $p$ is
a path and $\delta p$ is a \og variation\\fg of $p$.  So,
since the result depends only on the ends of the path,
let us choose, for any points $x$ and $y$ in $\RR^n$, the
connecting path $p : t \mapsto x + t(y-x)$. Let us remind
that $\Psi(p) = \hat p^*(\eK\omega)$. Let $u$ and $\delta
u$ in $\RR^n$. Note that $\hat p_*(\et_u) = \et_u \circ p
= [t \mapsto p(t) + u]$. So,  
 \begin{eqnarray*}
 \Psi(p)_u(\delta u) & = & \hat p^*(\eK\omega)_u(\delta u)
\\
 & = & ({\eK}\omega)_{\et_u \circ p}
(\delta(\et_u \circ p)), \qmbox{with}
\delta p = 0 \\
 & = & \int_0^1
\omega(\dot p(t),
\delta u) \ d\, t \\ & = &
\omega(y-x,\delta u)
 \end{eqnarray*} 
 So $\Psi(p) = \psi(x,y) = \omega(y-x)= \omega(y) -
\omega(x)$. And, $\mu : x \mapsto \omega(x)$, for all
$x$ in $\RR^n$. 

Now, let us consider $\omega_\Q$. Since $\RR^n$ is the
universal covering of $\Q$, every loop $\ell \in
\Loops(\Q,0)$ can be lifted into a path $p$ of $\RR^n$
starting at $0$ and ending in $\K$. In other words,
 $$
 \Gamma =  \left\{ \Psi (\ell) \mid \ell \in \Loops(\Q)
\right\} = \left\{ \Psi (t \mapsto tk) \mid k \in \K
\right\} = \omega (\K)
 $$ 
The other propositions are then a direct application of
the functoriality of the moment map
described in \art{Pushing-forward- moment-maps}, and
standard analysis on quotients and fibrations.
 \end{proof}

\begin{article}[The corner orbifold]
\label{The-corner-orbifold}
Let us consider the quotient $\cQ$ of $\RR^2$ by the
action of the finite subgroup $\K \simeq \{\pm1\}^2$,
embedded in $\GL(2,\RR)$ by 
 $$
 \K = \bigg\{ \mymatrix{\varepsilon & 0 \cr 0 &
\varepsilon'} \bigg| \ \varepsilon, \varepsilon' \in
\{\pm1\} \bigg\}.
 $$
 The space $\cQ = \RR^2/\K$ is an orbifold, according to
\cite{IKZ05}. It is diffeomorphic to the quarter space
$[0,\infty[ \times [0,\infty[ \subset \RR^2$, equipped
with the pushforward of the standard diffeology of
$\RR^2$ by the map 
 $\pi : \RR^2 \to [0,\infty[ \times [0,\infty[$, defined
by,
 $$
 \pi(x,y) = (x^2,y^2)
\qmbox{and} \cQ \simeq \pi_*(\RR^2).
 $$
 %
%%###########
\begin{figure}[th]
  \centering
  \includegraphics[width=0.75\textwidth]{fig-corner-orbifold.pdf}
  \caption{The corner orbifold}
  \label{fig-corner-orbifold}
\end{figure}
%%###########
%
So the letter $\cQ$ will denote indifferently the
quotient $\RR^2/\K$ or the quarter space
$\pi_*(\RR^2)$. And the meaning of the letter $\pi$ 
follows. Now, let us remark that, the decomposition of
$\cQ$ in terms of point's structure is
given by,
 $$
 \Str(0,0) = \{\pm1\}^2, \quad \Str(x,0) = \Str(0,y) =
\{\pm1\} \qmbox{and} \Str(x,y) = \{1\},
 $$
where $x$ and $y$ are positive real numbers. So, since
the structure of a point is preserved by diffeomorphisms
\cite{IKZ05}, there are at least three orbits of
$\Diff(\cQ)$, the point $0_\cQ=(0,0)$, the regular
stratum $\dot \cQ = ]0,\infty[^2$ and the union of the
two axes, $ox$ and $oy$. So, in particular any
diffeomorphism of $\cQ$ preserves the origin $0_\cQ$.
Actually, these are exactly the orbits of
$\Diff(\cQ)$. Let us remark that, $\dim(\cQ) =
2$ \cite{Piz06-b}. So, every 2-form is closed. Now,

1) Every 2-form of $\cQ$ is proportional to the 2-form
$\omega$ defined on $\cQ$ by
 $$
 \pi^*(\omega) : \mymatrix{x \cr y} \mapsto 4 xy \times dx
\wedge dy.
 $$
 That is, for any other 2-form $\omega'$ there exists a
smooth function $\phi \in \Cinfty(\cQ,\RR)$ such that
$\omega' = \phi \times \omega$.

2) The space $(\cQ,\omega)$ is
hamiltonian $\Gamma_\omega = \{0\}$. And, the action of
$\G_\omega$ is exact, that is $\sigma_\omega =
0$. In particular, the
universal moment map $\mu_\omega$ defined by
$\mu_\omega(0_\cQ) = 0$, is equivariant.

3) The universal equivariant moment map
$\mu_\omega$ vanishes on the singular strata $\{0\}$, $ox$
and $oy$, and is injective on the regular stratum $\dot
\cQ$. So, the image $\mu_\omega(\cQ)$ is diffeomorphic
 to an open disc with a point attached on the boundary.
 \end{article}

\begin{proof} 1) Let $\omega'$ be a 2-form on $\cQ$ and
let $\tilde \omega'$ be its pullback by $\pi$, $\tilde
\omega' = \pi^*(\omega')$. So, there exists a smooth real
function $\F$ such that $\tilde \omega' = \F \times dx
\wedge dy$. But, since $\pi \circ k = \pi$, for all $k
\in \K$ we get $\varepsilon \varepsilon' \F(\varepsilon x,
\varepsilon' y) = \F(x,y)$, for all $(x,y) \in \RR^2$ and
all $\varepsilon$, $\varepsilon'$ in $\{\pm1\}$. Thus, 
$\F(-x,y) = -\F(x,y)$ and $\F(x,-y) = -\F(x,y)$. In
particular, $\F(0,y) = 0$ and $\F(x,0) = 0$. Therefore,
since $\F$ is smooth, there exists $f \in
\Cinfty(\RR^2,\RR)$ such that $\F(x,y) = 4xyf(x,y)$, with
$f(\varepsilon x,
\varepsilon' y) = f(x,y)$. Therefore,
$\tilde \omega' = f \times \tilde \omega$, with
$\tilde \omega = 4xy \times dx \wedge dy$. Now $\tilde
\omega = d(x^2) \wedge d(y^2)$, but $x\mapsto x^2$ and $y
\mapsto y^2$ are invariant by $\K$ so, they are the
pullback by $\pi$ of some smooth real functions on
$\cQ$. Thus, $d(x^2)$ and $d(y^2)$ are the pullback of 
1-forms on $\cQ$, let us say $d(x^2) = \pi^*(ds)$ and
$d(y^2) = \pi^*(dt)$, so $\tilde \omega =
\pi^*(\omega)$, where $\omega = ds \wedge dt$ is a well
defined 2-form on $\cQ$. Now, since $f(\epsilon x,
\epsilon'y) = f(x,y)$ means just that $f$ is the pullback
of a smooth real function $\phi$ on $\cQ$, it follows
that any 2-form $\omega'$ on $\cQ$ is proportional to
$\omega$, that is $\omega' = \phi \times \omega$, with
$\phi \in \Cinfty(\cQ,\RR)$.

2) The orbifold is contractible. The deformation
retraction $(s,x,y) \mapsto (sx,sy)$ of $\RR^2$ to
$\{(0,0)\}$ projects on a smooth deformation retraction of
$\cQ$. So, there is no holonomy, $\Gamma = \{0\}$.
Now, since the origin $0_\cQ$ is the only point with
structure $\{\pm1\}$, every diffeomorphism of $\cQ$
preserves the origin $0_\cQ$. So, the 2-point moment map
is exact, see the note 2 of \art{Souriau-s-cocycles}, 
Souriau's cocycle vanishes, $\sigma_\omega = 0$. Let $q$
be any point of $\cQ$ and let $\mu_\omega(q) =
\psi(0_\cQ,q)$. This is an equivariant moment map and 
$\mu_\omega(0_\cQ) =  \psi(0_\cQ,0_\cQ) =0$.

3) Let $q \in \cQ$, thus $\mu_\omega(q) =
\Psi(p)$ for any path $p$
connecting $0_\cQ$ to $q$. Now, let $q$ belongs to a
semi-axis $ox$ or $oy$, and let us choose $p = t \mapsto
\lambda(t)q$, where $\lambda$ is a smashing function
equal to $0$ on $]-\infty,0]$ and equal to $1$ on
$[1,+\infty[$. Thus for all $t \in \RR$, $p(t)$ belongs
to the same semi-axis than $q$. Thanks to the expression
$\heartsuit$ of \art{Evaluation-of-the-paths-moment-map},
we have for any 1-plot $\phi$ of
$\Diff(\cQ,\omega_\omega)$, centered at the identity, 
 $$
 \Psi(p)(\phi)_0(1) =
\int_0^1 \omega \left[ \mymatrix{s \cr r} \mapsto
\phi(r) (\lambda(s + t)q) \right]_{\left({0 \atop
0}\right)}\mymatrix{1 \cr 0} \mymatrix{0 \cr 1} dt,
 $$
 But, now $(s,r) \mapsto
\phi(r) (\lambda(s + t)q)$ is a plot of the semi-axis, and
thanks to the item 1,
the form $\omega$ vanishes on the semi-axis. So, the
integrand vanishes and $\Psi(p)(\phi)_0(1) = 0$. Now,
since 1-forms are characterized by 1-plots and since
momenta are characterized by centered plots,
$\mu_\omega(q) = 0$ for all $q \in \cQ$ belonging to any
semi-axis. 

On the other hand, let $q$ and $q'$ be two points of
the regular stratum $\dot\cQ$. Since $\pi \restriction
\{ (x,y) \mid x>0 \ \& \ y>0 \}$ is a diffeomorphism, and
since $\tilde \omega \restriction \{ (x,y) \mid x>0 \ \& \
y>0 \}$ is symplectic there exists always a
symplectomorphism $\phi$ with compact support $\cS \subset
\{ (x,y) \mid x>0 \ \& \ y>0 \}$ which exchange $q$ and
$q'$. So, the image of this diffeomorphism on $\dot\cQ$
can be extended by the identity on the whole $\cQ$.
Therefore, the automorphisms of $\omega$ are transitive on
the regular stratum.
 \end{proof}

\begin{article}[The cone orbifold]
\label{The-cone-orbifold}
Let $\cQ_m$ be the quotient  of the
smooth complex plane $\CC$ by the
action of the cyclic subgroup
 $$\Z_m \simeq \{\zeta \in
\CC \mid \zeta^m = 1\} \qmbox{with} m>1.
 $$
 The space $\cQ_m$ is an orbifold, according to
\cite{IKZ05}. We identify $\cQ_m$ to the complex
plane $\CC$, equipped with the pushforward of the standard
diffeology by the map 
 $\pi_m : z \mapsto z^m$. That is, a plot of
$\cQ_m$ is any parametrization $\P$ of $\CC$ which writes
locally $\P(r) = \phi(r)^m$, where $\phi$ is a smooth
parametrization of $\CC$.
%
%%###########
\begin{figure}[th]
  \centering
  \includegraphics[width=0.75\textwidth]{fig-cone-orbifold.pdf}
  \caption{The cone orbifold}
  \label{fig-cone-orbifold}
\end{figure}
%%###########
%
 Let us remark first that the decomposition of
$\cQ_m$, in terms of structure group, is
given by
 $$
 \Str(0) = \Z_m, \qmbox{and} \Str(z) = \{1\}
\qmbox{if} z \neq 0.
 $$
 And secondly that there is two orbits of
$\Diff(\cQ_m)$, the point $0$ and the regular
stratum $\dot \cQ_m = \CC - \{0\}$. In particular any
diffeomorphism of $\cQ_m$ preserves the origin $0$. It is
not difficult to check that $\dim(\cQ_m) =
2$ \cite{Piz06-b}, so every 2-form on $\cQ_m$ is closed.
Now,

1) Every 2-form of $\cQ_m$ is proportional to the 2-form
$\omega$ uniquely defined  by
 $$
 \pi_m^*(\omega) : z \mapsto dx
\wedge dy \qmbox{with} z = x +iy.
 $$
 That is, for any other 2-form $\omega'$ there exists a
smooth function $f \in \Cinfty(\cQ_m,\RR)$ such that
$\omega' = f \times \omega$.

2) The space $(\cQ,\omega)$ is
hamiltonian $\Gamma_\omega = \{0\}$. And, the action of
$\G_\omega$ is exact, that is $\sigma_\omega = 0$. In
particular, the universal moment map $\mu_\omega$ defined
by $\mu_\omega(0) = 0$, is equivariant.

3) The universal moment map
$\mu_\omega$ is injective. Its image is the reunion
of two coadjoint orbits, the point $0 \in
\cG^*_\omega$, value of the origin of $\cQ_m$, and the
image of the regular stratum $\dot \cQ_m$.
 \end{article}

\begin{proof}
Let us first prove that the usual surface form $\Surf =
dx \wedge dy$ is the pullback of a 2-form $\omega$
defined on $\cQ_m$. We shall apply the
standard criterion and prove that for any two plots
$\phi_1$ and $\phi_2$ of $\CC$ such that $\pi_m \circ
\phi_1 = \pi_m \circ \phi_2$ we have $\Surf(\phi_1) =
\Surf(\phi_2)$. That is, $\phi_1(r)^m = \phi_2(r)^m$
implies $\Surf(\phi_1) = \Surf(\phi_2)$. First of all let
us recall that, since we are dealing with 2-forms, is is
sufficient to consider 2-plots. So, let the $\phi_i$
be defined on some numerical domain $\U \subset \RR^2$.
Let $r_0 \in \U$, we split the problem into 2 cases.

1) $\phi_1(r_0) \neq 0$ --- Thus $\phi_2(r_0) \neq 0$,
there exists a open disk $\B$ centered at $r_0$ on which
the $\phi_i$ do not vanishes. Thus, the map $r
\mapsto \zeta(r) = \phi_2(r)/\phi_1(r)$ defined on $\B$ is
smooth with values in $\Z_m$. But, since $\Z_m$ is
discrete there exists $\zeta \in \Z_m$ such that
$\phi_2(r) = \zeta \times \phi_1(r)$ on $\B$. Now,
$\Surf$ is invariant by $\U(1) \supset \Z_m$. Therefore
$\Surf(\phi_1) = \Surf(\phi_2)$ on $\B$.

2) $\phi_1(r_0) = 0$ --- Thus, $\phi_2(r_0) = 0$. Now,
we have $\Surf(\phi_i) = \det(\D(\phi_i)) \times \Surf$,
where $\D(\phi_i)$ denotes the tangent map of $\phi_i$.
We split this case into two sub-cases:

2.a) $\D(\phi_1)_{r_0}$ is non-degenerate --- Thus, thanks
to the implicit function theorem, there exists a small
open disk $\B$ around $r_0$ where $\phi_1$ is a local
diffeomorphisms onto its image. Since $\phi_1(r)^m =
\phi_2(r)^m$, the common zero $r_0$ of both $\phi_1$ and
$\phi_2$ is isolated. Thus, the
map $r
\mapsto \zeta(r) = \phi_2(r)/\phi_1(r)$ defined on $\B -
\{r_0\}$ is smooth, and for the same reason than in the
first case, $\zeta$ is constant. So, $\phi_2(r) =
\zeta \times \phi_1(r)$  on  $\B - \{r_0\}$. But,
since $\phi_i(r_0) = 0$, this equality extends on
$\B$. Therefore $\Surf(\phi_1) = \Surf(\phi_2)$ on $\B$.

2.b) $\D(\phi_1)_{r_0}$ is degenerate --- Let
$u$ be in the kernel of $\D(\phi_1)_{r_0}$. We have
$\phi_1(r_0 + su)^m = \phi_2(r_0 + su)^m$ for enough small
real $s$. Then, differentiating this equality $m$ times
with respect to $s$, for $s=0$ we get
$ 0 = \D(\phi_1)_{r_0}(u)^m = \D(\phi_2)_{r_0}(u)^m$.
Therefore,  $\D(\phi_2)_{r_0}$ is also degenerate at
$r_0$ and thus $0 = \Surf(\phi_1)_{r_0} = 
\Surf(\phi_2)_{r_0}$.

So, we have proved that for any $r \in \U$,
$\Surf(\phi_1)_r = \Surf(\phi_2)_r$. Therefore, there
exists a 2-form $\omega$ on $\cQ_m$ such that
$\pi_m^*(\omega) = \Surf$, and this form $\omega$ is
completely defined by its pullback. Now, since the
pullback by $\pi_m$ of any other 2-form $\omega'$ on
$\cQ_m$ is proportional to $\Surf$, the form $\omega'$ is
proportional to $\omega$. 

Now, for the same reasons than in
\art{The-corner-orbifold} the universal holonomy
$\Gamma_\omega$ and Souriau's class $\sigma_\omega$
vanish, and the universal moment map $\mu_\omega$
defined by $\mu_\omega(0) = 0_{\cG^*}$ is equivariant.
Moreover, the regular stratum $\dot \cQ$ is just a
symplectic manifold for the restriction of $\omega$. Any
symplectomorphism with compact support which doesn't
contain $0$ can be extended to an automorphism of
$(\cQ,\omega)$. Thus, since the compactly supported
symplectomorphisms of a connected symplectic manifold are
transitive, the regular stratum $\dot \cQ$ is an orbit of
$\Diff(\cQ,\omega)$. Therefore, the moment map
$\mu_\omega$ maps $\cQ$ onto two orbits, $\{0_{\cG^*}\}$
and $\mu_\omega(\dot \cQ)$.
 \end{proof}

\begin{article}[The infinite projective space]
\label{The-infinite-projective-space}
This example of the symplectic structure of the
infinite projective space is extracted from
\cite{Piz06-a}, everything not proved here can be found
there. Let $\cH$ be the Hilbert space of the square
summable complex series.
 $$
 \cH = \bigg\{ \Z = (\Z_i)_{i=1}^\infty \ \bigg| \
\sum_{i=1}^n \Z_i \cdot \Z_i < \infty \bigg\}.
 $$
Where the dot denotes the hermitian product. The space
$\cH$ is equipped with the {\em fine structure} of
complex diffeological vector space. That
is, its diffeology is generated by the linear injections
from $\CC^n$ to $\cH$, or if we prefer, let $\P : \U \to
\cH$ be a plot, then for every $r_0 \in \U$, there exists
an integer $n$, an open superset $\V \subset \U$ of
$r_0$, a finite family $\cF = \{(\lambda_a,\Z_a)\}_{a \in
\A}$, where the $\Z_a \in \cH$, and the $\lambda_a \in
\Cinfty(\V,\CC^n)$ such that $\P \restriction \V : r
\mapsto \sum_{a \in \A} \lambda_a(r)  \times \Z_a$. Such
a family  $\{(\lambda_a,\Z_a)\}_{a \in \A}$ is called a
{\em local family} of $\P$ at the point $r_0$. We
defined the symbol $d\Z$ which associates to every local
family $\cF = \{(\lambda_a,\Z_a)\}_{a \in \A}$ defined on
the domain $\V$, the complex valued 1-form of $\V$
 $$
 d\Z(\cF) : r \mapsto \sum_{a \in \A}
d\lambda_a(r) \Z_a.
 $$
 For every $\lambda_a = x_a +i y_a$, where $x_a$ and
$y_a$ are real smooth parametrizations, $d\lambda_a = dx_a
+ i dy_a$. Now, there exists on $\cH$ a 1-form $\alpha$
defined by
 $$ \alpha = \unsurdeuxi[\Z\cdot d\Z
-d\Z\cdot \Z]. 
 $$
1) As an additive group $(\cH,+)$ acts on itself,
preserving $d\alpha$. Let $\Z \in \cH$ and let $\et_\Z$
be the translation by $\Z$, then $\et^*_\Z(d\alpha) =
d\alpha$. This action is hamiltonian but not exact. Let
$\mu$ be the moment map of the translations $(\cH,+)$,
defined by $\mu(0_\cH) = 0$. So 
 $$
\mu(\Z) = 2d[w(\Z)] \qmbox{with} w(\zeta) : \Z \mapsto
\unsurdeuxi [ \zeta \cdot \Z - \Z \cdot \zeta ] \in
\Cinfty(\cH,\RR).
 $$
 The moment map $\mu$ is injective and $(\cH,d\alpha)$ is
an homogeneous symplectic space.

2) Let $\UU(\cH)$ be the group of unitary transformations
of $\cH$, equipped with the functional diffeology. The
group $\UU(\cH)$ acts on $\cH$ preserving $\alpha$.  The
action of $\UU(\cH)$ on $(\cH,d\alpha)$ is exact and
hamiltonian. Let $\P : \U \to \UU(\cH)$ be a $n$-plot.
The value of the moment map $\mu$ of the action of
$\UU(\cH)$ on $(\cH,d\alpha)$, evaluated on $\P$ is given
by 
 $$
 \mu(\Z)(\P)_r(\delta r) =
\unsurdeuxi\bigg[ \P(r)(\Z) \cdot {\partial \P(r)(\Z)
\over \partial r}(\delta r) - {\partial \P(r)(\Z) \over
\partial r}(\delta r) \cdot \P(r)(\Z)\bigg], $$
where, $r \in \U$ , $\delta r \in \RR^n$ and: 
 $$
 \mbox{If} \quad \P(r)(\Z) =_{\rm loc}
\sum_{\alpha \in \A} \lambda_\alpha(r) \Z_\alpha,
\qmbox{then} {\partial \P(r)(\Z) \over \partial r}(\delta
r) =_{\rm loc} \sum_{\alpha \in \A} {\partial
\lambda_\alpha(r) \over \partial r}(\delta r) \Z_\alpha. 
 $$

3) The unit sphere $\cS \subset \cH$ is an homogeneous
space of  $\UU(\cH)$. The fibers of the equivariant
moment map $\mu$ of the action of $\UU(\cH)$ on
$(\cS,d\alpha \restriction \cS)$ are the fibers of the
infinite Hopf fibration $\pi : \cS \to \cP = \cS/\S^1$,
where $\S^1 \in \CC$ acts multiplicatively on $\cS$.
There exists a symplectic form $\omega$ on $\cP$, such
that $\pi^*(\omega) = d\alpha \restriction \cS$. The
equivariant moment map of the induced action of
$\UU(\cH)$ on $\cP$ is injective. So, the infinite
projective space $\cP$, equipped with the Fubini-Study
form, is an homogeneous symplectic space and can be
regarded as a coadjoint orbit of $\UU(\cH)$.
 \end{article}

\begin{proof} Many of what is asserted here has been
proved in \cite{Piz06-a}. So, we shall just check what is
not in this paper.

1) Since $\cH$ is contractible, there is no holonomy.
Now, let $\zeta \in \cH$ and $\et_\zeta$ be the
translation $\et_\zeta(\Z) = \Z + \zeta$. A direct
computation shows that, $\et_\zeta^*(\alpha) = \alpha +
d[w(\zeta)]$. Thus, $d\alpha$ is invariant by translation
$\et_\zeta^*(d\alpha) = d\alpha$. Now, let $p$ be any
path connecting $0_\cH$ to $\Z$, we have $\mu(\Z) =
\Psi(p) = \hat p^*\eK(d\alpha) = \hat \Z^*(\alpha) - \hat
0_\cH^*(\alpha) - d[\eK\alpha \circ \hat p]$. But, on one
hand we have $\hat \Z = \et_\Z$, thus $\hat \Z^*(\alpha)
- \hat 0_\cH^*(\alpha) = \et_\Z^*(\alpha) -
\id_\cH^*(\alpha) = \alpha + d[w(\Z)] - \alpha =
d[w(\Z)]$. And, on the other hand we have, $\hat p (\zeta)
= \et_\zeta \circ p$, and thus $\eK\alpha \circ \hat p =
\int_{\et_\zeta \circ p} \alpha = \int_p
\et_\zeta^*(\alpha) = \int_p \alpha + \int_p d[w(\zeta)]
= \int_p \alpha + w(\zeta)(\Z)$, since $w(\zeta)(0_\cH) =
0$. So, $\mu(\Z) = d[w(\Z)] - d[\zeta \mapsto
w(\zeta)(\Z)]$. But, $w(\zeta)(\Z) = - w(\Z)(\zeta)$ so
$\mu(\Z) = d[w(\Z)] - d[\zeta \mapsto -
w(\Z)(\zeta)] = 2d[w(\Z)]$. Now, let $\Z$ be in the
kernel of $\mu$, so $w(\Z) = \const = w(0_\cH) = 0$. But
$w(\Z)(\Z') = 0$ for all $\Z' \in \cH$ if and only if $\Z
= 0_\cH$, we have just to decompose $\Z$ into real and
imaginary parts and use the fact that the hermitian norm
on $\cH$ is not degenerated. Therefore, $\mu$ is
injective.

2)  Since the 1-form $\alpha$ is
invariant by $\UU(\cH)$, this statement is a direct
application of  \art{The-exact-case}.
 \end{proof}

\begin{article}[The Virasoro coadjoint orbits]
\label{The-Virasoro-coadjoint-orbit}
Let $\Imm(\S^1,\RR^2)$ be the space of all the
immersions of the circle $\S^1 = \RR/2\pi\ZZ$ into
$\RR^2$, equipped with the functional diffeology. For
every $n$-plot $\P : \U \to \Imm(\S^1,\RR^2)$ let us
defined the 1-form $\alpha(\P)$ on $\U$ by
 $$
\alpha(\P)_r(\delta r) = 
\int_0^{2\pi} 
{1 \over \norm{\P(r)'(t)}^2} 
\bigg\langle\P(r)''(t) \ \bigg| \ {\partial \P(r)'(t)
\over \partial r}(\delta r) \bigg\rangle \ dt.
 $$
for every $r
\in \U$ and $\delta r \in \RR^n$. Where the prime denotes
the derivative with respect to the parameter $t$, and the
bracket $\langle \cdot \mid \cdot \rangle$ denotes the
ordinary scalar product of the vector space $\RR^2$.
 \begin{itemize}
 \item[1.] As defined above, $\alpha$ is a 1-form of
$\Imm(\S^1,\RR^2)$.
 \end{itemize}
Let us consider now the group $\Diff_+(\S^1)$ of
positive diffeomorphisms of the circle, and its 
action on $\Imm(\S^1,\RR^2)$ by re-parametrization. For
every $\varphi \in \Diff_+(\S^1)$, for every $x \in
\Imm(\S^1,\RR^1)$, let us denote by $\bar \varphi (x)$
the pushforward of $x$ by $\varphi$,
 $$
 \bar \varphi(x) = \varphi_*(x) = x \circ \varphi^{-1}.
 $$
 And, let $\F :
\Diff_+(\S^1) \to \Cinfty(\Imm(\S^1,\RR^2),\RR)$ be the
map defined, for all $\varphi \in \Diff_+(\S^1)$, by
 $$
 \F(\varphi) : x \mapsto 
\int_0^{2\pi} \log \norm{x'(t)} \ d\log(\varphi'(t))
 $$
 \begin{itemize}
 \item[2.] The map $\F$ is smooth and for every $\varphi
\in \Diff(\S^1)$,
 $$
 \bar \varphi^*(\alpha) = \alpha - d[\F(\varphi)].
 $$
So, the 2-form $\omega = d\alpha$, defined on
$\Imm(\S^1,\RR^2)$, is closed and invariant by the action
of $\Diff(\S^1)$. Moreover, the action of $\Diff(\S^1)$
is hamiltonian.
  \item[3.] Let $x_0 : \class(t) \mapsto
(\cos(t),\sin(t))$ be the {\em standard immersion} from
$\S^1 = \RR/2\pi\ZZ$ to $\RR^2$. The moment maps for
$\omega$, of $\Diff_+(\S^1)$ on the connected
component of $x_0 \in \Imm(\S^1,\RR^2)$,
are translated by a constant from 
 $$
 \mu(x)(r \mapsto \varphi)_r(\delta r) = \int_0^{2\pi}
\left\{ { \norm{x''(u)}^2 \over \norm{x'(u)}^2} - {d^2
\over du^2} \log \norm{x'(u)}^2 \right\} \delta u \ du.
 $$
 Where $ r \mapsto \varphi$ is any plot of
$\Diff_+(\S^1)$ defined on some $n$-domain $\U$, $r$ is
a point of $\U$, $\delta r \in \RR^n$, $u =
\varphi^{-1}(t)$, and $\delta u = \D(r \mapsto
u)(r)(\delta r)$. 
\item[4.] With the same conventions as in item
3, Souriau's cocycles of the $\Diff_+(\S^1)$ action on
$\Imm(\S^1,\RR^2)$ are cohomologous to $\theta$ defined by,
 $$
 \theta(g)(r \mapsto \varphi)_r(\delta r) =
\int_0^{2\pi} {3 \gamma''(u)^2 -2 \gamma'''(u) \gamma'(u)
\over \gamma'(u)^2} \ \delta u \ du,
 $$  
 where $g \in \Diff_+(\S^1)$ and $\gamma = g^{-1}$. We
recognize the integrand of the right hand side as the
so-called Schwartzian derivative of $\gamma$.
 \item[5.] Let $\beta$ be the function for all $g$ and
$h$ in $\Diff_+(\S^1)$ by
 $$
 \beta(g,h) = \int_0^{2\pi} \log(g \circ h)'(t) \
d\log h'(t).
 $$
 So, for all $g$ and
$h$ in $\Diff_+(\S^1)$ we have
 $$
 \F(g \circ g') = \F(g)\circ \bar g' +
\F(g') - \beta(g,g').
 $$
 This function $\beta$ is known as {\em Bott's
cocycle} \cite{Bot78}. The
central extension of $\Diff_+(\S^1)$ by $\beta$ is the
so-called Virasoro group. Its action on
$\Imm(\S^1,\RR^2)$, through $\Diff_+(\S^1)$, is still
hamiltonian, but now exact. This is a well known
construction which will be not more developed here. 
 \end{itemize}
 This example which has been built on purpose
\cite{Igl95}, gathers the main ingredients found in the 
literature on the construction of Virasoro's group. I
regard this example as a nice illustration of the whole
theory.
 \end{article}

\begin{proof} The proof is actually a long and tedious
series of computations. To make it as clear as possible,
we shall split the computations in a few steps.

{\em The 1-form $\alpha$} --- We prove first that
$\alpha$ is a well defined 1-form on $\Imm(\S^1,\RR^2)$.
Let $ \F : \U \to\U$ be a smooth $m$-parametrization. We
have, for all $s \in \V$ and all $\delta s \in \RR^m$,
  $$
 \alpha(\P \circ \F)(s)(\delta s)
= \int_0^{2\pi}  {1 \over \norm{(\P \circ
\F)(s)'(t)}^2} \bigg\langle(\P \circ
\F)(s)''(t) \ \bigg| \ {\partial (\P \circ \F)(s)'(t)
 \over \partial s}(\delta s)\bigg\rangle \ dt 
 $$
That is,
 $$\alpha(\P \circ \F)(s)(\delta s) = \int_0^{2\pi} {1
\over \norm{\P(\F(s))'(t)}^2}
\bigg\langle \P(\F(s))''(t) \ \bigg | \ {\partial
\P(\F(s))'(t) \over \partial s}(\delta s) \bigg\rangle
\ dt.
 $$
Let us denote by $r$ the point $\F(s)$. We get,
 \begin{eqnarray*}
 \alpha(\P \circ \F)(s) (\delta s) & = & \int_0^{2\pi} 
{1 \over \norm{\P(r)'(t)}^2} \bigg\langle \P(r)''(t) \
\bigg| \  {\partial \P(r)'(t)  \over \partial r}  \bigg(
{\partial \F(s) \over \partial s}
 (\delta s) \bigg) \bigg\rangle \ dt \\ & = & 
\alpha(\P)_{r = \F(s)}  \bigg({\partial \F(s) \over
\partial s}(\delta s)  \bigg) \\ & = &
{\F}^*(\alpha(\P))_s(\delta s).
 \end{eqnarray*}
So, $\alpha(\P \circ \F) = \F^*(\alpha(\P))$, and
$\alpha$ satisfies the differential
form axiom. 

Let us consider now the action of $\Diff_+(\S^1)$ on
$\Imm(\S^1,\RR^2)$. This action is obviously smooth from
the very definition of the functional diffeology of
$\Diff_+({\S}^1)$. Let us denote $\varphi^{-1}$ by $\phi$
such that
 $$ 
 {\bar \varphi}^*(\alpha)(\P) = \alpha(\bar
\varphi \circ \P)  = \alpha[r \mapsto 
 \P(r) \circ \varphi^{-1}] =\alpha[r \mapsto 
 \P(r) \circ \phi].
 $$
Note that
$\Diff_+({\S}^1)$ acts on {\em speed} and
{\em acceleration} of any immersion $x$, by
  \begin{equation}
 \renewcommand{\theequation}{$\heartsuit$}
 \begin{array}{lcl}
 (x \circ \phi)\,'(t) & = &
x'(\phi(t))\cdot \phi'(t) \\
\vspace{-9pt} \\
 (x \circ \phi)''(t)& =
&x''(\phi(t))\cdot \phi'(t)^2 + x'(\phi(t))
\cdot\phi''(t).
 \end{array}
 \end{equation}
Let us denote by $\Q$ the plot $\bar \varphi \circ
\P$, that is $\Q = [r \mapsto \P(r) \circ \phi]$. Such
that,
 $$
\alpha(\bar \varphi \circ \P)_r(\delta r) = 
\int_0^{2\pi} 
{1 \over \norm{\Q(r)'(t)}^2}
\bigg\langle \Q(r)''(t) \ \bigg| \ {\partial \Q(r)'(t)
\over \partial r} (\delta r)\bigg\rangle  \ dt 
 $$
for all $r \in \U$ and all $\delta r \in \RR^n$. But,
from $\heartsuit$,
 \begin{eqnarray*}
{\Q}(r)'(t) & = & (\P(r) \circ \phi)\,'(t)
\ = \ \P(r)'(\phi(t))\cdot \phi'(t) \\
{\Q}(r)''(t) &=& (\P(r) \circ \phi)''(t) \ = \
\P(r)''(\phi(t))\cdot \phi'(t)^2 +
\P(r)'(\phi(t))\cdot \phi''(t)
 \end{eqnarray*}
 So, $\alpha(\bar \varphi \circ \P)_r(\delta
r)$ is equal to  the sum $\A + \B$ of the two following
integrals, related to the decomposition of $\Q(r)''(t)$, 
 $$
 \A = \int_0^{2\pi} {1 \over \norm{\P(r)'(\phi(t))
\cdot \phi'(t)}^2}
\bigg\langle\P(r)''(\phi(t))\cdot \phi'(t)^2 \
\bigg| \ {\partial \P(r)'(\phi(t)) \cdot \phi'(t)
\over \partial r}(\delta r) \bigg\rangle \ dt,
 $$
 $$
 \B = \int_0^{2\pi} {1 \over \norm{\P(r)'(\phi(t))
\cdot \phi'(t)}^2} \bigg\langle\P(r)'(\phi(t)) \cdot
\phi''(t) \ \bigg| \ {\partial \P(r)'(\phi(t)) \cdot
\phi'(t) \over \partial r}(\delta r) \bigg\rangle \ dt.
  $$
The first integral is equal to
 $$
 \A = \int_0^{2\pi} 
{1 \over \norm{\P(r)'(\phi(t))}^2}
\bigg\langle \P(r)''(\phi(t)) \ \bigg| \
{\partial \P(r)'(\phi(t)) \over \partial
r}(\delta r) \bigg\rangle \ \phi'(t) dt.
 $$
 And, since $\varphi$, and thus $\phi$, is a positive
diffeomorphism, after the change of variable
$t \mapsto \phi(t)$,  we get 
 $$
 \A = \alpha(\P)_r(\delta r).
 $$
The second integral is given by
 $$
 \B = \int_0^{2\pi} 
{1 \over \norm{\P(r)'(\phi(t))}^2}
\bigg\langle \P(r)'(\phi(t)) \ \bigg| \ {\partial
\P(r)'(\phi(t)) \over \partial r}(\delta r)
\bigg\rangle \ {\phi''(t) \over \phi'(t)} \ dt 
 $$
Let us denote for short, 
 $$
x = \P(r), \quad x' = \P(r)', \qmbox{and} \delta x' =
\bigg[t \mapsto {\partial \P'(r)(t) \over \partial
r}(\delta r) \bigg],
 $$
 such that the last integral writes
 $$
 \B = \int_0^{2\pi} 
{1 \over \norm{x'(\phi(t))}^2}
\langle x'(\phi(t)) \mid \delta x'(\phi(t)) \rangle
\ {\phi''(t) \over \phi'(t)} dt.
 $$
 Let us remind that, for any variation $\delta$
 $$
 \delta \norm{v} = {1 \over \norm{v}} \langle v \mid
\delta v \rangle \quad \Rightarrow \quad \delta
\log\norm{v} =  {1 \over \norm{v}} \delta \norm{v} = 
{1\over\norm{v}^2} \langle v \mid \delta v \rangle.
 $$ 
 So, the integrand in the last expression of $\B$ writes, 
 $$
{1 \over \norm{x'(\phi(t))}^2}
\langle x'(\phi(t)) \mid \delta x'(\phi(t)) \rangle
= \delta \log \norm{x'(\phi(t))}.
 $$
Thus, the term $\B$ becomes 
 \begin{eqnarray*}
 {\B} & = & \int_0^{2\pi} 
\delta \log \norm{x'(\phi(t))} \
d\log(\phi'(t)) \\
& = & \delta \int_0^{2\pi} 
\log \norm{x'(\phi(t))} \
d\log(\phi'(t)) \\
& = & \delta \int_0^{2\pi} 
\log \norm{x'(\varphi^{-1}(t))} \
d\log((\varphi^{-1})'(t)) 
\end{eqnarray*}
Let us make the change of variable $s =
\varphi^{-1}(t)$, we get,
 \begin{eqnarray*}
 {\B}  & = & + \ \delta \int_0^{2\pi} 
\log \norm{x'(s)} \
d\log[(\varphi^{-1})'(\varphi(s))] \\
 & = & - \ \delta \int_0^{2\pi} 
\log \norm{x'(s)} \
d\log(\varphi'(s)) \\ 
 &= & - \ {\partial \over \partial
r}\bigg\{  \int_0^{2\pi} 
\log \norm{\P(r)'(s)} \
d \log(\varphi'(s)) \bigg\} (\delta
r) \\
 & = & - \ {\partial \over \partial r} \bigg\{
\F(\varphi)(\P(r)) \bigg\} (\delta r) \\
 & = & - \ d[\F(\varphi)](\P)_r(\delta r).
 \end{eqnarray*}
 Coming back to $\alpha(\bar \varphi \circ
\P)_r(\delta r)$ we get finally, 
 $$
 \alpha(\bar \varphi \circ \P)_r(\delta r) = 
\alpha(\P)_r(\delta r) - d[\F(\varphi)](\P)_r(\delta r)
\quad \mbox{that is} \quad \bar \varphi^*(\alpha) =
\alpha - d[\F(\varphi)].
 $$
 Thus, the exterior differential $\omega = d\alpha$
is invariant by the action of $\Diff_+({\S}^1)$. And
since the difference $\bar \varphi^*(\alpha) -
\alpha$ is exact, this
action is hamiltonian. 

{\em The 2-point moment map} --- Now, let us compute the
2-points moment maps $\psi$ of the
action of $\Diff_+(\S^1)$ on $(\Imm(\S^1,\RR^2),\omega)
$. Let $p$ be a path connecting
two immersions $x_0$ and $x_1$. We have $\Psi(p) = \hat
p^*(\eK\omega) = \hat p^*(\eK d\alpha) = \hat
p^*(\but^*(\alpha) - \source^*(\alpha) - d(\eK\alpha)) =
\hat x_1^*(\alpha) - \hat x_0^*(\alpha) - d(\eK\alpha
\circ \hat p)$. But, for all $\varphi \in \Diff_+(\S^1)$,
$$\eK\alpha \circ \hat p(\varphi) = \int_{\bar \varphi
(p)} \alpha = \int_p \bar \varphi^*(\alpha) =
\int_p\alpha - \int_p d\F(\varphi) = \int_p\alpha -
\F(\varphi)(x_1) + \F(\varphi)(x_0).$$ So, we get finally
 $$\Psi(p) = \psi(x_0,x_1) = \{\hat x_1^*(\alpha) +
d[\varphi \mapsto \F(\varphi)(x_1)]\} - \{\hat
x_0^*(\alpha) + d[\varphi \mapsto \F(\varphi)(x_0)]\}.
 $$
 But notice that, $\hat x^*(\alpha) + d[\varphi
\mapsto \F(\varphi)(x)$ is not a momentum of
$\Diff_+(\S^1)$.

{\em The 1-point moment maps} --- Let us
compute the moment map $\psi(x_0,x)$. Let 
 $$
 m = \{\hat x^*(\alpha) + d[\varphi \mapsto
\F(\varphi)(x)]\}(r \mapsto \varphi)_r(\delta r).
 $$
 And, let us denote for short
 \begin{eqnarray*}
 {\A} &=& \hat x^*(\alpha)(r \mapsto \varphi)_r(\delta r)
\\
 {\B} &=& d[\varphi \mapsto \F(\varphi)(x)](r
\mapsto \varphi)_r(\delta r) = {\partial \F(\varphi)(x)
\over \partial r} \delta r.
 \end{eqnarray*}
 We shall use the notation $m_0$, $\A_0$ and $\B_0$ for
the immersion $x_0$. Thus,
 $$
 \psi(x_0,x)(r \mapsto \varphi)_r(\delta r) = m - m_0 =
\A + \B - \A_0 - \B_0.
 $$
 We have, $ \hat x^*(\alpha)(r \mapsto
\varphi) = \alpha(\hat x \circ [r \mapsto \varphi]) =
\alpha(r \mapsto x \circ \varphi^{-1})$. Let
$\phi = \varphi^{-1}$, so
 $$
 \A
= \int_0^{2\pi} {1 \over
\norm{(x \circ \phi)'(t)}^2} \bigg\langle (x \circ
\phi)''(t) \ \bigg| \ {\partial (x \circ \phi)'(t) \over
\partial r}(\delta r) \bigg\rangle.
 $$
Let us introduce now,
 $$ 
 u = \phi(t), \quad u'=\phi(t) \qmbox{and} u'' =
\phi''(t).
 $$
So, the decomposition given by $\heartsuit$, 
writes 
 $$
(x \circ \phi)'(t) = x'(u) \cdot u' \qmbox{and}
(x \circ \phi)''(t) =
x''(u) \cdot u'^2 +
x'(u) \cdot u''.
 $$
Then, we shall use the prefix $\delta$ for every
variation associated to $\delta r$, that is $\delta \star
= \D(r \mapsto \star)(r)(\delta r)$. So,
  $$
 {\partial (x \circ \phi)'(t) \over \partial r}(\delta r)
= \delta[x'(u) \cdot u'] = x''(u) \cdot \delta u \cdot u'
+ x'(u) \cdot \delta u'.
 $$
 Thus,
 \begin{eqnarray*}
 {\A}
&=& \int_0^{2\pi} {1 \over \norm{x'(u)}^2 
u'^2} \langle x''(u)  u'^2 +
x'(u) u'' \mid  x''(u)  u' \delta u 
+ x'(u)  \delta u' \rangle \ dt \\
 &=& \int_0^{2\pi} {\norm{x''(u)}^2 \over
\norm{x'(u)}^2} \ \delta u \ u' dt +
\int_0^{2\pi} {\langle
x'(u), x''(u)\rangle \over \norm{x'(u)}^2}  \bigg[ \delta u' +
{u'' \over u'} \delta u \bigg] \ dt + \int_0^{2\pi}
{u'' \over u'} \delta
u' \ dt 
 \end{eqnarray*}
Now, 
 $$
\B = {\partial \F(\varphi)(x)
\over \partial r} \delta r = - {\partial \bar \F(\phi)(x)
\over \partial r} \delta r = - \delta [\bar \F(\phi)(x)],
 $$
with 
 $$
 \bar \F(\phi)(x)
= \int_0^{2\pi} \log \norm{x'(\phi(t))} \ d \log
\phi'(t) = \int_0^{2\pi} \log\norm{x'(u)} \ d\log(u').
 $$
So, after the variation with respect to $\delta r$ and an
integration by part, we get
 \begin{eqnarray*}
 {\B} &=& - \int_0^{2\pi}  {\langle
x'(u), x''(u)\rangle \over \norm{x'(u)}^2}  \delta u 
{u'' \over u'}  \ dt -  \int_0^{2\pi} \log
\norm{x'(u)} \ \delta d \log(u') \\
&=& - \int_0^{2\pi}  {\langle
x'(u), x''(u)\rangle \over \norm{x'(u)}^2}  \delta u 
{u'' \over u'}  \ dt +  \int_0^{2\pi} {\langle
x'(u), x''(u)\rangle \over \norm{x'(u)}^2} u' \ \delta
\log(u') \ dt \\
 &=& - \int_0^{2\pi}  {\langle
x'(u), x''(u)\rangle \over \norm{x'(u)}^2}  \delta u 
{u'' \over u'}  \ dt +  \int_0^{2\pi} {\langle
x'(u), x''(u)\rangle \over \norm{x'(u)}^2} \ \delta
 u' \ dt
 \end{eqnarray*}
Therefore, grouping the terms and integrating again
by part, we get
 \begin{eqnarray*}
 {\A} + \B
 &=& \int_0^{2\pi} {\norm{x''(u)}^2 \over
\norm{x'(u)}^2}  \delta u \ du + 2
\int_0^{2\pi} {\langle
x'(u), x''(u)\rangle \over \norm{x'(u)}^2}  \delta
u' \ dt + \int_0^{2\pi}
{u'' \over u'} \delta u' \ dt \\
 &=& \int_0^{2\pi} {\norm{x''(u)}^2 \over
\norm{x'(u)}^2}  \delta u \ du - 2
\int_0^{2\pi} {d^2  \over du^2}\log\norm{x'(u)}  \delta
u \ du + \int_0^{2\pi}
{u'' \over u'} \ \delta
u' \ dt  \\
 &=& \int_0^{2\pi} \left\{{\norm{x''(u)}^2 \over
\norm{x'(u)}^2} - {d^2  \over du^2}\log\norm{x'(u)}^2
\right\} \delta u \ du + \int_0^{2\pi}
{u'' \over u'} \ \delta
u' \ dt  \end{eqnarray*}
Now, since $\norm{x_0'(t)} = 1$ we
get the value of the 2-point moment map,
 $$
  \psi(x_0,x)(r \mapsto \varphi)_r(\delta r) =
\int_0^{2\pi} \left\{{\norm{x''(u)}^2 \over
\norm{x'(u)}^2} - {d^2  \over du^2}\log\norm{x'(u)}^2
\right\} \delta u \ du - \int_0^{2\pi} \delta u \ du.
 $$
 The second term of the right hand side of the equality is
a constant momentum of $\Diff_+(\S^1)$, so it can be
avoided. And, every moment map is, up to a constant, equal
to the moment $\mu$ announced.

{\em Souriau's cocycles} --- Souriau's
cocycle associated to 
immersion $x_0$ is defined by $\theta(g) = \psi(x_0,\bar
g(x_0))$, see \art{Souriau-s-cocycles}. So, we have to
replace, in the expression of $\psi$ above, $x$ by $\bar
g (x_0) = x_0 \circ g^{-1}$, that is $x= x_0 \circ
\gamma$. So, $\theta(g)(r \mapsto \varphi)_r(\delta r) =
\psi(x_0,x_0 \circ \gamma)$. So, note first that
 $$
 (x_0 \circ \gamma)'(u) = x_0'(\gamma(u)) \gamma'(u)
\qmbox{and} (x_0 \circ \gamma)''(u) = x_0''(\gamma(u))
\gamma'(u)^2 + x_0'(u) \gamma''(u).
 $$
 And, let us remind that $\norm{x_0'} = \norm{x_0''}
= 1$ and $\langle x_0' \mid x_0'' \rangle = 0$. We get,
 $$
 \norm{x'(u)}^2 = \gamma'(u)^2 \qmbox{and}
\norm{x''(u)}^2 = \gamma'(u)^4 + \gamma''(u)^2.
 $$
 This gives
 $$
 {\norm{x''(u)}^2 \over
\norm{x'(u)}^2} = \gamma'(u)^2 + {\gamma''(u)^2 \over
\gamma'(u)^2} \qmbox{and} {d^2  \over
du^2}\log\norm{x'(u)}^2 = 2{\gamma'''(u) \gamma'(u) -
\gamma''(u)^2 \over \gamma''(u)^2}.
 $$
 Thus, 
 \begin{eqnarray*}
 \theta(g)(r \mapsto \varphi)_r(\delta r) &=&
\int_0^{2\pi} { 3 \gamma''(u)^2 - 2 \gamma'''(u)
\gamma'(u) \over \gamma'(u)^2} \ \delta u \ du \\
 &+& 
\int_0^{2\pi} \gamma'(u)^2 \ \delta u \ du - \int_0^{2\pi}
\delta u \ du.
 \end{eqnarray*}
But, after a change of variable $u \mapsto v =
\gamma(u)$, we get $$\int_0^{2\pi} \gamma'(u)^2 \ \delta u
\ du = \int_0^{2\pi} (\delta u  \gamma'(u)) \ \gamma'(u)
du = \int_0^{2\pi} \delta v \ dv.$$ So the two last
terms cancel each other, and we get the value announced
for Souriau's cocycle $\theta$.

{\em Bott's cocycle} --- The real function $\F(g \circ h)
- \F(g) \circ \bar h - \F(h)$ is constant since $\X$ is
connected, and its differential is equal to $(\bar g
\circ \bar h)^*(\alpha) - \bar h^*(\bar g^*(\alpha))$,
that is $0$. Now, to explicit  $\beta(g,g') = 
  \F(g)\circ \bar g' +
\F(g') - \beta(g,g') - \F(g \circ g')$, it is sufficient
to compute the right hand member on the standard immersion
$x_0$, for which the speed norm is equal to 1, and thus
$\log \norm{x'(t)} = 0$ for all real $t$. So we get,
 \begin{eqnarray*}
\beta(g,h) &=& {\F}(g)(x_0 \circ h^{-1}) - \F(h)(x_0) -
\F(g \circ h)(x_0) \\
&=& + \int_0^{2\pi} \log\norm{(x_0 \circ h^{-1})'(t)} \ d
\log g'(t) \\
&=& + \int_0^{2\pi} \log (h^{-1})'(t) \ d\log g'(t) \\
&=& -   \int_0^{2\pi} \log h'(h^{-1}(t)) \ d\log g'(t) \\
&=& -  \int_0^{2\pi} \log h'(s) \ d\log g'(h(s)) \\
&=& +  \int_0^{2\pi} \log(g \circ h)'(t) \ d\log h'(t)
 \end{eqnarray*}
And this is the standard expression of Bott's cocycle. 
\end{proof}


%************************************************
%***** Bibliographie
%************************************************

%%%%%%%%%%%%%%% macros pour labibliographie %%%%%%%%%%%%%%%%%%%%%%%

\newcommand{\Author}[1]{{#1}.}
\newcommand{\Title}[1]{{\em #1},}
\newcommand{\Journal}[1]{#1,}
%\newcommand{\proceedings}[1]{proc.~{\em #1},}
%\newcommand{\tome}[1]{tome~#1,\ }
\newcommand{\Volume}[1]{volume #1,}
\newcommand{\Series}[1]{serie~#1,}
\newcommand{\Number}[1]{number~#1,}
\newcommand{\Pages}[1]{pages~#1,}
\newcommand{\Collection}[1]{{#1},}
\newcommand{\Publisher}[2]{#1, #2,} % premiere entree le nom de l'editeur, deuxieme la ville.
\newcommand{\Report}[1]{#1,}
\newcommand{\Year}[1]{#1.}

%%%%%%%%%%%%%%%%%%%%%%%%%%%%%%%%%%%%%%%%%%%%%%%%%%%%
% La bibliographie 
%%%%%%%%%%%%%%%%%%%%%%%%%%%%%%%%%%%%%%%%%%%%%%%%%%%%
\newpage 
\begin{thebibliography}{CDM88}

\bibitem[Ban78]{Ban78}
\Author{Augustin Banyaga}
\Title{Sur la structure du
groupe des diff\'eomorphismes qui pr\'eservent une forme
symplectique}
\Journal{Comment. Math. Helv.}
\Volume{53}
\Pages{174--227}
\Year{1978}

\bibitem[Boo69]{Boo69}
\Author{William M. Boothby}
\Title{Transitivity of the automorphisms of certain
geometric structures}
\Journal{Trans. Amer. Math. Soc.}
\Volume{137}
\Pages{93--100}
\Year{1969}

\bibitem[Bot78]{Bot78}
Raoul~Bott.
\newblock {\em On some formulas for the characteristic
classes of group actions,
  differential topology, foliations and Gelfand-Fuchs
cohomology}. \newblock In {Proceed. Rio de Janeiro,
1976}, volume 652 of {Springer
  Lectures Notes}. Springer Verlag, 1978.

\bibitem[Che77]{Che77}
\Author{Kuo Tsai Chen}
\Title{{Iterated path integral}}
\Journal{Bull. of Am. Math. Soc.}
\Volume{83}
\Number{5}
\Pages{831 -- 879}
\Year{1977}

\bibitem[CDM88]{CDM88}
{M. Condevaux, P. Dazord and P. Molino.}
{\em G\'eom\'etrie du moment.}
{Travaux du S\'eminaire Sud-Rhodanien de
G\'eom\'etrie, I, Publ. D\'ep. Math. Nouvelle S\'er.
B 88-1, Univ. Claude-Bernard, pp. 131 -- 160, Lyon, 1988.}

\bibitem[Dnl99]{Dnl99}
Simon K. Donaldson.
\newblock {\em Moment maps and diffeomorphisms\/}.
Asian Journal of Math, vol. 3, pp. 1--16, 1999.

\bibitem[Don84]{Don84}
\Author{Paul Donato}
\Title {{Rev\^etement et groupe fondamental des espaces diff\'e\-rentiels homog\`enes}}
\newblock {Th\`ese de doctorat d'\'etat, Universit\'e de Provence, Marseille,
1984.}

\bibitem[Don88]{Don88}
Paul Donato.
\newblock{\em G\'eom\'etrie des orbites coadjointes des
groupes de
  diff\'eomorphismes\/}.
\newblock In {Lect. Notes In Maths}, vol. 1416, pp.
84--104, 1988.

\bibitem[Igl85]{Igl85}
\Author{Patrick Iglesias}
\Title { {Fibr\'es diff\'eologiques et homotopie}}
\newblock {Th\`ese de doctorat d'\'e\-tat, Universit\'e de Provence, Marseille, 1985.}
\\
\newblock{\tt 
http://math.huji.ac.il/$\sim$piz/documents/These.pdf}

\bibitem[IKZ05]{IKZ05}
	\Author {Patrick Iglesias, Yael Karshon and Moshe
Zadka}
	\Title{Orbifolds as diffeologies}
	\Year {2005} \newblock{{\tt
http://arxiv.org/abs/math.DG/0501093}}

\bibitem[IL90]{IL90}
\newblock{Patrick Iglesias \& Gilles Lachaud.} 
\newblock{\em Espaces diff\'erentiables singuliers et
corps de nombres alg\'ebriques.\/} 
\newblock{Ann. Inst.
Fourier, Grenoble,} 
\newblock{volume 40,}
\newblock{number 1,}
\newblock{pages 723 -- 737,}
\newblock{1990.}

\bibitem[Igl95]{Igl95}
Patrick Iglesias.
\newblock {\em La trilogie du moment\/}.
\newblock {Ann. Inst. Fourier}, 45, 1995.

\bibitem[Piz05]{Piz05}
\Author{Patrick Iglesias-Zemmour} 
\Title{Diffeology}
\newblock{eprint}	\Year {2005--07}
\\
\newblock{{\tt
http://math.huji.ac.il/$\sim$piz/diffeology/}}

\bibitem[Piz06-a]{Piz06-a}
\Author{Patrick Iglesias-Zemmour} 
\Title{Diffeology of the Infinite Hopf Fibration}
\newblock{eprint,}	\Year {2006}
\newblock{{\tt
http://math.huji.ac.il/$\sim$piz/documents/DIHF.pdf}}

\bibitem[Piz06-b]{Piz06-b}
Patrick Iglesias-Zemmour.
{\em Dimension in diffeology\/}, eprint 2006.
\\
\newblock{{\tt
http://math.huji.ac.il/$\sim$piz/documents/DID.pdf}}

\bibitem[Piz07-a]{Piz07-a}
\Author{Patrick Iglesias-Zemmour} 
\Title{Variations of integrals in diffeology}
\newblock{eprint}	\Year {2007}
\newblock{{\tt
http://math.huji.ac.il/$\sim$piz/documents/VOIID.pdf}}

\bibitem[Piz07-c]{Piz07-c}
Patrick Iglesias-Zemmour
\newblock{\em Every symplectic manifold is a coadjoint
orbit} \newblock{eprint}	\Year {2007}
\\
\newblock{{\tt
http://math.huji.ac.il/$\sim$piz/documents/ESMIACO.pdf}}

\bibitem[Kir74]{Kir74}
Alexandre A. Kirillov.
\newblock{\em Elements de la th{\'e}orie des
repr{\'e}sentations.}
	\newblock {Editions MIR, Moscou, 1974.}

\bibitem[Kos70]{Kos70}
Bertram Kostant.
\newblock {\em Orbits and quantization theory\/}.
\newblock In Congr{\`e}s international des
math{\'e}maticiens 1970-1971.

\bibitem[Omo86]{Omo86}
Stephen Malvern Omohundro.
\newblock {\em Geometric Perturbation Theory in Physics}.
\newblock World Scientific, 1986.

\bibitem[Sou70]{Sou70}
\Author{Jean-Marie Souriau}
\Title{{Structure des syst\`emes dynamiques}}
\Publisher {Dunod}{Paris} 
\Year {1970}

\bibitem[Sou81]{Sou81}
\Author{Jean-Marie Souriau}
\Title{Groupes diff\'erentiels}
\Journal{{Lecture notes in mathematics}}
\Publisher{Springer Verlag}{New-York}
\Volume{836}
\Pages{91 -- 128}
\Year{1981}

\bibitem[Sou84]{Sou84}
\Author{Jean-Marie Souriau}
\Title{{Groupes diff\'erentiels et physique math\'ematique}}
\Journal{Lecture Notes in Physics}
\Publisher{Springer Verlag}{Berlin -- Heidelberg}
\Volume{201}
\Pages{511 -- 513}
\Year{1984}

\bibitem[Zie96]{Zie96}
\Author{Fran\c cois Ziegler}
\Title{Th\'eorie de Mackey symplectique, in M\'ethode des
or\-bi\-tes et repr\'esentations quantiques}
\newblock{Th\`ese de doctorat d'Uni\-ver\-sit\'e,
Universit\'e de Provence, Marseille, 1996.}

\end{thebibliography}


%%%%%%%%%%%%%%%%%%%%%%%%%%%%%%%%%%%%%%%%%%%%%%%%%%%%
% Fin du document 
%%%%%%%%%%%%%%%%%%%%%%%%%%%%%%%%%%%%%%%%%%%%%%%%%%%%
\end{document}
