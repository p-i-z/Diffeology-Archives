\documentclass[11pt]{article}
\usepackage[utf8]{inputenc}
\usepackage{geometry}
\geometry{a4paper, margin=1in}
\usepackage{hyperref}

\title{Handoff Summary: The Diffeology Archives Project (Window 3)}
\author{Patrick Iglesias-Zemmour \& Gemini}
\date{December 23, 2025}

\begin{document}

\maketitle

\section{Project Status}

The \textbf{Diffeology Archives} project continues to grow. We have successfully archived two major papers in this session and refined the theoretical links between them. The repository is live at \url{https://github.com/p-i-z/Diffeology-Archives}.

\section{Work Completed in This Session}

\subsection{New Archives}
\begin{itemize}
    \item \textbf{2010-Orbifolds-as-Diffeologies}:
    \begin{itemize}
        \item Transmuted legacy source to modern \texttt{amsart}.
        \item \textbf{Major Task:} Recreated all 6 legacy \texttt{eepic} figures using \texttt{TikZ}, making the file completely self-contained.
        \item Added historical context regarding the collaboration with Yael Karshon and Moshe Zadka.
    \end{itemize}
    \item \textbf{2021-An-Introduction-To-Diffeology}:
    \begin{itemize}
        \item Transmuted to \texttt{amsart} and applied "ventilated prose".
        \item Cleaned the preamble significantly.
        \item Added a "Significance" section to the README highlighting the "Shift in Perspective" (Diffeology vs. Topology) and the connection to Felix Klein's Erlangen Program.
    \end{itemize}
\end{itemize}

\subsection{Documentation and Theoretical Linking}
\begin{itemize}
    \item \textbf{The Boman Paradox (2025):} Updated the existing README to explicitly state that this paper invalidates the strict Erlangen Program view proposed in earlier works.
    \item \textbf{Cross-Referencing:} Added a note to the 2021 "Introduction" README citing the 2025 "Boman Paradox" as a necessary refinement of the Kleinian perspective. This creates a dialectical link between the archives.
\end{itemize}

\subsection{Technical Updates}
\begin{itemize}
    \item \textbf{Macro Library (\texttt{diffeology-piz.sty}):} Updated to \textbf{Version 2.13}.
    \item Added functional space operators: \texttt{\textbackslash Paths}, \texttt{\textbackslash Loops}, \texttt{\textbackslash Params}, \texttt{\textbackslash Maps}.
    \item Added geometry operators: \texttt{\textbackslash Geod}, \texttt{\textbackslash Ham}, \texttt{\textbackslash Surf}, \texttt{\textbackslash Sq}.
    \item Added algebraic operators: \texttt{\textbackslash Ad}, \texttt{\textbackslash Hom}, \texttt{\textbackslash ev}, etc.
    \item Added specific symbols: \texttt{\textbackslash dR}, \texttt{\textbackslash CHK} (Chain-Homotopy), \texttt{\textbackslash dt}.
\end{itemize}

\section{Current Workflow}
\begin{enumerate}
    \item \textbf{Transmutation:} Convert to \texttt{amsart}, apply ventilated prose, convert accents to UTF-8.
    \item \textbf{Figures:} Recreate legacy diagrams in \texttt{TikZ} whenever possible to ensure self-containment.
    \item \textbf{Macros:} Extract general macros to the central \texttt{.sty} file; keep specific ones in the local preamble.
    \item \textbf{Context:} Write rich READMEs that include historical anecdotes and theoretical significance.
\end{enumerate}

\section{Next Steps}
The user has indicated a pause. The next session should pick up by identifying the next monograph or key paper to archive. The goal remains to create a high-quality, semantic dataset for future AI training.

\end{document}