%%%%%%%%%%%%%%%%%%%%%%%%%%%%%%%%%%%%%%%%%%%%%%%%%%%%%%%%%%
%%
%%  Modeling spaces, Manifolds etc. MARK: -
%%
%%%%%%%%%%%%%%%%%%%%%%%%%%%%%%%%%%%%%%%%%%%%%%%%%%%%%%%%%%

\chapter{Modeling Spaces, Manifolds, etc.}

\label{Modeling-spaces-Manifolds-etc}
\newcommand{\ChapterMSME}{Modeling spaces, Manifolds etc.}

\begin{chaphead}
  Manifolds are the main objects of differential geometry as we know it.
  It is not necessary to develop the whole theory of diffeological spaces just to introduce or study manifolds.
  However,
  manifolds can be found again as a  full subcategory of the  category $\Diffeology$,
  which is reassuring.
  Roughly speaking,
  manifolds are diffeological spaces which look like locally real vector spaces.
  Diffeology gives another insight into this respectable domain:
  manifolds are not any more regarded as sets equipped with a {\em manifold structure},
  but rather as diffeological spaces whose diffeology satisfies the special property to be generated by local diffeomorphisms with a given vector space.
  
  This rediscovery of manifolds,
  through diffeology,
  suggests some generalizations.
  First of all,
  we can replace,
  in the definition of a manifold,
  real vector spaces by any diffeological vector space \art{Diffeological-vector-spaces},
  and we get a larger category of manifolds which contains not only the usual ones,
  but also infinite dimensional manifolds,
  which find,
  that way,
  a formal framework for their study.
  If this generalization brings some satisfaction
  ---~many infinite dimensional spaces become then manifolds,
  for example the infinite projective space \art{The-infinite-complex-projective-space},
  but not only~---
  it also raises new questions.
  For example,
  if we have a notion of orientability for diffeological spaces of finite dimension \art{Volumes-on-manifolds-and-diffeological-spaces},
  what about orientability of infinite dimensional diffeological manifolds?
  This is not the only question,
  and it may not be the most relevant.
  With infinite dimension spaces things are more subtle,
  and the usual concepts need to be revisited with care if we want to preserve their relevance.
  
  On the other hand,
  this  conception suggests the notion of {\em locally modeled spaces},
  I mean,
  diffeological spaces which are locally diffeomorphic to some members of a family of {\em diffeological models}.
  For manifolds,
  the family of modeling spaces is usually reduced to a single given diffeological vector space,
  but diffeological orbifolds \art{Orbifolds-as-diffeologies} are defined as diffeological spaces,
  modeled on the family of the quotients of real spaces by a finite subgroup of its diffeomorphisms.
\end{chaphead}

%%%%%%%%%%%%%%%%%%%%%%%%%%%%%%%%%%%%%%%%%%%%%%%%%%%%%%%%%%

\section*{Standard Manifolds, the Diffeological Way}
\label{Section-Manifolds-from-diffeological-viewpoint}

\begin{sechead}
  We should begin with the general definition of diffeological manifolds \art{Diffeological-Manifolds},
  but,
  to
  preserve the intuition of the geometer,
  I chose to begin with an presentation on usual manifolds
  ---~which are regarded as special cases of diffeological spaces~---
  and then to introduce the more general concept of manifolds modeled on general diffeological vector spaces.
\end{sechead}

\begin{article}\artlabel{Manifolds as diffeologies}
  \addcontentsline{toc}{section}{\small\hspace{10pt} Manifolds as diffeologies}
  \label{Manifolds-as-diffeologies}
  Let $\M$ be a diffeological space,
  $\M$ is said to be a {\em manifold of dimension $n$},
  or an {$n$-manifold}\index{Manifold} if and only if $\M$ is locally diffeomorphic at each point to $\RR^n$.
  This means precisely that,
  for each point $m \in \M$,
  there exist a local diffeomorphism $\F : \RR^n \supset \U \to \M$ \art{Local-diffeomorphisms},
  and a point $r \in \U$ such that $\F(r) = m$.
  Such local diffeomorphisms are called {\em charts}\index{Chart} of $\M$.
  The set of all the charts of $\M$ is called the {\em saturated atlas} of $\M$.
  Speaking in terms of diffeologies,
  we shall also say that the diffeology $\cD$ of $\M$ is a {\em manifold diffeology}\index{Diffeology!manifold diffeology}.
  Manifolds form the subcategory $\Manifolds$ of the category $\Diffeology$.
\end{article} %% Manifolds-as-diffeologies

\begin{article}\artlabel{Local modeling of manifolds}
  \addcontentsline{toc}{section}{\small\hspace{10pt} Local modeling of manifolds}
  \label{Local-modeling-of-manifolds}
  The previous definition of manifolds \art{Manifolds-as-diffeologies} can be formulated again in terms of generating family \art{Generating-diffeology}.
  Let $\M$ be a diffeological space.
  
  1. A family $\cA$ of local diffeomorphisms \art{Local-diffeomorphisms},
  from $\RR^n$ to $\M$,
  such that
  $$%
  \bigcup_{\F \in \cA} \Val(\F) = \M
  $$%
  is a generating family of the diffeology of $\M$.
  
  2. The diffeological space $\M$ is an $n$-manifold \art{Manifolds-as-diffeologies} if and only if there exists a generating family $\cA$ of $\M$,
  made of local diffeomorphisms from $\RR^n$ to $\M$.
  
  \Note{1} If $\M$ is a manifold,
  any generating family $\cA$,
  made of local diffeomorphisms of $\M$,
  is called an {\em atlas}\index{Atlas} of $\M$.
  The elements of $\cA$ are called the {\em charts of the atlas}.
  Obviously,
  if $\M$ is a manifold,
  the set of all local diffeomorphisms from $\RR^n$ to $\M$ is a generating family of the diffeology of $\M$.
  This set is the saturated atlas of the manifold $\M$ \art{Manifolds-as-diffeologies}.
  
  \Note{2} An $n$-manifold is a diffeological space of constant dimension $n$,
  since local diffeomorphisms preserve the dimension \art{The-dimension-map}.
  Conversely,
  if the dimension of a manifold is $n$ \art{Dimension-of-a-diffeological-space},
  then it is an $n$-manifold.
  This is why we sometimes use the wording {\em $n$-dimensional manifold} in place of $n$-manifold.
\end{article} %% Local-modeling-of-manifolds

\begin{proof}
  1. Let us assume that $\cA$ is a family of local diffeomorphisms from $\RR^n$ to $\M$ such that $\bigcup_{\F \in \cA} \Val(\F) = \M$.
  Thus,
  $\M$ is locally diffeomorphic to $\RR^n$ at each point,
  since each point $m$ of $\M$ is in the set of values of a local diffeomorphism.
  Let us choose,
  for each point $m \in {\M}$,
  a local diffeomorphism $\F : \U \to {\M}$ such that $\F(0) = m$,
  where $\U$ is an open neighborhood of $0 \in \RR^n$.
  Such local diffeomorphisms exist since we have just to compose any local diffeomorphism $\F$ which maps $r$ to $m$ with the translation mapping $0$ to $r$.
  Now,
  let $\cA$ be the set of all these chosen local diffeomorphisms,
  where $m$ runs over $\M$.
  Let us check that they form a generating family of the diffeology of $\M$.
  Let ${\P}: \V \to {\M}$ be a plot and $r \in \V$,
  and let $m = {\P}(r)$ and $\F \in \cA$ such that $\F: \U \to {\M}$,
  $\F(0) = m$.
  Let $\Q = \F^{-1} \circ {\P} \restriction {\W}$,
  where ${\W} = {\P}^{-1}(\F(\U))$.
  Since $\F$ is a local diffeomorphism,
  $\F(\U)$ is D-open \art{Local-smooth-maps-are-defined-on-D-opens},
  and because ${\P}$ is D-continuous \art{Smooth-maps-are-D-continuous},
  ${\W}$ is a domain.
  Thus,
  $\Q$ is a smooth local lift of ${\P}$ along $\F$.
  Hence,
  any plot of ${\M}$ can be lifted along a member of the family $\cA$,
  and this is the criterion for a generating family \art{Criterion-of-generation}.
  Hence,
  the diffeology of $\M$ is generated by $\cA$.
  
  2. Let us assume that the diffeology of $\M$ is generated by a family $\cA$ of local diffeomorphisms with $\RR^n$.
  Let $m \in \M$ be any point,
  the constant 0-parametrization ${\bf m} : 0 \mapsto m$ is a plot.
  Thus,
  by definition of generating families \art{Generating-diffeology},
  there exist an element $\F : \U \to \M$ of $\cA$ and a point $r \in \U$ such that ${\bf m} = \F \circ {\bf r}$,
  where ${\bf r}(0) = r$.
  Hence, there exist a local diffeomorphism $\F : \RR^n \supset \U \to \M$ and a point $r \in \U$ such that $\F(r) = m$.
  Therefore,
  $\M$ is a manifold \art{Manifolds-as-diffeologies}.
  The converse is a consequence of 1.
  If $\M$ is an $n$-manifold,
  it is locally diffeomorphic at each point to $\RR^n$.
  So,
  there exists a family $\cA$ of local diffeomorphisms from $\RR^n$ to $\M$ such that $\bigcup_{\F \in \cA} \Val(\F) = \M$.
  Therefore,
  $\cA$ is a generating family of the diffeology of $\M$.
  
  Now,
  since $n$-manifolds are locally diffeomorphic to $\RR^n$,
  their dimension,
  at every point,
  is the same dimension as an $n$-domain \art{The-dimension-map-is-a-local-invariant}.
  But the dimension of an $n$-domain is constant and equal to $n$ \art{dimension-of-real-domains}.
  Hence,
  the dimension of an $n$-manifold is constant and equal to $n$.
  Conversely,
  the dimension map of a manifold is constant and equal to the dimension of the domains of its charts,
  as a direct consequence of the definition.
  Since $\dim(\RR^n) = \dim(\RR^m)$ if and only if $n=m$,
  a manifold has dimension $n$ if and only if it is an $n$-manifold.
\end{proof}

\begin{article}\artlabel{Manifolds, the classical way}
  \addcontentsline{toc}{section}{\small\hspace{10pt} Manifolds, the classical way}
  \label{Manifolds-the-classical-way}
  A full presentation on manifolds,
  in classical differential geometry context, can be found in every book of differential geometry;
  see for example \cite{Bou82},
  \cite{BG72},
  \cite{Die70c},
  \cite{Doc76},
  \cite{DNF82}.
  Let us just summarize the basic definitions.
  We choose here the Bourbaki definition \cite{Bou82},
  but we make the inverse convention, made also by some other authors,
  to regard charts defined from real domains to a manifold $\M$,
  rather than from subsets of $\M$ into real domains.
  
  ($\clubsuit$) Let $\M$ be a nonempty set.
  A {\em chart} of $\M$ is a bijection $\F$ defined on an $n$-domain $\U$ to a subset of $\M$.
  The dimension $n$ is a part of the data.
  Let $\F : \U \to \M$ and $\F' : \U' \to \M$ be two charts of $\M$.
  The charts $\F$ and $\F'$ are said to be {\em compatible} if and only if the following conditions are fulfilled:
  \begin{itemize}
    \item[(a)] The sets $\F^{-1}(\F'(\U'))$ and $\F'^{-1}(\F(\U))$ are open.
    \item[(b)] The two maps,
    each one the inverse of the other,
    $\F'^{-1} \circ \F : \F^{-1}(\F'(\U')) \to \F'^{-1}(\F(\U))$ and $\F^{-1} \circ \F' : \F'^{-1}(\F(\U)) \to \F^{-1}(\F'(\U'))$, are either empty or smooth.
    They are called {\em transition maps}\index{Transition map}.
  \end{itemize}
  An {\em atlas}\index{Atlas} is a set of charts,
  compatible two-by-two,
  such that the union of the values is the whole $\M$.
  Two atlases are said to be compatible if their union is still an atlas.
  This relation is an equivalence relation.
  A {\em structure of manifold} on $\M$ is the choice of an equivalence class of atlases or,
  which is equivalent,
  the choice of a {\em saturated atlas}.
  Once a structure of manifold is chosen for $\M$,
  every compatible chart is called a {\em chart of the manifold}\index{Chart}.
  
  ($\heartsuit$) The simplest example of a classical manifold is a real domain $\U$,
  with the structure of a manifold given by the atlas reduced to the identity $\{\id_ \U\}$.
  
  ($\diamondsuit$) Let $\M$ and $\M'$ be two classical manifolds.
  A map $f: \M \to \M'$ is said to be infinitely differentiable,
  or of class $\Cinfty$,
  or smooth,
  if and only if,
  for every pair of charts $\F : \U \to \M$ and $\F' : \U' \to \M'$,
  the following conditions are fulfilled:
  \begin{itemize}
    \item[(a)] The set $(f \circ \F)^{-1}(\F'(\U'))$ is open.
    \item[(b)] The parametrization $\F'^{-1} \circ f \circ \F : (f \circ \F)^{-1}(\F'(\U')) \to \U'$ is either empty or smooth.
  \end{itemize}
  The composition of smooth maps between classical manifolds is again a smooth map. Classical manifolds,
  together with smooth maps,
  form a category whose isomorphisms are called diffeomorphisms.
  A diffeomorphism $f$ from a manifold  to another is a smooth bijection such that $f^{-1}$ is smooth.
  %%###########
  \begin{figure}[htb]
    \centerline{\includegraphics{Figures-PDF/fig-manifold-map}}
    \caption{A smooth map between manifolds.}
    \label{A-smooth-map-between-manifolds}
  \end{figure}
  %%###########
  
  That was for the classical way.
  Now,
  manifolds are naturally embedded in the category of diffeological spaces,
  thanks to the functor defined as follows.
  
  (A) Let $\M$ be a classical manifold.
  Let $\cD$ be the set of all the parametrizations $\P : \U \to \M$ which are smooth according to ($\diamondsuit$),
  where $\U$ is regarded as the manifold described by ($\heartsuit$).
  Then,
  $\cD$ is a manifold diffeology for which any atlas of $\M$,
  according to ($\clubsuit$),
  is a generating family \art{Manifolds-as-diffeologies}.
  This diffeology is the {\em canonical diffeology} associated with $\M$.
  
  (B)  Conversely,
  let $\M$ be a manifold and let $\cA$ be any atlas generating the diffeology $\cD$ of $\M$,
  according to the definitions of \art{Manifolds-as-diffeologies}.
  Then,
  $\cA$ is an atlas,
  according to ($\clubsuit$),
  for $\M$.
  The atlas $\cA$ gives to $\M$ the canonical structure of classical manifold,
  associated with $\cD$.
  
  (C) The constructions (A) and (B) are the inverse
  of each other.
  They define a full faithful functor from the classical category of manifolds to the category $\Diffeology$.
  The image of this functor is the category $\Manifolds$ defined in \art{Manifolds-as-diffeologies}.
  
  %
  Therefore,
  there is no need to distinguish between these two classes of objects.
  We shall regard finite dimensional real manifolds,
  always,
  as defined by \art{Manifolds-as-diffeologies}.
  One can say then,
  that diffeology is a generalization of the notion of manifolds,
  but reducing diffeology to be just a generalization of manifolds would be misleading.
\end{article} %% Manifolds-the-classical-way

\begin{proof}
  (A) Let $\M$ be a classical manifold,
  and let $\cA$ be an atlas of $\M$ according to ($\clubsuit$).
  Let $\cD$ be the set of all the smooth parametrizations of $\M$,
  according to ($\diamondsuit$).
  The domains of these parametrizations are regarded as manifolds,
  according to ($\heartsuit$).
  Let us prove the three axioms of diffeology:
  
  D1. Let $m \in \M$ and ${\bf m} : r \to m$ be a constant parametrization.
  Let $\F$ be a chart of $\cA$ such that $\F(x) = m$.
  Since $\F$ is injective,
  ${\bf m}$ lifts along $\F$ by the constant parametrization $r \mapsto x$.
  Since constant parametrizations of domains are smooth,
  $\cD$ contains the constant parametrizations of $\M$.
  
  D2. Let $\P : \V \to \M$ be a parametrization which satisfies locally the condition ($\diamondsuit$).
  For all $r \in \V$ there exists an open neighborhood $\W$ of $r$ such that $\P \restriction \W$ is a smooth parametrization,
  according to ($\diamondsuit$).
  In other words,
  for every chart $\phi$ of $\W$,
  for every chart $\F : \U \to \M$, $(\P \circ \phi)^{-1}(\F(\U))$ is open,
  and $\F^{-1} \circ \P \circ \phi : (\P \circ \phi)^{-1}(\F(\U)) \to \U$ is smooth.
  But charts of domains are just local diffeomorphisms \art{Local-diffeomorphisms},
  thus $\phi$ can be simply replaced by the inclusion $j_\W : \W \hookrightarrow \V$.
  These conditions reduce then to the following one:
  there exists an open neighborhood $\W$ of $r$ such that $(\P \restriction \W)^{-1}(\F(\U))$ is open and $\F^{-1} \circ (\P \restriction \W) : (\P \restriction \W)^{-1}(\F(\U)) \to \U$ is either empty or smooth.
  But this is clearly a local condition.
  Therefore, if $\P$ belongs locally to $\cD$,
  then it belongs to $\cD$.
  
  D3. Let $\P : \V \to \M$ be a parametrization.
  We have seen just above that $\P$ belongs to $\cD$ if only if,
  for any chart $\F : \U \to \M$, $\P^{-1}(\F(\U))$ is open and $\F^{-1} \circ \P : \P^{-1}(\F(\U)) \to \U$ is either empty or smooth.
  Now,
  let $\Q : \W \to \V$ be a smooth parametrization.
  Let $\F : \U \to \M$ be a chart.
  Since $\Q$ is a smooth parametrization and $\P^{-1}(\F(\U))$ is open,
  $\Q^{-1}(\P^{-1}(\F(\U)))$ is open, that is,
  $(\P \circ \Q)^{-1}(\F(\U))$ is open.
  Then,
  since $\Q$ is smooth and $\F^{-1} \circ \P : \P^{-1}(\F(\U)) \to \U$ is either empty or smooth,
  $\F^{-1} \circ \P \circ \Q : (\P \circ \Q)^{-1}(\F(\U)) \to \U$ is either empty or smooth.
  Therefore,
  if
  $\P$ belongs to $\cD$ and $\Q$ is a smooth parametrization in the domain of $\P$,
  then $\P \circ \Q$ belongs to $\cD$.
  
  In order to complete the first point,
  it remains to check that any atlas $\cA$ of $\M$ is a generating family of $\cD$.
  Let $\P : \U \to \M$ be a plot of $\M$,
  that is, $\P \in \cD$.
  Let $r \in \U$,
  there exist a chart $\F$ of $\M$ and a point $x \in \Dom(\F)$ such that $\F(x) = \P(r)$,
  let $\U' = \P^{-1}(\Val(\F))$.
  Since $\P$ belongs to $\cD$,
  $\U'$ is an open neighborhood of $r$,
  and $\P' = \P \restriction \U'$ is still a plot of $\M$.
  But $\Val(\P') \subset \Val(\F)$ implies that $\Q = \F^{-1} \circ \P' \in \Cinfty(\U',\Dom(\F))$.
  Thus, $\Q$ is a smooth parametrization in $\Dom(\F)$ and $\F \circ \Q = \P \restriction \U'$.
  This is the condition to be a generating family of the diffeology $\cD$.
  
  (B) Let $\M$ be a manifold,
  and let $\cA$ be any atlas generating the diffeology $\cD$ of $\M$,
  according to the definitions of \art{Manifolds-as-diffeologies}.
  Since every element $\F$ of $\cA$ is a local diffeomorphism \art{Local-diffeomorphisms},
  $\F$ is an injection defined on a domain.
  Since compositions of local diffeomorphisms are either empty or local diffeomorphisms,
  the transition maps associated with the charts of the atlas $\cA$ are smooth.
  Therefore, $\cA$ defines on $\M$ a structure of classical manifold.
  
  (C) Let $\M$ be a set,
  $\cA$ and $\cA'$ two maximal atlases according to ($\clubsuit$),
  defining the same diffeology $\cD$.
  Every chart $\F \in \cA$ is a local diffeomorphism for $\cD$,
  as well as every chart $\F' \in \cA'$.
  Since the composite of two local diffeomorphisms for a given diffeology is either empty or again a local diffeomorphism \art{Local-diffeomorphisms},
  $\F'^{-1} \circ \F$,
  defined on $\F^{-1}(\Val(\F'))$,
  is either empty or a local diffeomorphism between real domains,
  necessarily of the same dimension.
  But this is the condition for $\F$ and $\F'$ to be compatible.
  Since $\cA$ and $\cA'$ are maximal,
  $\cA = \cA'$, and the two classical manifold structures are equal.
  Thus,
  the map defined in (A) is injective,
  and clearly surjective,
  thanks to (B).
  Therefore,
  it defines a one-to-one correspondence between the objects of the category of classical manifolds and the objects of the category $\Manifolds$ defined in \art{Manifolds-as-diffeologies}.
  Then, according to (A) and (B) the saturated atlas of a given classical $n$-dimensional manifold $\M$ is the set of local diffeomorphisms from $\RR^n$ to $\M$,
  regarded as a diffeological space.
  Therefore,
  (A) and (B) are the inverse of each other.
  
  Finally,
  let us check that,
  given two classical manifolds $\M$ and $\M'$,
  the smooth maps from $\M$ to $\M'$,
  according to ($\diamondsuit$),
  are exactly the smooth maps with $\M$ and $\M'$ regarded as diffeological spaces.
  In fact,
  since the charts for a given classical $n$-dimensional manifold are just local diffeomorphisms with $\RR^n$,
  with respect to the associated diffeology,
  and since the associated diffeology is precisely generated
  by these local diffeomorphisms, this proposition is a
  direct application of the criterion \art{Lifting-smooth-maps-along-generating-families};
  see also Figure \ref{A-smooth-map-between-manifolds}.
  Therefore,
  the functor,
  mapping a classical manifold to its associated diffeology,
  is a full faithful functor to $\Manifolds$.
\end{proof}

\begin{article}\artlabel{Submanifolds of a diffeological  space}
  \addcontentsline{toc}{section}{\small\hspace{10pt} Submanifolds of a diffeological space}
  \label{Submanifolds-of-a-diffeological-space}
  The definition of {\em submanifolds} follows immediately the definition of manifolds \art{Manifolds-as-diffeologies}.
  Let $\X$ be a diffeological space and let $\M$ be a subset of $\X$.
  If $\M$,
  equipped with the subset diffeology \art{Subspaces-and-subset-diffeology},
  is a manifold,
  we shall say that $\M$ is a {\em submanifold}\index{Submanifold} of $\X$.
  
  \Note{1} The vocabulary used here ``$\M$ is a submanifold of $\X$'',
  can lead to confusion since $\X$ itself is not necessarily a manifold (like in the example below).
  But we have no real choice.
  We could say ``$\M$ is a manifold in $\X$'' but this is longer,
  it is not compatible with the standard vocabulary when $\X$ is itself a manifold,
  and it attenuates the fact that $\M$ is regarded as equipped with the subset diffeology.
  So,
  we choose to say ``submanifold'' even if the ambient space is not itself a manifold.
  We have just to
  be aware of that,
  and cautious.
  
  \Note{2} By definition,
  a submanifold is always {\em induced},
  $\M$ is a subspace of $\X$,
  the inclusion $\M \hookrightarrow \X$ is an induction.
  It can also be {\em embedded}
  ---~if the inclusion of ${\M}$ into $\X$ is moreover an embedding \art{Embeddings}.
  
  \begin{figure}[t]
    \centerline{\includegraphics[width=0.6\textwidth]{Figures-PDF/fig-semi-cubic}}
    \caption{The semi-cubic $y^2 = x^3$.}
    \label{fig-semi-cubic}
  \end{figure}
  
  \Example{1}~The sphere (\exref{The-sphere-as-diffeological-subspace}) is an example of a embedded submanifold,
  while the irrational solenoid (\exref{The-irrational-solenoid-is-not-embedded}) is again an example of submanifold,
  but not embedded.
  In  \exref{Embedding-GL-n-R-in-Diff-R-n},
  we have seen that the group $\GL(n,\RR)$ is embedded in $\Diff(\RR^n)$,
  equipped with the functional diffeology.
  Since the group $\GL(n,\RR)$ is a manifold for the subset diffeology,
  equivalent to the open subset of the $n\times n$ matrices with nonzero determinant,
  $\GL(n,\RR)$ is an embedded submanifold of $\Diff(\RR^n)$.
  
  \Example{2}~The semi-cubic\index{Semi-cubic} $y^2 = x^3$ in $\RR^2$ (Figure \ref{fig-semi-cubic}) is a counter-intuitive example of a diffeological submanifold.%
  \footnote{This was still unknown to me for the first edition.
  I had to leave this question open.}
  This is a corollary of the Joris theorem \cite{Jor82} that states,
  in particular,
  that if a map $t \mapsto f(t)$ is such that $f(t)^2$ and $f(t)^3$ are smooth,
  then $f$ is smooth.
  Therefore,
  the smooth injection $j : t \mapsto (t^2,t^3)$ is an induction from $\RR$ into $\RR^2$.
  Its image $\fS$,
  the semi-cubic,
  is then a submanifold of $\RR^2$ diffeomorphic to $\RR$,
  in spite of the cusp at $0$.
  Moreover,
  $j$ is an embedding \art{Embeddings}:
  $\fS$,
  equipped with the subset diffeology,
  is an embedded submanifold in $\RR^2$,
  but not smoothly embedded \xart{Embeddings}{Note 2}.
  The nature of the singularity of the cusp does not lie in the induction itself but in its relationship with the ambient space:
  a local diffeomorphism of $\RR^2$ preserving the semi-cubic $\fS$ would fix the cusp,
  while a local diffeomorphism of $\fS$ would send the cusp anyywhere;
  see \cite{PIZ22c}.
\end{article} %% Submanifolds-of-a-diffeological-space

\begin{proof}
  That concerns only the semi-cubic example.
  Clearly $j : t  \mapsto (x = t^2,y = t^3)$ is smooth,
  and injective:
  $t = \sqrt[3]{y}$. Let $r \mapsto \P(r) = (x(r),y(r))$ be a plot in $\RR^2$ with value in $\fS$.
  Then,
  $j^{-1}\circ\P(r) = t(r)$ such that $r \mapsto t(r)^2$ and $r \mapsto t(r)^3$ are smooth.
  To apply Joris theorem we need to come back to a map from $\RR$ to $\RR$.
  We can use Boman's theorem that states that $r \mapsto t(r)$ is smooth at a point $r_0$ if and only if,
  the composite $s \mapsto r(s) \mapsto t(r(s))$ is smooth for any smooth path $s \mapsto r(s)$,
  passing through $r_0$ at $s_0$ \cite{Bom67};
  see \art{Smooth-parametrizations-in-domains}.
  But,
  for each such smooth path,
  the composite $s \mapsto t(r(s))$ is smooth,
  thanks indeed to Joris theorem.
  Thus,
  $r \mapsto \P(r)$ is smooth.
  Now,
  assume that the embedding is smooth and let $\varphi : t \mapsto t+1$ regarded as a local diffeomorphism of $\RR$ mapping $0$ to $1$.
  Suppose that there exists a local difffeomorphism $\phi$ of $\RR^2$ that fix $\fS$ and maps $j(0) = (0,0)$ to $j(1) = (1,1)$;
  so,
  $\phi \circ j =_\loc j \circ \varphi$.
  The tangent linear map at $t = 0$ gives $\D(\phi)(j(0)) \circ \D(j)(0) = \D(j)(\varphi(0)) \circ \D(\varphi)(0)$,
  that is $\D(\phi)(0,0) \circ \D(j)(0) = \D(j)(1) \circ \D(\varphi)(0)$,
  and then $0=1$,
  which is absurd.
\end{proof}

\begin{article}\artlabel{Immersed submanifolds are not submanifolds?}%
  \addcontentsline{toc}{section}{\small\hspace{10pt} Immersed submanifolds are not submanifolds?}%
  \label{Immersed-submanifolds-are-not-submanifolds}%
  ~Let $\N$ and $\M$ be two manifolds,
  and let $j: \N \to \M$ be an immersion,
  see \exref{Immersions-of-real-domains}.
  Then,
  the image $j(\N)$ is not necessarily a submanifold,
  because an immersion is not necessarily an induction;
  see \exref{The-infinite-symbol}.
  Therefore,
  if there is a well defined concept of {\em immersion},
  the concept of ``immersed submanifold'' is more ambiguous.
  Immersions do not always have sub-manifolds as images,
  and we even know now that a smooth injection can have a submanifold as an image without being an immersion,
  as proved by the semi-cubic example.
\end{article} %% Immersed-submanifolds-are-not-submanifolds

\begin{article}\artlabel{Quotients of manifolds}
  \addcontentsline{toc}{section}{\small\hspace{10pt} Quotients of manifolds}
  \label{Quotients-of-manifolds}
  This question is often asked in differential geometry:
  When is the quotient of a manifold
  ---~or more generally,
  the quotient of a diffeological space~---
  by an equivalence relation,
  a manifold?
  There is no simple answer to this question because it depends too much on the nature of the diffeological space and of the equivalence relation.
  But,
  in some circumstances,
  the following characterization can be useful.
  Let $\X$ be a diffeological space,
  and let $\sim$ be an equivalence relation defined on $\X$.
  Let $\M = \quotient{\X}{\sim}$ be the diffeological quotient \art{Quotient-and-quotient-diffeology}.
  Let $\pi: \X \to {\M}$ be the associated projection.
  The quotient space $\M$ is a manifold of dimension $n$ if and only if there exists a family of $n$-plots $\{\phi_i: \U_i \to \X\}_{i \in \cI}$,
  where $\cI$ is any set of indices,
  such that the following conditions are fulfilled:
  
  1. For every $i \in \cI$,
  the map $\phi_i: \U_i \to \X$ is an induction such that $\pi \circ \phi_i$ is injective.
  Equivalently,
  the image of $\U_i$ by the induction $\phi_i$ cuts each class of the relation $\sim$ in one point,
  at most.
  
  2. The projections of the values of the inductions $\phi_i$ cover $\M = \bigcup_{i \in \cI} \pi\circ\phi_i(\U_i)$.
  
  3. For every $r_i \in \U_i$ and $r_j \in \U_j$ such that $\pi(\phi_i(r_i)) = \pi(\phi_j(r_j))$,
  there exists a local diffeomorphism $\psi$,
  defined on some open neighborhood $\V$ of $r_i$ to $\U_j$,
  mapping $r_i$ to $r_j$,
  and such that $\pi\circ\phi_j \circ \psi = \pi \circ (\phi_i \restriction \V)$.
  
  4. For every plot ${\P}: \W \to \X$,
  for every $r \in {\W}$,
  there exist an open neighborhood $\V \subset {\W}$ of $r$,
  some index $i \in \cI$ and a plot $\Q: \V \to \phi_i(\U_i) \subset \X$ such that $\pi \circ {\P} \restriction \V =  \pi \circ \Q$.
  
  If these conditions are satisfied,
  each map $\F_i = \pi \circ \phi_i$,
  with $i \in \cI$, is a chart of $\M$,
  and the set $\cA = \{\F_i\}_{i \in \cI}$ is an atlas of $\M$.
  
  \Note~The diffeology of the quotient $\M$ is well defined as the quotient diffeology of $\X$,
  there is no choice here.
  This condition,
  stated above,
  answers only this question:
  Is the quotient diffeology a manifold diffeology?
  It may be the case,
  or not.
\end{article} %% Quotients-of-manifolds

%%###########
\begin{figure}[tb]
  \centerline{\includegraphics[width=.95\textwidth]{Figures-PDF/fig-manifold-quotient}}
  \caption{The quotient is a manifold.}
  \label{fig-The-quotient-is-a-manifold}
\end{figure}
%%###########

\begin{proof}
  Let us assume that the quotient $\M$ is a manifold of dimension $n$.
  Then there exists a family of local diffeomorphisms from $\RR^n$ to $\M$ which generates the diffeology of $\M$ \art{Local-modeling-of-manifolds}.
  Let $\F: \U \to {\M}$ be such a diffeomorphism,
  and let $r \in \U$.
  Then,
  since $\pi: \X \to {\M}$ is a subduction,
  there exists a local lift $\phi: \V \to \X$ such that $\phi \in \cD(\V)$ and $\pi \circ \phi = \F \restriction \V$.
  Let us check that $\phi$ is an induction,
  using the criterion \art{Criterion-for-being-an-induction}.
  First of all $\phi$ is smooth.
  Since $\pi \circ \phi$ is injective,
  so is $\phi$.
  Now,
  let $\Q: {\W} \to \phi(\V)$ be a plot in $\X$,
  then $\pi \circ \Q \in \cC({\W})$.
  Since $\F$ is a chart of $\M$,
  that is,
  a local diffeomorphism,
  the composite $\F^{-1} \circ \pi \circ \Q$,
  restricted to its domain,
  is smooth.
  But,
  $\pi \circ \phi = \F \restriction \V \Rightarrow \phi^{-1} =  \F^{-1} \circ \pi$,
  where $\F^{-1}$ is restricted to $\pi(\phi(\V))$,
  thus $\F^{-1} \circ \pi \circ \Q = \phi^{-1} \circ \Q$.
  Hence,
  $\phi^{-1} \circ \Q$ is smooth,
  and $\phi$ is an induction.
  Now,
  since $\pi \circ \phi = \F \restriction \V$,
  $\pi \circ \phi$ is injective,
  which means that if $\phi(r)$ and $\phi(r')$ belong to the same equivalence class,
  then $r = r'$.
  Then,
  since for every point $m \in \M$ there exist a chart $\F: \U \to \M$ and a point $r \in \U$ such that $\F(r) = m$,
  there exists a family of such local lifts $\phi$ whose projections $\pi \circ \phi$ cover ${\M}$.
  We proved the first and second points.
  Let us prove the third one.
  Let us consider such a family $\{\phi_i: \U_i \to \X\}_{i \in \cI}$ of inductions made of local lifts of charts of $\M$.
  Let $r_i \in \U_i$ and $r_j \in \U_j$ such that $\phi_i(r_i) = \phi_j(r_j) = m$.
  Let $\F_i$ and $\F_j$ be two charts of $\M$ such that $\F_i = \pi \circ \phi_i$ and $\F_j = \pi \circ \phi_j$.
  Then,
  $\F_i(r_i) = \F_j(r_j) = m$ and there exists a local diffeomorphism $\psi$ defined on an open neighborhood of $r_i$ such that $\psi(r_i) = r_j$ and $\F_j \circ \psi = \F_i$.
  Hence,
  $\pi \circ \phi_j \circ \psi = \pi \circ \phi_i$,
  and this is the third point.
  Finally,
  let us consider a plot ${\P}: {\W} \to \X$,
  thus $\pi \circ {\P}$ is a plot of $\M$.
  Let $r \in {\W}$ and $m = \pi \circ {\P}(r)$,
  there exist a chart $\F_i: \U_i \to \M$,
  an open neighborhood $\V \subset {\W}$ of $r$ and a plot ${\P}': \V \to \U_i$
  such that $\F_i \circ {\P}' = \pi \circ {\P}\restriction \V$.
  Now,
  $\F_i = \pi \circ \phi_i$,
  then $\pi \circ \phi_i \circ {\P}' = \pi \circ {\P}\restriction \V$,
  denoting by $\Q = {\P}' \circ \phi_i: \V \to \phi_i(\U_i)$
  we get $\pi \circ \Q = \pi \circ {\P}\restriction \V$.
  The fourth point is proved.
  
  Thus,
  we proved that the four points of the proposition are satisfied if the quotient $\M$ is a manifold;
  see Figure \ref{fig-The-quotient-is-a-manifold}.
  Let us prove now that if these four points are satisfied,
  then $\M$ is a manifold.
  %%###########
  %    \begin{figure}[tb]
  %      \centerline{\includegraphics{Figures-PDF/fig-manifold-quotient}}
  %      \caption{The quotient is a manifold.}
  %      \label{fig-The-quotient-is-a-manifold}
  %    \end{figure}
  %%###########
  Let us define $\F_i = \pi \circ \phi_i$, $\F_i$ is injective and smooth.
  Let us show that $\F_i$ is a local diffeomorphism \art{Local-diffeomorphisms}.
  Let ${\P}: \cO \to {\M}$ be a plot,
  and let $r \in \cO$ such that $m = {\P}(r) \in \F_i(\U_i)$.
  Then,
  there exists a local lift ${\P}': {\W} \to \X$ such that $\pi \circ {\P}' = {\P}\restriction {\W}$ and $r \in {\W}$.
  Thanks to the fourth point of the proposition,
  there exist an index $j \in \cJ$,
  a subset $\V \subset {\W}$,
  and a plot $\Q: \V \to \phi_j(\U_j)$ of $\X$ such that $\pi \circ {\P}'\restriction \V = \pi \circ \Q$.
  Hence,
  $\pi \circ \Q = {\P}\restriction \V$ and $\Q(\V)\subset \phi_j(\U_j)$.
  Let $r_j = \phi_j^{-1}(\Q(r)) \in \U_j$ and $r_i \in \U_i$ such that $\F_i(r_i) = m$.
  We have $m = \pi \circ \phi_i(r_i) = \pi \circ \phi_j(r_j)$,
  and thanks to the third condition of the proposition,
  there exists a local
  diffeomorphism $\psi$ defined on an open neighborhood of $r_j$,
  mapping $r_j$ to $r_i$ and such that $\pi\circ\phi_i \circ \psi = \pi \circ \phi_j$,
  restricted to this neighborhood.
  Thus,
  $\F_i^{-1} \circ \pi \restriction \Q(\V) = \psi \circ \phi_j^{-1}\restriction \Q(\V)$,
  hence $\F_i^{-1} \circ {\P}\restriction \V = \F_i^{-1} \circ \pi \circ \Q \restriction \V = \psi \circ \phi_j^{-1} \circ \Q$.
  But,
  $\phi_j$ being an induction, $\phi_j^{-1} \circ \Q$ is smooth,
  and so is $\psi \circ \phi_j^{-1} \circ \Q$.
  Thus,
  $\F_i^{-1} \circ {\P}\restriction \V$ is smooth.
  Therefore,
  $\F_i^{-1} \circ \P$ is smooth,
  that is, $\F_i^{-1}$ is smooth.
  Therefore, $\F_i$ is a local diffeomorphism.
  In conclusion,
  the set $\cA = \{\F_i\}_{i \in \cI}$ is an atlas of $\M$, where
  $\F_i = \pi \circ \phi_i$, and $\M$ is a $n$-manifold.
\end{proof}

%%%%%%%%%%%%%%%%%%%%%%%%%%%%%%%%%%%%%%%%%%%%%%%%%%%%%%%%%%
%
%   Exercises
%
%%%%%%%%%%%%%%%%%%%%%%%%%%%%%%%%%%%%%%%%%%%%%%%%%%%%%%%%%%

\Exercises

\begin{exercise}[The irrational torus is not a manifold]
  \label{The-irrational-torus-is-not-a-manifold}
  Using the criterion \art{Quotients-of-manifolds},
  show that the irrational torus $\Torus_\alpha$ of \exref{Diffeomorphisms-between-irrational-tori},
  is not a manifold.
\end{exercise} %% The-irrational-torus-is-not-a-manifold

\begin{exercise}[The sphere as paragon]
  \label{The-sphere-as-parangon}
  Let us come back to the
  sphere $\S^n$ of \exref{The-sphere-as-diffeological-subspace}.
  The sphere is an important example which deserves to be watched under several points of view.
  Let us consider $\RR^{n+1}$ equipped with the standard diffeology,
  the sphere $\S^n$ is the subspace defined by\,:
  $$%
  \S^n = \{ x \in \RR^{n+1} \mid \norm{x} = 1 \},
  $$%
  where $\norm{\cdot}$ denotes the usual Euclidean norm associated with the scalar product,
  denoted by $(u,v) \mapsto u \cdot v$.
  Let us recall that a plot of $\S^n$ is a plot of $\RR^{n+1}$ with values in $\S^n$ \art{Subspaces-and-subset-diffeology}.
  With every $x \in \S^n$,
  let us associate $\E_x\subset \RR^{n+1}$,
  defined as the subspace orthogonal to $x$;
  see Figure \ref{The-stereographic-projection}.
  $$%
  \E_x = \{v \in \RR^{n+1} \mid x \cdot v = 0\}.
  $$%
  Let $\F_x$ be the map defined by
  $$%
  \F_x: \E_x \to \S^n, \text{ with }  v \mapsto u = 2{v+x \over 1+\norm{v}^{2}} - x.
  $$%
  The map $\F_x$ is called the {\em stereographic projection}
  with respect to  $x$.
  
  \Question{1)} Show that $\F_x$ is a local diffeomorphism from $\E_x$
  to $\S^n$.
  
  \Question{2)} Deduce that $\S^n$ is a manifold of dimension $n$.
  
  \Question{3)} Show that the pair $\set{\F_\N,\F_{-\N}}$,
  where $\N = (0_n,1) \in \S^n$ is the North Pole,
  is also a generating family for $\S^n$,
  that is, an atlas of $\S^n$.
  %%###########
  \begin{figure}[tb]
    \centerline{\includegraphics{Figures-PDF/fig-stereographic}}
    \caption{The stereographic projection.}
    \label{The-stereographic-projection}
  \end{figure}
  %%###########
\end{exercise} %% The-sphere-as-parangon

%%%%%%%%%%%%%%%%%%%%%%%%%%%%%%%%%%%%%%%%%%%%%%%%%%%%%%%%%%
%% MARK: Diffeological Manifolds
%%%%%%%%%%%%%%%%%%%%%%%%%%%%%%%%%%%%%%%%%%%%%%%%%%%%%%%%%%

\section*{Diffeological Manifolds}
\label{Section-Diffeological-manifolds}

\begin{sechead}
  The construction of manifolds \art{Manifolds-as-diffeologies} on the one hand,
  and the existence of diffeological vector spaces \art{Diffeological-vector-spaces} on the other hand,
  suggest the introduction of a new diffeological category: the {\em diffeological manifolds},
  which are diffeological spaces {\em modeled} on diffeological vector spaces.
  Since finite dimensional vector spaces are diffeological vector spaces,
  the traditional manifolds then become a special case (or a subcategory) of diffeological manifolds.
  The goal of this section is to give the general definition of these new objects,
  and to express some of their properties.
  We shall illustrate these definitions by two examples:
  the {\em infinite sphere} of the fine standard Hilbert space \art{The-fine-standard-Hilbert-space},
  and the associated {\em infinite projective space}.
  We shall have the opportunity,
  through these examples,
  to see how diffeology works with infinite dimensional manifolds.
\end{sechead}

\begin{article}\artlabel{Diffeological Manifolds}
  \addcontentsline{toc}{section}{\small\hspace{10pt} Diffeological Manifolds}
  \label{Diffeological-Manifolds}
  Let $\E$ be a diffeological vector space \art{Diffeological-vector-spaces}.
  Let $\X$ be a diffeological space.
  We shall say that $\X$ is a {\em diffeological manifold modeled on $\E$}\index{Diffeological manifold} if $\X$ is locally diffeomorphic to $\E$ at any point.
  That is,
  if for every $x \in \X$ there exists a local diffeomorphism \art{Local-diffeomorphisms} $\F : \E \supset \U \to \X$ such that $x \in \F(\U)$.
  Such a local diffeomorphism will be called a chart of $\X$\index{Chart}.
  Diffeological manifolds form a subcategory of $\Diffeology$ still denoted by $\Manifolds$.
  The basic examples of diffeological manifolds are diffeological vector spaces and ordinary manifolds modeled on $\RR^n$ \art{Manifolds-as-diffeologies},
  but also manifolds modeled on a Banach space;
  see \exref{Banach-s-diffeology}.
  
  \Note{1} In practice,
  it happens that the space $\X$ is,
  relatively obviously,
  locally diffeomorphic to an element of a family $\cE = \{\E_i\}_{i \in \cI}$ of isomorphic diffeological vector spaces;
  the set of indices $\cI$ often being $\X$ itself.
  This is an elementary case of modeling diffeological spaces (see \art{In-conclusion-on-modeling}).
  We encountered this situation,
  in particular,
  with the sphere in \exref{The-sphere-as-parangon}.
  
  \Note{2} Some definitions of manifolds,
  in differential geometry,
  allow the model vector space to change its type from one point to another,
  resulting in variable dimensional manifolds for the different connected components.
  I does not seem necessary to introduce this subtlety here,
  but it is always possible\ldots
\end{article} %% Diffeological-Manifolds

\begin{article}\artlabel{Generating diffeological manifolds}
  \addcontentsline{toc}{section}{\small\hspace{10pt} Generating diffeological manifolds}
  \label{Generating-diffeological-manifolds}
  The proposition \art{Local-modeling-of-manifolds} adapts itself to the case of diffeological manifolds.
  Let $\X$ be a diffeological space.
  Let $\E$ be a diffeological vector space.
  Every family $\cA$ of local diffeomorphisms \art{Local-diffeomorphisms},
  from $\E$ to $\X$,
  such that
  $$%
  \bigcup_{\F \in \cA} \Val(\F) = \X
  $$%
  {\em generates} the diffeology of $\X$ in the following meaning:
  for every plot $\P : \U \to \X$,
  for every point $r \in \U$,
  there exist an open neighborhood $\V$ of $r$,
  an element $\F$ of $\cA$,
  and a plot $\Q : \V \to \E$,
  such that $\P \restriction \V = \F \circ \Q$.
  If there exists such a family $\cA$ of local diffeomorphisms,
  from $\E$ to $\X$,
  $\cA$ will be called an {\em $\E$-atlas},
  or simply an {\em atlas}\index{Atlas},
  of $\X$,
  and the elements of $\A$ will be called the {\em charts}\index{Chart} of the atlas.
  Thus,
  a diffeological space $\X$ is a manifold modeled on $\E$ \art{Diffeological-Manifolds} if and only if there exists an atlas $\cA$ of $\X$ made of local diffeomorphisms from $\E$ to $\X$.
  Note that,
  in this case,
  there exists an atlas made up with all the local
  diffeomorphisms from $\E$ to $\X$,
  this atlas is called the {\em saturated atlas} of $\X$.
\end{article} %% Generating diffeological manifolds

\begin{proof}
  Let $\P : \U \to \X$ be a plot and $r \in \U$.
  Let $x = \P(r)$ and $\F \in \cA$ such that $x \in \Val(\F)$.
  Such an $\F$ exists by hypothesis.
  Now,
  let $\V = \P^{-1}(\Val(\F))$ and $\Q = \F^{-1}\circ (\P \restriction \V)$,
  that is,
  $\P \restriction \V = \F \circ \Q$.
  Now,
  since $\F$ is a local diffeomorphism,
  $\Val(\F)$ is D-open \art{Local-smooth-maps-are-defined-on-D-opens},
  and since $\P$ is D-continuous \art{Smooth-maps-are-D-continuous},
  $\V$ is a domain.
  Moreover,
  since $\F$ is a local diffeomorphism,
  $\Q$ is a plot of $\E$.
  Therefore,
  every plot of $\X$ can be smoothly lifted,
  locally,
  along some element of the family $\cA$,
  as it is claimed by the proposition.
  Now,
  let $\X$ be generated by a family $\cA$ of local diffeomorphisms from $\E$ to $\X$.
  Pick any point $x \in \X$, and let  $\P:\{0\}\to \X$ be the constant plot such that $\P(0) = x$.
  By hypothesis,
  there exist a chart $\F \in \cA$ and a lift $\Q : \{0\} \to \E$ such that $\P = \F \circ \Q$.
  Hence,
  $\F$ is a local diffeomorphism from $\E$ to $\X$,
  such that $x \in \Val(\F)$.
  Therefore,
  $\X$ is a diffeological manifold modeled on $\E$.
\end{proof}

\begin{article}\artlabel{The infinite sphere}
  \addcontentsline{toc}{section}{\small\hspace{10pt} The infinite sphere}
  \label{The-infinite-sphere}
  Let $\cH_\R$ be the real Hilbert space of square summable real sequences,
  equipped with the fine diffeology \art{The-fine-standard-Hilbert-space}.
  The unit sphere $\cS_\RR$ of the Hilbert space $\cH_\RR$ is defined as usual by
  $$%
  \cS_\RR = \bigg\{ \X \in \cH_\RR \ \bigg\vert \ \X \cdot \X = \sum_{k = 1}^\infty \X_k^2 = 1 \bigg\}.
  $$%
  The sphere $\cS_\RR$ will be called the {\em infinite sphere} in the following,
  and it will be equipped with the subset diffeology \art{Subspaces-and-subset-diffeology} of the fine diffeology \art{The-fine-diffeology-of-vector-spaces} of $\cH_\RR$.
  Let us recall that a plot of $\cH_\RR$ for the fine diffeology is a parametrization such that,
  for all $r_0 \in \U$,
  there exist an open neighborhood $\V$ of $r_0$ and a finite local family $(\lambda_\alpha,\X_\alpha)_{\alpha \in \A}$,
  defined on $\V$, such that
  $$%
  \P \restriction \V : r \mapsto \sum_{\alpha \in \A} \lambda_\alpha(r) \X_\alpha.
  $$%
  Let us recall that to be a {\em finite local family defined on $\V$} means that the set of indices $\A$ is finite,
  and for every $\alpha \in \A$,
  $\lambda_\alpha \in \Cinfty(\V,\RR)$,
  and $\X_\alpha \in \cH_\RR$.
  Now,
  the plot $\P$ is a plot of the sphere $\cS_\RR$ if moreover it takes its values in $\cS_\RR$,
  that is, if $\sum_{\alpha \in \A}\sum_{\alpha' \in \A} \lambda_\alpha(r) \lambda_{\alpha'}(r)\ \X_\alpha \cdot \X_{\alpha'} =1$ for all $r \in \V$.
  Let us then define the following {\em stereographic maps\/}:
  $$%
  \F_+:\cH_{\RR} \to \cS_{\RR}, \text{ with }  \F_+: \xi \mapsto \X = { 1 \over \norm{\xi}^2 + 1}
  \begin{pmatrix}
    \norm{\xi}^2 - 1 \\ 2 \xi
  \end{pmatrix},
  $$%
  %
  $$%
  \F_-: \cH_{\RR} \to \cS_{\RR}, \text{ with } \F_-: \xi \mapsto \X = { 1 \over 1 +\norm{\xi}^2}
  \begin{pmatrix}
    1 - \norm{\xi}^2 \\ 2 \xi
  \end{pmatrix},
  $$%
  where the matrix notation denotes the corresponding sequences,
  belonging to the vector space $\cH_{\RR}$ defined by
  $$%
  \vect{a \\ \xi} = (a, \xi_1,\xi_2,\ldots)\,, \text{ where } a \in \RR \text{ and } \xi = (\xi_1,\xi_2,\ldots) \in \cH_{\RR}.
  $$%
  Let us use the notation of \art{The-fine-standard-Hilbert-space},
  where $\ee_k$ is the vector whose $k$-th coordinate is $1$ and the others are zero.
  The images of $\F_+$ and $\F_-$ are
  $$%
  \F_+(\cH_\RR) = \cS_{\RR} - \{\ee_1\}, \text{ and } \F_-(\cH_\RR) = \cS_{\RR} - \{-\ee_1\}.
  $$%
  Let us denote now,
  for every $\X \in \cH_{\RR}$
  $$%
  \X = (\X_1,\X_+), \text{ with } \X_1 = \pr_1(\X), \text{ and } \X_+ = (\X_2,\X_3,\ldots) \in \cH_{\RR},
  $$%
  where $\pr_k$ denotes the $k$-th projection on $\RR$,
  notations of \art{The-fine-standard-Hilbert-space}.
  The stereographic maps are injective and their inverses are given by
  $$%
  \F_+^{-1} : \cS_{\RR} - \{\ee_1\} \to \cH_{\RR}, \text{ with } \F_+^{-1} : \X \mapsto \xi = { \X_+ \over 1 - \X_1},
  $$%
  %
  $$%
  \F_-^{-1} : \cS_{\RR} - \{-\ee_1\} \to \cH_{\RR}, \text{ with } \F_-^{-1} : \X \mapsto \xi = { \X_+ \over 1 + \X_1}.
  $$%
  Now,
  the stereographic maps are local diffeomorphisms.
  Since their images cover the whole sphere $\cS_\RR$,
  the sphere $\cS_\RR$ is a diffeological manifold modeled on $\cH_\RR$ \art{Generating-diffeological-manifolds}.
  Moreover,
  the sphere $\cS_\RR$ is embedded in $\cH_\RR$ \art{Embedded-subsets-of-a-diffeological-space}.
\end{article} %% The infinite sphere

\begin{proof}
  Let us show that the stereographic maps are local diffeomorphisms from $\cH_{\RR}$ to $\cS_{\RR}$.
  We shall just consider $\F_+$,
  since the case $\F_-$ is completely analogous.
  The fact that the map $\F_+$ is injective is a simple verification.
  Its domain is $\cH_\RR$ which is,
  of course,
  D-open in $\cH_\RR$.
  We shall prove that $\cS_\RR - \{ \ee_1 \}$,
  the image of $\F_+$,
  is D-open in $\cS_\RR$.
  Then,
  we shall prove that $\F_+$ is smooth as well as  $\F_+^{-1}$,
  defined on $\cS - \{ \ee_1 \}$,
  equipped with the subset diffeology.
  Finally,
  applying the criterion \art{Local-smooth-maps-are-defined-on-D-opens},
  it will follow that $\F_+$ is a local diffeomorphism.
  
  (a) {\em The map $\F_+$ is injective:}
  We already exhibited $\F_+^{-1}$.
  
  (b) {\em The map $\F_+$ is smooth:}
  Let us consider a parametrization $\P: \U \to \cH_{\RR}$.
  For every $r_0 \in \U$ there exist an open neighborhood $\V$ of $r_0$ in $\U$ and a local family $(\lambda_\alpha,\X_\alpha)_{\alpha \in \A}$,
  defined on $\V$, such that
  $$%
  \P \restriction \V : r \mapsto \sum_{\alpha \in \A} \lambda_\alpha(r) \X_\alpha.
  $$%
  Thus,
  $$%
  \F_+\circ (\P \restriction \V) : r \mapsto \vect{ \epsilon(r) \\
  \sum_{\alpha \in \A} \mu_\alpha(r) \X_\alpha}
  $$%
  with
  $$%
  \epsilon(r) = { \norm{\sum_{\alpha \in \A} \lambda_\alpha(r)\X_\alpha}^2 - 1 \over \norm{\sum_{\alpha \in \A} \lambda_\alpha(r)\X_\alpha}^2 +1},
  \text{ and } \mu_\alpha(r) = { 2 \lambda_\alpha(r) \over \norm{\sum_{\beta \in \A} \lambda_\beta(r)\X_\beta}^2 +1}\,.
  $$%
  The denominator of $\epsilon$ and $\mu_\alpha$ never vanishes.
  Hence,
  the functions $\epsilon$ and $\mu_\alpha$ belong to $\cC^{\infty}(\V,\RR)$.
  Now $\F_+\circ (\P \restriction \V)$ rewrites
  $$%
  \F_+\circ (\P \restriction \V)(r) = \epsilon(r) \vect{1 \\ 0} + \sum_{\alpha \in \A} \mu_\alpha(r) \vect{0 \\ \X_\alpha}.
  $$%
  This exhibits the map $\F_+\circ (\P \restriction \V)$ as a finite linear combination of vectors of $\cH_{\RR}$ with smooth parametrizations of $\RR$ as coefficients.
  Therefore,
  $\F_+\circ \P$ is a plot of $\cH_{\RR}$.
  Since $\F_+\circ \P$ takes its values in $\cS_{\RR}$,
  it is a plot of the subset diffeology of $\cS_{\RR} \subset \cH_{\RR}$,
  where $\cH_{\RR}$ is equipped with the fine diffeology.
  Hence,
  $\F_+$ is smooth.
  
  (c) {\em The map $\F_+^{-1}$ is smooth:}
  Let $\P:\U\to \cS_{\RR}-\{\ee_1\}$ be a plot.
  For any $r_0 \in \U$,
  there exist an open neighborhood $\V$ of $r_0$ in $\U$ and a local family $(\lambda_\alpha,\X_\alpha)_{\alpha \in \A}$,
  defined on $\V$,
  such that
  $$%
  \P \restriction \V : r \mapsto \sum_{\alpha \in \A}
  \lambda_\alpha(r) \X_\alpha,
  $$%
  then,
  $$%
  \F_+^{-1} \circ (\P \restriction \V) : r \mapsto \sum_{\alpha \in \A} \mu_\alpha(r) \X_{\alpha,+},
  \text{ with } \mu_\alpha(r) = { \lambda_\alpha(r) \over 1 - \sum_{\beta \in \A}\lambda_\beta(r)\X_{\beta,1}}.
  $$%
  Now,
  the $\X_{\beta,1}$ form a finite set of constant numbers,
  thus the parametrization $r \mapsto \sum_{\beta \in \A}\lambda_\beta(r)\X_{\beta,1}$ is smooth and is never equal to 1 since $\P$ takes its values in $\cS_{\RR}-\{\ee_1\}$.
  Thus,
  for each $\alpha \in \A$,
  $\mu_\alpha(r)$ is a smooth parametrization in $\RR$.
  The parametrization $\F_+^{-1} \circ (\P \restriction \V)$ is clearly a finite linear combination of vectors of $\cH_{\RR}$,
  with smooth parametrizations of $\RR$ as coefficients.
  Hence,
  $\F_+^{-1} \circ (\P \restriction \V)$ is a plot for the fine diffeology of $\cH_{\RR}$.
  Now,
  $\F_+^{-1} \circ \P$ is locally, at each point of $\U$,
  a plot of $\cH_{\RR}$, so it is a plot of $\cH_{\RR}$.
  Therefore,
  $\F_+^{-1}$ is a smooth map from $\cS_{\RR}-\{\ee_1\}$ to $\cH_{\RR}$.
  
  (d) {\em The subset $\cS_{\RR}-\{\ee_1\}$ is open for the D-topology:}
  Let us recall that a set is D-open if and only if its preimage by every plot is open \art{The-D-Topology-of-diffeological-spaces}.
  Let $\P:\U\to \cS_{\RR}$ be a plot,
  for every $r_0 \in \U$ there exist an open neighborhood $\V$ of $r_0$ in $\U$ and a local family $(\lambda_\alpha,\X_\alpha)_{\alpha \in \A}$,
  defined on $\V$,
  such that
  $$%
  \P \restriction \V : r \mapsto \sum_{\alpha \in \A} \lambda_\alpha(r) \X_\alpha.
  $$%
  Hence,
  $$%
  (\P \restriction \V)^{-1}(\cS_{\RR}-\{\ee_1\}) = \{ r \in \V \mid \sum_{\alpha \in \A} \lambda_\alpha(r) \X_{\alpha,1} \neq 1 \}.
  $$%
  But,
  the real parametrization $ \P_\V: r\mapsto \sum_{\alpha \in \A} \lambda_\alpha(r) \X_{\alpha,1}$ is smooth,
  {\em a fortiori\/} continuous.
  So, the preimage of $\RR - \{1\}$ is open.
  Thus,
  $\P^{-1}(\cS_{\RR}-\{\ee_1\})$ is a union of subdomains of $\U$,
  and therefore a domain.
  Hence,
  $\P^{-1}(\cS_{\RR}-\{\ee_1\})$ is open for any plot $\P$,
  that is, $\cS_{\RR}-\{\ee_1\}$ is D-open.
  
  In conclusion,
  the diffeology of the real infinite sphere $\cS_{\RR}$ is generated by $\F_+$ and $\F_-$.
  Therefore,
  $\cS_{\RR}$ is a diffeological manifold,
  modeled on $\cH_{\RR}$ \art{Generating-diffeological-manifolds}.
  Now,
  let us prove that $\cS_\RR$ is embedded in $\cH_\RR$ \art{Embedded-subsets-of-a-diffeological-space}.
  Let $\Omega$ be D-open in $\cS_\RR$.
  Let us consider the projection
  $$%
  \pr_\cS : \cH_\RR - \{0\} \to \cS_\RR, \text{ defined by } \pr_\cS(\X) = {\X \over \norm{\X}}.
  $$%
  Since $\norm{\X}$ never vanishes on $\cH_\RR - \{0\}$,
  the projection $\pr_\cS$ is smooth,
  thus D-continuous \art{Smooth-maps-are-D-continuous}.
  Hence,
  the set
  $$%
  \tilde \Omega = \pr_\cS^{-1}(\Omega) = \bigg\{ \X \in \cH_\RR - \{0\} \ \bigg\vert \ {\X \over \norm{\X}} \in \Omega \bigg\}
  $$%
  is D-open in  $\cH_\RR - \{0\}$.
  But $\cH_\RR - \{0\}$ is itself D-open in $\cH_\RR$,
  since $\cH_\RR - \{0\}$ is the pullback of the domain $]0,\infty[$ by the smooth map $\norm{\cdot}$.
  Hence,
  $\tilde \Omega$ is D-open in $\cH_\RR$.
  Now,
  $\Omega = \tilde \Omega \cap \cS_\RR$.
  Therefore,
  $\cS_\RR$ is embedded.
\end{proof}

\begin{article}\artlabel{The infinite sphere is contractible}
  \addcontentsline{toc}{section}{\small\hspace{10pt} The infinite sphere is contractible}
  \label{The-infinite-sphere-is-contractible}
  The infinite Hilbert sphere $\cS_\RR$,
  defined in \art{The-infinite-sphere},
  is smoothly {\em contractible}\index{Contractible},
  that is,
  there exists
  $$%
  \rho \in \Cinfty(\RR, \Cinfty(\cS_\RR)), \text{ such that } \rho(0) = \hat\ee_1, \text{ and } \rho(1) = \id_\cS,
  $$%
  where $\hat\ee_1$ is the constant map $\X \mapsto \ee_1$,
  with $\ee_1 = (1,0,0,\ldots)$,
  and  $\id_\cS$ is the identity of $\cS_\RR$.
  The {\em path} $\rho$,
  connecting the identity to the constant map,
  is called a {\em smooth retraction}\index{Smooth retraction}
  \art{Retractions-and-deformation-retracts}. Under $\rho$,
  the image $\rho(t)(\cS_\RR)$ retracts smoothly,
  when $t$ passes from 1 to 0,
  from the sphere $\cS_\RR$ to the point $\ee_1$.
  
  \Note~This strange property,
  for a sphere,
  has been proved for the topological structure of $\cS_\RR$ by S. Kakutani in 1943 \cite{Kak43},
  and we see that it still persists in diffeology,
  here for the fine diffeology of the sphere.
\end{article} %% The infinite sphere is contractible

\begin{proof}
  The proof of this proposition uses the following two preliminary constructions:
  
  1. The {\em shift operator\/}.
  It is defined as the linear map
  $$%
  {\shift} : \cH_{\RR} \to \cH_{\RR}, \text{ with } {\shift}(\X) = (0,\X) = (0,\X_1,\X_2,\ldots).
  $$%
  Since the shift operator is linear,
  it is smooth for the fine diffeology \art{Linear-maps-and-fine-diffeology}.
  It is injective and preserves the scalar product.
  It injects strictly the infinite sphere into an equator.
  
  2. {\em Connecting points}.
  Let $\X$ be some diffeological space,
  we say that a path $\gamma$ connects $x_0$ and $x_1$ if $\gamma(0) = x_0$ and $\gamma(1) = x_1$.
  We also say that $x_0$ and $x_1$ are {\em homotopic},
  and that $\gamma$ is a {\em homotopy} between $x_0$ and $x_1$ \art{Pathwise-connectedness}.
  Now,
  if there exist a smooth path $\gamma$ connecting $x_0$ to $x_1$ and a smooth path $\gamma'$ connecting $x_1$ to $x_2$,
  then the path $\gamma''$ defined by
  $$%
  \gamma'' =
  \left\{
  \begin{array}{lll}
    \gamma(\lambda(2t)) & \text{ if } & t \leq 1/2, \\
    \vspace{-5pt} \\
    \gamma'(\lambda(2t-1)) & \text{ if } & t \geq 1/2,
  \end{array}
  \right.
  $$%
  connects $x_0$ to $x_2$,
  where $\lambda$ is the smashing function defined in \art{Homotopy-of-paths},
  and described in Figure \ref{fig-smashing-function}.
  
  Now,
  we prove the contractibility of the infinite sphere in two steps,
  first we shall show that the constant map $\hat\ee_1$ is homotopic to the shift operator,
  and then,
  that the shift operator is homotopic to the identity $\id_{\cS}$.
  If we want,
  we can apply the smashing function to this pair of homotopies to get a smooth path connecting the constant map to the identity.
  
  (a) {\em Homotopy between $\hat\ee_1$ and ${\shift}$}.
  Let us consider the $1$-parameter family of deformations,
  $$%
  \rho_t(\X) = \cos\left({\pi t \over 2}\right) \ee_1 + \sin\left({\pi t \over 2}\right){\shift}(\X),
  \text{ for all } t \in \RR, \text{ and all } \X \in \cS_{\RR}.
  $$%
  For all $t \in \RR$, $\rho_t(\X) \in \cS_{\RR}$.
  Since addition and multiplication by a smooth function are smooth,
  the map $(t,\X)\mapsto \rho_t(\X)$ is smooth.
  Thus,
  the map $t\mapsto \rho_t$ is a path in $\cC^{\infty}(\cS_{\RR})$ connecting $\hat\ee_1$ and ${\shift}$,
  precisely,
  $$%
  \rho_0 = \hat\ee_1, \text{ and } \rho_1 = {\shift}.
  $$%
  
  (b) {\em Homotopy between ${\shift}$ and $\id_{\cS}$}.
  Let us consider the following $1$-parameter family of deformations,
  $$%
  \sigma_t(\X) = t\X + (1-t){\shift}(\X), \text{ for all } t \in \RR, \text{ and all } \X \in \cH_{\RR}.
  $$%
  Note that $\ker(\sigma_t) = {0}$.
  This is clear for $t = 0$,
  and for nonzero $t$ it follows inductively by observing that the condition $\sigma_t (\X) = 0$ writes
  $$%
  (\X_1 , \X_2 , \X_3 , \ldots) = {t - 1 \over t} (0, \X_1, \X_2 , \ldots).
  $$%
  In particular,
  $\sigma_t$ is nowhere zero on the sphere,
  we can define $\rho_t : \cS_{\RR} \to \cS_{\RR}$
  $$%
  \rho_t(\X) = {\sigma_t(\X) \over \norm{\sigma_t(\X)}}.
  $$%
  Let us check that $(t,\X)\mapsto \rho_t(\X)$ is smooth.
  First of all, $(t,\X)\mapsto \sigma_t(\X)$ is clearly smooth.
  Since the scalar product is smooth,
  it follows that $(t, \X) \mapsto \norm{\sigma_t (X)}^2$ is smooth,
  and because this map takes its values in $]0, \infty [$,
  its square root is smooth.
  In conclusion, $t\mapsto \rho_t$ is a path in $\cC^{\infty}(\cS_{\RR})$,
  and
  $$%
    \rho_0 = {\shift}, \text{ and } \rho_1 = \id_{\cS}.
  $$%
  We proved, with (a) and (b) above,
  that $\cS_{\RR}$ is contractible.
\end{proof}

\begin{article}\artlabel{The infinite complex projective space}
  \addcontentsline{toc}{section}{\small\hspace{10pt} The infinite complex projective space}
  \label{The-infinite-complex-projective-space}
  Let us recall some set-theoretic constructions,
  nowadays classical.
  Let us introduce
  $$%
  \CC^{\star} = \CC- \{0\}, \text{ and } \cH_\CC^{\star} = \cH_\CC - \{0\},
  $$%
  where the Hilbert space $\cH_\CC$ has been described in \art{The-fine-standard-Hilbert-space}.
  Then,
  let us consider the multiplicative action of the group $\CC^{\star}$ on $\cH_\CC^{\star}$,
  defined by
  $$%
  (z,\Z)\mapsto z\Z \in \cH_\CC^{\star}\,,
  \ \text{for all}\ (z,\Z) \in \CC^{\star} \times \cH_\CC^{\star}.
  $$%
  The quotient of $\cH_\CC^{\star}$ by this action of $\CC^{\star}$ is called the {\em infinite complex projective space},
  or simply the {\em infinite projective space}.
  We will denote it by
  $$%
  \cP_\CC = \cH_\CC^\star / \CC^\star.
  $$%
  Now,
  let us consider the space $\cH_\CC$ equipped with the fine diffeology \art{The-fine-diffeology-of-vector-spaces},
  the space $\cH_\CC^{\star}$ equipped with the subset diffeology \art{Subspaces-and-subset-diffeology},
  and the infinite projective space $\cP_\CC$ equipped with the quotient diffeology \art{Quotient-and-quotient-diffeology}.
  Let us denote by
  $\pi : \cH_\CC^{\star} \to \cP_\CC$ the canonical projection.
  Next,
  for every $k = 1, \ldots, \infty$,
  let us define the injection $j_k : \cH_\CC \to \cH_\CC^{\star}$ by
  $$%
  j_1(\Z) = (1, \Z), \text{ and } j_k(\Z) = (\Z_1,\ldots, \Z_{k-1}, 1, \Z_k,\ldots), \text{ for } k> 1.
  $$%
  Let the maps $\F_k$ be defined by
  $$%
  \F_k :\cH_\CC \to \cP_\CC, \text{ with } \F_k = \pi \circ j_k, \  k=1, \ldots, \infty.
  $$%
  %
  1. For every $k = 1, \ldots, \infty$,
  $j_k$ is an induction from $\cH_\CC$ into $\cH_\CC^{\star}$.
  
  2. For every $k = 1, \ldots, \infty$,
  $\F_k$ is a local diffeomorphism from $\cH_\CC$ to $\cP_\CC$.
  Moreover, their values cover $\cP_\CC$,
  $$%
  \bigcup_{k=1}^\infty \Val(\F_k) = \cP_\CC.
  $$%
  Thus,
  $\cP_\CC$ is a diffeological manifold modeled on $\cH_\CC$,
  for which the family $\set{\F_k}_{k=1}^\infty$ is an atlas \art{Generating-diffeological-manifolds}.
  
  3. The pullback $\pi^{-1}(\Val(\F_k)) \subset \cH_\CC^\star$ is isomorphic to the product $\cH_\CC \times \CC^\star$,
  where the action of $\CC^\star$ on $\cH_\CC^\star$ is transmuted into the trivial action on the factor $\cH_\CC$ and the multiplicative action on the factor $\CC^\star$.
  We say that the projection $\pi$ is a {\em locally trivial $\CC^\star$-principal fibration};
  see \art{Diffeological-fibrations} and \art{Principal-diffeological-fiber-bundles}.
\end{article} %% The-infinite-complex-projective-space

\begin{proof}
  1. Let us prove first that the maps $j_k$ are inductions from $\cH_\CC$ to $\cH_\CC^\star$.
  Let $\cH_k = j_k(\cH_\CC)$,
  that is,
  $$%
  \cH_k = \{ \Z \in \cH \mid \Z_k = 1 \}.
  $$%
  Now,
  let us consider a plot $\P$ of $\cH_\CC$ with values in $\cH_k$.
  By definition of the fine diffeology,
  we have
  $$%
  \P(r) =_{\rm loc} \sum_{\alpha \in \A} \lambda_\alpha(r) \Z_\alpha
  \text{ and }
  \P_k(r) =_{\rm loc} \sum_{\alpha \in \A} \lambda_\alpha(r)\Z_{\alpha,k} = 1,
  \text{ with }
  \P_k = \pr_k\circ \P.
  $$%
  The $\pr_k$ are the $k$-projections from $\cH_\CC$ to $\CC$,
  notations of \art{The-fine-standard-Hilbert-space}.
  Let us now define $\zeta_\alpha$ by
  $$%
  \zeta_\alpha = (\Z_{\alpha,_1}, \ldots, \Z_{\alpha,k-1},1, \Z_{\alpha,k+1},\ldots).
  $$%
  For each $\alpha$ in $\A$,
  $\zeta_\alpha$ belongs to $\cH_k$.
  Let $\ee_k$ be defined by $\pr_k(\ee_k) = 1$ and $\pr_j(\ee_k) = 0$ for $j\neq k$,
  notations of \art{The-fine-standard-Hilbert-space}.
  From the condition above,
  we have
  \begin{align*}
    {\P}(r) & =_{\rm loc} \sum_{\alpha \in \A}\lambda_\alpha(r)\zeta_\alpha -\sum_{\alpha \in  \A}\lambda_\alpha(r)\ee_k + \sum_{\alpha \in \A}\lambda_\alpha(r) \Z_{\alpha,k} \ee_k \\
            & =_{\rm loc} \sum_{\alpha \in \A}\lambda_\alpha(r)\zeta_\alpha + \bigg(1-\sum_{\alpha \in \A}\lambda_\alpha(r)\bigg)\ee_k.
  \end{align*}
  Now,
  since the vectors $\zeta_\alpha$ and $\ee_k$ belong to $\cH_k$, the plot $j_k^{-1}\circ \P$ writes
  $$%
  j_k^{-1}\circ \P(r) =_{\rm loc} \sum_{\alpha \in \A}\lambda_\alpha(r)j_k^{-1}(\zeta_\alpha) + \bigg(1-\sum_{\alpha \in \A}\lambda_\alpha(r)\bigg)j_k^{-1}(\ee_k),
  $$%
  but $j_k(0) = \ee_k$ implies $j^{-1}_k(\ee_k) = 0$,
  hence
  $$%
  j_k^{-1}\circ \P(r) =_{\rm loc} \sum_{\alpha \in \A}\lambda_\alpha(r)j^{-1}(\zeta_\alpha).
  $$%
  This exhibits the parametrization $j_k^{-1}\circ \P$ as a plot of $\cH$,
  hence $j_k$ is an induction.
  
  2. Let us prove now that the $\F_k$ are local diffeomorphisms.
  
  (a) {\em $\F_k$ is injective:}
  First of all,
  the maps $\F_k$ are clearly smooth.
  Now,
  since $\Z \in \cH_k$ implies $\Z_k = 1$,
  the space $\cH_k$ intersects the orbits of the group $\CC^{\star}$ in at most one point.
  The orbits which do not intersect $\cH_k$ are those such that $\Z_k = 0$.
  Hence, $\F_k$ is injective.
  
  (b) {\em The images of the $\F_k$ cover $\cP_\CC$:}
  The orbit of any point $\Z \in \cH_\CC^{\star}$ intersects some $\cH_k$,
  in other words,
  $$%
  \bigcup_{k = 1}^\infty \CC^{\star} \cH_k = \cH_\CC^{\star}, \text{ or } \pi\bigg(\bigcup_{k = 1}^\infty\cH_k \bigg) = \cP, \text{ or } \bigcup_{k=1}^\infty \Val(\F_k) = \cP_\CC.
  $$%
  
  (c) {\em The $\F_k$ are inductions:}
  Let $\Q:\U\to \cP_\CC$ be a plot with values in $\F_k(\cH_\CC)$,
  and let $r_0 \in \U$.
  By definition of the quotient diffeology,
  there exist an open neighborhood $\V$ of $r_0$ and a plot $\P:\V\to \cH_\CC^{\star}$ such that $\Q \restriction \V = \pi\circ \P$.
  By hypothesis,
  for each $r \in \V$, $\P_k(r) \neq 0$,
  where $\P_k = \pr_k\circ \P$,
  therefore $\P':\V\to \cH_\CC^{\star}$,
  defined by $\P'(r) = \P(r)/\P_k(r)$,
  takes its values in $\cH_k$.
  Since $\P_k$ is smooth,
  $\P'$ is smooth and $\Q \restriction \V = \pi\circ \P'$.
  The plot $\P'$ takes its values in $\cH_k$,
  and $j_k$ is an induction,
  thus the composite $j_k^{-1}\circ \P'$ is a plot of $\cH_\CC$.
  But,
  by construction,
  $j_k^{-1}\circ \P' = \F_k^{-1}\circ \Q$,
  thus $\F_k^{-1}\circ \Q$ is a plot of $\cH_\CC$ and $\F_k^{-1}$ is smooth.
  Therefore,
  $\F_k$ is an induction.
  
  (d) {\em The image of each $\F_k$ is D-open:}
  Since the D-topology of the quotient diffeology is the quotient topology of the D-topology \art{Quotients-and-D-topology},
  it is sufficient to prove that the pullback by $\pi$ of the $\F_k(\cH_\CC)$ is D-open in $\cH_\CC^{\star}$.
  We saw that $\pi^{-1}(\F_k(\cH_\CC))$ is the set of all $\Z \in \cH_\CC$ such that $\Z_k\neq 0$,
  {\em i.e.},
  $\pr_k^{-1}(\CC^{\star})$.
  But $\pr_k$ is linear,
  hence smooth,
  hence continuous.
  Since $\CC^\star$ is open it follows that $\pi^{-1}(\F_k(\cH_\CC))$ is open.
  
  Thus,
  by application of \art{Smooth-maps-are-D-continuous},
  we proved that the $\F_k$ are local diffeomorphisms.
  Since their images cover $\cP_\CC$,
  the space $\cP_\CC$ is a diffeological manifold modeled on $\cH_\CC$ \art{Generating-diffeological-manifolds}.
  
  3. Let us prove now that $\pi$ is a locally trivial $\CC^\star$-principal fibration.
  Let
  $$%
  \Phi_k : \cH_\CC \times \CC^{\star} \to \cH_\CC^{\star}, \text{ defined by } \Phi_k(\Z,z) = zj_k(\Z).
  $$%
  %
  (a) The $\Phi_k$ are local diffeomorphisms.
  Indeed,
  let $\Phi_k(\Z,z) = \Phi_k(\Z',z')$,
  that is,
  $zj_k(\Z) = z'j_k(\Z')$.
  Thus,
  $\pr_k(zj_k(\Z)) = \pr_k(z'j_k(\Z'))$,
  that is,
  $z = z'$.
  Hence,
  $\Z = \Z'$.
  Thus,
  the $\Phi_k$ are injective.
  They are obviously smooth.
  Their inverses are
  $$%
  \Phi_k^{-1}(\Z) = \left({\, \Z\phantom{{}_k} \over\, \Z_k}, \Z_k \right),
  \text{ for all } \Z \in \Val(\Phi_k) = \CC^\star \cH_k.
  $$%
  Since $\Z_k$ never vanishes,
  $\Phi_k^{-1}$ also is smooth.
  Moreover $\Val(\Phi_k)$ is D-open since it is the pullback $\pi^{-1}(\Val(\F_k))$,
  and we have seen that $\Val(\F_k)$ is D-open.
  Therefore,
  the $\Phi_k$ are local diffeomorphisms.
  
  (b) The $\Phi_k$ commute obviously with the action of $\CC^{\star}$ defined above.
  Therefore,
  $\set{\Phi_k}_{k=1}^\infty$ is a family of local diffeomorphisms from $\cH_\CC \times \CC^\star$ to $\cH_\CC^\star$ such that
  $$%
  \Phi_k(z'(\Z,z)) = z' \Phi_k(\Z,z), \text{ and } \bigcup_{k=1}^\infty \Val(\Phi_k) = \cH_\CC^\star,
  $$%
  where $z'(\Z,z) = (\Z,z'z)$ is just the action of $\CC^\star$ on the second factor.
  This atlas gives the structure of a locally trivial $\CC^\star$-principal bundle to the triple $(\cH_\CC^\star,\cP_\CC,\pi)$ for the given action of $\CC^\star$ \xart{Fibrations-and-local-triviality-along-the-plots}{Note 2}.
\end{proof}

%%%%%%%%%%%%%%%%%%%%%%%%%%%%%%%%%%%%%%%%%%%%%%%%%%%%%%%%%%
%
%   Exercises
%
%%%%%%%%%%%%%%%%%%%%%%%%%%%%%%%%%%%%%%%%%%%%%%%%%%%%%%%%%%

\Exercises

\begin{exercise}[The space of lines in $\Cinfty(\RR,\RR)$]
  \label{The-space-of-lines-in-Cinfty-R-R}
  Consider $\cE = \Cinfty(\RR,\RR)$,
  equipped with the functional diffeology.
  A line of $\cE$ is a subset $\eL \subset \cE$ for which there exist $f, g \in \cE$,
  with $g \neq 0$,
  such that $\eL = \{ f + s g \mid s \in \RR \}$.
  The space of lines,
  denoted by $\Lines(\cE)$,
  will be regarded as the quotient of $\cE \times (\cE - \{0\})$
  by the equivalence relation $(f,g) \sim (f + \lambda g, \mu g)$
  where $\lambda \in \RR$ and $\mu \in \RR - \{0\}$. For all $r \in \RR$,
  let
  $$%
  \cE_r^0 = \{ \alpha \in \Cinfty(\RR,\RR) \mid \alpha(r) = 0 \}, \text{ and } \cE_r^1 = \{ \beta \in \Cinfty(\RR,\RR)  \mid \beta(r) = 1 \}.
  $$%
  Let $\eF_r : \cE_r^0 \times \cE_r^1 \to \Lines(\cE)$ be defined by
  $$%
  \eF_r : (\alpha, \beta) \mapsto \{ \alpha + s \beta \mid s \in \RR \}.
  $$%
  
  \Question{1)} Describe the image of $\eF_r$ and show that $\eF_r$ is a local diffeomorphism.
  
  \Question{2)} Show that the images of the $\eF_r$,
  when $r \in \RR$,
  cover $\Lines(\cE)$.
  Deduce that $\Lines(\cE)$ is a manifold modeled on $\cE_0^0 \times \cE_0^0$.
  
  \Question{3)} Adapt this construction to $\Lines(\RR^2)$,
  where $\RR^2$ is regarded as $\Cinfty(\{1,2\}, \RR)$.
  Deduce that this gives an atlas for the M\"obius strip;
  \exref{A-diffeology-for-the-space-of-lines}, question 4.
\end{exercise} %% The-space-of-lines-in-Cinfty-R-R

\begin{exercise}[The Hopf $\S^1$-bundle]
  \label{The-Hopf-S1-bundle}
  Check that the quotient of the unit sphere $\cS_\CC = \set{ \Z \in \cH_\CC \mid \Z \cdot \Z = 1}$ by the multiplicative  action of the group $\U(1)$ of complex numbers with modulus 1,
  $$%
  \U(1) = \set{z \in \CC \mid z^*z = 1},
  $$%
  is naturally diffeomorphic to $\cP_\CC$.
  Transpose this identification to the product $\cH_\RR \times \cH_\RR \simeq \cH_\CC$;
  see \exref{HC-is-isomorphic-to-HR-X-HR}.
\end{exercise} %% The-Hopf-S1-bundle

\begin{exercise}[$\U(1)$ as subgroup of diffeomorphisms]
  \label{U1-as-subgroup-of-diffeomorphisms}
  Let $\GL(\cH_\CC)$ be the group of $\CC$-linear isomorphisms of $\cH_\CC$.
  Let $\U(\cH_\CC)$ be the unitary group of $\cH_\CC$,
  that is,
  the subgroup of the elements of $\GL(\cH_\CC)$ which preserves the Hermitian form,
  $$%
  \U(\cH_\CC) = \{ \A \in \GL(\cH_\CC) \mid \text{ for all } \Z, \Z' \in \cH_\CC, \ (\A\Z) \cdot (\A\Z') = \Z \cdot \Z' \}.
  $$%
  The group $\U(\cH_\CC)$ is equipped with the functional diffeology associated with the fine diffeology of $\cH_\CC$ \art{Functional-diffeology-between-fine-spaces}.
  Let $\U(1)$ be the group of complex numbers with modulus 1;
  see \exref{The-Hopf-S1-bundle}.
  Let $j : \U(1) \to \GL(\cH_\CC)$ be the map
  $$%
  j(z) : \Z \mapsto z\times \Z, \ \text{with} \ \Z \in \cH_\CC.
  $$%
  Show that $j$ is a monomorphic induction from $\U(1)$ to $\GL(\cH_\CC)$,
  and thus a mono\-morphic induction into $\U(\cH_\CC)$.
\end{exercise} %% 1-as-subgroup-of-diffeomorphisms

%%%%%%%%%%%%%%%%%%%%%%%%%%%%%%%%%%%%%%%%%%%%%%%%%%%%%%%%%%
%% MARK: Modeling Diffeologies
%%%%%%%%%%%%%%%%%%%%%%%%%%%%%%%%%%%%%%%%%%%%%%%%%%%%%%%%%%
\section*{Modeling Diffeologies}
\label{SecModeling-diffeologies}

\begin{sechead}
  We have seen,
  in this chapter, how diffeology offers a new approach for usual objects in classical differential geometry.
  Manifolds are not regarded anymore as a special class of structure,
  needing a lot of preparatory material,
  but as diffeological spaces which satisfy a given property:
  they are {\em modeled} on finite dimensional real vector spaces.
  The notion of local diffeomorphism and generating family in diffeology simplify their  description.
  Thanks to this point of view,
  we have been able to generalize the notion of classical manifolds to diffeological manifolds,
  simply by changing the {\em model},
  from finite dimensional real vector spaces to general diffeological vector spaces.
  We have also been able to begin the exploration of this way with two important examples:
  the Hilbert sphere \art{The-infinite-sphere} and the projective space of the Hilbert space \art{The-infinite-complex-projective-space}.
  
  Modeling diffeological spaces gives us a simple mechanism to understand and present a variety of geometrical objects,
  which we shall illustrate with two more examples.
  The first one,
  manifold with boundary,
  involves usually a not so simple definition \art{Classical-manifolds-with-boundary}.
  Thanks to the natural subset diffeology of {\em half-spaces} \art{Half-spaces},
  a manifold with boundary becomes simply a diffeological space modeled on a half-space \art{Diffeology-of-manifolds-with-boundary};
  and manifolds with boundary and corners follow by being diffeological spaces modeled on corners.
  
  The second example concerns {\em orbifolds}.
  They have become,
  since Satake's first original definition \cite{Sat56},
  \cite{Sat57},
  ordinary objects in mathematics.%
  \footnote{Originally, {\em orbifolds} were
  introduced by Ichiro Satake as {\em $V$-manifolds}.
  Later in a series of lectures on the subject,
  Thurston changed the name from $V$-manifolds to
  orbifolds.}
  But the concept remained problematic,
  and different authors give different
  definitions.
  Roughly speaking,
  an orbifold is a topological space which looks like,
  locally,
  a quotient $\RR^n/\Gamma$,
  where $\Gamma$ is a finite subgroup of linear transformations,
  for some kind of additional structure.
  Apart from the variety of definitions,
  what is strange is that orbifolds,
  or V-manifolds,
  come to us alone,
  without being included in a well defined category,
  as it is usual for mathematical objects.
  In one of his papers,
  Satake even quoted:
  ``{\em The notion of $\Cinfty$-map thus defined is inconvenient in the point that a composite of two $\Cinfty$-maps defined in a different choice of defining families is not always $\Cinfty$-map.}''
  
  The relative complexity and incompleteness of these different definitions,
  their conceptual obstacles as well as the endless technical discussions they generate,
  led us to test diffeology on the subject and introduce orbifolds in the category $\Diffeology$,
  following on the path opened by manifolds \art{Manifolds-as-diffeologies}:
  an orbifold is defined as a diffeological space which is locally diffeomorphic,
  at each point,
  to some quotient $\RR^n/\Gamma$,
  where $\Gamma$ is a finite subgroup of linear transformations.
  Then,
  orbifolds become just a subcategory of diffeological spaces.
  In this way,
  orbifolds take advantage of the whole theory of diffeology.
  In particular,
  smooth maps between orbifolds are well defined and compose correctly.
  Fiber bundles will be just a specialization of the notion of diffeological bundles,%
  \footnote{The notion of {\em orbibundle} does not seem to coincide with the notion of diffeological bundle \art{Diffeological-fibrations} over an orbifold.
  It looks like an intertwining of two orbifolds;
  see for example the definition in \cite{BG07}.
  The notion of orbibundle,
  if relevant,
  needs to be included properly later in the diffeological framework.}
  differential forms, covering, etc.
  We state in \art{Orbifolds-as-diffeologies} and \art{Structure-group-of-orbifolds} two main results which show that diffeological orbifolds completely fulfill the conditions of Satake's definition and make it a legitimate category in this regard.
  We shall not detail the proofs,
  the complete comparison between this definition and the original Satake's work was published in \cite{IKZ10}.
  It is worth adding that since then two directions on diffeological orbifols have been explored:
  their relation to $\CC^*$-algebra \cite{IZL18} and the complete description of their Klein stratification \cite{GIZ22}.
\end{sechead}

\begin{article}\artlabel{Half-spaces}
  \addcontentsline{toc}{section}{\small\hspace{10pt} Half-spaces}
  \label{Half-spaces}
  We denote by $\HH_n$ the {\em standard half-space}\index{Half-space} of $\RR^n$,
  that is,
  the set of points $x=(r,t) \in \RR^n$ such that $r \in \RR^{n-1}$ and $t \in [0,+\infty[$,
  and we denote by $\partial \HH_n$ its boundary $\RR^{n-1} \times \{0\}$.
  The subset diffeology of $\HH_n$,
  inherited from $\RR^n$,
  is made of all the smooth parametrizations $\P : \U \to \RR^n$ such that $\P_n(r) \geq 0$ for all $r \in \U$,
  $\P_n(r)$ being the $n$-th coordinate of $\P(r)$.
  The D-topology of $\HH_n$ is the usual topology defined by its inclusion into $\RR^n$.
\end{article} %% Half-spaces

\begin{article}\artlabel{Smooth real maps from half-spaces}
  \addcontentsline{toc}{section}{\small\hspace{10pt} Smooth real maps from half-spaces}
  \label{Smooth-real-maps-from-half-spaces}
  A map $f : \HH_n \to \RR^p$ is smooth for the subset diffeology of $\HH_n$ if and only if there exists an ordinary smooth map $\F$,
  defined on an open neighborhood of $\HH_n$,
  such that $f = \F \restriction \HH_n$.
  Actually,
  there exists such an $\F$ defined on the whole $\RR^n$.
  
  \Note~As an immediate corollary,
  any map $f$ defined on $\cC \times [0,\varepsilon[$ to $\RR^p$,
  where $\cC$ is an open cube of $\partial \HH_n$,
  centered at some point $(r,0)$,
  smooth for the subset diffeology,
  is the restriction of a smooth map $\F : \cC \times ]-\varepsilon, +\varepsilon[ \to \RR^p$.
\end{article} %% Smooth-real-maps-from-half-spaces

\begin{proof}
  First of all,
  if $f$ is the restriction of a smooth map $\F : \RR^n \to \RR^p$,
  it is obvious that for every smooth parametrization $\P : \U \to \HH_n$,
  $f\circ \P = \F \circ \P$ is smooth.
  Conversely,
  let $f_i$ be a coordinate of $f$.
  Let $x =(r,t)$ denote a point of $\RR^n$,
  where $r \in \RR^{n-1}$ and $t \in \RR$.
  If $f_i$ is smooth for the subset diffeology,
  then $\phi_i : (r,t) \mapsto f_i(r,t^2)$,
  defined on $\RR^n$,
  is smooth.
  Now, $\phi_i$ is even in the variable $t$, $\phi_i(r,t) = \phi_i(r,-t)$.
  Thus,
  according to Hassler Whitney \cite[Theorem 1 and final remark]{Whi43} there exists a smooth map $\F_i : \RR^n  \to \RR$ such that $\phi_i(r,t) = \F_i(r,t^2)$.
  Hence,
  $f_i(r,t) = \F_i(r,t)$ for all $r \in \RR^{n-1}$ and all $t \in [0,+\infty[$.
\end{proof}

\begin{article}\artlabel{Local diffeomorphisms of half-spaces}
  \addcontentsline{toc}{section}{\small\hspace{10pt} Local diffeomorphisms of half-spaces}
  \label{Local-diffeomorphisms-of-half-spaces}
  A map $f : \A \to \HH_n$,
  with $\A \subset \HH_n$,
  is a local diffeomorphism for the subset diffeology of $\RR^n$ if and only if $\A$ is open in $\HH_n$,
  $f$ is injective,
  $f(\A \cap \partial \HH_n) \subset \partial \HH_n$,
  and for all $x \in \A$ there exist an open ball $\Ball \subset \RR^n$ centered at $x$ and a local diffeomorphism $\F : \Ball \to \RR^n$ such that $f$ and $\F$ coincide on $\Ball \cap \HH_n$.
  
  \Note~This implies,
  in particular,
  that there exist an open neighborhood $\cU$ of $\A$ and an \'etale application $g : \cU \to \RR^n$ such that $f$ and $g$ coincide on $\A$.
\end{article} %% Local-diffeomorphisms-of-half-spaces

\begin{proof}
  Let us assume that $f$ is a local diffeomorphism for the subset diffeology.
  Since $f$ is a local diffeomorphism for the D-topology,
  $f$ is a local homeomorphism,
  and $\A$ is open in $\HH_n$.
  In particular,
  $f$ maps the boundary $\partial \A = \A \cap \partial \HH_n$ into $\partial \HH_n$.
  As well,
  $f$ maps the complementary $ \A - \partial \A$ into $\HH_n - \partial \HH_n$.
  Now,
  since $\A$ is open in $\HH_n$,
  $\A - \partial \A$ is open in $\HH_n - \partial \HH_n$,
  thus the restriction $f \restriction \A - \partial \A$ is a local diffeomorphism to $\HH_n - \partial \HH_n$.
  Therefore,
  for every $x \in \A - \partial \A$ there exists an open ball $\Ball \subset \A - \partial \A$,
  centered at $x$,
  such that $ \F = f \restriction \Ball$ is a local diffeomorphism.
  
  Now,
  let $(r,0) \in \partial \A$.
  Since $\A$ is open in $\HH_n$,
  $\partial \A$ is open in $\partial \HH_n$.
  Therefore,
  there exist an open cube $\cC \subset \partial \A$ centered at $(r,0)$,
  and $\varepsilon >0$,
  such that $\cC \times [0,+\varepsilon[ \subset \A$.
  The restriction of $f$ to $\cC \times [0,+\varepsilon[$ is a local diffeomorphism,
  for the subset diffeology,
  to $\HH_n$.
  Thanks to \art{Smooth-real-maps-from-half-spaces},
  there exists a smooth map $\F$ defined on $\cC \times ]-\varepsilon,+\varepsilon[$ to $\RR^n$,
  such that $f$ and $\F$ coincide on $\cC \times [0,+\varepsilon[$.
  Since $f$ is a diffeomorphism,
  $f$ maps $\cC \times [0,+\varepsilon[$ to some open set $\A' \subset \HH_n$ and maps $\cC$ to some open subset of $\partial \HH_n$.
  We have then $(r',0) = f(r,0) \in   \partial \A' = \A' \cap \partial \HH_n$.
  Considering now $f^{-1}$,
  for the same reason,
  there exists an open cube $\cC' \subset  \partial \A'$ centered at $(r',0)$,
  there exist $\varepsilon'>0$ such that $\cC' \times [0,+\varepsilon'[ \subset \A'$,
  and a smooth map $\G$ defined on $\cC' \times ]-\varepsilon',+\varepsilon'[$ to $\RR^n$ such that $f^{-1}$ and $\G$ coincide on $\cC' \times [0,+\varepsilon'[$.
  Now,
  let $\cO = \F^{-1}(\cC' \times ]-\varepsilon',+\varepsilon'[)$,
  and $\cO' = \G^{-1}(\cC \times ]-\varepsilon,+\varepsilon[)$,  $\cO$ and $\cO'$ are open subsets of $\RR^n$,
  with $(r,0) \in \cO$ and $(r',0) \in \cO'$.
  For every $t \geq 0$ such that $(r,t) \in \cO$,
  we have $\D(\G \circ \F)(r,t) = \D(f^{-1} \circ f)(r,t) = \id_{n+1}$.
  Thus,
  since $\F$ and $\G$ are smooth parametrizations,
  we have on the one hand $\lim_{t \mathop{\rightarrow} 0^+} \D(\G \circ \F)(r,t) = \id_{n+1}$,
  and on the other hand $\lim_{t \mathop{\rightarrow} 0^+} \D(\G \circ \F)(r,t) = \D(\G)(r',0) \circ \D(\F)(r,0)$.
  So,
  $\D(\G)(r',0) \circ \D(\F)(r,0) = \id_{n+1}$,
  and thus $\D(\F)(r,0)$ is nondegenerate.
  Therefore,
  thanks to the implicit function theorem,
  there exists an open ball $\Ball$ centered at $x =(r,0)$ such that $\F \restriction \Ball$ is a local diffeomorphism to $\RR^n$,
  and such that $f$ and $\F$ coincide on $\Ball \cap \HH_n$.
  
  Conversely,
  let us assume that $\A$ is open in $\HH_n$, $f : \A \to \HH_n$ is injective,
  and for each $x \in \A$ there exist an open ball $\Ball$ of $\RR^n$ centered at $x$,
  and a local diffeomorphism $\F: \Ball \to \RR^n$ such that $f$ and $\F$ coincide on $\Ball$.
  Let us prove that $f$ is a local smooth map for the subset diffeology.
  Let $\P : \U \to \RR^n$ be a smooth parametrization taking its values in $\HH_n$.
  Since $\A$ is open in  $\HH_n$ and $\P$ is continuous,
  for the D-topology,
  $\P^{-1}(\A)$ is open.
  Now let $r \in \P^{-1}(\A)$ and $x = \P(r)$.
  Since $\P$ is continuous,
  $\P^{-1}(\Ball)$ is open,
  and since $\F$ is a local diffeomorphism from $\Ball$ to $\RR^n$,
  $f \circ \P \restriction \P^{-1}(\Ball) =  \F \circ \P \restriction \P^{-1}(\Ball)$ is a smooth parametrization.
  Thus $f \circ \P$ is locally smooth at every point,
  thus $\P$ is smooth,
  and therefore $f$ is smooth.
  For the smoothness of $f^{-1}$,
  we need only check that $f(\A)$ is open,
  and the rest will follow the same way as for $f$.
  So,
  let $x \in \A$.
  If $x \in \A - \partial \A$,
  the ball $\Ball$ can be chosen small enough to fit into $\A - \partial \A$.
  Now,
  by hypothesis $\partial \A$ is mapped into $\partial \HH_n$ and $f$ is injective,
  thus $f$ maps the complementary $\A- \partial \A$ into $\HH_n - \partial \HH_n$.
  Then,
  since $f \restriction \Ball$ is a local diffeomorphism to $\RR^n$,
  $f(\Ball) \subset f(\A)$ is an open subset of $\HH_n - \partial \HH_n$. Next,
  let $x \in  \partial \A$,
  then $f(x) \in \partial \HH_n$ and $f(x) \in f(\Ball \cap \HH_n) = \F(\Ball) \cap \HH_n \subset f(\A)$.
  Since $\F$ is a local diffeomorphism from $\Ball$ to $\RR^n$,
  $\F(\Ball)$ is open in $\RR^n$, and $\F(\Ball) \cap \HH_n$ is an open subset of $\HH_n$.
  Finally,
  for every $x \in \A$, $f(x)$ is contained in some open subset $\cO$ of $\HH_n$ with $\cO \subset f(\A)$.
  Therefore $f(\A)$ is a union of open subsets of $\HH_n$,
  that is,
  an open subset of $\HH_n$.
\end{proof}

\begin{article}\artlabel{Classical manifolds with boundary}
  \addcontentsline{toc}{section}{\small\hspace{10pt} Classical manifolds with boundary}
  \label{Classical-manifolds-with-boundary}
  Smooth manifolds with boundary have been precisely defined in \cite{GuPo74}.
  We use here a more recent definition except that,
  for our subject, the charts have been inverted.
  
  \Definition~\cite{Lee06}.~A {\em smooth $n$-manifold with boundary\/}\index{Manifold with boundary} is a topological space $\M$,
  together with a family of local homeomorphisms $\F_i$ defined on some open sets $\U_i$ of the half-space $\HH_n$ to $\M$,
  such that the values of the $\F_i$ cover $\M$ and,
  for any two elements $\F_i$ and $\F_j$ of the family,
  the transition homeomorphism $\F_i^{-1} \circ \F_j$,
  defined on $\F_i^{-1}(\F_i(\U_i) \cap \F_j(\U_j))$ to $\F_j^{-1}(\F_i(\U_i) \cap \F_j(\U_j))$,
  is the restriction of some smooth map defined on an open neighborhood of $\F_i^{-1}(\F_i(\U_i) \cap \F_j(\U_j))$.
  The boundary $\partial \M$ is the union of the $\F_i(\U_i \cap \partial \HH_n)$.
  Such a family $\cF$ of homeomorphisms is called an atlas of $\M$,
  and its elements are called {\em charts\/}\index{Chart}.
  There exists a {\em maximal atlas}\index{Atlas} $\cA$ containing $\cF$,
  made of all the local homeomorphisms from $\HH_n$ to $\M$ such that the transition homeomorphisms with every element of $\cF$ satisfy the condition given just above.
  We say that $\cA$ gives to $\M$ its {\em structure of manifold with boundary}.
\end{article} %% Classical-manifolds-with-boundary

\begin{article}\artlabel{Diffeology of manifolds with boundary}
  \addcontentsline{toc}{section}{\small\hspace{10pt} Diffeology of manifolds with boundary}
  \label{Diffeology-of-manifolds-with-boundary}
  Let $\M$ be a smooth $n$-manifold with boundary.
  Let us recall that a parametrization $\P : \U \to \M$ is smooth if for every $r \in \U$ there exist an open neighborhood $\V$ of $r$,
  a chart $\F :  \Omega \to \M$ of $\M$,
  and a smooth parametrization,
  $\Q : \V \to \Omega$ such that $\P \restriction \V = \F \circ \Q$.
  The set of smooth parametrizations of $\M$ is then a natural diffeology.
  Regarded as a diffeological space,
  $\M$ is modeled on $\HH_n$ \art{Half-spaces}.
  Conversely, every diffeological space modeled on $\HH_n$ is naturally a manifold with boundary.
  Moreover,
  the smooth maps form a manifold with boundary to another,
  for the category \{Smooth Manifolds with Boundary\} or for the category $\Diffeology$ coincide.
  
  \Note~Nothing now prevents us from defining directly the {\em manifolds with corners} as the diffeological spaces
  modeled on corners\index{Corner} $\KK_n = \{ x = (x_1,\ldots,x_n) \in \RR^n \mid  x_i \geq 0, i = 1, \ldots, n\}$.
  Indeed, the following proposition,
  shows already that $\Cinfty(\KK_n,\RR^m)$ is made of the restrictions to $\KK_n$ of smooth maps defined on an open superset.
  
  \Proposition A map $f : \KK_n \to \RR$ is smooth for the subset diffeology of $\KK_n$ if and only if there exist an open superset $\W$ of $\KK_n$ and a smooth map $\F : \W \to \RR$,
  such that $\F \restriction \KK_n = f$.
  
  This result,
  extended to local diffeomorphisms of $\KK_n$ as was done for half-spaces \art{Local-diffeomorphisms-of-half-spaces},
  leads to a perfect correspondence between diffeological {\em manifolds with corners}\index{Manifold with coners} and the usual notion found in the literature;
  see \cite{GIZ19}.
\end{article} %% Diffeology-of-manifolds-with-boundary

\begin{proof}
  We denote by $\cD$ the set of smooth parametrizations of $\M$.
  Now,
  let us prove that any chart $\F : \U \to \M$ is a local diffeomorphism from $\HH_n$ to $\M$,
  where $\HH_n$ is equipped with the subset diffeology,
  and $\M$ is equipped with $\cD$.
  Since for any plot $\P$ of $\HH_n$,
  $\F \circ \P$
  ---~defined on $\P^{-1}(\U)$ which is open~---
  belongs obviously to $\cD$,
  so $\F$ is smooth.
  Then,
  let us prove now that $\F^{-1}$ is smooth.
  Let $\P : \U \to \M$ be an element of $\cD$,
  let $r \in \U$,
  let $\V$, $\Q$ and $\F'$ be as above,
  such that $\P \restriction \V = \F' \circ \Q$.
  Thus,
  $\F^{-1} \circ \P \restriction \V = \F^{-1}\circ \F' \circ \Q$.
  But since $\F$ and $\F'$ are charts of $\M$,
  $\F^{-1} \circ \F'$ is the restriction of some ordinary smooth map defined on some open neighborhood of $\F^{-1}(\F(\U) \cap \F'(\U'))$,
  therefore $\F^{-1}\circ \F' \circ \Q$ is a smooth parametrization in $\HH_n$,
  that is,
  $\F^{-1} \circ \P \restriction \V$.
  Now,
  since $\P^{-1}(\F(\U))$ is open,
  and since the parametrization $\P \circ \F^{-1}$ is locally smooth everywhere on $\P^{-1}(\F(\U))$,
  $\F^{-1} \circ \P$ is smooth.
  Thus,
  $\F^{-1}$ is a local smooth map.
  Therefore,
  $\F$ is a local diffeomorphism.
  This proves that,
  for the diffeology $\cD$,
  $\M$ is modeled on $\HH_n$.
  Conversely,
  let us assume that $\M$ is a diffeological space modeled on $\HH_n$.
  Let $\cA$ be the set of all local diffeomorphisms from $\HH_n$ to $\M$,
  and let us equip $\M$ with the D-topology.
  So,
  the elements of $\cA$ are local homeomorphisms.
  Let $\F : \U \to \M$ and $\F' : \U' \to \M$ be two elements of $\cA$.
  Let us assume that $\F(\U) \cap \F'(\U')$, which is open,
  is not empty.
  Thus $\F^{-1}(\F(\U) \cap \F'(\U'))$ and $\F'^{-1}(\F(\U) \cap \F'(\U'))$ are open.
  But $\F'^{-1} \circ \F \restriction \F^{-1}(\F(\U) \cap \F'(\U'))$ and $\F^{-1} \circ \F' \restriction \F'^{-1}(\F(\U) \cap \F'(\U'))$ are local diffeomorphisms for the subset diffeology of $\HH_n$,
  and according to \art{Local-diffeomorphisms-of-half-spaces} they are the restrictions of ordinary smooth maps.
  Therefore,
  the set $\cA$ gives to $\M$ a structure of smooth manifold with boundary.
  It is clear that these two operations just described are inverse one of each other.
  %
  Now,
  thanks to \art{Local-diffeomorphisms-of-half-spaces} and \art{Smooth-real-maps-from-half-spaces},
  it is clear that to be a smooth map for the category of smooth manifolds with boundary or to be smooth for the natural diffeology associated is identical.
  
  Let us give now a proof of the proposition in the note.
  First of all,
  if $f : \KK_n \to \RR$ is the restriction of a smooth map $\F : \RR^n \to \RR$,
  then it is clear that $f$ is smooth.
  Conversely,
  let $f : \KK_n \to \RR$ be a smooth map for the subset diffeology.
  Then,
  the map
  $\phi : (x_1, x_2, \ldots,x_n) \mapsto f(x_1^2, x_2^2, \ldots, x_n^2)$,
  defined on $\RR^n$,
  is smooth and is even in every variable.
  Consider the first variable,
  thanks to Hassler Whitney \cite[Theorem 1 and final remark]{Whi43},
  there exists a smooth map $\F_1(x_1,x_2,\ldots,x_n)$,
  defined on $\RR^n$,
  such that $\phi(x_1,x_2,\ldots,x_n) = \F_1(x_1^2,x_2,\ldots, x_n)$,
  that is,
  $f(x_1^2,x_2^2,\ldots,x_n^2) = \F_1(x_1^2,x_2,\ldots, x_n)$.
  Next,
  since $\F_1(x_1^2,x_2,\ldots, x_n) = \F_1(x_1^2,-x_2,\ldots, x_n)$,
  for the same reason,
  there exists a smooth map $\F_2$,
  defined on $\RR^n$,
  such that $\F_1(x_1^2,x_2,\ldots, x_n) = \F_2(x_1^2,x_2^2,\ldots,x_n)$,
  that is,
  $f(x_1^2,x_2^2,\ldots,x_n^2) = \F_2(x_1^2,x_2^2,\ldots,x_n)$.
  By recursion,
  there exists a smooth map $\F = \F_n$,
  also defined on $\RR^n$,
  such that $f(x_1^2,x_2^2,\ldots, x_n^2) = \F_n(x_1^2,x_2^2,\ldots,x_n^2)$.
  Therefore:
  $f(x_1,x_2,\ldots, x_n) = \F(x_1,x_2,\ldots,x_n)$,
  for all $x_i\geq 0$,
  $i=1,\ldots n$,
  where $\F$ is a smooth map from $\RR^n$ to $\RR$.
  %  Now, $\phi_i$ is invariant by the action of\n
  %  the discrete group $\{-1, +1\}^n$, acting on $\RR^n$\n
  %  by $(\varepsilon_1, \ldots, \varepsilon_n) \cdot (x_1, \ldots, x_n)\n
  %  = (\varepsilon_1 x_1, \ldots, \varepsilon_n x_n)$,\n
  %  by a result of Gerald Schwarz \cite{Sch75}, there exists a smooth\n
  %  function $\F : \RR^n \to \RR$ such that\n
  %  $\phi(x_1, \ldots, x_n) = \F(x_1^2, \ldots, x_n^2)$.\n
  %  Thus, $f = \F \restriction \KK_n$ is the restriction of a smooth function\n
  %  defined on an open superset of $\KK_n$. Thus smoothness coincides in the two meanings.\n
\end{proof}

\begin{article}\artlabel{Orbifolds as diffeologies}
  \addcontentsline{toc}{section}{\small\hspace{10pt} Orbifolds as diffeologies}
  \label{Orbifolds-as-diffeologies}
  A diffeological space $\M$ is said to be an  {\em orbifold of dimension $n$}\index{Orbifold},
  or a $n$-orbifold,
  if $\M$ is locally diffeomorphic at every point to $\RR^n/\Gamma$,
  where $\Gamma\subset \GL(\RR^n)$ is finite,
  and $\RR^n/\Gamma$ is equipped with the quotient diffeology.
  The group $\Gamma$ may change from one point to another.
  Note that $n$ is the dimension,
  in the diffeological sense \art{The-dimension-map},
  of the orbifold at each point.
  
  We could say this in other words:
  an orbifold is a diffeological space {\em modeled} by the family
  $$%
  \cM = \{\RR^n /\Gamma \mid \Gamma \subset \GL(n,\RR)
  \text{ and } \# \Gamma < \infty \}.
  $$%
  The elements of $\cM$ can be called the {\em models} of the category $\Orbifolds$,
  see Figure~\ref{orbifold-chart}.
  
  \Note~There is an approach on orbifolds through groupoids,
  developed the last decades around Ieke Moerdijk;
  see \cite{Mor02}.
  Orbifolds are not defined anymore by their geometry but by a superstructure defined as a groupoid which may contain some non-effective action of the structures groups $\Gamma$.
  There exists,
  however,
  a correspondance from these orbifolds groupoids to the diffeological orbifolds.
  The relationship between the two approaches was recently discussed by Yael Karshon
  %\footnote{Started with Masrour Zoghi in 2012.}
  and David Miyamoto \cite{KaMi22}.
\end{article} %% Orbifolds-as-diffeologies

\begin{article}\artlabel{Structure group of orbifolds}
  \addcontentsline{toc}{section}{\small\hspace{10pt} Structure groups of orbifolds}
  \label{Structure-group-of-orbifolds}
  Let $\M$ be an orbifold of dimension $n$ and $x \in \M$.
  There exist always a finite linear group $\Gamma \subset \GL(\RR^n)$,
  unique up to conjugation,
  and a local diffeomorphism $\psi : \RR^n/\Gamma \supset \cO \to \M$ mapping $0$ to $x$,
  that is,
  $0 \in \cO$ and $\psi(0) = x$.
  The conjugacy class of the group $\Gamma$ is called the {\em structure group}\index{Structure group} of $\M$ at the point $x$.
  The point $x$ is said to be {\em regular} if $\Gamma$ is trivial,
  {\em singular} otherwise.%
  \footnote{This corresponds to the diffeological notion of singularity,
  since a local diffeomorphism can only exchange points with the same structure group.}
  %
  Note that an orbifold with no singular point is just a manifold.
  For a comprehensive presentation,
  see \cite{IKZ10}.
  
  \Note{1}~Recently the definition of structure groups of an orbifold has been extended to a {\em structure groupoid}\index{Structure groupoid}.
  It is defined as the groupoid of the germs of local diffeomorphisms of the nebula of an atlas of the orbifold (actually the strict generating family associated),
  which project on the identity along the evaluation map;
  see \cite{IZL18}.
  The structure groups are the isotropies of the structure groupoid.
  
  \Note{2}~The category of orbifolds had been extended by the concept of {\em quasifold}\index{Quasifold},
  isolated by Elisa Prato in \cite{Pra01} and recently included in a precise way as a subcategory of $\Diffeology$ in \cite{IZP21}.
  A quasifold is built on the model of orbifolds where the finitude of the $\Gamma$ is relaxed to allow countable subgroups of the affine group.
  The irrational torus \cite{DI83} is the model {\em par excellence} of quasifold.
\end{article} %% Structure-group-of-orbifolds

%%###########
\begin{figure}[tb]
  \centerline{\includegraphics{Figures-PDF/fig-orbifold-chart}}
  \caption{The three levels of an orbifold structure.}
  \label{orbifold-chart}
\end{figure}
%%###########

\begin{article}\artlabel{In conclusion on modeling}
  \addcontentsline{toc}{section}{\small\hspace{10pt} In conclusion on modeling}
  \label{In-conclusion-on-modeling}
  The examples,
  described in this section,
  suggest a precise concept of {\em modeling} of diffeological spaces,
  close to the notion of generating family introduced in \art{Generating-diffeology},
  but stronger.
  We can say formally that a diffeological space $\X$ is {\em modeled} on a family  of diffeological spaces $\cM$,
  if $\X$ is locally diffeomorphic to some member of $\cM$,
  everywhere.
  That is,
  for any $x \in \X$ there exists $\A \in \cM$ and  a local diffeomorphis $\phi$ from $\A$ to $\X$,
  such that $x \in \Val(\phi)$.
  Precisely,
  for every plot $\P : \U \to \X$ and for every $r \in \U$,
  there exist an open neighborhood $\V \subset \U$ of $r$,
  a plot $\Q : \V \to \A$ and a local diffeomorphism $\phi$ from $\A$ to $\X$,
  such that $\phi \circ \Q = \P \restriction \V$.
  In this case,
  $x = \P(r)$.
  This definition covers the various examples we already looked at:
  the $n$-dimensional manifolds for which the family is reduced to $\cM = \{\RR^n\}$,
  the diffeological manifolds modeled on some diffeological vector space $\E$,
  or a family $\cM = \{\E_i\}_{i \in \cI}$ of diffeological vector spaces;
  the $n$-manifolds with boundary and corners,
  modeled on $\KK_n$,
  or the orbifolds/quasifolds where $\cM$ is the family of
  quotients $\{\R^n/\Gamma\}_{n \in \NN}$,
  where the $\Gamma$ are the finite,
  or countable,
  subgroups of $\Aff(\RR^n)$.
\end{article} %% In-conclusion-on-modeling

%%%%%%%%%%%%%%%%%%%%%%%%%%%%%%%%%%%%%%%%%%%%%%%%%%%%%%%%%%
%
%   Exercises
%
%%%%%%%%%%%%%%%%%%%%%%%%%%%%%%%%%%%%%%%%%%%%%%%%%%%%%%%%%%

\Exercises

\begin{exercise}[Reflexive diffeologies]
  \label{Reflexive-diffeologies}
  Let $\X$ be any set and $\cF$ be any subset of $\Maps(\X,\RR)$.
  
  \Question{1)} Describe the coarsest diffeology $\cD$ on $\X$ such that $ \cF \subset \cD(\X,\RR)$.
  What about the finest one?
  
  We say that the diffeology $\cD$ is {\em subordinated} to $\cF$.
  We shall say that a diffeology $\cD$ of $\X$ is {\em reflexive}\item{Diffeology!reflexive diffeology} if the diffeology $\cD'$ subordinated to $\cF = \cD(\X,\RR)$ coincides with $\cD$.
  
  \Question{2)} Why is the diffeology subordinated to any $\cF \subset \Maps(\X,\R)$ reflexive?
  
  \Question{3)} Show that the finite dimensional manifolds \art{Manifolds-as-diffeologies} are reflexive.
  
  \Question{4)} Show that the irrational tori (\exref{Diffeomorphisms-between-irrational-tori}) are not reflexive.
  
  Reflexive spaces form a full subcategory of the category $\Diffeology$,
  extending the ordinary full subcategory $\Manifolds$.
\end{exercise} %% Reflexive-diffeologies

\begin{exercise}[Fr\"olicher spaces]
  \label{Frolicher-spaces}
  Let $\X$ be a set,
  a Fr\"olicher structure on $\X$ is defined by a pair of sets:
  $\cF \subset \Maps(\X,\RR)$ and $\cC \subset \Maps(\RR,\X)$ such that $\cC = \fC(\cF)$ and $\cF = \fF(\cC)$,
  where
  \begin{align*}
    \fC(\cF) & = \{ c \in \Maps(\RR,\X) \mid \cF \circ c \subset \Cinfty(\RR,\RR) \}, \\
    \fF(\cC) & = \{ f \in \Maps(\X,\RR) \mid f \circ \cC \subset \Cinfty(\RR,\RR) \}.
  \end{align*}
  Said differently,
  $\cC = \fC(\cF)$ and $\cF = \fF(\cC)$ are what we call the {\em Fr\"olicher condition\/}.
  Thanks to the second question of \exref{Reflexive-diffeologies},
  we know that the diffeology subordinated to $\cF$ is reflexive.
  We shall admit this version of Boman's theorem.
  
  \Theorem\cite{Bom67}.~{\em If $\F \in \Maps(\RR^n,\RR)$ is such that,
  for all $\gamma \in \Cinfty(\RR, \RR^n)$,
  $\F \circ \gamma \in \Cinfty(\RR,\RR)$,
  then $\F \in \Cinfty(\RR^n,\RR)$.}
  
  Show that, for every reflexive diffeological space $\X$,
  $\cC = \Cinfty(\RR,\X)$ and $\cF = \Cinfty(\X,\RR)$ satisfy the Fr\"olicher condition.
  
  \Note~Actually it follows from these statements that there exists an equivalence between the category of Fr\"olicher spaces\index{Fr\"olicher space} and the strict subcategory of reflexive diffeological spaces.%
  \footnote{These results have been established collectively in a seminar in Toronto, during the fall of 2010,
  together with Augustin Batubenge, Yael Karshon and Jordan Watts.}
  Note that,
  as reflexive diffeological spaces,
  all the diffeological constructions developed in this textbook (homotopy, differential calculus, fiber bundles, etc.) apply in particular to Fr\"olicher spaces.
  We cannot expect however the same results for Fr\"olicher spaces as for general diffeological spaces.
  One of many examples:
  the diffeological construction of the integration bundle of a closed $2$-form \art{Integration-bundles-of-closed-2-forms} does not exist in this subcategory,
  except for the very special case of an integral form.
  Indeed,
  the group of periods of a closed $2$-form is,
  in general,
  dense in $\RR$ and the torus of periods,
  which is a fundamental ingredient of the integration bundle,
  is then trivial as a Fr\"olisher space.
\end{exercise}  %% Frolicher-spaces

