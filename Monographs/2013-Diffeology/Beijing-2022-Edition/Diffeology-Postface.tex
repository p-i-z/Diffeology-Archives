%%%%%%%%%%%%%%%%%%%%%%%%%%%%%%%%%%%%%%%%%%%%%%%%%%%%%%%%%%
%% 
%% MARK: Souvenirs
%% 
%%%%%%%%%%%%%%%%%%%%%%%%%%%%%%%%%%%%%%%%%%%%%%%%%%%%%%%%%%
  
\chapter*{Afterword}

\begin{chaphead}
  I was once a student of Jean-Marie Souriau,
  working on my doctoral dissertation,%
  \footnote{I am talking about the state doctorate.
  This was before the standardization on the American PhD model.}
  %
  when he introduced \guillemots{diffeologies}.
  I remember well,
  we used to meet for a seminar at that time
  ---~the early 1980s~---
  every Tuesday,
  at the Center for Theoretical Physics,
  on the Luminy campus in Marseille.
  Jean-Marie was trying to generalize his quantization procedure to a certain type of coadjoint orbits of infinite dimensional diffeomorphism groups.
  He wanted to regard these groups of diffeomorphisms as Lie groups,
  like everybody else,
  but he also wanted to avoid topological finesses,
  considering that they were not essential for this purpose.
  He then invented a lighter \guillemots{differentiable} structure that he called \guillemots{diffeologies} on groups of diffeomorphisms.
  These groups quickly became autonomous objects.
  Finally,
  he abandoned groups of diffeomorphisms for abstract groups with an abstract differential structure.
  He called them \guillemots{{\em groupes diff\'erentiels}},
  this was the first name for the future diffeological groups.
  
  {\bf Differential spaces} was born.
  Listening to Jean-Marie talking about his differential groups,
  It was clear that the axiomatics of
  differential groups,
  should be extended to any set,
  not only to groups,
  and I remember a particularly hot discussion on this question in the cafeteria of the Luminy campus.
  It was during a break in our seminar.
  We were there:
  JMS (as we called him),
  Jimmy Elhadad,
  Christian Duval,
  Paul Donato,
  Henry-Hugues Fliche,
  Roland Triay,
  and myself.
  Souriau insisted on not considering anything other than differential groups or their orbits.
  (Souriau was really obsessed by groups),
  and I was thinking that it was time to generalize diffeologies to any sets.
  But at that moment,
  I was working on the classification of $\SO(3)$-symplectic manifolds which had nothing to do with diffeology,
  and I had no time for anything else.
  A few weeks later,
  he presented the general theory of \guillemots{{\em espaces diff\'erentiels}} as he called it.
  I must say that at the time,
  this theory seemed to us,
  his students,
  as a curiosity,
  but so general that we saw no real stake in it,
  except a certain intellectual satisfaction...
  We were doubtful.
  I decided to forget differential spaces and stay focused on \guillemots{real maths},
  sympletic manifolds.
  I went to Moscow,
  spent a year there,
  and came back with a complete classification in dimension 4 and some general results in any dimension.
  This work represented for me a probable doctoral thesis.
  It was the first global classification theorem in symplectic geometry after the homogeneous case,
  the Kirillov-Kostant-Souriau theorem,
  which states that any homogeneous symplectic manifold by a Lie group is a covering of some coadjoint orbit.
  But Jean-Marie did not pay much attention to my work,
  absorbed as he was by his \guillemots{diffeologies}.
  I was really disappointed,
  I thought that this work deserved to become my doctorate.
  At the same time,
  Paul Donato gave a general construction of the universal covering for any
  quotient of \guillemots{differential groups},
  that is,
  the universal covering of any homogeneous \guillemots{differential space}.
  This construction became his doctoral thesis.
  I then decided to abandon,
  for a while,
  symplectic geometry and to be more interested in differential spaces,
  since it was the only subject that JMS could,
  or wanted,
  to talk about at that time.
  
  {\bf The coming of the irrational torus}.
  It was the year 1983,
  we were participating in a conference on symplectic geometry,
  in Lyon,
  when we decided,
  together with Paul,
  to test diffeology on the {\em irrational torus},
  the quotient of the $2$-torus by an irrational line.
  This quotient is not a manifold but remains a diffeological space,
  moreover a diffeological group.
  We decided to call it $\T_\alpha$,
  where $\alpha$ is the slope of the line.
  The interest for this example came,
  of course,
  from the Denjoy-Poincar\'e flow that we have heard so much about during this conference.
  What had diffeology to say about this group,
  for which topology is completely silent?
  We used the techniques worked out by Paul and computed its fundamental group,
  we found $\ZZ + \alpha \ZZ \subset \RR$.
  The real line  $\RR$ itself appeared as the universal covering of $\T_\alpha$.
  I remember how we were excited by this computation,
  as we didn't believe really in the capabilities of diffeology for saying anything serious about such \guillemots{singular} spaces or groups.
  Don't forget that diffeologies had been introduced for studying infinite dimensional groups and not singular quotients.
  We continued to explore this group and discovered that,
  as diffeological space,
  $\T_\alpha$ is characterized by $\alpha$,
  up to a conjugation by $\GL(2,\ZZ)$,
  and we found that the components of the group of diffeomorphisms of $\T_\alpha$ distinguish the cases where $\alpha$ is quadratic or not.
  It became clear that diffeology was not such a trivial theory and deserved to be developed further.
  At the same time,
  Alain Connes introduced the first elements of noncommutative geometry and applied them to the irrational flow on the torus
  ---~our favorite example~---
  and his techniques didn't give anything more (and at this moment, less) than the diffeological approach,
  which we considered more in the spirit of differential geometry.
  We were well placed to know the application of Connes' theory of irrational flows because he had many fans,
  at the Center for Theoretical Physics at the time,
  who developed his ideas.
  
  Eventually,
  this example convinced me that diffeology was a good tool,
  not as weak as it seemed.
  And I decided to continue exploring this path.
  The result of the computation of the fundamental group of $\T_\alpha$ made me think that everything was as if the irrational flow was a true fibration of the $2$-torus:
  the fiber $\RR$ being contractible the homotopy of the quotient $\T_\alpha$ had to be the same as the total space $\T^2$,
  and one should avoid Paul's group specific techniques to get it.
  But,
  of course,
  $\T_\alpha$ being topologically trivial it could not be an ordinary locally trivial fibration.
  I decided to investigate this question and,
  finally,
  together with a higher homotopy theory,
  I gave a definition of diffeological fiber bundles,
  which are not locally trivial,
  but locally trivial along the plots
  ---~the smooth parametrizations defining the diffeology.
  It showed two important things for me:
  The first one was that the quotient of a diffeological group by any subgroup is a diffeological fibration,
  and thus $\T^2 \to \T_\alpha$.
  The second point was that diffeological fibrations satisfy the exact homotopy sequence.
  I clarified in passing the relation between diffeology and topology by introducing the D-topology,
  since this was clearly a question raised by these new definitions.
  I was done,
  I understood why the homotopy of $\T_\alpha$,
  computed with direct techniques elaborated by Paul,
  gave the homotopy of $\T^2$,
  because of the exact homotopy sequence.
  I spent one year on this work,
  and I returned to Jean-Marie with that and some examples.
  He agreed to listen to me and decided that it could be my dissertation.
  I defended it in November 1985,
  and since then I have become completely involved in the adventure of diffeology.
  
  {\bf Differential, differentiable, or diffeological spaces?}
  The choice of the formula \guillemots{{\em differential spaces}} or \guillemots{{\em differential groups}} was not very happy,
  because \guillemots{{\em differen\-tial}} was already used in math and had a certain usage,
  especially \guillemots{{\em differential groups}} which are groups with a derivation operation.
  This was often pointed out to us.
  I remember that Daniel Kastler insisted that JMS change this name.
  From time to time we tried to find something else,
  without success.
  Finally,
  it was during Paul's thesis defense,
  if I remember correctly,
  that Van Est suggested the word \guillemots{{\em diffeological}} as \guillemots{{\em topological}} to replace \guillemots{{\em differential}}.
  We agreed and decided to use it,
  \guillemots{{\em differential spaces}} became \guillemots{{\em diffeological spaces}}.
  
  There was a damper,
  however,
  \guillemots{{\em diff\'erentiel}} as well as \guillemots{{\em topologique}} have four spoken syllables when \guillemots{{\em diff\'eologique}} has five.
  Anyway,
  I used and abused this new name,
  and some colleagues made fun of me.
  One of them once told me:
  Your \guillemots{{\em dix f\'ees au logis}}\ldots\
%  ---~which means ``ten fairies at home''~---\n
  For a long time then these ten fairies at home pursued me.
  Later,
  Daniel Bennequin pointed out to me that Kuo-Tsai Chen,
  in his work,
  {\em Iterated path integrals} \cite{Che77} in the 1970s,
  defined \guillemots{differentiable spaces} which looked a lot like \guillemots{diffeological spaces}.
  I went to the library,
  read Chen's paper and drew the quick (but unfounded) conclusion that our \guillemots{diffeological spaces} were just equivalent to Chen's \guillemots{differentiable spaces},
  with a slight difference in the definition.
  I was very disappointed,
  I was working on a subject I thought completely new and it appeared to be known and already worked out.
  I decided to drop \guillemots{diffeology} for \guillemots{differentiable} and to give credit to Chen,
  but my attempt to use Chen's vocabulary failed
  ---~the word \guillemots{diffeology} had already entered the practice,
  having helped popularize it myself.
  However,
  it is worth noting that,
  although the axiomatics of Chen and Souriau are similar,
  Souriau's choice is better suited from the point of view of geometry.
  The definition of the plots on open domains,
  rather than on standard simplices or convex subsets,
  radically changes the scope of the theory.
  
  {\bf Last word?} I would add some words about the use or misuse of diffeology.
  Some colleagues were skeptical about diffeology,
  and told me that they are waiting for diffeology to prove something great.
  Well,
  I don't know any theory proving anything,
  but I know mathematicians proving theorems.
  Let me put it differently:
  number theory doesn't prove any theorem,
  mathematicians solve problems raised by number theory.
  A theory is just a framework to express questions and pose problems,
  it is a playground. 
  The solutions of these problems depend on the skill of the mathematicians who are interested in them.
  As a framework for formulating questions in differential geometry,
  I think diffeology is a very good one,
  it offers good tools,
  simple axioms,
  simple vocabulary,
  simple but rich  objects,
  it is a stable category,
  and it opens a wide field of research.
  Now,
  I understand my colleagues,
  there are so many attempts to extend the usual category of differential geometry,
  and so many expectations,
  that it is legitimate to be doubtful.
  Nevertheless,
  I think that we now have enough convincing examples,
  simple or more elaborate,
  for which diffeology brings concrete and formal results.
  And this is an encouragement to persist on this path,
  to develop new diffeological tools,
  and perhaps to prove some day,
  some great theorem :).
  
  
  %\smallskip
  \hfill Jerusalem 2011 (revised 2022).
  
  \bigskip
  {\em When I started this book,
  Jean-Marie Souriau was still alive.
  He frequently asked me about my progress.
  He was eager to know if people were buying into his theory,
  and he was happy when I could tell him sometimes that,
  yes,
  some people in Tel Aviv or Texas had mentioned it in a paper or discussed it on an Internet forum.
  Today,
  as I finish this book and write the last sentences,
  Jean-Marie is no longer with us.
  He will not see the book published and finished.
  It is sad,
  diffeology was his last program,
  in which he had high expectations concerning geometric quantization.
  I don't know if diffeology will live up to his expectations,
  but I am sure that it is now a mature theory,
  and I dedicate this book to his memory.
  Whether it is the right framework for carrying out Souriau's quantization program remains an open question.
  
  \medskip
  \hfill Aix-en-provence 2012.
  }
  
\end{chaphead}

