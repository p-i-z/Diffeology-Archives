\chapter*{Preface to the revised print}
%\addcontentsline{toc}{chapter}{Preface II}

\begin{chaphead}
  
  This textbook was completed in 2011 and was published for the first time in 2013,
  by the American Mathematical Society,
  almost 10 years ago.
  Now Beijing World Publishing Corporation has decided to reprint it for mainland China.
  This is the perfect opportunity to correct the mistakes or misprints of the first print.
  I would like to believe that there are no mistakes or misprints left, but I fear that zero-fault is only an asymptotic notion.%
  \footnote{I will maintain an errata file on my page, if needed.\\
  \texttt{http://math.huji.ac.il/\raisebox{-2pt}{\~{}}piz/documents/erratawpc.pdf}
  }
  %
  That being said,
  this revision has also been an opportunity to update a few things that were undecided or unfinished at that time.
  In particular,
  the question of whether any local induction was an immersion was settled,
  see \exref{Induction-of-intervals-into-domains}.
  In fact,
  this question was solved by Henri Joris already in 1982 but I was unaware of it:
  the semicubic $y^2 = x^3$ is a submanifold for the subset diffeology while the induction $t \mapsto(t^2, t^3)$ that describes it is not an immersion.
  Thus,
  there are indeed local inductions that are not immersions,
  and the case is closed.
  This example was the last brick to understand and situate the different subcategories relative to each other:
  induction/subduction,
  local-induction/local-subduction and immersion/submersion.
  This example was an opportunity to refine the notion of submanifolds in a diffeological space by distinguishing between simply submanifolds,
  embedded submanifolds and smoothly embedded submanifolds.
  These last ones do not only put into play the D-topologies of the space and the subspace but also the germs of diffeomorphisms of the subspace that extend to the ambient space.
  All this is described in \xart{Submanifolds-of-a-diffeological-space}{Note 2},
  \xart{Embeddings}{Definition 2} and \art{Embedded-subsets-of-a-diffeological-space}.
  
  Another question has been solved in the meantime but is not included in details in the revision:
  every symplectic manifold is a coadjoint orbit (maybe affine) of the group of Hamiltonian diffeomorphisms \art{Symplectic-manifolds-are-coadjoint-orbits}.
  This result has been improved,
  and every symplectic manifold is a coadjoint orbit for the linear action of the group of automorphisms of the integration bundle of the symplectic manifold,
  which is a central extension of the group of Hamiltonian diffeomorphisms by the group of periods of the symplectic form \art{Integration-bundles-of-closed-2-forms}.%
  \footnote{The construction of the integration bundle for a manifold has been published in \cite{Igl95}.}
  This theorem which is only mentioned calls for several remarks;
  among other things:
  
  1) Diffeology is not just a convenient language,
  it is a rigorous formal framework for theorems which do not exist otherwise.
  In this case the group is not a Lie group,
  but above all:
  the integration bundle is almost never a manifold.
  The torus of periods,
  is in general topologically trivial.
  And yet,
  diffeology offers a full generalization of the Kirillov-Kostant-Souriau theorem.
  
  2) Various attempts to obtain this theorem,
  using infinitesimal approaches,
  have failed because the torus of periods has the same ``Lie algebra'' as $\RR$ or $\S^1$ or any quotient $\RR/\Gamma$,
  where $\Gamma$ is any strict subgroup of $\RR$.
  The only way to solve this problem is to obtain directly the central extension by a finite method,
  and in this degree of generality,
  this is only possible by diffeology.

  3) This is a new example where diffeology combines at the same time,
  in a non trivial way,
  singularity and infinite dimension.
  A first example was the singular reduction in symplectic diffeology.%
  \footnote{See \cite{PIZ15}.}
  
  On a different note,
  since the publication of \emph{Diffeology} in 2013,
  a way has been opened to a convergence with non-commutative geometry through two papers,
  that I just mention in passing (the book has its limitations).
  One can functorially associate a $\CC^*$-algebra with any orbifold,
  the same thing with any quasifold \xart{Structure-group-of-orbifolds}{Note 2},
  a (not so) newcomer in diffeology.%
  \footnote{See \cite{IZL18} and \cite{IZP21}.}
  This is a big program that has just been scratched, how far can we go on this path?
  
  The existence of Klein's stratification initiated a new thinking about the general notion of stratification \art{Klein-structure-and-singularities-of-a-diffeological-space}.
  The fact that any subset of a diffeological space is by default equipped with the subset diffeology makes it possible to decompose the ordinary geometric properties of stratifications,
  usually presented in block,
  into distinct subclasses that simplify the reading of the many different situations of stratified spaces.
  Basically,
  and this is the minimal condition in diffeology:
  any partition that satisfies the frontier condition
  ---~the adherence of a stratum is a union of strata~---
  is a diffeological stratification.
  Then,
  the strata are equipped with the subset diffeology.
  It is not necessary anymore to require that the strata are submanifolds,
  and there are now examples where they are not.%
  \footnote{The space of geodesics of the $2$-torus in particular,
  see \cite{PIZ22b}.}
  The property of being locally-closed is characterized by the Alexandrov D-topology of the quotient:
  defining or not a partial order,
  and so on.
  I mentioned this discussion but did not elaborate on it.%
  \footnote{A recent paper on the Klein stratification of orbifolds was the opportunity to clarify all these aspects of stratifications in diffeology;
  see \cite{GIZ22}.}
  Here again there is a lot to do in this direction.
  To conclude,
  I hope that with this book cleaned up a bit,
  the reader will benefit from a new look at differential geometry,
  in a global and inclusive version.

  \textsc{Last word.} As this is a reprint and not a new edition,
  I have not added new paragraphs or exercises (but an index) in order to preserve references that have appeared in other publications since then.
  This manual does not pretend to cover everything in diffeology.
  I have written it to serve as a reference text to establish the basic constructs of diffeology,
  and also to continue the reflection on symplectic diffeology as it is at the origin of this theory.
  Since the first edition,
  diffeology has developed in many directions,
  one finds diffeology in category theory,
  in differential homotopy,
  and even in formal calculus and machine learning.
  A growing number of mathematicians contribute to this development in the various fields where diffeology has branched out,
  by publishing original papers,
  by organizing summer schools,
  conferences,
  and recently a monthly world seminar has been instituted,
  open to all who wish to participate.
  All information is available at
  
  \hfill \verb|http://diffeology.net|
\end{chaphead}
