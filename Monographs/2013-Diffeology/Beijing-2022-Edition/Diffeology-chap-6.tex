%%%%%%%%%%%%%%%%%%%%%%%%%%%%%%%%%%%%%%%%%%%%%%%%%%%%%%%%%%
%%
%%  De Rham Calculus MARK: -
%%
%%%%%%%%%%%%%%%%%%%%%%%%%%%%%%%%%%%%%%%%%%%%%%%%%%%%%%%%%%

\chapter{Cartan-De Rham Calculus}

\label{Chapter-De-Rham-Calculus}
\newcommand{\ChapterDFADRC}{Cartan De-Rham Calculus}

\begin{chaphead}
  The general philosophy of diffeology is to carry,
  functorially,
  geometrical objects defined and constructed on real domains \art{Real-vector-spaces-and-domains} to diffeological spaces.
  By nature,
  since diffeology on a set $\X$ is defined by parametrizations in $\X$,
  geometrical covariant objects are perfectly adapted to this approach.
  Differential forms are one of the clearest examples.
  This chapter is devoted to introducing the differential calculus in diffeology.
  
  The first section of this chapter,
  on {\em Multilinear Algebra},
  puts the notations in place and summarizes the basic notions of linear algebra on which all this building is founded.
  For the professional mathematician,
  there is nothing new here,
  and he can skip it.
  The reason for this section is to give a self-consistent expository text for the graduate student,
  or post-graduate,
  where we regroup what is then used in diffeology.
  The same holds for the second section,
  on {\em Smooth Forms on Real Domains},
  where the foundations of differential calculus in $\RR^n$ are presented.
  It is here for the same reason as the first section.
  Again, the professional may skip this section.
  
  It is in the third section \art{Differential-forms-on-diffeological-spaces},
  on {\em Differential Forms on Diffeological Spaces},
  where the specificity of differential calculus in diffeology is introduced,
  and where we begin to establish the main theorems relative to differential forms in diffeology.
  It continues then along the following sections of this chapter.
  
  It could help,
  before absorbing all these axiomatics,
  to first get an idea on what we shall talk about.
  A {\em differential form} on a diffeological space is defined as the family of its pullbacks by the plots of the space.
  In other words,
  a differential form is defined as a map which associates with each plot a smooth form of its domain,
  satisfying a compatibility condition when the plots are composed with a smooth parametrization.
  This definition makes sense,
  if the diffeological space is a manifold,
  diffeological forms are just the ones we are used to.
  Then,
  exterior {\em Cartan calculus} and {\em De Rham cohomological calculus},
  the pullbacks by smooth maps,
  every basic construction associated with differential calculus,
  pass naturally to diffeology.
  The Lie derivative of forms extends to diffeological spaces by adapting the ordinary definition to the diffeological context \art{The-Lie-derivative-of-differential-forms}.
  The Cartan formula for diffeological spaces is established  \art{The-Cartan-Lie-formula}.
  The sets of differential forms on diffeological spaces are naturally diffeological vector spaces.
  
  The definition of {\em pointed $p$-forms} on a diffeological space $\X$ introduces the space $\Forms^p(\X)$ of $p$-forms over $\X$ \art{The-p-forms-bundle-of-diffeological-spaces}.
  The space $\Forms^p(\X)$ carries a natural functional diffeology such that the projection on $\X$ is a subduction,
  and such that a global form is just one smooth sections.
  This construction gives,
  in particular,
  a {\em cotangent space} ${\T}^*\X$ defined over any diffeological space \art{The-cotangent-space},
  without need to refer to some ``tangent space''.
  The cotangent space in diffeology is not the dual of a tangent space.
  The {\em Liouville forms} \art{The-Liouville-forms-on-the-spaces-of-p-forms} are introduced and their invariance by the action of the group of diffeomorphisms of the base is established.
\end{chaphead}

%%%%%%%%%%%%%%%%%%%%%%%%%%%%%%%%%%%%%%%%%%%%%%%%%%%%%%%%%%
%% MARK: Multilinear Algebra
%%%%%%%%%%%%%%%%%%%%%%%%%%%%%%%%%%%%%%%%%%%%%%%%%%%%%%%%%%

\section*{Multilinear Algebra}
\label{Section-Multilinear-algebra}

\begin{sechead}
  This section establishes some notation and presents a summary of the basic definitions and constructions in real linear and multilinear algebra,
  used thereafter for Cartan calculus in diffeological spaces.
  I chose to follow the presentation of Souriau's book ``Calcul Lin\'eaire'' \cite{Sou64}.
  I shall mention if a definition or a
  proposition applies just for finite dimensional real vector spaces.
  Otherwise,
  if no such a mention is made,
  it is assumed to work with any real vector space.
\end{sechead}

\begin{article}\artlabel{Spaces of linear maps}
  \addcontentsline{toc}{section}{\small\hspace{10pt} Spaces of linear maps}
  \label{Spaces-of-linear-maps}
  Let $\E$ and $\F$ be two real vector spaces.
  We denote by $\E*\F$ the space $\Lin(\E,\F)$ of {\em linear maps}\index{Linear map} from $\E$ to $\F$.
  $$%
    \E*\F = \{ \A: \E \to \F \mid \A(ax+by) = a \A(x) + b \A(y), \ \forall a,b \in \RR, \ \forall x,y \in \E \}.
    $$%
  The space $\E*\F$ is itself a real vector space with
  $$%
    (a\,\A)(x) = a \, \A(x), \text{ and } (\A + \B)(x) = \A(x) + \B(x),
    $$%
  for all $a \in \RR$ and all $\A, \B \in \E*\F$.
\end{article} %% Spaces-of-linear-maps

\begin{article}\artlabel{Bases and linear groups}
  \addcontentsline{toc}{section}{\small\hspace{10pt} Bases and linear groups}
  \label{Basis-and-linear-groups}
  Let $\E$ be a real vector space of finite dimension,
  with $\dim(\E) = n$.
  A {\em basis}\index{Basis} of $\E$ is a set $\cB$ of $n$ independent vectors $b_1,\ldots,b_n$ of $\E$ generating the whole space $\E$.
  
  \begin{itemize}
    \item[{1.}] {\em Independency} --- For all $s_i \in \RR$, $i = 1,\ldots, n$,
    $\sum_{i = 1}^n s_i b_i = 0$ if and only if $s_i = 0$ for all
    $i = 1,\ldots, n$.
    \item[{2.}] {\em Generating} --- For all $u \in \E$,
    there exist $u_i \in \RR$, $i = 1,\ldots, n$,
    such that $u = \sum_{i = 1}^n u_i b_i$.
  \end{itemize}
  %
  The numbers $u_i \in \RR$ such that $u = \sum_{i=1}^n u_ib_i$ are called the {\em coordinates}\index{Coordinates} of $u$ in the basis $\cB$.
  The condition to be independent for the $b_i$ implies that the decomposition of $u$ in terms of coordinates is unique.
  Every basis $\cB$ defines a natural isomorphism from $\RR^n$ to $\E$,
  which will be denoted by the same letter,
  $$%
    \cB \in \Isom(\RR^n,\E), \ \cB \vect{s_1 \\ \vdots \\ s_n} = \sum_{i = 1}^n s_i b_i.
    $$%
  The two conditions on the $b_i$,
  to be independent and generators,
  is equivalent for $\cB$ to be an isomorphism.
  Conversely,
  every isomorphism $\cB \in \Isom(\RR^n,\E)$ defines $n$ independent vectors $b_i = \cB(\ee_i)$,
  generating $\E$,
  where the $\ee_i$ are the vectors of the canonical basis of $\RR^n$.
  Therefore,
  we identify a basis with its associated isomorphism,
  and we consider indifferently the set of vectors of a basis,
  or the associated isomorphism,
  as the same thing,
  \ie\ $\cB = (b_1,\ldots,b_n)$.
  We shall denote
  $$%
    \Bases(\E) = \Isom(\RR^n,\E).
    $$%
  We denote by the indexed bold letters $\ee_i$, $i = 1, \ldots, n$,
  the vectors of the canonical basis of $\RR^n$,
  that is,
  $\ee_i$ is the vector  having all its coordinates zero,
  except at the rank $i$,
  where it has the value $1$.
  For example:
  $$%
    \ee_1 = \vect{1 \\ 0 \\ \downarrow },
    \
    \ee_2 = \vect{0 \\ 1 \\ 0\cr \downarrow },
    \
    \ee_3 = \vect{0 \\ 0 \\ 1 \\ 0\\ \downarrow },
    \
    \text{etc.}
    $$%
  Regarded as an isomorphism of $\RR^n$,
  the canonical basis $\cI_n = (\ee_1,\ldots,\ee_n)$ of $\RR^n$ is just the identity $\id_n$.
  The set of all the bases of $\RR^n$ is just the group of linear transformations,
  $$%
    \Bases(\RR^n) = \GL(n,\RR).
    $$%
  As a group, $\Bases(\RR^n)$ acts on the bases of $\E$,
  by right composition:
  $$%
    \text{For all } \M \in \GL(n,\RR), \text{ for all } \cB \in \Bases(\E), \ \cB \circ \M^{-1} \in \Bases(\E).
    $$%
  Note that this action is transitive.
  For every pair of bases $\cB$ and $\cB'$ the composition $\cB^{-1} \circ \cB' \in \GL(n,\RR)$.
  This operation is called the {\em change of basis}.
  The notion of basis is just a means to reduce vector spaces to standard representatives under the equivalence defined by isomorphisms.
  Also note that this definition of basis can be extended to any other kind of vector space,
  finite or infinite dimensional,
  on any field $\KK$,
  as soon as we have chosen in each class of isomorphisms a favorite representative.
\end{article} %% Basis-and-linear-groups

\begin{article}\artlabel{Dual of a vector space}
  \addcontentsline{toc}{section}{\small\hspace{10pt} Dual of a vector space}
  \label{Dual-of-a-vector-space}
  Let $\E$ be a real vector space,
  the {\em dual\/}\index{Dual vector space} of $\E$,
  denoted by $\E^*$,
  is the space of linear maps from $\E$ to $\RR$,
  that is,
  $$%
    \E^* = \E*\RR  = \Lin(\E,\RR).
    $$%
  An element of $\E^*$ is also called a {\em covector}.
  If $\E$ is finite dimensional,
  then $\E$ and $\E^*$ are isomorphic,
  $$%
    \dim(\E) < \infty \ \Rightarrow \ \E \sim \E^*, \text{ and } \dim(\E^*) = \dim(\E).
    $$%
\end{article} %% Dual-of-a-vector-space

\begin{proof}
  Let $\cB = (\ee_1,\ee_2,\ldots,\ee_n)$,
  where $\dim(\E) = n$.
  Let us define the {\em dual basis} $\cB^* = (\ee^1,\ee^2,\ldots,\ee^n)$ by
  $$%
    \ee^i(\ee_j) = \delta^i_j,
    $$%
  where $\delta^i_j$ is the Kronecker symbol:
  $\delta^i_j = 1$ if $i = j$ and 0 otherwise.
  Then,
  for every $w \in \E^*$ and for every vector $x \in \E$,
  we have
  $$%
    w(x) = w\bigg(\sum_{i = 1}^n x^i \ee_i \bigg) = \sum_{i = 1}^n x^i w(\ee_i) = \sum_{i = 1}^n x^i w_i = \sum_{i = 1}^n w_i \ee^i(x),
    $$%
  where $w_i = w(\ee_i)$.
  Then,
  $w = \sum_{i = 1}^n w_i \ee^i$,
  and ${\B}^*$ is a generator system of $\E^*$.
  It is independent,
  since $\sum_{i=1}^n s_i \ee^i = 0$ implies $\big(\sum_{i=1}^n s_i \ee^i\big)(\ee_j) = s_j =0$.
  Therefore,
  $\cB^*$ is a basis of $\E^*$.
\end{proof}

\begin{article}\artlabel{Bilinear maps}
  \addcontentsline{toc}{section}{\small\hspace{10pt} Bilinear maps}
  \label{Bilinear-maps}
  Let $\E$,
  $\F$ and $\G$ be three real vector spaces.
  Let us interpret the space $\E*(\F*\G)$.
  By definition,
  $\E*(\F*\G)$ is the space of linear maps from $\E$ to the space of linear maps from $\F$ to $\G$.
  Let us consider a linear map $\A : \E \to \F*\G$.
  For any $x \in \E$,
  $\A(x) \in \F*\G$,
  then,
  for any $y \in \F$, $[\A(x)](y) \in \G$.
  We can summarize this double mapping by
  $$%
    \A = [x \mapsto [y \mapsto z]],
    $$%
  that is,
  for all $x \in \E$ and all $y \in \F$ there exists $z \in \G$,
  such that $z = [\A(x)](y)$.
  We can forget the superfluous brackets and denote $z = \A(x)(y)$.
  So,
  the map $\A$ can be identified with some map from $\E \times \F$ to $\G$,
  and $\A(x)(y)$ could also be be denoted by $\A(x,y)$.
  From the linearity of the operator $\A$,
  and from the linearity of the operator $\A(x)$,
  for all $x \in \E$,
  we get the following characterization of $\A(x)(y)$:
  $$%
    \begin{array}{cl} \left.
    \begin{array}{rcl}
    \A(x+x')(y) & = & \A(x)(y) + \A(x')(y) \\
    \A(sx)(y) & = & s\A(x)(y)
    \end{array}
    \right\}
    &
    \text{linearity of $\A$,}
    \\
    \\
    \left.
    \begin{array}{rcl}
    \A(x)(y+y') & = & \A(x)(y) + \A(x)(y') \\
    \A(x)(sy) & = & s\A(x)(y)
    \end{array}
    \right\}
    &
    \text{linearity of $\A(x)$,}
    \end{array}
    $$%
  where $x, x' \in \E$, $y, y' \in \F $,
  and $s \in \RR$.
  This is equivalent to
  $$%
    \A\bigg(\sum_i t_i x_i\bigg)\bigg(\sum_j s_j y_j\bigg) = \sum_{i,j} t_is_j \A(x_i)(y_j),
    $$%
  where $s_i, t_j \in \RR$, $x_i \in \E$ and $y_j \in \F$ are finite families.
  Such operators are called {\em bilinear maps}\index{Bilinear map}.
\end{article} %% Bilinear-maps

\begin{article}\artlabel{Symmetric and antisymmetric bilinear maps}
  \addcontentsline{toc}{section}{\small\hspace{10pt} Symmetric and antisymmetric bilinear maps}
  \label{Symmetric-and-antisymmetric-bilinear-maps}
  Let $\E$ and $\F$ be two real vector spaces.
  The {\em symmetric}\index{Symmetric} of a bilinear map $\A \in \E*(\E*\F)$ is the map ${\B} \in \E*(\E*\F)$ defined by ${\B}(y)(x) = \A(x)(y)$.
  The operator $\A$ is said to be {\em symmetric} if ${\B} = \A$,
  and to be {\em antisymmetric}\index{Antisymmetric} if ${\B} = -\A$.
  
  Any bilinear map $\A \in \E*(\E*\F)$ decomposes,
  in a unique way,
  into the sum of a symmetric and an antisymmetric component,
  $$%
    \A(x)(y) = \undemi \ [\A(x)(y) + \A(y)(x)] + \undemi \ [\A(x)(y) - \A(y)(x)].
    $$%
  We deduce, from this decomposition, that an operator is antisymmetric if and only if $\A(x)(x) = 0$.
\end{article} %% Symmetric-and-antisymmetric-bilinear-maps

\begin{proof}
  If $\A$ is antisymmetric,
  clearly $\A(x)(x) = -\A(x)(x) = 0$.
  Now,
  if $\A(x)(x) =0$,
  then $0 = \A(x+y)(x+y)  = \A(x)(x) + \A(x)(y) + \A(y)(x) + \A(y)(y) = \A(x)(y) + \A(y)(x)$.
  Therefore,
  $\A(x)(y) = -\A(y)(x)$.
\end{proof}

\begin{article}\artlabel{Multilinear operators}
  \addcontentsline{toc}{section}{\small\hspace{10pt} Multilinear operators}
  \label{Multilinear-operators}
  Let $\E_1,\ldots,\E_\N, \E_{\N+1}$ be a family of real vector spaces.
  Extending the previous construction \art{Bilinear-maps},
  we can interpret the space $\E_1*(\E_2*(\E_3*(\cdots *\E_{{\N}+1})))$ as an operator which associates with ${\N}$ vectors $(x,y,\ldots,z) \in \E_1\times \E_2 \times \cdots \times \E_{\N}$,
  some element of $\E_{{\N}+1}$,
  such that
  $$%
    \A \bigg(\sum_i s_i x_i\bigg)\bigg(\sum_j t_j y_j\bigg) \cdots \bigg(\sum_k r_k z_k\bigg) = \sum_{i,j, \ldots ,k}s_i t_j \cdots r_k \A(x_i)(y_j) \cdots (z_k),
    $$%
  where $s_i, t_j,\ldots r_k \in \RR$,
  $x_i \in \E_1$,
  $y_j \in \E_2$,
  \ldots, $z_k \in \E_{\N}$ are finite families.
  This last condition expresses the linearity of all the spaces involved in the definition of $\A$.
  Such operators are called {\em multilinear operators}\index{Multilinear operator}.
  The sum of two elements of  $\E_1*(\E_2*(\E_3*(\cdots *\E_{{\N}+1})))$,
  and the multiplication of an element of $\E_1*(\E_2*(\E_3*(\cdots *\E_{{\N}+1})))$ by a scalar,
  are given by
  \begin{align*}
    [\A + {\B}](x)(y)\cdots (z)   & =  \A(x)(y) \cdots (z) + {\B}(x)(y) \cdots (z), \\
    \text{$[sA](x)(y)\cdots (z)$} & =  \text{$s[\A(x)(y)\cdots (z)].$}
  \end{align*}
  For a multilinear operator $\A$,
  we use indifferently the multiparentheses notation $\A(x)(y)\cdots(z)$ or the simple parentheses notation $\A(x, y, \ldots, z)$,
  depending upon which is more suitable.
\end{article} %% Multilinear-operators

\begin{article}\artlabel{Symmetric and antisymmetric multilinear operators}
  \addcontentsline{toc}{section}{\small\hspace{10pt} Symmetric and antisymmetric multilinear operators}
  \label{Symmetric-and-antisymmetric-multilinear-operators}
  Let $\E$ and $\F$ be two real vector spaces.
  Let $\A \in \E*(\E*(\E*(\cdots *\F)))$ be an $n$-multilinear operator.
  
  1. The operator $\A$ is said to be {\em totally symmetric},
  or simply {\em symmetric},
  if its value $\A(x)(y)\cdots (z)$ does not change when one exchanges any two vectors.
  
  2. The operator $\A$ is said to be {\em totally antisymmetric},
  or simply {\em antisymmetric},
  if its value $\A(x)(y)\cdots (z)$ changes sign when one exchanges any two vectors.
  
  Let us consider the group of {\em permutations}\index{Permutation} $\Perms(n)$ of the set of indices $\{1,\ldots,n\}$,
  acting on the $n$-tuple $(x_1,\ldots,x_n) \in \E^n$ by
  $$%
    \sigma(x_1,\ldots,x_n) =  (x_{\sigma(1)},\ldots, x_{\sigma(n)}),
    $$%
  for all $\sigma \in \Perms(n)$ and all $(x_1,\ldots,x_n) \in \E^n$.
  Totally symmetric or antisymmetric operators are also characterized by
  $$%
    \A(x_{\sigma(1)})\cdots(x_{\sigma(n)}) =
    \left\{
    \begin{array}{rl}
    \A(x_1)\cdots(x_n)               & \text{if $\A$ is symmetric,}\\
    \sgn(\sigma)\, \A(x_1)\cdots(x_n) & \text{if $\A$ is antisymmetric,}
    \end{array}
    \right.
    $$%
  where $\sigma \in \Perms(n)$,
  and $\sgn(\sigma)$ is the {\em signature} of the permutation $\sigma$,
  that is,
  $\sgn(\sigma)$ is equal to 1 if $\sigma$ decomposes into an even number of {\em transpositions}\index{Transposition} (permutations of only two distinct elements),
  and $\sgn(\sigma) = -1$ otherwise.
  The number $n$ is called the {\em order} or {\em degree} of the operator $\A$,
  and it is usually denoted by $\deg(\A)$.
  Note that $\A$ is antisymmetric if and only if,
  evaluated on any family of vectors with two repeated vectors,
  $\A$ vanishes:
  $$%
    \text{$\A$ antisymmetric} \  \Leftrightarrow \ \A \cdots(x) \cdots (x) \cdots = 0.
    $$%
\end{article} %% Symmetric-and-antisymmetric-multilinear-operators

\begin{proof}
  The first part of the proposition,
  that is,
  the formula with the permutations,
  results immediately from a decomposition of a permutation in a finite product of transpositions.
  Now,
  it is clear that if $\A$ is antisymmetric,
  then
  $$%
    \A \cdots(x) \cdots (x) \cdots = - \A \cdots(x) \cdots (x) \cdots = 0.
    $$%
  Conversely,
  if $\A \cdots(x) \cdots (x) \cdots = 0$,
  we have
  \begin{align*}
    \A \cdots(x+x') \cdots (x +x') \cdots & = \A \cdots(x) \cdots (x)\cdots + \A \cdots(x) \cdots (x')\cdots \\
    & + \A \cdots(x') \cdots (x) \cdots + \A \cdots(x') \cdots (x')\cdots \\
    & =  \A \cdots(x) \cdots (x') \cdots + \A \cdots(x') \cdots (x) \cdots\,.
  \end{align*}
  Therefore,
  $\A \cdots(x) \cdots (x') \cdots = - \A \cdots(x') \cdots (x)\cdots$.
\end{proof}

\begin{article}\artlabel{Tensors}
  \addcontentsline{toc}{section}{\small\hspace{10pt} Tensors}
  \label{Tensors}
  Let $\E$ be a real vector space,
  and let $\E^* = \E*\RR = \Lin(\E,\RR)$ be the dual space \art{Dual-of-a-vector-space}.
  
  1. A {\em covariant $p$-tensor}\index{Covariant tensor} of $\E$ is a $p$-multilinear operator $\A$ defined $p$-times on $\E$ with values in $\RR$,
  that is, an element of $\E*(\E*(\cdots *\E)*\RR)$,
  where $\E$ is repeated $p$-times.
  A covariant 1-tensor is just an element of $\E^*$.
  $$%
    \A(x_1)(x_2)\cdots (x_p) \in \RR, \text{ where } x_1, x_2,\ldots, x_p \in \E.
    $$%
  %
  2. A {\em contravariant $p$-tensor}\index{Contravariant tensor} of $\E$ is a $p$-multilinear map ${\B}$ defined $p$-times on $\E^*$ with values in $\RR$,
  that is,
  an element of $\E^* * (\E^* * (\cdots * \E^*) * \RR)$,
  where $\E^*$ is repeated $p$-times
  $$%
    \B(w_1)(w_2)\cdots(w_p) \in \RR, \text{ where } w_1, w_2,\ldots, w_p \in \E^*.
    $$%
  
  3. A {\em mixed tensor}\index{Mixed tensor},
  $p$-covariant and $q$-contravariant,
  is a $(p+q)$-multilinear operator ${\C}$ defined $p$-times on $\E$ and $q$-times on $\E^*$ with values in $\RR$.
  $$%
    \C(x_1)(x_2)\cdots (x_p)(w_1)(w_2)\cdots(w_q) \in \RR, \text{ where }
    \left\{
    \begin{aligned}
    x_1, x_2,\ldots, x_p & \in \E, \\
    w_1, w_2,\ldots, w_q & \in \E^*.
    \end{aligned}
    \right.
    $$%
  Note that a $0$-tensor is any map from $\E^0 = \{0\}$ to $\RR$;
  thus, just a number.
  Therefore,
  the space of $0$-tensors of any vector space is $\RR$.
\end{article} %% Tensors

\begin{article}\artlabel{Vector space and bidual}
  \addcontentsline{toc}{section}{\small\hspace{10pt} Vector space and bidual}
  \label{Vector-space-and-bidual}
  Every element of a vector space $\E$ defines an element of its {\em bidual}\index{Bidual} $(\E^*)^* = (\E*\RR)*\RR$,
  as follows:
  %
  $$%
    \text{For all $x \in \E$, for all $w \in \E^*$,} \ \hat x(w) = w(x) \text{ and } \hat x \in (\E^*)^* = \E^* * \RR.
    $$%
  %
  If $\dim(\E) < \infty$,
  then the map $x \mapsto \hat x$ is a canonical isomorphism which identifies the space $\E$ with its bidual $(\E^*)^*$.
\end{article} %% Vector-space-and-bidual

\begin{proof}
  The map $\hat x$ is obviously linear.
  Let $x \in \E$ such that $\hat x = 0$,
  so $\hat x(w) = w(x) = 0$ for all $w \in \E^*$.
  If $\dim(\E) < \infty$,
  then $x=0$,
  and
  $x \mapsto \hat x$ is injective.
  Thus, $\dim(\E) = \dim(\E^*) = \dim((\E^*)^*)$ \art{Dual-of-a-vector-space},
  and $x \mapsto \hat x$ injective implies $x \mapsto \hat x$ is bijective.
\end{proof}

\begin{article}\artlabel{Tensor products}
  \addcontentsline{toc}{section}{\small\hspace{10pt} Tensor products}
  \label{Tensor-products}
  Let $\E$ be a real vector space.
  Let $\A$ and $\B$ be any two tensors of $\E$,
  covariant,
  contravariant or mixed.
  Let $\A$ be of order $p$ and $\B$ be of order $q$.
  We define the tensor product $\A \otimes \B$ by
  $$%
    \A \otimes {\B} (a_1)\cdots(a_p)(b_1)\cdots(b_q) = \A(a_1)\cdots(a_p) \times {\B}(b_1)\cdots(b_q),
    $$%
  where the $(a_i)$ are in the domain of $\A$ and the $(b_j)$ in the domain of ${\B}$.
  The sign $\times$ denotes the ordinary multiplication of real numbers.
  The tensor product $\A \otimes \B$ is a $(p+q)$-tensor.
  It can be covariant if $\A$ and ${\B}$ are covariant,
  contravariant if $\A$ and ${\B}$ are contravariant,
  mixed otherwise.
  Moreover,
  the tensor product is associative,
  $(\A \otimes {\B}) \otimes {\C} = \A \otimes({\B} \otimes {\C})$.
  If $\A$ or $\B$ is a $0$-tensor,
  that is,
  a real $s$,
  then $s \otimes \A = \A \otimes s = s \times \A$.
\end{article} %% Tensor-products

\begin{article}\artlabel{Components of a tensor}
  \addcontentsline{toc}{section}{\small\hspace{10pt} Components of a tensor}
  \label{Components-of-a-tensor}
  Let $\E$ be a real vector space of finite dimension,
  and let $\dim(\E) = n$.
  Let $\A$ be a tensor of $\E$.
  Let $\cB = (\ee_1, \ldots, \ee_n)$ be a basis of $\E$,
  and let $\cB^* = (\ee^1, \ldots, \ee^n)$ be the dual basis of $\cB$ \art{Dual-of-a-vector-space}.
  Let us consider,
  for example, a tensor $\A$,
  $2$-times covariant and $1$-time contravariant.
  Then,
  for any $x,y \in \E$ and any $w \in \E^*$,
  %
  \begin{align*}
    \A(x)(y)(w) & =  \A \bigg(\sum_i x^i \ee_i\bigg) \bigg(\sum_j y^j \ee_j\bigg) \bigg(\sum_k w_k \ee^k\bigg) \\
    & =  \sum_{i,j,k} x^i y^j w_k \A_{ij}^k\,, \text{ where } \A_{ij}^k = \A(\ee_i)(\ee_j)(\ee^k).
  \end{align*}
  %
  The numbers $\A^k_{ij}$ are called the {\em components of the tensor $\A$}\index{Component} for the basis $\cB$.
  In fact,
  the space of these tensors
  ---~$2$-times covariant and $1$-time contravariant~---
  has a natural basis associated with $\cB$,
  the set of all the tensors $\ee^i\otimes \ee^j \otimes \ee_k$,
  where $i, j, k = 1\cdots n$.
  In other words,
  $$%
    \A = \sum_{i,j,k} \A^k_{ij} \ \ee^i\otimes \ee^j \otimes \ee_k.
    $$%
  Let us consider the product of two tensors $\A$ and ${\B}$.
  Let us assume that $\A^k_{ij}$ are the components of $\A$ for the basis $\cB$ and ${\B}^{mn}_{pqr}$ are the components of ${\B}$,
  then
  $$%
    [\A \otimes {\B}]^{k\,mn}_{ij\,pqr} = \A^k_{ij} \times {\B}^{mn}_{pqr}.
    $$%
  Note that the symmetry or the antisymmetry of a tensor
  ---~covariant or contravariant~---
  can be checked on its components for any basis.
  Let us consider,
  for example,
  a covariant tensor $\A$,
  let $\cB$ be a basis of $\E$,
  $$%
    \A_{\sigma(i) \sigma(j)\cdots \sigma(k)} =
    \left\{
    \begin{aligned}
    \A_{ij\ldots k}                      && \text{if $\A$ is symmetric,} \\
    \sgn(\sigma) \times \A_{ij\ldots k}  && \text{if $\A$ is antisymmetric.}
    \end{aligned}
    \right.
    $$%
  For example,
  for covariant 2-tensors the condition of antisymmetry writes
  $$%
    \A_{ij} + \A_{ji} = 0.
    $$%
\end{article} %% Components-of-a-tensor

\begin{article}\artlabel{Symmetrization and antisymmetrization of tensors}%
  \addcontentsline{toc}{section}{\small\hspace{10pt} Symmetrization and antisymmetrization of tensors}%
  \label{Symmetrization-and-antisymmetrization-of-tensors}%
  Let $\E$ be a real vector space,
  and let $\T$ be a covariant $p$-tensor of $\E$.
  
  1. The {\em symmetrization}\index{Symmetrization} of the tensor $\T$ is the tensor $\Symm(\T)$ defined by
  $$%
    \Symm({\T})(x_1)\cdots(x_p) = {1 \over p!} \sum_{\sigma \in \Perms(p)} {\T}(x_{\sigma(1)})\cdots(x_{\sigma(p)}).
    $$%
  The tensor $\T$ is symmetric if and only if $\Symm(\T) = \T$.
  The map $\Symm$ is a projector from the space of all covariant $p$-tensors onto the subspace of symmetric covariant $p$-tensors,
  $\Symm \circ \Symm = \Symm$.
  
  2. The {\em antisymmetrization}\index{Antisymmetrization} of $\T$ is the tensor $\Alt(\T)$ defined by
  $$%
    \Alt({\T})(x_1)\cdots(x_p) = {1 \over p!} \sum_{\sigma \in \Perms(p)} \sgn(\sigma) \times {\T}(x_{\sigma(1)})\cdots(x_{\sigma(p)}).
    $$%
  The tensor ${\T}$ is antisymmetric if and only if $\Alt(\T) = {\T}$.
  The map $\Alt$ is a projector from the space of all covariant $p$-tensors onto the subspace of antisymmetric covariant $p$-tensors,
  $\Alt \circ \Alt = \Alt$.
  
  What has been said above,
  for covariant tensors,
  can be transposed to contravariant tensors and,
  then,
  give projectors to the spaces of symmetric or antisymmetric contravariant tensors.
\end{article} %% Symmetrization-and-antisymmetrization-of-tensors

\begin{proof}
  1. Let us check that $\Symm(\T)$ is symmetric.
  Let $\epsilon \in \Perms(p)$.
  We have
  %
  \begin{eqnarray*}
    \Symm(\T)(x_{\epsilon(1)}) \cdots (x_{\epsilon(p)}) & = & {1 \over p!} \sum_{\sigma \in \Perms(p)} \T(x_{\sigma \circ \epsilon(1)})\cdots(x_{\sigma \circ \epsilon(p)}) \\
    & = & {1 \over p!} \sum_{\sigma' \in \Perms(p)} {\T}(x_{\sigma'(1)})\cdots(x_{\sigma'(p)}),
    \text{ with $\sigma' = \sigma \circ \epsilon$} \\
    & = & \Symm(\T)(x_1) \cdots (x_p).
  \end{eqnarray*}
  %
  If $\T$ is already symmetric,
  then for all $\sigma \in \Perms(p)$ $\T(x_{\sigma(1)})\cdots(x_{\sigma(p)}) = \T(x_1) \cdots (x_p)$,
  and since there are $p!$ permutations in $\Perms(p)$, $\Symm(\T)(x_1) \cdots (x_p) = \T(x_1) \cdots (x_p)$,
  that is,
  $\Symm(\T) = \T$.
  
  2. Let us check that $\Alt(\T)$ is antisymmetric.
  Let $\epsilon \in \Perms(p)$,
  we have
  \begin{eqnarray*}
    \Alt(\T)(x_{\epsilon(1)}) \cdots (x_{\epsilon(p)}) & = & {1 \over p!} \sum_{\sigma \in \Perms(p)} \sgn(\sigma) \times \T(x_{\sigma \circ \epsilon(1)})\cdots(x_{\sigma \circ \epsilon(p)}) \\
    & = & {1 \over p!} \sum_{\sigma' \in \Perms(p)} \sgn(\sigma' \circ \epsilon^{-1}) \times \T(x_{\sigma'(1)})\cdots(x_{\sigma'(p)}) \\
    & = & \sgn(\epsilon) \times \Alt(\T)(x_1) \cdots (x_p),
  \end{eqnarray*}
  where we denoted $\sigma' = \sigma \circ \epsilon$,
  and used $\sgn(\epsilon^{-1}) = \sgn(\epsilon)$.
  Now,
  if $\T$ is already antisymmetric,
  then for all $\sigma \in \Perms(p)$,
  $\T(x_{\sigma(1)})\cdots(x_{\sigma(p)}) = \sgn(\sigma) \times \T(x_1) \cdots (x_p)$.
  Thus,
  \begin{eqnarray*}
    \Alt(\T)(x_1) \cdots (x_p)  & = & {1 \over p!} \sum_{\sigma \in \Perms(p)} \sgn(\sigma) \times {\T}(x_{\sigma(1)})\cdots(x_{\sigma(p)}) \\
    & = & {1 \over p!} \sum_{\sigma \in \Perms(p)} \sgn(\sigma)^2 \times {\T}(x_1)\cdots(x_p) \\
    & = &  {1 \over p!} \sum_{\sigma \in \Perms(p)} {\T}(x_1)\cdots(x_p) \\
    & = & \T(x_1) \cdots (x_p).
  \end{eqnarray*}
  Therefore,
  $\Alt(\T) = \T$.
\end{proof}

\begin{article}\artlabel{Linear $p$-forms}
  \addcontentsline{toc}{section}{\small\hspace{10pt} Linear $p$-forms}
  \label{Linear-p-forms}
  The antisymmetric covariant tensors play an important role in differential geometry;
  they are at the basis of the Cartan calculus.
  It is why they got a special name and notation.
  Let $\E$ be a real vector space,
  a covariant antisymmetric $p$-tensor of $\E$ is called a {\em linear $p$-form}\index{Linear $p$-form} of $\E$.
  The vector space of linear $p$-forms of $\E$ is denoted by $\Lambda^p(\E)$.
  Note that $\Lambda^0(\E) = \RR$ and $\Lambda^1(\E) = \E^*$.
  
  \Note~On the other hand,
  the space of symmetric covariant $p$-tensors of $\E$ is sometimes denoted by $\V^p(\E)$.
  But symmetric covariant $p$-tensors did not get a special name,
  except in special cases,
  as for Euclidean scalar product for instance.
\end{article} %% Linear-p-forms

\begin{article}\artlabel{Inner product}
  \addcontentsline{toc}{section}{\small\hspace{10pt} Inner product}
  \label{Inner-product}
  Let $\E$ be a real vector space.
  Let $\A$ be a covariant $p$-tensor,
  $p \geq 1$.
  For any $x \in \E$,
  we denote by $\A(x)$ the covariant $(p-1)$-tensor defined by
  $$%
    [\A(x)](x_1)\cdots(x_{p-1}) = \A(x)(x_1) \cdots (x_{p-1}).
    $$%
  The tensor $\A(x)$ is called the {\em inner} (or {\em interior}) {\em product}\index{Inner/interior product} of $\A$ with $x$.
  The map $\A \mapsto \A(x)$,
  from $\Lambda^p(\E)$ to $\Lambda^{p-1}(\E)$,
  is a linear operator denoted by $\Int(x)$,
  that is,
  $\Int(x)(\A) = \A(x)$.
  Note that if $\A$ is antisymmetric or symmetric,
  that is,
  it changes or not its sign under a permutation of two vectors,
  then $\A(x)$ is still antisymmetric or symmetric.
  Then, the interior product is also an inner product inside the subspace of antisymmetric or symmetric multilinear covariant tensors.
\end{article} %% Inner-product

\begin{article}\artlabel{Exterior product}
  \addcontentsline{toc}{section}{\small\hspace{10pt} Exterior product}
  \label{Exterior-product}
  Let $\E$ be a real vector space.
  Let $\A\in \Lambda^p(\E)$ and $\B \in \Lambda^q(\E)$ \art{Linear-p-forms},
  the {\em exterior product}\index{Exterior product} $\A \wedge \B \in \Lambda^{p+q}(\E)$ is defined by
  $$%
    \A \wedge \B = {\fact{(p+q)} \over \fact{p} \fact{q}} \Alt(\A \otimes \B),
    $$%
  where $\Alt$ has been defined in \art{Symmetrization-and-antisymmetrization-of-tensors}.
  The exterior product with a $0$-form $s \in \RR$ is simply given by $s \wedge \A = \A \wedge s = s \times \A$.
  
  1. The exterior product is associative, for any three linear forms $\A,\B,\C$,
  $$%
    \A \wedge(\B \wedge \C) = (\A \wedge \B) \wedge \C.
    $$%
  
  2. The commutation of the exterior product is given by the rule,
  $$%
    \A \wedge \B = (-1)^{pq} \ \B \wedge \A.
    $$%
  
  3. For all $a \in \E^* = \Lambda^1(\E)$,
  the map $\B \mapsto a \wedge \B$,
  from $\Lambda^p(\E)$ to $\Lambda^{p+1}(\E)$,
  is a linear operator denoted by $\Ext(a)$,
  that is,
  $\Ext(a)(\B) = a \wedge \B$.
  The exterior product $a \wedge \B$ takes a special expression,
  \begin{align*}
    (a \wedge \B)(x)(x_1)\cdots(x_q) & = a(x) \times \B(x_1)(x_2) \cdots (x_q) \\
    & - a(x_1) \times \B(x)(x_2) \cdots (x_q) \\
    & - a(x_2) \times \B(x_1)(x)(x_3) \cdots (x_q) \\
    & - \cdots \\
    & - a(x_q) \times \B(x_1) \cdots (x_{q-1})(x).
  \end{align*}
  
  \Example~Let $a,b \in \E^*$,
  be two $1$-forms,
  then $a\, \wedge\, b = a\otimes b - b\otimes a$.
  Note that this is different from the antisymmetrization $\Alt(a \otimes b) = (1/2) \ a \wedge b$.
\end{article} %% Exterior-product

\begin{proof}
  1. Let $\C$ be of order $r$,
  and let $x_1,\ldots, x_{p+q+r} \in \E$.
  Let us denote $a.bc = \A\wedge(\B \wedge \C)(x_1)\cdots(x_{p+q+r})$ and $ab.c = (\A\wedge\B) \wedge \C(x_1)\cdots(x_{p+q+r})$.
  We have,
  \begin{eqnarray*}
    a.bc & = & {1 \over \fact{p} \fact{(q+r)}} \sum_{\sigma \in \Perms(p+q+r)} \sgn(\sigma) \times \A(x_{\sigma(1)})\cdots(x_{\sigma(p)}) \\
    & \times & {1 \over \fact{q} \fact{r}} \sum_{\epsilon \in \{\sigma(p+1) \cdots \sigma(p+q+r)\}!} \!\sgn(\epsilon)\! \times\! (\B \otimes \C) (x_{\epsilon \circ \sigma(p+1)}) \cdots (x_{\epsilon \circ \sigma(p+q+r)}) \\
    & = & {1 \over \fact{p} \fact{q} \fact{r}} \sum_{\sigma \in \Perms(p+q+r)} {1 \over \fact{(q+r)}} \sum_{\epsilon \in \{\sigma(p+1) \cdots \sigma(p+q+r)\}!} \sgn(\sigma) \times \sgn(\epsilon) \\
    & \times & (\A \otimes \B \otimes \C) (x_{\sigma(1)})\cdots(x_{\sigma(p)})(x_{\epsilon \circ \sigma(p+1)}) \cdots (x_{\epsilon \circ \sigma(p+q+r)}) \\
    & = &  {1 \over \fact{p} \fact{q} \fact{r}} \sum_{\bar\sigma \in \Perms(p+q+r)} {1 \over \fact{(q+r)}} \sum_{\epsilon \in \{\sigma(p+1) \cdots \sigma(p+q+r)\}!} \sgn(\bar \sigma) \\
    & \times & (\A \otimes \B \otimes \C)(x_{\bar\sigma(1)}) \cdots (x_{\bar\sigma(p+q+r)}) \\
    & = & {1 \over \fact{p} \fact{q} \fact{r}} \sum_{\bar\sigma \in \Perms(p+q+r)} \sgn(\bar\sigma) \times (\A \otimes \B \otimes \C)(x_{\bar\sigma(1)}) \cdots (x_{\bar\sigma(p+q+r)}),
  \end{eqnarray*}
  where $\{\sigma(p+1) \cdots \sigma(p+q+r)\}!$ denotes the set of permutations of the indices
  \linebreak
  $\sigma(p+1) \cdots \sigma(p+q+r)$,
  and where $\bar \sigma = \sigma \circ \bar \epsilon$,
  with $\bar \epsilon$ defined by $\bar \epsilon \restriction \{\sigma(1) \cdots \sigma(p)\} = \id$ and $\bar \epsilon \restriction \{\sigma(p+1) \cdots \sigma(p+q+r)\} = \epsilon$.
  So,
  except the fact that this last expression of $a.bc$ is clearly associative,
  a simple analogous computation gives the same expression for $ab.c$,
  thus the exterior product is associative.
  
  2.  Let $x_1,\ldots,x_{p+q} \in \E$,
  let us denote $ab = (\A \wedge \B)(x_1)\cdots(x_{p+q})$,
  and let
  \linebreak
  $ba = (\B \wedge \A)(x_1)\cdots(x_{p+q})$.
  We have
  \begin{eqnarray*}
    ab & = & {1 \over p!q!} \sum_{\sigma \in \Perms(p+q)} \sgn(\sigma) \times \A(x_{\sigma(1)})\cdots (x_{\sigma(p)}) \times \B(x_{\sigma(p+1)})\cdots (x_{\sigma(p+q)}) \\
    & = & {1 \over p!q!} \sum_{\sigma \in \Perms(p+q)} \sgn(\sigma) \times \B(x_{\sigma(p+1)})\cdots (x_{\sigma(p+q)}) \times \A(x_{\sigma(1)})\cdots (x_{\sigma(p)}) \\
    & = & {1 \over p!q!} \sum_{\sigma \in \Perms(p+q)} \sgn(\sigma) \times \B(x_{\sigma \circ \epsilon(1)})\cdots (x_{\sigma \circ \epsilon(q)}) \times \A(x_{\sigma \circ \epsilon (q+1)})\cdots (x_{\sigma \circ \epsilon(q + p)}),
  \end{eqnarray*}
  where $\epsilon$ is the permutation
  $$%
    \epsilon = \left\{
    \begin{array}{cccccccc}
    1 & 2 & \cdots & q & q+1 & q+2 & \cdots & q +p \\
    \downarrow & \downarrow  &   & \downarrow  & \downarrow  & \downarrow  &   & \downarrow \\
    p+1 & p+2 & \cdots & p+q & 1 & 2 & \cdots & p
    \end{array} \right. .
    $$%
  Thus,
  %
  \begin{align*}
    ab & =\ {1 \over p!q!} \sum_{\sigma' \in \Perms(p+q)} {\sgn(\sigma') \over \sgn(\epsilon)} \times \B(x_{\sigma'(1)})\cdots (x_{\sigma'(q)}) \times \A(x_{\sigma'(q+1)})\cdots (x_{\sigma'(q + p)}) \\
    & =\ {\sgn(\epsilon) \over p!q!}  \sum_{\sigma' \in \Perms(p+q)} \sgn(\sigma')\! \times\! \B(x_{\sigma'(1)})\cdots (x_{\sigma'(q)})\! \times\! \A(x_{\sigma'(q+1)})\cdots (x_{\sigma'(q + p)}) \\
    &  =\ \sgn(\epsilon) \times ba.
  \end{align*}
  Then,
  since the permutation $\epsilon$ is made of $pq$ successive transpositions,
  $\sgn(\epsilon) = (-1)^{pq}$.
  Hence,
  $\A \wedge \B = (-1)^{pq} \, \B \wedge \A$.
  
  3. To compute $a \wedge \B$,
  let $x_0,x_1,\ldots, x_q$ be $q+1$ vectors of $\E$,
  and let $\R =$
  \linebreak
  $(a \wedge \B)(x_0)(x_1)\cdots(x_q)$.
  Then, by definition:
  %
  \begin{eqnarray*}
    \R & = & {\fact{(p+q)} \over \fact{1} \fact{q}} {1 \over \fact{(p+q)}} \sum_{\sigma \in \Perms(1+q)} \sgn(\sigma) \times a(x_{\sigma(0)}) \times \B(x_{\sigma(1)})\cdots (x_{\sigma(q)}) \\
    & = & {1 \over \fact{q}} \sum_{\sigma | \sigma(0) = 0} \sgn(\sigma) \times a(x_0) \times \B(x_{\sigma(1)})\cdots (x_{\sigma(q)}) \\
    & + & {1 \over \fact{q}} \sum_{\sigma | \sigma(0) \neq 0} \sgn(\sigma) \times a(x_{\sigma(0)}) \times \B(x_{\sigma(1)})\cdots (x_{\sigma(q)}) \\
    & = & {1 \over \fact{q}} \sum_{\sigma | \sigma(0) = 0} \sgn(\sigma) \times a(x_0) \times \sgn(\sigma \restriction \{1 \cdots q \}) \times \B(x_1)\cdots (x_q) \\
    & + & {1 \over \fact{q}} \sum_{i=1}^q \sum_{\sigma | \sigma(0) = i} \sgn(\sigma) \times a(x_i) \times \B(x_{\sigma(1)})\cdots (x_{\sigma(q)}).
  \end{eqnarray*}
  But,
  on the one hand,
  we have $\sgn(\sigma) = \sgn(\sigma \restriction \{1 \cdots q \})$.
  On the other hand we can decompose every permutation $\sigma$ such that $\sigma(0) = i$ into the product of the following two permutations.
  %
  $$%
    \sigma = \left\{
    \begin{array}{ccccccr}
    0 & 1 & \cdots & i & \cdots & q  & \\
    \downarrow & \downarrow  &   & \downarrow  &  & \downarrow  & \bigg\} \ \tau \\
    i & 1 & \cdots & 0 & \cdots & q & \\
    \downarrow & \downarrow  &   & \downarrow  &  & \downarrow & \bigg\} \ \epsilon \\
    i & \sigma(1) & \cdots & \sigma(0) & \cdots & \sigma(q) & {}
    \end{array} \right.
    $$%
  %
  Thus,
  for each $\sigma$ and $i$ such that $\sigma(0) = i$,
  we have $\B(x_{\sigma(1)})\cdots (x_{\sigma(q)}) = \sgn(\epsilon) \times \B(x_1)\cdots(x_0)\cdots(x_q)$.
  Therefore,
  because $\sgn(\tau) = -1$,
  we get
  %
  \begin{eqnarray*}
    \R & = & {1 \over \fact{q}} \sum_{\sigma \in \Perms(q)} \sgn(\sigma) \times a(x_0) \times  \sgn(\sigma) \times \B(x_1)\cdots (x_q) \\
    & + & {1 \over \fact{q}} \sum_{i=1}^q \sum_{\epsilon \in \Perms(q)} \sgn(\epsilon) \times \sgn(\tau) \times a(x_i) \times \sgn(\epsilon) \times \B(x_1)\cdots (x_0) \cdots (x_q) \\
    & = & a(x_0) \times \B(x_1)\cdots(x_q) \\
    & - & \sum_{i=1}^q a(x_i) \times \B(x_1)\cdots(x_0)\cdots(x_q),
  \end{eqnarray*}
  %
  where,
  in the second term of the equalities,
  $x_0$ is at the rank $i$.
\end{proof}

\begin{article}\artlabel{Exterior monomials and basis of $\Lambda^p(\E)$}
  \addcontentsline{toc}{section}{\small\hspace{10pt} Exterior monomials and basis of $\Lambda^p(\E)$}
  \label{Exterior-monomials-and-basis-of-Lambda-p-E}
  Let $\E$ be a real vector space.
  Let $a _1, \ldots, a_p$ be $p$ covectors of $\E$.
  Let us define
  $$%
    a_1\wedge a_2 \wedge \cdots\wedge a_p = a_1 \wedge (a_2 \wedge (\cdots \wedge a_p)\cdots),
    $$%
  where the exterior product has been defined in \art{Exterior-product}.
  Such linear $p$-forms $a_1\wedge a_2 \wedge \cdots\wedge a_p$ are called {\em exterior monomials} of degree $p$.
  
  1. The  exterior monomial  $ a_1\wedge \cdots\wedge a_p$ is explicitly given by
  $$%
    a_1\wedge\cdots\wedge a_p = \sum_{\sigma \in \Perms(p)} \sgn(\sigma) \times a_{\sigma(1)}\otimes \cdots \otimes a_{\sigma(p)}.
    $$%
  
  2. The value of $a_1\wedge\cdots\wedge a_p$,
  applied to $v_1, \ldots, v_p \in \E$,
  is given by
  $$%
    (a_1\wedge\cdots\wedge a_p)(v_1)\cdots(v_p) = \sum_{\sigma \in \Perms(p)} \sgn(\sigma) \times a_1(v_{\sigma(1)}) \cdots a_p(v_{\sigma(p)}).
    $$%
  
  3. If one of the covectors of the family is a linear combination of the others,
  the exterior monomial $a_1\wedge\cdots\wedge a_p$ is zero.
  
  4. Let us assume that $\E$ is finite dimensional,
  with $\dim(\E) = n$,
  and let $\cB = (\ee_1,\cdots,\ee_n)$ be a basis.
  The  set of monomials of degree $p$
  $$%
    \{\ee^i\wedge \ee^j \wedge \cdots \wedge \ee^k\}, \text{ where } i < j < \cdots < k \in \{1, \cdots n\},
    $$%
  is a basis of the vector space $\Lambda^p(\E)$.
  In other words, for every linear $p$-form $\A$,
  there exists a family of numbers $\A_{ij\cdots k}$ such that
  $$%
    \renewcommand{\arraystretch}{1.5}
    \A = \sum_{i < j < \cdots < k } \A_{i j \cdots k}\ \ee^i\wedge \ee^j\wedge\cdots \wedge\ee^k,
    \text{ with }
    \A_{i j \cdots k} = \A(\ee_i)(\ee_j)\cdots(\ee_k).
    $$%
  The numbers $\A_{ij\cdots k}$,
  for all increasing sequences $i<j<\cdots<k$,
  where the indices are running from $1$ to $n$,
  are called the {\em components},
  or the {\em coordinates}\index{Coordinates},
  of the linear $p$-form $\A$ in the basis $\cB$.
  
  5. Let $\dim(\E) = n$.
  Since every element of a basis of $\Lambda^p(\E)$ is defined by a choice of $p$ different indices out of $n$,
  the dimension of $\Lambda^p(\E)$ is the Pascal binomial coefficient $\Cnp$,
  that is:
  $$%
    \dim(\Lambda^p(\E)) = {n! \over p! \ (n-p)!}.
    $$%
  Note that,
  thanks to this decomposition and to antisymmetry,
  $$%
    \text{if } p>\dim(\E), \text{ then } \Lambda^p(\E) = \{0\}.
    $$%
\end{article} %% Exterior-monomials-and-basis-of-Lambda-p-E

\begin{proof}
  1. We prove the first assertion by recurrence.
  Let us assume that the proposition is true for $a_1\wedge\cdots\wedge a_p$,
  then,
  \begin{eqnarray*}
    a_0\wedge (a_1\wedge\cdots\wedge a_p) & = & a_0\wedge \sum_{\sigma \in \Perms(p)} \sgn(\sigma) \times a_{\sigma(1)}\otimes \cdots \otimes a_{\sigma(p)} \\
    & = & \sum_{\sigma \in \Perms(p)} \sgn(\sigma) \times a_0 \wedge(a_{\sigma(1)}\otimes \cdots \otimes a_{\sigma(p)}) \\
    & = & \sum_{\sigma \in \Perms(p)} \sgn(\sigma) \times a_0 \otimes a_{\sigma(1)}\otimes \cdots \otimes a_{\sigma(p)} \\
    & - & \sum_{\sigma \in \Perms(p)} \sgn(\sigma) \times a_{\sigma(1)} \otimes a_0\otimes \cdots \otimes a_{\sigma(p)} \\
    & - & \cdots \\
    & - & \sum_{\sigma \in \Perms(p)} \sgn(\sigma) \times a_{\sigma(p)} \otimes a_{\sigma(1)}\otimes \cdots \otimes a_0 \\
    & = & \sum_{\sigma' \in \Perms(p+1)} \sgn(\sigma') \times a_{\sigma'(0)} \otimes a_{\sigma'(1)}\otimes \cdots \otimes a_{\sigma'(p)} \\
    & = & a_0\wedge a_1\wedge\cdots\wedge a_p.
  \end{eqnarray*}
  Now, the proposition being true for $p = 1$,
  it is true for any $p$.
  
  2. Let us make the change of variable,
  $\sigma$ into $\sigma^{-1}$,
  which does not change the sum,
  and since the products involved in the sum are independent of the ordering,
  we get the result:
  \begin{eqnarray*}
    (a_1\wedge\cdots\wedge a_p)(v_1)\cdots(v_p) & = & \sum_{\sigma \in \Perms(p)} \sgn(\sigma) \times a_{\sigma(1)}(v_1) \cdots a_{\sigma(p)}(v_p) \\
    & = & \sum_{\sigma \in \Perms(p)} \sgn(\sigma^{-1}) \times a_{1}(v_{\sigma(1)}) \cdots a_{p}(v_{\sigma(p)}) \\
    & = & \sum_{\sigma \in \Perms(p)} \sgn(\sigma) \times a_{1}(v_{\sigma(1)}) \cdots a_{p}(v_{\sigma(p)}).
  \end{eqnarray*}
  
  3. Let us first assume that $a_i = a_j = a $.
  By 2 we have
  \begin{eqnarray*}
    (a_1\wedge\cdots\wedge a_p)(v_1)\cdots(v_p) & = & \sum_{\sigma \in \Perms(p)} \sgn(\sigma) \times a_1(v_{\sigma(1)}) \cdots a_p(v_{\sigma(p)}) \\
    & = & \sum_{\sigma \in \Perms(p)} \sgn(\sigma) \times a(v_{\sigma(i)}) a(v_{\sigma(j)}) \times (\star),
  \end{eqnarray*}
  where $(\star)$ contains terms involving all the indices different from $i$ and $j$.
  Thus,
  denoting by $\varepsilon$ the transposition $i \leftrightarrow j$,
  we have
  \begin{eqnarray*}
    (a_1\wedge\cdots\wedge a_p)(v_1)\cdots(v_p) & = & \sum_{\sigma \in \Perms(p)} \sgn(\sigma) \times a(v_{\sigma(i)}) a(v_{\sigma(j)}) \times (\star) \\
    & = & \undemi \bigg[ \sum_{\sigma \in \Perms(p)} \sgn(\sigma) \times a(v_{\sigma(i)}) a(v_{\sigma(j)}) \times (\star) \\
    & + & \phantom{{1 \over 2} \bigg[}  \sum_{\sigma \in \Perms(p)} \sgn(\sigma) \times a(v_{\sigma(j)}) a(v_{\sigma(i)}) \times (\star)\bigg] \\
    & = & \undemi \bigg[ \sum_{\sigma \in \Perms(p)} \sgn(\sigma) \times a(v_{\sigma(i)}) a(v_{\sigma(j)} \times (\star) \\
    & + & \phantom{{1 \over 2} \bigg[}\sum_{\sigma \in \Perms(p)} \sgn(\sigma) \times a(v_{\sigma\varepsilon(i)}) a(v_{\sigma\varepsilon(j)}) \times (\star)\bigg].
  \end{eqnarray*}
  Let $\sigma' = \sigma \varepsilon$.
  Since $\varepsilon$ acts trivially on $(\star)$,
  and since $\sgn(\sigma') = \sgn(\sigma) \times \sgn(\varepsilon) = \sgn(\sigma)  \times (-1) = - \sgn(\sigma)$,
  we get
  \begin{eqnarray*}
    (a_1\wedge\cdots\wedge a_p)(v_1)\cdots(v_p) & = & \undemi \bigg[\sum_{\sigma \in \Perms(p)} \sgn(\sigma) \times a(v_{\sigma(i)}) a(v_{\sigma(j)} \times (\star) \\
    & - & \phantom{{1 \over 2} \bigg[}  \sum_{\sigma' \in \Perms(p)} \sgn(\sigma') \times a(v_{\sigma'(i)}) a(v_{\sigma'(j)}) \times (\star)\bigg] \\
    & = & 0.
  \end{eqnarray*}
  We then get the result by linearity.
  
  4. Let us consider a tensor $\A$.
  In the basis $\cB$,
  we get from \art{Components-of-a-tensor}
  $$%
    \A = \sum_{i,j,\cdots,k = 1}^n \A_{i j \cdots k}\ \ee^i\otimes \ee^j\cdots \otimes \ee^k, \text{ where } \A_{i j \cdots k} =  \A(\ee_i)(\ee_j)\cdots(\ee_k).
    $$%
  Moreover,
  the components of the tensor $\A$ in the basis $\cB$ are totally antisymmetric \art{Components-of-a-tensor},
  that is,
  $$%
    \text{for all } \sigma \in \Perms(p), \ \A_{\sigma(i) \sigma(j) \cdots \sigma(k)} = \sgn(\sigma) \times \A_{i j \cdots k}.
    $$%
  Thus, the form $\A$ writes:
  \begin{eqnarray*}
    \A & = & \sum_{i < j < \cdots < k = 1}^n \A_{i j \cdots k} \sum_{\sigma \in \Perms(p)}\sgn(\sigma) \times \ee^{\sigma(i)}\otimes \ee^{\sigma(j)}\cdots \otimes\ee^{\sigma(k)} \\
    & = & \sum_{i < j < \cdots < k = 1}^n \A_{i j \cdots k} \ \ee^i\wedge \ee^j\wedge\cdots \wedge \ee^k.
  \end{eqnarray*}
  Hence,
  the set of monomials $\ee^i\wedge \ee^j\wedge\cdots \wedge \ee^k$,
  where $i < j < \cdots < k$ run from 1 to $n$,
  is a basis of the vector space $\Lambda^p(\E)$.
  The numbers $\A_{i j \cdots k}$,
  where $i < j < \cdots < k$,
  are the {\em components}, or the {\em coordinates}\index{Coordinates},
  of the form $\A$ in the basis $\cB$.
  
  5. If $p > \dim(\E)$,
  then each monomial of the basis will contain necessarily two repeated terms,
  thus all the monomials of the basis are zero,
  thanks to 3.
  Hence,
  $\Lambda^p(\E) = 0$.
\end{proof}

\begin{article}\artlabel{Pullbacks of tensors and forms}
  \addcontentsline{toc}{section}{\small\hspace{10pt} Pullbacks of tensors and forms}
  \label{Pullbacks-of-tensors-and-forms}
  Let $\E$ and $\F$ be two real vector spaces,
  let ${\M}: \E \to \F$ be a linear map, ${\M} \in \E*\F = \Lin(\E,\F)$.
  Let $\A$ be a covariant $p$-tensor of $\F$ \art{Tensors}.
  The {\em pullback} of $\A$ by ${\M}$ is the covariant $p$-tensor of $\E$ defined by
  $$%
    {\M}^*(\A)(v_1)\cdots(v_p) = \A({\M}(v_1))\cdots({\M}(v_p)),
    \text{ where } v_1,\ldots, v_p \in \E.
    $$%
  Here are some of the most important properties of the pullback operation.
  The pullback is a linear operator,
  for any pair of covariant $p$-tensors $\A$ and ${\B}$,
  and for any real $\lambda$
  $$%
    \M^*(\A + \B) = \M^*(\A) + \M^*(\B), \text{ and } \M^*(\lambda \times \A) = \lambda \times \M^*(\A).
    $$%
  The pullback is contravariant,
  let $\G$ be a third vector space,
  $\N: \F \to \G$ be a linear map,
  $\A$ a $p$-tensor on $\G$,
  then
  $$%
    ({\N} \circ {\M})^* = {\M}^* \circ {\N}^*, \ \text{that is,} \ ({\N} \circ {\M})^*(\A) = {\M}^*({\N}^*(\A)).
    $$%
  The pullback clearly respects the tensor product of covariant tensors \art{Tensor-products}.
  Now, if $\A$ is a $p$-form,
  that is,
  $\A \in \Lambda^p(\F)$ \art{Linear-p-forms},
  then its pullback by $\M$ also is a $p$-form.
  The pullback respects the exterior product too,
  that is,
  for any pair of linear forms $\A$ and ${\B}$,
  we have
  $$%
    {\M}^*(\A\wedge {\B}) = {\M}^*(\A)\wedge {\M}^*({\B}).
    $$%
  In other words,
  the pullback is a morphism for the exterior algebra of linear forms.
\end{article} %% Pullbacks-of-tensors-and-forms

\begin{article}\artlabel{Coordinates of the pullback of linear forms}
  \addcontentsline{toc}{section}{\small\hspace{10pt} Coordinates of the pullback of forms}
  \label{Coordinates-of-the-pullback-of-linear-forms}
  Let $\E$ and $\F$ be two real vector spaces of dimensions $n$ and $m$.
  Let $\A$ be a linear $p$-form of $\F$.
  Let $\cE = (\ee_1,\ldots,\ee_n)$ be a basis of $\E$ and $\cF = (\ff_1,\ldots,\ff_m)$ be a basis of $\F$.
  Let $\A_{i \cdots k}$ be the coordinates of $\A$ in the basis $\cF$ \art{Exterior-monomials-and-basis-of-Lambda-p-E},
  $$%
    \A = \sum_{i<\cdots<j} \A_{i \cdots j} \times \ff\,{}^i\wedge \cdots \wedge \ff\,{}^j.
    $$%
  Then,
  the coordinates of $\M^*(\A)$ in the basis $\cE$ are given by
  $$%
    [\M^*(\A)]_{k\cdots \ell} = \sum_{i<\cdots< j} \ \sum_{\sigma \in \{i \cdots j\}!} \sgn(\sigma) \times {\M}_k^{\sigma(i)}\cdots {\M}_\ell^{\sigma(j)} \A_{i \cdots j},
    \text{ with } {\M}_k^i =  \ff\,{}^i({\M} \ee_k),
    $$%
  where $\{i \cdots j\}!$ denotes the group of permutations of the set of indices $\{i \cdots j\}$.
\end{article} %% Coordinates-of-the-pullback-of-linear-forms

\begin{proof}
  By definition $[{\M}^*(\A)]_{k\cdots \ell} = {\M}^*(\A)(\ee_k)\cdots(\ee_\ell) = \A({\M}\ee_k)\cdots({\M}\ee_\ell)$,
  but ${\M}\ee_k = \sum_i {\M}^i_k \ff_i$.
  Thus,
  \begin{eqnarray*}
    [{\M}^*(\A)]_{k\cdots \ell} & = & \A \bigg(\sum_i {\M}^i_k \ff_i\bigg)\cdots\bigg(\sum_j {\M}^j_\ell \ff_j\bigg) \\
    & = & \sum_i \cdots \sum_j {\M}^i_k \cdots {\M}^j_\ell \A(\ff_i)\cdots \A(\ff_j) \\
    & = & \sum_i \cdots \sum_j {\M}^i_k \cdots {\M}^j_\ell \A_{i \cdots j} \\
    & = & \sum_{i<\cdots<j} \ \sum_{\sigma \in \{i \cdots j\}!} \sgn(\sigma) \times {\M}_k^{\sigma(i)}\cdots {\M}_\ell^{\sigma(j)} \A_{i \cdots j},
  \end{eqnarray*}
  since $\A_{i \cdots j} = \sgn(\sigma) \times \A_{\sigma(i)\cdots\sigma(j)}$ \art{Exterior-monomials-and-basis-of-Lambda-p-E}.
\end{proof}

\begin{article}\artlabel{Volumes and determinants}
  \addcontentsline{toc}{section}{\small\hspace{10pt} Volumes and determinants}
  \label{Volumes-and-determinants}
  Let $\E$ be a real vector space of dimension $n$.
  The space  $\Lambda^n(\E)$ of linear $n$-forms of $\E$ has dimension 1.
  Thanks to the decomposition of forms on any basis $\cB = (\ee_1 , \ldots , \ee_n)$ \art{Exterior-monomials-and-basis-of-Lambda-p-E},
  every linear $n$-form of $\E$ is proportional to $\vol_\cB$,
  defined by
  %
  \begin{equation}
    \renewcommand{\theequation}{$\diamondsuit$}
    \vol_\cB = \ee^1\wedge\ee^2\cdots\wedge\ee^n, \text{ and } \Lambda^n(\E) =  \RR\times \vol_\cB.
  \end{equation}
  %
  We call {\em volume}\index{Volume} any nonzero form belonging to $\Lambda^n(\E)$.
  We shall denote the set of all the volumes of $\E$ by $\Volumes(\E)$,
  $$%
    \Volumes(\E) = \{\vol \in \Lambda^n(\E) \mid \vol \neq 0 \}.
    $$%
  For $\E = \RR^n$,
  the canonical basis \art{Basis-and-linear-groups} defines by $(\diamondsuit)$ a {\em canonical volume} denoted by $\vol_n$.
  Now,
  let ${\M}$ be a linear map from $\E$ to $\E$,
  and let ${\M}^*(\vol)$ be the pullback \art{Pullbacks-of-tensors-and-forms} of some volume $\vol \in \Lambda^n(\E)$ by ${\M}$.
  Let us recall that,
  by definition,
  for all $v_1,\ldots, v_n \in \E$,
  ${\M}^*(\vol) (v_1)\cdots(v_n) = \vol(\M v_1)\cdots(\M v_n)$.
  Since the space $\Lambda^n(\E)$ is 1-dimensional and $\vol \neq 0$,
  the $n$-form ${\M}^*(\vol)$ is proportional to $\vol$.
  The coefficient of proportionality is independent on the choice of the volume,
  it is called the {\em determinant} of the linear map ${\M}$,
  and it is denoted by $\det({\M})$,
  \begin{equation}
    \renewcommand{\theequation}{$\clubsuit$}
    {\M}^*(\vol) = \det({\M}) \times \vol, \text{ or } \det({\M}) = {{\M}^*(\vol) \over \vol}.
  \end{equation}
  So,
  for every family of vectors $v_1,\ldots,v_n \in \E$,
  $$%
    \vol(\M v_1)\cdots(\M v_n) = \det(\M) \times \vol(v_1)\cdots(v_n).
    $$%
  Actually,
  the computation of $\det({\M})$ can be done by evaluating the previous identity on the vectors of some basis $\cB = (\ee_1, \ldots, \ee_n)$ of $\E$,
  that is,
  \begin{equation}
    \renewcommand{\theequation}{$\heartsuit$}
    \det({\M}) = \vol_\cB({\M}\ee_1)\cdots({\M}\ee_n),\text{ with } \vol_\cB = \ee^1\wedge\ee^2\cdots\wedge\ee^n.
  \end{equation}
  Now,
  let ${\M}_1 = {\M}(\ee_1), \ldots, {\M}_n = {\M}(\ee_n)$ be the images of the vectors of the basis $\cB$ by the linear map $\M$.
  The determinant of $\M$ is given,
  in terms of the matrix coefficients $\M_i^j$ of $\M$ in the basis $\cB$,
  by
  \begin{equation}
    \renewcommand{\theequation}{$\spadesuit$}
    \det({\M}) =
    \sum_{\sigma \in \Perms(n)} \sgn(\sigma) \times {\M}_1^{\sigma(1)} \times \cdots \times {\M}_n^{\sigma(n)}, \text{ with } \M_i^j = \ee^j(\M\ee_i).
  \end{equation}
  
  The choice of a basis $\cB$ of $\E$ defines an {\em orientation},
  that is,
  the set of the bases $\cB'$ such that $\det(\cB'^{-1} \cB) = 1$.
  With what precedes,
  it is equivalent to choose an orientation or to choose a volume for the vector space $\E$.
  Finally,
  by direct application of the definition (\exref{Determinant-of-a-product}) we can establish the usual representation formulas:
  $$%
    \det(\M\N) = \det({\M}) \times \det({\N}), \text{ and } \det(s\times {\M}) = s^n \times \det({\M}),
    $$%
  where $\M$, $\N$ are two linear maps from $\RR^n$ to $\RR^n$, and $s \in \RR$.
\end{article} %% Volumes-and-determinants

\begin{proof}
  Let us  check first that the determinant of a
  linear map $\M$ does not depend on the choice of the
  volume. Let $\vol'$ be another volume. Thus, $\vol' = c
  \times \vol$, where $c \neq 0$. Now, $\M^*(\vol') =
  \M^*(c \times \vol) = c \times \M^*(\vol) = c
  \times \det(\M) \times \vol = \det(\M) \times \vol'$,
  that is, $\M^*(\vol') =  \det(\M) \times \vol'$.
  Now, let us prove the formula $(\spadesuit)$. On the one
  hand, by application of the definition of the exterior
  product
  \art{Exterior-monomials-and-basis-of-Lambda-p-E}, we get
  $ \vol_\cB({\M}_1)\cdots({\M}_n) =
  \sum_{\sigma \in \Perms(n)} \sgn(\sigma) \times
  {\M}_1^{\sigma(1)} \times \cdots \times
  {\M}_n^{\sigma(n)}$, where the ${\M}^j_i =
  \ee^j({\M}_i) = \ee^j(\M \ee_i)$, $j = 1 \cdots n$, are
  the coordinates of the vector ${\M}_i$ in the basis
  $\cB$. On the other hand, $\vol_\cB(\M\ee_1) \cdots
  (\M \ee_n) =  \det(\M) \times \vol_\cB(\ee_1) \cdots
  (\ee_n)$. But $\vol_\cB(\ee_1)\cdots(\ee_n)
  = (\ee^1\wedge\ee^2\cdots\wedge\ee^n)(\ee_1)\cdots(\ee_n) =
  1$. So,  $\det({\M}) = \sum_{\sigma \in \Perms(n)}
  \sgn(\sigma) \times {\M}_1^{\sigma(1)} \times \cdots \times
  {\M}_n^{\sigma(n)}$.
\end{proof}

%%%%%%%%%%%%%%%%%%%%%%%%%%%%%%%%%%%%%%%%%%%%%%%%%%%%%%%%%%
%
%   Exercises
%
%%%%%%%%%%%%%%%%%%%%%%%%%%%%%%%%%%%%%%%%%%%%%%%%%%%%%%%%%%

\Exercises

\begin{exercise}[Antisymmetric $3$-form]
  \label{Antisymmetric-3-form}
  Let $\A$ be a  $3$-form on a finite dimensional vector space.
  Show that an antisymmetric tensor $\A$ is zero if and only if,
  for some basis,
  its coordinates satisfy $ \A_{ijk} + \A_{jki} + \A_{kij} = 0$,
  for every triple of indices $i,j,k$.
\end{exercise} %% Antisymmetric-3-form

\begin{exercise}[Expanding the exterior product]
  \label{Expanding-the-exterior-product}
  Let $\E$ be a vector space.
  Let $a$,
  $b$ and $c$ be three $1$-forms of $\E$.
  Expand the evaluation of the triple product $\Ext(a)(\Ext(b)(c))$ on the triple of vectors $(x_1)(x_2)(x_3)$,
  where $\Ext$ is defined in \art{Exterior-product}.
  Check that
  $$%
    [\Ext(a)(\Ext(b)(c))](x_1)(x_2)(x_3) = \sum_{\sigma \in \Perms(3)}\sgn(\sigma) a(x_{\sigma(1)}) b(x_{\sigma(2)}) c(x_{\sigma(3)}).
    $$%
\end{exercise} %% Expanding-the-exterior-product

\begin{exercise}[Determinant and isomorphisms]
  \label{Determinant-and-isomorphisms}
  Let $\E$ be an $n$-dimensional vector space,
  and let $\vol$ be a volume.
  Check that,
  for $n$ vectors $v_i \!\in\! \E$,
  $\vol(v_1)\cdots(v_n) \!= 0$ if and only if the $v_i$ are not linearly independent.
  Deduce that the determinant of  $\M \in \Lin(\E)$ is not zero if and only if $\ker(\M) = \set{0}$,
  that is,
  if and only if $\M$ is an isomorphism.
\end{exercise} %% Determinant-and-isomorphisms

\begin{exercise}[Determinant is smooth]
  \label{Determinant-is-smooth}
  Check that the determinant is a smooth map from $\Lin(\E)$ to $\RR$.
  Let $\P : \U \to \GL(\E)$ be any $m$-plot.
  Let $r \in \U$, $\M = \P(r)$ and $\delta r$ be any vector of $\RR^m$.
  Let $\delta \M$ be the variation $\delta \M = \D(\P)(r)(\delta r)$,
  and let $\delta[\det(\M)] = \D(r \mapsto \det(\P(r))(r)(\delta r)$ be the variation of the determinant.
  Show that
  $$%
    \delta[\det(\M)] = \det(\M) \times \Tr(\M^{-1} \delta \M),
    $$%
  where $\Tr$ is the trace operator on matrices.
\end{exercise} %% Determinant-is-smooth

\begin{exercise}[Determinant of a product]
  \label{Determinant-of-a-product}
  Show that for every pair $\M$,
  $\N$,
  of elements of $\Lin(\E)$,
  where $\E$ is a finite dimensional vector space,
  we have $\det(\M \N) = \det(\M) \times \det(\N)$.
  Deduce that $\det(s \times \M) = s^n \times \det(\M)$,
  with $s \in \NN$.
\end{exercise} %% Determinant-of-a-product

\newpage

%%%%%%%%%%%%%%%%%%%%%%%%%%%%%%%%%%%%%%%%%%%%%%%%%%%%%%%%%%
%% MARK: Smooth Forms on Real Domains
%%%%%%%%%%%%%%%%%%%%%%%%%%%%%%%%%%%%%%%%%%%%%%%%%%%%%%%%%%

\section*{Smooth Forms on Real Domains}
\label{sec-Smooth-forms-on-numerical-domains}

\begin{sechead}
  This section introduces the notion of smooth forms defined on real domains,
  and most of the related material.
  Then,
  this definition and associated constructions will be naturally extended to diffeological spaces,
  in order to define the Cartan-De Rham calculus in the larger context of diffeology,
  \art{Differential-forms-on-diffeological-spaces} and after.
\end{sechead}

\begin{article}\artlabel{Smooth forms on real domains}
  \addcontentsline{toc}{section}{\small\hspace{10pt} Smooth forms on numerical domains}
  \label{Smooth-forms-on-numerical-domains}
  Let $\U \subset \RR^n$ be a real domain \art{Real-vector-spaces-and-domains}.
  Equip $\Lambda^p(\RR^n)$ with its smooth structure of finite dimensional vector space \art{Exterior-monomials-and-basis-of-Lambda-p-E}.
  A {\em smooth $p$-form}\index{Smooth $p$-form} on $\U$ is any smooth map $\omega: \U \to \Lambda^p(\RR^n)$.
  The set of all the smooth $p$-forms of $\U$,
  $$%
    \Cinfty(\U,\Lambda^p(\RR^n)),
    $$%
  is naturally a real vector space.
  Let $\omega, \omega' \in \Cinfty(\U,\Form^p(\RR^n))$ and $s \in \RR$.
  The sum $\omega + \omega'$ and the product $s\omega$  are given by,
  $$%
    \text{for all } x \in \U:
    \ \left\{
    \begin{array}{rcl}
    (\omega + \omega')(x) & = &\omega(x) + \omega'(x), \\
    (s\times \omega)(x) & = & s\times (\omega(x)).
    \end{array}
    \right.
    $$%
  For example,
  any form $\A \in \Form^p(\RR^n)$ defines the constant smooth $p$-form $x \mapsto \A$ on $\U$.
  Thus,
  $\Lambda^p(\RR^n)$ is naturally a subspace of the space $\Cinfty(\U,\Form^p(\RR^n))$ of smooth $p$-forms on $\U$.
  
  \Note~More generally,
  {\em smooth tensors}\index{Smooth tensor} on $\U$ are defined the same way,
  as smooth maps defined on $\U$ with values in some fixed space of linear tensors \art{Tensors}.
  But the smooth tensors which may have a role in diffeology are the {\em covariant smooth tensors},
  since they can be transported by pullback;
  for example \art{Differential-forms-on-diffeological-spaces}.
\end{article} %% Smooth-forms-on-numerical-domains

\begin{article}\artlabel{Components of smooth forms}
  \addcontentsline{toc}{section}{\small\hspace{10pt} Components of smooth forms}
  \label{Components-of-smooth-forms}
  Let $\cB = (\ee_1,\ldots,\ee_n)$ be the canonical basis of $\RR^n$ \art{Basis-and-linear-groups}.
  So,
  for every $x \in \U$,
  for every smooth $p$-form $\omega$,
  the $p$-form $\omega(x)$ breaks down into components over the basis $\{\ee^i\wedge \ee^j \wedge \cdots\wedge\ee^k\}_{i < j < \cdots < k}$ \art{Exterior-monomials-and-basis-of-Lambda-p-E},
  $$%
    \omega(x) = \sum_{i < j <\cdots< k} \omega_{i j \cdots k} (x) \times \ee^i\wedge \ee^j \wedge \cdots\wedge\ee^k.
    $$%
  Since $\omega$ is a smooth map,
  the $\Cnp$ parametrizations $\omega_{i j \cdots k}: \U \to \RR$ are smooth.
  They are still called the {\em components of the smooth form $\omega$ in the basis $\cB$},
  even if they are not anymore constant but smooth real parametrizations.
  Conversely, $\Cnp$ smooth real parametrizations $\{\omega_{i j \cdots k} \}_{i < j< \cdots < k}$,
  defined on $\U$,
  define by the expression above a unique smooth $p$-form $\omega$ of $\U$.
  
  \textsc{Notation.} We shall sometimes denote by $\omega_x$ or by $\omega(x)$ the value of the smooth form $\omega$,
  at the point $x$.
\end{article} %% Components-of-smooth-forms

\begin{article}\artlabel{Pullbacks of smooth forms}
  \addcontentsline{toc}{section}{\small\hspace{10pt} Pullbacks of smooth forms}
  \label{Pullbacks-of-smooth-forms}
  Let $\U \subset \RR^m$ and $\V \subset \RR^n$ be two real domains.
  Let $\omega \in \Cinfty(\V,\Form^p(\RR^n))$,
  and let $f: \U \to \V$ be a smooth parametrization.
  The {\em pullback} of $\omega$ by $f$,
  denoted by
  $f^*(\omega)$,
  is defined by
  $$%
    f^*(\omega)(u)(x_1)\cdots(x_p) = \omega(f(u))(\D(f)(u)(x_1))\cdots (\D(f)(u)(x_p)),
    $$%
  for every $u \in \U$, and every $x_1,\ldots,x_p \in \RR^n$.
  We have denoted by $\D(f)(u)$ the tangent map of $f$ at the point $u$ \art{Smooth-parametrizations-in-domains}.
  Now,
  since components of the pullback $f^*(\omega)$ are linear combinations of the components of $\omega$ with the components of the linear map $\D(f)$,
  which are also smooth functions,
  the pullback $f^*(\omega)$ is a smooth $p$-form of $\U$,
  that is, $f^*(\omega) \in \Cinfty(\U,\Form^p(\RR^m))$. Note that,
  by definition \art{Pullbacks-of-tensors-and-forms},
  $$%
    f^*(\omega)(u) = (\D(f)(u))^*(\omega(f(u))).
    $$%
  Next,
  since tangent  maps satisfy the chain-rule property,
  $$%
    \D(g \circ f)(u) = \D(g)(f(u)) \circ \D(f)(u),
    $$%
  and thanks to the composition of pullbacks of linear forms \art{Pullbacks-of-tensors-and-forms},
  we get the contravariant property of the pullback.
  Let $\U$, $\V$ and $\W$ be three real domains.
  Let $f: \U \to \V$ and $g: \V \to \W$ be two smooth maps,
  for any smooth $p$-form  $\omega$ on $\W$,
  $$%
    (g \circ f)^*(\omega) = f^*(g^*(\omega)).
    $$%
  
  \Note~The definition of pullback of smooth forms,
  and its contravariant property,
  applies to general covariant smooth tensors,
  and not just to forms.
\end{article} %% Pullbacks-of-smooth-forms

\begin{proof}
  We need only check the last sentence.
  Let $u \in \U$ and $v = f(u)$.
  We have
  \begin{align*}
    (g \circ f)^*(\omega)(u) & = (\D(g \circ f)(u))^*(\omega(g \circ f(u))) \\
    & = [\D(g)(f(u)) \circ \D(f)(u)]^*[\omega(g(f(u)))] \\
    & = \D(f)(u)^*[\D(g)(v)^*(\omega(g(v))] \\
    & = \D(f)(u)^*[g^*(\omega)(v)] \\
    & = f^*(g^*(\omega))(u).
  \end{align*}
  Thus,
  $ (g \circ f)^*(\omega) = f^*(g^*(\omega))$.
\end{proof}

\begin{article}\artlabel{The differential of a function as pullback}
  \addcontentsline{toc}{section}{\small\hspace{10pt} The differential of a function as pullback}
  \label{The-differential-of-a-function-as-pullback}
  As an example of pullback of form \art{Pullbacks-of-smooth-forms},
  the pullback of the volume form of $\RR$ is interesting.
  Let us consider the canonical
  constant $1$-form $\theta$ on $\RR$\,:
  $$%
    \Lambda^1(\RR) = \RR\,\theta, \text{ with } \theta(t) = \id_\RR, \text{ \ie\ } \theta_t(\delta t) = \delta t, \  \forall \delta t \in \RR.
    $$%
  Let $f: \U \to \RR$ be an smooth $n$-parametrization.
  Let $u \in \U$ and $x \in \RR^n$.
  Then, \begin{align*}
    f^*(\theta)_u(x) & = \theta_{f(x)}(\D(f)(u)(x)) \\
    & =  \D(f)(u)(x) \\
    & =  df_u(x).
  \end{align*}
  Hence,
  $$%
    f^*(\theta) = df
    $$%
  is the {\em differential} of the function $f$.
  Applied to $\RR$ itself,
  the form $\theta$ is the differential of the identity map $\id_\RR = [t \mapsto t]$,
  and the form $\theta$ is also denoted by $dt$,
  which is summarized by
  $$%
    f^*(dt) = df, \text{ where } \Lambda^1(\RR) = \RR\,dt.
    $$%
  Note that this notation,
  which is the standard notation,
  is not very coherent.
  The differential $dt$ should be denoted by $d\,[t \mapsto t]$ or $d\,\id_\RR$,
  and then $df = f^*(d\,\id_\RR)$,
  which is the coherent notation.
  Otherwise,
  we have to interpret differently the symbolic construction $dt$.
  In the same way,
  let us consider $\RR^n$,
  and the
  projections {\em coordinates\/}\index{Coordinates}:
  $$%
    \xx^k: \RR^n \to \RR, \text{ such that } \xx^k: x \mapsto  x^k.
    $$%
  For all points $x \in \RR^n$ and all vectors $u \in \RR^n$,
  the differential of the function $\xx^k$,
  that is,
  $d\xx^k = (\xx^k)^*(\theta)$,
  satisfies
  $$%
    d\xx^k_x(u) = d\xx^k_x\bigg(\sum_{i = 1}^n u^i \ee^i\bigg) = u^k = \ee^k(u), \text{ which implies } d\xx^k = \ee^k.
    $$%
  Thus,
  any exterior monomial writes also
  $$%
    \ee^i\wedge\ee^j\wedge\cdots\wedge \ee^k = d\xx^i\wedge d\xx^j \wedge \cdots \wedge d\xx^k.
    $$%
  By abuse of notation,
  or for some better  reason,
  the monomial $d\xx^i\wedge d\xx^j \wedge \cdots \wedge d\xx^k$ is also simply written $dx^i\wedge dx^j \wedge \cdots \wedge dx^k$.
\end{article} %% The-differential-of-a-function-as-pullback

\begin{article}\artlabel{Exterior derivative of smooth forms}
  \addcontentsline{toc}{section}{\small\hspace{10pt} Exterior derivative of smooth forms}
  \label{Exterior-derivative-of-smooth-forms}
  There exists an operation called {\em exterior differentiation},
  denoted by the letter $d$ and defined by the following properties:
  
  1. For all integers $n$ and all $n$-domains $\U$,
  $$%
    d : \Cinfty(\U,\Form^p(\RR^n)) \to \Cinfty(\U,\Form^{p+1}(\RR^n)).
    $$%
  
  2. If $p = 0$,
  for every smooth real parametrization $f: x \mapsto a$,
  $$%
    df(x) =  \sum_{\ell = 1}^n {\Der a \over \Der x^\ell}\, \ee^\ell = \sum_{\ell = 1}^n {\Der a \over \Der x^\ell}\, dx^\ell.
    $$%
  
  3. If $\alpha$ is monomial,
  $\alpha : x \mapsto a\, \ee^i\wedge \cdots \wedge \ee^k$ with $[x \mapsto a] \in \Cinfty(\U,\RR)$,
  then
  $$%
    (d\alpha)(x) = \sum_{\ell = 1}^n {\Der a \over \Der x^\ell}\, \ee^\ell\wedge \ee^i\wedge \cdots \wedge \ee^k = \sum_{\ell = 1}^n {\Der a \over \Der x^\ell}\, dx^\ell\wedge dx^i\wedge \cdots \wedge dx^k.
    $$%
  
  4. The operator $d$ is extended by linearity to any smooth $p$-form $\omega$,
  $$%
    \omega = \sum_{i<\cdots<k} \omega_{i \cdots k}\, \ee^i\wedge \cdots \wedge \ee^k \  \Rightarrow \  d\omega =  \sum_{i<\cdots<k} d[\omega_{i \cdots k}\,\ee^i\wedge \cdots \wedge \ee^k].
    $$%
  The  smooth form $d\omega$ is called the {\em exterior derivative}\index{Exterior derivative} of $\omega$.
  For example,
  the exterior derivative of a $1$-form $\alpha = \sum_{i = 1}^n a_i\, dx^i$,
  defined on some domain $\U$, is given by
  $$%
    d\alpha = \sum_{k = 1}^n\sum_{i = 1}^n {\Der a_k \over \Der x^i}\ dx^i\wedge dx^k = \sum_{1\leq i<k\leq n} \bigg({\Der a_k \over \Der x^i} - {\Der a_i \over \Der x^k} \bigg)\ dx^i\wedge dx^k.
    $$%
\end{article} %% Exterior-derivative-of-smooth-forms

\begin{article}\artlabel{Exterior derivative commutes with pullback}
  \addcontentsline{toc}{section}{\small\hspace{10pt} Exterior derivative commutes with pullback}
  \label{Exterior-derivative-commutes-with-pullback}
  Let $\U \subset \RR^n$ and $\U' \subset \RR^{n'}$ be two real domains.
  Let $f: \U' \to \U$ be a smooth map,
  and let $\alpha \in \Cinfty(\U,\Form^p(\RR^n))$.
  We have,
  $f^*(d\alpha) = d(f^*\alpha)$.
  In other words, the exterior derivative and pullback commutes,
  $f^* \circ d = d \circ f^*$.
  
  \begin{center}
    \begin{tikzcd}[column sep=large, row sep=large, every label/.append style = {font = \small}]
      \Cinfty(\U,\Form^{p}(\RR^n)) \arrow[d, swap, "f^*"] \arrow[r,"d"] & \Cinfty(\U,\Form^{p+1}(\RR^n)) \arrow[d, "f^*"]  \\
      \Cinfty(\U',\Form^{p}(\RR^{n'})) \arrow[r, swap, "d"] & \Cinfty(\U',\Form^{p+1}(\RR^{n'}))
    \end{tikzcd}
  \end{center}
\end{article} %% Exterior-derivative-commutes-with-pullback

\begin{proof}
  Let $x'$ be a point in $\U'$,
  and $x = f(x')$.
  Let $\M$ be the linear tangent map $\M = \D(f)(x')$,
  and $\M_i = \M\ee_i$, where the $\ee_i$ are the vectors of the canonical basis.
  We shall use the expression of the exterior derivative given in \exref{Coordinates-of-the-exterior-derivative},
  $(d\omega)_{ijk\cdots \ell} = \Der_i\omega_{jk\cdots\ell}
  - \Der_j\omega_{ik\cdots\ell}
  -\Der_k\omega_{ji \cdots\ell} - \cdots
  -\Der_\ell\omega_{jk\cdots i}$,
  to show that,
  for every family of indices,
  we have $[d(f^*(\alpha))]_{ijk\cdots \ell} = [f^*(d\alpha)]_{ijk\cdots \ell}$.
  First of all,
  we have
  $
  [f^*(\alpha)]_{j\cdots \ell} = \sum_{r \cdots t} \M^r_j \cdots \M^t_\ell \ \alpha_{r \cdots t}
  $.
  Hence,
  the coordinates of the differential $d(f^*(\alpha))$ are given by
  \begin{eqnarray*}
    [d(f^*(\alpha))]_{ij\cdots \ell} & = & \partial_i \big[\sum_{r \cdots t} \M^r_j \cdots \M^t_\ell \ \alpha_{r \cdots t}\big] \\
    & - & \partial_j \big[\sum_{r \cdots t} \M^r_i \cdots \M^t_\ell \ \alpha_{r \cdots t}\big] \\
    & - & \cdots\\
    & - & \partial_\ell \big[\sum_{r \cdots t} \M^r_j \cdots \M^t_i \ \alpha_{r \cdots t}\big] \\
    & = & \sum_{r \cdots t} \partial_i[\M^r_j \cdots \M^t_\ell] \ \alpha_{r \cdots t} + \sum_{r \cdots t} \M^r_j \cdots \M^t_\ell \ \partial_i \alpha_{r \cdots t} \\
    & - & \sum_{r \cdots t} \partial_j[\M^r_i \cdots \M^t_\ell] \ \alpha_{r \cdots t} - \sum_{r \cdots t} \M^r_i \cdots \M^t_\ell \ \partial_j \alpha_{r \cdots t} \\
    & - & \cdots \\
    & - & \sum_{r \cdots t} \partial_\ell[\M^r_j \cdots \M^t_i] \ \alpha_{r \cdots t} - \sum_{r \cdots t} \M^r_j \cdots \M^t_i \ \partial_\ell \alpha_{r \cdots t}.
  \end{eqnarray*}
  
  Let $\T^{r \cdots t}_{j\cdots \ell} = \M^r_j \cdots \M^t_\ell$,
  and let us develop $\partial_i \alpha_{s \cdots t} = \sum_r \M^r_i \partial_r\alpha_{s \cdots t}$.
  After reordering some of the indices,
  we get
  
  \begin{eqnarray*}
    [d(f^*(\alpha))]_{ij\cdots \ell} & = & \sum_{rs \cdots t} \M^r_i\M^s_j \cdots \M^t_\ell (\Der_r\alpha_{s\cdots t} - \Der_s\alpha_{r \cdots t} - \cdots  -\Der_t\alpha_{s\cdots r} )\\
    & - & \sum_{r \cdots t}[\partial_j \T^{r \cdots t}_{i \cdots \ell} + \cdots + \partial_\ell \T^{r \cdots t}_{j\cdots i} - \partial_i \T^{r \cdots t}_{j\cdots \ell}] \ \alpha_{r \cdots t}.
  \end{eqnarray*}
  But
  \begin{eqnarray*}
    [f^*(d\alpha)]_{ij\cdots\ell} & = & \sum_{rs\cdots t}\M^r_i\M^s_j \cdots \M^t_\ell \ [d(\alpha)]_{rs\cdots t} \\
    & = & \sum_{rs\cdots t}\M^r_i\M^s_j \cdots \M^t_\ell (\Der_r\alpha_{s\cdots t} - \Der_s\alpha_{r \cdots t} - \cdots  -\Der_t\alpha_{s\cdots r}).
  \end{eqnarray*}
  Hence,
  $$%
    [d(f^*(\alpha))]_{ij\cdots \ell} = [f^*(d\alpha)]_{ij\cdots \ell} - \sum_{r \cdots t}[\partial_j \T^{r \cdots t}_{i \cdots \ell} + \cdots + \partial_\ell \T^{r \cdots t}_{j\cdots i} - \partial_i \T^{r \cdots t}_{j\cdots \ell}] \ \alpha_{r \cdots t}.
    $$%
  Now,
  let us consider the coefficients of the summands involved in the right-hand side,
  $$%
    \S^{\phantom{i,}r \cdots t}_{i,j\cdots \ell} = \partial_j \T^{r \cdots t}_{i \cdots \ell} + \cdots + \partial_\ell \T^{r \cdots t}_{j\cdots i} - \partial_i \T^{r \cdots t}_{j\cdots \ell},
    $$%
  that is,
  \begin{eqnarray*}
    \S^{\phantom{i,}r \cdots t}_{i,j\cdots \ell} & = &  \partial_j(\M^r_i \cdots \M^t_\ell) + \cdots +  \partial_\ell(\M^r_j \cdots \M^t_i)  - \partial_i(\M^r_j \cdots \M^t_\ell) \\
    & = & \partial_j\M^r_i (\cdots \M^t_\ell) + \M^r_i \partial_j(\cdots \M^t_\ell) \\
    & + & \cdots \\
    & + & (\M^r_j \cdots) \partial_\ell\M^t_i  + \M^t_i \partial_\ell (\M^r_j \cdots) \\
    &-& \partial_i\M^r_j(\cdots \M^t_\ell) - \cdots - (\M^r_j\cdots) \partial_i\M^t_\ell\,.
  \end{eqnarray*}
  But $\partial_i\M^r_j = \partial_i \partial_j x^r = \partial_j \partial_i x^r = \partial_j\M^r_i$,
  thus each term of the last line of the second equality cancels with a corresponding term in the previous lines.
  So,
  the sum rewrites
  \begin{eqnarray*}
    \S^{\phantom{i,}r s \cdots t}_{i,j k\cdots \ell}
    & = & \M^r_i \partial_j(\M^s_k\cdots \M^t_\ell) + \M^s_i \partial_k (\M^r_j \cdots \M^t_l) + \cdots + \M^t_i \partial_\ell(\M^r_j \M^s_k\cdots) \\
    & = & \M^r_i \partial_j \M^s_k (\cdots) \M^t_\ell +\M^r_i \M^s_k \partial_j(\cdots)\M^t_\ell + \M^r_i\M^s_k (\cdots) \partial_j\M^t_\ell \\
    & + & \M^s_i \partial_k \M^r_j (\cdots) \M^t_\ell +\M^s_i \M^r_j \partial_k(\cdots)\M^t_\ell + \M^s_i\M^r_j (\cdots) \partial_k\M^t_\ell \\
    &+& \cdots \\
    &+& \M^t_i \partial_\ell \M^r_j  \M^s_k (\cdots) +\M^t_i \M^r_j \partial_\ell\M^s_k (\cdots) + \M^t_i\M^r_j \M^s_k \partial_\ell(\cdots).
  \end{eqnarray*}
  Each term looks like $\M^r_i\partial_j\M^s_k[\cdots]$,
  but each such term has its counterpart in the sum,
  which writes $\M^s_i\partial_k\M^r_j [\cdots]$.
  Replacing the letter $\M$ by what it represents,
  the term $\S^{\phantom{i,}r s \cdots t}_{i,j k\cdots \ell}$ is the sum of pairs of the following kind:
  $$%
    \partial_ix^r \partial_j \partial_k x^s[\cdots] +  \partial_ix^s \partial_k \partial_j x^r[\cdots] =
    (\partial_i x^r \partial_j\partial_k x^s + \partial_i x^s \partial_j\partial_k x^r  )[\cdots].
    $$%
  The term between parentheses is clearly symmetric in $(r,s)$.
  Hence,
  summed with the coefficient $\alpha_{rs\cdots t}$,
  which is antisymmetric in $(r,s)$,
  it vanishes.
  Thus,
  all these terms vanish and $[d(f^*(\alpha))]_{ij\cdots \ell} = [f^*(d\alpha)]_{ij\cdots \ell}$,
  for every family of indices.
  Therefore,
  $ d(f^*(\alpha))= f^*(d\alpha)$.
\end{proof}

\begin{article}\artlabel{Exterior product of smooth forms}
  \addcontentsline{toc}{section}{\small\hspace{10pt} Exterior product of smooth forms}
  \label{Exterior-product-of-smooth-forms}
  Let $\U \subset \RR^n$ be some real domain.
  Let $\alpha \in \Cinfty(\U,\Form^p(\RR^n))$ and $\beta \in \Cinfty(\U,\Form^q(\RR^n))$.
  The {\em exterior product} $\alpha\wedge\beta$ is defined by:
  $$%
    \text{For all $x \in \U$}, \  (\alpha\wedge\beta)(x) = \alpha(x)\wedge\beta(x),
    $$%
  where $\alpha(x) \wedge \beta(x)$ is given in \art{Exterior-product}.
  So,
  $\alpha \wedge \beta$ is a smooth $(p+q)$-form on $\U$.
  The exterior product is a bilinear map,
  $$%
    \wedge: \Cinfty(\U,\Form^p(\RR^n)) \times \Cinfty(\U,\Form^q(\RR^n)) \to \Cinfty(\U,\Form^{p+q}(\RR^n)).
    $$%
\end{article} %% Exterior-product-of-smooth-forms

\begin{proof}
  The components of $\alpha\wedge\beta$ are linear combinations of products of the components of $\alpha$ and $\beta$.
  Since the components of $\alpha$ and $\beta$ are smooth, the components of $\alpha\wedge\beta$ are smooth.
\end{proof}

\begin{article}\artlabel{Integration of smooth $p$-forms on $p$-cubes}
  \addcontentsline{toc}{section}{\small\hspace{10pt} Integration of smooth $p$-forms on $p$-cubes}
  \label{Integration-of-a-smooth-p-form-on-a-p-cube}
  Let $\omega$ be a smooth $p$-form defined on a domain $\U \subset \RR^p$,
  that is,
  $\omega \in \Cinfty(\U,\Form^p(\RR^p))$.
  Let $\cB = (\ee_1,\ldots,\ee_p)$ be the canonical basis of $\RR^p$.
  The basis $\cB$ defines the {\em positive orientation} of $\RR^p$,
  and let $\vol_p$ be the associated volume $\vol_p = \ee^1\wedge\cdots\wedge \ee^p$ \art{Volumes-and-determinants}.
  At each point $x \in \U$,
  the form $\omega$ is proportional to $\vol_p$.
  Let us define $f: \U \to \RR$ by $f(x) = \omega(x)(\ee_1)\cdots(\ee_p)$.
  Thus,
  the smooth form $\omega$ writes unambiguously
  $$%
    \omega (x) = f(x) \times \vol_p, \text{ with } f \in \Cinfty(\U,\RR).
    $$%
  Now,
  let us define a {\em $p$-cube}\index{Cube} in $\U$ as a product of closed intervals,
  $$%
    \C = [a_1,b_1] \times \cdots \times [a_p,b_p] \subset \U.
    $$%
  The {\em integral of the smooth form $\omega$}\index{Integral of a smooth form} on the {\em cube} ${\C}$ is defined as the real number
  $$%
    \int_\C \omega = \int_\C f \times \vol_p = \int_{a_1}^{b_1} dx_1\cdots\int_{a_p}^{b_p} dx_p \ f(x_1,\ldots,x_p).
    $$%
  The ordinary {\em multiple integral} of the right-hand side is defined as usual.
  For each given value $(x_1,\ldots,x_{p-1})$,
  the function $x \mapsto f(x_1,\ldots,x_{p-1},x)$,
  defined on a small open neighborhood of $[a_1,b_1]$,
  is smooth.
  Thus,
  this function has primitives,
  let $\F[x_1\cdots x_{p-1}]$ be one of them,
  $$%
    \F[x_1 \cdots x_{p-1}]'(x) = f(x_1,\ldots,x_{p-1},x).
    $$%
  Now,
  the integral of $x \mapsto f(x_1,\ldots,x_{p-1},x)$ on $[a_p,b_p]$,
  $$%
    \int_{a_p}^{b_p} f(x_1, \ldots, x_{p-1},x) dx = \F[x_1 \cdots x_{p-1}](b_p) - \F[x_1 \cdots x_{p-1}](a_p),
    $$%
  is a smooth real function of the variables $(x_1,\ldots,x_{p-1})$.
  Then,
  the integral of $\omega$ on the cube ${\C}$ is given by
  \begin{eqnarray*}
    \int_{\C} \omega & = &\int_{a_1}^{b_1}dx_1 \cdots \int_{a_p}^{b_p} dx_p \ f(x_1,\ldots,x_p) \\
    & = & \int_{a_1}^{b_1}dx_1 \cdots \int_{a_{p-1}}^{b_{p-1}} dx_{p-1}  \{ \F[x_1 \cdots x_{p-1}](b_p) - \F[x_1 \cdots x_{p-1}](a_p)\} \\
    & = & \int_{a_1}^{b_1} \! dx_1 \cdots \int_{a_{p-2}}^{b_{p-2}} \! dx_{p-2} \{ \F[x_1 \cdots x_{p-2}](b_{p-1})(b_p) \\
    & - & \int_{a_1}^{b_1} \! dx_1 \cdots \int_{a_{p-2}}^{b_{p-2}} \! dx_{p-2} \F[x_1 \cdots x_{p-2}](a_{p-1})(b_p)\} \\
    & - & \int_{a_1}^{b_1} \! dx_1 \cdots \! \int_{a_{p-2}}^{b_{p-2}} \! dx_{p-2} \{\F[x_1 \cdots x_{p-2}](b_{p-1})(a_p) \\
    & + & \int_{a_1}^{b_1} \! dx_1 \cdots \! \int_{a_{p-2}}^{b_{p-2}} \! dx_{p-2} \F[x_1 \cdots x_{p-2}](a_{p-1})(a_p)\},
  \end{eqnarray*}
  where $x \mapsto \F[x_1 \cdots x_{p-2}](x)(x_p)$ is a primitive of $x \mapsto \F[x_1 \cdots x_{p-2} \ x](x_p)$,
  and so on.
  Finally,
  the integration of $\omega$ over $\C$ is obtained after $p$ iterations of this process.
  Note that the result does not depend on the choice of the primitives.
  Let us illustrate this construction by integrating the first forms of low degree.
  
  {\em $1$-form.} If $\omega$ is a smooth $1$-form on $\U \subset \RR$,
  that is,
  $\omega(t) = f(t) \times dt$,
  then
  $$%
    \int_{[a,b]} \omega = \int_a^b f(t)\dt = \F(b) -\F(a),
    $$%
  where $\F$ is a primitive of $f$, $\F'=f$.
  
  {\em $2$-form.} If $\omega$ is a smooth $2$-form on $\U \subset \RR^2$,
  and ${\C} = [a_1,b_1] \times [a_2,b_2]$,
  then
  $$%
    \int_{\C} \omega = [\F(b_1)(b_2) -\F(a_1)(b_2)] - [\F(b_1)(a_2) - \F(a_1)(a_2)],
    $$%
  where $x \mapsto \F(x)(x_2)$ is a primitive of $x \mapsto \F[x](x_2)$,
  and $x \mapsto \F[x_1](x)$ is a primitive of $x \mapsto f(x_1,x)$.
  
  \Note~Since,
  by convention,
  $\vol_p = \ee^1 \wedge \cdots \wedge \ee^p = dx^1 \wedge \cdots \wedge dx^p$ \art{The-differential-of-a-function-as-pullback},
  we have the equivalence of notation
  $$%
    \int_{\C} f \times dx^1 \wedge \cdots \wedge dx^p = \int_{a_1}^{b_1} dx_1 \cdots \int_{a_p}^{b_p} dx_p \ f(x_1,\ldots,x_p).
    $$%
\end{article} %% Integration-of-a-smooth-p-form-on-a-p-cube

%%%%%%%%%%%%%%%%%%%%%%%%%%%%%%%%%%%%%%%%%%%%%%%%%%%%%%%%%%
%
%   Exercises
%
%%%%%%%%%%%%%%%%%%%%%%%%%%%%%%%%%%%%%%%%%%%%%%%%%%%%%%%%%%

\Exercises

\begin{exercise}[Coordinates of the exterior derivative]
  \label{Coordinates-of-the-exterior-derivative}
  Let $\U \subset \RR^n$ be a domain.
  Let $\omega$ be a smooth $1$-form defined on $\U$,
  and let $x$ denote a generic point of $\U$.
  Let
  $$%
    \omega = \sum_{i = 1}^n \omega_i \ee^i, \text{ and } x = \sum_{i = 1}^n x^i \ee_i.
    $$%
  Let $\Der_i$ denote the partial derivative with respect to $x^i$.
  Using antisymmetry and reordering indices, we have
  \begin{eqnarray*}
    d\omega & = & \sum_{j = 1}^n \sum_{i = 1}^n \Der_i \omega_j\,\ee^i\wedge \ee^j \\
    & = & \sum_{1\leq i<j\leq n} \Der_i \omega_j\, \ee^i\wedge \ee^j + \sum_{1\leq j<i\leq n} \Der_i \omega_j\, \ee^i\wedge \ee^j \\
    & = & \sum_{1\leq i<j\leq n} \Der_i \omega_j\, \ee^i\wedge \ee^j + \sum_{1\leq i<j\leq n} \Der_j \omega_i\,\ee^j\wedge \ee^i \\
    & = & \sum_{1\leq i<j\leq n} \Der_i \omega_j\, \ee^i\wedge \ee^j - \sum_{1\leq i<j\leq n} \Der_j \omega_i\, \ee^i\wedge \ee^j \\
    & = & \sum_{1\leq i<j\leq n} (\Der_i \omega_j - \Der_j \omega_i) \ee^i\wedge \ee^j.
  \end{eqnarray*}
  Show that,
  for a general smooth $p$-form $\omega$,
  the coordinates of the exterior derivative $d\omega$ are given by
  $$%
    (d\omega)_{ijk\cdots \ell} = \Der_i\omega_{jk\cdots\ell} - \Der_j\omega_{ik\cdots\ell} -\Der_k\omega_{ji \cdots\ell} - \cdots -\Der_\ell\omega_{jk\cdots i}.
    $$%
\end{exercise} %% Coordinates-of-the-exterior-derivative

\begin{exercise}[Integral of a $3$-form on a $3$-cube]
  \label{Integral-of-a-3-form-on-a-3-cube}
  Make explicit the integral of a smooth $3$-form on a 3-cube using iterated primitives;
  see \art{Integration-of-a-smooth-p-form-on-a-p-cube}.
\end{exercise} %% Integral-of-a-3-form-on-a-3-cube

%%%%%%%%%%%%%%%%%%%%%%%%%%%%%%%%%%%%%%%%%%%%%%%%%%%%%%%%%%
%% MARK: Differential Forms on Diffeological Spaces
%%%%%%%%%%%%%%%%%%%%%%%%%%%%%%%%%%%%%%%%%%%%%%%%%%%%%%%%%%

\section*{Differential Forms on Diffeological Spaces}
\label{Section-Differential-forms-on-diffeological-spaces}

\begin{sechead}
  Differential forms on diffeological spaces are defined by their evaluations on the plots,
  which are regarded as the pullbacks of the forms by the plots.
  These pullbacks are ordinary smooth forms \art{Smooth-forms-on-numerical-domains}.
  Hence,
  a differential form of a diffeological space is known as soon as we know all its pullbacks by all the plots.
  This is the idea behind the definition of differential forms on diffeological spaces.
  The condition on the pullbacks,
  to represent a differential form of a diffeological space,
  expresses just the condition of compatibility under composition.
\end{sechead}

\begin{article}\artlabel{Differential forms on diffeological spaces}
  \addcontentsline{toc}{section}{\small\hspace{10pt} Differential forms on diffeological spaces}
  \label{Differential-forms-on-diffeological-spaces}
  Let $\X$ be a diffeological space.
  A {\em differential $k$-form}\index{Differential form} on $\X$ (or of $\X$) is a map $\alpha$ which associates,
  with every plot $\P$ of $\X$,
  a smooth $k$-form $\alpha(\P)$ defined on the domain of $\P$ (see \art{Smooth-forms-on-numerical-domains}) such that,
  for every smooth parametrization $\F$ in the domain of the plot $\P$,
  \begin{equation}
    \renewcommand{\theequation}{$\clubsuit$}
    \alpha(\P \circ \F) = \F^*(\alpha(\P)).
  \end{equation}
  To be more precise,
  $\alpha$ is a $k$-form on $\X$  if and only if the following two conditions are fulfilled:
  
  1. For all integers $n$, for all $n$-plots $\P : \U \to \X$, $\alpha(\P) \in \Cinfty(\U,\Form^k(\RR^n))$.
  
  2. For all $m$-domains $\V$,
  for all smooth parametrizations $\F : \V \to \U$,
  for all $v \in \V$ and for all $k$ vectors $\xi_1\cdots \xi_k \in \RR^m$,
  $$%
    \alpha({\P} \circ \F)(v)(\xi_1)\cdots(\xi_k) = \alpha({\P})(\F(v))(\D(\F)(v) (\xi_1))\cdots(\D(\F)(v)(\xi_k)).
    $$%
  The condition $\alpha(\P \circ \F) = \F^*(\alpha(\P))$ is the {\em smooth compatibility condition},
  and we shall say that $\alpha(\P)$ {\em represents} the differential form $\alpha$ in the plot $\P$.
  The set of differential $k$-forms of $\X$ is clearly a real vector space;
  it will be denoted by $\Omega^k(\X)$.
  
  \Note~Since smooth forms are a special case of covariant smooth tensors,
  which satisfy the same contravariant pullback property,
  there is no reason {\em a priori\/} to not define a larger class of objects,
  the {\em differential covariant tensors},
  on any diffeological space $\X$.
  By definition,
  a differential covariant $p$-tensor $\tau$ on $\X$ is a map $\P \mapsto \tau(\P)$ such that $\tau(\P)$ is a smooth covariant $p$-tensor,
  maybe with a given type of symmetry,
  satisfying the compatibility condition $(\clubsuit)$ above.
\end{article} %% Differential-forms-on-diffeological-spaces

\begin{article}\artlabel{Functional diffeology of the space of forms}
  \addcontentsline{toc}{section}{\small\hspace{10pt} Functional diffeology of the space of forms}
  \label{Functional-diffeology-of-the-space-of-forms}
  The set $\Omega^k(\X)$ of all differential $k$-forms on a diffeological space $\X$ is a real vector space.
  For all $\alpha,\alpha' \in \Omega^k(\X)$,
  for all real numbers $s$,
  for all plots $\P$ of $\X$:
  $$%
    \left\{
    \begin{array}{r@{\,\,}c@{\,\,}l}
    (\alpha + \alpha')({\P}) & = & \alpha(\P) + \alpha'(\P), \\
    (s\times \alpha)(\P) & = & s\times \alpha(\P).
    \end{array}
    \right.
    $$%
  The sum $\alpha({\P}) + \alpha'(\P)$ and the product by a scalar $s\times \alpha(\P)$,
  of smooth differential forms,
  have been defined in \art{Smooth-forms-on-numerical-domains}.
  The set of parametrizations $\phi: \V \mapsto \DForms^k(\X)$,
  defined by the following condition,
  is a diffeology of vector space \art{Diffeological-vector-spaces}.
  \begin{enumerate}
    \item[($\clubsuit$)] For every plot ${\P}: \U \to \X$,
    the map $(s,r) \mapsto  \phi(s)({\P})(r)$ defined from $\V \times \U$ to $\Lambda^k(\RR^n)$ is smooth.
    Briefly,
    $[(s,r) \mapsto \phi(s)({\P})(r)] \in \Cinfty(\V \times \U,\Lambda^k(\RR^n))$.
  \end{enumerate}
  We shall call this diffeology the {\em standard functional diffeology},
  or simply the {\em functional diffeology},
  of $\Omega^k(\X)$.
\end{article} %% Functional diffeology of the space of forms

\begin{proof}
  Let us check the axioms of the diffeology.
  
  D1. By the very definition of differential forms,
  the constant $m$-plots $s \mapsto \omega$ satisfy the condition $[(s,r) \mapsto \omega({\P})(r)] \in \Cinfty(\RR^m\times \U, \Form^k(\RR^n))$.
  
  D2'. Let $\phi: s \mapsto \omega_s$ be a parametrization in $\DForms^k(\X)$,
  defined on $\V$ such that,
  for all $s_0 \in \V$ there exists an open neighborhood $\W \subset \V$ of $s_0$ such that $[(s,r) \mapsto \omega_s({\P})(r)]  \restriction \W \times \U \in \Cinfty({\W} \times \U, \Form^k(\RR^n))$.
  Then,
  the map $(s,r) \mapsto \omega_s({\P})(r)$ is a parametrization in $\Form^k(\RR^n)$, locally smooth at each point,
  thus smooth.
  
  D3. Let $\F \in \Cinfty({\W},\V)$ and $\phi: s \mapsto \omega_s$, defined on $\V$,
  satisfying $(\clubsuit)$.
  Then,
  $\phi \circ \F = [(t,r) \mapsto (s = \F(t),r) \mapsto \omega_s({\P})(r)]$ is smooth,
  since it is the composite of two smooth parametrizations.
  Thus, $\phi \circ \F$ satisfies $(\clubsuit)$.
  
  Therefore,
  the condition $(\clubsuit)$ defines  a diffeology of $\Omega^k(\X)$.
  Moreover,
  since for smooth forms,
  defined on domains,
  addition and multiplication by a scalar are smooth operations,
  these operations are also smooth for differential forms defined on diffeological spaces.
  In conclusion, the diffeology  of $\DForms^k(\X)$ defined by $(\clubsuit)$ is a vector space diffeology.
\end{proof}

\begin{article}\artlabel{Smooth forms and differential forms}
  \addcontentsline{toc}{section}{\small\hspace{10pt} Smooth forms and differential forms}
  \label{Smooth-forms-and-differential-forms}
  Let $\U \subset \RR^n$ be a real domain,
  regarded as a diffeological space \art{Real-domains-as-diffeological-spaces}.
  Let $a$ be a smooth $k$-form on $\U$ \art{Smooth-forms-on-numerical-domains},
  that is,
  $a \in \Cinfty(\U,\Form^k(\RR^n))$.
  Let us define,
  for every plot in $\U$, $\P \in \Cinfty(\V,\U)$,
  the smooth $k$-form on $\V$,
  $$%
    \alpha({\P}) = {\P}^*(a).
    $$%
  Thanks to the chain-rule property of the pullback \art{Pullbacks-of-smooth-forms},
  for any smooth parametrization $\F \in \Cinfty({\W},\V)$,
  we have
  $$%
    \alpha({\P} \circ \F) = ({\P} \circ \F)^*(a) = \F^*({\P}^*(a)) = \F^*(\alpha({\P})).
    $$%
  Thus,
  $\alpha$ is a differential $k$-form of $\U$ \art{Differential-forms-on-diffeological-spaces}.
  Now,
  the map defined by $a \mapsto \alpha$ is clearly a linear isomorphism from $\Cinfty(\U,\Form^k(\RR^n))$ to $\DForms^k(\U)$.
  Its inverse is given by $a = \alpha(\id_\U)$,
  where $\id_\U$ is the identity plot of $\U$.
  Therefore,
  for real domains,
  this map is a natural identification between the smooth forms defined in \art{Smooth-forms-on-numerical-domains} and the differential forms defined in \art{Differential-forms-on-diffeological-spaces},
  $\Omega^k(\U) \simeq \Cinfty(\U,\Lambda^k(\RR^n))$.
  We may sometimes identify a differential form $\alpha$,
  defined on a domain $\U$,
  with its value on the identity $a = \alpha(\id_\U)$.
\end{article} %% Smooth forms and differential forms

\begin{article}\artlabel{Zero-forms are smooth functions}
  \addcontentsline{toc}{section}{\small\hspace{10pt} Zero-forms are smooth functions}
  \label{Zero-forms-are-smooth-functions}
  Let $\X$ be a diffeological space.
  Let us associate with each 0-form $\varphi \in \Omega^0(\X)$ \art{Differential-forms-on-diffeological-spaces} the map $f : \X \to \RR$ defined by
  $$%
    f(x) = \varphi([0 \mapsto x]),
    $$%
  where $[0 \mapsto x]$ is the 0-plot with value $x$.
  Then,
  $f$ is a smooth real map from $\X$ to $\RR$,
  $f \in \Cinfty(\X, \RR)$.
  This construction is a natural smooth identification
  $$%
    \Omega^0(\X) \simeq \Cinfty(\X,\RR).
    $$%
\end{article} %% Zero-forms are smooth functions

\begin{proof}
  Let ${\P}: \U \to \X$ be a plot,
  for every $r \in \U$, $f \circ {\P}(r) = \varphi([0 \mapsto {\P}(r)] = \varphi({\P} \circ [0 \mapsto r]) = [0 \mapsto r]^*(\varphi({\P}))$,
  but $[0 \mapsto  r]^*(\varphi({\P})) = \varphi({\P})(r)$,
  by definition.
  Then,
  $f \circ {\P}$ is smooth.
  Thus,
  $f$ is a smooth real function of $\X$.
  Conversely,
  every smooth function $f$ from $\X$ to $\RR$ defines a 0-form $\varphi$ by $\varphi({\P}) = f \circ {\P}$.
  Therefore,
  $\Omega^0(\X) \simeq \Cinfty(\X,\RR)$.
\end{proof}

\begin{article}\artlabel{Pullbacks of differential forms}
  \addcontentsline{toc}{section}{\small\hspace{10pt} Pullbacks of differential forms}
  \label{Pullbacks-of-differential-forms}
  Let $\X$ and $\X'$ be two diffeological spaces.
  Let $\alpha' \in \Omega^k(\X')$ and $f: \X \to \X'$ be a smooth map.
  There exists a differential $k$-form on $\X$,
  denoted by $f^*(\alpha')$ and defined by
  $$%
    (f^*(\alpha'))({\P}) = \alpha'(f \circ {\P}), \text{ for all plots $\P$ of $\X$.}
    $$%
  The $k$-form $f^*(\alpha')$ is called {\em pullback}\index{Pullback} of $\alpha'$ by $f$.
  The pullback of differential forms is contravariant.
  Let $\X$, $\X'$ and $\X''$ be three diffeological spaces.
  Let $f: \X \to \X'$ and $g: \X' \to \X''$ be two smooth maps,
  let $\alpha'' \in \Omega^k(\X'')$,
  then
  $$%
    (g \circ f )^*(\alpha'') = f^*(g^*(\alpha'')).
    $$%
  Moreover,
  the pullback operation
  $$%
    f^*: \Omega^k(\X') \to \Omega^k(\X)
    $$%
  is a smooth linear map for the functional diffeology of the spaces of forms defined in \art{Differential-forms-on-diffeological-spaces}.
\end{article} %% Pullbacks of differential forms

\begin{proof}
  Let us check that $f^*(\alpha')$ is a form on $\X$.
  Let ${\P}: \U \to \X$ be a plot of $\X$ and $\F \in \Cinfty(\V, \U)$,
  where $\V$ is some real domain.
  Then,
  $(f^*(\alpha))({\P} \circ \F) = \alpha(f \circ {\P} \circ \F) = \F^*(\alpha(f \circ {\P})) = \F^*((f^*(\alpha))({\P}))$.
  Now,
  let us prove that $f^*: \Omega^k(\X') \to \Omega^k(\X')$ is smooth.
  Let $\phi: \V \to \Omega^k(\X')$ be a plot,
  we want to check that $f^* \circ \phi$ is smooth.
  Let ${\P}: \U \to \X$ be a plot,
  then for every $(s,r) \in \V \times \U$,
  $(f^* \circ \phi)(s)(\P)(r) = f^*((\phi)(s))(\P)(r) = \phi(s)(f \circ \P)(r)$.
  But,
  since $f$ is smooth,
  $\P' = f \circ {\P}$ is a plot of $\X'$ and since $\phi$ is smooth,
  $(s,r) \mapsto \phi(s)({\P}')(r)$ is smooth.
  Therefore,
  $f^*$ is smooth.
\end{proof}

\begin{article}\artlabel{Pullbacks by the plots}
  \addcontentsline{toc}{section}{\small\hspace{10pt} Pullbacks by the plots}
  \label{Pullbacks-by-the-plots}
  Let $\X$ be a diffeological space and $\alpha \in \Omega^k(\X)$.
  Let ${\P}: \U \to \X$ be a plot and $\U$ regarded as a smooth diffeological space \art{Real-domains-as-diffeological-spaces}.
  The plot $\P$ is a smooth map from $\U$ to $\X$ \art{Plots-are-smooth},
  so $\P^*(\alpha) \in \Omega^k(\U)$.
  The $k$-form $\P^*(\alpha)$ is characterized by its values on the plot $\id_\U$ \art{Smooth-forms-and-differential-forms},
  that is,
  $\P^*(\alpha)(\id_\U) = \alpha(\P \circ \id_\U) = \alpha(\P)$.
  Therefore,
  the values of $\alpha$ on the plots,
  which define the form $\alpha$,
  represent just the pullbacks of $\alpha$ by the plots.
  In other words,
  the differential form $\alpha$ is just defined by its pullbacks by the plots of $\X$,
  and the condition of compatibility just writes $(\P \circ \F)^*(\alpha) = \F^*(\P^*(\alpha))$,
  for any plot $\P$ of $\X$ and any smooth parametrization $\F$ in the domain of $\P$.
\end{article} %% Pullbacks by the plots

\begin{article}\artlabel{Exterior derivative of forms}
  \addcontentsline{toc}{section}{\small\hspace{10pt} Exterior derivative of forms}
  \label{Exterior-derivative-of-forms}
  Let $\X$ be a diffeological space.
  Let $\alpha$ be a $p$-form of $\X$.
  The exterior derivative of $\alpha$ is the differential $(p+1)$-form defined by
  $$%
    (d\alpha)(\P) = d(\alpha(\P)),
    $$%
  for all plots $\P$ of $\X$.
  The exterior derivative $d$ is a smooth linear operator,
  $$%
    d: \Omega^p(\X) \to \Omega^{p+1}(\X),
    $$%
  with $\Omega^p(\X)$ and $\Omega^{p+1}(\X)$ equipped with the functional diffeology \art{Functional-diffeology-of-the-space-of-forms}.
  
  \Note~Thanks to the commutativity between pullback and exterior derivative of smooth forms \art{Pullbacks-of-tensors-and-forms},
  the pullback of differential forms on diffeological spaces commutes with the exterior derivative.
  Let $\X'$ be another diffeological space,
  and let $f: \X \to \X'$ be a smooth map.
  Then,
  for all differential forms
  $\alpha'$ of $\X'$ we have
  $$%
    d(f^*(\alpha')) = f^*(d\alpha').
    $$%
\end{article} %% Exterior-derivative-of-forms

\begin{proof}
  First of all, $d\alpha$ is well defined.
  Indeed,
  since,
  for differential forms,
  the pullback commutes with the exterior derivative \art{Exterior-derivative-of-smooth-forms},
  the definition of $d\alpha$ gives a well defined form of $\X$,
  $$%
    (d\alpha)({\P} \circ \F) = d(\alpha({\P} \circ \F)) = d(\F^*(\alpha({\P}))) = \F^*(d(\alpha({\P}))) = \F^*((d\alpha)({\P})).
    $$%
  Next,
  the differential $d$ is smooth.
  Indeed,
  for every plot $\P$ of $\X$,
  the components of $(d\alpha)({\P})$ are linear combinations of partial derivatives of smooth real functions,
  thus they are smooth real functions.
\end{proof}

\begin{article}\artlabel{Exterior product of differential forms}
  \addcontentsline{toc}{section}{\small\hspace{10pt} Exterior product of differential forms}
  \label{Exterior-product-of-differential-forms}
  Let $\X$ be a diffeological space,
  let $\alpha \in \Omega^k(\X)$ and $\beta \in \Omega^\ell(\X)$.
  The {\em exterior product}\index{Exterior product} $\alpha \wedge \beta$ is the differential $(k + \ell)$-form defined on $\X$ by
  $$%
    (\alpha \wedge \beta)(\P) = \alpha(\P) \wedge \beta(\P),
    $$%
  for all plots $\P$ of $\X$.
  Regarded as the map
  $$%
    \wedge: \Omega^k(\X) \times \Omega^\ell(\X) \to \Omega^{k+\ell}(\X), \text{ with } \wedge(\alpha,\beta) = \alpha\wedge \beta,
    $$%
  the exterior product is smooth and bilinear,
  for the functional diffeology of the spaces of forms defined in \art{Differential-forms-on-diffeological-spaces}.
  The following properties of the exterior product of smooth forms pass naturally to the exterior product of differential forms on diffeological spaces,
  $$%
    \alpha\wedge \beta = (-1)^{k\ell} \beta\wedge\alpha
    \text{ and }
    f^*(\alpha\wedge \beta) = f^*(\alpha)\wedge f^*(\beta),
    $$%
  where $f : \X' \to \X$ is a smooth map.
\end{article} %% Exterior-product-of-differential-forms

\begin{proof}
  First of all, the exterior product is well defined.
  Indeed, let $\P : \U \to \X$ be a plot,
  and  $\F \in \Cinfty(\V, \U)$ be a smooth parametrization,
  then
  \begin{align*}
    (\alpha \wedge \beta) (\P \circ \F) & = \alpha (\P \circ \F) \wedge \beta (\P \circ \F) \\
    & =  \F^*(\alpha(\P)) \wedge \F^*(\beta(\P)) \\
    & =  \F^*({\P}^*(\alpha) \wedge {\P}^*(\beta)) \qquad \text{\art{Pullbacks-of-tensors-and-forms}} \\
    & =  \F^*((\alpha \wedge \beta)(\P)).
  \end{align*}
  Hence,
  the map ${\P} \mapsto {\P}^*(\alpha \wedge \beta)$ defines a differential form $\alpha\wedge \beta \in \Omega^{k+\ell}(\X)$.
  Next,
  the exterior product is smooth.
  Indeed,
  let $r \mapsto (\phi(r),\psi(r))$ be a plot of $\Omega^k(\X) \times \Omega^\ell(\X)$.
  Thus,
  $\phi: \V \to \Omega^k(\X)$ and $\psi: \V \to \Omega^\ell(\X)$ are two plots.
  Let $\P : \U \to \X$ be a plot.
  Then,
  $(s,r) \mapsto [\phi(s)\wedge \psi(s)]({\P})(s) =  [\phi(s)({\P})(r)]\wedge [\psi(s)({\P})(r)]$.
  But the product of smooth plots is smooth \art{Exterior-product-of-smooth-forms},
  thus the map $(s,r) \mapsto  [\phi(s)({\P})(r)]\wedge [\psi(s)({\P})(r)]$ is smooth.
  Therefore,
  the exterior product is a smooth operation.
  What remains of the proposition is a direct consequence of the properties of linear forms \art{Exterior-product} and \art{Pullbacks-of-tensors-and-forms}.
\end{proof}

\begin{article}\artlabel{Differential forms are local}
  \addcontentsline{toc}{section}{\small\hspace{10pt} Differential forms are local}
  \label{Differential-forms-are-local}
  Let $\X$ be a diffeological space.
  Let $\alpha$ and $\beta$ be two differential $p$-forms of $\X$.
  Let $x$ be a point of $\X$.
  We shall say that $\alpha$ and $\beta$ {\em have the same germ} at $x$ if for every plot $\P : \U \to \X$,
  such that $0 \in \U$ and $\P(0) = x$ (that is, $\P$ is centered at $x$)\index{Centered plot},
  there exists an open neighborhood $\V$ of $0 \in \U$ such that $\alpha(\P) \restriction \V = \beta(\P) \restriction \V$.
  
  1) The form $\alpha$ is zero if and only if its germ vanishes at each point.
  
  The following statements are equivalent to the first one:
  
  2) The form $\alpha$ is zero if and only if  for every $x \in \X$ there exists a D-open neighborhood $\V$ of $x$ \art{The-D-Topology-of-diffeological-spaces} such that $\alpha \restriction \V = 0$.
  
  3) Two differential forms $\alpha$ and $\beta$ of $\X$ coincide if and only if they have the same germ at every point.
  
  4) Two forms $\alpha = \beta$ of $\X$ coincide if and only if there exists a  D-open covering $\cU = \{\U\}_{i \in \cI}$ of $\X$ such that $\alpha \restriction \U_i = \beta\restriction \U_i$.
  
  \Note~This is not in contradiction with the fact that the D-topology of diffeological spaces can be trivial,
  as it happens,
  for example, with the irrational torus;
  see \exref{Diffeomorphisms-between-irrational-tori},
  and \exref{The-irrational-torus-is-not-a-manifold}.
  It just says that,
  in this case,
  a differential form is only defined globally;
  see \exref{Forms-bundles-of-irrational-tori}.
\end{article} %% Differential forms are local

\begin{proof}
  1) It is obvious that the zero form,
  defined by $\alpha(\P) = 0$ for all plots $\P$ of $\X$,
  has a zero germ at every point of $\X$.
  Conversely,
  if the germ of $\alpha$ vanishes at each point,
  for every plot $\P : \U \to \X$,
  for every $r \in \U$ such that $\P(r) = x$,
  there exists an open neighborhood $\V$ of $r$ such that $\alpha(\P) \restriction \W = 0$.
  Indeed,
  we just have to compose the plot $\P$ with the translation mapping $0$ to $r$ to get a centered plot at $x$.
  Thus, $\alpha(\P)$ vanishes locally everywhere.
  Since smooth forms on real domains are local,
  $\alpha(\P) = 0$, that is,
  $\alpha = 0$.
  
  2) It is obvious that if $\alpha = 0$,
  its restriction on every D-open vanishes.
  Conversely,
  let us assume that $\alpha$ vanishes D-locally everywhere on $\X$.
  Let $\P : \U \to \X$ be a plot,
  let $r \in \U$ and $x = \P(r)$.
  Let $\V$ be a D-open neighborhood of $x$ such that $\alpha \restriction \V = 0$.
  Since, by definition of the D-topology,
  plots are continuous, $\W = {\P}^{-1}(\V)$ is an open neighborhood of $r$ and $\alpha(\P \restriction \W) = (\alpha \restriction \V)(\P \restriction \W)= 0$.
  Hence,
  $\alpha({\P})$ vanishes locally at each point $r \in \U$,
  thus $\alpha({\P}) = 0$,
  for all plots ${\P}$ of $\X$, that is,
  $\alpha = 0$.
  
  3) We apply the item 2) to $\alpha - \beta$.
  
  4) From item 2) we have immediately that a form $\alpha$ of $\X$ is zero if and only if there exists a D-open covering of $\X$ such that $\alpha$ vanishes on each of its elements.
  Then,
  we apply that to $\alpha - \beta$.
\end{proof}

\begin{article}\artlabel{The $\bf k$-forms are defined by the $\bf k$-plots}
  \addcontentsline{toc}{section}{\small\hspace{10pt} The $\bf k$-forms are defined by the $\bf k$-plots}
  \label{The-k-forms-are-defined-by-the-k-plots}
  A $k$-form $\alpha$,
  on a diffeological space $\X$,
  is zero if and only if $\alpha(\P) = 0$ for every $k$-plot $\P$ of $\X$.
  Formally,
  let us denote by $\cD_k(\X)$ the set of the $k$-plots of $\X$,
  $$%
    \alpha = 0  \text{ if and only if } \alpha(\P) = 0, \text{ for all } \P\in \cD_k(\X).
    $$%
  Then,
  for $\alpha$ and $\beta$,
  two $k$-forms on $\X$,
  $\alpha = \beta$ if and only if $\alpha(\P) = \beta(\P)$ for every $k$-plot of $\X$.
  $$%
    \alpha = \beta \text{ if and only if } \alpha(\P) = \beta(\P), \text{ for all } \P\in \cD_k(\X).
    $$%
  In other words,
  differential $k$-forms of $\X$ are defined uniquely by their values on the $k$-plots.
  But
  the smoothness,
  defined by the compatibility axiom,
  still needs to be checked against all the plots of the space.
\end{article} %% The-k-forms-are-defined-by-the-k-plots

\begin{proof}
  Only one way of this proposition needs to be proved.
  Assume that $\alpha(\P) = 0$ for all $k$-plots.
  Let $\Q: \U \to \X$ be any plot,
  let $n = \dim(\U)$,
  let $r \in \U$ and $v_1,\ldots, v_k$ be $k$ vectors of $\RR^n$.
  Define the following smooth parametrization in $\RR^n$:
  $$%
    {\T}^r_{v_1\cdots v_k}: (s_1,\ldots,s_k) \mapsto r + s_1 \times v_1 + \cdots + s_k \times v_k.
    $$%
  The parametrization ${\T}^r_{v_1\cdots v_k}$ maps $0 \in \RR^k$ to $r$ and,
  restricted to some open neighborhood of $0$, ${\T}^r_{v_1\cdots v_k}$ is smooth.
  Then, $\Q \circ {\T}^r_{v_1\cdots v_k}$ is a $k$-plot of $\X$ and,
  by hypothesis,
  $\alpha (\Q \circ {\T}^r_{v_1\cdots v_k}) = 0$.
  But
  \begin{eqnarray*}
    \alpha (\Q \circ {\T}^r_{v_1\cdots v_k}) (0)(\ee_1)\cdots(\ee_k) & = & ({\T}^r_{v_1\cdots v_k})^*(\alpha(\Q))(0)(\ee_1)\cdots(\ee_k) \\
    & = & \alpha(\Q)(r)(v_1)\cdots(v_k).
  \end{eqnarray*}
  Hence,
  $\alpha(\Q)(r)(v_1)\cdots(v_k) = 0$,
  for any $r \in \U$ and any $v_1,\ldots,v_k \in \RR^n$.
  Therefore,
  $\alpha(\Q) = 0$.
  Now,
  if $\alpha(\P) = \beta(\P)$ for any $k$-plot $\P$,
  then $(\alpha -\beta)(\P) = 0$,
  thus $\alpha-\beta = 0$ and $\alpha = \beta$.
\end{proof}

\begin{article}\artlabel{Pushing forms onto quotients}
  \addcontentsline{toc}{section}{\small\hspace{10pt} Pushing forms onto quotients}
  \label{Pushing-forms-onto-quotients}
  Let $\X$ and $\X'$ be two diffeological spaces.
  Let $\pi: \X \to \X'$ be a subduction \art{What-is-a-subduction},
  and let $\alpha$  be a differential $k$-form on $\X$.
  The $k$-form $\alpha$ is the pullback of  a $k$-form $\beta$ defined on $\X'$,
  $\alpha = \pi^*(\beta)$,
  if and only if,
  for any two plots $\P$ and $\Q$ of $\X$  such that $\pi \circ \P = \pi \circ \Q$,
  $\alpha(\P) = \alpha(\Q)$.
  We also say that $\beta$ is the {\em pushforward} of $\alpha$ by $\pi$.
  The differential forms of $\X$ satisfying this property  may be called {\em basic forms}\index{Basic differential form},
  with respect to $\pi$.
  
  \Note{1} For every integer $k$,
  the pullback $\pi^*: \Omega^k(\X') \to \Omega^k(\X)$ is always a smooth linear map as soon as $\pi$ is smooth \art{Pullbacks-of-differential-forms}.
  The previous proposition is a characterization of the image of $\pi^*$, when $\pi$ is a subduction.
  
  \Note{2} This property can be expressed with the help of a diagram and will be used further this way \art{Integration-bundles-of-closed-2-forms}.
  Consider the pullback of $\pi$ by itself
  $$%
    \pi^*(\X) = \{ (x_1,x_2) \in \X \times \X \mid \pi(x_1) = \pi(x_2)\},
    $$%
  with projections $\pr_1$ and $\pr_2$.
  \begin{center}
    \begin{tikzcd}[column sep=large, row sep=large, every label/.append style = {font = \small}]
      \pi^*(\X) \arrow[d, swap, "\pr_2"] \arrow[r,"\pr_1"] & \X \arrow[d, "\pi"]  \\
      \X \arrow[r, swap, "\pi"] & \X'
    \end{tikzcd}
  \end{center}
  Then,
  $\alpha$ is basic with respect to $\pi$ if and only if $\pr_1^*(\alpha) - \pr_2^*(\alpha) = 0$,
  {\em i.e.\/} if and only if $\pr_1^*(\alpha) = \pr_2^*(\alpha)$.
\end{article} %% Pushing forms onto quotients

\begin{proof}
  One way is clear,
  if $\pi^*(\beta) = \alpha$,
  then for every pair of plots such that $\pi \circ {\P} = \pi \circ \Q$,
  we have ${\P}^*(\alpha) = {\P}^*(\pi^*(\beta)) = (\pi \circ {\P})^*(\beta) = (\pi \circ \Q)^*(\beta) = \Q^*(\pi^*(\beta))= \Q^*(\alpha)$.
  Conversely,
  let $\alpha$ be a $k$-form of $\X$ such that for every pair of plots ${\P}$ and $\Q$ of $\X$,
  $\pi \circ {\P} = \pi \circ \Q$ implies ${\P}^*(\alpha) = \Q^*(\alpha)$.
  Let $\F: {\W} \to \X'$ be a plot.
  Since $\pi$ is a subduction,
  there exists a family $\cP = \{ {\P}_i: {\W}_i \to \X\}_{i \in \cI}$ of plots,
  such that $\bigcup_{i \in \cI} {\W}_i = {\W}$,
  and for every $i \in \cI$,
  $\pi \circ {\P}_i = \F\restriction {\W}_i$.
  We shall say that $\cP$ is a covering of $\F$.
  So,
  let us define $\beta_i = {\P}_i^*(\alpha)$,
  $\beta_i$ is a form defined on ${\W}_i$.
  Now,
  $\beta_i\restriction {\W}_i \cap {\W}_j = {\P}_i^*(\alpha) \restriction {\W}_i \cap {\W}_j$,
  but $\pi \circ {\P}_i \restriction {\W}_i \cap {\W}_j = \pi \circ {\P}_j\restriction {\W}_i\cap {\W}_j = \F \restriction {\W}_i\cap {\W}_j$,
  then $\beta_i\restriction {\W}_i\cap {\W}_j = \beta_j\restriction {\W}_i\cap {\W}_j$.
  Hence,
  there exists a form $\beta(\F) \in \Omega^k({\W})$, such that
  $$%
    \beta(\F) = \sup_{i \in \cI} {\P}_i^*(\alpha), \text{ with } \beta(\F)\restriction {\W}_i = \beta_i.
    $$%
  The $\sup_{i\in \cI}$ denotes,
  as usual,
  the smallest common extension.
  But $\beta(\F)$ could depend {\em a priori\/} on the choice of the covering $\cP$.
  Let us prove that it does not. Let $\cP' = \{{\P}'_{i'}: {\W}'_{i'} \to \X\}_{i' \in \cI'}$ be another covering of $\F$,
  and $\beta'(\F)$ be the form associated with $\cP'$.
  Then,
  the union $\cP'' = \{ {\P}''_{i''}: {\W}''_{i''} \to \X\}_{i'' \in \cI''}$ of the coverings $\cP$ and $\cP'$ is another covering of $\F$.
  But the form $\beta''(\F)$, associated with $\cP''$,
  coincides by construction with $\beta(\F)$,
  by choosing just the members of $\cP$ for building it,
  and coincides with $\beta'(\F)$,
  by choosing just the members of $\cP'$.
  Hence,
  $\beta(\F)$ does not depend on the covering.
  
  Now,
  we have to prove that the map $\beta$,
  we have just defined,
  is a $k$-form of $\X'$.
  Let $\phi: {\T} \to {\W}$ be a smooth parametrization.
  Let $\cP = \{ {\P}_i: {\W}_i \to \X\}_{i \in \cI}$ be a covering of $\F$,
  let $\cQ = \{{\Q}_i: {\T}_i \to \X\}_{i \in \cI}$ be the pullback of the covering $\cP$,
  that is,
  ${\T}_i = \phi^{-1}({\W}_i)$ and $\Q_i = {\P}_i \circ \phi$.
  By the previous construction we have:
  $\beta(\F \circ \phi) =
  \sup_{i \in \cI}\Q_i^*(\alpha) =
  \sup_{i \in \cI}({\P}_i \circ \phi)^*(\alpha) =
  \sup_{i \in \cI} \phi^*({\P}_i^*(\alpha)) =
  \phi^*(\sup_{i \in \cI} {\P}_i^*(\alpha)) =
  \phi^*(\beta(\F))$.
  Therefore,
  $\beta$ is a $k$-form of $\X'$.
  We still need to check that $\pi^*(\beta) = \alpha$.
  By definition \art{Pullbacks-of-differential-forms},
  $\alpha$ is the pullback of $\beta$ by $\pi$ if and only if,
  for every plot ${\P}$ of $\X$, $\alpha({\P}) = \beta(\pi \circ {\P})$.
  But,
  by construction,
  $\beta(\pi \circ {\P}) =  \alpha({\P})$,
  thus $\alpha = \pi^*(\beta)$.
  
  For Note 2, we remark that a plot of $\pi^*(\X)$ is just a pair $(\P,\Q)$ of plots of $\X$,
  such that $\pi \circ \P = \pi \circ \Q$,
  and that $[\pr_1^*(\alpha) - \pr_2^*(\alpha)](\P, \Q) = \alpha(\P) - \alpha(\Q)$.
\end{proof}

\begin{article}\artlabel{Vanishing forms on quotients}
  \addcontentsline{toc}{section}{\small\hspace{10pt} Vanishing forms on quotients}
  \label{Vanishing-forms-on-quotients}
  Let $\X$ and $\X'$ be two diffeological spaces and $f:\X \to \X'$ be a subduction.
  Let $\alpha$ be a $p$-form on $\X'$,
  $\alpha \in \Omega^p(\X')$,
  $p \in \NN$.
  Then,
  $f^*(\alpha) = 0$ if and only if $\alpha = 0$.
  Equivalently,
  for any two $p$-forms $\alpha$ and $\beta$ on $\X'$ and for  every subduction $f$ from $\X$ to $\X'$,
  \begin{equation}
    \renewcommand{\theequation}{$\diamondsuit$}
    f^*(\alpha) = f^*(\beta) \ \Rightarrow \ \alpha = \beta.
  \end{equation}
  In other words,
  for every subduction $f:\X \to \X'$,
  the pullback by $f^* : \Omega^p(\X') \to \Omega^p(\X)$ is injective.
  
  \Note~This implies in particular that if a diffeological space $\X$ has a finite dimension $n \in \NN$ \art{Dimension-of-a-diffeological-space},
  then every $n+k$ differential form,
  with $k>0$,
  is zero.
  Formally,
  \begin{equation}
    \renewcommand{\theequation}{$\heartsuit$}
    \dim(\X) = n<\infty, \text{ implies } \Omega^{n+k}(\X) = \{0\}, \text{ for all } k>0.
  \end{equation}
\end{article} %% Vanishing-forms-on-quotients

\begin{proof}
  Let us consider a plot $\P : \U \to \X'$, $r \in \U$.
  Since $f$ is a subduction,
  there exist an open neighborhood $\V$ of $r$ and a plot $\Q: \V \to \X$ such that $f \circ \Q = \P \restriction \V$,
  then $\alpha(\P \restriction \V) = \alpha(f \circ \Q) = (f \circ \Q)^*(\alpha) = \Q^*(f^*(\alpha)) = 0$.
  Thus,
  since $\alpha(\P \restriction \V) = \alpha(\P) \restriction \V$,
  the form $\alpha(\P)$ vanishes locally at each point $r \in \U$,
  then $\alpha(\P) = 0$.
  Since $\alpha(\P) = 0$ for all plots $\P$, $\alpha = 0$.
  Now,
  let us assume that $\dim(\X) = n < \infty$.
  Then,
  there exists a generating family $\cF$ such that the dimension of all its elements is less than or equal to $n$.
  But the space $\X$ is the quotient of $\Nebula(\cF)$ by the evaluation $\ev_\cF$,
  from $\Nebula(\cF)$ onto $\X$ \art{Nebula-of-a-generating-family}.
  Now,
  the nebula is the sum of the domains of the elements of $\cF$,
  whose dimensions are less or equal than $n$.
  Hence,
  for $k>0$,
  every $(n+k)$-form on the nebula is zero,
  since
  every linear $(n+k)$-form on $\RR^m$ is zero,
  for $m \leq n$ \art{Exterior-monomials-and-basis-of-Lambda-p-E}.
  Thus,
  for every $(n+k)$-form $\alpha$ of $\X$,
  with $k>0$,
  $\ev_\cF^*(\alpha) = 0$.
  By application of the first proposition, $\alpha = 0$.
\end{proof}

\begin{article}\artlabel{The values of a differential form}
  \addcontentsline{toc}{section}{\small\hspace{10pt} The values of a differential form}
  \label{The-values-of-a-differential-form}
  Let $\X$ be a diffeological space.
  Let us recall that a plot {\em centered}\index{Centered plot} at $x \in \X$ is a plot ${\P}:
  \U \to \X$ such that $0 \in \U$ and ${\P}(0) = x$.
  Let $p$ be any integer,
  and let us consider the following relation:
  %
  \begin{enumerate}
    \item[($\diamondsuit$)\ ] We say that two $p$-forms $\alpha,
    \beta \in \DForms^p(\X)$ {\em have the same value in $x$}\index{Value of differential form} if and only if,
    for every plot ${\P}$ centered at $x$,
    we have $\alpha({\P})(0) = \beta({\P})(0)$,
    or,
    which is equivalent,
    $(\alpha-\beta)({\P})(0) = 0$.
  \end{enumerate}
  %
  Having the same value at the point $x$ is an equivalence relation,
  we shall denote it by $\sim_x$.
  The class of $\alpha$ for this relation will be called the {\em value of $\alpha$ at the point $x$},
  and will be denoted by $\alpha_x$.
  The set of all the values at the point $x$,
  of all the $p$-forms of $\X$,
  will be denoted by:
  $$%
    \Form_x^p(\X) = \quotient{\DForms^p(\X)}{\sim_x} = \{ \alpha_x \mid \alpha \in \DForms^p(\X) \}.
    $$%
  An element $a \in \Form_x^p(\X)$ will be called a {\em $p$-form of $\X$ at the point $x$},
  the {\em basepoint} of $a$,
  and the space $\Form_x^p(\X)$ will be called the {\em space of $p$-forms of $\X$ at the point $x$}.
  We shall say that a form $\alpha$ {\em vanishes at the point $x$} if and only if,
  for every plot $\P$ centered at $x$,
  $\alpha({\P})(0) = 0$,
  that is, if $\alpha$ is equivalent to the zero form in this point,
  $\alpha\sim_x 0$.
  We shall denote that by $\alpha_x = 0_x$.
  Then,
  two $p$-forms $\alpha$ and $\beta$ have the same value at the point $x$ if and only if their difference vanishes at this point,
  $$%
    \alpha_x = \beta_x \text{ if and only if } (\alpha-\beta)_x = 0_x.
    $$%
  Now,
  it is clear that the set $\{\alpha \in \DForms^p(\X) \mid \alpha_x = 0_x\}$ of the $p$-forms of $\X$ vanishing at the point $x$ is a vector subspace of $\DForms^p(\X)$,
  and it is also clear that
  $$%
    \Form_x^p(\X) = \DForms^p(\X)/\{\alpha \in \DForms^p(\X) \mid \alpha_x = 0_x\}.
    $$%
  Thus,
  as a quotient of a diffeological vector space by a vector subspace,
  the space $\Form_x^p(\X)$ is naturally a diffeological vector space \art{Quotient-of-diffeological-vector-spaces}.
  The addition and the multiplication by a scalar on $\Form_x^p(\X)$ are given by
  $$%
    \alpha_x + \beta_x = (\alpha + \beta)_x, \text{ and } s(\alpha_x) = (s\alpha)_x,
    $$%
  where $\alpha, \beta \in \DForms^p(\X)$ and $s \in \RR$.
  Naturally,
  the zero form $0_x$ is the zero of the vector space $\Form_x^p(\X)$.
\end{article} %% The-values-of-a-differential-form

\begin{article}\artlabel{Differential forms through generating families}
  \addcontentsline{toc}{section}{\small\hspace{10pt} Differential forms through generating families}
  \label{Differential-forms-through-generating-families}
  Let $\X$ be a diffeological space,
  and let $\cF$ be a generating family of its diffeology $\cD$ \art{Generating-diffeology}.
  A collection $\{\alpha_\F\}_{\F \in \cF}$ of smooth $k$-forms is made of the values of a differential form $\alpha \in \Omega^k(\X)$,
  that is,
  $\alpha_\F = \alpha(\F)$
  ---~or equivalently $\alpha_\F = \F^*(\alpha)$ \art{Pullbacks-by-the-plots}~---
  if and only if the following two conditions are fulfilled:
  
  1) For all $\F \in \cF$, $\alpha_\F$ is a smooth $k$-form defined on $\Dom(\F)$.
  
  2) For all $\F, \F' \in \cF$,
  for every smooth parametrization $\P$ of the domain of $\F$,
  and for every smooth parametrization $\P'$ of the domain of $\F'$,
  $$%
    \F \circ {\P} = \F' \circ {\P}' \ \Rightarrow \ {\P}^*(\alpha_\F) = {\P}'^*(\alpha_{\F'}).
    $$%
  
  \Note{1} Every differential $k$-form $\alpha$ of $\X$,
  pulled back by the elements of the family $\cF$,
  satisfies these properties.
  It seems that there is no better characterization of $\alpha$ using only its values on the elements of the family $\cF$.
  We cannot hope for anything really better since all the plots of $\X$ form a generating family (see Note 2).
  However,
  in some cases, in particular for manifolds but not only,
  this condition is useful and used \art{Differential-forms-on-manifolds}.
  
  \Note{2} Applied to the whole diffeology $\cD$,
  which is obviously a generating family,
  this condition is reduced to the compatibility condition \art{Differential-forms-on-diffeological-spaces},
  $$%
    \P' = \P \circ \F \ \Rightarrow \ \alpha_{\P'} = \F^*(\alpha_\P),
    $$%
  where $\P$ is a plot of $\X$ and $\F$ is a smooth parametrization in $\Dom(\P)$.
  This is another way to talk about a differential $k$-form $\alpha$,
  as the family of smooth $k$-forms $\{\alpha_\P\}_{\P \in \cD}$ such that $\alpha_\P$ is a smooth $k$-form defined on $\Dom(\P)$,
  satisfying the compatibility condition above.
  
  \Note{3} Consider the diffeology $\cD$ itself as a generating family.
  Let $\tilde\alpha = \ev^*(\alpha)$ be the pullback of $\alpha$ by the evaluation map  $\ev : \cN \to \X$,
  where $\cN$ is the nebula of $\cD$ \art{Nebula-of-a-generating-family}.
  Then $\tilde\alpha$ is invariant by the monoid $\cM$ of smooth fonctions from $\cN$ to itself that project on the identity of $\X$\,:
  $$%
    \cM = \{ \phi \in \Cinfty(\cN) \mid \phi \circ \ev = \ev \}.
    $$%
  Conversely,
  the subspace of $\Omega^k(\cN)$,
  made of $k$-forms invariant by $\cM$,
  identifies with $\Omega^k(\X)$.
  This remark plays a important role in the definition of the \v{C}ech cohomology of a diffeological space,
  and the relation between this cohomology and the De Rham cohomology;
  see \cite{PIZ21b}.
\end{article} %% Differential-forms-through-generating-families

\begin{proof}
  Let $\Nebula(\cF) = \{ (\F,r) \mid \F \in \cF  \text{ and } r \in \Dom(\F)\}$ \art{Nebula-of-a-generating-family}.
  We know that the evaluation map $\ev: \Nebula(\cF) \to \X$,
  defined by $\ev(\F,r) = \F(r)$, is a subduction \art{Nebula-of-a-generating-family}.
  Let $\alpha \in \Omega^k(\X)$,
  the pullback of $\alpha$ by $\ev$ is exactly given by
  $
  \ev^*(\alpha) = \{\F^*(\alpha)\}_{\F \in \cF}
  $.
  Conversely,
  let $a \in \Omega^k(\Nebula(\cF))$,
  that is,
  $a = \{a_\F\}_{\F \in \cF}$.
  The form $a$ is a pullback of $\alpha \in \Omega^k(\X)$ if and only if,
  for any pair of plots ${\P}$ and ${\P}'$ of $\Nebula(\X)$,
  $\ev \circ {\P} = \ev \circ {\P}'$ implies ${\P}^*(a) = {\P}'^*(a)$ \art{Pushing-forms-onto-quotients}.
  Now,
  a plot ${\P}$ of $\Nebula(\cF)$ is just,
  locally at every point,
  a smooth parametrization in some domain $\U = \Dom(\F)$,
  with $\F \in \cF$.
  Then ${\P}^*(a) = {\P}^*(a\restriction \F) = {\P}^*(a_\F)$,
  at least locally.
  On the other hand,
  $\ev \circ {\P}(s) = \F({\P}(s))$,
  that is,
  $\ev \circ {\P} = {\P} \circ \F$.
  Therefore,
  the last condition writes $\F \circ \P = \F' \circ \P'$,
  which implies $\P^*(a_\F) = \P'^*(a_{\F'})$.
  Now,
  let us consider the special case where $\cF = \cD$.
  Let us assume that the condition above is satisfied,
  let $\F, \F' \in \cD$,
  let $\P : \Dom(\F') \to \Dom(\F)$ be a smooth parametrization,
  let $\P' = \id_{\Dom(\F')}$,
  and let us assume that $\F \circ \P = \F' \circ \P'$,
  that is,
  $\F'= \F \circ \P$.
  Then,
  condition 2) above gives $\alpha_{\F'} = \P^*(\alpha_\F)$.
  This is the compatibility condition of \art{Differential-forms-on-diffeological-spaces},
  where the role of $\P$ and $\F$ have been inverted.
  Conversely,
  let us assume that the compatibility condition of \art{Differential-forms-on-diffeological-spaces} is satisfied,
  and let $\P$, $\P'$, $\F$, $\F'$ as above,
  such that $\F \circ \P = \F' \circ \P'$. So, $\alpha(\F \circ \P) = \alpha(\F' \circ \P')$,
  thus $\P^*(\alpha(\F)) = {\P'}^*(\alpha(\F'))$,
  that is, $\P^*(\alpha_\F) = {\P'}^*(\alpha_{\F'})$.
\end{proof}

\begin{article}\artlabel{Differential forms on manifolds}
  \addcontentsline{toc}{section}{\small\hspace{10pt} Differential forms on manifolds}
  \label{Differential-forms-on-manifolds}
  Let ${\M}$ be a manifold modeled on some diffeological vector space $\E$ \art{Manifolds-as-diffeologies}.
  Let $\cA$ be an atlas,
  that is,
  a generating family of ${\M}$ whose elements are local diffeomorphisms from $\E$ to $\X$.
  A family of smooth $k$-forms $\{\alpha_\F\}_{\F \in \cA}$ are the values of a differential form $\alpha \in \Omega^k({\M})$ if and only if,
  for every nonempty transition function $\phi = \F^{-1} \circ \F'$,
  defined on $\F'^{-1}(\Val(\F))$, we have
  $$%
    \alpha_\F' \restriction \F'^{-1}(\Val(\F)) = \phi^*(\alpha_\F).
    $$%
  This condition is the specialization of the characterization of forms on manifolds,
  through generating families \art{Differential-forms-through-generating-families}.
  This is actually the usual way of defining differential forms in classical differential geometry,
  in the case of finite dimensional manifolds.
\end{article} %% Differential forms on manifolds

\begin{article}\artlabel{Linear differential forms on diffeological vector spaces}%
  \addcontentsline{toc}{section}{\small\hspace{10pt} Linear differential forms on diffeological vector spaces}%
  \label{Linear-differential-forms-on-diffeological-vector-spaces}
  Let $\E$ be a diffeological vector space on $\RR$ \art{Diffeological-vector-spaces}.
  Let $\E^* = \E * \RR = \Lin(\E,\RR)$,
  and let $\E^*_\infty$ be the space of smooth linear maps from $\E$ to $\RR$,
  $\E^*_\infty = \Lin^\infty(\E,\RR) = \Lin(\E,\RR) \cap \Cinfty(\E, \RR)$.
  Every element $\alpha \in \E^*_\infty$ defines naturally a differential $1$-form $\underline \alpha \in \Omega^1(\E)$.
  Indeed,
  let $\P : \U \to \E$ be an $n$-plot, so $\alpha \circ \P \in \Cinfty(\U,\RR)$.
  We define,
  for all $r \in \U$ and $\delta r \in \RR^n$,
  $$%
    \underline\alpha(\P)_r(\delta r) = d[\alpha \circ \P]_r(\delta r) = {\partial [\alpha \circ \P(r)] \over \partial r}(\delta r).
    $$%
  This makes sense because $\alpha \circ \P$ is a smooth real function on $\U$,
  and then $\underline\alpha(\P)$ is a smooth $1$-form on $\U$.
  Moreover,
  the map $\alpha \mapsto \underline \alpha$ is linear and injective,
  indeed,
  for all $u \in \E$, $\alpha(u) = \underline \alpha(s \mapsto su)_{s=0}(1)$.
  Thus,
  we can identify $\alpha$ with $\underline \alpha$.
  In other words,
  $\E^*_\infty$ can be regarded as a subset of $\Omega^1(\E)$,
  which is satisfactory.
  This construction is a particular case of the following general construction,
  which appears in some examples coming from mathematical physics;
  see \art{On-the-intersection-form-of-a-surface-I} and after.
  Let $\Lambda^k(\E)$ be the vector space of $k$-linear forms of $\E$ and $\omega \in \Lambda^k(\E)$.
  We shall say that $\omega$ is a  {\em smooth linear $k$-form of $\E$} if,
  for all plots $\Q : \V \to \E^k$,
  $\omega \circ \Q \in \Cinfty(\V,\RR)$.
  Note that $\omega \circ \Q(s) = \omega(\Q_1(s))\cdots(\Q_k(s))$,
  with $\Q = (\Q_1,\ldots,\Q_k)$.
  We shall denote by $\Lambda^k_\infty(\E)$ the space of smooth linear $k$-forms of $\E$,
  it is a vector subspace of $\Lambda^k(\E)$.
  Let $\P : \U \to \E$ be an $n$-parametrization and $\P^k : \U^k \to \E^k$ be the $k$-th power of $\P$,
  $$%
    \text{for all $(r_1,\ldots,r_k) \in \U^k$}, \ \P^k(r_1,\ldots,r_k) = (\P(r_1),\ldots,\P(r_k)).
    $$%
  Then,
  let us define $\underline \omega$ by
  %
  \begin{equation}
    \renewcommand{\theequation}{$\diamondsuit$}
    \underline \omega(\P)_r(\delta_1 r, \ldots, \delta_k r) =
    \left.{\partial^k [\omega \circ \P^k(r_1,\ldots,r_k)]
    \over \partial r_1 \cdots \partial r_k}\right|_{r_1=\cdots=r_k=r}(\delta_1 r) \cdots (\delta_k r),
  \end{equation}
  %
  where $r \in \U$ and the $\delta_i r$ belong to $\RR^n$.
  For $k =1$,
  we find again the above definition of a smooth linear $1$-form.
  For $k = 2$, $r \in \U$ and $u_1, u_2 \in \RR^n$,
  we have
  %
  \begin{align*}
    \underline \omega(\P)_r(u_1,u_2) & = \left.{\partial^2 [\omega ( \P(r_1),\P(r_2))] \over \partial r_1  \partial r_2}\right|_{r_1=r_2=r} (u_1,u_2) \\
    & = \D\{r_1 \mapsto \D[r_2 \mapsto \omega(\P(r_1),\P(r_2))](r_2=r)(u_2)\}(r_1 = r)(u_1),
  \end{align*}
  %
  and so on.
  Now,
  $\underline \omega$ defined by $(\diamondsuit)$ is a smooth $k$-form of $\E$.
  The map $\omega \mapsto \underline \omega$ is linear and injective.
  Indeed, for any $k$ vectors $u_1,\ldots,u_k$ of $\E$,
  we have
  $$%
    \omega(u_1,\ldots,u_k) = \underline \omega(\P)_0(\ee_1,\ldots,\ee_k), \text{ with } \P : (t_1,\ldots,t_k) \mapsto \sum_{i=1}^k t_i u_i,
    $$%
  where the $t_i$ belong to $\RR$ and the $\ee_i$ are the
  vectors of the canonical basis of $\RR^k$.
  Therefore,
  we can regard $\Lambda^k_\infty(\E)$ as a vector subspace of $\Omega^k(\E)$.
\end{article} %% Linear-differential-forms-on-diffeological-vector-spaces

\begin{proof}
  Since $\alpha \in \Cinfty(\E,\RR)$ and $\P$ is an $n$-plot of $\E$,
  $\alpha \circ \P \in \Cinfty(\U, \RR)$ and the notation $d[\alpha \circ \P]$ makes sense,
  it is a smooth $1$-form on $\U$,
  $d[\alpha \circ \P] \in \Cinfty(\U, \Lin(\RR^n,\RR)) = \Cinfty(\U,\Lambda^1(\RR^n))$.
  Thus,
  the first condition for $\underline \alpha$ to be a differential $1$-form of $\E$ is satisfied.
  Next,
  let $\F : \V \to \U$ be a smooth parametrization,
  then $\underline\alpha(\P \circ \F) = d[\alpha \circ (\P \circ \F)] = d[(\alpha \circ \P) \circ \F] = \F^*[d(\alpha \circ \P)] = \F^*(\underline \alpha(\P))$.
  Hence,
  the second condition is satisfied and $\underline\alpha$ is a differential $1$-form of $\E$.
  The linearity of $\alpha \mapsto \underline \alpha$ is clear.
  Let us consider its kernel: let $\alpha$ be such that $\underline \alpha = 0$,
  that is,
  for all plots $\P$,
  $d[\alpha \circ \P] = 0$.
  Let $u \in \E$ be any vector and let $\P : s \mapsto su$,
  with $s \in \RR$,
  $\P$ is a $1$-plot by the very definition of diffeological vector spaces.
  Then, $d[\alpha \circ \P] = 0$,
  and $\alpha \circ \P \in \Cinfty(\RR,\RR)$ implies that $\alpha \circ \P$ is a constant map,
  thus $\alpha \circ \P(1) = \alpha \circ \P(0)$,
  that is,
  $\alpha (u) = \alpha(0) =0$.
  Hence,
  $\alpha = 0$ and $\alpha \mapsto \underline \alpha$ is injective.
  Now,
  since $\P : s \mapsto su$ is a plot of $\E$,
  we have $\underline \alpha(\P)_{s=0}(1) = \D(\alpha \circ \P)(0)(1) = \D(s \mapsto \alpha(su))(0)(1) = \D(s \mapsto s \alpha(u))(0)(1) = \D(s \mapsto [\alpha(u)] \times s)(0)(1) = \alpha(u)$.
  
  Let us consider the general case.
  First of all,
  the partial derivative is linear in each of its arguments,
  thus $\underline \omega(\P)_r$ is multilinear.
  Since $\underline \omega \circ \P^k$ is smooth,
  all of its partial derivatives,
  of any order,
  are smooth,
  hence $r \mapsto \underline \omega(\P)_r$ is smooth.
  Now,
  the partial derivatives are totally symmetric,
  thus,
  for any couple of indices $j$ and $\ell$,
  we have
  \begin{multline*}
    \left.{\partial^k [\omega \circ \P^k(r_1,\ldots, r_j, \ldots, r_\ell, \ldots, r_k)] \over \partial r_1 \cdots \partial r_j \cdots \partial r_\ell \cdots \partial r_k}\right|_{r_i=r}(u_1,\ldots, u_j,\ldots, u_\ell,\ldots, u_k) \\
    = \left.{\partial^k [\omega \circ \P^k(r_1,\ldots, r_j, \ldots, r_\ell, \ldots, r_k)] \over \partial r_1 \cdots \partial r_\ell \cdots \partial r_j \cdots \partial r_k}\right|_{r_i=r}(u_1,\ldots, u_\ell,\ldots, u_j,\ldots, u_k).
  \end{multline*}
  But $\omega \circ \P^k(r_1,\ldots, r_j, \ldots, r_\ell, \ldots, r_k) = - \omega \circ \P^k(r_1,\ldots, r_\ell, \ldots, r_j, \ldots, r_k)$,
  hence
  %
  \begin{multline*}
    \left.{\partial^k [\omega \circ \P^k(r_1,\ldots, r_j, \ldots, r_\ell, \ldots, r_k)] \over \partial r_1 \cdots \partial r_\ell \cdots \partial r_j \cdots \partial r_k}\right|_{r_i=r}(u_1,\ldots, u_\ell,\ldots, u_j,\ldots, u_k) \\
    = - \left.{\partial^k [\omega \circ \P^k(r_1,\ldots, r_\ell, \ldots, r_j, \ldots, r_k)] \over \partial r_1 \cdots \partial r_\ell \cdots \partial r_j \cdots \partial r_k}\right|_{r_i=r}(u_1,\ldots,
    u_\ell,\ldots, u_j,\ldots, u_k).
  \end{multline*}
  %
  Hence,
  $\underline \omega(\P)_r(u_1,\ldots, u_j,\ldots, u_\ell,\ldots, u_k) = -\ \underline \omega(\P)_r(u_1,\ldots, u_\ell,\ldots, u_j,\ldots, u_k)$,
  and $\underline \omega(\P)_r$ is totally antisymmetric.
  Therefore, $\underline \omega(\P) \in \Cinfty(\U, \Lambda^p(\RR^n))$.
  Now,
  let $\F : \V \mapsto \U$ be some smooth $m$-parametrization.
  Let $s_i$ denote a point in $\V$ and $r_i = \F(s_i)$.
  Let $s \in \V$, $r = \F(s)$,
  $\M = \D(\F)(s)$ and let $v_1,\ldots,v_k \in \RR^m$.
  We have,
  %
  \begin{align*}
    \underline \omega(\P \circ \F)_s (v_1,\ldots,v_k) & = \left.{\partial^k [\omega (\P \circ \F(s_1),\ldots,\P \circ \F(s_k))] \over \partial s_1 \cdots \partial s_k}\right|_{s_i=s} (v_1,\ldots,v_k) \\
    & =\left.{\partial^k [\omega (\P(r_1),\ldots,\P(r_k))] \over \partial r_1 \cdots \partial r_k}\right|_{r_i=r} (\M v_1,\ldots,\M v_k) \\
    & = \underline \omega(\P)_{\F(s)} (\D(\F)(s)(v_1),\ldots,\D(\F)(s)(v_k)) \\
    & = \F^*(\underline \omega(\P))_s (v_1,\ldots,v_k).
  \end{align*}
  %
  Therefore,
  $\underline \omega(\P \circ \F) = \F^*(\underline \omega(\P))$,
  and $\underline \omega$ is a differential $k$-form of $\E$.
  %
  Let us consider now $\underline \omega(\P)_0(\ee_1,\ldots,\ee_k)$ with $\P(r) = \sum_{i=1}^k t_i u_i$ and $r = (t_1,\ldots,t_k)$,
  as defined above.
  The plot $\P : \RR^k \to \E$ is a linear map,
  thus $\omega \circ \P^k(r_1,\ldots,r_k) = \omega(\P(r_1),\ldots,\P(r_k)) = \P^*(\omega)(r_1,\ldots,r_k)$,
  by definition of the pullback of a linear form by a linear map.
  But since $\P^*(\omega)$ is a linear $k$-form of $\RR^k$,
  $\P^*(\omega)$ is proportional to the canonical volume,
  that is,
  the determinant.
  Let $\P^*(\omega) = c \times \det$,
  the coefficient $c$ is given by $c = c \times \det(\ee_1,\ldots,\ee_k) = \P^*(\omega)(\ee_1,\ldots,\ee_k) = \omega(\P(\ee_1),\ldots,\P(\ee_k)) = \omega(u_1,\ldots,u_k)$.
  Thus,
  $\omega \circ \P^k(r_1,\ldots,r_k) = \omega(u_1,\ldots,u_k) \times \det(r_1,\ldots,r_k)$.
  Now, since the determinant is multilinear,
  we get
  $$%
    \left.{\partial^k \det(r_1,\ldots,r_k) \over \partial r_1 \cdots \partial r_k}\right|_{r_i=0} (\ee_1,\ldots,\ee_k) = \det(\ee_1,\ldots,\ee_k) =1.
    $$%
  Therefore,
  $\underline \omega(\P)_0(\ee_1,\ldots,\ee_k) = \omega(u_1,\ldots,u_k)$.
  The map $\omega \mapsto \underline \omega$,
  from $\Lambda^k_\infty(\E)$ to $\Omega^k(\E)$,
  is clearly linear.
  If $\underline \omega = 0$,
  then,
  applied to any linear plot $\P$ as defined just above,
  we get $\omega(u_1,\ldots,u_k) = 0$ for all $k$ vectors $u_1,\ldots,u_k$ of $\E$, and $\omega = 0$.
  The map $\omega \mapsto \underline \omega$ is thus injective.
\end{proof}

\begin{article}\artlabel{Volumes on manifolds and diffeological spaces}
  \addcontentsline{toc}{section}{\small\hspace{10pt} Volumes on manifolds and diffeological spaces}
  \label{Volumes-on-manifolds-and-diffeological-spaces}
  Let $\M$ be a finite dimensional manifold \art{Manifolds-as-diffeologies},
  and let $n = \dim(\M)$.
  Recall that a {\em volume} of $\M$ is any $n$-form $\omega$ of $\M$ nowhere vanishing.
  The set of volumes of $\M$ will be denoted by
  $$%
    \Volumes(\M) = \{ \omega \in \Omega^n(\M) \mid \forall x \in {\M}, \omega_x \neq 0 \}.
    $$%
  The manifold $\M$ is said to be {\em orientable} if there exists a volume;
  if $\Volumes(\M)$ is not empty.
  A pair $(\M, \vol)$,
  where $\vol \in \Volumes(\M)$,
  is called
  an {\em oriented manifold}\index{Oriented manifold},
  and the manifold $\M$ is said to be {\em oriented} by $\vol$.
  The following proposition is a classic result.
  
  1. If $\M$ is oriented by $\vol$,
  then,
  for every $n$-form $\omega \in \Omega^n(\M)$,
  there exists a function $f \in \Cinfty(\M,\RR)$ such that
  $$%
    \omega = f \times \vol, \text{ and we denote } f = {\omega \over \vol}.
    $$%
  %
  More generally,
  let $\X$ be a diffeological space of dimension $n < \infty$.
  By analogy with the definition above,
  we could call {\em volume}\index{Volume} of $\X$ any differential $n$-form $\omega$ of $\X$ nowhere vanishing.
  We denote the same way,
  by $\Volumes(\X)$,
  the set of all the volumes of $\X$.
  In these conditions the following proposition gives more precision.
  
  2. If $\omega$ is a volume of $\X$ and $\dim(\X) = n$,
  then the dimension of $\X$ at each point $x \in \X$ is $n$.
  Moreover,
  there exists $\cF$,
  a parametrized covering family of plots,
  such that for all $\F \in \cF$,
  $\omega(\F)$ is a volume on $\Dom(\F)$.
  
  \Note{1} In the case of an arbitrary diffeological space of finite dimension,
  the property of proportionality between volumes is uncertain.
  There exist indeed diffeological spaces,
  which are not manifolds,
  with volumes;
  see for example \exref{Irrational-tori-are-orientable}.
  But it is not clear in general why,
  except for particular situations,
  two volumes should be proportional.
  
  \Note{2} In the second statement,
  if the diffeology $\gen{\cF}$ generated by $\cF$ \art{Generating-diffeology} is obviously finer than the diffeology $\cD$ of $\X$,
  nothing proves {\em a priori\/} that $\gen{\cF}$  coincides with $\cD$.
\end{article} %% Volumes-on-manifolds-and-diffeological-spaces

\begin{proof}
  1. For the space $\M$,
  to be a manifold of dimension $n$ means that its diffeology is generated by a family $\cA$ of local diffeomorphisms from $\RR^n$ to $\M$ \art{Manifolds-as-diffeologies}.
  Let us assume that there exists a volume $\vol \in \Volumes(\M)$,
  and let $\omega \in \Omega^n(\M)$.
  For any chart $[\F: \U \to {\M}] \in \cA$,
  $\omega(\F)$ and $\vol(\F)$ are $n$-forms.
  But since $\F$ is a local diffeomorphism,
  $\vol(\F)$ is a volume of $\U$.
  Hence,
  there exists a function $f(\F) \in \Cinfty(\U,\RR)$ such that $\omega(\F) = f(\F) \times \vol(\F)$.
  Now,
  let $\phi \in \Diffloc(\RR^n)$ be a transition function between two charts,
  that is,
  $\F'\restriction {\W} = \F \circ \phi$, where ${\W} = \Def(\phi)$.
  Then,
  on the one hand $\omega(\F'\restriction {\W}) =  (f(\F') \times \vol(\F'))\restriction {\W} = f(\F'\restriction {\W}) \times \vol(\F \circ \phi) = f(\F'\restriction {\W}) \times \phi^*(\vol(\F))$,
  and on the other hand $\omega(\F'\restriction {\W}) = \omega(\F \circ \phi) =  \phi^*(\omega(\F)) = \phi^*(f(\F)\times\vol(\F)) = \phi^*(f(\F))\times\phi^*(\vol(\F))$.
  Thus,
  $f(\F'\restriction {\W}) \times \phi^*(\vol(\F)) = \phi^*(f(\F))\times\phi^*(\vol(\F))$,
  which implies $f(\F'\restriction {\W}) = f(\phi^*(f(\F))$,
  that is,
  $f(\F \circ \phi) = f(\F) \circ \phi$.
  Hence, $\F \mapsto f(\F)$ is the expression of a smooth real function defined on ${\M}$ in the atlas $\cA$ \art{Differential-forms-on-manifolds}.
  Therefore,
  there exists a function $f \in \Cinfty(\M,\RR)$ such that $\omega = f\times \vol$.
  
  2.  Let us prove first that if there exists a volume on $\X$ and if $\dim(\X) = n$,
  then the dimension at each point of $\X$ is $n$ \art{The-dimension-map}.
  We know that,
  for all $x \in \X$, $\dim_x(\X)\leq n$ \art{Global-dimension-and-dimension-map}.
  Let us assume that $x \in \X$ and $\dim_x(\X) = p<n$.
  Since $\omega_x\neq 0$,
  there exists a plot ${\P}: \U \to \X$ such that $0 \in \U$,
  ${\P}(0) = x$ and $\omega({\P})(0)\neq 0$ \art{The-values-of-a-differential-form}.
  But $\dim_x(\X) = p$ implies that there exists a family $\cF_x$ of centered plots,
  generating the germ of the diffeology at the point $x$,
  with $\dim(\cF_x) = p$.
  Thus,
  there exist an open neighborhood $\V$ of $0 \in \U$,
  a plot $\F \in \cF_x$ and a smooth parametrization $\Q$ in $\Dom(\F)$ such that ${\P}\restriction \V = \F \circ \Q$.
  Hence,
  $\omega({\P}\restriction \V) = \omega(\F \circ \Q) =  \Q^*(\omega(\F))$.
  But, $\dim(\F)\leq p< n$ implies $\omega(\F) = 0$ \art{Exterior-monomials-and-basis-of-Lambda-p-E}.
  Thus,
  $\Q^*(\omega(\F)) = 0$,
  which implies $\omega({\P})(0) = 0$.
  But $\omega({\P})(0)\neq 0$,
  therefore,
  $\dim_x(\X) = n$.
  Moreover,
  what we proved there is that for each point $x \in \X$ there exists a plot $\phi = \F \restriction \V$ such that $\omega(\phi)$ is a volume.
\end{proof}

%%%%%%%%%%%%%%%%%%%%%%%%%%%%%%%%%%%%%%%%%%%%%%%%%%%%%%%%%%
%
%   Exercises
%
%%%%%%%%%%%%%%%%%%%%%%%%%%%%%%%%%%%%%%%%%%%%%%%%%%%%%%%%%%

\Exercises

\begin{exercise}[Functional diffeology of $0$-forms]
  \label{Functional-diffeology-of-0-forms}
  Let $\X$ be a diffeological space.
  Show that the functional diffeology of $\Omega^0(\X)$,
  defined in \art{Functional-diffeology-of-the-space-of-forms},
  coincides with the functional diffeology of $\Cinfty(\X,\RR)$ defined in \art{Functional-diffeologies}.
\end{exercise} %% Functional-diffeology-of-0-forms

\begin{exercise}[Differential forms against constant plots]
  \label{Differential-forms-against-constant-plots}
  Check that for any differential form $\alpha$ on a diffeological space $\X$,
  if $\P : \U \to \X$ is a locally constant plot,
  then $\alpha(\P) = 0$.
\end{exercise} %% Differential-forms-against-constant-plots

\begin{exercise}[The equi-affine plane]
  \label{The-equi-affine-plane}
  Let us consider the {\em wire plane},
  that is,
  $\RR^2$ equipped with the wire diffeology \art{The-wire-diffeology},
  generated by the $1$-plots.
  Let $\gamma$ be a  $1$-plot of $\RR^2$,
  let $\dot \gamma$ and $\ddot \gamma$ be the first and second derivatives of $\gamma$,
  and let $\omega$ be the canonical volume on $\RR^2$,
  that is,
  $\omega(\V,\W) = \det[\V \ \W]$.
  Let $\alpha(\gamma)$ be the covariant 3-tensor defined on $\Dom(\gamma)$ by
  $$%
    \alpha(\gamma)_t(\delta t, \delta'\!t,\delta''\!t) = \omega(\dot\gamma(t)(\delta t),\ddot\gamma(t)(\delta'\!t)(\delta''\!t)),
    $$%
  where $\delta t, \delta'\!t, \delta''\!t \in \RR$.
  Use the criterion of \art{Differential-forms-through-generating-families} to prove that $\alpha$ defines,
  through the generating family of $1$-plots,
  a differential covariant 3-tensor on the wire plane.
  The cubic root $\sqrt[3]{\alpha}$ appears in the geometric analysis of human arm movements,
  and it is called the {\em equi-affine arc length}\index{Equi-affine arc length};
  see for example \cite{BFBF09}.
\end{exercise} %% The-equi-affine-plane

\begin{exercise}[Liouville $1$-form of the Hilbert space]
  \label{Liouville-1-form-of-the-Hilbert-space}
  Let $\cH_\RR$ be the Hilbert space equipped with the fine diffeology \art{The-fine-standard-Hilbert-space}.
  Let us recall that a parametrization $\P:\U\to \cH_{\RR} \times \cH_{\RR}$ is a plot if for all $r \in \U$ there exist an open neighborhood $\V$ of $r$ and a finite local family $(\lambda_\alpha, (\X_\alpha, \Y_\alpha))$,
  where the $\lambda_\alpha$ are smooth real functions defined on $\V$ and the $(\X_\alpha,\Y_\alpha)$ are vectors of $\cH_{\RR} \times \cH_{\RR}$,
  such that
  $$%
    \P \restriction \V : r \mapsto \sum_{\alpha \in \A}\lambda_\alpha(r)(\X_\alpha, \Y_\alpha), \  \# \A < \infty.
    $$%
  Let $\Lambda(\P \restriction \V)$ be the following $1$-form,
  defined on $\V$:
  $$%
    \Lambda(\P \restriction \V) = {{ 1\over 2}} \sum_{\alpha, \beta \in \A} (\X_\alpha\cdot \Y_\beta -\Y_\alpha\cdot \X_\beta) (\lambda_\alpha d\lambda_\beta - \lambda_\beta d\lambda_\alpha).
    $$%
  
  \Question{1)}~Show that if $\P'$ is another plot of $\cH_{\RR} \times \cH_{\RR}$ such that $\P \restriction \V = \P'\restriction \V$,
  then $\Lambda(\P \restriction \V) = \Lambda(\P'\restriction \V)$.
  
  \Question{2)}~Show that there exists a $1$-form $\Lambda(\P)$ on $\U$ such that,
  for every open subset $\V \subset \U$, $\Lambda(\P \restriction \V) = \Lambda(\P)\restriction \V$.
  
  \Question{3)}~Show that the map $\Lambda: \P\mapsto \Lambda(\P)$ is a $1$-form of $\cH_{\RR}\times \cH_{\RR}$.
\end{exercise} %% Liouville-1-form-of-the-Hilbert-space

\begin{exercise}[The complex picture of the Liouville form]
  \label{The-complex-picture-of-the-Liouville-form}
  Let us identify $\cH_\CC$ with $\cH_{\RR} \times \cH_{\RR}$,
  defined by the unique decomposition $\Z = \X +i \Y$,
  with $(\X, \Y) \in \cH_{\RR} \times \cH_{\RR}$;
  see \exref{HC-is-isomorphic-to-HR-X-HR}.
  Let $\P : r \mapsto \sum_{\alpha \in \A} \lambda_\alpha(r) \Z_\alpha$ be a plot of $\cH_\CC$,
  where $(\lambda_\alpha,\Z_\alpha)_{\alpha \in \A}$ is a local family.
  The $\lambda_\alpha$ are complex valued functions and the $\Z_\alpha$ are vectors of $\cH_\CC$.
  Let us define the symbol $d\Z$ by
  $$%
    d\Z(\P) : r \mapsto \sum_{\alpha \in \A} d\lambda_\alpha(r) \Z_\alpha, \text{ with } \P:r\mapsto \sum_{\alpha \in \A} \lambda_\alpha(r) \Z_\alpha.
    $$%
  In this formula,
  $d\lambda_\alpha$ needs to be understood as
  $$%
    d\lambda_\alpha = da_\alpha + i db_\alpha,
    \text{ where }
    \lambda_\alpha = a_\alpha + i b_\alpha.
    $$%
  Show that the pullback of the Liouville form $\Lambda$ by the isomorphism $\Phi : \Z \mapsto (\X, \Y)$ writes
  $$%
    \Phi^*(\Lambda) = {1\over 2i}[\Z\cdot d\Z -d\Z\cdot \Z].
    $$%
\end{exercise} %% The-complex-picture-of-the-Liouville-form

\begin{exercise}[The Fubini-Study $2$-form]
  \label{The-Fubini-Study-2-form}
  Let $\cS_\CC$ be the Hilbert sphere $\cS_\CC = \set{\Z \in \cH_\CC \mid \Z \cdot \Z = 1}$,
  equipped with the fine diffeology \art{The-fine-standard-Hilbert-space}.
  Let $\U(1)$ be the group of complex numbers with modulus 1,
  acting on $\cS_\CC$ by multiplication (\exref{U1-as-subgroup-of-diffeomorphisms}).
  Let $\cP_\CC = \cS_\CC/\U(1)$ be the Hilbert projective space (\exref{The-Hopf-S1-bundle}),
  and $\pi : \cS_\CC \to \cP_\CC$ be the projection.
  Let $\varpi$ be the $1$-form of $\cS_\CC$ defined by restriction of the Liouville $1$-form of $\cH_\CC$ (\exref{The-complex-picture-of-the-Liouville-form}).
  
  \Question{1)}~Show that the $1$-form $\varpi$ is
  invariant under the action of $\U(1)$, that is, for all
  $z \in \U(1)$, $z^*(\varpi) = \varpi$.
  
  \Question{2)}~Show that if $\P : \cO \to \cS_\CC$ and $\P' : \cO \to \cS_\CC$ are two plots such that $\pi \circ \P = \pi \circ \P'$.
  Then there exists a smooth parametrization $\zeta : \cO \to \U(1)$ such that $\P'(r) = \zeta(r) \times \P(r)$,
  for all $r \in \cO$.
  
  \Question{3)}~With $\P$ and $\P'$ of question 2,
  show that $\varpi(\P') = \varpi(\P) + \zeta^*(\theta)$,
  where $\theta$ is the $1$-form of $\U(1)$ defined by $[t \mapsto e^{it}]^*(\theta) = \dt$.
  
  \Question{4)}~Deduce that there exists a closed $2$-form $\omega$ on $\cP_\CC$ such that $\pi^*(\omega) = d\varpi$.
  
  \Note~The $2$-form $\omega$ is called the Fubini-Study form of the projective space $\cP_\CC$.
\end{exercise} %% The-Fubini-Study-2-form

\begin{exercise}[Irrational tori are orientable]
  \label{Irrational-tori-are-orientable}
  Let $\Gamma \subset \RR^n$ be a generating discrete subgroup of $(\RR^n,+)$.
  Let us recall that {\em diffeologically discrete} does not mean {\em topologically discrete},
  but that the plots of the subset diffeology of $\Gamma$ are locally constant \art{Subspaces-and-subset-diffeology}.
  Generating $\RR^n$ means here that the vector space $\RR^n$ is spanned by $\Gamma$.
  Let ${\T}_{\Gamma} = \RR^n/\Gamma$, $\T_\Gamma$ is a {\em torus}\index{Torus},
  {\em rational} or {\em irrational},
  depending if $\Gamma$ is embedded or not in $\RR^n$ \art{Embedded-subsets-of-a-diffeological-space}.
  Let us remark that this implies $\dim({\T}_{\Gamma}) = n$ \art{Dimension-of-tori}.
  Let $\pi_\Gamma$ be the canonical projection from $\RR^n$ onto its quotient ${\T}_{\Gamma}$.
  The projection $\pi_\Gamma$ is a generating family of ${\T}_{\Gamma}$.
  The canonical volume form $\vol_n =  \ee^1\wedge\cdots\wedge\ee^n \in \Lambda^n(\RR^n)$ is invariant by translation,
  {\em a fortiori\/} invariant by $\Gamma$.
  
  \Question{1)}~Show that there exists,
  on $\T_\Gamma$,
  a nowhere vanishing $n$-form $\vol_\Gamma \in \Omega^n({\T}_{\Gamma})$ such that $\pi_\Gamma^*(\vol_\Gamma) = \vol_n$.
  
  \Question{2)}~Conclude that,
  even if the group $\Gamma$ is not embedded,
  that is,
  not topologically discrete
  ---~which implies that the quotient ${\T}_{\Gamma}$ is not a manifold \art{The-irrational-torus-is-not-a-manifold}~---
  the torus ${\T}_{\Gamma}$ is orientable,
  and oriented by $\vol_\Gamma$.
  
  \Question{3)}~Show that,
  if $\Gamma$ is dense in $\RR^n$,
  then all the volumes of $\T_\Gamma$ are proportional,
  by a constant,
  to $\vol_\Gamma$.
\end{exercise} %% Irrational-tori-are-orientable

%%%%%%%%%%%%%%%%%%%%%%%%%%%%%%%%%%%%%%%%%%%%%%%%%%%%%%%%%%
%% MARK: Forms Bundles on Diffeological Spaces
%%%%%%%%%%%%%%%%%%%%%%%%%%%%%%%%%%%%%%%%%%%%%%%%%%%%%%%%%%

\section*{Forms Bundles on Diffeological Spaces}

\label{Section-Forms-bundles-on-diffeological-spaces}

\begin{sechead}
  In this section we {\em bundle} all the vector spaces $\Form^p_x(\X)$ of values of $p$-forms of $\X$ \art{The-values-of-a-differential-form},
  when $x$ runs over $\X$,
  into one geometrical object:
  the {\em $p$-form bundle} $\Forms^p(\X)$.
  This {\em bundle} is equipped with the pushforward diffeology of the product $\X\times \DForms^p(\X)$ by the map associating with the pair $(x,\alpha)$ the value $\alpha_x$.
  Then,
  every differential $p$-form of $\X$ becomes a smooth section of the projection $\pi: \Form^p_x(\X) \to \X$,
  which maps each value of a $p$-form to its basepoint,
  and vice versa.
  
  Then,
  we define on the product $\X\times \DForms^p(\X)$ a tautological $p$-form denoted by $\Taut$,
  satisfying $\Taut\restriction \X\times \{\alpha\} = \alpha$.
  We show that this tautological form passes to the quotient $\Forms^p(\X)$ into a form called Liouville form,
  and denoted by $\Liouv$.
  The form $\Liouv$ is characterized by the following property:
  for every $p$-form $\alpha \in \Omega^p(\X)$,
  $\balpha^*(\Liouv) = \alpha$,
  where $\balpha$ is the section $x \mapsto \alpha_x$ of the bundle $\Forms^p(\X)$.
  
  We suggest in the last paragraph a construction for the $p$-vector bundles of a diffeological space.
  We give,
  in particular,
  a definition for the tangent bundle.
  This question,
  of a good definition of a tangent bundle in diffeology,
  has been recurrently discussed,
  and different answers have been advanced by various authors.
  The reason for which there is no unique answer to the question of the tangent bundle in diffeology lies in the covariant nature of diffeology.
  There are indeed many ways to think about tangent spaces,
  which are equivalent for manifolds but not when applied to diffeological spaces.
  We discuss this question in \xart{The-p-vectors-bundles-of-diffeological-spaces}{Note 5} and we give a construction which tries to fit the general philosophy of the theory:
  since the nature of diffeology is to be covariant,
  contravariant objects need to be defined by duality with their covariant natural counterparts,
  and not the opposite way.
  But, actually,
  different kind of questions may need different versions of tangent bundles.
  
  I must add that what makes the notion of vector field interesting in classical differential geometry,
  and unavoidable,
  is the Cauchy-Lipschitz integration theorem.
  Many questions,
  in physics in particular,
  are expressed by a system of ordinary differential equations whose integral curves are the solutions,
  and whose existence is assured by this famous theorem.
  However,
  nothing like this exists in diffeology,
  a ``vector field'' can represent a system of partial differential equations which may not have a solution in finite time.
  This situation seriously weakens the interest in generalizing to diffeology a concept which is operative in ordinary differential geometry,
  but not anymore in this new framework.
  It is remarkable,
  however,
  that the difficulties of some problems involving vector fields in ordinary differential geometry could be circumvented and solved by avoiding the ``vector field'' pitfall.
  For example the integration of the closed $2$-forms \cite{Igl95} or the definition of the {\em moment map} in this new framework \cite{PIZ10},
  which lead in particular to the generalization of a classical theorem on the classification of homogeneous symplectic manifolds \cite{PIZ16} \cite{DIZ22},
  do not require Lie algebra,
  wich are classically defined as invariant vector fields.
  
\end{sechead}

\begin{article}\artlabel{The $p$-form bundle of diffeological spaces}
  \addcontentsline{toc}{section}{\small\hspace{10pt} The $p$-form bundle of diffeological spaces}
  \label{The-p-forms-bundle-of-diffeological-spaces}
  Let $\X$ be a diffeological space.
  Let us recall that the vector space $\Form^p_x(\X)$,
  of values of $p$-forms at the point $x \in \X$,
  is the quotient of the space of differential $p$-forms $\DForms^p(\X)$ by the subspace of all $p$-forms of $\X$ vanishing in $x$ \art{The-values-of-a-differential-form}.
  We define the {\em bundle of $p$-forms of $\X$}\index{Bundle} (or {\em over $\X$}) as the union of all these spaces $\Form^p_x(\X)$,
  and we denote it by
  $$%
    {\Forms}^p(\X) = \coprod_{x \in \X} \Form^p_x(\X) = \{ (x, a) \mid x \in \X \text{ and } a \in \Form^p_x(\X) \}.
    $$%
  This bundle will be equipped with the
  pushforward diffeology of $\X \times \DForms(\X)$ by the projection
  $$%
    \pp : \X \times \DForms^p(\X) \to \Forms^p(\X), \text{ such that } \pp(x,\alpha) = (x, \alpha_x).
    $$%
  We shall denote by $\pi :(x,a) \mapsto x$ the natural projection from the bundle $\Forms^p(\X)$ to $\X$.
  The construction is illustrated by the following diagram.
  
  \begin{center}
    \begin{tikzcd}[column sep=normal, row sep=large, every label/.append style = {font = \small}]
      \X \times \DForms^p(\X) \arrow[dr, swap, "\pr_1"] \arrow[rr,"\pp"] & {} & \Forms^p(\X) \arrow[dl, "\pi"]  \\
      {} & \X  & {}
    \end{tikzcd}
  \end{center}
  
  \Note~The space $\Forms^p(\X)$ is not the diffeological sum of the subsets $\Form^p_x(\X)$,
  $x \in \X$.
  Actually,
  $\Forms^p(\X)$ is the diffeological quotient of $\X \times \DForms^p(\X)$ by the equivalence relation $(x,\alpha) \sim (x',\alpha')$ if and only if $x = x'$ and $\alpha_x = \alpha'_{x'}$.
\end{article} %% The-p-forms-bundle-of-diffeological-spaces

\begin{article}\artlabel{Plots of the bundle $\Forms^p(\X)$}
  \addcontentsline{toc}{section}{\small\hspace{10pt} Plots of the bundle $\Forms^p(\X)$}
  \label{Plots-of-the-bundle-of-p-forms}
  A parametrization $\Pi: r \mapsto (\Q(r), {\P}(r))$ in $\Forms^p(\X)$,
  defined on some domain $\U$,
  is a plot if and only if the following two conditions are fulfilled:
  \begin{itemize}
    \item[1.] The parametrization $\Q$ is a plot of $\X$.
    \item[2.] For all $r_0 \in \U$ there exist an open neighborhood $\V$ of $r_0$ and a plot $\A : \V \to \DForms^p(\X)$ such that,
    for all $r \in \V$,
    ${\P}(r) = \A(r)_{{\Q}(r)}$,
    \ie\ ${\P}(r)$ is the value of $\A(r)$ at the point $\Q(r)$.
  \end{itemize}
  %%###########
  \begin{figure}[tb]
    \centerline{\includegraphics{Figures-PDF/fig-p-forms-bundle}}
    \caption{The $p$-form bundle.}
    \label{fig-space-of-p-forms}
  \end{figure}
  %%###########
  Note that the
  standard diffeology of $\Forms^p(\X)$,
  illustrated in Figure \ref{fig-space-of-p-forms},
  satisfies the following properties:
  \begin{itemize}
    \item[(a)] The projection $\pi: \Forms^k(\X) \to \X$ is a local subduction \art{Local-subductions}.
    \item[(b)] Each subspace $\pi^{-1}(x)$ is smoothly isomorphic to $\Form^p_x(\X)$.
  \end{itemize}
\end{article} %% Plots-of-the-bundle-of-p-forms

\begin{proof}
  We begin by the point 2:
  By definition of the pushforward diffeology \art{Pushforward-of-diffeologies},
  for every $r_0 \in \U$ there exist an open neighborhood $\V$ of $r_0$ and a plot $\A : \V \to \DForms^p(\X)$,
  such that $\pp \circ \A = \Pi \restriction \V$,
  that is,
  for any $r \in \V$,
  $\pp (\Q(r), \A(r)) = \Pi(r) = (\Q(r),\P(r))$.
  But $\pp (\Q(r), \A (r)) = (\Q(r), \A(r)_{\Q(r)})$,
  hence $\P(r) = \A(r)_{\Q(r)}$.
  
  1. Since $\pp(x,\alpha) = (x,\alpha_x)$ it follow from 2. that $\Q$ is a plot of $\X$.
  
  (a) Let $\Q: \U \to \X$ be a plot,
  $x = \Q(r)$ and $(x,a) \in \Forms^p(\X)$.
  By construction,
  there exists a $p$-form $\alpha \in \DForms^p(\X)$ such that $a = \alpha_x$.
  The parametrization $\Q \times \hat\alpha: r \mapsto (\Q(r),\alpha)$ is a plot of the product $\X\times \DForms^p(\X)$.
  Hence,
  $\pp \circ (\Q \times \hat\alpha)$ is a plot of $\Forms^p(\X)$,
  lifting $\Q$ and passing through $(x,a)$.
  Therefore,
  $\pi: \Forms^k(\X) \to \X$ is a local subduction.
  
  (b) Now,
  for every $x \in \X$ the injection $\Form^p_x(\X) \to \Forms^p(\X)$,
  defined by $a \mapsto (x,a)$, is an induction.
  Indeed,
  every plot of $\Forms^p(\X)$ with values in $\{x\} \times \Form^p_x(\X)$ is just a parametrization $a$ of $\Form^p_x(\X)$ such that there exists a plot $\A$ of $\DForms^p(\X)$ with $a(r) = \A(r)_x$.
  This is exactly the condition to be an element of the diffeology of $\Form^p_x(\X)$.
\end{proof}

\begin{article}\artlabel{The zero-forms bundle}
  \addcontentsline{toc}{section}{\small\hspace{10pt} The zero-forms bundle}
  \label{The-zero-forms-bundle}
  The bundle of $0$-forms $\Forms^0(\X)$ of a diffeological space $\X$ is just the product $\X\times \RR$.
  Indeed,
  the value of a $0$-form $f \in \DForms^0(\X) = \Cinfty(\X,\RR)$,
  at any point $x \in \X$,
  is just a number,
  and any number is the value of a $0$-form,
  at least the value of a constant function,
  $$%
    \Forms^0(\X) \simeq \X \times \RR.
    $$%
  The projection $(x,f) \mapsto (x,f(x))$,
  from $\X\times \Cinfty(\X,\RR)$ to $\X\times \RR$,
  is equivalent to the projection from $\X\times \Cinfty(\X,\RR)$ to $\Forms^0(\X)$.
\end{article} %% The-zero-forms-bundle

\begin{article}\artlabel{The cotangent space}
  \addcontentsline{toc}{section}{\small\hspace{10pt} The \guillemots{cotangent} space}
  \label{The-cotangent-space}
  The space $\Forms^1(\X)$ will be called the {\em cotangent space}\index{Cotangent space} of $\X$,
  even if it is not defined {\em a priori\/} by duality with some {\em tangent space};
  see the discussion in \art{The-p-vectors-bundles-of-diffeological-spaces}.
  Since in classical differential geometry,
  for a finite dimensional manifold $\M$,
  the cotangent space is denoted by ${\T}^*(\M)$,
  we should also denote sometimes $\Forms^1(\X)$ by $\T^*(\X)$.
\end{article} %% The-cotangent-space

\begin{article}\artlabel{The Liouville forms on the spaces of $p$-forms}
  \addcontentsline{toc}{section}{\small\hspace{10pt} The Liouville forms on spaces of $p$-forms}
  \label{The-Liouville-forms-on-the-spaces-of-p-forms}
  Let $\X$ be a diffeological space.
  For every plot $\Q \times \A : \U \to \X \times \DForms^p(\X)$,
  let us define
  %
  \begin{equation}
    \renewcommand{\theequation}{$\diamondsuit$}
    \Taut(\Q \times \A) = [r \mapsto \A(r)(\Q)(r)] \in \DForms^p(\U).
  \end{equation}
  %
  $\Taut$ is a differential $p$-form of $\X\times \DForms^p(\X)$ such that:
  %
  \begin{equation}
    \renewcommand{\theequation}{$\heartsuit$}
    \text{For all } \alpha \in \DForms^p(\X), \ \Taut \restriction \X \times \{\alpha\} = \alpha.
  \end{equation}
  %
  We shall call this $p$-form the {\em tautological $p$-form} of $\X \times \DForms^p(\X)$.
  Moreover,
  there exists a differential $p$-form of $\Forms^p(\X)$,
  denoted by $\Liouv$,
  such that
  $$%
    \Taut = \pp^*(\Liouv), \text{ with } \Liouv \in \DForms^p(\Forms^p(\X)),
    $$%
  where $\pp$ is the projection from $\X\times \DForms^p(\X)$ onto its quotient $\Forms^p(\X)$ \art{The-p-forms-bundle-of-diffeological-spaces}.
  We shall call this form the {\em Liouville $p$-form}\index{Liouville form} of $\Forms^p(\X)$.
\end{article} %% The-Liouville-forms-on-the-spaces-of-p-forms

\begin{proof}
  1. Let us prove that the condition $(\diamondsuit)$ above defines a differential form on $\X\times \DForms^p(\X)$ satisfying $(\heartsuit)$.
  Let $(\Q, \A)$ be a plot of $\X\times \DForms^p(\X)$,
  defined on an $n$-domain $\U$.
  Let $\F: \V \to \U$ be a smooth parametrization with $\dim(\V) = m$.
  Now,
  let $s \in \V$,
  $v_1,\ldots v_p \in \RR^m$,
  $r = \F(s)$ and $u_i = \D(\F)(s)(v_i)$,
  $i = 1\cdots p$.
  Then,
  %
  \begin{align*}
    \Taut((\Q, \A) \circ \F)(s)(v_1)\cdots(v_p) & = \Taut((\Q \circ \F) \times (\A \circ \F))(s)(v_1)\cdots(v_p)\\
    & = [(\A \circ \F)(s)](\Q \circ \F)(s)(v_1)\cdots(v_p) \\
    & = [\A(\F(s))](\Q \circ \F)(s)(v_1)\cdots(v_p) \\
    & = \A (r)(\Q \circ \F)(s)(v_1)\cdots(v_p) \\
    & = \F^*[\A(r)(\Q)](s)(v_1)\cdots(v_p) \\
    & = \A(r)(\Q)(r)(\D(\F)_s(v_1))\cdots(\D(\F)_s(v_p)) \\
    & = \A(r)(\Q)(r)(u_1)\cdots(u_p) \\
    & = \Taut(\Q,\A)(r)(u_1)\cdots(u_p) \\
    & = \F^*[\Taut(\Q,\A)](s)(v_1)\cdots(v_p).
  \end{align*}
  %
  Hence, $\Taut((\Q, \A) \circ \F) =
  \F^*(\Taut(\Q, \A))$, and $\Taut$ is a well
  defined $p$-form on $\X \times \DForms^p(\X)$.
  Now, let us
  consider the restriction of $\Taut$ to a subspace $\X\times
  \{\alpha\}$. A plot of $\X\times \{\alpha\}$ is just a
  parametrization $(\Q,\balpha) : r \mapsto (\Q(r),\alpha)$,
  where $\Q$ is a plot of $\X$. Then,
  by definition,
  $[\Taut \restriction \X \times \{\alpha\}](\Q,\balpha)(r) = \alpha(\Q)(r)$.
  Hence $\Taut \restriction \X\times \{\alpha\} = \alpha$.
  
  2. Now, let us prove that there exists a differential $p$-form $\Liouv$ on $\Forms^p(\X)$ such that $\pp^*(\Liouv) = \Taut$.
  We shall use the criterion \art{Pushing-forms-onto-quotients}.
  Let $(\Q,\A)$ and $(\Q',\A')$ be two plots of $\X \times \DForms^p(\X)$,
  defined on a domain $\U$,
  such that $\pp \circ (\Q,\A) = \pp \circ (\Q',\A')$,
  that is,
  $$%
    \Q = \Q', \text{ and for every $r \in \U$}, \ \A(r)_{\Q(r)} = \A'(r)_{\Q'(r)}.
    $$%
  Thus,
  the value of $\A(r)$ and $\A'(r)$ coincide at the point $\Q(r)$,
  for all $r \in \U$.
  That means that for any  plot ${\P}$ centered at $\Q(r)$,
  $\A(r)({\P})(0) = \A'(r)({\P})(0)$.
  Let $\T_r$
  \linebreak
  be the translation ${\T}_r(r') = r+r'$,
  and ${\P} = \Q \circ {\T}_r$.
  Then,
  $\P$ is a plot centered at $\Q(r)$,
  since $\Q \circ {\T}_r(0) = \Q(r)$.
  Hence,
  $\A(r)(\Q \circ {\T}_r)(0) = \A'(r)(\Q \circ {\T}_r)(0)$,
  that is,
  ${\T}_r^*(\A(r)(\Q))(0) = {\T}_r^*(\A'(r)(\Q))(0)$.
  But since ${\T}_r$ is a translation,
  ${\T}_r^*(\A(r)(\Q))(0) = \A'(r)(\Q)(r)$ and ${\T}_r^*(\A(r)(\Q))(0) = \A'(r)(\Q)(r)$.
  Thus,
  for every $r \in \U$,
  $\A(r)(\Q)(r) = \A'(r)(\Q)(r)$.
  But $\A(r)(\Q)(r) = \Taut(\Q,\A)(r)$ and $\A'(r)(\Q)(r) = \Taut((\Q,\A')(r)$,
  thus,
  $\Taut(\Q,\A) = \Taut(\Q,\A')$.
  Therefore,
  for any two plots $(\Q,\A)$ and $(\Q',\A')$ of $\X \times \DForms^p(\X)$ such that $\pp \circ (\Q,\A) = \pp \circ (\Q',\A')$,
  $\Taut(\Q,\A) = \Taut(\Q',\A')$.
  By application of the criterion \art{Pushing-forms-onto-quotients},
  there exists a $p$-form $\Liouv \in \DForms^p(\Forms^p(\X))$ such that $\pp^*(\Liouv) = \Taut$.
\end{proof}

\begin{article}\artlabel{Differential forms are sections of bundles}
  \addcontentsline{toc}{section}{\small\hspace{10pt} Differential forms are sections of  bundles}
  \label{Differential-forms-are-sections-of-bundles}
  Let $\X$ be a diffeological space.
  Let $\Forms^p(\X)$ be the bundle of $p$-forms over $\X$,
  and let $\pi: \Forms^p(\X) \to \X$ be the canonical projection \art{The-p-forms-bundle-of-diffeological-spaces}.
  A {\em smooth section} of $\pi$ is a smooth map $\sigma: \X \to \Forms^p(\X)$ such that $\pi \circ \sigma = \id_\X$.
  Let
  $$%
    \DSec(\pi) = \{\sigma \in \Cinfty(\X,\Forms^p(\X)) \mid \pi \circ \sigma = \id_\X \}.
    $$%
  Let $\chi$ be the map
  $$%
    \chi : \DSec(\pi) \to \DForms^p(\X), \text{ defined by } \chi(\sigma) = \sigma^*(\Liouv).
    $$%
  This map is smooth and surjective,
  and for every $\alpha \in \Omega^p(\X)$,
  $$%
    \alpha = [x \mapsto \alpha_x]^*(\Liouv), \text{ \ie, } \alpha = \chi([x \mapsto \alpha_x]).
    $$%
  Hence,
  every differential form can be regarded as the pullback of the Liouville form by its associated section $x \mapsto \alpha_x$.
  
  \Note~For the $0$-forms,
  $\Forms^0(\X) = \X\times \RR$ \art{The-zero-forms-bundle},
  a section $\ff: \X \to \Forms^0(\X)$ is just a map $\ff: x \mapsto (x,f(x))$,
  where $f: \X \to \RR$,
  and the section $\ff$ is smooth if and only if $f$ is smooth,
  which is satisfactory.
  Now,
  the Liouville form on $\X\times\DForms^0(\X)$ is a $0$-form,
  that is,
  a real function,
  and it is given by $\Taut(\Q \times \varphi)(r) = f_r(\Q(r))$,
  where $f_r = \varphi(r)$,
  and $\Q\times\varphi$ is any plot of $\X\times \DForms^0(\X)$.
  Hence,
  the Liouville form on $\Forms^0(\X)$ is just the second projection $\Liouv(x,t) = \pr_2(x,t) = t $,
  and we can check that $\ff^*(\Liouv)(x) = \Liouv(x,f(x)) = f(x)$.
\end{article} %% Differential-forms-are-sections-of-bundles

\begin{proof}
  Since the pullback is a smooth map \art{Pullbacks-of-smooth-forms},
  the map $\chi$ is smooth.
  Let us check then that the map $\alpha \mapsto \balpha = [x \mapsto \alpha_x]$,
  from $\DForms^p(\X)$ to $\DSec(\pi)$,
  where $\alpha_x$ is the value of $\alpha$ at $x$,
  is smooth.
  Note that we commit an abuse with the notation $[x \mapsto \alpha_x]$,
  since $\alpha_x$ does not belong to $\Forms^p(\X)$,
  actually $\balpha(x) = (x, \alpha_x)$.
  Next,
  $\balpha$ is smooth because $\balpha = [x \mapsto \pp(x, \alpha)]$ is the composite of two smooth maps $\balpha = \pp \circ \hat \alpha$,
  where $\hat \alpha =  [x \mapsto (x,\alpha)]$,
  and $\pp$ has been defined in \art{The-p-forms-bundle-of-diffeological-spaces}.
  Now,
  let $\alpha$ be a differential $p$-form,
  and let $\balpha: x \mapsto (x,\alpha_x)$ be the associated section.
  Thus,
  $\balpha^*(\Liouv) = (\pp \circ \hat \alpha)^*(\Liouv) = \hat\alpha^*(\pp^*\Liouv) = \hat\alpha^*(\Taut)$.
  But for any plot $\Q$ of $\X$,
  $\hat\alpha^*(\Taut)(\Q) = [r \mapsto \alpha(\Q)(r)]$,
  hence $\balpha^*(\Liouv) = \alpha$,
  and the map $\chi$ is surjective.
\end{proof}

\begin{article}\artlabel{Pointwise pullback or pushforward of forms}
  \addcontentsline{toc}{section}{\small\hspace{10pt} Pointwise pullback or pushforward of forms}
  \label{Pointwise-pullback-or-pushforward-of-forms}
  Let $\X$ and $\X'$ be two diffeological spaces.
  Let $f: \X \to \X'$ be a smooth map.
  
  1. Let $x \in \X$ and $x' = f(x)$.
  For every $a' \in \Form^p_{x'}(\X')$,
  the  {\em pointwise pullback}\index{Pointwise pullback} $f^*_x(a') \in \Form^p_{x}(\X)$ is defined by
  $$%
    f^*_x(a') = (f^*\alpha')_x\,,
    $$%
  for any $\alpha' \in \DForms^p(\X')$ such that $a' = \alpha'_{x'}$.
  %
  \begin{center}
    \begin{tikzcd}[column sep=large, row sep=large, every label/.append style = {font = \small}]
      \Form^p_x(\X) \arrow[d, swap, "\pi_\X"] & \Form^p_{x'}(\X') \arrow[l,"f^*_x"] \arrow[d, "\pi_{{\X}'}"]  \\
      \{x\} \arrow[r, swap, "f"] & \{x'\}
    \end{tikzcd}
  \end{center}
  %
  The composition of pointwise pullbacks is contravariant,
  $$%
    (g \circ f)^*_x = f^*_{x} \circ g^*_{x'},
    $$%
  where $\X$,
  $\X'$ and $\X''$ are three diffeological spaces,
  $f: \X \to \X'$ and $g: \X' \to \X''$ are smooth maps, and $x' = f(x)$,
  with $x \in \X$.
  
  2. Let $f$ be a diffeomorphism from $\X$ to $\X'$,
  and let $a \in \Form^p_x(\X)$.
  The {\em pointwise pushforward}\index{Pointwise pushforward} $f^x_*(a)$ is the $p$-form at the point $x' = f(x)$,
  defined by
  $$%
    f^x_*(a) = (f^{-1}_{x'})^*(a) \in \Form^p_{x'}(\X').
    $$%
  As a consequence of the chain rule of pointwise pullbacks,
  the pointwise pushforward is covariant,
  $(g \circ f)^x_* = g^{x'}_* \circ f^x_*$,
  where $f: \X \to \X'$ and $g: \X' \to \X''$ are diffeomorphisms.
  The pushforward of forms,
  by diffeomorphisms,
  defines an {\em action} of the group $\Diff(\X)$ on the bundle $\Forms^p(\X)$.
  The map $f_*$ is a diffeomorphism of $\Forms^p(\X)$ for each $f \in \Diff(\X)$.
\end{article} %% Pointwise-pullback-or-pushforward-of-forms

\begin{proof}
  What we have to check is only that if two forms $\alpha'$ and $\beta'$ have the same value at the point $x' = f(x)$,
  their pullbacks $\alpha = f^*(\alpha')$ and $\beta = f^*(\beta')$ have the same value in $x$.
  But this is a consequence of linearity of pullback of linear forms.
\end{proof}

\begin{article}\artlabel{Diffeomorphisms invariance of the Liouville form}
  \addcontentsline{toc}{section}{\small\hspace{10pt} Diffeomorphisms invariance of the Liouville form}
  \label{Diffeomorphisms-invariance-of-the-Liouville-form}
  Let $\X$ be a diffeological space.
  The group of diffeomorphisms of $\X$ acts naturally on the product $\X \times \DForms^p(\X)$ by:
  $$%
    \bvarphi(x,\alpha) = (\varphi(x), \ \varphi_*(\alpha)),
    $$%
  for all $\varphi \in \Diff(\X)$ and all $(x,\alpha) \in \X \times \DForms^p(\X)$,
  where $\varphi_*(\alpha) = (\varphi^{-1})^*(\alpha)$.
  The Liouville form $\Taut$ \art{The-Liouville-forms-on-the-spaces-of-p-forms} is invariant by this action,
  that is,
  $$%
    \bvarphi^*(\Taut) = \Taut, \text{ for all } \varphi \in \Diff(\X).
    $$%
  The action of $\Diff(\X)$ on $\X \times \DForms^p(\X)$ projects to a pointwise action \art{Pointwise-pullback-or-pushforward-of-forms} on the bundle of $p$-forms $\Forms^p(\X)$,
  $$%
    \bvarphi(x,a) = (\varphi(x), (\varphi^x_*(a)),
    $$%
  for all $\varphi \in \Diff(\X)$ and all $(x,a) \in \Forms^p(\X)$.
  Moreover,
  the Liouville form of the $p$-form bundle \art{The-Liouville-forms-on-the-spaces-of-p-forms} is invariant by this projected action,
  that is,
  $$%
    \bvarphi^*(\Liouv) = \Liouv, \text{ for all } \varphi \in \Diff(\X).
    $$%
\end{article} %% Diffeomorphisms-invariance-of-the-Liouville-form

\begin{proof}
  The pullback of the Liouville form $\Taut$ on $\X \times \DForms^p(\X)$ is given,
  for every plot $\Q \times \A$,
  by
  $$%
    \bvarphi^*(\Taut)(\Q \times \A) = \Taut(\bvarphi\circ(\Q \times \A)) = \Taut((\varphi \circ \Q) \times (\varphi_* \circ \A)).
    $$%
  But $\Taut((\varphi \circ \Q) \times (\varphi_* \circ \A))(r) = \varphi_*(\A(r))(\varphi \circ \Q(r)) = \A(r)(\varphi^{-1} \circ \varphi \circ \Q)(r) = \A(r)(\Q)(r) = \Taut(\Q \times \A)(r)$.
  Therefore,
  $\bvarphi^*(\Taut) = \Taut$.
  For equivariance reasons,
  by projection, $\bvarphi^*(\Liouv) = \Liouv$.
\end{proof}

\begin{article}\artlabel{The $p$-vector bundles of diffeological spaces}
  \addcontentsline{toc}{section}{\small\hspace{10pt} The $p$-vector bundles of diffeological spaces}
  \label{The-p-vectors-bundles-of-diffeological-spaces}
  Let $\X$ be a diffeological space.
  Let $\Plots_p(\X)$ be the set of $p$-plots of $\X$,
  for some integer $p$.
  Let $\P \in \Plots_p(\X)$ and $r \in \Dom(\P)$,
  for every $\alpha \in \Omega^p(\X)$, $\alpha(\P)(r) \in \Forms^p(\RR^p)$,
  thus $\alpha(\P)(r)$ is proportional to the standard volume $\vol_p$ of $\RR^p$ \art{Volumes-and-determinants}.
  Let us denote by $\dot \P(r)(\alpha)$ the coefficient of proportionality.
  The real map
  $$%
    \dot\P(r) : \Omega^p(\X) \to \RR, \text{ defined by } \dot\P(r) : \alpha \mapsto {\alpha(\P)(r) \over \vol_p}
    $$%
  belongs to the dual vector space $\Omega^p(\X)^*$,
  and moreover to the smooth dual.
  
  1. Let $\P : \U \to \X$ be a $p$-plot,
  and let $r_0 \in \U$.
  There exist an open neighborhood $\V \subset \U$ of $r_0$ and a plot $r \mapsto \Q_r$ of $\DPaths{p}(\X)$ \art{Iterating-paths},
  defined on $\V$,
  such that,
  for all $r \in \V$, $\P(r) = \Q_r(0)$ and $\dot \P(r) = \dot \Q_r(0)$.
  
  2. Thanks to the previous proposition,
  we redirect our attention to the set of global $p$-plots $\DPaths{p}(\X)$.
  For every $x \in \X$,
  we define $\Vgt{p}{x}{\X} \subset \Omega^p(\X)^*$ by
  $$%
    \Vgt{p}{x}{\X} = \{ \dot \Q(0) \in \Omega^p(\X)^* \mid \Q \in \DPaths{p}(\X) \text{ and } \Q(0) = x \}.
    $$%
  For all $x \in \X$,
  $0 \in \Vgt{p}{x}{\X}$.
  For all $x \in \X$,
  for all $v \in \Vgt{p}{x}{\X}$ and for all $s \in \RR$,
  $s \times v \in \Vgt{p}{x}{\X}$.
  Thus, $\Vgt{p}{x}{\X}$ is a star-shaped subset of $\Omega^p(\X)^*$ with origin $0$.
  
  3. However,
  if $\Vgt{p}{x}{\X}$ is not {\em a priori\/} a vector subspace of $\Omega^p(\X)^*$,
  then we shall use the vector space structure of the dual space $\Omega^p(\X)^*$ to extend,
  by linearity,
  the star $\Vgt{p}{x}{\X}$ into a vector subspace.
  Let $\Tgt{p}{x}{\X} = \Span(\Vgt{p}{x}{\X})$,
  $\Tgt{p}{x}{\X} \subset \Omega^p(\X)^*$,
  be the smallest vector subspace containing $\Vgt{p}{x}{\X}$.
  The elements of $\Tgt{p}{x}{\X}$ may be called the {\em tangent $p$-vectors} of $\X$ at the point $x$.
  The space $\Tgt{p}{x}{\X}$ is made of the finite sums of elements of $\Vgt{p}{x}{\X}$,
  $$%
    \Tgt{p}{x}{\X} = \bigg\{ v = \sum_{i \in \cI} v_i \biggm \vert \#\, \cI < \infty \text{ and for all $i \in \cI$, } v_i \in \Vgt{p}{x}{\X} \bigg\}.
    $$%
  
  The bundle of all the $p$-vector spaces $\Tgt{p}{x}{\X}$,
  when $x$ runs over $\X$,
  defines the {\em $p$-vector bundle}\index{pvector-bundle@$p$-vector bundle} of $\X$,
  denoted by $\Tgt{p}{}{\X}$,
  that is,
  $$%
    \Tgt{p}{}{\X} = \{ (x,v) \mid x \in \X \text{ and } v \in \Tgt{p}{x}{\X} \}.
    $$%
  
  4. The space $\Tgt{p}{}{\X}$ may be equipped with a {\em standard diffeology}.
  A parametrization  $\P : \U \to \Tgt{p}{}{\X}$,
  with $\P(r) = (x_r,v_r)$,
  is a plot for the standard diffeology if for all $r_0 \in \U$ there exist an open neighborhood $\V$ of $r_0$,
  a finite family of indices $\cI$,
  a family $\{[r \mapsto \Q_{i,r}]\}_{i \in \cI}$ of plots of $\DPaths{p}(\X)$ defined on $\V$,
  such that $\Q_{i,r}(0) = x_r$ for all $i \in \cI$,
  and $v_r = \sum_{i \in \cI} \dot\Q_{i,r}(0)$.
  
  5. The restriction of the standard diffeology to each $\Tgt{p}{x}{\X}$ is a diffeology of vector space \art{Diffeological-vector-spaces}.
  The zero section $x \mapsto (x,0)$ of $\Tgt{p}{}{\X}$ is smooth.
  The first projection $\pi : (x,v) \mapsto x$ is obviously a subduction.
  
  \Note{1}~By analogy with classical differential geometry, for $p = 1$,
  the space $\Tgt{1}{x}{\X}$ will simply be denoted by $\Tangent{x}{\X}$,
  and may be interpreted as the {\em tangent space} to $\X$ at the point $x$.
  The space $\Tangent{}{\X} = \Tgt{1}{}{\X}$ becomes then the {\em tangent bundle}\index{Tangent bundle} of $\X$.
  Moreover,
  for $\gamma \in \Plots_1(\X)$,
  the vector $\dot \gamma(t) \in  \Tangent{\gamma(t)}{\X}$ is the {\em speed} of $\gamma$ at {\em time} $t$.
  
  \Note{2}~By definition each vector space $\Tgt{p}{x}{\X}$ is a subspace of the smooth dual of the space of $p$-forms $\Forms^p_x(\X)$.
  Conversely,
  $\Forms^p_x(\X)$ is a subspace of the smooth dual of $\Tgt{p}{x}{\X}$,
  thanks to the pairing from $\Forms^p_x(\X) \times \Tgt{p}{x}{\X}$ to $\RR$, $(a,v) \mapsto \alpha(\Q)(0)$ where $v = \dot \Q(0)$ and $a = \alpha_x$ is the value of $\alpha$ at the point $x$ \art{The-values-of-a-differential-form}.
  
  \Note{3}~The map $\dot \Q(0)$ is not just local but depends only on the {\em $1$-jet} of $\Q$ at $0$.
  Precisely,
  let $\Q = \Q' \circ \F$,
  with $\F \in \Cinfty(\RR^p,\RR^p)$ and $\F(0) = 0$.
  If $\D(\F)(0) = \id_{\RR^p}$ then $\dot \Q = \dot \Q'$.
  Actually,
  since the plot $\Q$ is evaluated on the $p$-forms,
  $\dot \Q$ is ``antisymmetric'' in the sense that it depends only on the determinant of $\D(\F)(0)$, $\dot \Q(0) = \det(\D(\F)(0)) \times \dot \Q'(0)$.
  
  \Note{4}~Since $\Tgt{p}{}{\X}$ is a subset of the product $\X \times \DLin(\DForms^p(\X),\RR)$,
  where the smooth dual  $\DLin(\DForms^p(\X),\RR) = \Omega^p(\X)^* \cap \Cinfty(\Omega^p(\X))$ of $\Omega^p(\X)$ is equipped with the functional diffeology,
  $\Tgt{p}{}{\X}$ can also be equipped with the subset diffeology of this product.
  We shall call this subset diffeology  the {\em functional diffeology}\index{Functional diffeology} of $\Tgt{p}{}{\X}$.
  The standard diffeology of $\Tgt{p}{}{\X}$ is finer than its functional diffeology.
  
  \Note{5}~There are a few ways to look at vectors in differential geometry.
  We can see a vector $v$ at a some point $x$ as the class of a path $\gamma$.
  We can see this vector as a derivation $f \mapsto df_x(v)$ on the space of smooth functions.
  We can see also this vector $v$ as the value of a vector field $\F$,
  $v = \F(x)$.
  In classical differential geometry,
  for real domains or manifolds,
  all these approaches lead to the same objects.
  It is no more the case in diffeology.
  Essentially,
  a vector $v$ at $x$ is the class of a path $\gamma$ pointed at $x$,
  two paths $\gamma$ and $\gamma'$
  define the same tangent vector at $x$ if $\alpha(\gamma)_0 = \alpha(\gamma')_0$,
  for all $1$-forms $\alpha$.
  This is the more obvious way to define a vector at a point.
  On the other hand,
  considering a vector as a derivation consists of testing the paths $\gamma$ on exact $1$-forms only,
  that is,
  $\alpha = df$,
  and that shrinks {\em a priori\/} the tangent space.
  For instance,
  regarded this way,
  the tangent space of an $n$-dimensional irrational torus is reduced to $\{0\}$,
  while the construction above gives $\RR^n$,
  which coincides with what we expect;
  see \exref{Forms-bundles-of-irrational-tori}.
  In relation with vector fields in ordinary differential geometry,
  and thanks to the Cauchy-Lipschitz theorem,
  a vector field $\F$ generates a local flow $\varphi_t$,
  and $v = \F(x)$ is just the vector associated with the path
  $t \mapsto \varphi_t(x)$,
  which brings us back to the construction above.
  As we can see,
  this is the key concept,
  which drags everything else.
  But every new concept is justified by its achievements,
  until now there is no real deep result involving or needing tangent vectors or tangent spaces in diffeology.
  In general, using directly smooth paths,
  for instance here \art{Integrating-closed-1-forms} or there \art{Integration-bundles-of-closed-2-forms},
  or the local flows,
  for example in \art{The-Cartan-Lie-formula},
  to prove or construct what we need is sufficient.
\end{article} %% The-p-vectors-bundles-of-diffeological-spaces

\begin{proof}
  1.~First we claim that for any real $\varepsilon >0$ there exists a diffeomorphism $\lambda : \RR \to \openinterval{-\varepsilon,+\varepsilon}$ such that $\lambda \restriction \openinterval{-\varepsilon/2,+\varepsilon/2}$ is the identity;
  see Figure \ref{fig-diff-R-interval}.
  Now,
  let $\P : \U \to \X$ be a $p$-plot and $r_0 \in \Dom(\P)$,
  there exists $\varepsilon >0$ such that the open ball $\cB(r_0, 2\varepsilon) \subset \U$.
  For all $r \in \cB(r_0, \varepsilon)$ let $\phi_r :  \RR^p \to \RR^p$ be defined by
  $$%
    \phi_r(t) = r + {\lambda(\norm{t}) \over \norm{t}} \times t.
    $$%
  The map $\phi_r$ is well defined since on an open neighborhood of $0 \in \RR^p$,
  $\lambda(\norm{t}) = \norm{t}$ and $\phi_r(t) = r + t$.
  Now,
  \begin{align*}
    \norm{\phi_r(t) - r_0} & =  \norm{r-r_0 + {\lambda(\norm{t}) \over \norm{t}} \times t} \\
    & < \norm{r-r_0} + \modulus{\lambda(\norm{t})} \\
    &<  \varepsilon + \varepsilon.
  \end{align*}
  Thus,
  $\phi_r(t) \in \cB(r_0,2\varepsilon)$ for all $t \in \RR^p$.
  Then,
  we can define
  $$%
    \Q_r = \P \circ \phi_r,\ \text{that is,}\ \Q_r(t) = \P\bigg(
    r + {\lambda(\norm{t}) \over \norm{t}} \times t\bigg).
    $$%
  For all $r \in \cB(r_0,\varepsilon)$,
  $\Q_r$ is a global $p$-plot of $\X$,
  and since the map $r \mapsto \phi_r$ is clearly smooth for the functional diffeology,
  $r \mapsto \Q_r$ is a plot of $\DPaths{p}(\X)$.
  We have obviously $\Q_r(0) = \P(r)$. Now,
  for all $\alpha \in \Omega^p(\X)$,
  for computing $\alpha(\Q_r)(0)$,
  just restrict $\Q_r$ to an open neighborhood of $0 \in \RR^p$.
  Let us choose the ball $\cB(0, \varepsilon/2)$,
  on which $\Q_r(t) = \P(r+t)$.
  We have $\alpha(\Q_r)(0) = \alpha(\Q_r \restriction \cB(0,\varepsilon/2))(0) = \alpha(t \mapsto \P(r+t))(0) = \alpha(r \mapsto r+t \mapsto \P(r+t))(0) = [t\mapsto r+t]^*(\alpha(\P))(0) = \alpha(\P)(r)$.
  Thus,
  $\dot\Q_r(0) = \dot\P(r)$.
  
  %%###########
  \begin{figure}[tb]
    \centerline{\includegraphics{Figures-PDF/fig-diff-R-interval}}
    \caption{Contraction of $\RR$ to an interval.}
    \label{fig-diff-R-interval}
  \end{figure}
  %%###########
  
  2. Every differential form evaluated on a constant plot vanishes (see \exref{Differential-forms-against-constant-plots}).
  Thus,
  for all $x \in \X$, $\alpha({\bmx}) = 0$ for all $\alpha \in \Omega^p(\X)$, with ${\bmx} = [r \mapsto x]$, $r \in \RR^p$,
  that is,
  $\dot{\bmx}(0) = 0 \in \Omega^p(\X)^*$,
  hence $0 \in \Vgt{p}{x}{\X}$.
  Let us show now that if  $v = \dot\Q(0)$,
  then $s \times v = \dot\Q_s(0)$ with $\Q_s(t) =  \Q(st)$.
  For all $\alpha \in \Omega^p(\X)$ we have $\dot\Q_s(0)(\alpha) \times \vol_p = \alpha(\Q_s)(0) = \alpha(t \mapsto \Q(st))(0) = \alpha(t \mapsto st \mapsto \Q(st))(0) =$
  \linebreak
  $\{[s \mapsto st]^*(\alpha(\Q))\}(0) = s \times \alpha(\Q)(0) = s \times \dot\Q(0)(\alpha) \times \vol_p$. So, $s \times v = \dot\Q_s(0) \in  \Vgt{p}{x}{\X}$.
  
  3. Let us check now that $\Span(\Vgt{p}{x}{\X})$ is made of the finite sums of elements of $\Vgt{p}{x}{\X}$.
  First of all,
  the set $\Tgt{p}{x}{\X}$ of these finite sums is a vector subspace of $\Omega^p(\X)^*$.
  It is obviously closed by addition and by scalar multiplication because $\Vgt{p}{x}{\X}$ is already closed by scalar multiplication.
  Now,
  by definition,
  $\Span(\Vgt{p}{x}{\X})$ is the intersection of all the vector subspaces of $\Omega^p(\X)^*$ containing $\Vgt{p}{x}{\X}$.
  But every vector subspace of $\Omega^p(\X)^*$ containing $\Vgt{p}{x}{\X}$ will contain every finite sum of elements of $\Vgt{p}{x}{\X}$,
  so will contain $\Tgt{p}{x}{\X}$,
  thus $\Tgt{p}{x}{\X} \subset \Span(\Vgt{p}{x}{\X})$.
  Conversely,
  every vector of $\Span(\Vgt{p}{x}{\X})$ is by definition contained in every vector subspace of $\Omega^p(\X)^*$ containing $\Vgt{p}{x}{\X}$,
  but $\Tgt{p}{x}{\X}$ is one of them,
  thus $\Span(\Vgt{p}{x}{\X}) \subset \Tgt{p}{x}{\X}$.
  Therefore,
  $\Span(\Vgt{p}{x}{\X}) = \Tgt{p}{x}{\X}$.
  
  4. The proof that the property described in the fourth point is a diffeology is left as an exercise.
  Let us just remark that this diffeology is the image of the sum $\coprod_{k \in \NN} \Paths_p^k(\X,\bullet)$ by the map $(k,\Q_1,\ldots,\Q_k) \mapsto (\Q_1(0), \sum_{i=1}^k \dot\Q_i(0))$,
  where $\Paths_p^k(\X,\bullet)$ is the subspace of $[\Paths_p(\X)]^k$ of elements $(\Q_1,\ldots,\Q_k)$ such that $\Q_1(0) = \cdots = \Q_k(0)$.
  The fact that the restriction of this diffeology to the subspaces $\Tgt{p}{x}{\X}$ is a diffeology of vector space is a simple verification,
  and the fact that the zero section is smooth follows that $0 = \dot {\bmx}(0)$ (see point 2 above).
\end{proof}

%%%%%%%%%%%%%%%%%%%%%%%%%%%%%%%%%%%%%%%%%%%%%%%%%%%%%%%%%%
%
%   Exercises
%
%%%%%%%%%%%%%%%%%%%%%%%%%%%%%%%%%%%%%%%%%%%%%%%%%%%%%%%%%%

\Exercises

\begin{exercise}[The $k$-forms bundle on a real domain]
  \label{The-k-forms-bundle-on-a-real-domain}
  Let $\U$ be an $n$-domain.
  For $\alpha \in \Omega^k(\U)$,
  let $a = \alpha(\id_\U) \in \Cinfty(\U,\Lambda^k(\RR^n))$.
  
  \Question{1)}~Let $\P : \V \to \Omega^k(\U)$ be a parametrization,
  $\alpha_r = \P(r)$ and $a_r = \alpha_r(\id_\U)$.
  Show that $\P$ is a plot for the functional diffeology \art{Functional-diffeology-of-the-space-of-forms} if and only if $(r,x) \mapsto a_r(x)$,
  defined on $\V \times \U$ with values in $\Lambda^k(\RR^n)$,
  is smooth.
  
  \Question{2)}~ Let $\alpha, \beta \in \Omega^k(\U)$ and $x \in \U$.
  Show that $\alpha_x = \beta_x$ if and only if $a(x) = b(x)$,
  where $b = \beta(\id_\U)$.
  Deduce that $\Lambda^k_x(\U) \simeq \Lambda^k(\RR^n)$.
  
  \Question{3)}~Show that the map $\phi : (x,\alpha_x) \mapsto (x,a(x) = \alpha(\id_\U)(x))$,
  from $\Forms^k(\U)$ to $\U \times \Lambda^k(\RR^n)$,
  is a diffeomorphism.
\end{exercise} %% The-k-forms-bundle-on-a-real-domain

\begin{exercise}[The $p$-form bundle on a manifold]
  \label{The-p-forms-bundle-on-a-manifold}
  Let $\M$ be a finite dimensional manifold, $\dim(\M) = n$ \art{Manifolds-as-diffeologies}.
  Let $m \in \M$, $\F : \U \to \M$ and $\F \in \cA$ such that $\F(x) = m$,
  let $a \in \Form^p(\RR^n)$.
  
  \Question{1)}~Show that,
  for $\varepsilon > 0$ small enough,
  there exists a $p$-form $\bar a$ defined on $\U$ such that $\bar a$ is zero outside a ball $\B(x,\varepsilon)$,
  centered at $x$ with radius $\varepsilon$,
  and equal to $a$ at the point $x$, $\bar a(x) = a$.
  
  \Question{2)}~Show that there exists a $p$-form $\alpha$,
  defined on $\M$, such that $\F^*(\alpha) = \bar a$.
  
  \Question{3)}~Deduce that $\Forms^p_m(\M) \simeq \Form^p(\RR^n)$.
  
  \Question{4)}~For every chart $\F : \U \to \M$,
  define
  $$%
    \cF : \U \times \Form^p(\RR^n) \to \Forms^p(\M), \text{ by } \cF(r,a) = (\F(r),\F_*(a)),
    $$%
  where $\F_*(a) \in \Forms_{\F(r)}^p(\M)$ is the pushforward of the form $a \in \Form^p(\RR^n)$ by $\F$,
  according to \art{Pointwise-pullback-or-pushforward-of-forms}.
  Check that $\cF$ is a chart of $\Forms^p(\M)$.
  
  \begin{center}
    \begin{tikzcd}[column sep=large, row sep=large, every label/.append style = {font = \small}]
      \U \times \Form^p(\RR^n) \arrow[d, swap, "\pr_1"] \arrow[r,"\cF"] & \Forms^p(\M) \arrow[d, "\pi"]  \\
      \U \arrow[r, swap, "\F"] & \M
    \end{tikzcd}
  \end{center}
  
  Conclude that the bundle of $p$-forms $\Forms^p(\M)$ is a manifold of dimension $n+ \Cnp$,
  modeled on $\RR^n \times \Form^p(\RR^n)$,
  and that the set of $\cF$,
  when $\F$ runs over an atlas of $\M$,
  is an atlas of $\Forms^p(\M)$.
\end{exercise} %% The-p-forms-bundle-on-a-manifold

\begin{exercise}[Smooth forms on diffeological vector spaces]
  \label{Smooth-forms-on-diffeological-vector-spaces}
  Let $\E$ be a diffeological vector space and $\cO \subset \E$ be a D-open subset.
  
  \Question{1)}~Let $\alpha$ be a differential $1$-form on $\cO$ and $\A(x)(u) = \alpha(t \mapsto x + tu)_0(1)$,
  for all $x \in \cO$ and all $u \in \E$.
  Show that $\A \in \Cinfty(\cO,\E^*_\infty)$.
  
  \Question{2)}~Conversely,
  let $\A : \cO \to \E^*_\infty$ be a smooth map.
  Let $\P : \U \to \cO$ be a plot.
  For all $r \in \U$,
  let $\alpha(\P)_r = d[\A(x) \circ \P]_r$,
  where $x = \P(r)$.
  Show that $\alpha$ is a differential $1$-form on $\cO$.
  
  \Question{3)}~Let $\pi : \DForms^1(\cO) \to \Cinfty(\cO,\E^*_\infty)$ and $\sigma : \Cinfty(\cO,\E^*_\infty) \to \DForms^1(\cO)$ be the maps $\pi : \alpha \mapsto \A$ and $\sigma : \A \mapsto \alpha$.
  Show that $\pi \circ \sigma = \id$.
  
  \Question{4)}~Assume that every arc in $\E$ at the origin is tangent to some ray,
  that is, for every pointed $1$-plot $c$, $c(0) =0_\E$, there exists $u \in \E$ such that for every differential $1$-form $\alpha$,
  $\alpha(c)_0(1) = \alpha (t \mapsto tu)_0(1)$.
  Observe that $\A(x)$ represents the value of $\alpha$ at $x$ \art{The-values-of-a-differential-form} and show that $\pi$ is an isomorphism.
  
  \Note{1}~For the last question,
  we cannot exclude {\em a priori\/} the case for which an arc could be not tangent to some vector,
  which does not happen in finite dimensions, for example.
  
  \Note{2}~This construction is related to the characterization of the cotangent space $\T^*(\M)$ \art{The-cotangent-space} of a manifold $\M$ modeled on $\E$ \art{Diffeological-Manifolds}.
  It is possible to show that,
  if
  for every $m \in \M$,
  every element $a \in \E^*_\infty$ represents,
  in some chart $\F$ of $\M$, the value of a differential $1$-form $\alpha$ at $m$,
  that is,
  if $a$ is the value of $\F^*(\alpha)$ at $x$,
  with $m = \F(x)$,
  then the cotangent space is a (locally trivial) fiber bundle \art{Diffeological-fibrations} with fiber $\E^*_\infty$.
  The same kinds of consideration can help to deconstruct the bundles of differential forms in any degree over diffeological manifolds.
\end{exercise} %% Smooth-forms-on-diffeological-vector-spaces

\begin{exercise}[Forms bundles of irrational tori]
  \label{Forms-bundles-of-irrational-tori}
  Let ${\T}_{\Gamma}$ be the irrational torus $\RR^n/\Gamma$ where $\Gamma$ is a dense generating subgroup of $\RR^n$.
  Let $\pi : \RR^n \to \T_\Gamma$ be the canonical projection.
  
  \Question{1)}~Let $\alpha$ be a $p$-form of $\T_\Gamma$.
  Let $a = \pi^*(\alpha)$.
  Show that $a$ is invariant under the group $\Gamma$,
  that is,
  for all $\gamma \in \Gamma$ $\gamma^*(a) = a$.
  Deduce that every component of the form $a$ is invariant under $\Gamma$.
  Deduce that $a$ is constant and $\DForms^p(\T_\Gamma) \simeq \Form^p(\RR^n)$.
  
  \Question{2)}~Deduce that $\Forms^p(\T_\Gamma) \simeq \T_\Gamma \times \Form^p(\RR^n)$.
\end{exercise} %% Forms-bundles-of-irrational-tori

\begin{exercise}[Vector bundles of irrational tori]
  \label{Vectors-bundles-of-irrational-tori}
  Let ${\T}_{\Gamma}$ be the irrational torus $\RR^n/\Gamma$ where $\Gamma$ is a dense generating subgroup of $\RR^n$;
  see \exref{Forms-bundles-of-irrational-tori}.
  Show that,
  for the $p$-vector bundles \art{The-p-vectors-bundles-of-diffeological-spaces} of $\T_\Gamma$,
  $$%
    \Tgt{p}{}{{\T}_{\Gamma}} \simeq {\T}_{\Gamma} \times [\Form^p(\RR^n)]^* \simeq {\T}_{\Gamma} \times \RR^{\raisebox{2pt}{$\scriptstyle{ n! \over p!(n-p)!}$}}.
    $$%
  In particular,
  for the tangent space of the irrational torus,
  $$%
    \Tangent{x}{{\T}_{\Gamma}} \simeq \RR^n, \text{ and } \T({\T}_{\Gamma}) \simeq {\T}_{\Gamma}\times\RR^n.
    $$%
\end{exercise} %% Vectors-bundles-of-irrational-tori

\begin{exercise}[Differential $1$-forms on $\RR/\{\pm 1\}$]
  \label{Differential-1-forms-on-pm-1}
  Let $\Delta = \RR/\{\pm 1\}$ be the quotient of the real line by the action of the multiplicative group $\{\pm 1\}$.
  Identify $\Delta$ with the half-line $[0,\infty[$,
  equipped with the pushforward of the smooth diffeology of $\RR$ by the square map $\sq(x) = x^2$.
  
  \Question{1)}~Show that every differential $1$-form on $\Delta$ writes $\alpha = f \times \theta$,
  where $\theta$ is the $1$-form defined by $\sq^*(\theta) = d[t^2] = 2t\times dt$, and $f \in \Cinfty(\Delta,\RR)$.
  
  \Question{2)}~Deduce that every differential $1$-form on $\Delta$ vanishes at the origin,
  as well as its tangent space at this point.
\end{exercise} %% Differential-1-forms-on-pm-1

%%%%%%%%%%%%%%%%%%%%%%%%%%%%%%%%%%%%%%%%%%%%%%%%%%%%%%%%%%
%% MARK: Lie Derivative of Differential Forms
%%%%%%%%%%%%%%%%%%%%%%%%%%%%%%%%%%%%%%%%%%%%%%%%%%%%%%%%%%

\section*{Lie Derivative of Differential Forms}
\label{Section-Lie-derivative- of-differential-forms}

\begin{sechead}
  In this section,
  we introduce the notion of {\em Lie derivative} of differential forms,
  on a diffeological space,
  along {\em slidings},
  that is,
  germs of paths of diffeomorphisms.
  We introduce the notion of {\em contraction of forms} by {\em arcs of plots} which generalizes the notion of contraction of smooth forms by a vectors on real domains.
  We generalize then the famous Cartan-Lie formula for differential forms in the context of diffeology.
\end{sechead}

\begin{article}\artlabel{The Lie derivative of differential forms}
  \addcontentsline{toc}{section}{\small\hspace{10pt} The Lie derivative of differential forms}
  \label{The-Lie-derivative-of-differential-forms}
  Let $\X$ be a diffeological space,
  let $\alpha \in \Omega^k(\X)$,
  and let $h: \RR \to \Diff(\X)$ be a $1$-plot for the functional diffeology \art{Functional-diffeology-on-groups-of-diffeomorphisms} centered at the identity $h(0) = \id_\X$.
  The {\em Lie derivative}\index{Lie derivative} $\DLie_h(\alpha)$,
  of $\alpha$ by $h$,
  is the $k$-form defined by
  $$%
    \DLie_h(\alpha) = {\partial \over \partial t}
    \bigg\{ h(t)^* \alpha \bigg\}_{t = 0}.
    $$%
  Explicitly,
  that means that for all $n$-plot $\P : \U \to \X$,
  for all $r \in \U$ and all set of $k$ vectores $v_1,\ldots,v_k$ in $\RR^n$,
  $$%
    \DLie_h(\alpha)({\P})(r)(v_1)\cdots(v_k) = {\partial \over \partial t}\bigg\{\alpha(h(t) \circ {\P})(r)(v_1)\cdots(v_k)\bigg\}_{t = 0}.
    $$%
  \Note{1} The Lie derivative $\DLie_h(\alpha)$ depends only on the 1-jet of the plot $h$ at $0$.
  Precisely,
  if $h$ and $h'$ are two $1$-plots of $\Diff(\X)$ centered at the identity such that $h' = h \circ f$, with $f(0) = 0$ and $\D(f)(0) = 1$,
  then $\DLie_h(\alpha) = \DLie_{h'}(\alpha)$.
  This is the case in particular when $h$ and $h'$ coincide on an open neighborhood of $0$ (they have the same germ at $0$).
  
  \Note{2} The Lie derivative $\DLie_h: \Omega^k(\X) \to \Omega^k(\X)$ is a smooth linear map for the functional diffeology of forms art{Functional-diffeology-of-the-space-of-forms}.
\end{article} %% The-Lie-derivative-of-differential-forms

\begin{proof}
  First of all,
  let us show that $\DLie_h(\alpha)$ is a well defined differential form of $\X$.
  Let ${\P}: \U \to \X$ be an $n$-plot of $\X$, let $h\cdot {\P}$ be the $(p+1)$-plot defined on $\RR\times \U$ by
  $$%
    h\cdot {\P}(t,r) =  (h(t) \circ {\P})(r) = h(t)({\P}(r)), \ \text{where} \ (t,r) \in \RR\times \U.
    $$%
  By definition,
  $\alpha(h\cdot {\P})$ is a differential form on $\RR\times \U$.
  But $\alpha(h(t) \circ {\P})$ is the pullback of $\alpha(h\cdot {\P})$ by the injection $j_t: r \mapsto (t,r)$ from $\U$ into $\RR \times \U$,
  \ie,
  $\alpha(h(t) \circ {\P})$ is the restriction of $\alpha(h\cdot {\P})$ to $\{t\,\} \times \U$.
  Since $\alpha(h(t) \circ {\P})$ is a smooth form,
  the map $t \mapsto \alpha(h(t) \circ {\P})\restriction \{t\,\} \times \U$ is a smooth parametrization in $\Forms^k(\RR^n)$.
  Thus, its derivative is still a smooth parametrization of $\Forms^k(\RR^n)$.
  Therefore,
  the definition $\DLie_h(\alpha)$ makes sense.
  Let us check now the fundamental property of differential forms \art{Differential-forms-on-diffeological-spaces}.
  Let $\F: \V \to \U$ be a smooth parametrization in $\U$:
  %
  $$%
    \def\arraystretch{2.5}
    \begin{array}{lllll}
    \DLie_h(\alpha)({\P} \circ \F) & = & \displaystyle{\partial \over \partial t}\bigg\{\alpha(h(t) \circ ({\P} \circ \F))\bigg\}_{t = 0}  & = & \displaystyle{\partial \over\partial t}\bigg\{\alpha((h(t) \circ {\P}) \circ \F)\bigg\}_{t = 0} \\
    & = & \displaystyle{\partial \over \partial t}\bigg\{\F^*(\alpha(h(t) \circ {\P}))\bigg\}_{t = 0}  & = & \F^*\bigg(\displaystyle{\partial \over\partial t}\bigg\{\alpha(h(t) \circ {\P})\bigg\}_{t = 0}\bigg) \\
    & = & \F^*\bigg(\DLie_h(\alpha)({\P})\bigg). & &
    \end{array}
    $$%
  Then $\DLie_h(\alpha)$ is a differential $k$-form of $\X$.
  For the first note,
  since the derivative is local,
  the Lie derivative is local and $\DLie_h(\alpha)$ does not depend on more than the germ of $h$.
  Now, if $h' = h \circ f$ with $f(0) = 0$, then
  \begin{eqnarray*}
    \DLie_{h'}(\alpha)(\P)(r)[v] & = &\D(s \mapsto \alpha(h(f(s)) \circ \P)(r)[v])(s=0)(1) \\
    & = & \D(s \mapsto t =f(s) \mapsto \alpha(h(t) \circ \P)(r)[v])(s=0)(1) \\
    & = & \D(t \mapsto \alpha(h(t) \circ \P)(r)[v])(0)(\D(f)(0)(1)) \\
    & = & \D(f)(0)(1) \times \DLie_h(\alpha)(\P)(r)[v],
  \end{eqnarray*}
  where we denoted a $k$-uple of vectors $(v_1)\cdots(v_k)$ by $[v]$.
  Thus,
  if  $\D(f)(0)(1) = 1$,
  then $\DLie_h(\alpha) = \DLie_{h'}(\alpha)$.
  For the second note,
  it is clear that the Lie derivative is linear.
  Let us prove that $\DLie_h$ is smooth.
  Let $\phi: \V \to \Omega^k(\X)$ and ${\P}: \U \to \X$ be two plots.
  We want to check that $(s,r) \mapsto \DLie_h(\phi(s))({\P})(r)$ is smooth.
  But
  \begin{eqnarray*}
    \DLie_h(\phi(s))({\P})(r) & = & {\partial \over \partial t} \Llpar h(t)^* (\phi(s))({\P})(r)\Lrpar_{t = 0} \\
    & = & {\partial \over \partial t} \Llpar \phi(s)(h(t) \circ {\P})(r)\Lrpar_{t = 0} \\
    & = & {\partial \over \partial t} \Llpar j_t^*(\phi(s)(h\cdot {\P}))(r)\Lrpar_{t = 0}\,.
  \end{eqnarray*}
  Now, $(s,t,r) \mapsto \phi(s)(h\cdot {\P})(t,r)$ is a smooth map,
  and $j_t^*(\phi(s)(h\cdot {\P}))(r)$ is just its restriction to $\V \times\{t\} \times \U$.
  Thus,
  $\DLie_h(\phi(s))({\P})(r)$ is a partial derivative of a smooth map,
  hence it is smooth.
  Therefore,
  the Lie derivative $\DLie_h$ is a smooth endomorphism of $\Omega^k(\X)$.
\end{proof}

\begin{article}\artlabel{Lie derivative along homomorphisms}
  \addcontentsline{toc}{section}{\small\hspace{10pt} Lie derivative along
  homomorphisms} \label{Lie-derivative-along-homomorphisms}
  Let $\X$ be a diffeological space.
  We call a {\em ray\/}\index{Ray} of $\Diff(\X)$ every smooth homomorphism from $\RR$ to $\Diff(\X)$,
  and we call a {\em flow\/} every $1$-plot $h$ of $\Diff(\X)$,
  defined on an interval $\I$ and centered at the identity,
  such that for all $t,t', t + t' \in \I$,
  $h(t+t') = h(t) \circ h(t')$.
  
  1. Let $h$ be a ray of $\Diff(\X)$.
  Then
  $$%
    \left.{\partial h(t)^*(\alpha) \over \partial t}\right\vert_{t = t_0} = h(t_0)^*(\DLie_h(\alpha)),
    \text{ for all } t_0 \in \RR.
    $$%
  
  2. Let $\alpha$ be a $k$-form on $\X$ and $h$ be a flow defined on $\I$.
  If $\DLie_h(\alpha) = 0$,
  then $h(t)^*(\alpha) = \alpha$,
  for all $t \in \I$.
  
  3. Let $\SkDiff{\X}\subset \Diff(\X)$ be the subgroup generated by the elements $h(t) \in \Diff(\X)$ where $h$ is a flow.
  The subgroup $\SkDiff{\X}$ is made of finite composites $h_1(t_1) \circ \cdots \circ h_{\N}(t_{\N})$,
  where ${\N}$ runs over $\NN$,
  and the $h_i$ are flows.%
  \footnote{The group generated by the flows is clearly a subgroup of the identity component of the group of diffeomorphisms.
  For a compact second countable manifold they coincide,
  but I do not think that they coincide in general,
  for diffeological spaces.}
  %
  The subgroup $\SkDiff{\X}$ is normal in $\Diff(\X)$,
  that is,
  for all $g \in \Diff(\X)$,
  and all $h \in \SkDiff{\X}$,
  $g \circ h \circ g^{-1} \in \SkDiff{\X}$.
  Now,
  let $\H \subset \SkDiff{\X}$ be a subgroup and let $\alpha$ be a $k$-form of $\X$.
  If $\DLie_\F(\alpha) = 0$ for every $1$-plot $\F$ of ${\H}$,
  centered at the identity,
  then,
  for every $h \in {\H}$,
  $h^*(\alpha) = \alpha$.
\end{article} %% Lie-derivative-along-homomorphisms

\begin{proof}
  1.~Let us compute the derivative of $[t \mapsto h(t)^*(\alpha)]$ at the point $t_0$:
  \begin{eqnarray*}
    \left.{\partial h(t)^*(\alpha) \over \partial t}\right\vert_{t = t_0}& = & \lim_{\epsilon \mathop{\rightarrow} 0} {h(t_0+\epsilon)^*(\alpha) -h(t_0)^*(\alpha) \over \epsilon} \\
    & = & \lim_{\epsilon \mathop{\rightarrow} 0} h(t_0)^*\left[{h(\epsilon)^*(\alpha) -\alpha \over \epsilon}\right].
  \end{eqnarray*}
  Let $\beta_\epsilon = (h(\epsilon)^*(\alpha) -\alpha)/ \epsilon$.
  Then for every $n$-plot ${\P}: \U \to \X$,
  for every point $r \in \U$, and for any $k$ vectors $u_1,\ldots,u_k \in \RR^n$,
  \begin{eqnarray*}
    \left.{\partial h(t)^*(\alpha) \over \partial t}\right\vert_{t = t_0}({\P})(r)(u_1)\cdots(u_k) & = & \lim_{\epsilon \mathop{\rightarrow} 0} h(t_0)^*(\beta_\epsilon)({\P})(r)(u_1)\cdots(u_k) \\
    & = & \lim_{\epsilon \mathop{\rightarrow} 0}\beta_\epsilon(h(t_0) \circ {\P})(r)(u_1)\cdots(u_k) \\
    & = & \DLie_h(\alpha)(h(t_0) \circ {\P})(r)(u_1)\cdots(u_k) \\
    & = & h(t_0)^*(\DLie_h(\alpha))({\P})(r)(u_1)\cdots(u_k).
  \end{eqnarray*}
  Hence,
  $$%
    \left.{\partial h(t)^*(\alpha) \over \partial t}\right\vert_{t = t_0}  \hspace{-1em}{=} \ \ h(t_0)^*(\DLie_h(\alpha)).
    $$%
  
  2. Now,
  if $\DLie_h(\alpha) = 0$,
  then $[\partial(h(t_0)^*(\alpha))/\partial t]_{t = t_0} = 0$ for all $t_0 \in {\I}$,
  and $h(t)^*(\alpha)$ is constant on $\I$.
  Hence,
  $h(t)^*(\alpha)$ is equal to $h(0)^*(\alpha) = \id_\X^*(\alpha) = \alpha$.
  
  3. This is a direct application of 2.
\end{proof}

\begin{article}\artlabel{Contraction of differential forms}
  \addcontentsline{toc}{section}{\small\hspace{10pt} Contraction of differential forms}
  \label{Contraction-of-differential-forms}
  Let $\X$ be a diffeological space,
  and let $\cD$ be its diffeology.
  Let $\P : \U \to \X$ be an $n$-plot.
  We shall call an {\em arc of plots}\index{Arc of plots} or an {\em arc in $\cD$},
  any $1$-parameter family $\F : s \mapsto \P_s$,
  centered at some $\P \in \cD$,
  defined on an interval $\openinterval{-\varepsilon, +\varepsilon}$ with values in $\cD$ such that
  \begin{itemize}
    \item[(a)] $\P_0 = \P$ and $\Dom(\P_s) = \U$,
    for all $s$,
    \item[(b)] $\bar \P : (s,r) \mapsto \P_s(r)$,
    defined on $\openinterval{-\varepsilon,
    \varepsilon} \times \U$,
    is a plot.
  \end{itemize}
  In particular,
  $\F$ is a 1-plot of $\cD$,
  equipped with the functional diffeology \art{Functional-diffeology-of-a-diffeology}.
  The plot $\P = \P_0$ will be called the {\em target} of $\F$.
  Let $\F$ be an arc in $\cD$,
  centered at $\P$,
  and let $\alpha$ be a $p$-form of $\X$,
  $p \geq 1$.
  For all $r \in \U$,
  $\alpha(\bar \P)$ is a smooth $p$-form of $\openinterval{-\varepsilon, +\varepsilon} \times \U$.
  The restriction to  $\set{0} \times \U$ of the contraction of $\alpha(\bar \P)$ by the vector $(1,0) \in \RR \times \RR^n$ \art{Inner-product},
  is the following smooth $(p-1)$-form of $\U$
  $$%
    r \mapsto \bigg[
    (v_2,\ldots,v_{p}) \mapsto \alpha(\bar \P)_{0 \choose r}
    \vect{1 \\ 0}
    \vect{0 \\ v_2}
    \cdots
    \vect{0 \\ v_{p}}
    \bigg]
    \in \Cinfty(\U,\Form^{p-1}(\RR^n)),
    $$%
  with $r$ in $\U$,
  and $v_2,\ldots,v_{p}$ being $(p-1)$ vectors of $\RR^n$.
  Let $\F : s \mapsto \P_s$ and $\F' : s \mapsto \P'_s$ be two arcs of plots centered at the same plot $\P = \P_0 = \P'_0$.
  We shall say that $\F$ and $\F'$ define the same {\em variation\/} $\delta \P$ of the $n$-plot $\P$ if,
  for every $p$-form $\alpha$ of $\X$,
  $p \geq 1$,
  for all $r$ in $\U$ and for all $v_2,\ldots,v_{p} \in \RR^n$,
  \begin{equation}
    \renewcommand{\theequation}{$\heartsuit$}
    \alpha(\bar \P')_{0 \choose r} \vect{1 \\ 0} \vect{0 \\ v_2} \cdots \vect{0 \\ v_{p}} = \alpha(\bar \P)_{0 \choose r} \vect{1 \\ 0} \vect{0 \\ v_2} \cdots \vect{0 \\ v_{p}}.
  \end{equation}
  Thus,
  the variation $\delta \P$ of the plot $\P$ can be regarded as the class of the arc of plots $\F$,
  for the equivalence relation defined by $(\heartsuit)$.
  Let $\alpha \in \Omega^p(\X)$ with $p \geq 1$,
  and $\P : \U \to \X$ be an $n$-plot,
  let $\delta\P$ be a variation of the plot $\P$.
  We shall call a {\em contraction}\index{Contraction differential form} of $\alpha$ by $\delta \P$ the smooth $(p-1)$-form defined on $\U$ by
  \begin{equation}
    \renewcommand{\theequation}{$\diamondsuit$}
    \alpha(\delta \P)(r)(v_1)\cdots(v_{p-1}) =  \alpha(\bar \P)_{0 \choose r} \vect{1 \\ 0} \vect{0 \\ v_1} \cdots \vect{0 \\ v_{p-1}},
  \end{equation}
  where $\F : s \mapsto \P_s$ represents $\delta \P$, $r \in \U$,
  and $v_1,\ldots v_{p-1} \in \RR^n$.
  Be aware that
  $$%
    \alpha(\delta \P) \in \Cinfty(\U, \Form^{p-1}(\RR^n)).
    $$%
  
  \Note{1} To be coherent and to avoid possible confusion,
  we shall denote formally by $\alpha \rfloor \delta \P \in \Omega^{p-1}(\U)$ the differential form,
  where $\U$ is regarded as a diffeological space,
  defined by $(\alpha \rfloor \delta \P)(\id_\U) = \alpha(\delta \P)$.
  
  \Note{2} The contraction of a $p$-form of $\X$,
  by some arc of plots $\F$,
  is not a $(p-1)$ form of $\X$,
  since it is not defined on the plots of $\X$ but on the domain of the target $\P = \F(0)$.
  However,
  in some particular situations,
  for example in \art{Contracting-differential-forms-on-slidings},
  this definition may lead to a true $(p-1)$-form of $\X$.
  
  \Note{3} In the definition of the variation $\delta \P$,
  the value $s=0$,
  where the variation is computed,
  does not really matter.
  We can define as well the variation denoted by $\delta\P_{s}$,
  as the variation for $s=0$ of the translated arc $\F_s: s' \mapsto \F(s'+s)$,
  that is,
  $$%
    \alpha(\delta \P_s)(r)(v_2)\cdots(v_{p}) =
    \alpha(\bar\P)_{s \choose r}
    \vect{1 \\ 0}
    \vect{0 \\ v_2}
    \cdots
    \vect{0 \\ v_{p}}.
    $$%
\end{article} %% Contraction-of-differential-forms

\begin{article}\artlabel{Contracting differential forms on slidings}
  \addcontentsline{toc}{section}{\small\hspace{10pt} Contracting differential forms on slidings}
  \label{Contracting-differential-forms-on-slidings}
  Let $\X$ be a diffeological space and $\Diff(\X)$ be its group of diffeomorphisms,
  equipped with the
  functional diffeology
  \art{Functional-diffeology-on-groups-of-diffeomorphisms}.
  Let $\alpha \in \Omega^p(\X)$ be any differential $p$-form with $p\geq 1$.
  Let $\F : \openinterval{-\epsilon, +\epsilon} \to \Diff(\X)$,
  $\epsilon > 0$, be a {\em sliding} of $\X$,
  that is,
  any
  1-plot centered at the identity,
  $\F(0) = \id_\X$.
  Let ${\P}: \U \to \X$ be any $n$-plot of $\X$.
  We denote by $\F \cdot {\P}$ the $(1+n)$-plot of $\X$ defined by
  $$%
    \F \cdot {\P}: \openinterval{-\epsilon, +\epsilon} \times \U \to \X, \text{ and } \F \cdot {\P} : (t,r) \mapsto \F(t)({\P}(r)).
    $$%
  %
  1. The contraction of $\alpha$ by the arc of plots $t \mapsto [r \mapsto \F \cdot {\P}(t,r)]$ \art{Contraction-of-differential-forms} will be denoted by $i_\F(\alpha)({\P})$.
  It is defined,
  for every $r \in \U$ and for any $(p-1)$ vectors $v_1,\ldots,v_{p-1} \in \RR^n$,
  by
  \begin{equation}
    \renewcommand{\theequation}{$\diamondsuit$}
    i_\F(\alpha)({\P})_r(v_1) \cdots (v_{p-1}) =  \alpha(\F \cdot {\P})_{\left(0 \atop r\right)} \vect{1 \\ 0} \vect{0 \\ v_1} \cdots \vect{0 \\ v_{p-1}}.
  \end{equation}
  The mapping $i_\F(\alpha)$ defined by $(\diamondsuit)$ is a differential $(p-1)$-form of $\X$.
  It will be called the {\em contraction of $\alpha$ by the sliding $\F$}\index{Sliding}.
  
  2. The contraction operation $i_\F: \Omega^p(\X) \to \Omega^{p-1}(\X)$,
  with $p \geq 1$,
  is a smooth linear map,
  where the spaces of differential forms are equipped with the functional diffeology \art{Differential-forms-on-diffeological-spaces}.
  
  \Note{1} Diffeomorphisms are automorphisms,
  they preserve the {\em morphology} of the space.
  A $1$-plot $\F$ of diffeomorphisms of $\X$,
  centered at the origin represents some sliding from $\X$ onto itself,
  and this explains the vocabulary.
  
  \Note{2} Actually,
  the map $(\F,\alpha) \mapsto i_\F(\alpha)$,
  where $\F$ belongs to the space of slidings of $\X$ and $\alpha$ belongs to $\Omega^p(\X)$,
  is smooth.
  
  \Note{3} To use the notation of \art{Contraction-of-differential-forms},
  $i_\F(\alpha)(\P) = \alpha(\delta \P)$,
  where $\delta \P$ is the variation of the arc of plots $[t \mapsto [r \mapsto \F(t)(\P(r))]]$.
\end{article} %% Contracting-differential-forms-on-slidings

\begin{proof}
  1.~The condition for $i_\F(\alpha)$ to be a differential form on $\U$ is to satisfy the compatibility relation
  $$%
    i_\F(\alpha)({\P} \circ \psi) = \psi^*(i_\F(\alpha)({\P})),
    $$%
  where $\psi: \V \to \U$ is some smooth parametrization in $\U$.
  Let $\psi$ be a $m$-plot of $\U$,
  let $s \in \V$ and let $u_1,\ldots,u_{p-1} \in \RR^m$.
  We have then
  $$%
    i_\F(\alpha)({\P}\cdot\psi)_s(u_1,\ldots,u_{p-1}) = \alpha(\F \cdot ({\P} \circ \psi)) \vect{0 \\ s} \vect{1 \\ 0} \vect{0 \\ u_1} \cdots \vect{0 \\ u_{p-1}}.
    $$%
  But
  \begin{eqnarray*}
    \F \cdot({\P} \circ \psi)(t,s) & = & \F(t)({\P} \circ \psi(s))\\
    & = & \F(t)({\P}(\psi(s)))\\
    & = & \F \cdot {\P}(t,\psi(s)) \\
    & = & (\F \cdot {\P}) \circ (\id\times\psi)(t,s),
  \end{eqnarray*}
  where $\id\times\psi(t,v) = (t,\psi(v))$.
  Let us use the more compact notation
  $$%
    \sqvect{0 \\ u} = \vect{0 \\ u_1}\cdots\vect{0 \\ u_{p-1}}.
    $$%
  By posing $v_i = \D(\psi)_s(u_i)$,
  we get from above:
  %
  \begin{eqnarray*}
    \alpha(\F \cdot (\P \circ \psi)) \vect{0 \\ s} \vect{1 \\ 0} \sqvect{0 \\ u} & = & \alpha((\F \cdot \P) \circ (\id\times\psi)) \vect{0 \\ s} \vect{1 \\ 0} \sqvect{0 \\ u} \\
    & = & (\id\times\psi)^*(\alpha(\F \cdot \P)) \vect{0 \\ s} \vect{1 \\ 0} \sqvect{0 \\ u} \\
    & = & \alpha(\F \cdot \P) \vect{0 \\ \psi(s)} \vect{1 \\ 0} \sqvect{0 \\ v}.
  \end{eqnarray*}
  Hence,
  we get finally,
  with $\D(\psi)_s[u]$ for $(\D(\psi)_s(u_1)) \cdots (\D(\psi)_s(u_{p-1}))$,
  \begin{eqnarray*}
    \alpha(\F \cdot (\P \circ \psi)) \vect{0 \\ s} \vect{1 \\ 0} \sqvect{0 \\ u} & = & i_\F(\alpha)({\P})_{\psi(s)}(\D(\psi)_s[u]) \\
    & = & \psi^*(i_\F(\alpha)({\P}))_s[u].
  \end{eqnarray*}
  Therefore, $i_\F(\alpha)({\P} \circ \psi) = \psi^*(i_\F(\alpha)({\P}))$,
  and $i_\F(\alpha)$ is a $(p-1)$-form of $\X$.
  
  2. The contraction $i_\F$ is obviously linear.
  Let us prove that it is smooth.
  Let $s \mapsto \beta_s$ be a plot of $\Omega^p(\X)$.
  We have to check that $s \mapsto i_\F(\beta_s)$ is a plot of $\Omega^{p-1}(\X)$.
  Let ${\P}: \U \to \X$ be a plot,
  we have
  $$%
    i_\F(\beta_s)({\P})_r[v] =  \beta_s(\F \cdot \P) \vect{0 \\ r} \vect{1 \\ 0} \sqvect{0 \\ v},
    $$%
  with the same notation as above for $v$.
  But $\F \cdot \P : (t,r) \mapsto \F(t)({\P}(r))$ is a plot of $\X$,
  thus $(t,r,s) \mapsto \beta_s(\F \cdot \P)_{(t,r)}$ is smooth.
  Then,
  $$%
    (r,s) \mapsto
    \beta_s(\F \cdot \P)
    \vect{0 \\ r}
    \vect{1 \\ 0}
    $$%
  is smooth.
  Hence, $(r,s) \mapsto i_\F(\beta_s)({\P})_r$ is a plot of $\Omega^{p-1}(\X)$ and the contraction $i_\F: \Omega^p(\X) \to \Omega^{p-1}(\X)$ is a smooth linear map.
\end{proof}

%%%%%%%%%%%%%%%%%%%%%%%%%%%%%%%%%%%%%%%%%%%%%%%%%%%%%%%%%%
%
%   Exercises
%
%%%%%%%%%%%%%%%%%%%%%%%%%%%%%%%%%%%%%%%%%%%%%%%%%%%%%%%%%%

\Exercises

\begin{exercise}[Anti-Lie derivative]
  \label{Anti-Lie-derivative}
  Let $\X$ be a diffeological space,
  let $\alpha$ be a $p$-form of $\X$,
  and let $\F$ be a 1-plot of $\Diff(\X)$ centered at the identity.
  For all $\A \in \Diff(\X)$,
  let $\A_*$ denote the pushforward $\A_*(\alpha) = (\A^{-1})^*(\alpha)$.
  Use the identity $\id_\X = \F(t) \circ \F(t)^{-1}$,
  and that $\F(0) = \id_\X$,
  to show that
  $$%
    {\partial \over \partial t} \bigg\{ \F(t)_*(\alpha) \bigg\}_{t=0} = - \DLie_\F(\alpha).
    $$%
\end{exercise} %% Anti-Lie-derivative

\begin{exercise}[Multi-Lie derivative]
  \label{Multi-Lie-derivative}
  The notion of Lie derivative can be extended to any plot of $\Diff(\X)$.
  Let us define,
  with the kinds of notation and conventions of \art{The-Lie-derivative-of-differential-forms},
  for every $q$-plot $h$ of $\Diff(\X)$  centered at the identity and for every vector $v \in \RR^q$,
  $$%
    \DLie_h(\alpha)(v) = {\partial \over \partial s} \bigg\{ h(s)^* \alpha \bigg\}_{s = 0}(v).
    $$%
  
  \Question{1)}~Check that $\DLie_h(\alpha)(v)$ is a $k$-form of $\X$.
  Actually,
  check that $\DLie_h(\alpha)$ is a smooth linear map from $\RR^q$ to $\DForms^k(\X)$.
  
  \Question{2)}~Let $v = \sum_i v_i\, \ee_i$, where the $\ee_i$ are the vectors of the canonical basis of $\RR^q$.
  And let
  $h_i :
  t \mapsto h(t \ee_i)$ be defined on some open neighborhood of $0 \in \RR$.
  Show that
  $$%
    \DLie_h(\alpha)(v) = \sum_{i = 1}^q v^i \,
    \DLie_{h_i}(\alpha).
    $$%
  As a corollary of this linearity,
  $\DLie_h(\alpha)({\P}) = 0$ for every $q$-plot $h$ of $\Diff(\X)$ if and only if $\DLie_h(\alpha)({\P}) = 0$ for every $1$-plot $h$ of $\Diff(\X)$.
\end{exercise} %% Multi-Lie-derivative

\begin{exercise}[Variations of points of domains]
  \label{Variations-of-points-of-domains}
  Let $\U$ be an $n$-domain and $x \in \U$.
  Let $\bmx = [0 \mapsto x]$ be the $0$-plot with value $x$.
  Describe the variations $\delta \bmx$ of $\bmx$.
\end{exercise} %% Variations-of-points-of-domains

\begin{exercise}[Liouville rays]
  \label{Liouville-rays}
  Let $\omega$ be a $p$-form on a diffeological space $\X$,
  assume $\omega \neq 0$.
  Let $h \in \DHom(\RR,\Diff(\X))$ be a ray such that,
  for all $t \in \RR$, $h(t)^*(\omega) = \lambda(t) \times \omega$.
  Show that $\lambda(t) = e^t$;
  see \exref{Liouville-rays-and-closed-forms}.
\end{exercise} %% Liouville-rays

%%%%%%%%%%%%%%%%%%%%%%%%%%%%%%%%%%%%%%%%%%%%%%%%%%%%%%%%%%
%% MARK: Cubes and Homology of Diffeological Spaces
%%%%%%%%%%%%%%%%%%%%%%%%%%%%%%%%%%%%%%%%%%%%%%%%%%%%%%%%%%

\section*{Cubes and Homology of Diffeological Spaces}
\label{Section-Cubes-and-homology-of-diffeological-spaces}

\begin{sechead}
  We shall use the singular cubic homology and cohomology because they are naturally adapted,
  in the diffeological framework,
  to the integration of forms \art{Integration-of-a-smooth-p-form-on-a-p-cube} and to the variational calculus \art{Variation-of-the-integral-of-a-form-on-a-cube}.
  We shall see further the main theorems relative to the De Rham theory in this context.
  For a short description of cubic homology in classical differential geometry, see for example \cite{HoYo61}.
\end{sechead}

\begin{article}\artlabel{Cubes and cubic chains on diffeological spaces}
  \addcontentsline{toc}{section}{\small\hspace{10pt} Cubes and cubic chains on diffeological spaces}
  \label{Cubes-and-cubic-chains-on-diffeological-spaces}
  We call {\em standard $p$-cube}\index{pcube@$p$-Cube} the subset $[0,1]^p$ of $\RR^p$,
  and we denote it by $\cub{p}$,
  $$%
    \cub{p} = [0,1]^p\subset \RR^p.
    $$%
  Let $\X$ be a diffeological space.
  We call a {\em smooth $p$-cube}\index{pcube@$p$-Cube} in $\X$ any smooth map from $\RR^p$ to $\X$.
  And we denote by $\DCubes{p}(\X)$ the set of all the smooth $p$-cubes in $\X$,
  $$%
    \DCubes{p}(\X) = \Cinfty(\RR^p,\X).
    $$%
  The set $\DCubes{p}(\X)$ will be equipped with the functional diffeology;
  see \art{Functional-diffeologies} and \art{Functional-diffeology-and-products}.
  Note that since $0$-cubes are any maps from $\cub{0} = \RR^0 = \{0\}$ to $\X$,
  then $\Cub_0(\X)$ is naturally equivalent to $\X$,
  thanks to $x \mapsto \bmx = [0 \mapsto x]$.
  Hence,
  $$%
    \Cub_0(\X) \simeq \X.
    $$%
  Then,
  we define the {\em smooth cubic $p$-chains}\index{Cubic chain} in $\X$,
  with coefficients in $\ZZ$,
  as the free Abelian group generated by $\DCubes{p}(\X)$,
  and we denote it by $\DChains{p}(\X)$.
  Thus,
  a (smooth) cubic $p$-chain $c$,
  in $\X$,
  is any finite $\ZZ$-linear combination  of $p$-cubes,
  that is,
  $$%
    c = \sum_\sigma n_\sigma \, \sigma,
    \text{ with }
    \sigma \in \DCubes{p}(\X), \text{ and } n_\sigma \in \ZZ,
    $$%
  where the sum is performed over a finite set of $p$-cubes called the {\em support} of $c$,
  and denoted by
  $$%
    \Supp(c) = \{ \sigma \in \DCubes{p}(\X) \mid n_\sigma \neq 0 \}.
    $$%
  The group of cubic $p$-chains $\DChains{p}(\X)$ can also be represented by
  $$%
    \DChains{p}(\X) \simeq \{ c \in \Maps(\DCubes{p}(\X),\ZZ) \mid \#\Supp(c) <\infty \}.
    $$%
  Note that in the writing $\sum_\sigma n_\sigma \sigma$ of the chain $c$,
  $n_\sigma = c(\sigma)$.
  Then,
  the sum of two cubic $p$-chains $c$ and $c'$,
  and the multiplication of a cubic $p$-chain $c$ by an integer $m$,
  are defined as usual:
  $$%
    (c + c')(\sigma) = c(\sigma)  + c'(\sigma), \text{ and } (mc)(\sigma) = m \times c(\sigma).
    $$%
  \Note{1} A cubic chain can also be regarded as any finite family $\{(n_i,\sigma_i)\}_{i \in \cI}$ and can be written $\sum_{i \in \cI} n_i \sigma_i$,
  with the convention that if $\sigma_i = \sigma_j$,
  then $\sum_{i \in \cI} n_i \sigma_i = \sum_{i' \in \cI'} n_{i'} \sigma_{i'} + (n_i + n_j)\sigma_i$,
  where $\cI' = \cI - \{i,j\}$.
  Since the family is finite,
  the sum of the coeffcients of a same cube is finite and both aspects are equivalent.
  %
  As a free Abelian group,
  the set of cubic $p$-chains is a quotient of the free group $\W_p(\X)$ over the set of $p$-cubes.
  A word $w$ is a string $w = \prod_{i=1}^\N \sigma_i^{\pm 1}$,
  $\sigma_i \in \DCubes{p}(\X)$ and $\sigma_i^{-1}$ is the ``opposite'' of $\sigma_i$,
  with the usual rules of cancelation.
  The integer $\N$ is the {\em length} of $w$.
  The {\em Abelianization} consists of the formal map $\prod_{i=1}^\N \sigma_i^{\pm 1} \mapsto \sum_{i=1}^\N \pm \sigma_i$,
  from $\W_p(\X)$ to $\DChains{p}(\X)$,
  with the usual rules of addition and cancelation.
  The set $\W_p$ is the sum over $\N$ of the set $\W_p^\N$ of words of length $\N$,
  and the set of words of length $\N$ is the $\N$-th power of $\DCubes{p}(\X) \times \{\pm 1\}$.
  Then,
  $\W_p$ can be equipped with the diffeology sum of the product diffeology,
  a plot takes its values locally in some $\W_p^\N$ and writes locally $r \mapsto \prod_{i=1}^\N \sigma_i^{\pm 1}(r)$,
  where $r \mapsto \sigma_i(r)$ is a plot in $\DCubes{p}(\X)$.
  The set $\DChains{p}(\X)$ is equipped with the pushforward of this diffeology by the Abelianization.
  %
  %%###########
  \begin{figure}[tb]
    \centerline{\includegraphics{Figures-PDF/fig-cube-smashing-function}}
    \caption{The cube's smashing function.}
    \label{fig-cube-smashing-function}
  \end{figure}
  %%###########
  
  \Note{2} With smooth homology or cohomology in mind,
  there is no contradiction in defining smooth $p$-cubes in $\X$ as smooth maps from $\RR^p$ to $\X$,
  as we do here,
  or as the maps from $\cub{p}$ to $\X$ which are the restrictions of smooth maps defined on an open neighborhood of $\cub{p}$,
  as we could have also chosen to do.
  Indeed the following proposition addresses this issue.
  \begin{enumerate}
    \item[($\diamondsuit$)] Every $p$-plot of $\X$ defined on a small open neighborhood of $\cub{p}$ coincides,
    on $\cub{p}$,
    with some global $p$-plot of $\X$.
  \end{enumerate}
  This is why, for sake of simplicity and without loss of generality,
  the smooth $p$-cubes in $\X$ have been defined as the global $p$-plots of $\X$.
  But to focus our attention on $\cub{p} \in \RR^p$ we have introduced a special name,
  $p$-cube instead of global $p$-plot,
  and a special notation $\DCubes{p}(\X)$ for $\Cinfty(\RR^p,\X)$).
\end{article} %% Cubes-and-cubic-chains-on-diffeological-spaces

\begin{proof}
  For the second note,
  let ${\P}$ be a $p$-plot of $\X$,
  defined on some open neighborhood of $\cub{p}$.
  Since $\I^p$ is compact there exists a real $\epsilon >0$,
  small enough,
  such that $]-\epsilon,1+\epsilon[\,{}^p \subset \Dom(\P)$.
  Let us restrict $\P$ to this small open $p$-cube.
  Now,
  let $\lambda \in \Cinfty(\RR,\RR)$ be a smooth function,
  as shown in Figure \ref{fig-cube-smashing-function},
  equal to $1$ on a superset of $[0,1]$,
  equal to $0$ on a superset of $]-\infty,\epsilon] \cup [1+\epsilon,+\infty[$,
  increasing between $-\epsilon$ and $0$,
  decreasing between $1$ and $1+\epsilon$.
  Then, let us define
  $$%
    \P'(t_1,\ldots,t_p) = \P(\lambda(t_1)\times t_1,\ldots,\lambda(t_p)\times t_p),
    \text{ if } (t_1,\ldots,t_p) \in \left]-\epsilon,1+\epsilon\right[{}^p;
    $$%
  thus
  ${\P}'(t_1,\ldots,t_p) = 0$ outside $\left]-\epsilon,1+\epsilon\right[{}^p$.
  The parametrization ${\P}'$ is a global $p$-plot of $\X$,
  equal to ${\P}$ on $\cub{p}$ and $0$ outside of a small open neighborhood of $\cub{p}$.
\end{proof}

\begin{article}\artlabel{Boundary of cubes and chains}
  \addcontentsline{toc}{section}{\small\hspace{10pt} Boundary of cubes and chains}
  \label{Boundary-of-cubes-and-chains}
  Let $\X$ be a diffeological space.
  The integration of a smooth $p$-form on a $p$-cube \art{Integration-of-a-smooth-p-form-on-a-p-cube} suggests the introduction of a {\em boundary}\index{Cube} operation on cubic chains,
  denoted by  $\Der$,
  and satisfying the {\em homological condition} $\Der \circ \Der = 0$.
  Let us consider the simplest example: Let $f$ be a smooth function defined on the cube $c = [a,b]$.
  Let $\F$ be a primitive of $f$,
  $\F' = f$ or $d\F = f$.
  We can write
  $$%
    \int_c f = \int_c d\F = \F(b) - \F(a) = [\F]_a^b = \int_{\Der c} \F, \text{ with } \Der c = b-a.
    $$%
  Here,
  the string $\Der c = b-a$ is interpreted as a $0$-chain $\Der c = 1\times \bb + (-1) \times \aa$,
  where $\aa: 0 \mapsto a$ and $\bb: 0 \mapsto b$.
  We shall extend this particular case to the general case on $\X$ as follows.
  Let us describe the cubic chain-complex.
  
  \textbf{$0$-Chains.} For every $x \in \X$, let $\xx: 0 \mapsto x$ be the $0$-cube with value $x$.
  Let
  $$%
    \Der \xx \text{ (or $\Der x$) } = 0.
    $$%
  Now, $\Der$ is extended by linearity on every $0$-chain $c = \sum_x n_x x \in \DChains{0}(\X)$,
  and
  $$%
    \Der \left(\sum_x n_x x\right) =  \sum_x n_x \Der x = 0.
    $$%
  In brief,
  $\Der(\DChains{0}(\X)) = \{0\}$,
  or $\Der : \DChains{0}(\X) \to \{0\}$.
  
  \textbf{$1$-Chains.} Let $\sigma \in \DCubes{1}(\X)$ be a smooth $1$-cube,
  and we define
  $$%
    \Der \sigma = \sigma(1) - \sigma(0).
    $$%
  Extended by linearity on any $1$-chain $c = \sum_\sigma n_\sigma \sigma \in \DChains{1}(\X)$,
  we have $\Der: \DChains{1}(\X) \to \DChains{0}(\X)$ with
  $$%
    \Der
    \left(\sum_\sigma n_\sigma \sigma\right) = \sum_\sigma n_\sigma \Der\sigma = \sum_\sigma n_\sigma (\sigma(1) - \sigma(0)).
    $$%
  Now,
  since for all $1$-chains $c$,
  $\Der(c)$ is a $0$-chain,
  and since $\Der$ vanishes on the $0$-chain,
  we have $\Der(\Der(c)) =0$, for all $c \in \DChains{1}(\X)$.
  
  \textbf{$2$-Chains.} Let $\sigma \in \DCubes{2}(\X)$ be any smooth $2$-cube,
  and we define
  $$%
    \Der \sigma (t) = [\sigma(1)(t) - \sigma(0)(t)] - [\sigma(t)(1) - \sigma(t)(0)].
    $$%
  Extended by linearity on any $2$-chain $c = \sum_\sigma n_\sigma \sigma \in \DChains{2}(\X)$,
  we have $\Der: \DChains{2}(\X) \to \DChains{1}(\X)$ with
  $$%
    \Der \left(\sum_\sigma n_\sigma \sigma\right) = \sum_\sigma n_\sigma \Der\sigma = \sum_\sigma n_\sigma \{[\sigma(1)(t) - \sigma(0)(t)] - [\sigma(t)(1) - \sigma(t)(0)]\}.
    $$%
  Now, for any $2$-cube $\sigma$ we have
  \begin{eqnarray*}
    \Der(\Der \sigma) & = &\Der([\sigma(1)(t) - \sigma(0)(t)] - [\sigma(t)(1) - \sigma(t)(0)]) \\
    & = & \{[\sigma(1)(1) - \sigma(1)(0)] - [\sigma(0)(1) - \sigma(0)(0)]\} \\
    &-& \{[\sigma(1)(1) - \sigma(0)(1)] - [\sigma(1)(0) - \sigma(0)(0)]\} \\
    & = & \sigma(1)(1) - \sigma(1)(0) - \sigma(0)(1) + \sigma(0)(0) \\
    & - & \sigma(1)(1) + \sigma(0)(1) + \sigma(1)(0) - \sigma(0)(0) \\
    & = & 0.
  \end{eqnarray*}
  Hence,
  by linearity,
  for all $2$-chains $c \in \DChains{2}(\X)$,
  we have also $\Der (\Der c) = 0$.
  
  \textbf{$3$-Chains.} Let us continue once more,
  to see clearly the pattern,
  before giving the general definition.
  Let $\sigma \in \DCubes{3}(\X)$ be a smooth $3$-cube,
  and we define its boundary
  \begin{eqnarray*}
    \Der\sigma(t_1)(t_2) & = & [\sigma(1)(t_1)(t_2) - \sigma(0)(t_1)(t_2)] \\
    & -& [\sigma(t_1)(1)(t_2) - \sigma(t_1)(0)(t_2)] \\
    & + & [\sigma(t_1)(t_2)(1) - \sigma(t_1)(t_2)(0)].
  \end{eqnarray*}
  As usual the boundary operator $\Der$ is extended by linearity on any $3$-chain $c \in \DChains{3}(\X)$,
  and $\Der: \DChains{3}(\X) \to \DChains{2}(\X)$.
  The verification that $\Der(\Der \sigma) = 0$,
  for any $\sigma \in \DCubes{3}(\X)$, is left as an exercise.
  
  \textbf{$p$-Chains.} We are now ready to give the general definition for the boundary of a cubic $(p+1)$-chain of a diffeological space $\X$.
  Let $\sigma$ be a $(p+1)$-cube of $\X$,
  that is,
  $\sigma \in \DCubes{p+1}(\X) = \Cinfty(\RR^{p+1},\X)$,
  with $p\geq 0$.
  We define the boundary $\Der \sigma$ of the $(p+1)$-cube $\sigma$ as the following cubic $p$-chain in $\X$:
  
  \pagebreak
  \begin{eqnarray*}
    \Der\sigma(t_1)(t_2)\cdots(t_p) & = & (-1)^1 \times [\sigma(0)(t_1)(t_2)\cdots(t_p) - \sigma(1)(t_1)(t_2)\cdots(t_p)] \\
    & + & (-1)^2 \times [\sigma(t_1)(0)(t_2)\cdots(t_p) - \sigma(t_1)(1)(t_2)\cdots(t_p)] \\
    & + & (-1)^3 \times [\sigma(t_1)(t_2)(0)\cdots(t_p) - \sigma(t_1)(t_2)(1)\cdots(t_p)] \\
    & \cdots & \\
    &+ & (-1)^{p+1} \times [\sigma(t_1)(t_2)\cdots(t_p)(0) - \sigma(t_1)(t_2)\cdots(t_p)(1)].
  \end{eqnarray*}
  To give a more compact expression of the boundary operator $\Der$, we introduce the following family of injections $j_k(a) : \RR^p \to \RR^{p+1}$, $k = 1, \ldots, p+1$ and $a\in\RR$:
  $$%
    \renewcommand{\arraystretch}{1.3}
    \begin{array}{crcl}
    k = 1 & j_1(a): (t_1)\cdots(t_p) & \mapsto & (a)(t_1)\cdots(t_p), \\
    1 < k \leq p & j_k(a): (t_1)\cdots(t_p) & \mapsto & (t_1)\cdots(t_{k-1})(a)(t_k)\cdots(t_p), \\
    k = p+1 & j_{p+1}(a): (t_1)\cdots(t_p) & \mapsto & (t_1)\cdots(t_p)(a).
    \end{array}
    $$%
  Given a $p$-tuple of numbers,
  $j_k(a)$ puts $a$ at the place number $k$,
  preserving the numbers before and shifting the numbers after.
  Therefore,
  the boundary operator $\Der$ defined above writes again,
  for $p \geq 1$\,:
  \begin{equation}
    \renewcommand{\theequation}{$\diamondsuit$}
    \text{For all } \sigma \in \DCubes{p}(\X), \ \Der\sigma =  \sum_{k = 1}^p (-1)^k[\sigma \circ j_k(0) - \sigma \circ j_k(1)].
  \end{equation}
  The operator $\Der$ defined by $(\diamondsuit)$ is naturally extended by linearity on all cubic $p$-chains,
  with $p \geq 1$,
  and
  \begin{equation}
    \renewcommand{\theequation}{$\heartsuit$}
    \text{for all } c = \sum_\sigma n_\sigma \sigma \in \DChains{p}(\X),
    \ \Der c = \sum_\sigma n_\sigma \sum_{k = 1}^p (-1)^k[\sigma \circ j_k(0) - \sigma \circ j_k(1)].
  \end{equation}
  The operator $\Der$ defined by $(\heartsuit)$ is a boundary operator:
  %that is,
  %$\Der \circ \Der = 0$,
  %and
  \begin{equation}
    \renewcommand{\theequation}{$\clubsuit$}
    \cdots
    {\buildrel\Der\over{\ \longrightarrow \ }}
    \DChains{p}(\X)
    {\buildrel\Der\over{\ \longrightarrow \ }}
    \DChains{p-1}(\X)
    {\buildrel\Der\over{\ \longrightarrow \ }}
    \cdots
    {\buildrel\Der\over{\ \longrightarrow \ }}
    \DChains{0}(\X)
    {\buildrel\Der\over{\ \longrightarrow \ }}
    \{0\},
    \text{ with } \Der \circ \Der = 0.
  \end{equation}
\end{article} %% Boundary-of-cubes-and-chains

\begin{proof}
  What we have to prove is that the operator $\Der$ defined by $(\heartsuit)$ satisfies $\Der \circ \Der = 0$.
  According to the definition $(\diamondsuit)$,
  we have just to check it on cubes.
  Let $p\geq 0$ and $\sigma$ be a $(p+2)$-cube,
  thus $\Der\sigma$ is a cubic $(p+1)$-chain,
  and $\Der^2\sigma = \Der\Der\sigma$ is a cubic $p$-chain.
  We have
  $$%
    \Der\sigma  =  \sum_{\ell = 1}^{p+2}(-1)^\ell[\sigma \circ j_\ell(0) - \sigma \circ j_\ell(1)].
    $$%
  Then,
  \begin{eqnarray*}
    \Der^2\sigma & = & \sum_{\ell = 1}^{p+2}(-1)^\ell\{\Der[\sigma \circ j_\ell(0)] - \Der[\sigma \circ j_\ell(1)]\} \\
    & = & \sum_{\ell = 1}^{p+2}(-1)^\ell\Der[\sigma \circ j_\ell(0)] - \sum_{\ell = 1}^{p+2}(-1)^\ell \Der[\sigma \circ j_\ell(1)].
  \end{eqnarray*}
  But $\sigma \circ j_\ell(a)$ is a $(p+1)$-cube,
  with here $a=0$ or $1$,
  hence
  \begin{eqnarray*}
    \Der[\sigma \circ j_\ell(a)] & = & \sum_{k = 1}^{p+1}(-1)^k [(\sigma \circ j_\ell(a)) \circ j_k(0) - (\sigma \circ j_\ell(a)) \circ j_k(1)] \\
    & = & \sum_{k = 1}^{p+1}(-1)^k [\sigma \circ j_\ell(a) \circ j_k(0) - \sigma \circ j_\ell(a) \circ j_k(1)].
  \end{eqnarray*}
  Thus,
  \begin{eqnarray*}
    \Der^2\sigma & = & \sum_{\ell = 1}^{p+2} (-1)^\ell \sum_{k = 1}^{p+1}(-1)^k \sigma \circ j_\ell(0) \circ j_k(0) - \sum_{\ell = 1}^{p+2} (-1)^\ell \sum_{k = 1}^{p+1}(-1)^k \sigma \circ j_\ell(0) \circ j_k(1)\\
    &-& \sum_{\ell = 1}^{p+2} (-1)^\ell \sum_{k = 1}^{p+1}(-1)^k \sigma \circ j_\ell(1) \circ j_k(0) + \sum_{\ell = 1}^{p+2} (-1)^\ell \sum_{k = 1}^{p+1}(-1)^k \sigma \circ j_\ell(1) \circ j_k(1) \\
    & = & \sum_{\ell = 1}^{p+2}\sum_{k = 1}^{p+1}(-1)^{k+l} \sigma \circ j_\ell(0) \circ j_k(0) \\
    &-& \sum_{\ell = 1}^{p+2}\sum_{k = 1}^{p+1}(-1)^{k+l} [\sigma \circ j_\ell(0) \circ j_k(1) + \sigma \circ j_\ell(1) \circ j_k(0)] \\
    &+& \sum_{\ell = 1}^{p+2}\sum_{k = 1}^{p+1}(-1)^{k+l} \sigma \circ j_\ell(1) \circ j_k(1).
  \end{eqnarray*}
  Let us denote
  \begin{eqnarray*}
    (\A) & = & \sum_{\ell = 1}^{p+2}\sum_{k = 1}^{p+1}(-1)^{k+l} \sigma \circ j_\ell(0) \circ j_k(0), \\
    ({\B}) & = & \sum_{\ell = 1}^{p+2}\sum_{k = 1}^{p+1}(-1)^{k+l} [\sigma \circ j_\ell(0) \circ j_k(1) + \sigma \circ j_\ell(1) \circ j_k(0)], \\
    ({\C}) & = & \sum_{\ell = 1}^{p+2}\sum_{k = 1}^{p+1}(-1)^{k+l} \sigma \circ j_\ell(1) \circ j_k(1).
  \end{eqnarray*}
  Now,
  let us prove that $(\A)$,
  $(\B)$, and $(\C)$ are zero.
  Let us define first,
  for any pair $(a,b) \in \{0,1\}^2$,
  the map
  $$%
    j_{\ell, k}(a)(b): \RR^p \to \RR^{p+2},
    $$%
  which puts $a$ at the place $\ell$ and $b$ at the place $k$,
  distributing the other variables into the free places,
  respecting their initial distribution.
  This map is defined for any pair of indices $k\neq \ell$ which makes sense.
  $$%
    \begin{array}{rl}
    \text{For example} & j_{1,2}(a)(b)() \to (a)(b) \\
    \text{or} & j_{1,3}(a)(b)(t) \to (a) \bullet (b) \to (a)(t)(b), \\
    \text{etc.} & j_{4,2}(a)(b)(r)(t)(s) \to \bullet (b) \bullet (a) \bullet \to  (r)(b)(t)(a)(s).
    \end{array}
    $$%
  Note the variance of $j_{\ell,k}$,
  $$%
    j_{\ell,k}(a)(b) = j_{k,\ell}(b)(a).
    $$%
  Now, a simple verification shows that
  $$%
    (*) \quad \left\{
    \begin{array}{ll}
    j_\ell \circ j_k = j_{\ell,k+1} & \text{if \  $\ell\leq k$,} \\
    j_\ell \circ j_k = j_{\ell,k} & \text{if \ $\ell> k$.}
    \end{array}
    \right.
    $$%
  Let us denote,
  for the sake of convenience,
  $$%
    \S_p(a)(b) = \sum_{\ell = 1}^{p+2}\sum_{k = 1}^{p+1}(-1)^{k+\ell} \sigma \circ j_\ell(a) \circ j_k(b),
    $$%
  such that $(\A) = \S_p(0)(0)$,
  $(\B) = \S_p(1)(0)+\S_p(0)(1)$,
  $(\C) = \S_p(1)(1)$.
  Let us show that $\S_p$ is antisymmetric,
  $\S_p(a)(b) = -\S_p(b)(a)$,
  and we shall conclude that $(A)$, $(B)$, and $(C)$ are zero.
  Using the identity $(*)$ above,
  we have
  \begin{eqnarray*}
    \sum_{\ell = 1}^{p+2}\sum_{k = 1}^{p+1}(-1)^{k+l} \sigma \circ j_\ell(a) \circ j_k(b) & = & \sum_{1\leq \ell \leq k \atop 1\leq k<p+2} (-1)^{k+\ell} \sigma \circ j_\ell(a) \circ j_k(b) \\
    &+& \sum_{k< \ell \leq p+2 \atop 1\leq k<p+2} (-1)^{k+\ell} \sigma \circ j_\ell(a) \circ j_k (b) \\
    & = & \sum_{1\leq \ell \leq k \atop 1\leq k<p+2} (-1)^{k+\ell} \sigma \circ j_{\ell,k+1}(a)(b) \\
    &+& \sum_{k< \ell \leq p+2 \atop 1\leq k<p+2} (-1)^{k+\ell} \sigma \circ j_{\ell,k}(a)(b).
  \end{eqnarray*}
  Now,
  the first term of the right-hand side of the last equality becomes
  \begin{eqnarray*}
    \sum_{1\leq \ell \leq k \atop 1\leq k<p+2} (-1)^{k+\ell} \sigma \circ j_{\ell,k+1}(a)(b) & = & \sum_{1\leq \ell < k \atop 2\leq k\leq p+2} (-1)^{k+\ell-1} \sigma \circ j_{\ell,k}(a)(b) \\
    & = & \sum_{1\leq \ell < k \leq p+2} (-1)^{k+\ell-1} \sigma \circ j_{\ell,k}(a)(b) \\
    & = & - \sum_{1\leq \ell < k \leq p+2} (-1)^{k+\ell} \sigma \circ j_{\ell,k}(a)(b).
  \end{eqnarray*}
  Then, using the variance of $j_{\ell,k}$,
  the second term
  writes
  \begin{eqnarray*}
    \sum_{k< \ell \leq p+2 \atop 1\leq k<p+2} (-1)^{k+\ell} \sigma \circ j_{\ell,k}(a)(b) & = & \sum_{\ell < k \leq p+2 \atop 1\leq \ell<p+2} (-1)^{\ell+k} \sigma \circ j_{k,\ell}(a)(b) \\
    & = & \sum_{\ell < k \leq p+2 \atop 1\leq \ell<p+2} (-1)^{\ell+k} \sigma \circ j_{\ell,k}(b)(a) \\
    & = & \sum_{1\leq \ell < k \leq p+2} (-1)^{k+\ell} \sigma \circ j_{\ell,k}(b)(a).
  \end{eqnarray*}
  Hence, joining the last two expressions,
  we get
  \begin{eqnarray*} \sum_{\ell = 1}^{p+2}\sum_{k = 1}^{p+1}(-1)^{k+\ell} \sigma \circ j_\ell(a) \circ j_k(b) & = & \sum_{1\leq \ell < k \leq p+2} (-1)^{k+\ell} \sigma \circ j_{\ell,k}(b)(a) \\
    &-& \sum_{1\leq \ell < k \leq p+2} (-1)^{k+\ell} \sigma \circ j_{\ell,k}(a)(b).
  \end{eqnarray*}
  And,
  finally
  $$%
    \S_p(a)(b) = \sum_{1\leq \ell < k \leq p+2} (-1)^{k+\ell} [\sigma \circ j_{\ell,k}(b)(a) - \sigma \circ j_{\ell,k}(a)(b)].
    $$%
  Thus,
  $\S_p(a)(b)$ is clearly antisymmetric,
  hence,
  $(A)$,
  $(B)$,
  and $(C)$ are zero.
  Therefore,
  $\Der(\Der\sigma) = (\A)-(\B)+(\C) = 0$,
  and the operator $\Der$ is a boundary operator.
\end{proof}

\begin{article}\artlabel{Degenerate cubes and chains}
  \addcontentsline{toc}{section}{\small\hspace{10pt} Degenerate cubes and chains}
  \label{Degenerate-cubes-and-chains}
  Let $p$ and $q$ be two integers such that $0 \leq q < p$.
  We call a {\em reduction} from $\RR^p$ to $\RR^q$ any projection $\pr : \RR^p \to \RR^q$ such that $\pr(t_1, \ldots, t_p) = (t_{i_1}, \ldots ,t_{i_q})$,
  where $\{i_1,\ldots,i_q\} \subset \{1, \ldots, p\}$ is a subset of indices,
  and $i_1 < \cdots <i_q$.
  For $q = 0$,
  there is only one reduction:
  the constant map $\hat 0 : (t_1,\ldots,t_p) \mapsto 0$.
  So,
  a reduction from $\RR^p$ to $\RR^q$ consists of just ``forgeting'' some,
  or all,
  of the components of $t = (t_1, \ldots , t_p) \in \RR^p$.
  Now,
  let $\X$ be a diffeological space.
  Let $p > 0$ be an integer,
  we say
  that a $p$-cube $\sigma \in \DCubes{p}(\X)$ is {\em degenerate}\index{Degenerate} if there exist an integer $q$ such that $0 \leq q < p$,
  a reduction $\pr$ from $\RR^p$ to $\RR^q$ and a $q$-cube $\sigma' \in \DCubes{q}(\X)$ such that $\sigma = \sigma' \circ \pr$.
  Said otherwise,
  a $p$-cube is degenerate if it does not depend on some coordinates of $\RR^p$.
  Let us denote by $\DCubes{p}^\bullet(\X)$ the set of degenerate $p$-cubes of $\X$,
  and let us denote by $\DChains{p}^\bullet(\X)$ the free Abelian group generated by $\DCubes{p}^\bullet(\X)$.
  The elements of $\DChains{p}^\bullet(\X)$ will be called the {\em degenerate cubic $p$-chains} of $\X$.
  For $p = 0$,
  we agree that
  $$%
    \DCubes{0}^\bullet(\X) = \varnothing, \text{ and } \DChains{0}^\bullet(\X) = \{0\}.
    $$%
  We define the {\em reduced group of cubic $p$-chains of $\X$}\index{Reduced $p$-chain},
  denoted by $\QDChains{p}(\X)$,
  as the quotient of the group of cubic $p$-chains of $\X$ by its subgroup of degenerate cubic $p$-chains,
  that is,
  $$%
    \QDChains{p}(\X) = \DChains{p}(\X)/ \DChains{p}^\bullet(\X).
    $$%
  Note that $\QDChains{0}(\X) = \DChains{0}(\X)/ \DChains{0}^\bullet(\X) = \DChains{0}(\X)/\{0\} = \DChains{0}(\X)$.
  Now,
  for any integer $p > 0$,
  the boundary \art{Boundary-of-cubes-and-chains} of any degenerate $p$-cube is a degenerate cubic $p$-chain,
  that is,
  $$%
    \text{for all } \sigma \in \DCubes{p}^\bullet(\X), \ \Der \sigma \in \DChains{p-1}^\bullet(\X).
    $$%
  Then by linearity we get that,
  $$%
    \text{for all } c \in \DChains{p}^\bullet(\X), \ \Der c \in \DChains{p-1}^\bullet(\X), \text{ or } \Der[\DChains{p}^\bullet(\X)] \subset \DChains{p-1}^\bullet(\X).
    $$%
  Thus,
  there exists an operator,
  denoted again by $\Der$,
  from $\QDChains{p}(\X)$ to $\QDChains{p-1}(\X)$,
  such that the following diagram commutes
  
  \begin{center}
    \begin{tikzcd}[column sep=large, row sep=large, every label/.append style = {font = \small}]
      \DChains{p}(\X) \arrow[d, swap, "\pi_p"] \arrow[r,"\Der"] & \DChains{p-1}(\X) \arrow[d, "\pi_{p-1}"]  \\
      \QDChains{p}(\X) \arrow[r, swap, "\Der"] & \QDChains{p-1}(\X)
    \end{tikzcd}
  \end{center}
  
  where $\pi_k$ is the natural projection from $\DChains{k}(\X)$ onto its quotient $\QDChains{k}(\X)$.
  Moreover,
  the operator $\Der : \QDChains{p}(\X) \to \QDChains{p-1}(\X)$ again satisfies the boundary property $\Der \circ \Der = 0$.
\end{article} %% Degenerate-cubes-and-chains

\begin{proof}
  Let us begin with the case $p = 1$.
  Let $\sigma \in \DCubes{1}^\bullet(\X)$,
  for all $t \in \RR$,
  $\sigma(t) = \sigma(0) = \sigma(1)$,
  since the reduction can be only $\hat 0 : t \mapsto 0$.
  Now,
  $\Der \sigma = \sigma(1) - \sigma(0) = 0$,
  that is,
  $\Der \sigma \in \DChains{0}^\bullet(\X) = \{0\}$.
  Now,
  let us consider the case $p >1$,
  and let $\sigma$ be a degenerate $p$-cube,
  that is,
  $\sigma = \sigma' \circ \pr$ with $\pr(t_1,\ldots,t_p) = (t_{i_1},\ldots,t_{i_q})$,
  $\{i_1,\ldots,i_q\} \subset \{1, \ldots , p\}$,
  $0 \leq q < p$,
  and $i_1 < \cdots <i_q$.
  Applying the definition \art{Boundary-of-cubes-and-chains} $(\diamondsuit)$,
  we have
  \begin{eqnarray*}
    \Der\sigma & = &  \sum_{k = 1}^p (-1)^k[\sigma \circ j_k(0) - \sigma \circ j_k(1)] \\
    & = & \sum_{k = 1}^p (-1)^k[\sigma' \circ \pr \circ j_k(0) - \sigma' \circ \pr \circ j_k(1)].
  \end{eqnarray*}
  Let us consider the term $[\sigma' \circ \pr \circ j_k(0) - \sigma' \circ \pr \circ j_k(1)]$,
  for $1 \leq k \leq p$. There are two possibilities:
  
  (a) The index $k$ belongs to the complement of $\{i_1, \ldots, i_q\}$ in $\{1, \ldots, p\}$.
  Thus,
  $\pr \circ j_k(0)(t_1, \ldots, t_{p-1}) = \pr \circ j_k(1)(t_1,\ldots,t_{p-1})$ and the term $[\sigma' \circ (\pr \circ j_k(0)) - \sigma' \circ (\pr \circ j_k(1))]$ disappears from the sum.
  
  (b) The index $k$ belongs to $\{i_1,\ldots,i_q\}$.
  Let us assume that $q>0$.
  Now,
  $\sigma' \circ \pr \circ j_k(a)(t_1,\ldots,t_{p-1}) = \sigma'\circ \pr (\ldots,a,\ldots)$,
  where the dots represent the $(p-1)$ numbers $t_1,\ldots,t_{p-1}$,
  distributed around $a$,
  and $a = 0$ or $1$.
  Thus,
  $\sigma' \circ \pr \circ j_k(a)(t_1,\ldots,t_{p-1}) = \sigma'(\ldots,a,\ldots)$,
  where the dots represent $q-1$ of the numbers $\{t_{i_1},\ldots,t_{i_q}\} \subset \{t_1,\ldots,t_p\}$,
  distributed around $a$.
  Thus,
  $\sigma' \circ \pr \circ j_k(a)$ depends only on $q-1$ variables,
  and,
  as a $(p-1)$-cube,
  is degenerate.
  If $q = 0$, then $\sigma'$ is a $0$-cube and $\Der \sigma = 0$.
  
  Therefore,
  as a linear combination of degenerate $(p-1)$-cubes,
  $\Der \sigma$ is a degenerate cubic $(p-1)$-chain.
  Now,
  let us define $\Der$ on $\QDChains{p}(\X)$ naturally by $\Der \bar c = \pi_{p-1}(\Der c)$,
  with $\bar c \in \QDChains{p}(\X)$ and $\pi_p(c) = \bar c$.
  Since $\Der[\DChains{p}^\bullet(\X)] \subset \DChains{p-1}^\bullet(\X)$,
  $\Der \bar c$ does not depend on the choice of the representative $c$ and $\Der$ is well defined.
  The fact that $\Der$ on $\QDChains{p}(\X)$ is a boundary operator is then obvious.
\end{proof}

\begin{article}\artlabel{Cubic homology}
  \addcontentsline{toc}{section}{\small\hspace{10pt} Cubic homology}
  \label{Cubic-homology}
  Let $\X$ be a diffeological space.
  As usual in homology theory \cite{McL75},
  when we have a chain complex,
  here $\QDChains{\star}(\X)$ with $\star = 0,1, \ldots \infty$,
  and the boundary operator
  $$%
    \Der: \QDChains{\star}(\X) \to \QDChains{\star-1}(\X), \text{ with } \Der \circ \Der = 0,
    $$%
  the space of {\em $p$-cycles}\index{Cubic cycle} is defined as the kernel in $\QDChains{p}(\X)$ of the operator $\Der$,
  and the space of {\em $p$-boundary}\index{Cubic boundary} as the image,
  in $\QDChains{p}(\X)$,
  of the operator $\Der$ defined on $\QDChains{p+1}(\X)$.
  These spaces will be denoted by
  $$%
    \left\{
    \begin{array}{rcll}
    \QDZ_p(\X) & = &\ker[\Der: \QDChains{p}(\X) \to \QDChains{p-1}(\X)] & \text{with $p \geq 1$}, \\
    \QDB_p(\X) & = & \Der (\QDChains{p+1}(\X))\subset \QDZ_p(\X) \subset \QDChains{p}(\X) & \text{with $p \geq 0$.}
    \end{array}
    \right.
    $$%
  Then,
  the homology groups are defined as the quotients of the spaces of cycles by the spaces of boundaries
  $$%
    \QDH_p(\X) = \QDZ_p(\X)/\QDB_p(\X).
    $$%
  Let us recall that for $p = 0$,
  $\Der: \QDChains{0}(\X) \to \{0\}$,
  thus $\QDZ_0(\X) = \QDChains{0}(\X)$,
  and in this case $\QDH_0(\X) =  \QDChains{0}(\X)/\Der \QDChains{1}(\X) = \DChains{0}(\X)/\Der \QDChains{1}(\X)$.
  We call this homology $\QDH_*(\X)$,
  the {\em cubic homology}\index{Cubic homology}%
  \footnote{In topology the cubic homology and the singular homology coincide \cite[Ex. 8.1]{HoYo61}.
  For a natural singular homology in diffeology,
  these two homologies will coincide also.}
  of the space $\X$.
  
  \Note~It is not clear if cubic (or singular) homology will play,
  in diffeology,
  the crucial role it plays in the theory of manifolds,
  because as we know elsewhere,
  contrary to the case of manifolds,
  all cohomologies do not coincide in diffeology;
  see \cite{Igl87b} and \cite{PIZ21b}.
  But,
  since it is a traditional tool,
  closely associated with De Rham calculus,
  and since it is still a smooth invariant,
  it was worth extending it to the general case.
\end{article} %% Cubic-homology

\begin{article}\artlabel{Interpreting $\QDH_0$}
  \addcontentsline{toc}{section}{\small\hspace{10pt} Interpreting $\QDH_0$}
  \label{Interpreting-H-0}
  Let $\X$ be a diffeological space.
  The group $\QDH_0(\X)$ is the free Abelian group generated by the connected components $\pi_0(\X)$ \art{Connected-components} of $\X$,
  $$%
    \QDH_0(\X) = \Maps(\pi_0(\X),\ZZ).
    $$%
  In particular if $\X$ is connected \art{Pathwise-connectedness},
  then $\QDH_0(\X) = \ZZ$.
\end{article} %% Interpreting-H-0

\begin{proof}
  On $\QDChains{0}(\X)$,
  the boundary operator is the zero homomorphism,
  $\Der: \QDChains{0}(\X) \to \{0\}$.
  Then,
  $\QDH_0(\X) = \QDChains{0}(\X) / \Der(\QDChains{1}(\X))$.
  Let us consider the homomorphism $\nu: \QDChains{0}(\X) \to \ZZ$ defined by $\nu(\sum_x n_x x) = \sum_x n_x$.
  On the one hand, $\Der(\QDChains{1}(\X))\subset\ker(\nu)$.
  On the other hand,
  $\sum_x n_x = 0$ means that,
  by decomposing $\sum_x n_x$ as an indexed sum of $+1$ and $-1$,
  there are as many terms associated with $+1$ as terms associated with $-1$.
  
  1) {\em If $\X$ is connected}.
  For any pair $(x,x') \in \X \times \X$ there exists a $1$-cube connecting $x$ to $x'$,
  that is,
  a path $\gamma$ such that $\gamma(0) = x$ and $\gamma(1) = x'$.
  So,
  for any pair of $(+1,x)$ and $(-1,x')$ appearing in the sum,
  we can find a $1$-cube connecting  $x$ to $x'$.
  In other words,
  if $\sum_x n_x = 0$ there exists a $1$-chain $c$ such that $\Der c = \sum_x n_x$.
  Hence,
  $\ker(\nu) \subset \Der(\QDChains{1}(\X))$ and $\Der(\QDChains{1}(\X)) = \ker(\nu)$.
  Therefore,
  $\QDH_0(\X) = \QDChains{0}(\X)/ \ker(\nu) = \nu(\QDChains{0}(\X)) = \ZZ$.
  
  2) {\em If $\X$ is not connected}.
  First of all,
  we decompose every chain $c$ into a sum of chains $c = \sum_{a \in \A} c_a$,
  where $\A$ is a finite set of components of $\X$ and $c_a$ takes its values in $a$.
  Then,
  we define a set of homomorphisms $\nu_a$,
  for each index $a$,
  as above,
  such that $c \in \Der(\QDChains{1}(\X))$ if and only if each $c_a$ belongs to the kernel of $\nu_a$.
  Therefore,
  $\QDH_0(\X)$ is the free Abelian group generated by the connected components.
\end{proof}

\begin{article}\artlabel{Cubic cohomology}
  \addcontentsline{toc}{section}{\small\hspace{10pt} Cubic cohomology}
  \label{Cubic-cohomology}
  Let $\X$ be a diffeological space.
  Let $\DChains{p}(\X)$ be the space of cubic $p$-chains of $\X$ \art{Cubes-and-cubic-chains-on-diffeological-spaces}.
  A cubic $p$-cochain of $\X$,
  with coefficients in $\RR$,
  is any homomorphism from $\DChains{p}(\X)$ to $\RR$.
  But $\DChains{p}(\X)$ is the free Abelian group generated by $\DCubes{p}(\X)$,
  thus every cubic $p$-cochain is defined by its values on the set of generators $\DCubes{p}(\X)$.
  Since $\DCubes{p}(\X) = \Cinfty(\RR^p,\X)$ is naturally a diffeological space,
  we shall define the {\em smooth cubic $p$-cochains} as the homomorphisms from  $\DChains{p}(\X)$ to $\RR$ generated by $\Cinfty(\DCubes{p}(\X),\RR)$.
  Now,
  since we are only interested in smooth cubic $p$-cochains,
  we shall omit the adjective ``smooth''.
  So,
  a {\em cubic $p$-cochain of $\X$}\index{Cubic cochain} is a linear map $f: \DChains{p}(\X) \to \RR$ such that
  $$%
    f: \sum_{\sigma} n_\sigma \sigma \mapsto \sum_{\sigma} n_\sigma f(\sigma), \text{ and } f \restriction \DCubes{p}(\X) \in \Cinfty(\DCubes{p}(\X),\RR),
    $$%
  with the notations of \art{Cubes-and-cubic-chains-on-diffeological-spaces}.
  We shall denote the spaces of cubic $p$-cochains by $\DCochains{p}(\X)$,
  that is,
  $$%
    \DCochains{p}(\X) = \DHom(\DChains{p}(\X),\RR) \simeq  \Cinfty(\DCubes{p}(\X),\RR).
    $$%
  Note that,
  as spaces of smooth functions in $\RR$,
  the spaces of cubic $p$-cochains $\DCochains{p}(\X)$ are naturally real diffeological vector spaces \art{Diffeological-vector-spaces}.
  Now,
  the boundary $\Der$ defined from $\DChains{p+1}(\X)$ to $\DChains{p}(\X)$ \art{Cubic-homology} induces,
  by duality,
  a {\em coboundary operator} $d$ such that
  $$%
    d : \DCochains{p}(\X) \to  \DCochains{p+1}(\X), \text{ with } df(c) = f(\Der c),
    $$%
  for all $f \in \DCochains{p}(\X)$,
  and all $c \in \DChains{p+1}(\X)$.
  But the value of the cubic $(p+1)$-cochain $df$ is characterized by its values on the $(p+1)$-cubes,
  by definition of the operator $\Der\sigma$ \art{Cubic-homology} applied on cubes,
  $$%
    df(\sigma) =  \sum_{k = 1}^{p+1} (-1)^k[f(\sigma \circ j_k(0)) - f(\sigma \circ j_k(1))] %% PIZ I have fixed here a misprint about \sum_{k = 1}^{p+1}... and not \sum_{k = 1}^p
    $$%
  for all $\sigma \in \DCubes{p+1}(\X)$.
  The injections $j_k$ have been defined above \art{Cubic-homology}.
  Then,
  by transfer of property,
  the coboundary $d$ satisfies
  $$%
    d \in \Hom(\DCochains{p}(\X),\DCochains{p+1}(\X)), \text{ and } d \circ d = 0.
    $$%
  Now,
  a {\em reduced cubic $p$-cochain} is a cubic $p$-cochain $f$ which vanishes on the degenerate $p$-cubes \art{Cubes-and-cubic-chains-on-diffeological-spaces},
  or equivalently,
  which is defined on the set of reduced cubic $p$-chains $\QDChains{p}(\X)$.
  Let us denote the space of reduced cubic $p$-cochains by $\QDCochains{p}(\X)$,
  $$%
    \QDCochains{p}(\X) = \{ f \in \DCochains{p}(\X) \mid f \restriction \Cub_p^\bullet(\X) = 0 \}.
    $$%
  Since $\Der [\DChains{p+1}^\bullet(\X)] \subset \DChains{p}^\bullet(\X)$,
  if $f \in \QDCochains{p}(\X)$,
  then $df \in \QDCochains{p+1}(\X)$.
  Thus,
  the coboundary of a reduced cubic cochain is a reduced cubic cochain,
  and we can define,
  by duality with the cubic homology,
  a {\em cubic cohomology}\index{Cubic cohomology},
  with coefficients in $\RR$.
  The spaces of {\em cocycles} and {\em coboundaries} of this cohomology are defined,
  and denoted,
  by
  $$%
    \left\{
    \begin{array}{lcll}
    {\QDZ}^p(\X) & = & \ker[d: \QDCochains{p}(\X) \to \QDCochains{p+1}(\X)] & \text{for $p\geq 0$}, \\
    {\QDB}^p(\X) & = & d (\QDCochains{p-1}(\X))\subset {\QDZ}^p(\X) \subset \QDCochains{p}(\X)  & \text{for $p \geq 1$}.
    \end{array}
    \right.
    $$%
  The elements of ${\QDZ}^p(\X)$ are called {\em cubic $p$-cocycles of $\X$}\index{Cubic cocycle} and the elements of ${\QDB}^p(\X)$ are called {\em cubic $p$-coboundaries of $\X$}\index{Cubic coboundary},
  we omit the word ``reduced''.
  The {\em cubic cohomology spaces}\index{Cubic cohomology} are defined as the quotient of the spaces of cocycles by the spaces of coboundaries,
  $$%
    \left\{
    \begin{array}{lcll}
    \QDH^p(\X) & = & {\QDZ}^p(\X)/{\QDB}^p(\X) & \text{for $p>0$}, \\
    \QDH^0(\X) & = &{\QDZ}^0(\X) & \text{for $p = 0$}.
    \end{array}
    \right.
    $$%
  The Abelian group $\QDH^p(\X)$ is called the {\em $p$-th space of cubic cohomology of $\X$,
  with coefficients in $\RR$}.
  
  \Note~We may be interested in a cubic homology with coefficients in a group other than $\RR$,
  for example a cohomology with coefficients in some Abelian diffeological group $\A$ (an irrational torus for example).
  In this case the morphisms are generated by the maps from the space of $p$-cubes to $\A$,
  and the cohomology groups are denoted by $\QDH^p(\X,\A)$,
  and the spaces $\QDH^p(\X)$ also are denoted by $\QDH^p(\X,\RR)$.
  We use {\em $\A$-cubic cochains}\index{Cubic cochain} or {\em real cubic cochain} if we want to specify.
\end{article} %% Cubic-cohomology

\begin{article}\artlabel{Interpreting $\QDH^0(\X,\RR)$}%
  \addcontentsline{toc}{section}{\small\hspace{10pt} Interpreting $\QDH^0(\X,\RR)$}%
  \label{Interpreting-H-0-X-R}
  Let $\X$ be a diffeological space,
  and let us consider the group $\QDH^0(\X,\RR)$.
  By definition \art{Cubic-cohomology} this is the set of $f \in \Cinfty(\DCubes{0}(\X),\RR)$ such that $df = 0$ (there are no degenerate $0$-cubes).
  But $\DCubes{0}(\X) = \X$ \art{Cubes-and-cubic-chains-on-diffeological-spaces}.
  Hence $f \in \Cinfty(\X,\RR)$,
  and if $f$ is a cocycle,
  that is,
  if $df = 0$,
  then $f(\Der \gamma) = f(\gamma(1))-f(\gamma(0)) = 0$,
  for all $\gamma \in \DCubes{1}(\X) = \Paths(\X)$.
  But this just means that $f$ is constant on the connected components of $\X$ \art{Pathwise-connectedness}.
  Hence, $\QDH^0(\X,\RR)$ is the group of real maps defined on the space $\pi_0(\X)$ \art{Connected-components},
  $$%
    \QDH^0(\X,\RR) = \Maps(\pi_0(\X),\RR).
    $$%
  \Note~This is also the group $\HDR^0(\X)$ of De Rham cohomology \art{The-functions-whose-differential-is-zero},
  and also the group of homomorphisms from $\QDH_0(\X)$ to $\RR$,
  $$%
    \QDH^0(\X,\RR) = \Hom(\QDH_0(\X),\RR) = \HDR^0(\X).
    $$%
\end{article} %% Interpreting-H-0-X-R

%%%%%%%%%%%%%%%%%%%%%%%%%%%%%%%%%%%%%%%%%%%%%%%%%%%%%%%%%%
%
%   Exercises
%
%%%%%%%%%%%%%%%%%%%%%%%%%%%%%%%%%%%%%%%%%%%%%%%%%%%%%%%%%%

\Exercises

\begin{exercise}[The boundary of a $3$-cube]
  \label{The-boundary-of-a-3-cube}
  For a $3$-cube,
  $\sigma \in \DCubes{3}(\X)$,
  check explicitly that $\Der \Der \sigma = 0$.
\end{exercise} %% The-boundary-of-a-3-cube

\begin{exercise}[Cubic homology of a point]
  \label{Cubic-homology-of-a-point}
  Let $\H_*(\X)$ be the homology,
  defined by the boundary operator $\Der$ on the space of $p$-chains,
  before the reduction by the degenerate chains \art{Cubic-homology}.
  Let us call it the {\em nonreduced cubic homology} of $\X$.
  Let $\star = \{0\}$ be the singleton.
  Show that for the nonreduced homology $\H_p(\star)$ is equal to $\ZZ$ for all $p$.
  And check that $\QDH_0(\star) = \ZZ$ and $\QDH_p(\star) = 0$ for all $p > 0$,
  for the reduced homology.
\end{exercise} %% Cubic-homology-of-a-point

%%%%%%%%%%%%%%%%%%%%%%%%%%%%%%%%%%%%%%%%%%%%%%%%%%%%%%%%%%
%% MARK: Integration of Differential Forms
%%%%%%%%%%%%%%%%%%%%%%%%%%%%%%%%%%%%%%%%%%%%%%%%%%%%%%%%%%

\section*{Integration of Differential Forms}
\label{Section-Integration-of-differential-forms}

\begin{sechead}
  This section describes the integration of forms on chains,
  and the essential properties related to this construction.
  I chose,
  for the sake of simplicity,
  to
  integrate differential forms on cubic chains,
  which suggested the cubic homology \art{Cubic-homology},
  because it is closely related to the computation of multiple integrals \art{Integration-of-a-smooth-p-form-on-a-p-cube}.
  This is a simple way to introduce such further important formulas as,
  for example,
  the Cartan formula relating the Lie derivative to contraction and the exterior derivative of differential forms.
\end{sechead}

\begin{article}\artlabel{Integrating forms on chains}
  \addcontentsline{toc}{section}{\small\hspace{10pt} Integrating forms on chains}
  \label{Integrating-forms-on-chains}
  Let us consider the real vector space $\RR^p$,
  oriented by its canonical basis $\cB = (\ee_1,\ldots,\ee_p)$,
  that is,
  oriented by the canonical volume $\vol_p$ associated with $\cB$ \art{Volumes-and-determinants} \art{The-differential-of-a-function-as-pullback},
  $$%
    \vol_p = \ee^1\wedge\cdots\wedge \ee^p = d\xx^1\wedge\cdots\wedge d\xx^p.
    $$%
  Every smooth $p$-form $\omega$,
  on a real domain $\U \!\subset\! \RR^p$,
  is proportional to $\vol_p$ \art{Volumes-and-determinants}.
  For every $\omega \in \Cinfty(\U,\Form^p(\RR^p))$,
  there exists a unique $f \in \Cinfty(\RR^p,\RR)$,
  such that
  $$%
    \omega = f\times \vol_p\,, \text{ and } f(x) = \omega(x)(\ee_1,\ldots,\ee_p).
    $$%
  Let $\alpha$ be a $p$-form on a diffeological space $\X$.
  And let $\sigma \in \DCubes{p}(\X)$ be a smooth $p$-cube \art{Cubes-and-cubic-chains-on-diffeological-spaces}.
  The {\em integral}\index{Integral} of the $p$-form $\alpha$ on the $p$-cube $\sigma$ is defined by
  \begin{equation}
    \renewcommand{\theequation}{$\diamondsuit$}
    \int_\sigma \alpha = \int_{\cub{p}} \alpha(\sigma),
  \end{equation}
  where $\I^p = \closedinterval{0,1}^p$,
  and the integration of a smooth $p$-form on a $p$-cube has been defined in \art{Integration-of-a-smooth-p-form-on-a-p-cube}.
  Since $\alpha(\sigma)$ is a smooth $p$-form of $\RR^p$,
  there exists a smooth function $f_\sigma$ such that $\alpha(\sigma) = f_\sigma \times \vol_p$. Hence, the
  integral of $\alpha$ over $\sigma$ writes also
  $$%
    \int_\sigma \alpha = \int_{\cub{p}} f_\sigma \times d\xx^1\wedge\cdots\wedge d\xx^p, \text{ with } \alpha(\sigma) = f_\sigma \times \vol,,
    $$%
  or,
  in terms of multiple integrals,
  $$%
    \int_\sigma \alpha =  \int_0^1 dx_1 \cdots\int_0^1 dx_p \ f_\sigma(x)\,, \text{ with } x = (x_1,\ldots,x_p),
    $$%
  where
  $$%
    f_\sigma(x) =  \alpha(\sigma)(x)(\ee_1,\ldots,\ee_p).
    $$%
  Note that for the space $\RR^p$,
  having been oriented once and for all,
  there is no ambiguity on the value,
  or the sign,
  of the function $f_\sigma$.
  The integral of $p$-forms on cubic $p$-chains is defined by linear extension of the integral of $p$-forms on  $p$-cubes,
  $$%
    \int_{c} \alpha = \sum_\sigma n_\sigma \int_\sigma \alpha\,, \text{ with } c = \sum_\sigma n_\sigma\, \sigma.
    $$%
  The way the sum is written,
  it appears to be infinite but it is actually performed only on the support of $c$ which is a finite set of cubes.
  
  \Note~For $p = 0$, a $0$-form on $\X$ is any smooth function $f \in \Cinfty(\X,\RR)$,
  and a $0$-chain is any finite sum $c = \sum_x n_x\, x$.
  The integral of $f$ on $c$ is then defined as the sum:
  $$%
    \int_c f = \sum_x n_x \, f(x)\,, \text{ for all } c = \sum_x n_x\, x.
    $$%
\end{article} %% Integrating-forms-on-chains

\begin{article}\artlabel{Pairing chains and forms}
  \addcontentsline{toc}{section}{\small\hspace{10pt} Pairing chains and forms}
  \label{Pairing-chains-and-forms}
  The map defined above \art{Integrating-forms-on-chains} by integrating differential $p$-forms on cubic $p$-chains is known as the {\em pairing operation},
  $$%
    (c,\alpha) \mapsto \int_c\alpha\,, \text{ for all } (c,\alpha) \in \DChains{p}(\X) \times
    \Omega^p(\X).
    $$%
  This pairing of forms and chains satisfies, in particular, the
  following properties:
  
  1. The pairing is a {\em bilinear} operation,
  $$%
    \int_{nc+n'c'}(s\alpha +s'\alpha') = ns\int_c\alpha + ns'\int_c\alpha' + n's\int_{c'}\alpha + n's'\int_{c'}\alpha'.
    $$%
  
  2. On cubes,
  the pairing is smooth,
  $$%
    \bigg[(\sigma,\alpha) \mapsto \int_\sigma\alpha\bigg] \in \Cinfty(\DCubes{p}(\X) \times \Omega^p(\X),\RR),
    $$%
  where $\DCubes{p}(\X)$ is equipped with its functional diffeology of space of smooth maps from $\RR^p$ to $\X$ \art{Cubes-and-cubic-chains-on-diffeological-spaces},
  and $\Omega^p(\X)$ is equipped with its functional diffeology of space of forms \art{Differential-forms-on-diffeological-spaces}.
  
  3. A $p$-form $\alpha$ is zero if and only if its integral,
  on every smooth $p$-cube,
  vanishes.
  Formally,
  $$%
    \alpha = 0 \ \Leftrightarrow \ \int_\sigma \alpha = 0\,, \text{ for all } \sigma \in \DCubes{p}(\X).
    $$%
  Equivalently,
  two $p$-forms coincide if and only if their integrals,
  on every smooth $p$-cube, coincide.
\end{article} %% Pairing-chains-and-forms

\begin{proof}
  1. The bilinearity of the pairing is a direct consequence of the definitions of sums of chains and sums of forms.
  
  2. Let $r \mapsto (\sigma_r,\alpha_r)$ be a plot of $\Cinfty(\DCubes{p}(\X) \times \Omega^p(\X))$.
  By definition,
  $$%
    \int_{\sigma_r} \alpha_r = \int_{\cub{p}} \alpha_r(\sigma_r).
    $$%
  On the one hand $r \mapsto \alpha_r$ is a plot of $\Omega^p(\X)$,
  which means that for every plot $\P$ of $\X$ the map $(r,r') \mapsto  \alpha_r(\P)(r')$ is smooth \art{Differential-forms-on-diffeological-spaces}.
  On the other hand,
  $r \mapsto \sigma_r$ is a plot of $\DCubes{p}(\X)$,
  which means that $(r,t) \mapsto \sigma_r(t)$ is a plot of $\X$.
  Combined,
  we get that for all $v_1, \ldots, v_p \in \RR^p$ the map
  $$%
    (r,t) \mapsto \alpha_r \bigg( \vect{r' \\ s} \mapsto \sigma_{r'}(s)\bigg)_{r'=r \choose s=t} \vect{0 \\ v_1} \cdots \vect{0 \\ v_p} = \alpha_r(\sigma_r)_t (v_1) \cdots (v_p)
    $$%
  is smooth.
  Thus,
  $(r,t) \mapsto \alpha_r(\sigma_r)_t$ is smooth,
  therefore there exists a smooth function $(r,t) \mapsto f_r(t)$ such that $\alpha_r(\sigma_r)_t= f_r(t) \times \vol_p$.
  Hence,
  $$%
    \int_{\sigma_r} \alpha_r =  \int_{\cub{p}} f_r \times \vol_p,
    $$%
  and since the integration over the cube $\cub{p}$ is a smooth operation,
  the function $r \mapsto \int_{\sigma_r} \alpha_r$ is smooth.
  
  3. Let us assume that $\int_\sigma\alpha = 0$ for all $p$-cubes $\sigma$,
  and that $\alpha \neq 0$.
  Then,
  there exists a $p$-plot ${\P}: \U \to \X$ such that $\alpha({\P})\neq 0$,
  that is,
  there exists $r \in \U$ such that $\alpha({\P})(r)\neq 0$.
  But $\alpha$ is a $p$-form on a $p$-domain,
  thus there exists $f \in \Cinfty(\U,\RR)$ such that $\alpha(\P) = f\times \vol$ \art{Volumes-and-determinants},
  and $\alpha({\P})(r)\neq 0$ means that $f(r)\neq 0$.
  Let us assume that $f(r)>0$ (it would be equivalent to assume that $f(r)<0$).
  Since $f$ is smooth,
  there exists a small cube ${\C}$,
  centered at the point $r$,
  such that $f(r')>0$ for all $r' \in {\C}$.
  Since $f\restriction {\C}$ is positive,
  the integral of $f$ over the cube ${\C}$ is positive:
  $$%
    \int_{\C} f \times \vol_p > 0.
    $$%
  But there exists a
  positive diffeomorphism $\varphi$ from $\RR^p$ onto an open neighborhood of $r$,
  mapping the standard cube $\cub{p}$ to $\C$.
  Then, $\sigma = \P \circ \varphi$ is a smooth $p$-cube of $\X$ and
  $$%
    \int_\sigma \alpha = \int_{\cub{p}} \alpha(\P \circ \varphi) = \int_{\cub{p}} \varphi^*(\alpha(\P)) = \int_{\cub{p}} \varphi^*(f \times \vol_p).
    $$%
  But since $\varphi$ is a positive diffeomorphism,
  by a change of coordinates we get
  $$%
    \int_{\cub{p}} \varphi^*(f \times \vol_p) = \int_{\cub{p}} f \circ \varphi \times \det(\D(\varphi)) \times \vol_p = \int_\C f \times \vol_p.
    $$%
  Hence,
  $\int_\sigma \alpha >0$ and we have a smooth $p$-cube $\sigma$ on which the integral of $\alpha$ does not vanish.
  This is in contradiction with the hypothesis.
  Therefore,
  for each plot $\P$ of $\X$,
  $\P^*(\alpha) = 0$,
  that is,
  $\alpha = 0$.
\end{proof}

\begin{article}\artlabel{Pulling back and forth forms and chains}
  \addcontentsline{toc}{section}{\small\hspace{10pt} Pulling back and forth forms and chains}
  \label{Pulling-back-and-forth-forms-and-chains}
  Let $\X$ and $\X'$ be two diffeological spaces and $f \in \Cinfty(\X,\X')$.
  For all $p$-cubes $\sigma \in \Cub_p(\X)$,
  the {\em pushforward}\index{Pushforward} of $\sigma$ by $f$ is denoted and defined by
  $$%
    f_* : \Cub_p(\X) \to \Cub_p(\X')\,, \text{ with } f_*(\sigma) = f \circ \sigma\,, \text{ for all } \sigma \in \DCubes{p}(\X).
    $$%
  The pushforward of cubic $p$-chains by $f$ is defined by linear extension from the pushforward of $p$-cubes,
  briefly written as
  $$%
    f_*\bigg(\sum_\sigma n_\sigma \, \sigma\bigg) = \sum_{\sigma} n_\sigma \, f_*(\sigma) = \sum_{\sigma'} n_{\sigma'}\, \sigma'\,, \text{ with } n_{\sigma'} = \sum_{\sigma \atop f_*(\sigma) = \sigma'} n_\sigma.
    $$%
  More precisely,
  let $c = \sum_\sigma n_\sigma \, \sigma$,
  that is,
  $c = \sum_{\sigma \in \Supp(c)} n_\sigma \sigma$.
  For $\sigma$ and $\tau$ in $\Supp(c)$,
  let $\sigma \sim \tau$ if $f_*(\sigma) = f_*(\tau)$.
  Let $\cI = \quotient{\Supp(c)}{\sim}$,
  then choose,
  for each $i \in \cI$,
  one $\sigma_i \in i$,
  and let $\sigma'_i = f_*(\sigma_i)$.
  Therefore,
  $f_*(c) = \sum_{i \in \cI} n'_i \, \sigma'_i$,
  with
  $n'_i = \sum_{\sigma \in i} n_\sigma$.
  The support of $f_*(c)$ is the subset of these $\sigma'_i$,
  such that $n'_i \neq 0$, which is of course finite.
  All that being specified,
  we have now,
  for all $\alpha' \in \Omega^p(\X')$,
  $$%
    \int_{f_*(c)} \alpha' = \int_c f^*(\alpha').
    $$%
  
  \Note~Since for every $p$-cube $\sigma = {\id_p}_*(\sigma)$,
  where $\id_p : \RR^p \to \RR^p$ is the identity,
  we have an equivalent writing for the integral of a $p$-form on a $p$-cube,
  $$%
    \int_\sigma \alpha = \int_{\id_p} \sigma^*(\alpha).
    $$%
  This formulation is sometimes useful,
  to avoid ambiguity,
  in particular in the section about variation of integral of forms on chains.
\end{article} %% Pulling-back-and-forth-forms-and-chains

\begin{proof}
  By definition \art{Integrating-forms-on-chains},
  for every smooth $p$-cube $\sigma$,
  we have
  $$%
    \int_{f_*(\sigma)} \alpha' = \int_{\id_p} [f_*(\sigma)]^*(\alpha') = \int_{\id_p} (f \circ \sigma)^*(\alpha') = \int_{\id_p}\sigma^*(f^*(\alpha')) = \int_{\sigma}f^*(\alpha').
    $$%
  Now,
  let $c = \sum_\sigma n_\sigma \, \sigma$,
  and let $c' = f_*(c)$.
  On the one hand,
  we have
  $$%
    \int_{f_*(c)} \alpha' = \sum_{\sigma'} n_{\sigma'} \, \int_{\sigma'} \alpha' = \sum_{\sigma'} \bigg[ \sum_{\sigma \atop  f_*(\sigma) = \sigma'} n_\sigma \bigg] \int_{f_*(\sigma)} \alpha',
    $$%
  and on the other hand,
  $$%
    \int_c f^*(\alpha') = \sum_\sigma n_\sigma \int_\sigma f^*(\alpha') = \sum_\sigma n_\sigma \int_{f_*(\sigma)} \alpha' = \sum_{\sigma'} \bigg[ \sum_{\sigma \atop \sigma' = f_*(\sigma)} n_\sigma \bigg] \int_{f_*(\sigma)} \alpha'.
    $$%
  Therefore,
  $\int_{f_*(c)} \alpha' = \int_c f^*(\alpha')$.
\end{proof}

\begin{article}\artlabel{Changing the coordinates of a cube}
  \addcontentsline{toc}{section}{\small\hspace{10pt} Changing the coordinates of a cube}
  \label{Changing-the-coordinates-of-a-cube}
  Let $\X$ be a diffeological space.
  Let $\sigma \in \DCubes{p}(\X)$ and $\alpha \in \Omega^p(\X)$. Let $\varphi$ be a {\em positive diffeomorphism} of $\cub{p}$,
  that is,
  $\varphi \in \Diff(\RR^p)$, $\varphi(\cub{p}) = \cub{p}$,
  and for all $x \in \cub{p}$, $\det[\D(\varphi)(x)]>0$.
  Then,
  $$%
    \int_{\sigma_*(\varphi)} \alpha = \int_\sigma \alpha.
    $$%
  
  Note that in the notation $\sigma_*(\varphi)$,
  $\varphi$ is regarded as a smooth cube and $\sigma$ as a smooth map.
  Also note that $\varphi$ does not need to be defined on the whole $\RR^p$,
  but only on some open superset of the cube $\cub{p} \subset \RR^p$.
\end{article} %% Changing-the-coordinates-of-a-cube

\begin{proof}
  Let $f$ be such that $\alpha(\sigma) = f \times \vol_p$,
  then
  $$%
    \int_{\sigma_*(\varphi)}\alpha = \int_{\cub{p}}\alpha (\sigma \circ \varphi) = \int_{\cub{p}}\varphi^*(\alpha(\sigma)) = \int_{\cub{p}}\varphi^*(f \times \vol_p).
    $$%
  But $\varphi^*(f \times \vol_p) = (f \circ \varphi) \times \varphi^*(\vol_p)$,
  thus
  $$%
    \int_{\cub{p}}\varphi^*(f \times \vol_p) = \int_{\cub{p}}(f \circ \varphi) \times \varphi^*(\vol_p) = \int_{\cub{p}}(f \circ \varphi) \times \det(\D(\varphi)) \times \vol_p.
    $$%
  Next,
  since $\det(\D(\varphi))>0$,
  $$%
    \int_{\cub{p}}(f \circ \varphi) \times \det(\D(\varphi)) \times \vol_p = \int_{\cub{p}}(f \circ \varphi) \times \vert \det(\D(\varphi)) \vert \times \vol_p.
    $$%
  By application of the change of variables $x \mapsto \varphi(x)$ in a multiple integral,
  $$%
    \int_{\cub{p}}(f \circ \varphi) \times \vert \det(\D(\varphi)) \vert \times \vol_p = \int_{\varphi(\cub{p})}f \times \vol_p = \int_{\cub{p}}f \times \vol_p = \int_\sigma \alpha.
    $$%
  Therefore,
  $\int_{\sigma_*(\varphi)} \alpha = \int_\sigma \alpha$.
\end{proof}

%%%%%%%%%%%%%%%%%%%%%%%%%%%%%%%%%%%%%%%%%%%%%%%%%%%%%%%%%%
%% MARK: Variation of the Integrals of Forms on Chains
%%%%%%%%%%%%%%%%%%%%%%%%%%%%%%%%%%%%%%%%%%%%%%%%%%%%%%%%%%

\section*{Variation of the Integrals of Forms on Chains}
\label{Section-Variation-of-integrals}

\begin{sechead}
  In this section we shall establish some theorems relative to the variation of integrals of forms on chains.
  First of all,
  we give the diffeological version of the Stokes theorem.
  Then,
  we give a formula for any variation of the integral of a $p$-form on a $p$-chain.
  This formula mixes the form,
  its exterior derivative and the contractions with the variation of the chain.
  We deduce then the homotopic invariance of the De Rham cohomology \art{Homotopic-invariance-of-the-De-Rham-cohomology} and the diffeological version of the classical Cartan formula \art{The-Cartan-Lie-formula},
  relating the Lie derivative,
  contraction of forms and the exterior derivative.
\end{sechead}

\begin{article}\artlabel{The Stokes theorem}
  \addcontentsline{toc}{section}{\small\hspace{10pt} The Stokes theorem}
  \label{The-Stokes-theorem}
  Let $\X$ be a diffeological space,
  and let $\alpha$ be a $(p-1)$-form on $\X$ \art{Differential-forms-on-diffeological-spaces},
  with $p \geq 1$,
  let $c$ be a cubic $p$-chain in $\X$ \art{Cubes-and-cubic-chains-on-diffeological-spaces}.
  With the integration of differential forms on chains defined in \art{Integrating-forms-on-chains},
  we have
  $$%
    \int_c d\alpha = \int_{\partial c} \alpha.
    $$%
  This is the diffeological version of classical Stokes' theorem.
  
  \Note~A $p$-form $\alpha$ is zero if and only if its integral vanishes on every $p$-cube \art{Pairing-chains-and-forms}.
  As an immediate corollary of Stokes' theorem,
  $\alpha$ is closed if and only if its integral vanishes on every boundary $\partial c$,
  where $c$ is a $(p+1)$-cube:
  $$%
    d\alpha = 0 \text{ if and only if } \int_{\partial c} \alpha = 0.
    $$%
\end{article} %% The-Stokes-theorem

\begin{proof}
  Let us prove first the theorem for a $p$-cube $\sigma \in \DCubes{p}(\X)$.
  We have
  $$%
    \int_\sigma d\alpha = \int_{\cub{p}} d\alpha (\sigma) = \int_{\cub{p}} d[\alpha (\sigma)].
    $$%
  Let $a = \alpha(\sigma)$,
  $a$ is a smooth $(p-1)$-form of $\RR^p$,
  let
  \begin{equation}
    \renewcommand{\theequation}{$\spadesuit$}
    a = \sum_{k = 1}^p a_k \, \ee^1\wedge \cdots [\ee^k] \cdots \wedge \ee^p,
  \end{equation}
  where the brackets $[\ee^k]$ mean that $\ee^k$ is omitted.
  This is the general expression of a smooth $(p-1)$-form on $\RR^p$.
  The $a_k$ are smooth real functions defined on $\RR^p$.
  Now, the exterior derivative $da$ is given by
  \begin{eqnarray*}
    da(x) & = & \bigg\{{\Der a_1 \over \Der x_1} - \cdots + (-1)^{p-1}{\Der a_p \over \Der x_p} \bigg\}\, \ee^1 \wedge \cdots \wedge \ee^p \\
    & = & - \left\{ \sum_{k=1}^p(-1)^k {\Der a_k \over \Der x_k} \right\} \, \ee^1 \wedge \cdots \wedge \ee^p.
  \end{eqnarray*}
  Then,
  the integral of $d\alpha$ over $\sigma$ splits into
  $$%
    \int_\sigma d\alpha = \int_{\cub{p}} da = -\sum_{k = 1}^p (-1)^k \int_0^1dx_1\cdots \int_0^1 {\Der a_k \over \Der x_k} dx_x \cdots \int_0^1 dx_p.
    $$%
  Next,
  after integration by parts,
  we get
  %
  \renewcommand{\theequation}{$\heartsuit$}
  \begin{eqnarray}
    \int_\sigma d\alpha & = & - \sum_{k = 1}^p (-1)^k \int_0^1 dx_1 \cdots \bigg[a_k\bigg]_{x_k = 0}^{x_k = 1} \cdots \int_0^1 dx_p \nonumber \\
    & = &  + \sum_{k = 1}^p (-1)^k \int_0^1 dx_1 \cdots \bigg[a_k\bigg]_{x_k = 1}^{x_k = 0} \cdots \int_0^1 dx_p \nonumber  \\
    & = &  + \sum_{k = 1}^p (-1)^k \int_{\cub{p-1}} \bigg[a_k\bigg]_{x_k = 1}^{x_k = 0} \, \ee^1\wedge \cdots [\ee^k] \cdots \wedge \ee^p.
  \end{eqnarray}
  %
  Since we have the choice for the name of the coordinates in $\RR^{p-1}$,
  for each $k=1, \ldots, p$,
  we denote by $(x_1 \cdots [x_k] \cdots x_p)$ a current point in $\RR^{p-1}$,
  where the brackets $[x_k]$ mean that there is no coordinate with index $k$,
  and the $\ee_i$,
  in the last right-hand term above, are the vectors of the canonical basis of $\RR^{p-1}$ according to the new indexation.
  For example,
  $$%
    k = 1, \ p = 3, \ x \in \RR^2 \ : \ x = \vect{ x_2 \\ x_3}, \text{ and } \ee_1 = \vect{1 \\ 0} \ \ee_2 = \vect{0 \\ 1}.
    $$%
  Now, to compute the integral of $\alpha$ on the boundary $\Der \sigma$,
  with
  $$%
    \Der \sigma = \sum_{k=1}^p (-1)^k [\sigma \circ j_k(0) - \sigma \circ j_k(1)],
    $$%
  we need a workable expression of $j_k(t)^*(a)$.
  But as a $(p-1)$-form on $\RR^{p-1}$ $j_k(t)^*(a)$,
  it writes as:
  $$%
    j_k(t)^*(a)_{x} = c(x)\, \ee^1 \wedge \cdots [\ee^k] \cdots \wedge \ee^p,
    $$%
  with $c(x) = j_k(t)^*(a)_{x}(\ee_1) \cdots [\ee_k] \cdots (\ee_p)$.
  Thus,
  $$%
    c(x) = j_k(t)^*(a)_{x}(\ee_1) \cdots [\ee_k] \cdots (\ee_p) = a_{j_k(t)(x)} (\M \ee_1) \cdots [\M\ee_k] \cdots (\M\ee_p),
    $$%
  with  $\M = \D(j_k(t))(x)$.
  But
  $$%
    \D(j_k(t))(x)(\ee_i) = {\partial \over \partial s}\bigg\{ j_k(t)(x + s \ee_i) \bigg\}_{s=0} = j_k(0)(\ee_i) = \ee_i \in \RR^p.
    $$%
  Now,
  the $\ee_i$ are the vectors of the canonical basis of $\RR^p$.
  Thus,
  \begin{align*}
    c(x) & =  a_{j_k(t)(x)} (\M \ee_1) \cdots [\M\ee_k] \cdots (\M\ee_p) \\
    & =   a(x_k = t) (\ee_1) \cdots [\ee_k] \cdots (\ee_p) \\
    & =  \sum_{j=1}^p a_j (x_k = t) \ee^1 \wedge \cdots [\ee^j] \cdots \wedge \ee^p (\ee_1) \cdots [\ee_k] \cdots (\ee_p)\\
    & =   a_k(x_k = t).
  \end{align*}
  Hence,
  $j_k(t)^*(a)_{x} = \sum_{k=1}^p a_k(x_k=t) \, \ee^1 \cdots \wedge [\ee^k] \cdots \wedge \ee^p$,
  we have then
  %
  \renewcommand{\theequation}{$\diamondsuit$}
  \begin{eqnarray}
    \int_{\Der \sigma} \alpha & = & \sum_{k=1}^p (-1)^k \bigg[ \int_{\sigma \circ j_k(0)} \alpha - \int_{\sigma \circ j_k(1)} \alpha  \bigg] \nonumber \\
    & = & \sum_{k = 1}^p (-1)^k \int_{\cub{p-1}} j_k(0)^*(\alpha(\sigma)) - \int_{\cub{p-1}} j_k(1)^*(\alpha(\sigma)) \nonumber \\
    & = & \sum_{k = 1}^p (-1)^k \int_{\cub{p-1}} j_k(0)^*(a) - \int_{\cub{p-1}} j_k(1)^*(a) \nonumber \\
    & = & \sum_{k = 1}^p (-1)^k \int_{\cub{p-1}}a_k(x_k=0) \, \ee^1 \wedge \cdots [\ee^k] \cdots \wedge \ee^p \nonumber  \\
    &-& \sum_{k = 1}^p (-1)^k \int_{\cub{p-1}}a_k(x_k=1) \, \ee^1 \wedge \cdots [\ee^k] \cdots \wedge \ee^p \nonumber\\
    & = & \sum_{k = 1}^p (-1)^k \int_{\cub{p-1}} [a_k(x_k=0) - a_k(x_k=1)] \, \ee^1\wedge \cdots [\ee^k] \cdots \wedge \ee^p \nonumber \\
    & = & \sum_{k = 1}^p (-1)^k \int_{\cub{p-1}} \bigg[a_k\bigg]_{x_k = 1}^{x_k = 0} \, \ee^1\wedge \cdots [\ee^k] \cdots \wedge \ee^p.
  \end{eqnarray}
  Thus, $(\heartsuit) = (\diamondsuit)$, and the Stokes theorem is proved for the integral of forms on cubes.
  It extends by linearity on every chain:
  $$%
    \int_c d\alpha = \sum_\sigma
    n_\sigma \, \int_\sigma d\alpha =  \sum_\sigma n_\sigma \,
    \int_{\Der \sigma} \alpha = \int_{\Der c} \alpha\,,
    \text{ for all } c = \sum_\sigma
    n_\sigma \, \sigma.
    $$%
  Note that the covariant nature of diffeology reduces the Stokes theorem
  to the simplest case of smooth $(p-1)$-forms on standard $p$-cubes.
\end{proof}

\begin{article}\artlabel{Variation of the integral of a form on a cube}
  \addcontentsline{toc}{section}{\small\hspace{10pt} Variation of the integral of a form on a cube}
  \label{Variation-of-the-integral-of-a-form-on-a-cube}
  Let $\X$ be a diffeological space and $r \mapsto (\sigma_r,\alpha_r) \in \Cinfty(\U,\DCubes{p}(\X) \times \Omega^p(\X))$ be a plot of the product,
  defined on some real domain $\U$ of $\RR^m$,
  $m \in \NN$.
  Let us denote simply $\alpha$ for $\alpha_0$ and $\sigma$ for $\sigma_0$.
  Then,
  since the pairing of chains and forms is a smooth map \art{Pairing-chains-and-forms},
  we have
  $$%
    \left[ r \mapsto \int_{\sigma_r}\alpha_r\right] \in \Cinfty(\U,\RR).
    $$%
  The {\em variation of the integral} of $\alpha$ on $\sigma$,
  at some point $r \in \U$, applied to a vector $\delta r \in \RR^m$,
  is the number denoted and defined by
  $$%
    \delta \int_\sigma \alpha = {\partial \over \partial r} \left\{ \int_{\sigma_r}\alpha_r\right\}_{r} (\delta r).
    $$%
  The partial derivative $\Der/\Der r$ denotes the tangent linear map \art{Smooth-parametrizations-in-domains}.
  Let us give an equivalent formulation of the variation of the integral.
  Let us consider the following function,
  defined on a small real interval $]-\epsilon,\epsilon[$,
  with $\epsilon>0$,
  $$%
    s \mapsto \int_{\sigma_s} \alpha_s\,, \text{ where } \sigma_s = \sigma_{r+s\delta r} \text{ and } \alpha_s = \alpha_{r+s\delta r}.
    $$%
  Then,
  $$%
    \delta \int_\sigma \alpha = {\partial \over \partial s} \left\{ \int_{\sigma_s}\alpha_s\right\}_{s = 0}.
    $$%
  The variation of the integral involves only $1$-plot of cubes and forms.
  For this reason we shall continue only with {\em $1$-plot variations} of $\int_\sigma \alpha$.
  Now,
  for any arc of a $p$-cube $s \mapsto \sigma_s$ of $\X$ centered at $\sigma$ \art{Contraction-of-differential-forms},
  for any arc of a $p$-form $s \mapsto \alpha_s$ of $\X$ centered at $\alpha$,
  we have the following identity:
  \begin{equation}
    \renewcommand{\theequation}{$\diamondsuit$}
    \delta \int_\sigma \alpha = \int_{\id_p} d\alpha \rfloor \delta\sigma + \int_{\id_p} d[\alpha \rfloor \delta\sigma] + \int_{\id_p} \sigma^*(\delta \alpha).
  \end{equation}
  
  (a) $d\alpha$ is the exterior derivative of $\alpha$ defined in \art{Exterior-derivative-of-forms}.
  
  (b) $\delta \alpha$ is the $p$-form of $\X$ denoted,
  for every $n$-plot ${\P}$ of $\X$,
  by
  $$%
    \delta \alpha = {\P} \mapsto {\partial \over \partial s} \bigg\{\alpha_s({\P})\bigg\}_{s = 0},
    $$%
  and defined by
  $$%
    {\partial \over \partial s} \bigg\{\alpha_s({\P})\bigg\}_{s = 0}(r)(v_1) \cdots (v_p) = {\partial \over \partial s} \bigg\{\alpha_s({\P})(r)(v_1) \cdots (v_p)\bigg\}_{s = 0}\,,
    $$%
  for all $r \in \U$ and $v_1,\ldots, v_p \in \RR^n$.
  
  (c) $\alpha \rfloor \delta\sigma$ and $d\alpha \rfloor \delta\sigma$ are the contractions of the forms $\alpha$ and $d\alpha$ with the arc of plots\index{Arc of plots} $s \mapsto \sigma_s$ \art{Contraction-of-differential-forms}.
  
  \Note{1} Thanks to the Stokes theorem \art{The-Stokes-theorem},
  the variation of the integral $\alpha$ on the cube $\sigma$ writes also
  $$%
    \delta \int_\sigma \alpha = \int_{\id_p} d\alpha \rfloor \delta\sigma + \int_{\partial \id_p} \alpha \rfloor \delta\sigma + \int_{\id_p} \sigma^*(\delta \alpha).
    $$%
  
  \Note{2} The variation formula $(\diamondsuit)$ still applies {\em mutatis mutandis\/} for the variation $\delta\sigma_s$ \xart{Contraction-of-differential-forms}{Note 3},
  for any $s \in \openinterval{-\varepsilon,+\varepsilon}$.
\end{article} %% Variation-of-the-integral-of-a-form-on-a-cube

\begin{proof}
  Let us consider the decomposition of the pairing
  $$%
    s \mapsto \vect{s \\ s} \mapsto \int_{\sigma_s}\alpha_s, \text{ with } \vect{s \\ t} \mapsto \int_{\sigma_s}\alpha_t,
    $$%
  such that,
  $$%
    {\partial \over \partial s} \left\{ \int_{\sigma_s}\alpha_s\right\}_{s = 0} = {\Der \over \Der s} \left\{ \int_{\sigma_s}\alpha\right\}_{s = 0} + {\Der \over \Der t} \left\{ \int_{\sigma}\alpha_t\right\}_{t = 0}.
    $$%
  Let us use the variable $r \in \RR^p = \Dom(\sigma)$,
  and let $r = \sum_{k=1}^p r^k \ee_k$.
  The second term of the right-hand sum of this identity gives immediately,
  \begin{eqnarray*}
    {\Der \over \Der t} \left\{\int_{\sigma}\alpha_t\right\}_{t = 0} & = &  {\Der \over \Der t} \left\{\int_{\cub{p}}\alpha_t(\sigma) \right\}_{t = 0} \\
    & = & {\Der \over \Der t} \left\{\int_{\cub{p}} \alpha_t(\sigma)(r)(\ee_1)\cdots(\ee_p)\, dr^1\wedge\cdots \wedge dr^p \right\}_{t = 0} \\
    & = & \int_{\cub{p}} {\Der \over \Der t} \bigg\{\alpha_t(\sigma)(r)(\ee_1)\cdots(\ee_p)\bigg\}_{t = 0} dr^1\wedge\cdots \wedge dr^p \\
    & = & \int_{\cub{p}} (\delta\alpha)(\sigma)(r)(\ee_1)\cdots(\ee_p) dr^1\wedge\cdots \wedge dr^p \\
    & = & \int_\sigma \delta\alpha .
  \end{eqnarray*}
  Hence,
  the variation of the pairing splits into two parts,
  \begin{equation}
    \renewcommand{\theequation}{$\clubsuit$}
    \delta \int_\sigma \alpha = \delta \int_\sigma \alpha \bigg\vert_{\delta \alpha = 0} + \int_\sigma \delta \alpha.
  \end{equation}
  Let us focus on the first integral of the right-hand side of $(\clubsuit)$,
  which corresponds to $\delta\alpha = 0$,
  and let us define
  $$%
    \bsigma(s,r) = \sigma_s(r), \text{ and } j_s: r \mapsto (s,r).
    $$%
  Thus, $\bsigma$ is a $(p+1)$-plot defined on $]-\epsilon,\epsilon[ \times \cub{p}$ and $j_s$ is the injection from $\cub{p}$ to $]-\epsilon,\epsilon[ \times \cub{p}$ at the height $s$.
  Now,
  using $\sigma_s = \bsigma \circ j_s$,
  we have
  \begin{eqnarray*}
    \int_{\sigma_s}\alpha & = & \int_{\cub{p}} \alpha(\sigma_s)(r)(\ee_1) \cdots (\ee_p)\,dr^1\wedge\cdots \wedge dr^p \\
    & = & \int_{\cub{p}} \alpha(\bsigma \circ j_s)(r)(\ee_1)\cdots(\ee_p)\, dr^1\wedge\cdots \wedge dr^p \\
    & = & \int_{\cub{p}} j_s^*(\alpha(\bsigma))(r)(\ee_1)\cdots(\ee_p)\, dr^1\wedge\cdots \wedge dr^p \\ & = & \int_{\cub{p}} \alpha(\bsigma) \vect{s \\ r} \vect{0 \\ \ee_1} \cdots \vect{0 \\ \ee_p} \,dr^1\wedge\cdots \wedge dr^p.
  \end{eqnarray*}
  Note that $\alpha(\bsigma)$ is a $p$-form on the $(p+1)$-domain $\openinterval{-\epsilon,\epsilon} \times \RR^p$.
  Let us introduce then the $(p+1)$ coordinates $\aa_i = [(s,r) \mapsto a_i]$,
  with $i = 0, 1,\ldots, p$, in this way
  \begin{eqnarray*}
    \alpha(\bsigma)_{s \choose r} & = & a_0\,dr^1\wedge dr^2\wedge dr^3\wedge \cdots \wedge dr^p \\
    &+& a_1\,ds\wedge dr^2\wedge dr^3\wedge \cdots \wedge dr^p \\
    &+& a_2\,ds\wedge dr^1\wedge dr^3\wedge \cdots \wedge dr^p \\
    &+&\cdots \\ &+& a_p\, ds\wedge dr^1\wedge dr^2\wedge \cdots \wedge dr^{p-1},
  \end{eqnarray*}
  or,
  equivalently
  $$%
    \alpha(\bsigma)_{s \choose r} = \sum_{k = 0}^p a_k \, dr^0\wedge \cdots [dr^k] \cdots \wedge dr^p = \sum_{k = 0}^p a_k \, \ee^0\wedge \cdots [\ee^k] \cdots \wedge \ee^p,
    $$%
  where $r^0 = s$,
  and the brackets $[dr^k]$ mean that $dr^k$ is omitted.
  Thus,
  $$%
    \int_{\sigma_s}\alpha = \int_{\cub{p}} a_0 \,d r^1\wedge\cdots \wedge d r^p, \text{ with } a_0 = \alpha(\bsigma) \vect{s \\ r} \vect{0 \\ \ee_1}\cdots \vect{0 \\ \ee_p}.
    $$%
  Now,
  since everything is smooth,
  integration and derivation commute,
  and the derivative with respect to the variable $s$ becomes
  $$%
    {\Der \over \Der s} \left\{\int_{\sigma_s}\alpha\right\}_{s = 0} = \int_{\cub{p}} \left. {\Der a_0 \over \Der s}\right|_{s=0} dr^1\wedge\cdots  \wedge dr^p.
    $$%
  Next,
  let us introduce the exterior derivative of $\alpha(\bsigma)$,
  $$%
    d[\alpha(\bsigma)]_{s \choose r} = d\alpha(\bsigma)_{s \choose r} = \bigg\{ {\Der a_0 \over \Der s} - {\Der a_1 \over \Der r^1} + \cdots + (-1)^p{\Der a_p \over \Der r^p} \bigg\}\, ds \wedge dr^1 \wedge \cdots \wedge dr^p.
    $$%
  After contracting the two terms of this identity by the vector of coordinates $(1,0) \in \RR \times \RR^p$,
  we get
  $$%
    {\Der a_0 \over \Der s}\,d r^1\wedge\cdots \wedge dr^p  = d\alpha(\bsigma)_{s \choose r} \vect{1 \\ 0} - \bigg\{ \sum_{k = 1}^p (-1)^k \,{\Der a_k \over \Der r^k} \bigg\}  \, dr^1 \wedge \cdots \wedge dr^p.
    $$%
  The first term of the right-hand side is the inner product \art{Inner-product} of the value of the $(p+1)$-form $d\alpha(\bsigma)$ at the point $(s,r)$,
  by the vector of coordinates $(1, 0) \in \RR \times \RR^p$.
  Evaluated at the point $(0,r)$,
  and restricted to $\RR^p$,
  it is exactly the contraction $d\alpha$ by the arc of $p$-cubes $s \mapsto \sigma_s$ \xart{Contraction-of-differential-forms}{($\diamondsuit$)}.
  Hence,
  $$%
    \left. {\Der a_0 \over \Der s}\right|_{s=0} d r^1\wedge\cdots \wedge dr^p  = d\alpha(\delta\sigma)_r - \bigg\{ \sum_{k = 1}^p (-1)^k \,{\Der \aa_k(0,r) \over \Der r^k} \bigg\}  \, dr^1 \wedge \cdots \wedge dr^p.
    $$%
  The second term of the right-hand side of this identity is just the exterior derivative of the contraction $\alpha(\delta \sigma)$.
  Indeed,
  let $v_2, \ldots, v_p$ be  $(p-1)$ vectors of $\RR^p$.
  Let us abridge the notation,
  $$%
    [v] = (v_2)\cdots (v_p), \text{ and } \sqvect{0 \\v} = \vect{ 0 \\ v_2} \cdots \vect{ 0 \\ v_p}.
    $$%
  Using the above expression of $\alpha(\bsigma)$,
  we get
  \begin{eqnarray*}
    \alpha(\delta \sigma)_r[v] & = & \alpha(\bsigma)_{0 \choose r} \vect{ 1 \\ 0} \sqvect{0 \\v} \\
    & = & \bigg[\sum_{k = 0}^p \aa_k(0,r) \, dr^0\wedge \cdots [dr^k] \cdots \wedge dr^p \bigg] \vect{ 1 \\ 0} \sqvect{0 \\ v} \\
    & = & \bigg[\sum_{k=1}^p \aa_k(0,r) \, dr^1 \wedge \cdots [dr^k] \cdots \wedge dr^p\bigg][v].
  \end{eqnarray*}
  Thus,
  $$%
    \alpha(\delta\sigma)_r  = \sum_{k=1}^p \aa_k(0,r) \, dr^1 \wedge \cdots [dr^k] \cdots \wedge dr^p.
    $$%
  Then,
  \begin{eqnarray*}
    d[\alpha(\delta \sigma)]_r & = & \sum_{k=1}^p {\partial \aa_k(0,r) \over \partial r^k}\, dr^k \wedge dr^1 \wedge \cdots [dr^k] \cdots \wedge dr^p \\
    & = & \sum_{k=1}^p (-1)^{k-1} {\partial \aa_k(0,r) \over \partial r^k} \, dr^1 \wedge \cdots \wedge dr^k \wedge \cdots \wedge dr^p \\
    & = & - \bigg\{ \sum_{k = 1}^p (-1)^k \,{\Der \aa_k(0,r) \over \Der r^k} \bigg\}  \, dr^1 \wedge \cdots \wedge dr^p.
  \end{eqnarray*}
  Hence,
  $$%
    \left. {\Der a_0 \over \Der s}\right|_{s=0} d r^1\wedge\cdots \wedge dr^p  = d\alpha(\delta\sigma)_r + d[\alpha(\delta\sigma)]_r\,,
    $$%
  and
  \renewcommand{\theequation}{$\spadesuit$}
  \begin{eqnarray}
    {\Der \over \Der s} \left\{\int_{\sigma_s}\alpha\right\}_{s = 0} & = & \int_{\sigma_s} \left. {\Der a_0 \over \Der s}\right|_{s=0} d r^1\wedge\cdots \wedge dr^p \nonumber \\
    & = &\int_{\cub{p}} d\alpha(\delta\sigma) + d[\alpha(\delta\sigma)] \nonumber \\
    & = & \int_{\id_p} d\alpha \rfloor \delta\sigma + \int_{\id_p} d[\alpha \rfloor \delta\sigma].
  \end{eqnarray}
  Finally,
  combining $(\clubsuit)$ and $(\spadesuit)$,
  we obtain the formula $(\diamondsuit)$ of the variation of the integral of a $p$-form on a $p$-cube.
\end{proof}

\begin{article}\artlabel{Variation of the integral of forms on chains}
  \addcontentsline{toc}{section}{\small\hspace{10pt} Variation of the integral of forms on chains}
  \label{Variation-of-the-integral-of-a-form-on-chains}
  We defined the integral of a $p$-form on a cubic $p$-chain by $\int_c\alpha = \sum_\sigma n_\sigma \int_\sigma \alpha$,
  where $c = \sum_\sigma n_\sigma \sigma$ \art{Integrating-forms-on-chains}.
  To extend the variation of the integral of a $p$-form,
  from a $p$-cube \art{Variation-of-the-integral-of-a-form-on-a-cube} to a cubic $p$-chain,
  we need a suitable diffeology on the set of cubic $p$-chains,
  we got one in \xart{Cubes-and-cubic-chains-on-diffeological-spaces}{Note 1}.
  A parametrization $r \mapsto c_r$ from $\U$ to $\DChains{p}(\X)$,
  is a plot if it satisfies the following condition.
  \begin{itemize}
    \item[($\heartsuit$)] For all $r_0 \in \U$,
    there exist an open neighborhood $\V$ of $r_0$,
    a finite family of indices $\cI$,
    together with a family of integers $\{n_i\}_{i \in \cI}$ and a family $\{\sigma_i\}_{i \in \cI}$ of elements of $\Cinfty(\V,\Cub_p(\X))$ such that $c_r = \sum_{i \in \cI} n_i \, \sigma_{i,r}$ for all $r \in \V$.
  \end{itemize}
  Now,
  let $s\mapsto c_s$ be an arc of cubic $p$-chain centered at $c$ and $s \mapsto \alpha_s$ be an arc of $p$-form centered at $\alpha$,
  the variation of the integral of $\alpha$ on $c$ is given by
  $$%
    \delta \int_c \alpha = \int_{\id_p} d\alpha \rfloor \delta c + \int_{\partial \id_p} \alpha \rfloor \delta c + \int_{\id_p} c^*(\delta \alpha).
    $$%
  We need to explain the terms involved in the formula.
  Every cubic $p$-chain $c$ decomposes into a finite sum $c = \sum_{i \in \cI} n_i \, \sigma_i$.
  The pullback of a $p$-form $\alpha$ by $c$ is defined by linearity,
  $c^*(\alpha) = \sum_{i \in \cI} n_i\, \sigma_i^*(\alpha)$,
  as well $\alpha(c) = \sum_{i \in \cI} n_i\, \alpha(\sigma_i)$.
  The contraction $d\alpha\rfloor \delta c$ and $\alpha\rfloor \delta c$ also are given by linearity,
  $d\alpha \rfloor \delta c = \sum_{i \in \cI} n_i \, d\alpha \rfloor \delta \sigma_i$ and $\alpha \rfloor \delta c = \sum_{i \in \cI} n_i \, \alpha \rfloor \delta \sigma_i$.
  They are $p$-forms on $\RR^p$,
  or a $(p-1)$-form on $\RR^{p-1}$,
  and do not depend on the decomposition of $c$.
\end{article} %% Variation-of-the-integral-of-a-form-on-chains

\begin{proof}
  Consider a small interval around $0 \in \RR$ on which $c_s = \sum_{i \in \cI} n_i \, \sigma_{i,s}$,
  then
  $$%
    \delta\int_c \alpha = {\partial \over \partial s} \left\{\sum_{i \in \cI} n_i \int_{\sigma_{i,s}} \alpha_s\right\}_{s=0} = \sum_{i \in \cI} n_i {\partial \over \partial s} \left\{\int_{\sigma_{i,s}} \alpha_s\right\}_{s=0} = \sum_{i \in \cI} n_i\, \delta\! \int_{\sigma_i} \alpha.
    $$%
  The formula of the variation of the integral of a $p$-form on a $p$-cube \art{Variation-of-the-integral-of-a-form-on-a-cube} gives
  \begin{eqnarray*}
    \delta\int_c \alpha & = & \sum_{i \in \cI} n_i  \int_{\cub{p}} d\alpha (\delta\sigma_i) + \sum_{i \in \cI} n_i \int_{\cub{p}} d[\alpha(\delta\sigma_i)] + \sum_{i \in \cI} n_i \int_{\cub{p}} \delta \alpha(\sigma_i) \\
    & = & \int_{\cub{p}} d\alpha \bigg(\sum_{i \in \cI} n_i \, \delta\sigma_i\bigg) + \int_{\cub{p}} d\bigg[\alpha \bigg(\sum_{i \in \cI} n_i \, \delta\sigma_i\bigg)\bigg] + \int_{\cub{p}} \delta \alpha\bigg(\sum_{i \in \cI} n_i \, \sigma_i\bigg) \\
    & = & \int_{\cub{p}} d\alpha (\delta c) + \int_{\cub{p}} d[\alpha (\delta c)] + \int_{\cub{p}} \delta \alpha(c).
  \end{eqnarray*}
  This makes sense according to the definition \art{Contraction-of-differential-forms}.
  Let $\bmc : (s,t) \mapsto c_s(t)$ and $\bsigma_i : (s,t) \mapsto \sigma_{i,s}(t)$ such that $\bmc = \sum_{i \in \cI} \bsigma_i$.
  We have for all $t \in \RR^p$ and all $v_2, \ldots, v_p \in \RR^p$,
  \begin{eqnarray*}
    \alpha(\delta c)(t)(v_2) \cdots (v_{p}) & = & \alpha(\bmc)_{0 \choose t} \vect{1 \\ 0} \vect{0 \\ v_2} \cdots \vect{0 \\ v_{p}} \\
    & = & \sum_{i \in \cI} n_i \, \alpha(\bsigma_i)_{0 \choose t} \vect{1 \\ 0} \vect{0 \\ v_2} \cdots \vect{0 \\ v_{p}}.
  \end{eqnarray*}
  The same holds for $d\alpha(\delta c)$.
  We have still to be sure that the computation does not depend on the choice of the decomposition of the arc $s \mapsto c_s$.
  Let us check generally that the evaluation of a $p$-form $\alpha$ on a cubic $p$-chain $c$ does not depend on a decomposition $c = \sum_{i \in \cI} n_i \, \sigma_i$.
  Equivalently,
  if $\sum_{i \in \cI} n_i \, \sigma_i = 0$,
  then $\sum_{i \in \cI} n_i \, \alpha(\sigma_i) = 0$.
  Let $i \sim j$ if $\sigma_i = \sigma_j$,
  let $\cA = \quotient{\cI}{\sim}$,
  and let us denote $\sigma_a = \sigma_i$ for every $a \in \cA$, where $i \in a$,
  then $\sum_{i \in \cI} n_i \, \sigma_i = \sum_{a \in \cA} \big(\sum_{i \in a} n_i\big) \, \sigma_a$.
  The same factorization gives $\sum_{i \in \cI} n_i \, \alpha(\sigma_i) = \sum_{a \in \cA} \big(\sum_{i \in a} n_i\big) \, \alpha(\sigma_a)$.
  But since $\sum_{a \in \cA} \big(\sum_{i \in a} n_i\big) \sigma_a = 0$ and since the $\sigma_a$ are all different,
  $\sum_{i \in a} n_i = 0$ for all $a \in \cA$.
  Therefore,
  $\sum_{a \in \cA} \big(\sum_{i \in a} n_i\big) \, \alpha(\sigma_a) = 0$,
  that is,
  $\sum_{i \in \cI} n_i \, \alpha(\sigma_i) = 0$.
  Now,
  if for all $s$,
  $c_s = \sum_{i \in \cI} n_i \, \sigma_{i,s} = \sum_{i' \in \cI'} n_{i'} \, \sigma_{i',s}$,
  then, from what precedes,
  $\alpha_s\big(\sum_{i \in \cI} n_i \, \sigma_{i,s}\big) = \alpha_s\big(\sum_{i' \in \cI'} n_{i'} \, \sigma_{i',s}\big)$.
  Thus,
  the evaluation of the variation does not depend on the choice of the decomposition.
\end{proof}

\begin{article}\artlabel{The Cartan-Lie formula}
  \addcontentsline{toc}{section}{\small\hspace{10pt} The Cartan-Lie formula}
  \label{The-Cartan-Lie-formula}
  Let $\X$ be a diffeological space,
  and let $\Diff(\X)$ be its group of diffeomorphisms equipped with the functional diffeology \art{Functional-diffeology-on-groups-of-diffeomorphisms}.
  Let $\F: \RR \to \Diff(\X)$ be a sliding,
  that is,
  a $1$-plot centered at the identity \art{Contracting-differential-forms-on-slidings}.
  Let $\alpha$ be any differential $k$-form on $\X$,
  with $k\geq 1$.
  Then,
  $$%
    \DLie_\F(\alpha) = i_\F(d\alpha) + d(i_\F(\alpha)),
    $$%
  where $\DLie_\F(\alpha)$ is the Lie derivative of $\alpha$ by $\F$ \art{The-Lie-derivative-of-differential-forms},
  and $i_\F$ denotes the contraction by $\F$ \art{Contracting-differential-forms-on-slidings}.
  For a $0$-form $f$,
  that is,
  a smooth function from $\X$ to $\RR$,
  the Cartan formula is reduced to $\DLie_\F(f) = i_\F(df)$.
  The identity above extends,
  to the diffeological spaces,
  the {\em Cartan-Lie formula} of
  classical differential geometry.
  
  Actually,
  the Cartan-Lie formula plays a crucial role in the definition of the Chain-Homotopy operator \art{The-Chain-Homotopy-operator-K},
  which is,
  in particular,
  the main tool in the construction of the moment map in diffeology (see Chap. \ref{Chapter-Symplectic-Diffeology}).
\end{article} %% The-Cartan-Lie-formula

\begin{proof}
  Let $\sigma \in \DCubes{p}(\X)$ \art{Cubes-and-cubic-chains-on-diffeological-spaces} and
  $$%
    \left\{
    \begin{array}{l}
    \alpha_t = \F(t)_*(\alpha) = (\F(t)^{-1})^*(\alpha), \\[1ex]
    \sigma_t = (\F(t))_*(\sigma) = \F(t) \circ \sigma.
    \end{array}
    \right.
    $$%
  Thanks to \art{Pulling-back-and-forth-forms-and-chains},
  for all $t \in \RR$,
  $$%
    \int_{\sigma_t}\alpha_t = \int_{\F(t)_*(\sigma)}\F(t)_*(\alpha) =  \int_\sigma \F(t)^* \circ \F(t)_*(\alpha) =  \int_\sigma \alpha.
    $$%
  Now,
  by differentiation with respect to the parameter $t$,
  we get on the one hand
  $$%
    \delta\int_{\sigma_t}\alpha_t = \delta\int_{\sigma}\alpha = 0, \text{ with } \delta = \left.{\partial \over \partial t}\;\right\vert_{t = 0},
    $$%
  and on the other hand,
  by the formula of the variation of integral of differential forms \art{Variation-of-the-integral-of-a-form-on-a-cube},
  \begin{equation}\renewcommand{\theequation}{$\diamondsuit$}
    \delta\int_{\sigma_t}\alpha_t = \int_{\id_p} d\alpha \rfloor \delta \sigma + \int_{\id_p} d[\alpha \rfloor \delta \sigma] + \int_{\id_p} \delta\sigma^*(\alpha).
  \end{equation}
  But:
  \begin{itemize}
    \item[(a)] $\displaystyle{ \delta \alpha = \left.{\partial \alpha_t \over \partial t} \, \right\vert_{t = 0} = \left.{\partial \over \partial t} \F(t)_*(\alpha) \, \right\vert_{t = 0} = \left.{\partial \over \partial t} (\F(t)^{-1})^*(\alpha)\;\right\vert_{t = 0} = - \DLie_\F(\alpha)}$.
    \item[(b)] $\sigma_t = \F(t) \circ \sigma$ and $\displaystyle{\delta = \left.{\partial \over \partial t} \, \right\vert_{t = 0}}$,
    which implies $\alpha \rfloor \delta \sigma = \sigma^*(i_\F(\alpha))$.
  \end{itemize}
  The point (a) is \exref{Anti-Lie-derivative}.
  Then,
  the identity $(\diamondsuit)$ above becomes
  $$%
    0 = \int_\sigma i_\F[d\;\alpha] + \int_{\sigma}d\;[i_\F(\alpha)] - \int_\sigma\DLie_\F(\alpha) = \int_\sigma i_\F[d\;\alpha] + d\;[i_\F(\alpha)]- \DLie_\F(\alpha).
    $$%
  Since this is satisfied for any $p$-cube $\sigma$,
  we get $i_\F[d\;\alpha)] + d\;[i_\F(\alpha)] - \DLie_\F(\alpha) = 0$ \art{Pairing-chains-and-forms}.
  Thus,
  $\DLie_\F(\alpha) =  i_\F[d\;\alpha)] + d\;[i_\F(\alpha)]$.
\end{proof}

%%%%%%%%%%%%%%%%%%%%%%%%%%%%%%%%%%%%%%%%%%%%%%%%%%%%%%%%%%
%
%   Exercises
%
%%%%%%%%%%%%%%%%%%%%%%%%%%%%%%%%%%%%%%%%%%%%%%%%%%%%%%%%%%

\Exercises

\begin{exercise}[Liouville rays and closed forms]
  \label{Liouville-rays-and-closed-forms}
  Consider the notations and hypothesis of \exref{Liouville-rays}.
  Show that,
  for the $p$-form $\omega$,
  with $p \geq 1$,
  if $d\omega = 0$,
  then $\omega$ is exact and  $\varpi = i_h(\omega)$ is a primitive,
  $d\varpi = \omega$.
\end{exercise} %% Liouville-rays-and-closed-forms

\begin{exercise}[Integrals on homotopic cubes]
  \label{Integrals-on-homotopic-cubes}
  Let $\X$ be a diffeological space.
  Let $\alpha$ be a closed $p$-form,
  $\alpha \in \Omega^p(\X)$ and $d\alpha = 0$.
  Let $s \mapsto \sigma_s$ be an arc of $p$-cubes of $\X$ centered at $\sigma$ such that $\sigma_s \restriction \partial \I^p = \sigma \restriction \partial \I^p$ for all $s$,
  where $\partial \I^p = \bigcup_{k=1}^p \bigcup_{a = 0,1} j_p(a)(\I^{p-1})$ and $\I = \closedinterval{0,1}$.
  Use the formula of the variation of the integral of a $p$-form on a $p$-cube \art{Variation-of-the-integral-of-a-form-on-a-cube} to show that $\delta \int_\sigma \alpha = 0$.
\end{exercise} %% Integrals-on-homotopic-cubes

\begin{exercise}[Closed $1$-forms on connected spaces]
  \label{Closed-1-forms-on-connected-spaces}
  Let $\X$ be a connected diffeological space,
  and $\alpha$ be a closed $1$-form of $\X$.
  Show that the integral of $\alpha$ on any $\ell \in \Loops(\X,x)$ depends only on the fixed-ends homotopy classes of $\ell$ \art{Homotopy-of-paths}.
  Let $\Periods_\alpha$ be the set of all the numbers $\int_\ell \alpha \in \RR$ where $\ell \in \Loops(\X,x)$,
  show that $\Periods_\alpha$ does not depend on the basepoint $x$.
  Conclude that $\Periods_\alpha$ is a homomorphic image of $\pi_1(\X)$ \art{The-Poincare-groupoid-and-fundamental-group}.
\end{exercise} %% Closed-1-forms-on-connected-spaces

\begin{exercise}[Closed $1$-forms on simply connected spaces]
  \label{Closed-1-forms-on-simply-connected-spaces}
  Let $\X$ be a diffeological space and $\alpha$ be a closed $1$-form of $\X$.
  Show that if there exists a loop $\ell$ based at a point $x$ \art{The-space-of-Paths-of-a-diffeological-space} such that $\int_\ell \alpha \neq 0$,
  then the connected component of $x$ \art{Pathwise-connectedness} is not simply connected \art{The-Poincare-groupoid-and-fundamental-group}.
\end{exercise} %% Closed-1-forms-on-simply-connected-spaces

%%%%%%%%%%%%%%%%%%%%%%%%%%%%%%%%%%%%%%%%%%%%%%%%%%%%%%%%%%
%% MARK: De Rham Cohomology
%%%%%%%%%%%%%%%%%%%%%%%%%%%%%%%%%%%%%%%%%%%%%%%%%%%%%%%%%%

\section*{De Rham Cohomology}
\label{Section-De-Rham-cohomology}

\begin{sechead}
  In this section we introduce the De Rham cohomology of diffeological spaces,
  according to the definition of differential forms \art{Differential-forms-on-diffeological-spaces} and the exterior derivative \art{Exterior-derivative-of-forms}.
  We give some examples to show how homotopy and De Rham cohomology can be related.
  A precise statement will be given later \art{Homotopic-invariance-of-the-De-Rham-cohomology}.
\end{sechead}

\begin{article}\artlabel{The De Rham cohomology}
  \addcontentsline{toc}{section}{\small\hspace{10pt} The De Rham cohomology spaces}
  \label{The-De-Rham-cohomology}
  Let $\X$ be a diffeological space.
  The exterior derivative  defined above \art{Exterior-derivative-of-forms} satisfies the {\em coboundary condition}
  $$%
    d: \Omega^p(\X) \to \Omega^{p+1}(\X), \ p \geq 0
    \text{ and }
    d \circ d = 0.
    $$%
  As it is usual in cohomology theories \cite{McL75},
  when we have a chain complex
  ---~here the chain complex of real vector spaces $\Omega^\star(\X) = \set{\Omega^p(\X)}_{p=0}^\infty$ with a coboundary operator $d$~---
  the space of {\em $p$-cocycles}\index{De Rham cocycle} is defined as the kernel in $\Omega^p(\X)$ of the operator $d$,
  and the space of {\em $p$-coboundary}\index{De Rham coboundary} is defined as the image,
  in $\Omega^p(\X)$,
  of the operator $d$.
  They will be denoted by
  $$%
    \left\{
    \begin{array}{rcl}
    \ZDR^p(\X) & = &\ker \left[d: \Omega^p(\X) \to \Omega^{p+1}(\X)\right], \\
    \ \vspace{-2ex} \\
    \BDR^p(\X) & = & d (\Omega^{p-1}(\X))\subset \ZDR^p(\X),  \text{ with } \BDR^0(\X) = \{0\}.
    \end{array}
    \right.
    $$%
  The {\em De Rham cohomology groups}\index{De Rham cohomology group} of $\X$ are then defined as the quotients of the spaces of cocycles by the spaces of coboundaries,
  we denote them by
  $$%
    \HDR^p(\X) = \ZDR^p(\X)/\BDR^p(\X).
    $$%
  Since the operator $d$ is linear,
  and since the space of differential $p$-forms $\DForms^p(\X)$,
  equipped with the functional diffeology \art{Functional-diffeology-of-the-space-of-forms},
  is a diffeological vector space \art{Diffeological-vector-spaces},
  the De Rham cohomology group $\HDR^p(\X)$,
  equipped with the quotient diffeology  \art{Quotient-of-diffeological-vector-spaces},
  is a diffeological vector space.
\end{article} %% The-De-Rham-cohomology

\begin{article}\artlabel{The De Rham homomorphism}
  \addcontentsline{toc}{section}{\small\hspace{10pt} The De Rham homomorphism}
  \label{The-De-Rham-homomorphism}
  Let $\X$ be a diffeological space,
  let $p$ be any positive integer,
  and let $\alpha \in \DForms^p(\X)$.
  The integration of $\alpha$ on the cubic $p$-chains \art{Integrating-forms-on-chains} defines a cubic $p$-cochain $f_\alpha$ for the reduced cubic cohomology \art{Cubic-cohomology},
  for all $c \in \DChains{p}(\X)$,
  $$%
    f_\alpha(c) = \int_c \alpha, \text{ and } f_\alpha \in \QDCochains{p}(\X,\RR).
    $$%
  Then,
  thanks to Stokes' theorem \art{The-Stokes-theorem},
  if $\alpha$ is closed,
  $d\alpha = 0$,
  then $f_\alpha$ is closed as cochain,
  $df_\alpha = 0$.
  Thus,
  the integration of $\alpha$ on chains defines a morphism from $\ZDR^p(\X)$ to $\QDZ^p(\X)$.
  If $\alpha$ is exact,
  $\alpha = d\beta$,
  then $f_\alpha$ is exact as cochain,
  $f_\alpha = df_\beta$.
  Hence,
  the integration on a chain defines a linear map from $\HDR^p(\X)$ to $\QDH^p(\X)$.
  This morphism is called the {\em De Rham homomorphism}\index{De Rham homomorphism},
  and we shall denote it by $\hDR^p$,
  $$%
    \hDR^p \in \Lin(\HDR^p(\X),\QDH^p(\X)),
    \text{ with }
    \hDR^p : \class(\alpha) \mapsto \class\bigg( c \mapsto \int_c \alpha \bigg).
    $$%
  
  \Note{1} For a closed $p$-form $\alpha$,
  representing some class in $\HDR^p(\X)$,
  the set  of values of the $p$-cochain $f_\alpha$ on the $p$-cycles is a homomorphic image of the group $\QDH_p(\X)$ \art{Cubic-homology} in $\RR$,
  it is generally called the {\em group of periods}\index{Group of periods} of $\alpha$,
  and denoted by
  \begin{equation}
    \renewcommand{\theequation}{$\clubsuit$}
    \Periods_\alpha = \bigg\{ \int_c \alpha \in \RR \mathop{\big|} c \in \QDZ_p(\X) \bigg\}.
  \end{equation}
  We recall that $\QDZ_p(\X) = \{c \in \DChains{p}(\X) \mid \partial c = 0\}$.
  In diffeology the groups of periods of a closed $p$-form have some refinements and may better be defined by iterations through the space of loops;
  see for instance \art{Integration-bundles-of-closed-2-forms} and \exref{Periods-of-a-surface}.
  
  \Note{2} The De Rham homomorphism is generally not an isomorphism in diffeology;
  see \exref{De-Rham-homomorphism-and-irrational-tori}.
  There exists however a spectral sequence describing the relationship between the two cohomologies \cite{Igl87b, PIZ21b}.
  The first obstruction is interpreted in \art{The-cokernel-of-the-first-De-Rham-homomorphism},
  but very little,
  perhaps nothing,
  is known about a geometric interpretation of the other terms.
\end{article} %% The-De-Rham-homomorphism

\begin{proof}
  Since the integration of a $p$-form on a $p$-cube is smooth \art{Pairing-chains-and-forms},
  the De Rham homomorphism is smooth.
  Now,
  if $\sigma$ is a degenerate $p$-cube,
  that is,
  $\sigma = \sigma' \circ \pr$,
  where $\pr : \RR^p \to \RR^q$ and $q<p$,
  then
  $$%
    \int_\sigma \alpha = \int_{\sigma'_*(\pr)} \alpha = \int_{\pr} \sigma'^*(\alpha).
    $$%
  But $\sigma'^*(\alpha)$ is a $p$-form on $\RR^q$ with $q<p$,
  thus $\sigma'^*(\alpha) = 0$ and the integral vanishes.
  Hence,
  the morphism $c \mapsto \int_c \alpha$ vanishes on degenerate cubes,
  and then on degenerate cubic chains.
  Therefore,
  this is a cochain for the reduced cubic cohomology.
\end{proof}

\begin{article}\artlabel{The functions whose differential is zero}
  \addcontentsline{toc}{section}{\small\hspace{10pt} The functions whose differential is zero}
  \label{The-functions-whose-differential-is-zero}
  Let $\X$ be a diffeological space.
  Let us recall that two points $x_0$ and $x_1$ are said to be connected if there exists a path $\gamma \in \Paths(\X) = \Cinfty(\RR, \X)$ such that $\gamma(0) = x_0$ and $\gamma(1) = x_1$ \art{Pathwise-connectedness}.
  Connectedness is an equivalence relation on $\X$,
  the classes of this relation are the components of $\X$ and the set of components of $\X$ has been denoted by $\pi_0(\X)$ \art{Connected-components}.
  Next,
  according to the definition above \art{The-De-Rham-cohomology} and given $\Omega^0(\X) = \Cinfty(\X,\RR)$ \art{Zero-forms-are-smooth-functions},
  we have
  $$%
    \HDR^0(\X) = \ker \left[ d: \Omega^0(\X) \to \Omega^{1}(\X)\right] = \{f \in \Cinfty(\X,\RR) \mid df = 0 \}.
    $$%
  Now,
  every function whose differential is zero is constant on the connected components of $\X$,
  and is therefore characterized by the set of values it takes on each connected component,
  thus
  $$%
    \HDR^0(\X) = \Maps(\pi_0(\X),\RR).
    $$%
  
  \Note{1} The group of periods $\Periods_f$ of $f$  \xart{The-De-Rham-homomorphism}{($\clubsuit$)} is generated by the following set of periods
  $$%
    \PeriodsOf(f) = \{f(\X_i) \mid \X_i \in \pi_0(\X) \} \subset \RR.
    $$%
  
  \Note{2} The group $\HDR^0(\X)$ also coincides with the cubic cohomology group $\QDH^0(\X)$ \art{Interpreting-H-0-X-R}.
  The De Rham homomorphism $\hDR^0$ \art{The-De-Rham-homomorphism} is an isomorphism.
\end{article} %% The-functions-whose-differential-is-zero

\begin{proof}
  Let $x$ and $x'$ be two connected points,
  let $\gamma \in \Cinfty(\RR,\X)$ such that $x = \gamma(0)$,
  and let $x' = \gamma(1)$.
  Then,
  $df = 0$ implies $df(\gamma) = \gamma^*(df) = d[\gamma^*(f)] = 0$.
  Hence,
  $\gamma^*(f) =  f \circ \gamma = \const$.
  Therefore,
  the function $f$ is constant on any connected component of $\X$.
  Conversely,
  the differential of any smooth function,
  constant on each connected component of $\X$,
  is zero.
  Therefore,
  $\HDR^0(\X)$ is the Abelian group generated by the set of components of $\X$,
  that is,
  $\HDR^0(\X) = \Maps(\pi_0(\X),\RR)$.
\end{proof}

\begin{article}\artlabel{Vanishing De Rham 1-cohomology}
  \addcontentsline{toc}{section}{\small\hspace{10pt} The Poincar\'e Lemma for diffeological spaces}
  \label{Vanishing-De-Rham-1-cohomology}
  Let us recall that a diffeological space $\X$ is said to be {\em connected} if for any two points $x,x' \in \X$ there exists a path $\gamma \in \Paths(\X) = \Cinfty(\RR,\X)$ such that $\gamma(0)=x$ and $\gamma(1)=x'$ \art{Pathwise-connectedness}.
  This is denoted by $\pi_0(\X) = \{\X\}$.
  Then,
  the space $\X$ is said to be {\em simply connected} if $\X$ is connected,
  and if for any two points $x, x' \in \X$,
  and for any two paths $\gamma$ and $\gamma'$ connecting $x$ to $x'$,
  there exists a path $[s \mapsto  \gamma_s] \in \Paths(\Paths(\X)) = \Cinfty(\RR,\Paths(\X))$ such that $\gamma_0 = \gamma$,
  $\gamma_1 = \gamma'$,
  $\gamma_s(0) =x$ and $\gamma_s(1) = x'$,
  for all $s$ \art{The-Poincare-groupoid-and-fundamental-group}.
  This is denoted by  $\pi_1(\X) = \{0\}$.
  Now,
  for every simply connected diffeological space $\X$,
  the first De Rham cohomology group is trivial,
  $\HDR^1(\X) = \{0\}$,
  that is,
  every closed $1$-form is exact.
  Precisely,
  for all $\alpha \in \Omega^1(\X)$ such that $d\alpha = 0$ the functions
  $$%
    f : x \mapsto  \int_\xo^x \alpha + \const
    $$%
  are smooth and are the primitives of $\alpha$, $\alpha = df$.
  The integral is taken on any path in $\X$,
  connecting an arbitrary chosen basepoint $\xo$ to $x$.
\end{article} %% Vanishing-De-Rham-1-cohomology

\begin{proof}
  Let $\xo \in \X$ be a point,
  chosen as origin.
  Let $\Paths(\X,\xo,\star)$ be the subspace of paths in $\X$ having $\xo$ as origin \art{The-space-of-Paths-of-a-diffeological-space},
  that is,
  $$%
    \Paths(\X,\xo,\star) = \{\gamma \in \Cinfty(\RR,\X) \mid  \gamma(0) = \xo \}.
    $$%
  Then,
  let
  $$%
    \F : \Paths(\X,\xo,\star) \to \RR, \text{ with } \F(\gamma) =  \int_\gamma \alpha.
    $$%
  The value $\F(\gamma)$ is just the pairing of $\alpha$ with $\gamma$.
  Moreover,
  thanks to \art{Pairing-chains-and-forms},
  $\F$ belongs to $\Cinfty(\Paths(\X,\xo,\star),\RR)$.
  Since the space $\X$ is connected,
  the projection $\1: \Paths(\X,\xo,\star) \to \X$,
  defined by $\1(\gamma) = \gamma(1)$,
  is a subduction \art{The-space-of-Paths-of-a-diffeological-space}.
  Let us show the following propositions.
  %
  \begin{itemize}
    \item[1.] There exists $f: \X \to \RR$ such that $\F = f \circ \1$.
    Then,
    since $\1$ is a subduction and $\F$ is smooth,
    $f$ is smooth \art{Smooth-maps-from-quotients}.
    \item[2.] $d\F = \1^*(\alpha)$.
    Then,
    since $d\F = d(f \circ \1) =  \1^*(df)$, $\1^*(\alpha) = \1^*(df)$.
    Next,
    $\1$ being a subduction, $\alpha = df$ \art{Pushing-forms-onto-quotients}.
  \end{itemize}
  %
  Let us establish first that $d\F = \1^*(\alpha)$.
  Let $\P : \U \to \Paths(\X,\xo,\star)$ be an $n$-plot,
  and let us denote $\P(r) = \gamma_r$ for $r \in \U$.
  Let $\delta r \in \RR^n$ be any vector,
  then
  $$%
    d\F(\P)_r(\delta r) = {\Der \F(\P)(r) \over \Der r}( \delta r) =
    {\Der \over \Der r}\bigg\{\int_{\gamma_r}\alpha \bigg\} (\delta r) =
    \delta \int_\gamma \alpha.
    $$%
  But thanks to the formula of the variation of the integral \art{Variation-of-the-integral-of-a-form-on-a-cube},
  $$%
    \delta \int_\gamma \alpha = \int_0^1 d\alpha(\delta \gamma) + \int_0^1 d[\alpha(\delta \gamma)]
    = 0 + \bigg[\alpha(\delta \gamma)\bigg]_{t=0}^{t=1},
    $$%
  that is,
  $$%
    \delta \int_\gamma \alpha  = \bigg[\alpha(\bgamma)_{r \choose t} \vect{\delta r \\ 0} \bigg]_{t = 0}^{t = 1},
    $$%
  where $\bgamma(r,t) = \gamma_r(t) = {\P}(r)(t)$.
  Hence,
  \begin{eqnarray*}
    d\F({\P})_r(\delta r) & = & \alpha[(r,t) \mapsto {\P}(r)(t)]_{r \choose 1}{\scriptstyle \vect{\delta r \\ 0}} - \alpha[(r,t) \mapsto {\P}(r)(t)]_{r \choose 0}{\scriptstyle \vect{\delta r \\ 0}} \\
    & = & \alpha[r \mapsto {\P}(r)(1)]_r(\delta r) - \alpha[r \mapsto {\P}(r)(0)]_r(\delta r) \\
    & = & (\1^*(\alpha))({\P})_r(\delta r) - \alpha[r \mapsto  \xo]_r(\delta r) \\
    & = & (\1^*(\alpha))({\P})_r(\delta r) -0.
  \end{eqnarray*}
  Therefore,
  $d\F = \1^*(\alpha)$.
  Now,
  for all $x \in \X$,
  the restriction of $d\F$ to any pullback $\1^{-1}(x)$ vanishes,
  that is,
  $d\F\restriction \1^{-1}(x) = 0$.
  Thus,
  $d\F$ is locally constant on the subspace $\1^{-1}(x)$.
  But $\1^{-1}(x)$ is the subspace of paths in $\X$ with origin $\xo$ and end $x$.
  By hypothesis $\X$ is simply connected,
  hence $\1^{-1}(x)$ is connected.
  Therefore,
  $\F$ is constant on $\1^{-1}(x)$
  \exref{To-be-a-locally-constant-map},
  and there exists $f: \X \to \RR$ such that $\F = f \circ \1$.
\end{proof}

\begin{article}\artlabel{Closed $1$-forms on locally simply connected spaces}
  \addcontentsline{toc}{section}{\small\hspace{10pt} Closed $1$-forms on locally simply connected spaces}
  \label{Closed-1-forms-on-locally-simply-connected-spaces}
  Let $\X$ be a diffeological space,
  $\X$ is said to be {\em locally simply connected} if every D-open \art{The-D-Topology-of-diffeological-spaces} neighborhood of every point $x \in \X$ contains a simply connected \art{The-Poincare-groupoid-and-fundamental-group} D-open neighborhood of $x$.
  In particular,
  if $\X$ is locally simply connected,
  there exists a D-open covering $\cU = \{ \U_i\}_{i \in \cI}$ of $\X$,
  such that each element $\U_i$ of $\cU$ is simply connected.
  As a consequence of the previous proposition \art{Vanishing-De-Rham-1-cohomology},
  every closed $1$-form $\alpha$ on a locally simply connected diffeological space $\X$ is locally exact,
  that is,
  for every $x \in \X$ there exist a D-open neighborhood $\U$ of $x$ and a function $f \in \Cinfty(\U,\RR)$ such that $\alpha\restriction \U = df$.
  Finite dimensional manifolds \art{Manifolds-as-diffeologies},
  for example,
  are locally simply connected.
  Indeed,
  manifolds are locally diffeomorphic to real domains,
  and real domains are locally simply connected.
  Thus,
  any closed form defined on a finite dimensional manifold is locally exact.
\end{article} %% Closed-1-forms-on-locally-simply-connected-spaces

%%%%%%%%%%%%%%%%%%%%%%%%%%%%%%%%%%%%%%%%%%%%%%%%%%%%%%%%%%
%
%   Exercises
%
%%%%%%%%%%%%%%%%%%%%%%%%%%%%%%%%%%%%%%%%%%%%%%%%%%%%%%%%%%

\Exercises

\begin{exercise}[$1$-forms vanishing on loops]
  \label{1-forms-vanishing-on-loops}
  Let $\X$ be a diffeological space.
  Show that if the integral of a differential $1$-form $\alpha$ vanishes on every loop,
  then the form is closed.
  Use the Stokes theorem \art{The-Stokes-theorem} and the fact that a $2$-form is characterized by its values on the $2$-plots \art{The-k-forms-are-defined-by-the-k-plots}.
  Note that the form $\alpha$ is actually exact thanks to \art{Closed-1-forms-vanishing-on-loops}.
\end{exercise} %% Closed-1-forms-vanishing-on-loops

\begin{exercise}[Forms on irrational tori are closed]
  \label{Forms-on-irrational-tori-are-closed}
  Show that every differential form on an irrational torus $\Torus_\Gamma = \RR^n/\Gamma$,
  where $\Gamma$ is a dense discrete generating subgroup of $\RR^n$,
  $n \geq 1$,
  is closed.
  Deduce that $\HDR^p({\T}_{\Gamma}) \simeq \Lambda^p(\RR^n)$.
\end{exercise} %% Forms-on-irrational-tori-are-closed

\begin{exercise}[Is the group $\Diff(\S^1)$ simply connected?]
  \label{Is-the-group-Diff-S-1-simply-connected}
  Let $\S^1 \subset \RR^2$ be the subspace of unit vectors and consider the group $\Diff(\S^1)$ equipped with the functional diffeology.
  Let $\X \in \Cinfty(\RR,\S^1)$ and $\J \in \GL(\RR^2)$,
  defined by
  $$%
    \theta \mapsto \X(\theta) = \vect{\cos(\theta) \\ \sin(\theta)}, \text{ and } \J =
    \begin{pmatrix}
    0 & -1 \\
    1 & \phantom{-}0
    \end{pmatrix}.
    $$%
  For every $n$-plot $\P : \U \to \Diff(\S^1)$,
  for every $r \in \U$ and for every $\delta r \in \RR^n$,
  let
  $$%
    \alpha(\P)_r(\delta r) = \int_0^{2\pi} \bigg\langle \J [\P(r)(\X(\theta))] , {\partial \P(r)(\X(\theta)) \over \partial r}(\delta r) \bigg\rangle \ d\theta,
    $$%
  where
  \begin{eqnarray*}
    {\partial \P(r)(\X(\theta)) \over \partial r}(\delta r) & = & \D(r \mapsto \P(r)(\X(\theta)))(r)(\delta r) \\
    & = & \lim_{t \mathop{\rightarrow} 0}{\P(r + t \delta r)(\X(\theta)) - \P(r)(\X(\theta)) \over t},
  \end{eqnarray*}
  and the difference is computed in $\RR^2$.
  The brackets denote the standard scalar product on $\RR^2$.
  
  \Question{1)}~Check that $\alpha$ is a differential $1$-form on $\Diff(\S^1)$,
  and compute $d\alpha$.
  
  \Question{2)}~Compute the integral of $\alpha$ on the following loop in $\Diff(\S^1)$
  $$%
    \sigma : t \mapsto
    \begin{pmatrix}
    \cos(2\pi t) & -\sin(2\pi t) \\
    \sin(2\pi t) & \phantom{-}\cos(2\pi t)
    \end{pmatrix}.
    $$%
  
  \Question{3)}~Deduce that the identity component of $\Diff(\S^1)$ is not simply connected.
\end{exercise} %% Is-the-group-Diff-S-1-simply-connected

%%%%%%%%%%%%%%%%%%%%%%%%%%%%%%%%%%%%%%%%%%%%%%%%%%%%%%%%%%
%% MARK: Chain-Homotopy Operator
%%%%%%%%%%%%%%%%%%%%%%%%%%%%%%%%%%%%%%%%%%%%%%%%%%%%%%%%%%

\section*{Chain-Homotopy Operator}
\label{Section-Chain-Homotopy-Operator}

\begin{sechead}
  The {\em Chain-Homotopy operator} on a diffeological space $\X$ is a smooth linear map $\CHK : \DForms^p(\X) \to \DForms^{p-1}(\Paths(\X))$ which satisfies $\CHK \circ d\,+ d\, \circ \CHK = \1^* -\0^*$ \art{The-Chain-Homotopy-operator-K},
  with $\Paths(\X) = \Cinfty(\RR,\X)$,
  $\0$ and $\1$ map a path $\gamma$ to its source $\0(\gamma) = \gamma(0)$ and its target $\1(\gamma) = \gamma(1)$.
  Since the space of paths in $\X$ is naturally a diffeological space,
  it is legitimate to consider differential forms on $\Paths(\X)$ and its subspaces,
  it is the ability of the diffeological approach to stay in the same category and avoid parallel and tedious constructions.
  The Chain-Homotopy operator has multiple applications,
  for instance,
  it leads to the homotopic invariance of the De Rham cohomology \art{Homotopic-invariance-of-the-De-Rham-cohomology},
  and it is crucial for the construction of the moment map \mychap{Chapter-Symplectic-Diffeology}.
\end{sechead}

\begin{article}\artlabel{Integration operator of forms along paths}
  \addcontentsline{toc}{section}{\small\hspace{10pt} Integration operator of forms along paths}
  \label{Integration-operator-of-forms-along-paths}
  Let $\X$ be a diffeological space and let $\Paths(\X) = \Cinfty(\RR,\X)$ be the space of smooth paths in $\X$,
  equipped with the functional diffeology \art{Functional-diffeologies}.
  Let us consider,
  for each $t \in \RR$, the {\em evaluation map} of paths at the point $t$,
  that is,
  $$%
    {\bf t}: \Paths(\X) \to \X, \text{ with ${\bf t}(\gamma) = \gamma(t)$}.
    $$%
  The map ${\bf t}$ is a smooth map,
  this is an immediate consequence of the definition of the functional diffeology.
  
  1. We call the {\em integration operator}\index{Integration operator} the map $\Phi : \DForms^p(\X) \to \DForms^p(\Paths(\X))$ defined,
  for all integers $p >0$,
  by:
  $$%
    \text{For all } \alpha \in  \DForms^p(\X), \  \Phi(\alpha) = \int_0^1 {\bf t}^*(\alpha) \dt.
    $$%
  It maps any differential $p$-form $\alpha$ on $\X$ to the $p$-form on $\Paths(\X)$ obtained by integrating $\alpha$ along the paths.
  Precisely,
  let $\P : \U \to \Paths(\X)$ be an $n$-plot,
  let $r \in \U$,
  and let $v = (v_1 \cdots v_p)$ denote $p$ vectors of $\RR^n$.
  Then,
  $$%
    \Phi(\alpha)({\P})(r)(v) = \int_0^1\alpha({\bf t} \circ {\P})(r)(v) \dt, \  {\bf t} \circ {\P} = [r \mapsto {\P}(r)(t)].
    $$%
  
  2. The integration operator $\Phi$ is linear.
  For any two $p$-forms $\alpha$ and $\alpha'$ on $\X$,
  and for all $s \in \RR$,
  $$%
    \Phi(\alpha+\alpha') = \Phi(\alpha) + \Phi(\alpha'), \text{ and } \Phi(s\alpha) = s\times \Phi(\alpha).
    $$%
  
  3. The integration operator $\Phi$ is smooth,
  $\Phi \in \Cinfty(\Omega^p(\X),\Omega^p(\Paths(\X)))$.
\end{article} %% Integration-operator-of-forms-along-paths

\begin{proof}
  1. Let us check first that $\Phi(\alpha)$ is a well defined $p$-form on $\Paths(\X)$.
  Let ${\P}: \U \to \X$ be a plot and let $\F \in \Cinfty(\V, \U)$,
  where $\V$ is some real domain.
  Then,
  \begin{multline*}
    \Phi(\alpha)({\P} \circ \F)  =  \int_0^1\alpha({\bf t} \circ {\P} \circ \F) \dt =  \int_0^1 \F^*(\alpha({\bf t} \circ {\P})) \dt  \\
    = \F^*\left(\int_0^1 \alpha({\bf t} \circ {\P}) \dt \right) = \F^*(\Phi(\alpha)({\P})).
  \end{multline*}
  
  2. Now,
  let us check that the integration operator $\Phi$ is linear.
  Let $\alpha$ and $\alpha'$ be any two $p$-forms of $\X$.
  Let ${\P}_t = {\bf
  t} \circ {\P} $, we have
  \begin{eqnarray*}
    \Phi(\alpha + \alpha') & = & [\P \mapsto \int_0^1 (\alpha + \alpha')(\P_t) \dt] \\
    & = & \int_0^1 \alpha(\P_t) \dt + \int_0^1 \alpha'(\P_t) \dt] \\
    & = & \Phi(\alpha) + \Phi(\alpha').
  \end{eqnarray*}
  And,
  for all $s \in \RR$,
  $$%
    \Phi(s\alpha) = [\P \mapsto  \int_0^1 s\alpha(\P_t) \dt] = [\P \mapsto  s \int_0^1 \alpha(\P_t) \dt] = s\times \Phi(\alpha).
    $$%
  
  3. Since the map ${\bf t}: \Paths(\X) \to \X$ is smooth and since integration preserves smoothness,
  the integration operator $\Phi$ is a smooth linear map from $\DForms^p(\X)$ to $\DForms^p(\Paths(\X))$.
\end{proof}

\begin{article}\artlabel{The operator $\Phi$ is a morphism of De Rham complex}
  \addcontentsline{toc}{section}{\small\hspace{10pt} The operator $\Phi$ is a morphism of the De Rham complex}
  \label{The-operator-Phi-is-a-morphism-of-De-Rham-complex}
  The integration operator $\Phi$,
  of a diffeological space $\X$ \art{Integration-operator-of-forms-along-paths},
  is a morphism from the De Rham complex of $\X$ to the De Rham complex of $\Paths(\X)$,
  that is,
  $$%
    d \circ \Phi = \Phi \circ d.
    $$%
  This is summarized by the following commutative diagram,
  with $p>0$.
  
  \begin{center}
    \begin{tikzcd}[column sep=large, row sep=large, every label/.append style = {font = \small}]
      \DForms^p(\X) \arrow[d, swap, "d"] \arrow[r,"\Phi"] & \DForms^p(\Paths(\X)) \arrow[d, "d"]  \\
      \DForms^{p+1}(\X) \arrow[r, swap, "\Phi"] & \DForms^{p+1}(\Paths(\X))
    \end{tikzcd}
  \end{center}
  
\end{article} %% The-operator-Phi-is-a-morphism-of-De-Rham-complex

\begin{proof}
  Let $\alpha$ be a $p$-form on $\X$,
  $p>0$.
  Then,
  \begin{multline*}
    \Phi(d\alpha)({\P}) = \int_0^1 (d\alpha)(\P_t) \dt \\
    = \int_0^1 d[\alpha(\P_t)] \dt = d \left(\int_0^1 \alpha(\P_t) \dt \right) = d (\Phi(\alpha)({\P})).
  \end{multline*}
  Thus,
  $\Phi(d\alpha) = d(\Phi(\alpha))$.
\end{proof}

\begin{article}\artlabel{Variance of the integration operator $\Phi$}
  \addcontentsline{toc}{section}{\small\hspace{10pt} Variance of the integration operator $\Phi$}
  \label{Variance-of-the-integration-operator-Phi}
  Let $\X$ and $\X'$ be two diffeological spaces,
  and let $f: \X \to \X'$ be a smooth map.
  The map $f$ induces a smooth map on the spaces of paths,
  $$%
    \fPaths{f}: \Paths(\X) \to \Paths(\X'), \text{ with } \fPaths{f}(\gamma) = f \circ \gamma.
    $$%
  The map $f$ also induces the two pullbacks
  $$%
    f^*: \DForms^\star(\X') \to \DForms^\star(\X),\text{ and } \fPaths{f}^*: \DForms^\star(\Paths(\X')) \to \DForms^\star(\Paths(\X)).
    $$%
  Let  $\Phi_\X$ and $\Phi_{\X'}$ be the two associated integration operators \art{Integration-operator-of-forms-along-paths},
  then
  $$%
    \Phi_\X \circ f^* = \fPaths{f}^* \circ \Phi_{\X'}.
    $$%
  This is summarized by the following commutative diagram.
  
  \begin{center}
    \begin{tikzcd}[column sep=large, row sep=large, every label/.append style = {font = \small}]
      \DForms^p(\X') \arrow[d, swap, "f^*"] \arrow[r,"\Phi_{\X'}"] & \DForms^p(\Paths(\X')) \arrow[d, "\fPaths{f}^*"]  \\
      \DForms^p(\X) \arrow[r, swap, "\Phi_\X"] & \DForms^p(\Paths(\X))
    \end{tikzcd}
  \end{center}
\end{article} %% Variance-of-the-integration-operator-Phi

\begin{proof}
  Let $\alpha$ be a differential $p$-form on $\X'$ and $\P : \U \to \X$ be a plot.
  Let us denote ${\P}_t = {\bf t} \circ {\P} $.
  On the one hand we have
  $$%
    [(\Phi_\X \circ f^*)(\alpha)]({\P}) = [\Phi_\X(f^*(\alpha))]({\P}) =  \int_0^1 f^*(\alpha)(\P_t) \dt =  \int_0^1 \alpha(f \circ {\P}_t) \dt,
    $$%
  and on the other hand
  $$%
    [(\fPaths{f}^* \circ \Phi_{\X'})(\alpha)]({\P}) = [\fPaths{f}^*(\Phi_{\X'}(\alpha))]({\P}) = [\Phi_{\X'}(\alpha)](\fPaths{f} \circ {\P})=\int_0^1 \alpha[(\fPaths{f} \circ {\P})_t] \dt.
    $$%
  But,
  for all $r \in \U$,
  $(f \circ {\P}_t)(r) = f({\P}_t(r)) = f({\P}(r)(t))$,
  and $(\fPaths{f} \circ {\P})_t(r) = (\fPaths{f} \circ {\P})_t(r) = f({\P}(r)(t))$.
  Hence,
  $f \circ {\P}_t = (\fPaths{f} \circ {\P})_t$,
  thus $\Phi_\X \circ f^* = \fPaths{f}^* \circ \Phi_{\X'}$.
\end{proof}

\begin{article}\artlabel{Derivation along time reparametrization}
  \addcontentsline{toc}{section}{\small\hspace{10pt} Derivation along time reparametrization}
  \label{Derivation-along-time-reparametrization}
  Let $\X$ be a diffeological space and $\Paths(\X)$ be the space of its smooth paths,
  equipped with the functional diffeology.
  The group of translations $(\RR,+)$ acts by reparametrization on $\Paths(\X)$,
  as a $1$-parameter group of diffeomorphisms.
  Let us denote by $\tau$ this action.
  For all $\gamma \in \Paths(\X)$ and for  all $e \in \RR$,
  $$%
    \tau(e): \gamma \mapsto \gamma \circ {\T}_e, \text{ with } {\T}_e: t \mapsto t+e.
    $$%
  Let $\alpha$ be a $p$-form of $\X$,
  the Lie derivative \art{The-Lie-derivative-of-differential-forms} of the $p$-form $\Phi(\alpha)$ by the $1$-parameter group $\tau$ satisfies the identity
  $$%
    \DLie_\tau (\Phi(\alpha)) = \1^*\alpha -\0^*\alpha.
    $$%
\end{article} %% Derivation-along-time-reparametrization

\begin{proof}
  Let us check first that $\tau : e \mapsto [\gamma \mapsto \gamma \circ \T_e]$ is a smooth homomorphism from $(\RR,+)$ into $\Diff(\Paths(\X))$.
  Indeed,
  for all $e \in \RR$, $\gamma' = \tau(e)(\gamma)$ implies $\gamma = \tau(-e)(\gamma')$,
  $\tau(e)^{-1} = \tau(-e)$.
  For all $e \in \RR$, $\tau(e)$ (and thus $\tau(e)^{-1}$) is smooth by the very definition of the functional diffeology,
  and $\tau$ is clearly a homomorphism,
  $\tau(e+e') = \tau(e) \circ \tau(e')$,
  thus $\tau$ is a $1$-parameter group of diffeomorphisms of $\Paths(\X)$.
  
  Now,
  let us denote $\balpha = \Phi(\alpha)$,
  for every plot ${\P}$ de $\Paths(\X)$,
  $$%
    [\DLie_\tau\balpha]({\P}) = {\partial \over \partial t}\bigg\{[\tau(t)^*\balpha]({\P})\bigg\}_{t = 0} = {\partial \over \partial t}\bigg\{\balpha(\tau(t) \circ \P) \bigg\}_{t = 0}.
    $$%
  But $\tau(t) \circ {\P}: r \mapsto {\P}(r) \circ {\T}_t$,
  then
  $$%
    \balpha [\tau(t) \circ {\P}] = \int_0^1 \alpha[(\tau(t) \circ {\P})_s] \ds = \int_0^1 \alpha[r \mapsto {\P}(r)(t+s)] \ds.
    $$%
  Let $u = t+s$,
  we have
  $$%
    \balpha[\tau(t) \circ {\P}] = \int_t^{1+t} \alpha[r \mapsto  {\P}(r)(u)] \du.
    $$%
  After derivation we get,
  for all plots $\P$ of $\X$,
  $[\DLie_\tau\balpha]({\P}) = \alpha[r \mapsto {\P}(r)(1)] -$ %\alpha[r \mapsto {\P}(r)(0)] = [\1^*\alpha-\0^*\alpha]({\P})$
  \linebreak
  $\alpha[r \mapsto {\P}(r)(0)] = [\1^*\alpha-\0^*\alpha]({\P})$,
  that is,
  $\DLie_\tau (\Phi(\alpha)) = \1^*\alpha -\0^*\alpha$.
\end{proof}

\begin{article}\artlabel{Variance of the time reparametrization}
  \addcontentsline{toc}{section}{\small\hspace{10pt} Variance of time reparametrization}
  \label{Variance-of-time-reparametrization}
  Let $\X$ and $\X'$ be two diffeological spaces and $f: \X \to \X'$ be a smooth map.
  Let $\fPaths{f}: \Paths(\X) \to \Paths(\X')$ be the action of $f$,
  on the paths,
  defined in \art{Variance-of-the-integration-operator-Phi}.
  Let $\fPaths{f}^*: \DForms^p(\Paths(\X')) \to \DForms^p(\Paths(\X))$ be the induced action of $\fPaths{f}$ at the level of $p$-forms.
  Let $\tau$ and $\tau'$ denote the action $(\RR,+)$ on $\Paths(\X)$ and $\Paths(\X')$,
  as defined in \art{Derivation-along-time-reparametrization}.
  Let $i_\tau$ and $i_{\tau'}$ be the contractions associated with these $1$-parameter groups of diffeomorphisms \art{Contracting-differential-forms-on-slidings}.
  Then,
  $i_{\tau} \circ \fPaths{f}^* = \fPaths{f}^* \circ i_{\tau'}.$
  %
  \begin{center}
    \begin{tikzcd}[column sep=large, row sep=large, every label/.append style = {font = \small}]
      \DForms^p(\Paths(\X')) \arrow[d, swap, "i_{\tau'}"] \arrow[r,"\fPaths{f}^*"] & \DForms^p(\Paths(\X)) \arrow[d, "i_\tau"]  \\
      \DForms^{p-1}(\Paths(\X')) \arrow[r, swap, "\fPaths{f}^*"] & \DForms^{p-1}(\Paths(\X))
    \end{tikzcd}
  \end{center}
\end{article} %% Variance-of-time-reparametrization

\begin{proof}
  Let $\beta$ be a $p$-form on $\Paths(\X')$,
  ${\P}: \U \to \Paths(\X)$ be an $n$-plot,
  $r \in \U$ and $v$ represents $(p-1)$ vectors of $\RR^n$.
  By definition of the contraction of a $p$-form by a $1$-parameter group of diffeomorphisms \art{Contracting-differential-forms-on-slidings},
  we have
  \begin{eqnarray*}
    [(i_{\tau} \circ \fPaths{f}^*)(\beta)]({\P})_r(v) & = & [i_{\tau}(\fPaths{f}^*(\beta))]({\P})_r(v) \\
    & = & \fPaths{f}^*(\beta)(\tau\cdot {\P})_{0 \choose r} \vect{1 \\ 0} \vect{0 \\ v} \\
    & = & \beta(\fPaths{f} \circ (\tau\cdot {\P}))_{0 \choose r} \vect{1 \\ 0}\vect{0 \\ v},
  \end{eqnarray*}
  where $\tau\cdot {\P}(t,r) = \tau(t)({\P}(r))$.
  But
  \renewcommand{\theequation}{$\clubsuit$}
  \begin{eqnarray}
    [\fPaths{f} \circ (\tau\cdot {\P})](t,r) & = & \fPaths{f}((\tau\cdot {\P})(t,r)) \nonumber \\
    & = & \fPaths{f}(\tau(t)({\P}(r)) \nonumber \\
    & = & \fPaths{f}[s \mapsto \tau(t)({\P}(r))(s)]\nonumber \\
    & = & \fPaths{f}[s \mapsto {\P}(r)(s+t)] \nonumber \\
    & = & [s \mapsto f({\P}(r)(s+t))].
  \end{eqnarray}
  On the other hand,
  \begin{eqnarray*}
    [(\fPaths{f}^* \circ i_{\tau'})(\beta)]({\P})_r(v) & = & [\fPaths{f}^*(i_{\tau'}(\beta))]({\P})_r(v) \\
    & = & [(i_{\tau'}(\beta)(\fPaths{f} \circ {\P})]_r(v) \\
    & = & \beta(\tau'\cdot(\fPaths{f} \circ {\P}))_{(0,r)} \vect{1 \\ 0}\vect{0 \\ v}.
  \end{eqnarray*}
  But
  \renewcommand{\theequation}{$\spadesuit$}
  \begin{eqnarray}
    [\tau'\cdot(\fPaths{f} \circ {\P})](t,r) & = & \tau'(t)(\fPaths{f} \circ {\P}(r))\nonumber \\
    & = & [s \mapsto (\fPaths{f} \circ {\P}(r))(s+t)], \nonumber \\
    \left[\tau'\cdot(\fPaths{f} \circ {\P})\right] & = & [s \mapsto f({\P}(r)(s+t))].
  \end{eqnarray}
  Now, comparing $(\clubsuit)$ and $(\spadesuit)$, we get
  $$%
    [(i_{\tau} \circ \fPaths{f}^*)(\beta)]({\P})_r(v)
    = [(\fPaths{f}^* \circ i_{\tau'})(\beta)]({\P})_r(v),
    $$%
  that is, $i_{\tau} \circ \fPaths{f}^* = \fPaths{f}^* \circ i_{\tau'}$.
\end{proof}

\begin{article}\artlabel{The Chain-Homotopy operator $\CHK$}
  \addcontentsline{toc}{section}{\small\hspace{10pt} The Chain-Homotopy operator $\CHK$}
  \label{The-Chain-Homotopy-operator-K}
  Let $\X$ be a diffeological space,
  let $\Phi$ be the integration operator defined in \art{Integration-operator-of-forms-along-paths},
  let $\tau$ be the time-reparametrization defined in \art{Derivation-along-time-reparametrization},
  and let $i_\tau$ be the contraction by the sliding $\tau$ \art{Contracting-differential-forms-on-slidings}.
  The operator $\CHK: \DForms^p(\X) \to \DForms^{p-1}(\Paths(\X))$ defined,
  for all $p>0$,
  by
  \begin{equation}
    \renewcommand{\theequation}{$\spadesuit$}
    \label{defOH}
    \CHK =  i_\tau \circ \Phi,
  \end{equation}
  satisfies
  \begin{equation}
    \renewcommand{\theequation}{$\clubsuit$} \label{ophom}
    \CHK \circ d +d  \circ \CHK = \1^* - \0^*.
  \end{equation}
  The operator $\CHK$ will be called the {\em Chain-Homotopy operator}\index{Chain-Homotopy operator},
  it is smooth and linear,
  $$%
    \CHK \in \DLin(\DForms^p(\X), \DForms^{p-1}(\Paths(\X))).
    $$%
  Let $\alpha$ be a $p$-form of $\X$,
  with $p>1$,
  and $\P : \U \to \Paths(\X)$ be an $n$-plot.
  The value of $\CHK\!\alpha$ on the plot $\P$,
  at the point $r \in \U$,
  evaluated on $(p-1)$ vectors $v_2, \ldots, v_{p}$ of $\RR^n$,
  is explicitly given by
  \begin{equation}
    \renewcommand{\theequation}{$\diamondsuit$}
    (\CHK\!\alpha)(\P)_r(v_2) \cdots (v_{p}) = \int_0^1 \alpha \left( \vect{t \\ r} \mapsto \P(r)(t) \right)_{t \choose r} \vect{1 \\ 0} \vect{0 \\ v_2} \cdots \vect{0 \\ v_{p}} \dt.
  \end{equation}
  For $p=1$, see \art{The-Chain-Homotopy-operator-for-p-1}.
\end{article} %% The-Chain-Homotopy-operator-K

\begin{proof}
  Let $\alpha$ be a $p$-form of $\X$,
  $p>0$.
  On the one hand \art{Derivation-along-time-reparametrization} we have
  $$%
    \DLie_\tau (\Phi(\alpha)) = \1^*\alpha - \0^*\alpha,
    $$%
  and on the other hand,
  applying the Cartan formula \art{The-Cartan-Lie-formula} and the commutation $d \circ \Phi = \Phi \circ d$ \art{The-operator-Phi-is-a-morphism-of-De-Rham-complex},
  we have
  \begin{eqnarray*}
    \DLie_\tau (\Phi(\alpha))& = & d[i_\tau(\Phi (\alpha))]+i_\tau (d[\Phi (\alpha)]) \\
    & = & d[i_\tau\Phi (\alpha)]+i_\tau \Phi [d\alpha] \\
    & = & d[\CHK(\alpha)] + \CHK(d\alpha).
  \end{eqnarray*}
  Hence, $d[\CHK(\alpha)] + \CHK(d\alpha) = \1^*\alpha - \0^*\alpha$,
  that is,
  $\CHK \circ d +d  \circ \CHK = \1^* - \0^*$.
  Now,
  because the contraction operation and the map $\Phi$ are smooth \art{Contraction-of-differential-forms},
  \art{Integration-operator-of-forms-along-paths},
  the Chain-Homotopy operator $\CHK$ is a smooth linear map.
  A direct application of all these successive definitions gives
  $$%
    (\CHK\!\alpha)(\P)_r(v_2) \cdot (v_{p}) = \int_0^1 \alpha \left( \vect{s \\ r} \mapsto \P(r)(s + t) \right)_{0 \choose r} \vect{1 \\ 0} \vect{0 \\ v_2} \cdot \vect{0 \\ v_{p}} \dt,
    $$%
  which becomes $(\diamondsuit)$ after the change of variable $s \mapsto s + t$.
\end{proof}

\begin{article}\artlabel{Variance of the Chain-Homotopy operator}
  \addcontentsline{toc}{section}{\small\hspace{10pt} Variance of the Chain-Homotopy operator}
  \label{Variance-of-the-Chain-Homotopy-operator}
  Let $\X$ et $\X'$ be two diffeological spaces.
  Let $f: \X \to \X'$ be a smooth map.
  Let us use the notations of the proposition \art{Variance-of-the-integration-operator-Phi} and let $\CHK_\X$ et $\CHK_{\X'}$ be the two Chain-Homotopy operators of $\X$ and $\X'$.
  The variance of the Chain-Homotopy operators is given by
  $$%
    \CHK_\X \circ f^* = \fPaths{f}^* \circ \CHK_{\X'},
    $$%
  where $\fPaths{f}$ has been defined in \art{Variance-of-the-integration-operator-Phi},
  as the action of the function $f$ on the paths.
  This is summarized by the following commutative diagram.
  
  \begin{center}
    \begin{tikzcd}[column sep=large, row sep=large, every label/.append style = {font = \small}]
      \DForms^p(\X') \arrow[d, swap, "f^*"] \arrow[r,"\CHK_{\X'}"] & \DForms^{p-1}(\Paths(\X')) \arrow[d, "\fPaths{f}^*"]  \\
      \DForms^p(\X) \arrow[r, swap, "\CHK_\X"] & \DForms^{p-1}(\Paths(\X))
    \end{tikzcd}
  \end{center}
  
  \Note~Let $\Diff(\X,\alpha)$ be the {\em group of automorphisms}\index{Group of automorphisms} of $\alpha$,
  that is the group of diffeomorphisms of $\X$ preserving $\alpha$,
  $$%
    \Diff(\X,\alpha) = \{ f \in \Diff(\X) \mid f^*(\alpha) = \alpha \}.
    $$%
  As an application of the above proposition we get immediately that
  $$%
    f \in \Diff(\X,\alpha) \ \Rightarrow \ \fPaths{f} \in \Diff(\Paths(\X),\CHK\!\alpha).
    $$%
  Said with words,
  if a diffeomorphism $f$ of $\X$ preserves $\alpha$,
  then the action $\fPaths{f}$,
  of $f$ on $\Paths(\X)$, preserves $\CHK\!\alpha$.
\end{article} %% Variance-of-the-Chain-Homotopy-operator

\begin{proof}
  Let $\tau$ and $\tau'$ be the action of $(\RR,+)$ on $\Paths(\X)$ and $\Paths(\X')$,
  defined in \art{Derivation-along-time-reparametrization}.
  Let $\alpha \in \DForms^p(\X')$,
  by definition $\CHK_\X(f^*(\alpha)) =  (i_\tau \circ \Phi_\X)(f^*(\alpha)) =  i_{\tau}(\Phi_\X(f^*(\alpha)))$,
  but $\Phi_\X \circ f^* =  \fPaths{f}^* \circ \Phi_{\X'}$ \art{Variance-of-the-integration-operator-Phi},
  thus $\CHK_\X(f^*(\alpha)) =  i_{\tau}(\Phi_\X \circ f^*(\alpha)) =  i_{\tau}(\fPaths{f}^* \circ \Phi_{\X'}(\alpha)) = (i_{\tau} \circ \fPaths{f}^*)(\Phi_{\X'}(\alpha))$.
  Now,
  thanks to \art{Variance-of-time-reparametrization},
  $i_{\tau} \circ \fPaths{f}^* = \fPaths{f}^* \circ i_{\tau'}$.
  Hence,
  $\CHK_\X(f^*(\alpha)) = (\fPaths{f}^* \circ i_{\tau'})(\Phi_{\X'}(\alpha)) =  \fPaths{f}^*(i_{\tau'} \circ \Phi_{\X'}(\alpha)) =  \fPaths{f}^*(\CHK_{\X'}(\alpha))$.
  Therefore,
  $\CHK_\X \circ f^* = \fPaths{f}^* \circ \CHK_{\X'}.$
\end{proof}

\begin{article}\artlabel{Chain-Homotopy and paths concatenation} 
  \addcontentsline{toc}{section}{\small\hspace{10pt} Chain-Homotopy and paths concatenation}
  \label{Chain-Homotopy-and-paths-concatenation}
  Let $\X$ be a diffeological space and $\alpha$ be a differential $k$-form on $\X$,
  $k \geq 1$.
  Let $\gamma \in \stPaths(\X,x,x')$,
  the space of stationary paths,
  we consider the {\em preconcatenation} $\eL(\gamma)$ defined on $\stPaths(\X,x',\star)$ and the {\em postconcatenation} $\eR(\gamma)$ defined on $\stPaths(\X,\star,x)$,
  that is,
  $$%
    \eL(\gamma)(\gamma') = \gamma \vee \gamma', \text{ and } \eR(\gamma)(\gamma') = \gamma' \vee \gamma.
    $$%
  These operations preserve the $(k-1)$-form $\CHK\!\alpha$,
  where $\CHK$ is the Chain-Homotopy operator \art{The-Chain-Homotopy-operator-K},
  precisely,
  $$%
    \renewcommand{\arraystretch}{1.3}
    \left\{
    \begin{array}{lll}
    \eL(\gamma)^*(\CHK\!\alpha \restriction \stPaths(\X,x,\star)) & = & \CHK\!\alpha \restriction \stPaths(\X,x',\star)), \\
    \eR(\gamma)^*(\CHK\!\alpha \restriction \stPaths(\X,\star,x')) & = & \CHK\!\alpha \restriction \stPaths(\X,\star,x)).
    \end{array}
    \right.
    \renewcommand{\arraystretch}{1}
    $$%
\end{article} %% Chain-Homotopy-and-paths-concatenation

\begin{proof}
  Let $\P : \U \to \stPaths(\X,x',\star)$ be an $n$-plot.
  Let $r$ be a generic point in $\U$ and,
  in the following,
  as is usual now,
  let the surrounding square brackets represent $(k-1)$ vectors,
  for example $[\delta r] = (\delta r_2)\cdots (\delta r_k)$ with $\delta r_i \in \RR^n$, etc.
  We have,
  according to \xart{The-Chain-Homotopy-operator-K}{($\diamondsuit$)},
  \begin{eqnarray*}
    \renewcommand{\theequation}{$\bullet$}
    \eL(\gamma)^*(\CHK\!\alpha)(\P)_r[\delta r] & = & \CHK\!\alpha\,(\eL(\gamma) \circ \P)_r[\delta r]\\
    & =& \int_0^1 \alpha \left(\vect{t \\ r}  \mapsto \eL(\gamma) \circ \P(r)(t)\right)_{{t \choose r}}\vect{1 \\ 0} \sqvect{0 \\ \delta r} \ dt \\
    & = & \int_0^1 \alpha \left(\vect{t \\ r} \mapsto (\gamma \vee \P(r)) (t)\right)_{{t \choose r}}\vect{1 \\ 0} \sqvect{0 \\ \delta r} \ dt \\
    & = & \int_0^1 \alpha \left(\vect{t \\ r} \mapsto  {\left\{ \begin{array}{ll} \gamma(2t) & t \leq 1/2 \\ \P(r)(2t-1) & t \geq 1/2 \end{array}\!\! \right\}}\! \right)_{\!{t \choose r}}\!\vect{1 \\ 0}\! \sqvect{0 \\ \delta r} \, dt \\
    & = & \int_0^{1/2} \alpha \left(\vect{t \\ r} \mapsto \gamma(2t)\right)_{{t \choose r}}\vect{1 \\ 0} \sqvect{0 \\ \delta r} \, dt \\
    & + & \int_{1/2}^1 \alpha \left(\vect{t \\ r} \mapsto \P(r) (2t-1) \right)_{{t \choose r}}\vect{1 \\ 0} \sqvect{0 \cr \delta r} \, dt.
  \end{eqnarray*}
  The first term of the right-hand side of the last identity vanishes,
  since $\gamma (2t)$ does not depend on $r$,
  and after a change of variable $t' = 2t-1$, the second term gives
  
  \begin{eqnarray*}
    \eL(\gamma)^*(\CHK\!\alpha)(\P)_r[\delta r] & =& \int_0^1 \alpha \left(\vect{t' \\ r} \mapsto \P(r)(t')\right)_{{t' \choose r}}\vect{1 \\ 0} \sqvect{0 \\ \delta r} \, dt' \\
    & = & \CHK\!\alpha(\P)_r[\delta r].
  \end{eqnarray*}
  Thus, $\eL(\gamma)^*(\CHK\!\alpha)(\P)_r[\delta r] = \CHK\!\alpha(\P)_r[\delta r]$,
  and this proof applies {\em mutatis mutandis\/} to the postconcatenation operator $\eR(\gamma)$.
\end{proof}

\begin{article}\artlabel{The Chain-Homotopy operator for $p = 1$}
  \addcontentsline{toc}{section}{\small\hspace{10pt} The Chain-Homotopy operator for $p = 1$}
  \label{The-Chain-Homotopy-operator-for-p-1}
  Let $\X$ be a diffeological space.
  Let $\Paths(\X) = \Cinfty(\RR,\X)$ be the space of paths in $\X$,
  equipped with the functional diffeology.
  For every $1$-form $\alpha \in \DForms^1(\X)$, $\F_\alpha = \CHK(\alpha)$ belongs to $\DForms^0(\Paths(\X)) = \Cinfty(\Paths(\X),\RR)$,
  $$%
    \CHK : \DForms^1(\X) \to \Cinfty(\Paths(\X),\RR), \text{ and } \CHK(\alpha) = \F_\alpha = \left[ \gamma \mapsto \int_\gamma\alpha\right].
    $$%
  The function $\F_\alpha = \CHK(\alpha)$ can be extended,
  by linearity, over the whole space $\DChains{1}(\X)$ of $1$-chains of $\X$ \art{Cubes-and-cubic-chains-on-diffeological-spaces}\,:
  $$%
    \text{For all } \sum_\gamma n_\gamma \gamma \in \DChains{1}(\X), \  \F_\alpha\bigg(\sum_\gamma n_\gamma \gamma\bigg)
    = \sum_{\gamma} n_\gamma \F_\alpha(\gamma)
    = \sum_\gamma n_\gamma \int_\gamma \alpha.
    $$%
  Hence,
  for $p = 1$,
  the Chain-Homotopy operator is just the pairing of $1$-forms with $1$-chains \art{Pairing-chains-and-forms}.
  Moreover,
  if $\gamma$ and $\gamma'$ are two paths such that the concatenation $\gamma\vee\gamma'$ is defined \art{Pathwise-connectedness},
  then
  \begin{equation}
    \renewcommand{\theequation}{$\diamondsuit$}
    \F_\alpha(\gamma\vee\gamma') = \F_\alpha(\gamma + \gamma') =  \F_\alpha(\gamma) + \F_\alpha(\gamma').
  \end{equation}
  
  \Note~If $\gamma(1) = \gamma'(0)$ but the concatenation of $\gamma$ and $\gamma'$ is not smooth,
  then the smashed concatenation $\gamma\star\gamma' = \gamma^\star\vee\gamma'^\star$ is smooth \xart{Homotopy-of-paths}{Note},
  and also satisfies
  \begin{equation}
    \renewcommand{\theequation}{$\heartsuit$}
    \F_\alpha(\gamma \star\gamma') = \F_\alpha(\gamma + \gamma') = \F_\alpha(\gamma) + \F_\alpha(\gamma').
  \end{equation}
  In other words,
  $\F_\alpha = \CHK(\alpha)$ is a morphism from the magma\index{Magma} $(\Paths(\X),\star)$ to the Abelian group $(\RR,+)$.
\end{article} %% The-Chain-Homotopy-operator-for-p-1

\begin{proof}
  If $\gamma\vee\gamma'$ is a path in $\X$,
  then the identity $(\diamondsuit)$ is just the additivity of the integral,
  \begin{eqnarray*}
    \F_\alpha(\gamma\vee\gamma') & = & \int_0^1\alpha(\gamma\vee\gamma')(t) \dt \\
    & = & \int_0^{1/2}\alpha(s \mapsto \gamma(2s))(t) \dt + \int_{1/2}^1\alpha(s \mapsto \gamma'(2s-1))(t) \dt.
  \end{eqnarray*}
  After a suitable change of variable,
  we get
  $$%
    \F_\alpha(\gamma\vee\gamma') = \int_0^{1}\alpha(\gamma)(t) \dt + \int_{0}^1\alpha(\gamma')(t) \dt = \F_\alpha(\gamma) + \F_\alpha(\gamma').
    $$%
  Now,
  if we need to smash the paths $\gamma$ and $\gamma'$,
  we just remark that according to \art{Changing-the-coordinates-of-a-cube} we have
  $$%
    \F_\alpha(\gamma^\star) = \F_\alpha(\gamma \circ \lambda)
    = \int_{\eI} \alpha(\gamma \circ \lambda)
    = \int_{\eI} \lambda^*(\alpha(\gamma))
    = \int_{\lambda(\eI)} \alpha(\gamma)
    = \int_{\eI} \alpha(\gamma)
    = \F_\alpha(\gamma),
    $$%
  where $\lambda$ is the smashing function described in \art{Pathwise-connectedness},
  satisfying $\lambda(\eI) = \eI$.
  Therefore,
  $\F_\alpha(\gamma \star \gamma') = \F_\alpha(\gamma) + \F_\alpha(\gamma')$.
\end{proof}

\begin{article}\artlabel{The Chain-Homotopy operator for manifolds}
  \addcontentsline{toc}{section}{\small\hspace{10pt} The Chain-Homotopy operator for manifolds}
  \label{The-Chain-Homotopy-operator-for-manifolds}
  Let $\M$ be a manifold of finite dimension \art{Manifolds-as-diffeologies}.
  The Chain-Homotopy operator $\CHK$ \art{The-Chain-Homotopy-operator-K} of $\M$ can be expressed in terms of path integral.
  Let $\P : \U \to \Paths(\M)$ be an $n$-plot.
  Let us denote, for any $r \in \U$ and for any vector $\delta_i r \in \RR^n$,
  $$%
    \gamma_r = {\P}(r): t \mapsto \gamma_r(t), \text{ and } \delta \gamma_r: t \mapsto \D[r \to \gamma_r(t)](r)(\delta r).
    $$%
  Let $\alpha \in \Omega^p(\M)$, the real $\CHK\!\alpha({\P})(r)(\delta_2 r)\cdots(\delta_{p}r)$ can be interpreted as $\CHK\!\alpha$,
  computed at the point $\gamma_r$ and applied to the $(p-1)$ {\em variations} $\delta_i\gamma_r$ associated with the $(p-1)$ vectors $\delta_i r$.
  It can be written as follows:
  $$%
    \CHK\!\alpha_{\gamma_r}(\delta_2\gamma_r)\cdots (\delta_{p} \gamma_r) = \int_0^1\alpha_{\gamma_r(t)}\bigg({d\,\gamma_r(t) \over dt}\bigg) (\delta_2 \gamma_r(t))\cdots(\delta_{p}\gamma_r(t))\dt.
    $$%
\end{article} %% The-Chain-Homotopy-operator-for-manifolds

%%%%%%%%%%%%%%%%%%%%%%%%%%%%%%%%%%%%%%%%%%%%%%%%%%%%%%%%%%
%% MARK: Homotopy and Differential Forms, Poincar\'e's Lemma
%%%%%%%%%%%%%%%%%%%%%%%%%%%%%%%%%%%%%%%%%%%%%%%%%%%%%%%%%%

\section*{Homotopy and Differential Forms, Poincar\'e's Lemma}
\label{Section-Homotopy-and-differential-forms}

\begin{sechead}
  One of the crucial properties of De Rham cohomology is its homotopic invariance:
  the mapping induced in De Rham cohomology by the pullback of a smooth map depends only on the homotopy class of the map.
  The equivalent theorem in classical differential geometry,
  for the De Rham cohomology of finite dimensional manifolds,
  can be regarded as a specialization of general diffeological theory.
  The transition through the space of paths,
  which is impossible in the restricted category of manifolds,
  offers a great simplification of this theorem,
  even for the sole case of manifolds.
  The main tool used in this section is the Chain-Homotopy operator defined in the previous section \art{The-Chain-Homotopy-operator-K}.
  This proves that the nature of this invariance dwells deeply in the diffeological structure.
\end{sechead}

\begin{article}\artlabel{Homotopic invariance of the De Rham cohomology}
  \addcontentsline{toc}{section}{\small\hspace{10pt} Homotopic invariance of the De Rham cohomology}
  \label{Homotopic-invariance-of-the-De-Rham-cohomology}
  Let $\X$ and $\X'$ be two diffeological spaces and $f \in \Cinfty(\X,\X')$.
  Since the exterior derivative commutes with the pullback \art{Exterior-derivative-of-forms},
  the action of $f$ passes from the differential forms to the De Rham cohomology,
  as a linear action denoted by $f^*_{\rm dR}$.
  For all $\alpha' \in \DForms^\star(\X')$,
  $$%
    f^*_{\rm dR}(\class(\alpha')) = \class(f^*(\alpha')), \text{ and } f^*_{\rm dR} \in \DLin(\HDR^\star(\X'),\HDR^\star(\X)).
    $$%
  Let $s \mapsto f_s$ be a {\em homotopy of smooth maps} from $\X$ to $\X'$ and let $\alpha'$ be a closed differential form on $\X'$,
  that is,
  $$%
    [s \mapsto f_s] \in \Paths(\Cinfty(\X,\X'))
    \text{ and }
    \alpha' \in \DForms^p(\X'), \text{ with } d\alpha' = 0.
    $$%
  Let $f_0$ and $f_1$ be the ends of this path.
  Then,
  there exists a differential form $\beta \in \DForms^{p-1}(\X)$ such that
  $$%
    f_1^*(\alpha') = f_0^*(\alpha') + d\beta, \text{ and } {f^*_0}_{\rm dR} =  {f^*_1}_{\rm dR}.
    $$%
  In other words,
  the pullback $f^*_{\rm dR} : \HDR^\star(\X') \to \HDR^\star(\X)$ by smooth maps depends only on the homotopy class of $f$ \art{Smooth-maps-and-connectedness}.
  We say that {\em $f_0$ and $f_1$ are equivalent in De Rham cohomology}.
\end{article} %% Homotopic-invariance-of-the-De-Rham-cohomology

\begin{proof}
  Let us consider the following map $\varphi$ from $\X$ to $\Paths(\X')$,
  $$%
    \begin{array}{ccc}
    \begin{array}{c}
    \begin{tikzcd}[column sep=large, row sep=large, every label/.append style = {font = \small}]
    \X  \arrow[r,"\varphi"] & \Paths(\X') \arrow[d, "\ends"]  \\
    {}  & \X' \times \X'
    \end{tikzcd}
    \end{array}
    & \text{defined by} &
    \begin{array}{c}
    \varphi: x \mapsto [s \mapsto f_s(x)],
    \end{array}
    \end{array}
    $$%
  where $\ends(\gamma') = (\gamma'(0), \gamma'(1))$.
  The map $\varphi$ is clearly smooth for the functional diffeology,
  $\varphi \in \Cinfty(\X,\Paths(\X'))$.
  Let us compose $\varphi^*$ with the identity satisfied by the Chain-Homotopy operator $\CHK \circ d + d \circ \CHK = \1^* - \0^*$ \art{The-Chain-Homotopy-operator-K}.
  Using the hypothesis $d\alpha' = 0$ and the commutativity between the exterior derivative and pullback \art{Exterior-derivative-of-forms},
  we have on the one hand
  $$%
    \varphi^*(\CHK d \alpha' + d\CHK\!\alpha') = \varphi^*(d\CHK\!\alpha') = d(\varphi^*(\CHK\!\alpha')),
    $$%
  and on the other hand
  \begin{eqnarray*}
    \varphi^* \circ (\1^*-\0^*)(\alpha) & = & \varphi^* \circ \1^*(\alpha)-\varphi^* \circ \0^*(\alpha) \\
    & =  & (\1\circ\varphi)\,^*(\alpha) - (\0\circ\varphi)\,^*(\alpha) \\
    & = & f_1^*(\alpha)-f_0^*(\alpha).
  \end{eqnarray*}
  Thus, $f_1^*(\alpha) = f_0^*(\alpha) + d\,\beta$, with $\beta = \varphi^*(\CHK\!\alpha')$.
\end{proof}

\begin{article}\artlabel{Closed $1$-forms vanishing on loops}
  \addcontentsline{toc}{section}{\small\hspace{10pt} Closed $1$-forms vanishing on loops}
  \label{Closed-1-forms-vanishing-on-loops}
  Let $\X$ be a connected diffeological space \art{Pathwise-connectedness}.
  A closed $1$-form $\alpha$,
  $\alpha \in \Omega^1(\X)$ and $d\alpha = 0$,
  is exact,
  $\alpha = df$,
  if and only if its integral vanishes on every loop,
  that is,
  $\int_\ell \alpha = 0$ for all $\ell \in \Loops(\X)$.
  The primitive $f$ can be chosen
  as the integral of $\alpha$ along any path connecting a chosen origin $\xo \in \X$ to the current point,
  which is summarized by
  $$%
    \alpha = df, \text{ with } f(x) = \int_\xo^x \alpha + \const.
    $$%
  
  \Note{1} This proposition looks like \art{Vanishing-De-Rham-1-cohomology} for which the hypothesis to be simply connected
  ---~which means that every loop is homotopic to a constant loop~---
  implies,
  by homotopic invariance,
  that the integral of the $1$-form vanishes on every loop,
  and thus satisfies the above condition.
  
  \Note{2} Let $\eta$ be the {\em first De Rham homomorphism},
  $$%
    \eta : \HDR^1(\X) \to \Hom(\pi_1(\X),\RR), \text{ with } \eta(\class(\alpha)) = \bigg[\class(\ell) \mapsto \int_\ell \alpha \bigg].
    $$%
  The map $\eta$ is well defined since the integral of a closed $1$-form on a loop depends only on the homotopy class of the loop and the cohomology class of the form.
  The proposition above can be reformulated as follows:
  ``The first De Rham homomorphism is injective'',%
  \footnote{The first De Rham homomorphism with values in the \v{C}ech cohomology is also injective \cite{PIZ21b}.}
  that is,
  $$%
    \ker (\eta) = \{0\}.
    $$%
\end{article} %% Closed-1-forms-vanishing-on-loops

\begin{proof}
  One way is a direct consequence of the Stokes theorem \art{The-Stokes-theorem}.
  If $\alpha = df$,
  then
  $$%
    \int_\ell\alpha = \int_\ell df = \int_{\Der \ell} f = f(\ell(1)) - f(\ell(0)) = 0.
    $$%
  Conversely,
  let us assume that the integral of $\alpha$ vanishes on any loop in $\X$\,:
  $$%
    \text{For all } \ell \in \Loops(\X), \ \int_\ell \alpha =  \int_0^1\alpha(\ell)_t(1) \dt = 0.
    $$%
  Let $\gamma$ and $\gamma'$ be two paths in $\X$ such that $\gamma(0) = \gamma'(0)$ and $\gamma(1) = \gamma'(1)$.
  Let $ \bar\gamma$ defined by $ \bar\gamma(t) = \gamma'(1-t)$,
  then the path $\ell = \gamma'\star \bar\gamma = \gamma'^\star \vee \bar\gamma^\star$ \xart{Homotopy-of-paths}{note} is a loop in $\X$ based in $\xo = \gamma(0)$.
  Hence,
  thanks to \art{The-Chain-Homotopy-operator-for-p-1},
  we have
  $$%
    \F_\alpha(\ell) = \F_\alpha(\gamma') - \F_\alpha(\gamma), \ \text{that is,} \ \F_\alpha(\gamma') = \F_\alpha(\gamma) + \F_\alpha(\ell).
    $$%
  Since,
  by hypothesis,
  $\F_\alpha(\ell) = 0$,
  $\F_\alpha(\gamma') = \F_\alpha(\gamma)$.
  Hence,
  there exists a real function $f: \X \times \X \to \RR$ such that $\F(\gamma) = f(x,x')$, where $x = \gamma(0)$ and $x' = \gamma(1)$.
  Then,
  since $\0\times\1: \gamma \mapsto (\gamma(0),\gamma(1))$ is a subduction \art{The-space-of-Paths-of-a-diffeological-space},
  the map $f$ is smooth.
  The application of the Chain-Homotopy operator gives
  $$%
    d \F_\alpha = d[\CHK(\alpha)] = \1^*(\alpha) - \0^*(\alpha) - \CHK[d(\alpha)],
    $$%
  but $d\alpha = 0$, thus $d \F_\alpha = \1^*(\alpha) - \0^*(\alpha)$.
  Restricting this identity to the subspace $\Paths(\X,\xo,\star)$ of paths with origin $\xo$,
  we get
  $$%
    d\F_\alpha \restriction \Paths(\X,\xo,\star) = \1^*(\alpha), \ \text{that is,} \ d(f_\xo \circ \1) = \1^*(df_\xo) = \1^*(\alpha),
    $$%
  where $f_{\xo}(x) = f(\xo,x)$.
  Then,
  since $\1$ is a subduction,
  $\alpha =  df_{x_0}$ \art{Pushing-forms-onto-quotients}.
  This is summarized by
  $$%
    f : x \mapsto \int_\xo^x \alpha + \const,
    $$%
  as the general solution of the equation $\alpha = df$.
\end{proof}

\begin{article}\artlabel{Closed forms on contractible spaces are exact}
  \addcontentsline{toc}{section}{\small\hspace{10pt} Closed forms on contractible spaces are exact}
  \label{Closed-forms-on-contractible-spaces-are-exact}
  As a corollary of the above proposition \art{Homotopic-invariance-of-the-De-Rham-cohomology},
  we get that every closed form on a contractible diffeological space $\X$ \art{Contractible-diffeological-spaces} is exact.
  For finite dimensional manifolds,
  this theorem is a variant of the {\em Poincar\'e lemma}.
\end{article} %% Closed-forms-on-contractible-spaces-are-exact

\begin{proof}
  Let $\rho$ be a {\em deformation retraction}\index{Deformation retraction} to the point $x_0 \in \X$ \art{Retractions-and-deformation-retracts},
  that is,
  $\rho \in \Paths(\Cinfty(\X,\X))$ such that $\rho(0) = [x \mapsto x_0]$ and $\rho(1) = \id_\X$.
  With the same notation as above \art{Homotopic-invariance-of-the-De-Rham-cohomology},
  for every closed $p$-form $\alpha$,
  $\rho(1)^*(\alpha) = \rho(0)^*(\alpha) + d\beta$,
  but $\rho(1)^*(\alpha) = \alpha$ and $\rho(0)^*(\alpha) = [x \mapsto x_0]^*(\alpha) = 0$.
  Therefore $\alpha = d\,\beta$.
\end{proof}

\begin{article}\artlabel{Closed forms on centered paths spaces}
  \addcontentsline{toc}{section}{\small\hspace{10pt} Closed forms on centered paths spaces}
  \label{Closed-forms-on-centered-paths-spaces}
  Let $\X$ be a diffeological space and $x_0 \in \X$ be some point.
  The space $\Paths(\X,x_0,\star)$ of pointed paths \art{The-space-of-Paths-of-a-diffeological-space} is a particular example of contractible space;
  see \exref{Contractible-space-of-paths}.
  Therefore,
  any closed form on $\Paths(\X,x_0,\star)$ is exact \art{Closed-forms-on-contractible-spaces-are-exact}.
  It is, in particular,
  the role of the Chain-Homotopy operator\index{Chain-Homotopy operator} \art{The-Chain-Homotopy-operator-K} to give a primitive for the pullback,
  by the end map,
  of a closed form on $\X$.
\end{article} %% Closed-forms-on-centered-paths-spaces

\begin{article}\artlabel{The Poincar\'e lemma}
  \addcontentsline{toc}{section}{\small\hspace{10pt} The Poincar\'e lemma}
  \label{The-Poincare-lemma}
  Let $\X$ be a diffeological space.
  As a corollary of \art{Closed-forms-on-contractible-spaces-are-exact},
  if $\X$ is locally contractible \art{Local-contractibility},
  then every closed p-form $\alpha$, $p>0$,
  is locally exact.
  For each point $x \in \X$ there exist a D-open neighborhood $\U$ of $x$ \art{The-D-Topology-of-diffeological-spaces} and a $(p-1)$-form $\beta$, defined on $\U$,
  such that $\alpha\restriction \U = d\beta$.
  This proposition extends the Poincar\'e lemma about integration of closed forms on star-shaped domains.
\end{article} %% The-Poincare-lemma

\begin{article}\artlabel{The Poincar\'e lemma for manifold}
  \addcontentsline{toc}{section}{\small\hspace{10pt} The Poincar\'e lemma for manifolds}
  \label{The-Poincare-lemma-for-manifolds}
  As a corollary of \art{The-Poincare-lemma},
  since manifolds are locally diffeomorphic to $\RR^n$ and since $\RR^n$ is locally contractible,
  every closed form of a finite dimensional manifold is locally exact.
  This is not the case with diffeological spaces which are not locally contractible,
  for example the irrational tori,
  see \exref{De-Rham-homomorphism-and-irrational-tori}.
\end{article} %% The-Poincare-lemma-for-manifolds

%%%%%%%%%%%%%%%%%%%%%%%%%%%%%%%%%%%%%%%%%%%%%%%%%%%%%%%%%%
%
%   Exercises
%
%%%%%%%%%%%%%%%%%%%%%%%%%%%%%%%%%%%%%%%%%%%%%%%%%%%%%%%%%%

\Exercises

\begin{exercise}[The Fubini-Study form is locally exact]
  \label{The-Fubini-Study-form-is-locally-exact}
  Show that the Fubini-Study form  $\omega$,
  defined on the infinite projective space in \exref{The-Fubini-Study-2-form},
  is locally exact.
\end{exercise} %% The-Fubini-Study-form-is-locally-exact

\begin{exercise}[Closed but not locally exact]
  \label{Closed-but-not-locally-exact}
  Check that all forms on the irrational tori are closed but not locally exact;
  see \exref{Forms-on-irrational-tori-are-closed}.
\end{exercise} %% Closed-but-not-locally-exact

\begin{exercise}[A morphism from $\HDR^\star(\X)$ to $\HDR^\star(\Diff(\X))$]
  \label{A-morphism-from-HX-to-HDiffX}
  Let $\X$ be a diffeological space,
  and let $\Diff(\X)$ be its group of diffeomorphisms equipped with the functional diffeology \art{Functional-diffeology-on-groups-of-diffeomorphisms}.
  Let $\hat x$ be the orbit map of the point $x \in \X$,
  that is,
  $\hat x : \Diff(\X) \to \X$ with $\hat x(\varphi) = \varphi(x)$.
  Check that the morphism $\hat x^*_{\rm dR}$,
  from $\HDR^\star(\X)$ to $\HDR^\star(\Diff(\X))$,
  induced by  $\hat x$ \art{Homotopic-invariance-of-the-De-Rham-cohomology},
  depends only on the connected component of $x$.
  For $\X = \S^1$,
  use the result of \exref{Covering-tori},
  to show that this morphism is injective.
  What does this example make you think of?
\end{exercise} %% A-morphism-from-HX-to-HDiffX
