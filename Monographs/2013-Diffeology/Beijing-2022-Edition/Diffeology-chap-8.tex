%%%%%%%%%%%%%%%%%%%%%%%%%%%%%%%%%%%%%%%%%%%%%%%%%%%%%%%%%%
%%
%% Fiber Bundles MARK: -
%%
%%%%%%%%%%%%%%%%%%%%%%%%%%%%%%%%%%%%%%%%%%%%%%%%%%%%%%%%%%

\chapter{Diffeological Fiber Bundles}

\label{Chapter-Diffeological-Fiber-Bundles}
\newcommand{\ChapterDFB}{Diffeological Fiber Bundles}

\begin{chaphead}
  Finding the right notion of fiber bundle for diffeology \cite{Igl85} has been a question raised by the study of the irrational torus $\Torus_\alpha$ \cite{DI83}.
  The direct computation of homotopy groups showed that the projection $\Torus^2 \to {\T}_{\alpha}$ behaves like a fibration but without being locally trivial,
  since ${\T}_{\alpha}$ inherits the coarse topology.
  So,
  it was necessary to adapt the notion of fiber bundle from classical differential geometry to diffeology in order to include such objects,
  regarded at this time as {\em singular},
  without losing the main properties of this theory.
  We shall see,
  later on,
  two equivalent definitions of {\em diffeological bundles}.
  One is pedestrian and operative,
  involving {\em local triviality along the plots} \art{Fibrations-and-local-triviality-along-the-plots},
  the other one involves {\em groupoid} \art{Diffeological-fibrations},
  is more  sophisticated and seems also more profound.
  And indeed,
  according to this definition, diffeological fiber bundles satisfy the {\em exact homotopy sequence} \art{Exact-homotopy-sequence-of-a-diffeological-fibration},
  one of their major properties.
  The category of diffeological fibrations also includes all the quotients of diffeological groups by any subgroup,
  and this is what explains {\em a posteriori} the homotopy of the irrational torus,
  computed previously by another method.
  In parallel,
  many classical constructions can be extended to this category:
  {\em pullbacks bundles}, {\em products bundles},
  {\em principal bundles},
  {\em associated bundles}, etc.
  Because of the wide category of diffeological groups and of the flexibility of diffeological fibrations,
  the notion of principal bundle in diffeology plays an even more important role than in classical differential geometry,
  if only for the fact that every diffeological fiber bundle is associated with a principal fiber bundle.
  To reconnect with homotopy in diffeology,
  {\em coverings} are defined as fiber bundles with (diffeologically) discrete fiber \art{What-is-a-covering},
  and the main property is that every diffeological space has a unique simply connected covering up to equivalence \art{The-universal-covering},
  which is a principal bundle with structure group the fundamental group \art{Fundamental-group-acting-on-coverings}.
  Any other covering is a quotient of this universal covering by a subgroup of the fundamental group.
  Another important property is the {\em monodromy theorem} of lifting maps which applies in this context of diffeological coverings
  \art{Monodromy-theorem}.
  These constructions are then illustrated by the 1-dimensional irrational tori \art{The-Kronecker-flows-as-diffeological-fibrations}.
\end{chaphead}

%%%%%%%%%%%%%%%%%%%%%%%%%%%%%%%%%%%%%%%%%%%%%%%%%%%%%%%%%%

\section*{Building Bundles with Fibers}
\label{Building-bundles-with-fibers}

\begin{sechead}
  Intuitively,
  a fiber bundle is a kind of projection such that the preimage of any value is equivalent to any other,
  and equivalent to a given {\em fiber}.
  This is the minimal condition required for using the word ``fiber bundle''.
  The notion of equivalence depends uniquely and naturally on the category we are working in.
  Here the equivalence will obviously be understood  as diffeomorphic.
  But what is then of major importance is how these fibers are {\em glued together} to form a fiber bundle.
  This is the critical point which can make the difference between one choice and another,
  and this is the point we describe and discuss  in this section.
\end{sechead}

\begin{article}\artlabel{The category of smooth surjections}
  \addcontentsline{toc}{section}{\small\hspace{10pt} The category of smooth surjections}
  \label{The-category-of-smooth-surjections}
  We shall consider the category whose objects are {\em smooth surjections}\index{Smooth surjection} $\pi:\T \to \B$,
  $\T$ and $\B$ are diffeological spaces,
  $\pi$ is surjective and smooth.
  The space $\T$ will be called the {\em total space}\index{Total space} of the surjection and $\B$ its {\em base space}\index{Base space}.
  Let $\pi' : \T' \to \B'$ be another surjection,
  a {\em morphism}\index{Morphism} from $\pi$ to $\pi'$ is a couple of smooth maps $\Phi : \T \to \T'$ and $\phi : \B \to \B'$ such that the following diagram commutes.
  
  \begin{center}
    \begin{tikzcd}[column sep=large, row sep=large, every label/.append style = {font = \small}]
      \T \arrow[d, swap, "\pi"] \arrow[rr,"\Phi"] && \T' \arrow[d, "\pi'"]  \\
      \B \arrow[rr, swap, "\phi = \pr(\Phi)"] && \B'
    \end{tikzcd}
  \end{center}

  Note that,
  for any $\Phi$,
  if there exists a map $\phi$ such that $\pi' \circ \Phi = \phi \circ \pi$,
  then $\phi$ is unique,
  it is called the {\em projection}\index{Projection} of $\Phi$ on $\B$,
  and denoted by $\pr(\Phi)$.
  The identity morphism of the object $\pi : \T \to \B$ is the pair $(\id_\T, \id_\B)$.
  A morphism $(\Phi,\phi)$ from $\pi$ to $\pi'$ is an isomorphism if and only if $\Phi \in \Diff(\T,\T')$ and $\phi \in \Diff(\B,\B')$,
  in this case we say that $\pi$ and $\pi'$ are equivalent.
  We say particularly that $\pi$ and $\pi'$ are $\B$-equivalent if $\B'= \B$ and $\phi = \id_\B$.
  
  1. {\em Trivial projections}.
  We say that a projection $\pi : \T \to \B$ is {\em trivial} with {\em fiber}\index{Fiber} $\F$ if $\pi$ is equivalent to the first factor projection $\pr_1 : \B \times \F \to \B$,
  where $\F$ is some diffeological space.
  In this case $\pi$ is $\B$-equivalent to $\pr_1$,
  that is,
  there exists $\Phi \in \Diff(\T,\B \times \F)$ such that $\pr_1 \circ \Phi = \pi$.
  
  2. {\em Pullbacks}.
  Let $\pi' : \T' \to \B'$ be a smooth surjection.
  Let $\phi : \B \to \B'$ be a smooth map.
  The {\em pullback}\index{Pullback} of $\T'$ by $\phi$ is denoted by $\phi^*(\T')$,
  and defined by
  $$%
  \phi^*(\T') = \{ (b,t') \in \B \times \T' \mid \phi(b) = \pi'(t') \}.
  $$%
  It is equipped with the subset diffeology of the product diffeology.
  Now,
  let $\pi = \pr_1 \restriction \phi^*(\T')$, $\Phi = \pr_2 \restriction \phi^*(\T')$ and $\T = \phi^*(\T')$, so $\pi : \T \to \B$ is a smooth surjection,
  called {\em pullback} of $\pi$ by $\phi$,
  and $(\Phi, \phi)$ is the natural morphism from $\pi$ to $\pi'$ associated with this construction.
  
  \begin{center}
    \begin{tikzcd}[column sep=large, row sep=large, every label/.append style = {font = \small}]
      \T = \phi^*(\T') \subset \B \times \T' \arrow[d, swap, "\pi = \pr_1 \restriction \phi^*(\T')"] \arrow[rrr,"\Phi = \pr_2 \restriction \phi^*(\T')"] &&& \T' \arrow[d, "\pi'"]  \\
      \B \arrow[rrr, swap, "\phi"] &&& \B'
    \end{tikzcd}
  \end{center}

  The pullback is associative up to equivalence.
  Let us denote by $\pi'_\phi$ the pullback of $\pi'$ by $\phi$.
  If $\pi'' : \T'' \to \B''$ is a smooth projection,
  if $\phi' : \B' \to \B''$ and $\phi : \B \to \B'$ are two smooth maps,
  then $(\pi''_{\phi'})_\phi$ is naturally equivalent to $\pi''_{\phi' \circ \phi}$.
  Indeed,
  on the one hand we have $(\phi' \circ \phi)^*(\T'') = \{ (b,t'') \in \B \times \T'' \mid \pi''(t'') = \phi' \circ \phi(b) \}$,
  and on the other hand, $\phi^*(\phi'^*(\T'')) = \{ (b, (b',t'')) \in \B \times \B' \times \T'' \mid b' = \phi(b), \text{ and } \phi'(b') = \pi''(t'') \}$.
  Thus,
  the map $(b,t'') \mapsto (b, (\phi(b), t''))$ defined on $(\phi' \circ \phi)^*(\T'')$ takes its values in $\phi^*(\phi'^*(\T''))$ and is clearly an isomorphism.
  Also note that,
  if $\pi'$ is a subduction, then $\pi$ is also a subduction,
  and symmetrically,
  if $\phi$ is a subduction,
  then $\Phi$ is also a subduction.
  
  3. {\em Restriction}.
  Let $\pi : \T \to \B$ be a smooth surjection,
  and let $\A \subset \B$ be any subset.
  We call the {\em restriction of $\pi$ over $\A$} the restriction $\pi \restriction \pi^{-1}(\A) : \pi^{-1}(\A) \to \A$,
  denoted sometimes by $\pi^{\restriction \A}$,
  where $\pi^{-1}(\A)$ and $\A$ are equipped with the subset diffeology.
  It is equivalent to the pullback of $\pi$ by the inclusion $j_\A : \A \hookrightarrow \B$.
  When $\A = \{b\}$ is a single point,
  the restriction $\pi^{-1}(b)$ is denoted by $\T_b$ and is called the {\em fiber}\index{Fiber} of $\pi$ (or $\T$) over $b$.
  
  4. {\em Reductions and sections}.
  Let $\pi : \T \to \B$ be a smooth surjection,
  and let $\Sigma \subset \T$.
  If $\pi \restriction \Sigma$ is surjective,
  then it is a smooth surjection called the {\em reduction of $\pi$ to $\Sigma$}\index{Reduction}.
  If $\Sigma \cap \T_b$ is reduced to a single point for all $b \in \B$,
  we say that $\Sigma$ is a {\em section}\index{Section} of $\pi$ (over $\B$).
  In this case,
  we also call section the map $\sigma$ which associates with every $b \in \B$ the unique element contained in $\Sigma \cap \T_b$.
  Conversely,
  any map $\sigma : \B \to \T$ satisfying $\pi \circ \sigma = \id_\B$ defines a section $\Sigma = \sigma(\B)$.
  We say that the section $\Sigma$ is smooth when $\sigma$ is smooth.
  Note that,
  if $\sigma$ is smooth,
  then the projection $\pi$ is not just a smooth projection but also a subduction.
  
  5. {\em Local triviality}.
  Let $\pi : \T \to \B$ be a smooth surjection and $\F$ be some diffeological space.
  We say that $\pi$ is {\em locally trivial}\index{Locally trivial} with {\em fiber} $\F$ if there exists a cover of $\B$ by a family of D-open sets $\{\U_i\}_{i \in \cI}$ such that,
  the restriction of $\pi$ over each $\U_i$ is trivial with fiber $\F$.
\end{article} %% The-category-of-smooth-surjections

\begin{proof}
  There is not much to prove here.
  Let us consider the first point.
  Let us check that if $\pi : \T \to \B$ is trivial with fiber $\F$,
  then $\pi$ is $\B$-equivalent to $\pr_1 : \B \times \F \to \B$.
  Let $(\Psi, \psi)$ be an equivalence from $\pi$ to $\pr_1$.
  From $\pr_1 \circ \Psi = \psi \circ \pi$ we get $\Psi(x) = (\psi (\pi(x)), f(x))$,
  where $f \in \Cinfty(\X,\F)$.
  Now,
  since $\psi$ is an automorphism of $\B$, $\psi \times \id_\F : (b,y) \mapsto (\psi(b),z)$ is an automorphism of $\B \times \F$,
  thus $\Phi = (\psi \times \id_\F)^{-1} \circ \Psi : \X \to \B \times \F$ is a diffeomorphism satisfying $\pr_1 \circ \Phi = \pi$.
  Let us now consider the second point.
  On the one hand,
  $(\phi' \circ \phi)^*(\T'')$ is made of the elements $(b, t'')\in \B \times \T''$ such that $\phi' \circ \phi(b) = \pi''(t'')$,
  on the other hand  $\phi^*(\phi'^*(\T''))$ is made of the elements $(b,(b',t'')) \in \B \times \phi'^*(\T'')$ such that $\phi(b) = b'$,
  that is, $(b,(b',t'')) \in \B \times (\B' \times \T'')$ with $\phi(b) = b'$ and $\phi'(b')= \pi''(t'')$.
  Thus,
  the diffeomorphism $(b,t'') \mapsto (b,(\phi(b),t''))$ from  $(\phi' \circ \phi)^*(\T'')$ to  $\phi^*(\phi'^*(\T''))$ is a natural $\B$-equivalence from $\pi''_{\phi' \circ \phi}$ to $(\pi''_{\phi'})_\phi$.
  Now,
  if $\pi'$ is a subduction,
  let $\P: \U \to \B$ be a plot,
  and let $\phi \circ \P$ be a plot of $\B'$ which can be lifted locally in a plot $\Q$ of $\T'$ \art{Criterion-for-being-a-subduction}.
  Hence,
  $\pi' \circ \Q = \phi \circ \P$,
  everywhere $\Q$ is defined,
  thus $ r \mapsto (\P(r),\Q(r))$ is a plot of $\phi^*(\T')$,
  and $\pi$ is a subduction.
  A symmetrical reasoning applies for $\phi$ and $\Phi$.
  For the third point,
  let us notice that the map $t \mapsto (b= \pi(t),t)$,
  defined from $\pi^{-1}(\A)$ to $j_\A^*(\T)$ is an $\A$-equivalence from $\pi \restriction \pi^{-1}(\A)$ to $\pr_1 \restriction j_\A^*(\T)$.
  For the fourth point,
  note that if $\sigma : \B \to \T$ is a smooth section,
  then every plot $\P : \U \to \B$ lifts on $\T$ by $\Q = \sigma \circ \P$,
  and thus $\pi$ is a subduction.
\end{proof}

\begin{article}\artlabel{The category of groupoids}
  \addcontentsline{toc}{section}{\small\hspace{10pt} The category of groupoids}
  \label{The-category-of-groupoids}
  Let us recall some elements of the theory of groupoids,
  and let us set some notation;
  see \cite{McL75} and \cite{McK87} for a comprehensive presentation.
  A groupoid\index{Groupoid} $\KK$ is a category such that the objects constitute a set (and so do the morphisms) and such that every morphism\index{Morphism} (also called an {\em arrow}\index{Arrow}) is invertible,
  that is,
  every morphism is an isomorphism.
  
  1. {\em Objects and Morphisms}.
  We denote by $\Obj(\KK)$ and by $\Mor(\KK)$ respectively,
  the set of objects and morphisms of $\bf K$.
  For all $x, x' \in \Obj(\KK)$,
  we denote by  $\Mor_\KK(x,x')$ the set of morphisms from $x$ to $x'$.
  This set is also denoted sometimes by $\KK(x,y)$,
  or even by $\KK_{x,y}$.
  
  2. {\em Source and Target}.
  Let $f \in \Mor(\KK)$,
  $\source(f)$ and $\target(f)$ denote the {\em source}\index{Source} and the {\em target}\index{Target} of $f$,
  that is,
  if $f \in \Mor_\KK(x,x')$,
  then $x = \source(f)$ and $x' = \target(f)$.
  
  3. {\em Groupoid Operation}.
  We denote by $f \cdot g$ (multiplication) or by $g \circ f$ (composition) the groupoid composite of two morphisms $f$ and $g$,
  defined when $\target(f) = \source(g)$.
  
  4. {\em Characteristic Map}.
  We call {\em characteristic map}\index{Characteristic map} the map defined by
  $$%
  \car : \Mor(\KK) \to \Obj(\KK) \times \Obj(\KK), \text{ with } \car(f)=(\source(f),\target(f)).
  $$%
  And thus,
  $\Mor_\KK(x,x') = \car^{-1}(x,x')$.
  
  5. {\em Subgroupoids}.
  A {\em subgroupoid}\index{Subgroupoid} $\HH$ is a subcategory which is a groupoid,
  that is,
  a
  subcategory such that if $f \in \Obj(\KK)$,
  then $f^{-1} \in \Obj(\KK)$.
  A subgroupoid $\HH$ of a groupoid $\KK$ is said to be {\em wide} if $\Obj(\HH) = \Obj(\KK)$;
  we shall also say that $\HH$ is a {\em reduction} of $\KK$.
  
  6. {\em Structure Groups and Units}.
  For all $x \in \Obj(\KK)$,
  we denote $\KK_x$ instead of $\KK_{x,x} = \Mor_\KK(x,x)$.
  This is a group called the {\em isotropy}\index{Isotropy} of $x$,
  or the {\em structure group} of the groupoid $\KK$ at $x$.
  The identity of $\KK_x$ is denoted by $\id_x$ and is a {\em unit}\index{Unit} of $\KK$.
  The units of $\KK$ form a (wide) subgroupoid defined by
  
  $$%
  \Obj(\Units{\KK}) = \Obj(\KK), \text{ and } \Mor(\Units{\KK}) = \{\id_x \mid x \in \Obj(\KK)\}.
  $$%
  Said differently,
  $\Mor_{\Units{\KK}}(x,x') = \varnothing$ if $x \neq x'$,
  and $\Mor_{\Units{\KK}}(x,x) = \{\id_x\}$.
  We denote by $i_{\Obj(\KK)}$ the injection from the objects to the morphisms,
  $$%
  i_{\Obj(\KK)} : \Obj(\KK) \to \Mor(\KK), \text{ with } i_{\Obj(\KK)}(x) = \id_x.
  $$%
  
  7. {\em Transitivity components}. There exists a natural equivalence relation on the objects of a groupoid $\KK$.
  Two elements $x$ and $x'$ of $\Obj(\KK)$ are equivalent if there exists an arrow $f$ connecting $x$ to $x'$,
  that is,
  $\source(f) = x$ and $\target(f) = x'$.
  This relation decomposes $\Obj(\KK)$ into classes called {\em transitivity components}\index{transitivity component},
  or simply {\em components}\index{Component}.
  Each component defines naturally a subgroupoid,
  called again a transitivity component of $\KK$.
  The groupoid is said to be transitive, or connected,
  if it has only one transitivity component,
  that is,
  if the characteristic map is surjective.
  
  8. {\em Morphisms of Groupoids}.
  A {\em morphism} from a groupoid $\KK$ to a groupoid $\KK'$ is a covariant functor $\Phi$,
  which can be represented by a couple of maps $(\Phi_\eO, \Phi_\eM)$ with
  $$%
  \Phi_\O : \Obj(\KK) \to \Obj(\KK'), \text{ and } \Phi_\eM : \Mor(\KK) \to \Mor(\KK'),
  $$%
  such that the following diagram commutes:
  %
  \begin{equation}\renewcommand{\theequation}{$\diamondsuit$}
    \begin{tikzcd}[column sep=huge, row sep=large, every label/.append style = {font = \small}]
      \Mor(\KK) \arrow[d, swap, "\car"] \arrow[r,"\Phi_\eM"] & \Mor(\KK') \arrow[d, "\car'"]  \\
      \Obj(\KK) \times \Obj(\KK) \arrow[r, swap, "\Phi_\eO \times \Phi_\eO"] & \Obj(\KK') \times \Obj(\KK')
    \end{tikzcd}
  \end{equation}
  %
  and $\Phi_\eM$ distribute the product:
  $$%
  \Phi_\eM(f \cdot g) =
  \Phi_\eM(f) \cdot \Phi_\eM(g).
  $$%
  The groupoids and their morphisms form a category we shall denote by $\Groupoids$.
  The commutativity of the diagram $\car' \circ \Phi_\eM = (\Phi_\eO \times \Phi_\O) \circ \car$ also writes
  $$%
  \source \circ \Phi_\eM = \Phi_\eO \circ \source, \text{ and } \target \circ \Phi_\eM = \Phi_\eO \circ \target.
  $$%
  The {\em kernel}\index{Kernel} of $\Phi$,
  denoted by $\ker(\Phi)$,
  is the subgroupoid of $\KK$ defined by
  $$%
  \Obj(\ker(\Phi)) = \Obj(\KK), \text{ and } \Mor(\ker(\Phi)) = \Phi_\eM^{-1}(\Mor(\Units{\KK'})).
  $$%
  Note that the restrictions of $\Phi_\eM$ to the isotropy groups $\KK_x$ are group homomorphisms,
  precisely
  $$%
  \Phi_\eM \restriction \KK_x \in \Hom(\KK_x, \KK'_{\Phi_\eO(x)}).
  $$%
  In the following we will consider only {\em strong morphisms} \cite{Iva99},
  more precisely injective on the objects.
  In that case $\ker(\Phi)$ is totally disconnected,
  if  $x \neq x'$,
  then $\Mor_{\ker(\Phi)}(x,x') = \varnothing$,
  and $\nu \in \Mor(\ker(\Phi))$ means that,
  for some $x \in \Obj(\KK)$,
  $\nu \in \KK_x$ and $\Phi_\eM(\nu) = \id_{x'}$ with $x' =  \Phi_\eO(x)$.
  Thus,
  $$%
  \Mor(\ker(\Phi)) = \bigcup_{x \in \Obj(\KK)} \ker(\Phi_\eM \restriction \KK_x).
  $$%
  Remark that $\Units{\KK}$ is always a subgroupoid of $\ker(\Phi)$,
  we say that $\Phi$ is faithful if $\ker(\Phi)$ is reduced to $\Units{\KK}$.
  The {\em quotient groupoid}, denoted by $\KK/\ker(\Phi)$,
  is defined in any case by
  $$%
  \Obj(\KK/\ker(\Phi)) = \Obj(\KK)
  \ \text{and} \
  \Mor(\KK/\ker(\Phi)) = \Mor(\KK) / \Mor(\ker(\Phi)),
  $$%
  where the quotient of the set of morphisms is defined by the equivalence relation:
  for all  $f$ and $f'$ in $\Mor(\KK)$, $f \sim f'$ if there exists $\nu \in \Mor(\ker(\Phi))$ such that $f' = \nu \cdot f$.
  Note that $\nu \in \ker(\Phi_\eM \restriction \KK_x)$,
  where $x = \source(f)$,
  and $\car(f)=\car(f')$.
  The class of $f \in \Mor(\KK)$ is the subset of morphisms
  $$%
  \class(f) = \{ \nu \cdot f \mid \nu \in \ker(\Phi_\eM \restriction \KK_x), \text{ with } x = \source(f) \}.
  $$%
  The multiplication of classes on the quotient $\KK/\ker(\Phi)$ is naturally defined by
  $$%
  \class(f) \cdot \class(f') = \class(f \cdot f').
  $$%
  Then, $\class(f)^{-1} = \class(f^{-1})$ and the structure groups $[\KK/\ker(\Phi)]_x$ of the quotient are the quotient groups $\KK_x/\ker(\Phi_\eM \restriction \KK_x)$.
  There exist two natural functors, the {\em projection functors} $\pi$ from $\KK$ to $\KK/\ker(\Phi)$ and the {\em quotient functor} $\pr(\Phi)$ from $\KK/\ker(\Phi)$ to $\KK'$ such that the following diagram commutes.
  %
  \begin{center}
    \begin{tikzcd}[column sep=large, row sep=huge, every label/.append style = {font = \small}]
      \KK \arrow[rr,"\Phi"] \arrow[dr, swap, "\pi"] & {} & \KK' \\
      {} & \KK/\ker(\Phi) \arrow[ur, swap, "\pr(\Phi)"] & {}
    \end{tikzcd}
  \end{center}
  The projection functor $\pi$ is naturally defined by
  %
  $$%
  \pi_\eO = \id_{\Obj(\KK)}, \text{ and } \pi_\eM(f) = \class(f),
  $$%
  for all $f \in \Mor(\KK)$,
  where $\class(f) = \ker(\Phi_\eM \restriction \KK_x) \cdot f$,
  with $x = \source(f)$.
  The quotient functor is naturally defined by
  $$%
  \pr(\Phi)_\eO = \Phi_\eO, \text{ and } \pr(\Phi)_\eM(\class(f)) = \Phi_\eM(f).
  $$%
  \Note~Let $\KK$ be a connected groupoid,
  let us choose a point $x \in \Obj(\KK)$.
  The data of the groupoid is just the choice of an arrow $a_{x,x'} \in \Mor_\KK(x,x')$ by element
  \linebreak
  $x' \in \Obj(\KK)$ and the group $\KK_x= \Mor(x,x)$.
  Indeed, for any pair $x',x'' \in \Mor_\KK(x',x'')$ we have
  $$%
  \Mor_\KK(x',x'') = a_{x,x''} \circ \KK_x \circ a_{x,x'}^{-1}.
  $$%
  Now,
  if the groupoid is not connected,
  it splits into components for which the previous construction applies.
  And therefore,
  the structure groups of all components are isomorphic.
\end{article} %% The-category-of-groupoids

\begin{proof}
  For item 8\,:
  Let $f \in \Mor(\KK)$ with $\source(f) = x$,
  $\target(f) = y$.
  Assume that $f \in \Mor(\ker(\Phi))$,
  that is,
  $\Phi_\eM(f) = \id_{x'}$ for somme $x' \in \Obj(\KK')$.
  Thus,
  the diagram $(\diamondsuit)$ gives $\car'(\Phi_\eM(f)) = (\Phi_\eO \times \Phi_\eO)(\car(f))$,
  that is $(x',x') = (\Phi_\eO(x),\Phi_\eO(y))$.
  Hence,
  $\Phi_\eO(x)=\Phi_\eO(y)=x'$,
  but since $\Phi_\eO$ is injective:
  $x = y$,
  and $f \in \KK_x$.
  
  Let us check,
  now,
  that $f \sim f'$ is a well defined equivalence relation. It is
  reflexive, $f = \id_x \cdot f$. It is is symmetric,
  if $f' = \nu \cdot f$, then $f = \nu^{-1} \cdot f'$. It is transitive,
  if $f' = \nu \cdot f$ and $f'' = \nu' \cdot f'$, then
  $f'' = \nu' \cdot \nu \cdot f$, with $\nu' \cdot \nu \in \ker(\Phi)_x$
  and $x = \source(f)$.
  Next, let us check that the multiplication of classes is well defined. Let $f$ and $f'$
  be two composable arrows, let $\nu \in \ker(\Phi)_x$ and $\nu' \in \ker(\Phi)_{x'}$
  with $x = \source(f)$ and $x' = \source(f')$. Thus,
  $\nu \cdot f \cdot \nu' \cdot f' = \nu \cdot f \cdot \nu' \cdot f^{-1} \cdot f \cdot f'$,
  but $\Phi_\eM(\nu \cdot f \cdot \nu' \cdot f^{-1}) = \Phi_\eM(\nu) \cdot  \Phi_\eM(f) \cdot
  \Phi_\eM(\nu') \cdot  \Phi_\eM(f^{-1}) =
  \id_{\Phi_\eO(x)} \cdot \Phi_\eM(f) \cdot \id_{\Phi_\eO(x')} \cdot \Phi_\eM(f^{-1})
  = \Phi_\eM(f) \cdot \Phi_\eM(f^{-1}) = \Phi_\eM(f \cdot f^{-1})
  = \id_{\Phi_\eO(x)}$. Hence $\nu \cdot f \cdot \nu' \cdot f^{-1} \in
  \ker(\Phi_\eM \restriction \KK_x)$. Therefore,
  $\class(\nu \cdot f \cdot \nu' \cdot f') = \class(f \cdot f')$.
  Let us check also that $\pr(\Phi)_\M$ is well defined, let
  $\class(f) = \class(f')$, then $f' = \nu \cdot f$,
  with $\nu \in \ker(\Phi_\M \restriction \KK_x)$ and $x = \source(f)$. Therefore,
  $\Phi_\eM(f') = \Phi_\eM(\nu \cdot f) = \Phi_\eM(\nu) \cdot \Phi_\eM(f) =
  \id_{\Phi_\eO(x)} \cdot \Phi_\eM(f) = \Phi_\eM(f)$.
  For the note, every morphism
  $b \in \Mor_\KK(x,x')$ writes in a unique way $b
  = a_{x,x''} \circ a \circ a_{x,x'}^{-1}$, where $a
  = a_{x,x''}^{-1} \circ b \circ a_{x,x'} \in \KK_x$.
\end{proof}

\begin{article}\artlabel{Diffeological groupoids}
  \addcontentsline{toc}{section}{\small\hspace{10pt} Diffeological groupoids}
  \label{Diffeological-groupoids}
  Let $\KK$ be a groupoid,
  we shall use the notations introduced previously \art{The-category-of-groupoids}.
  We call {\em groupoid diffeology}\index{Diffeology!groupoid diffeology} on the groupoid $\KK$ a diffeology defined on  $\Mor(\K)$ and on $\Obj(\K)$ such that the following conditions are satisfied.%
  \footnote{This notion is close to the concept of Lie-groupoid but differs in that we just require that source and target are smooth maps and not necessarily stronger;
  see also \art{Fibrating-groupoids} and \art{A-simple-non-fibrating-groupoid}.}
  \begin{itemize}
    \item[(a)] The multiplication $(f,g) \mapsto f \cdot g$,
    defined on the set $\{ (f,g) \in \Mor(\KK) \times \Mor(\KK) \mid \target(f) = \source(g)\}$ equipped with the subset diffeology of the product diffeology,
    with values in $\Mor(\KK)$,
    is smooth.
    \item[(b)]  The inversion $f \mapsto f^{-1}$,
    from $\Mor(\KK)$ to itself,
    is smooth.
    \item[(c)] The two maps $\source$ and $\target$,
    from $\Mor(\KK)$ to $\Obj(\KK)$, are smooth.
    Or,
    which is equivalent,
    the characteristic map $\car$ is smooth.
    \item[(d)] The identity injection $i_{\Obj(\KK)} : x \mapsto \id_x$,
    from $\Obj(\KK)$ into $\Mor(\KK)$,
    is smooth.
  \end{itemize}
  A groupoid equipped with a groupoid diffeology is a {\em diffeological groupoid}\index{Diffeological groupoid}.
  
  1. {\em Morphisms}.
  A morphism from a diffeological groupoid $\KK$ to another one $\KK'$ is a morphism of groupoid $\Phi$ such that $\Phi_\eM : \Mor(\KK) \to \Mor(\KK')$ is smooth,
  which implies that $\Phi_\eO : \Obj(\KK) \to \Obj(\KK')$ is also smooth.
  This defines the category $\DGroupoids$ of diffeological groupoids.
  An isomorphism $\Phi$ is a morphism such that $\Phi_\eM$ and $\Phi_\eO$ are diffeomorphisms.
  
  2. {\em Subgroupoids}.
  Any subgroupoid $\HH$ of a diffeological groupoid $\KK$ is a diffeological groupoid,
  when $\Mor(\HH)$ and $\Obj(\HH)$ are equipped with the subset diffeology.
  In this case we say that $\HH$ is a {\em diffeological subgroupoid} of $\KK$.
  
  3. {\em Morphisms and quotients}.
  Let $\KK$ and $\KK'$ be two diffeological groupoids.
  Let $\Phi$ be a strong morphism from $\KK$ to $\KK'$,
  actually a morphism injective on the objects.
  Equipped with the quotient diffeology,
  the groupoid $\KK/\ker(\Phi)$ is a diffeological groupoid.
  The natural associated morphisms $\pi$, from $\KK$ to $\KK/\ker(\Phi)$,
  and $\pr(\Phi)$,
  from $\KK/\ker(\Phi)$ to $\KK'$ \art{The-category-of-groupoids},
  are smooth,
  that is,
  morphisms of diffeological groupoids.
  
  \Note{1} The inclusion $i_{\Obj(\KK)}$ of a diffeological groupoid is an induction \art{What-is-an-induction}.
  The space of objects $\Obj(\KK)$ identifies naturally,
  in the category $\Diffeology$,
  with the subspace of identities
  $$%
    \Obj(\KK) \simeq \{ \id_x  \mid x \in \Obj(\KK) \} \subset \Mor(\KK).
  $$%
  Therefore,
  a diffeological groupoid is uniquely characterized by $\Mor(\KK)$,
  its groupoid operation and its diffeology.
  
  \Note{2} The structure groups of a component of a diffeological groupoid $\KK$,
  equipped with the subset diffeology of $\Mor(\KK)$,
  are isomorphic in the category $\DGroups$,
  which is a full subcategory of the category $\DGroupoids$.
\end{article} %% Diffeological-groupoids

\begin{proof}
  For point 1.
  Note that $\Phi_\eO = \source \circ \Phi_\eM \circ i_{\Obj(\KK)}$,
  indeed we have $x \mapsto \id_x = i_{\Obj(\K)}(x) \mapsto \id_{\Phi_\eO(x)} = \Phi_\eM(\id_x) \mapsto \Phi_\eO(x) = \source(\id_{\Phi_\eO(x)})$.
  Thus,
  since $\source$ and $i_{\Obj(\KK)}$ are smooth,
  if $\Phi_\eM$ is smooth,
  then  $\Phi_\eO$ is smooth.
  For Note 1,
  the inverse map $i_{\Obj(\KK)}^{-1} : \id_x \mapsto x$ defined on $\{\id_x \mid x \in \Obj(\KK)\} \subset \Mor(\KK)$ coincides with the restriction of the source map,
  $\source(\id_x) =  x$ (and coincides with the restriction of the target map too).
  Thus,
  $i_{\Obj(\KK)}^{-1}$ is smooth when $\{\id_x \mid x \in \Obj(\KK)\}$ is equipped with the subset diffeology.
  Then,
  since $i_{\Obj(\KK)}$ and $i_{\Obj(\KK)}^{-1}$ are smooth,
  $i_{\Obj(\KK)}$ is an induction.
  Point 3 is a consequence of the game of subductions.
  For Note 2,
  the conjugation given in \xart{The-category-of-groupoids}{Note} is clearly smooth and thus the structure groups of every component are isomorphic in the category of diffeological groups.
\end{proof}

\begin{article}\artlabel{Fibrating groupoids}
  \addcontentsline{toc}{section}{\small\hspace{10pt} Fibrating groupoids}
  \label{Fibrating-groupoids}
  Let $\KK$ be a diffeological groupoid,
  we shall say that $\KK$ is {\em fibrating\/}\index{Fibrating groupoid} if the characteristic map $\car : \Mor(\KK) \to \Obj(\KK) \times \Obj(\KK)$ defined by $\car(f) = (\source(f),\target(f))$ is a subduction.%
  \footnote{In \cite{Igl85} these groupoids were called ``parfaits'' but this terminology is confusing since to be perfect for a group means something else.}
  %
  Note that this implies in particular that $\car$ is surjective and thus the groupoid is connected \xart{The-category-of-groupoids}{Note}.
  As well,
  all the isotropy groups are isomorphic in the category of diffeological groups \xart{Diffeological-groupoids}{Note 2}.
  %
  This definition is the central point of the theory of diffeological fiber bundles.
  We shall see in the following that the theory of diffeological fiber bundles is reduced to the theory of fibrating diffeological groupoids.
\end{article} %% Fibrating-groupoids

\begin{article}\artlabel{Trivial diffeological groupoids}
  \addcontentsline{toc}{section}{\small\hspace{10pt} Trivial fibrating groupoid}
  \label{Trivial-fibrating-groupoid}
  Let us describe a simple but important example of fibrating diffeological groupoid.
  Let $\X$ be a diffeological space and $\G$ be a diffeological group.
  Let us define $\Gamma$ by
  $$%
  \Obj(\Gamma) = \X, \text{ and } \Mor(\Gamma) = \X \times \G \times \X,
  $$%
  with the multiplication defined on
  $$%
  \Def(\cdot) = \{((x,g,y), (y,k,z)) \mid x,y,z \in \X, \text{ and } g,k \in \G \}
  $$%
  by
  $$%
  (x,g,y) \cdot (y,k,z) = (x,gk,z).
  $$%
  We clearly defined,
  this way,
  a groupoid.
  The source and the target are obviously
  $$%
  \source(x,g,y) = x, \text{ and } \target(x,g,y) = y.
  $$%
  The inverses,
  the identities,
  and the isotropy groups are given by
  $$%
  (x,g,y)^{-1} = (y,g^{-1},x), \  id_x = (x,\id_\G, x), \text{ and } \Gamma_x = \{(x,g,x) \mid g \in \G\}.
  $$%
  Note that $\Def(\cdot)$ is naturally diffeomorphic to $\X \times \G \times \X \times \G \times \X$ and the multiplication as well as the inversion  are clearly smooth.
  The identity injection from $\X$ to $\Mor(\Gamma)$ and the characteristic map from $\Mor(\Gamma)$ to $\X \times \X$
  $$%
  i_{\Obj(\Gamma)} : x \mapsto (x,\id_\G,x), \text{ and } \car : (x,g,y) \mapsto (x,y)
  $$%
  are again smooth.
  Therefore  $\Gamma$ is a diffeological groupoid.
  We shall name this groupoid  {\em the trivial groupoid}\index{Trivial groupoid} with {\em base $\X$} and {\em structure group} $\G$,
  and we may denote it sometimes by $\Triv(\X,\G)$.
  Every diffeological groupoid $\K$ isomorphic to $\Triv(\X,\G)$,
  for some $\X$ and $\G$,
  will be said {\em trivial} (with base $\X$ and structure group $\G$).
  Moreover,
  the map $\car$ being clearly a subduction,
  thanks to the section $(x,y) \mapsto (x,\id_\G,y)$,
  the trivial groupoid $\Triv(\X,\G)$ is fibrating.
  
  \Note~A diffeological groupoid $\KK$ is trivial if and only if there exists a map $\sigma : \Obj(\K)^2 \to \Mor(\KK)$ such that the following conditions are satisfied.
  \begin{itemize}
    \item[A.] $\sigma$ is a smooth section of the characteristic map $\car$\,:
    $\car \circ \sigma = \id_{\Obj(\KK)^2}$.
    \item[B.] $\sigma$ satisfies the cocycle relation $\sigma(x,y) \cdot \sigma(y,z) = \sigma(x,z)$,
    for all triples of points $x,y,z$ in $\Obj(\KK)$.
  \end{itemize}
  Picking a basepoint $\eo \in \Obj(\KK)$,
  and denoting by $\Mor(\KK,\eo)$ the subspace of arrows of $\KK$ with origin $\eo$,
  it is equivalent to say that
  \begin{itemize}
    \item[A\textprime.] There exists a smooth section $s$ of $\target \restriction \Mor(\KK,\eo)$,
    that is,
    a smooth map $s : \Obj(\KK) \to \Mor(\KK,\eo)$ such that $\target \circ s = \id_{\Obj(\KK)}$.
  \end{itemize}
\end{article} %% Trivial-diffeological-groupoids

\begin{proof}
  Let us assume first that $\KK$ is trivial with structure group $\G$,
  let us denote by $\X$ the space of objects.
  Let $\Phi$ be an isomorphism from $\KK$ to $\Triv(\X,\G)$.
  Thus,
  $\Phi_\eM : \Mor(\KK) \to \X \times \G \times \X$ is a diffeomorphism satisfying $\Phi_\eM(f \cdot g) = \Phi_\eM(f) \cdot \Phi_\eM(g)$.
  So,
  $\sigma$ defined by $\sigma(x,y) = \Phi_\eM^{-1}(x, \id_\G, y)$ satisfies the conditions above.
  Now,
  let us assume that there exists such a section $\sigma$.
  Since $\sigma$ is a section,
  we get $\car$ surjective,
  the groupoid $\KK$ is connected.
  Thus choosing a basepoint $\eo \in \X$,
  we can define $s(x) = \sigma(\eo,x)$,
  for all $x \in \X$.
  The map $s$ is a smooth section of $\target \restriction \Mor(\KK,\eo)$.
  Let us denote by $\G$ the isotropy group $\KK_\eo$,
  for each $f \in \Mor(\KK)$,
  $s(x) \cdot f \cdot s(y)^{-1}$ belongs to $\G$,
  where $x = \source(f)$ and $y = \target(f)$.
  The map $\Phi_\eM : f \mapsto (\source(f), s(x) \cdot f \cdot s(y)^{-1}, \target(f))$,
  with values in $\X \times \G \times \X$,
  is smooth and satisfies: 1)
  $\Phi_\eM(f \cdot g) = \Phi_\eM(f) \cdot \Phi_\eM(g)$ for all $f$ and $g$ in $\Mor(\KK)$ such that $\target(f) = \source(g)$,
  and 2) $\car \circ \Phi_\eM = \car$,
  where $\car$ denotes generically the characteristic maps.
  Moreover $\Phi_\eM$ is bijective,
  its inverse is given by $\Phi_\eM^{-1}(x, k , y) = s(x)^{-1} \cdot k \cdot s(y)$,
  which is clearly smooth.
  Therefore $\Phi = (\Phi_\eM, \id_{\Obj(\KK)})$ is an isomorphism from $\KK$ to $\Triv(\X,\G)$.
  By the way,
  we also proved the proposition A\textprime.
\end{proof}

\begin{article}\artlabel{A simple nonfibrating groupoid}
  \addcontentsline{toc}{section}{\small\hspace{10pt} A simple non-fibrating groupoid}
  \label{A-simple-non-fibrating-groupoid}
  It is of course not difficult to find a non-fibrating groupoid,
  but the following example is simple and instructive.
  Let $\cR$ be an equivalence relation defined on a diffeological space $\X$.
  We can regard the graph of $\mathop{\cR}$ as the groupoid $\Gamma(\mathop{\cR})$ given by
  $$%
  \Obj(\Gamma(\mathop{\cR})) = \X, \text{ and } \Mor(\Gamma(\mathop{\cR})) = \{(x,x') \in \X \times \X \mid x \mathop{\cR} x' \}.
  $$%
  The multiplication of $(x,x')$ by $(x'',x''')$ is defined only if $x' = x''$ by
  $$%
  (x,x') \cdot (x',x''') = (x,x''').
  $$%
  Equipped with their natural
  respective diffeologies these spaces define a diffeological groupoid.
  The space $\Mor(\Gamma(\mathop{\cR}))$ is just the graph $\graph(\mathop{\cR})$ of the relation $\mathop{\cR}$.
  The components of the groupoid are just the classes $\quotient{\X}{\cR}$ of the equivalence relation $\cR$,
  and it is fibrating if and only if there is only one class.
  In this case it coincides with the trivial groupoid with base $\X$ and trivial group $\{\id\}$.
\end{article} %% A-simple-non-fibrating-groupoid

\begin{article}\artlabel{Structure groupoid of a smooth surjection}
  \addcontentsline{toc}{section}{\small\hspace{10pt} Structure groupoid of a smooth surjection}
  \label{Structure-groupoid-of-a-smooth-surjection}
  Let $\X$ and $\B$ be two diffeological spaces.
  Let $\pi : \X \to \B$ be a smooth surjection.
  Let us define
  $$%
  \Obj(\KK) = \B, \text{ and for all $b,b' \in \B$}, \Mor_{\KK}(b,b') = \Diff(\X_b, \X_{b'}),
  $$%
  where the $\X_b = \pi^{-1}(b)$, $b \in \B$,
  are equipped with the subset diffeology.
  Let us define  on
  $$%
  \Mor(\KK) = \bigcup_{b,b' \in \B} \Mor_{\KK}(b,b')
  $$%
  the natural multiplication $f \cdot g = g \circ f$,
  for $f \in \Mor_{\KK}(b,b')$ and $g \in \Mor_{\KK}(b',b'')$,
  %%###########
  \begin{figure}[tb]
    \centerline{\includegraphics{Figures-PDF/fig-groupoid-projection}}
    \caption{The groupoid associated with a surjection.}
    \label{fig-groupoid-partition}
  \end{figure} %% The groupoid associated with a surjection
  %%###########
  $\KK$ is clearly a groupoid.
  The source and target maps are given by
  $$%
  \source(f) = \pi(\Def(f)), \text{ and } \target(f) = \pi(\Val(f)).
  $$%
  The groupoid $\KK$ is then equipped with a functional diffeology of $\KK$ defined as follows.
  Let $\X_\source$ be the total space of the pullback of $\pi$ by $\source$ \art{The-category-of-smooth-surjections},
  that is,
  $$%
  \X_\source = \{ (f,x) \in \Mor(\KK) \times \X \mid x \in \Def(f) \}.
  $$%
  We define the evaluation map $\ev$ as usual (see \art{Functional-diffeologies})
  $$%
  \ev : \X_\source \to \X, \text{ with } \ev(f,x) = f(x).
  $$%
  There exists a coarsest diffeology on $\Mor(\KK)$,
  which gives to $\KK$ the structure of a diffeological groupoid and such that the
  evaluation map  $\ev$ is smooth.
  It will be called the {\em functional diffeology}\index{Functional diffeology}.
  Equipped with this functional diffeology,
  the groupoid $\KK$ captures the {\em smooth structure} of the projection $\pi$.
  It is why we define $\KK$ as the {\em structure groupoid} of the  surjection $\pi$.
  
  \Note{1} If $\B$ is reduced to a point,
  $\Obj(\KK) = \{\star\}$,
  this diffeology coincides with the usual functional diffeology of $\Diff(\X) = \Mor(\KK)$ \art{Functional-diffeology-on-groups-of-diffeomorphisms}.
  
  \Note{2} This construction also applies when we have just a partition $\cP$ on a diffeological space $\X$.
  We can equip the quotient $\Q = \quotient{\X}{\cP}$ with the quotient diffeology,
  and we get the structure groupoid of the partition as the structure groupoid of the projection $\pi : \X \to \Q$.
\end{article} %% Structure-groupoid-of-a-smooth-surjection

\begin{proof}
  Let $\P : \U \to \Mor(\KK)$ be some parametrization,
  and let us define
  $$%
  \left\{
  \begin{array}{l}
    \X_{\source \circ \P} = \{(r,x) \in \U \times \X \mid x \in \Def(\P(r)) \}, \\
    \X_{\target \circ \P} = \{(r,x) \in \U \times \X \mid x \in \Val(\P(r)) \}.
  \end{array}
  \right.
  $$%
  The space $\X_{\source \circ \P}$ is the total space of the pullback of $\pi : \X \to \B$ by the map $\source \circ \P$,
  that is,
  $(\source \circ \P)^*(\X)$.
  It is equivalent to the total space $\P^*(\source^*(\X))$,
  thanks to the identification $(r,x) \mapsto (r,(\P(r),x))$.
  But,
  under this identification,
  the second projection mutes into $\P \times \id_\X$,
  that is,
  $(r,x) \mapsto (\P(r),x)$.
  The situation is summarized in the following diagram,
  where the maps are restricted to the indicated domains,
  it applies {\em mutatis mutandis\/} to $\X_{\target \circ \P}$.

\begin{center}
    \begin{tikzcd}[column sep=large, row sep=large, every label/.append style = {font = \small}]
      \X_{\source \circ \P} \arrow[r,"\P \times \id_\X"] \arrow[d, swap, "\pr_1"] & \X_\source \arrow[d, "\pr_1"] \arrow[r,"\pr_2"] & \X \arrow[d, "\pi"]  \\
      \U \arrow[r, swap, "\P"] & \Mor(\KK) \arrow[r, swap, "\source"] & \B
    \end{tikzcd}
  \end{center}

  Now, let us introduce the following maps
  $$%
  \left\{
  \begin{array}{l}
    \P_\source : \X_{\source \circ \P} \to \X, \text{ with } \P_\source(r,x) = \P(r)(x), \\
    \P_\target : \X_{\source \circ \P} \to \X, \text{ with } \P_\target(r,x) = \P(r)^{-1}(x).
  \end{array}
  \right.
  $$%
  Then,
  let us define $\cD(\U,\Mor(\KK))$ as the set of all the parametrizations $\P : \U \to \Mor(\KK)$ such that:
  $$%
  \renewcommand{\arraystretch}{1.1}
  \text{($\clubsuit$)} \ \ \car \circ \P \in \Cinfty(\U, \B \times \B), \text{ and }
  \left\{
  \begin{array}{ll}
    \text{($\diamondsuit$)} & \P_\source \in \Cinfty(\X_{\source \circ \P}, \X), \\
    \text{($\heartsuit$)} & \P_\target \in \Cinfty(\X_{\target \circ \P}, \X)),
  \end{array}
  \right.
  \renewcommand{\arraystretch}{1}
  $$%
  where $\X_{\source \circ \P}$ and $\X_{\target \circ \P}$ are equipped with the subset diffeology of the product $\U \times \X$.
  Let us show now that the union $\DD$ of all the families $\cD(\U,\Mor(\KK))$,
  where $\U$ runs over the set of real domains,
  is a diffeology.
  Note first that the condition ($\clubsuit$) defines already a diffeology:
  the pullback of the diffeology of $\B$ by $\car$ \art{Pullback-of-diffeologies}.
  So,
  we need only check that the conditions ($\diamondsuit$) and ($\heartsuit$) define a diffeology.
  And since the condition ($\heartsuit$) is just the condition ($\diamondsuit$) inverting the arrows,
  we have just to prove that the condition ($\diamondsuit$) defines a diffeology.
  
  {\em Covering axiom.}
  Let ${\bf f} = [0 \mapsto f]$,
  where $f \in \Diff(\X_b,\X_{b'})$.
  We have
  $$%
  \X_{\source \circ {\bf f}} = \X_b, \text{ and } {\bf f}_\source = f.
  $$%
  But ${\bf f}_\source : \X_{\source \circ {\bf f}} \to \X$ is smooth,
  thus ${\bf f}$ belongs to $\DD$.
  
  {\em Compatibility axiom.}
  Let $\P \in \DD$ and $\F$ be a smooth parametrization in $\Dom(\P)$.
  The map $(\P \circ \F)_\source$ decomposes as follows
  $$%
  \begin{array}{r@{\hspace{3pt}}c@{\hspace{3pt}}c@{\hspace{3pt}}c@{\hspace{3pt}}l@{\hspace{3pt}}}
    \X_{\source \circ \P \circ \F} & \longrightarrow & \X_{\source \circ \P} & \longrightarrow & \X  \\
    (r,x) & \longmapsto & (\F(r),x) & \longmapsto & \P(\F(r))(x).
  \end{array}
  $$%
%  \begin{center}\n
%    \begin{tikzcd}[column sep=small, row sep=0.15em, every label/.append style = {font = \small}]\n
%    \X_{\source \circ \P \circ \F} \arrow[r] & B \arrow[r] & C  \\\n
%    (r,x) \arrow[r, mapsto] & (\F(r),x) \arrow[r, mapsto] &  P(\F(r))(x)\n
%  \end{tikzcd}\n
%  \end{center}\n
%
  As a composite of smooth maps,
  $(\P \circ \F)_\source$ is a smooth map.
  Thus,
  $\P \circ \F \in \DD$.
  
  {\em Locality axiom.}
  Let $\P : \U \to \Mor(\KK)$ be a parametrization such that,
  for every $r \in \U$ there exists an open neighborhood $\A \subset \U$ of $r$ such that $(\P \restriction \A) \in \DD$.
  So,
  $(\P \restriction \A)_\source$,
  defined on $\X_{\source \circ (\P \restriction \A)} = \X_{\source \circ \P} \restriction \A = \pr_1^{-1}(\A)$ is smooth for the subset diffeology.
  Since $\X_{\source \circ \P} \subset \U \times \X$ is equipped with the subset diffeology and since $\pr_1 : \U \times \X \to \U$ is smooth,
  its restriction $\pr_1 \restriction \X_{\source \circ \P}$ is smooth
  \art{Subspaces-and-subset-diffeology},
  therefore D-continuous \art{The-D-Topology-of-diffeological-spaces} \art{Smooth-maps-are-D-continuous}.
  Thus,
  $(\pr_1 \restriction \X_{\source \circ \P})^{-1}(\A)$ is D-open.
  Now,
  since $(\P \restriction \A)_\source$ is smooth for the subset diffeology and defined on a D-open,
  $(\P \restriction \A)_\source$
  is a local smooth map \art{Local-smooth-maps-are-defined-on-D-opens}.
  Therefore,
  $\P_\source$ is everywhere locally smooth,
  so $\P_\source$ is smooth \art{To-be-smooth-or-locally-smooth},
  and $\P$ belongs to $\DD$.
  As we claimed above,
  what we said for the $\source$ part applies {\em mutatis mutandis\/} to the $\target$ part.
  Hence, $\DD$ is a diffeology of $\Mor(\KK)$.
  We know already that the characteristic function $\car$ is smooth,
  let us show now that the evaluation map is smooth.
  Let $\Phi$ be a plot of $\X_\source$,
  that is,
  $\Phi(r) = (\P(r), \Q(r))$ where $\P$ is a plot of $\Mor(\KK)$ and $\Q$ is a plot of $\X$,
  such that $\Q(r) \in \Def(\P(r))$.
  The map $\ev \circ \Phi$ decomposes into the product $\P_\source \circ \bar\Q$,
  where $\bar\Q : r \mapsto (r, \Q(r))$,
  $$%
  r \mapsto (r,\Q(r)) \mapsto \P(r)(\Q(r)) = \P_\source(r,\Q(r)).
  $$%
  But $\bar\Q$ is as smooth as $\Q$ and,
  by the very definition of $\DD$,
  $\P_\source$ is smooth,
  thus $\ev$ is smooth.
  Moreover,
  note that for any diffeology on $\Mor(\KK)$ such that the evaluation is smooth,
  for any plot $\P$ of  $\Mor(\KK)$,
  the map $\P_\source$ is smooth.
  
  Let us prove now that $\DD$ is a groupoid diffeology.
  Let us begin by showing that the multiplication is smooth.
  Let $\Phi : r \mapsto (\P(r), \P'(r))$ be a plot of the domain of the multiplication,
  that is,
  a plot of the square $\Mor(\KK)^2$ such that $\target \circ \P = \source \circ \P'$.
  Let us denote $\P''(r) = \P(r) \cdot \P'(r) = \P'(r) \circ \P(r)$ for all $r \in \Dom(\Phi)$.
  
  1. Since $\car \circ \P''(r) = (\source(\P(r)), \target(\P'(r)))$ and since $\source \circ \P$ and $\target \circ \P'$ are smooth, $\car$ is smooth.
  
  2. First of all,
  let us notice that $\X_{\source \circ \P''} = \X_{\source \circ \P}$ and $\X_{\target \circ \P''} = \X_{\target \circ \P'}$.
  Now,
  the map $\P''_\source : (r,x) \mapsto \P''(r)(x)$ is the composite of two smooth maps $(r,x) \mapsto (r, \P(r)(x)) = (r, \P_\source(r,x)) \mapsto \P'(r)(\P(r)(x)) = \P'_\source(r,\P_\source(r,x))$.
  Thus,
  the multiplication is smooth.
  
  3. Let $\P$ be a plot of $\Mor(\KK)$,
  lets us show that $\Q = [r \mapsto \P(r)^{-1}]$ is also a plot of $\Mor(\KK)$.
  First of all, since $\car(\Q(r)) = \car(\P(r)^{-1})$ and
  $$%
  \car(\P(r)^{-1}) = (\source(\P(r)^{-1}), \target(\P(r)^{-1})) = (\target(\P(r)), \source(\P(r))),
  $$%
  $\car \circ \Q$ is smooth.
  Now,
  $\X_{\source \circ \Q} = \X_{\target \circ \P}$ and $\X_{\target \circ \Q} = \X_{\source \circ \P}$,
  with
  $\Q_\source = \P_\target$ and $\Q_\target = \P_\source$.
  Thus, $\Q_\source$ and $\Q_\target$ are smooth.
  
  4. Finally,
  let $i_{\Obj(\KK)} = i_\B$ be the inclusion of $\B = \Obj(\KK)$ into $\Mor(\KK)$,
  $i_\B(b) = \id_{\X_b}$.
  Let $\F : \U \to \B$ be a plot and let $\P = i_\B \circ \F$,
  that is,
  $\P(r) = \id_{\X_{\F(r)}}$.
  Let us show that $\P$ is a plot of $\Mor(\KK)$.
  First of all,
  we have $\car \circ \P(r) = (\F(r),\F(r))$,
  which is smooth.
  Next,
  for all $r \in \Def(\F)$,
  we have
  $$%
  \X_{\source \circ \P} = \X_{\target \circ \P} = \F^*(\X) = \{ (r,x) \in \U \times \X \mid \pi(x) = \F(r) \}.
  $$%
  Then,
  $\P_\source(r,x) = \P(r)(x) = \id_{\X_{\F(r)}}(x) = x$,
  and as well $\P_\target(r,x) = x$.
  Thus,
  $\P_\source$ and
  % Ici peut être un saut de ligne
  $\P_\target$ are smooth because  they are the restrictions,
  to a subspace of $\U \times \X$,
  of the second projection,
  which is smooth.
  Hence,
  $\DD$ defines a groupoid diffeology for which the evaluation map is smooth.
  We also saw that,
  for any groupoid diffeology on $\Mor(\KK)$ such that the evaluation is smooth,
  $\P_\source$ and $\P_\target$ are smooth,
  for all plots $\P$ of $\Mor(\KK)$.
  Therefore,
  $\DD$ is the coarsest groupoid diffeology such that the evaluation map is smooth.
\end{proof}

\begin{article}\artlabel{Diffeological fibrations}
  \addcontentsline{toc}{section}{\small\hspace{10pt} Diffeological fibrations}
  \label{Diffeological-fibrations}
  Let $\pi : \T \to \B$ be a smooth projection,
  where $\T$ and $\B$ are two arbitrary diffeological spaces.
  
  \Definition We say that $\pi$ is a {\em diffeological fibration}\index{Diffeological fibration},
  or simply a {\em fibration},
  if the structure groupoid $\KK$ \art{Structure-groupoid-of-a-smooth-surjection} is fibrating,
  that is,
  if and only if the characteristic map $\car : \Mor(\KK) \to \B \times \B$ is a subduction \art{Fibrating-groupoids}.
  
  The space $\T$ is called the {\em total space}\index{Total space} of the fibration,
  the space $\B$ is called the {\em base space}\index{Base space} and $\pi$ the {\em projection}\index{Projection}.
  %%###########
  \begin{figure}[tb]
    \centerline{\includegraphics{Figures-PDF/fig-Fibration}}
    \caption{A fibration.}
    \label{A-fibration}
  \end{figure}
  %%###########
  We also say that $\T$ is {\em fibered} over $\B$ by $\pi$,
  or $\T$ is a {\em fiber bundle}\index{Fiber bundle} over $\M$.
  The preimages $\T_b = \pi^{-1}(b)$,
  necessarily all diffeomorphic since $\car$ is surjective,
  are called the {\em fibers}\index{Fiber} of the fibration,
  or of the fiber bundle.
  The fiber $\T_b$ is called the {\em fiber over $b$}.
  The diffeological class of the fibers is called the {\em type} of the fiber.
\end{article} %% Diffeological-fibrations

\begin{article}\artlabel{Fibrations and local triviality along the plots}
  \addcontentsline{toc}{section}{\small\hspace{10pt} Fibrations and local triviality along the plots}
  \label{Fibrations-and-local-triviality-along-the-plots}
  A smooth map $\pi : \T \to \B$,
  where $\T$ and $\B$ are two
  diffeological spaces,
  is a fibration \art{Diffeological-fibrations} if and only if there exists a diffeological space $\F$ such that the pullback of $\pi$ by any plot $\P$ of $\B$ is locally trivial,
  with fiber $\F$ \art{The-category-of-smooth-surjections}.
  The space $\F$ represents the type of the fibers of $\pi$,
  it is why it is also called {\em the fiber} of the fibration.
  We say that $\pi$ is {\em locally trivial along the plots} (of $\B$).
  
  \Note{1} A diffeological fibration $\pi : \T \to \B$ is,
  in particular,
  a local subduction \art{Local-subductions}.
  Precisely,
  for every plot $\P : \U \to \B$,
  for all $r \in \U$ and for all $t \in \T_b = \pi^{-1}(b)$,
  with $b = \P(r)$,
  there exists a plot $\Q$ of $\T$ defined on some open neighborhood $\V$ of $r$ lifting $\P \restriction \V$,
  that is,
  $\P \restriction \V = \pi \circ \Q$,
  and such that $\Q(r) = t$.
  
  \Note{2}  There is a hierarchy in the various notions of fiber bundles:
  trivial bundles are locally trivial (with respect to the D-topology),
  locally trivial bundles are locally trivial along the plots.
  The converse is not true.
  To be locally trivial along the plots does not mean that the fibration itself is locally trivial,
  as many examples will point it out.
  Look for example at the irrational torus $\Torus_\alpha$,
  quotient of $\T^2$ by the Kronecker flow (irrational solenoid) \art{The-Kronecker-flows-as-diffeological-fibrations}.
  
  \Note{3} If the base of a diffeological fiber bundle is a manifold,
  then the fiber bundle is locally trivial.
  This comes immediately from
  the definition,
  consider the pullback by local charts.
  If moreover the fiber is a manifold,
  then the diffeological fiber bundle is a fiber bundle in the category of manifolds.
  This shows in particular that the classical notion of fiber bundle can also be defined directly in diffeological terms as a property of its associated groupoid,
  but of course this definition leads to leave an instant the category of manifolds.
\end{article} %% Fibrations-and-local-triviality-along-the-plots

\begin{proof}
  Let us begin by assuming that $\pi$ is a fibration,
  according to the definition \art{Diffeological-fibrations}.
  Let $\P : \U \to \B$ be a plot,
  and let $\F = \X_b = \pi^{-1}(b)$,
  for some base point $b \in \B$.
  The map $\Phi : r \mapsto (b,\P(r))$ is obviously a plot of $\B \times \B$.
  Let $\car$ be the characteristic map of the groupoid $\KK$ associated with $\pi$.
  Since $\car$ is a subduction,
  by hypothesis,
  for every $r_0 \in \U$ there exist an open neighborhood $\V$ of $r_0$ and a plot $\phi : \V \to \Mor(\KK)$ such that $\car \circ \phi = \Phi \restriction \V$.
  So,
  for all $r \in \V$, $\source(\phi) = b$ and $\target(\phi) = \P(r)$,
  that is,
  $\phi \in \Diff(\F, \X_{\P(r)})$,
  see diagram ($\clubsuit$).
  Let us then define $\psi : (r,\xi) \mapsto (r, \phi(r)(\xi))$,
  where $(r,\xi) \in \ \V \times \F$.
  This map takes {\em a priori\/} its values in $\V \times \X$,
  but since $\phi(r) \in \Diff(\F,\X_{\P(r)})$,
  $\Val(\psi) \subset \P^*(\X)$.
  The map $\psi$ is obviously bijective,
  and smooth.
  Indeed,
  $\psi$ decomposes into $(r,\xi) \mapsto (r,\phi(r), \xi) \mapsto (r, \phi(r)(\xi))$,
  the first map $(r,\xi) \mapsto (r,\phi(r), \xi)$ is clearly smooth,
  and the second map $(r,(\phi(r), \xi)) \mapsto (r, \phi(r)(\xi))$ is just $\id_\V \times \ev$ which is smooth by definition of the functional diffeology.
  Its inverse is given by $\psi^{-1} : (r,x) \mapsto (r, \phi(r)^{-1}(x)$ which is again smooth,
  thanks to the smoothness of the inverse map in $\Mor(\KK)$ and to the smoothness of the evaluation map $\ev$ \art{Structure-groupoid-of-a-smooth-surjection}.
  %
  \begin{equation} \renewcommand{\theequation}{$\clubsuit$}
    \begin{tikzcd}[column sep=huge, row sep=large, every label/.append style = {font = \small}]
      \V \times \F \arrow[dr, swap,"\pr_1"] \arrow[rr, "\psi"] & {} & \P^*(\X) \arrow[dl, "\pr_1"] \arrow[r, "\pr_2"] & \X \arrow[d, "\pi"] \\
      {}  & \V \arrow[rr, swap, "\P"] &  {} & \B
    \end{tikzcd}
  \end{equation}
  %
  Thus,
  $\psi$ is a diffeomorphism from $\V \times \F$ to $\P^*(\X) \restriction \V$.
  Therefore,
  $\P^*(\X)$ is locally trivial with fiber $\F$.
  
  Conversely,
  let us assume that $\pi$ is locally trivial along the plots,
  with fiber $\F$.
  Let $\Phi : r \mapsto (\phi(r), \phi'(r))$ be a plot of $\B \times \B$,
  defined on $\U$.
  Let $r_0 \in \U$,
  since the pullbacks of $\pi$ by $\phi$ and $\phi'$ are locally trivial,
  let us choose two open neighborhoods $\V$ and $\V'$ of $r_0$ over which $\phi^*(\P)$ and $\phi'^*(\P)$ are trivial.
  Let us consider then $\W = \V \cap \V'$.
  Let $\Psi$ and $\Psi'$ be the respective trivializations,
  so $\Psi$ and $\Psi'$ write $\Psi(r,\xi) = (r, \psi(r)(\xi))$ and $\Psi'(r,\xi) = (r, \psi'(r)(\xi))$.
  Since the restriction of a diffeomorphism to a subspace is a diffeomorphism onto its image \art{Restricting-inductions-to-subspaces},
  $\psi(r)$ and $\psi'(r)$ take their values respectively in $\Diff(\W, \X_{\phi(r)})$,
  and in $\Diff(\W, \X_{\phi'(r)})$.
  Then,
  let us define $\Q : r \mapsto \psi'(r) \circ \psi(r)^{-1}$,
  $\Q$ is defined on $\W$ with values in $\Mor(\KK)$ and satisfies $\car \circ \Q(r) = (\phi(r),\phi'(r))$.
  Now,
  the spaces $\X_{\source \circ \Q}$ and $\X_{\target \circ \Q}$ of \art{Structure-groupoid-of-a-smooth-surjection} are given by
  $$%
  \left\{
  \begin{aligned}
    \X_{\source \circ \Q} & = \{(r,x) \in \W \times \X \mid x \in \phi(r) \}, \\
    \X_{\target \circ \Q} & = \{(r,x) \in \W \times \X \mid x \in \phi'(r) \}.
  \end{aligned}
  \right.
  $$%
  The maps $\Q_\source$ and $\Q_\target$ of \art{Structure-groupoid-of-a-smooth-surjection} decompose into
  $$%
  \left\{
  \begin{array}{l}
    \Q_\source : (r,x) \mapsto (r,\psi(r)^{-1}(x)) \mapsto (r,\psi'(r)(\psi(r)^{-1}(x))) \mapsto \psi'(r)(\psi(r)^{-1}(x)), \\[.8ex]
    \Q_\target : (r,x) \mapsto (r,\psi'(r)^{-1}(x)) \mapsto (r,\psi(r)(\psi'(r)^{-1}(x))) \mapsto \psi(r)(\psi'(r)^{-1}(x)),
  \end{array}
  \right.
  $$%
  that is,
  $\Q_\source = \pr_2 \circ \Psi' \circ \Psi^{-1}$ and $\Q_\target = \pr_2 \circ \Psi \circ \Psi'^{-1}$.
  Therefore,
  $\Q_\source$ and $\Q_\target$ are smooth,
  and since $\car \circ \Q$ is also smooth,
  $\Q$ is a plot of $\Mor(\KK)$ for the functional diffeology.
  Thus,
  $\Q$ is a smooth local lift of the plot $\Phi$ of $\B \times \B$.
  Hence,
  $\car : \Mor(\KK) \to \B \times \B$ is a subduction and $\pi$ is a fibration,
  according to \art{Diffeological-fibrations}.
\end{proof}

\begin{article}\artlabel{Diffeological fiber bundles category}
  \addcontentsline{toc}{section}{\small\hspace{10pt} Diffeological fiber bundles category}
  \label{Diffeological-fiber-bundles-category}
  Diffeological fibrations form a subcategory of the category of smooth surjections \art{The-category-of-smooth-surjections}.
  But note that for any two diffeological fiber bundles $\pi: \X \to \B$ and  $\pi': \X' \to \B'$,
  a morphism from $p$ to $p'$ is just a smooth map $\phi: \X \to \X'$ mapping fiber to fiber,
  since $\pi$ is a subduction the projection $\phi = \pr(\Phi) : \B \to \B'$ is necessarily smooth.
  
  1. {\em Pullbacks}\index{Pullback}.
  The pullback of any diffeological fibration by any smooth map of the base is a diffeological fibration,
  with same fiber.
    
  2. {\em Products}\index{Product}.
  Let $\pi : \T \to \B$ and $\pi' : \T' \to \B'$ be two diffeological fibrations with fibers $\F$ and  $\F'$.
  Then,
  the product $\pi \times \pi' : \T \times \T' \to \B \times \B'$ is a diffeological fibration with fiber $\F \times \F'$.
  
  3. {\em Fibered products}\index{Fibered product}.
  Let $\pi : \T \to \B$ and $\pi' : \T' \to \B$ be two diffeological fibrations with same base.
  The {\em fibered product} is the pullback of the product $\pi \times \pi' : \T \times \T' \to \B \times \B$ by the diagonal map $b \mapsto (b,b)$,
  it is denoted by $\pi \times_\B \pi' : \T \times_\B \T' \to \B$.

  4. {\em Restriction}\index{Restriction}.
  Let $\pi : \T \to \B$ be a diffeological fibration and $\A \subset \B$ a subset,
  equipped with the subset diffeology.
  The restriction of $\pi$ over $\A$ \art{The-category-of-smooth-surjections} is a diffeological fibration.
  
  5. {\em Subbundles}\index{Subbundle}.
  Let $\pi : \T \to \B$ be a diffeological fibration.
  We call {\em subbundle} the reduction of $\pi$ to some subset $\Sigma \subset \T$ \art{The-category-of-smooth-surjections},
  when it happens that this restriction is still a diffeological fibration.
\end{article} %% Diffeological-fiber-bundles-category

\begin{proof}
  We only check the first assertion,
  the other ones are obvious or direct consequences of the first.
  Let $f : \B' \to \B$ be a smooth map and $\pi : \X \to \B$ be a fiber bundle.
  Let $\P : \U \to \B'$ be a plot,
  the pullbacks $\P^*(f^*(\X))$ and $(f \circ \P)^*(\X)$ are equivalent.
  Indeed, an element of the first is a triple $(r,b',x) \in \U \times \B' \times \X$ such that $\P(r) = b'$ and $f(b') = \pi(x)$,
  an element of the second is a pair $(r,x) \in \U \times \X$ such that $f \circ \P(r) = \pi(x)$,
  the equivalence of the two pullbacks is given by $(r,x) \mapsto (r,b'=\P(r),x)$ and conversely $(r,b',x) \mapsto (r,x)$.
  Now,
  $\pi$ is a fiber bundle and $f \circ \P$ is a plot of $\B$,
  the pullback $\pr_1 : (f \circ \P)^*(\X) \to \B$ is locally trivial \art{Fibrations-and-local-triviality-along-the-plots},
  let $(r,\xi) \mapsto (r, \varphi(r)(\xi))$ be a local trivialization.
  By composition with the above equivalence we get a local trivialization of the pullback $\pr_1 : \P^*(f^*(\X)) \to \U$,
  that is,
  $(r,\xi) \mapsto (r, \P(r), \varphi(r)(\xi))$.
\end{proof}


%%%%%%%%%%%%%%%%%%%%%%%%%%%%%%%%%%%%%%%%%%%%%%%%%%%%%%%%%%
%
%   Exercises
%
%%%%%%%%%%%%%%%%%%%%%%%%%%%%%%%%%%%%%%%%%%%%%%%%%%%%%%%%%%

\Exercise

\begin{exercise}[Groupoid associated with $x \mapsto x^3$]
  \label{Groupoid-associated-with-x-x3}
  Describe the groupoid associated with the real map $x \mapsto x^3$ and its standard diffeology \art{Structure-groupoid-of-a-smooth-surjection}.
  Is this groupoid fibrating?
\end{exercise} %% Groupoid-associated-with-x-x3

%%%%%%%%%%%%%%%%%%%%%%%%%%%%%%%%%%%%%%%%%%%%%%%%%%%%%%%%%%
%% MARK: Principal and Associated Fiber Bundles
%%%%%%%%%%%%%%%%%%%%%%%%%%%%%%%%%%%%%%%%%%%%%%%%%%%%%%%%%%

\section*{Principal and Associated Fiber Bundles}
\label{Principal-and-associated-fiber-bundles}

\begin{sechead}
  The most important subcategory of the category of diffeological fiber bundles is the category of {\em principal fiber bundles}.
  They are bundles fibered by free actions of diffeological groups \art{Principal-diffeological-fiber-bundles}.
  They play a central role because in diffeology every diffeological fiber bundle is ``associated'' with a principal fiber bundle \art{Associated-fiber-bundles},
  even the principal fiber bundles themselves.
  Another important aspect of fiber bundles and principal fiber bundles is the notion of structure.
  The structure of the fiber bundle is understood as an equivalent structure for all of its fibers,
  for example a vector space structure,
  an affine structure, a group structure, etc.
  The groupoid approach and the fact that every fiber bundle is defined through a general structure groupoid \art{Diffeological-fibrations},
  which characterizes its {\em smooth structure},
  makes this concept clear:
  a {\em structure} is defined by a wide fibrating subgroupoid of the structure groupoid of the fiber bundle,
  the structure group of the structure groupoid being the  structure group of the fiber bundle structure \art{Structures-on-fiber-bundles}.
\end{sechead}

\begin{article}\artlabel{Principal diffeological fiber bundles}
  \addcontentsline{toc}{section}{\small\hspace{10pt} Principal diffeological fiber bundles}
  \label{Principal-diffeological-fiber-bundles}
  Let $\X$ be a diffeological space and $g \mapsto g_\X$ be a smooth action of a diffeological group $\G$ on $\X$,
  that is,
  a smooth homomorphism from $\G$ to $\Diff(\X)$ \art{Smooth-action-of-a-diffeological-group}.
  Let $\eF$ be the {\em action map},
  $$
  \eF : \X \times \G \to \X \times \X, \text{ with } \eF(x,g) = (x,g_\X(x)).
  $$
  \Proposition If $\eF$ is an induction,
  then the projection $\pi$ from $\X$ to its quotient $\X/\G$ is a diffeological fibration,
  with the group $\G$ as fiber.
  We shall say that the action of $\G$ on $\X$ is {\em principal}\index{Principal action}.
  
  (a) If $\eF$ is inductive,
  then it is in particular injective,
  which implies that the action of $\G$ on $\X$ is free,
  which is indeed a necessary condition.
  
  (b) If a projection $\pi : \X \to \Q$ is equivalent to $\class : \X \to \G/\H$,
  that is,
  if there exists a diffeomorphism $\varphi : \G/\H \to \Q$ such that $\pi = \varphi \circ \class$,
  we shall say that $\pi$ is a {\em principal fibration}\index{Principal fibration},
  or a {\em principal fiber bundle}\index{Principal fiber bundle},
  with {\em structure group}\index{Structure group} $\G$.
  
  \Note{1} The main assertion is based on the following constructions.
  If $\eF$ is an induction,
  then the subgroupoid $\KK_\G$ of equivariant arrows of the groupoid $\KK$ associated with the projection $\pi : \X \to \Q$ \art{Structure-groupoid-of-a-smooth-surjection},
  $$%
  \left\{
  \begin{aligned}
    \Obj(\KK_\G) & = \Q, \\
    \Mor(\KK_\G) & = \{ f \in \Mor(\KK) \mid f \circ g_\X = g_\X \circ f, \text{ for all } g \in \G\},
  \end{aligned}
  \right.
  $$%
  is fibrating,
  which implies immediately that $\KK$ is fibrating,
  and then that $\pi$ is a diffeological fibration.
  The groupoid $\KK_\G$ is the {\em principal fiber bundle structure groupoid}\index{Principal fiber bundle structure groupoid},
  for the action of $\G$ on $\X$ \art{Structures-on-fiber-bundles}.
  In order to prove that the groupoid $\KK_\G$ is fibrating,
  we introduce a third groupoid denoted by $\pi \times_\G \pi$ and defined by
  $$%
  \Obj(\pi \times_\G \pi) = \Q, \text{ and } \Mor(\pi \times_\G \pi) = \X \times_\G \X,
  $$%
  where $\X \times_\G \X$ is the quotient of the product $\X \times \X$ by the diagonal action of $\G$,
  that is,
  $g_{\X \times \X}(x,y) = (g_\X(x), g_\X(y))$.
  Let $[x,y] \in \X \times_\G \X$ be the class of the pair $(x,y)$,
  and let
  $$%
  \xi : [x,y] \mapsto (\pi(x), \pi(y))
  $$%
  be the characteristic map.
  The groupoid multiplication is then defined by
  $$%
  [x,y] \cdot [y',z'] = [x,g_\X^{-1}(z')], \text{ where }
  g_\X(y) = y',\ g\in \G.
  $$%
  The identities and inverses are given by
  $$%
  \id_q = [x,x], \text{ where } \pi(x) = q, \text{ and } [x,y]^{-1} = [y,x].
  $$%
  There exists a natural smooth faithful functor $\Phi$ from $\pi \times_\G \pi$ to $\KK_\G$ defined by
  $$%
  \Phi_\eO = \id_\Q, \text{ and } \Phi_\eM : [x,y] \mapsto \eR_y \circ \eR_x^{-1},
  $$%
  where $\eR_x$ is the orbit map $g \mapsto g_\X(x)$,
  with $x \in \X$ and $g \in \G$.
  Next,
  we check that $\pi \times_\G \pi$ is fibrating.
  Then,
  since $\Phi$ is smooth,
  $\KK_\G$,
  and therefore $\KK$ is also fibrating.
  Actually,
  the functor $\Phi$ is an equivalence of diffeological groupoid,
  its inverse $\Phi^{-1}$ from $\KK_\G$ to $\pi \times_\G \pi$ is smooth too.
  
  \Note{2} In the above construction,
  if $\pi : \X \to \Q$ and $\pi' : \X' \to \Q$ are two $\G$-principal fiber bundles and $\psi : \X \to \X'$ is a $\G$-equivariant diffeomorphism,
  $\psi(g_\X(x)) = g_{\X'}(\psi(x))$,
  then the map $\Psi: [x,y] \mapsto [\psi(x),\psi(y)]'$,
  from $\X \times_\G \X$ to $\X' \times_\G \X'$,
  is well defined and it is an isomorphism of diffeological groupoids.
\end{article} %% Principal-diffeological-fiber-bundles

\begin{proof}
  (a) Let us prove first that $\pi \times_\G \pi$ is a well defined groupoid.
  Let $[x,y]$ and $[s,t]$ be two elements of $\X \times_\G \X$ such that $s = g_\X(y)$, $g \in \G$.
  Let  $x',y',s',t'$ such that $[x',y'] = [x,y]$ and $[s',t'] = [s,t]$.
  Thus,
  there exists $k, h \in \G$ such that $x' = k_\X(x)$, $y' = k_\X(y)$,
  $s' = h_\X(s)$, $t' = h_\X(t)$.
  By definition we should have $[x,y] \cdot [s,t] = [x, g_\X^{-1}(t)]$,
  and we have $s' = h_\X(s) = h_\X \circ g_\X(y) = h_\X \circ g_\X \circ k_\X^{-1}(y')$,
  then $[x',y'] \cdot [s',t'] = [x', k_\X \circ g_\X^{-1} \circ h_\X^{-1}(t')] = [k_\X(x), k_\X(g_\X^{-1}(t))] = [x,g_\X^{-1}(t)] = [x,y] \cdot [s,t]$.
  Therefore,
  the multiplication is well defined.
  The inverse and the identities are then obvious and obviously smooth.
  
  (b) Let $p$ be the natural projection from $\X \times \X$ to its quotient $\X \times_\G \X$.
  Thanks to the following diagram and to the fact that $\pi$,
  and thus $\pi \times \pi$,
  is a subduction,
  the characteristic map $\xi$ is a subduction.
  %
  \begin{center}
    \begin{tikzcd}[column sep=large, row sep=large, every label/.append style = {font = \small}]
      \X \times \X \arrow[dr, swap, "\pi \times \pi"] \arrow[rr,"p"] & {} & \X \times_\G \X \arrow[dl, "\xi"]  \\
      {} & \Q \times \Q  & {}
    \end{tikzcd}
  \end{center}
  %
  (c) Let us prove now that the multiplication is smooth.
  Let $\Phi$ be a plot of the domain of the groupoid multiplication,
  that is,
  $\Phi(r) = (\phi(r),\phi'(r))$,
  where $\phi$ and $\phi'$ are plots of $\X \times_\G \X$ and $\target(\phi(r)) = \source(\phi'(r))$ for all $r \in \U = \Dom(\Phi)$.
  Since $p$ is a subduction,
  $\phi$ and $\phi'$ can be lifted locally in $\X \times \X$.
  Let $\Psi : r \mapsto (\psi_1(r),\psi_2(r))$ and $\Psi' : r \mapsto (\psi'_2(r),\psi'_3(r))$ be two smooth local lifts of $\phi$ and $\phi'$,
  respectively.
  The condition $\target \circ \phi = \source \circ \phi'$ writes $\pi \circ \psi_2 = \pi \circ \psi'_2$,
  for all $r$ in the common domain of the lifts.
  Since the action of $\G$ on $\X$ is free,
  there exists a map $\gamma$ such that $\psi'_2(r) = \gamma(r)_\X(\psi_2(r))$ wherever the two functions are defined.
  The map $r \mapsto (\psi_2(r),\psi'_2(r))$ is smooth with values in the image of $\eF$,
  since $\eF$ is an induction and $(\psi_2(r), \gamma(r)) = \eF^{-1}(\psi_2(r),\psi'_2(r))$,
  $\gamma$ is smooth.
  And since $\G$ is a diffeological group,
  the parametrization $r \mapsto \gamma(r)^{-1}$ is smooth too.
  Now,
  let $\psi_3(r) = \gamma(r)^{-1}(\psi'_3(r))$,
  the map $\Psi'' : r \mapsto (\psi_2(r),\psi_3(r))$ is another smooth local lift of $\phi'$ and then $\phi(r) \cdot \phi'(r) = p(\psi_1(r),\psi_3(r))$.
  Therefore,
  the map $r \mapsto \phi(r) \cdot \phi'(r)$ is locally smooth,
  and thus smooth.
  Hence,
  the groupoid $\pi \times_\G \pi$ is a diffeological groupoid,
  and since $\xi$ is a subduction,
  $\pi \times_\G \pi$ is a fibrating groupoid.
  
  (d) We shall exhibit a smooth morphism from $\pi \times_\G \pi$ to $\KK_\GG$,
  which will be used to prove that $\KK$ is fibrating.
  Let $\varphi$ be the map,
  from $\X \times \X$ to $\Mor(\KK_\G)$,
  defined by
  $$%
  \varphi(x,y) = \eR_y \circ \eR_x^{-1} : \G_\X(x) \to \G_\X(y), \text{ for all } x,y \in \X.
  $$%
  This is the only equivariant map $f$ from the orbit $\G_\X(x)$ to $\G_\X(y)$ mapping $x$ to $y$.
  Indeed,
  if another equivariant map $f' : \G_\X(x) \to \G_\X(y)$ maps $x$ to $y$,
  then for all $g \in \G$, $f(g_\X(x)) = g_\X(f(x)) = g_\X(f'(x)) = f'(g_\X(x))$,
  and since every $x' \in \G_\X(x)$ writes in a unique way $g_\X(x)$ for some $g \in \G$, $f(x') = f'(x')$ for all $x' \in \G_\X(x)$,
  and $f =f'$.
  Thus,
  $\Val(\Phi_\eM) = \{ f \in \Mor(\KK) \mid f \circ g_\X = g_\X \circ f, \text{ for all } g \in \G\} = \Mor(\KK_\G)$.
  Now,
  $\varphi(x,y) = \varphi(x',y')$ if and only if $(x',y') = (g_\X(x),g_\X(y))$ for some $g \in \G$.
  Indeed,
  if $\eR_y \circ \eR_x^{-1} = \eR_{y'} \circ \eR_{x'}^{-1}$,
  then $x'=g_\X(x)$ for some $g \in \G$,
  and $\eR_{y'} \circ \eR_{x'}^{-1}(x') = \eR_y \circ \eR_x^{-1}(g_\X(x))$ gives $y' = g_\X(y)$.
  Thus,
  there exists a map
  $$%
  \bar\varphi : \X \times_\G \X = \Mor(\pi \times_\G \pi) \to \Mor(\KK_\G)
  $$%
  such that $\bar\varphi \circ p = \varphi$.
  Let us check that $\varphi$ is smooth,
  since $p$ is a subduction it will imply that $\bar\varphi$ also is smooth.
  \begin{equation}
    \renewcommand{\theequation}{$\clubsuit$}
    \begin{tikzcd}[column sep=large, row sep=large, every label/.append style = {font = \small}]
      {} & \X \times \X \arrow[dl, swap, "p"] \arrow[dr, "\varphi"] & {} \\
      \X \times_\G \X \arrow[rr, swap, "\bar\varphi"] & {} & \Mor(\KK_\G)
    \end{tikzcd}
  \end{equation}
  %
  Let $\H : r \mapsto (\eta(r),\eta'(r))$ be a plot of $\X \times \X$,
  let $\HH = \varphi \circ \H$.
  The spaces $\X_{\source \circ \HH}$, $\X_{\target \circ \HH}$ and the maps $\HH_\source$ and $\HH_\target$,
  see \xart{Structure-groupoid-of-a-smooth-surjection}{proof},
  are given by
  $$%
  \begin{array}{ll}
    \X_{\source \circ \HH} = \{ (r,x) \in \Dom(\H) \times \X \mid \pi \circ \eta(r) = \pi(x) \}, & \HH_\source (r,x) = \HH(r)(x), \\
    \X_{\target \circ \HH} = \{ (r,x) \in \Dom(\H) \times \X \mid \pi \circ \eta'(r) = \pi(x) \}, & \HH_\target (r,x) = \HH(r)^{-1}(x).
  \end{array}
  $$%
  Next,
  $\pi \circ \eta(r) = \pi(x)$ means that there exists
  $$%
  \gamma : \X_{\source \circ \HH} \to \G, \text{ such that } \gamma(r,x)_\X(\eta(r)) = x.
  $$%
  As well,
  there exists $\gamma'$ for $\eta'$.
  Let us check that $\gamma$ is smooth.
  Let $\psi : t \mapsto (\R(t),\P(t))$ be a plot of $\X_{\source \circ \HH}$,
  then $\gamma \circ \psi(t) = \gamma(\R(t),\P(t))$ satisfies $\gamma(\R(t),\P(t))_\X(\eta(\R(t))) = \P(t)$,
  but the maps $\eta \circ \R$ and $\P$ are smooth and with values in $\Val(\eF)$.
  Since $\eF$ is an induction,
  the map $t \mapsto \gamma(\R(t),\P(t))$ is smooth,
  and thus $\gamma$ is smooth.
  Therefore,
  the map $\HH_\source$,
  which writes $\HH_\source(r,x) = \HH(r)(\gamma(r,x)_\X(\eta(r))$,
  is smooth because it is the composite of smooth maps.
  Actually,
  thanks to the equivariance of the map $\HH(r) \in \Mor(\pi \times_\G \pi)$ we have $\HH_\source(r,x) = \gamma(r,x)_\X(\H(r)(\eta(r)))$,
  but $\H(r)(\eta(r)) = \eta'(r)$,
  so $\HH_\source(r,x) = \gamma(r,x)_\X(\eta'(r))$.
  The same holds {\em mutatis mutandis\/} for $\eta'$,
  and we can check that the map $\HH_\target$ is smooth,
  using moreover the smoothness of the inversion in $\G$.
  %
  Now,
  on the one hand we have $\bar\varphi([x,y] \cdot [y',z']) = \bar\varphi([x,y] \cdot [y,z])$,
  where $y' = g_\X(y)$ and $z = g_\X^{-1}(z')$,
  and $\bar\varphi([x,y] \cdot [y,z]) = \bar\varphi([x,z]) = \eR_z \circ \eR_x^{-1}$,
  and on the other hand $\bar\varphi([x,y]) \cdot \bar\varphi([y,z]) = (\eR_z \circ \eR_y^{-1}) \circ (\eR_y \circ \eR_x^{-1}) = \eR_z \circ \eR_x^{-1}$.
  Then,
  $\Phi = (\Phi_\eO = \id_\Q, \Phi_\eM = \bar\varphi)$ is a (smooth faithful) functor from $\pi \times_\G \pi$ to $\KK_\G$.
  %
  Moreover,
  by the next commutative  diagram,
  where $p$ and $\bar\varphi$ are smooth and $\pi \times \pi$ and $\xi$ are subductions,
  we deduce that $\car$ is smooth and is a subduction.
  Thus,
  $\KK_\G$ is a fibrating groupoid,
  and so is $\KK$.
  
  Let us check now that $\Phi$ is an equivalence of diffeological groupoid.
  Let $r \mapsto f_r$ be a plot of $\Mor(\KK_\G)$.
  Thus,
  $r \mapsto q_r = \source(f_r)$ is a plot of $\Q$.
  Let $r \mapsto x_r$ be a smooth local lift of $r \mapsto q_r$, so $r \mapsto x'_r = f_r(x_r)$ is a plot of $\X$,
  and $f_r = \eR_{x'_r} \circ \eR_{x_r}^{-1}$,
  since $f_r$ is only determined by its value at one point.
  Then,
  the map $\varphi$ of the diagram $(\clubsuit)$ is a subduction,
  and $\bar\varphi$ is a diffeomorphism.
  Thus,
  $\KK_\G$ and $\pi \times_\G \pi$ are equivalent diffeological groupoids.
  %
  \begin{center}
    \begin{tikzcd}[column sep=large, row sep=large, every label/.append style = {font = \small}]
      \X \times \X \arrow[dr, swap, "\pi\times\pi"] \arrow[r,"p"] & X \times_\G \X \arrow[r,"\bar\varphi"] \arrow[d, fill=white, auto, "\xi"] & \Mor(\KK_\G) \arrow[dl, "\car"]  \\
      {} & \Q \times \Q  & {}
    \end{tikzcd}
  \end{center}
  %
  (e) Concerning the projection $\pi : \X \mapsto \Q$,
  if the preimages of $\pi$ are the orbits of $\G$,
  and if $\pi$ is a subduction,
  then $\Q$ is equivalent to $\X/\G$,
  that is,
  the map $\varphi : \X/\G \to \Q$ defined by $\varphi(\class(x)) = \pi(x)$ is a natural diffeomorphism.
  Then,
  the couple of morphisms $(\id_\X, \varphi)$ is an equivalence between the projection $\class : \X \to \X/\G$,
  and $\pi : \X \to \Q$, $\pi$ is thus a fibration.
\end{proof}

\begin{article}\artlabel{Category of principal fiber bundles}
  \addcontentsline{toc}{section}{\small\hspace{10pt} Category of principal fiber bundles}
  \label{Category-of-principal-fiber-bundles}
  From the definition of morphisms between diffeological groupoids \art{The-category-of-groupoids} we get a natural category of principal fiber bundles.
  Let $\pi : \X \to \Q$ and $\pi' : \X' \to \Q'$ be two principal fiber bundles with structure groups $\G$ and $\G'$.
  A {\em morphism of principal fiber bundles} from $\pi$ to $\pi'$ is a smooth map $ \phi : \X \to \X'$,
  and a smooth homomorphism $h : \G \to \G'$ such that
  $$%
  \phi \circ g_\X = h(g)_{\X'} \circ \phi, \text{ for all } g \in \G.
  $$%
  This defines a map $\varphi : \Q \to \Q'$ by $\varphi(q) = \pi'(\phi(x))$,
  with $\pi(x) = q$,
  and since $\pi$ is a subduction, $\varphi$ is smooth.
  When $\phi$ is a diffeomorphism and $h$ an isomorphism,
  they define an isomorphism of principal fiber bundle,
  $\varphi$ is then a diffeomorphism.
  
  \Note{1} When $\Q = \Q'$ and $\varphi = \id_\Q$,
  the definition above gives the subcategory of $\Q$-principal fiber bundles.
  
  \Note{2} Fixing the structure group $\G$ introduces the subcategory of {\em $\G$-principal fiber bundles}\index{G-principal fiber bundle@$\G$-principal fiber bundle}.
  In this category,
  the trivial principal bundle is the trivial projection with fiber $\G$,
  $\pr_1 : \Q \times \G \mapsto \Q$.
  A principal fiber bundle is trivial if and only if it has a global smooth section.
  
  \Note{3} Pullbacks of diffeological principal fiber bundles are naturally diffeological principal fiber bundles.
  Let $\pi : \X \to \Q$ be a principal fiber bundle with structure group $\G$,
  and let $f : \Q' \to \Q$ be a smooth map.
  The pullback of $\pi$ by $f$ is the projection $\pr_1 : f^*(\X) \to \Q'$ where $f^*(\X)$ is the set of pairs $(q',x) \in \Q' \times \X$ such that $f(q') = \pi(x)$,
  and this projection is a fiber bundle \art{Diffeological-fiber-bundles-category}.
  Then,
  the pullback $\pr_1 : f^*(\X) \to \Q'$ is a principal fiber bundle for the natural action of $\G$ on $f^*(\X)$ defined by $g_{f^*(\X)}(q',x) = (q', g_\X(x))$,
  where $g\in \G$ and $g_\X$ denotes the action of $\G$ on $\X$.
  The second projection $\pr_2 : f^*(\X) \to \X$ is clearly a morphism of a principal fiber bundle.
\end{article} %% Category-of-principal-fiber-bundles

\begin{proof}
  Let us prove the second note.
  A fiber bundle $\pi : \X \to \Q$ with fiber $\G$ is trivial if and only if there exists a diffeomorphism $\psi : \Q \times \G \to \X$ over the identity of $\Q$,
  that is,
  $\pi \circ \psi = \pr_1$ \art{The-category-of-smooth-surjections}.
  It is trivial as a $\G$-principal bundle if and only if this diffeomorphism is equivariant.
  In this case we get a smooth section $\sigma : q \mapsto \psi(q,\id_\G)$.
  Conversely,
  any smooth section $\sigma : \Q \to \X$ defines an equivariant diffeomorphism $\psi : \Q \times \G \to \X$ by $\psi(q,g) = g_\X(\sigma(q))$.
  For the third note.
  The proof of the assertion is a direct consequence of \xart{Diffeological-fiber-bundles-category}{point 1} and the characterization \art{Equivariant-plot-trivializations} of diffeological principal bundles by equivariant trivializations along plots.
\end{proof}

\begin{article}\artlabel{Equivariant plot trivializations}
  \addcontentsline{toc}{section}{\small\hspace{10pt} Equivariant plot trivializations}
  \label{Equivariant-plot-trivializations}
  Let $\pi : \X \to \Q$ be a $\G$-principal fiber bundle.
  There exists a family of {\em equivariant trivializations} along the plots.
  Let $\eq : \U \to \Q$ be a plot lifted in $\X$ by a plot $\P$,
  $\pi \circ \P = \eq$.
  We can restrict to some subdomain of $\U$ if necessary,
  it is always possible since $\pi$ is a subduction.
  Thus,
  the map
  $\psi : \U \times \G \to \eq^*(\X)$,
  defined by
  $$%
  \psi(r,g) = (r, g_\X(\P(r))),
  $$%
  is an equivariant diffeomorphism from $\U \times \G$ to $\eq^*(\X)$,
  over the identity of $\U$.
  Conversely,
  if there exits such a family of equivariant local trivializations along the plots,
  then $\pi : \X \to \Q$ is a principal fibration with structure group $\G$,
  that is,
  the action map $\eF : (x,g) \mapsto (x,g_\X(x))$ is an induction \art{Principal-diffeological-fiber-bundles}.
  
  \Note{1} For any two plots $\P : \U \to \X$ and $\P' : \U \to \X$ such that $\pi \circ \P = \pi \circ \P'$,
  there exists a plot $\gamma : \U \to \G$ such that $\P'(r) = \gamma(r)_\X(\P(r))$.
  
  \Note{2} Let $\pi' : \X' \to \Q'$ be a $\G'$-principal fiber bundle.
  Let $\phi$ be a smooth morphism of principal bundle from $\pi$ to $\pi'$ such that $\pi' \circ \phi = \varphi \circ \pi$,
  and $h : \G \to \G'$ be a smooth morphism such that $\phi(g_\X(x)) = h(g)_{\X'}(\phi(x))$.
  If $\varphi$ is a diffeomorphism and $h$ an isomorphism of diffeological groups,
  then $\phi$ is an isomorphism of diffeological principal fiber bundle.
\end{article} %% Equivariant-plot-trivializations

\begin{proof}
  We assume that there exists such a family of equivariant trivializations,
  which implies in particular that the action of $\G$ on $\X$ is free.
  Let $r \mapsto(\P(r), \P'(r))$ be a smooth parametrization defined on some real domain $\U$,
  with values in the image of the action map $\eF$.
  Thus,
  there exists a unique parametrization $r \mapsto g(r)$ of $\G$ such that $\P'(r) = g(r)_\X(\P(r))$,
  for all $r \in \U$.
  Now,
  $\pi \circ \P = \pi \circ \P'$ is a plot of $\Q$,
  let us denote it by $\eq$.
  Let $\phi : \eq^*(\X) \to \U \times \G$ be an equivariant trivialization.
  The plots $\P$ and $\P'$ define two sections of $\eq^*(\X)$,
  $r \mapsto (r, \P(r))$ and $r \mapsto (r, \P'(r))$.
  Thus,
  there exist two unique smooth parametrizations $\gamma$ and $\gamma'$ of $\G$ such that $\phi(r,\P(r)) = (r, \gamma(r))$ and $\phi(r,\P'(r)) = (r, \gamma'(r))$,
  then $(r, \gamma'(r)) = \phi(r,\P'(r)) = \phi(r, g(r)_\X(\P(r))) = g(r)_{\U \times \G}(\phi(r, \P(r))) = g(r)_{\U \times \G}(r, \gamma(r)) = (r, g(r) \gamma(r))$.
  Hence,
  $\gamma'(r) = g(r) \gamma(r)$ and finally $g(r) = \gamma'(r) \gamma(r)^{-1}$.
  Since $\gamma$ and $\gamma'$ are smooth and the inversion in $\G$ is smooth,
  $r \mapsto g(r)$ is a smooth parametrization,
  and $\eF$ is an induction.
  
  Let us check Note 1.
  Since $\pi : \X \to \Q$ is a principal fiber bundle,
  for all $r \in \U$,
  there exists a unique $\gamma(r) \in \G$ such that $\P'(r) = \gamma(r)_\X(\P(r))$.
  Let $\eq = \pi \circ \P = \pi \circ \P'$,
  and let us restrict $\eq^*(\X)$ over an open subset $\V$ of $\U$ where $\pr_1: \eq^*(\X) \to \U$ is trivial.
  Let $\psi : (\eq\restriction \V)^*(\X) \to \V \times \G$ be an equivariant trivialization.
  Let $r \mapsto g_r$ and $r \mapsto g'_r$ be the two plots of $\G$ such that $\psi(r,\P(r)) = (r, g_r)$ and $\psi(r,\P'(r)) = (r, g'_r)$.
  Thus,
  thanks to the equivariance property,
  $\psi(r,\P(r)) = \psi(r,\gamma(r)_\X(\P(r)))$,
  with $\gamma(r) = {g'_r}^{-1}g_r$.
  Therefore, $\gamma \restriction \V$ is a plot of $\G$,
  and so is $\gamma$.
  
  For Note 2,
  we check immediately that,
  under these assumptions $\varphi$ a diffeomorphism and $h$ an isomorphism,
  $\phi$ is bijective,
  that is,
  a smooth bijection.
  Now,
  let us check that $\phi^{-1}$ is smooth.
  Let $\P : \U \to \X'$ be a plot,
  thus $\varphi^{-1} \circ \pi' \circ \P$ is a plot of $\Q$.
  Hence,
  there exists a local smooth lift $\F$ such that $\pi \circ \F = \varphi^{-1} \circ \pi' \circ \P$ on the domain of $\F$.
  Thus,
  $\pi' \circ \phi \circ \F = \varphi \circ \pi \circ \F = \pi' \circ \P$,
  and thanks to Note 1,
  there exists a plot $\gamma'$ of $\G'$ such that $\phi \circ \F(r) = \gamma'(r)_{\X'}(\P(r))$.
  Let $\gamma(r) = h^{-1}(\gamma'(r))$, $\gamma$ is a plot of $\G$ and $\phi \circ \F(r) = h(\gamma(r))_{\X'}(\P(r))$,
  that is,
  $\phi (\gamma(r)_\X(\F(r))) = \P(r)$,
  then $\phi^{-1} \circ \P(r) = h(\gamma(r))_\X(\F(r))$.
  Hence,
  $\phi^{-1} \circ \P$ is locally smooth, thus smooth.
  Therefore, $\phi$ is an isomorphism of diffeological principal fiber bundle.
\end{proof}

\begin{article}\artlabel{Principal bundle attached to a fibrating groupoid}
  \addcontentsline{toc}{section}{\small\hspace{10pt} Principal bundle attached to a fibrating groupoid}
  \label{Principal-bundle-attached-to-a-fibrating-groupoid}
  Let $\KK$ be a fibrating groupoid \art{Fibrating-groupoids}.
  Let $\Q = \Obj(\KK)$ and pick a point $\eo \in \Q$,
  then define
  $$%
  \X = \source^{-1}(\eo), \  \pi = \target \restriction \X, \ \G = \KK_{\eo} = \car^{-1}(\eo,\eo),
  $$%
  and denote by $g \mapsto g_\X$ the natural left action of $\G$ on $\X$,
  $$%
  \text{for all } (g,x) \in \G \times \X, \  g_\X(x) = g \cdot x.
  $$%
  Then,
  the projection $\pi : \X \to \Q$ is a principal fibration with structure group $\G$.
  We shall call it the {\em principal fiber bundle attached  to $\KK$ at the point $\eo$}.
  
  \Note{1} The principal structure bundles $\pi$ and $\pi'$ of a fibrating groupoid,
  attached at different basepoints $\eo$ and $\eo'$,
  are naturally isomorphic.
  Let $f$ be any element of $\Mor_\KK(\eo',\eo)$,
  the map $x \mapsto f \cdot x$ from $\X$ to $\X'$ is such an isomorphism,
  with the structure groups $\G$ and $\G'$ naturally conjugated by $ g \mapsto f \cdot g \cdot f^{-1}$.
  
  \Note{2} Every diffeological fibration $p : \T \to \Q$ has a natural family of {\em principal structure bundles},
  the principal fiber bundle attached to the structure groupoid of $p$ \art{Structure-groupoid-of-a-smooth-surjection} at the various points of $\Q$.
  As well as the structure groupoid,
  they capture equivalently the smooth structure of the fibration $p$.
\end{article} %% Principal-bundle-attached-to-a-fibrating-groupoid

\begin{proof}
  First of all,
  let us notice that $\target$ is a subduction and then $\Q = \quotient{\X}{\G}$.
  Indeed,
  let $\P : \U \to \Q$ be a plot,
  thus $r \mapsto (\eo, \P(r))$ is a plot of $\Q \times \Q$.
  Since $\car$ is a subduction,
  this plot can be locally lifted into $\Mor(\K)$ by a plot $r \mapsto \bar\P(r)$,
  but since $\source(\bar\P(r)) = \eo$,
  $\bar\P$ is a plot of $\X$,
  lifting locally $\P$.
  Now,
  the action map $\eF : \X \times \G \to \X \times \X$,
  defined by $\eF(x,g) = (x,g_\X(x))$ just writes $\eF(x,g) = (x, g \cdot x)$,
  it is obviously smooth and its inverse is $\eF^{-1}(x,x') = (x, x' \cdot x^{-1})$,
  which is smooth thanks to the very property of diffeological groupoid.
  Hence,
  $\eF$ is an induction and $\pi$ is a principal fibration \art{Principal-diffeological-fiber-bundles}.
\end{proof}

\begin{article}\artlabel{Fibrations of groups by subgroups}
  \addcontentsline{toc}{section}{\small\hspace{10pt} Fibrations of groups by subgroups}
  \label{Fibrations-of-groups-by-subgroups}
  Let $\G$ be a diffeological group,
  and let $\H \subset \G$ be any subgroup.
  The projection $\pi : \G \to \G/\H$,
  is a principal diffeological fibration with structure group $\H$,
  where $\H$ acts on $\G$ by left or right multiplication and the coset $\G/\H$ is equipped with the quotient diffeology.
  
  \Note~In the example where $\G = \Torus^2 = \{(e^{ix},e^{iy}) \mid (x,y) \in \RR^2 \}$ and $\H = \{(e^{2\pi i t},e^{2\pi i \alpha t}) \mid t \in \RR\}$,
  with $\alpha \in \RR - \QQ$,
  the quotient $\G/\H$ is the {\em irrational torus $\Torus_\alpha$ of slope $\alpha$}\index{Irrational torus};
  see \exref{The-irrational-solenoid}.
  Since $\Torus_\alpha$ is not a manifold,
  the projection $\pi : \G \to \G/\H$ is a fibration in the category $\Diffeology$,
  but not in the category $\Manifolds$.
\end{article} %% Fibrations-of-groups-by-subgroups

\begin{proof}
  Let $\pi : \G \to \X$ be the projection from $\G$ onto its quotient $\X = \G/\H$.
  We choose $\H$ acting on $\G$ by left multiplication,
  \ie,
  $g \sim hg$, $g \in \G$ and $h \in \H$.
  Let $\P : \U \to \X$ be a plot and $r_0 \in \U$.
  By definition of the quotient diffeology,
  there exist an open neighborhood $\V$ of $r_0$ and a plot $\Q : \V \to \G$ such that $\pi \circ \Q = \P \restriction \V$.
  Then the map $\Phi : (r,h) \mapsto (r,h \cdot \Q(r))$,
  defined on $\V \times \H$,
  takes its values into $(\P\restriction \V)^*(\G) \subset \P^*(\G)$,
  and it is smooth because the multiplication is smooth.
  The inverse is given by $\Phi^{-1}(r,g) = (r, g \cdot \Q(r)^{-1})$,
  and it is smooth because the inversion is smooth.
  Thus,
  $\pr_1 : (\P\restriction \V)^*(\G) \to \V$ is trivial,
  hence $\pr_1 : \P^*(\G) \to \U$ is locally trivial,
  and the projection $\pi$ is a fibration \art{Fibrations-and-local-triviality-along-the-plots}.
  Now,
  let $\eL(g')(r,g) = (r,g'g)$ for every $r \in \U$,
  and $g,g' \in \G$.
  Then,
  for all $h \in \H$,
  we have $\Phi \circ \eL(h) = \eL(h) \circ \Phi$.
  The projection $\pi : \G \to \quotient{\G}{\H}$ is thus a principal fibration with structure group $\H$.
\end{proof}

\begin{article}\artlabel{Associated fiber bundles}
  \addcontentsline{toc}{section}{\small\hspace{10pt} Associated fiber bundles}
  \label{Associated-fiber-bundles}
  Let $\pi : \T \to \B$ be a principal fiber bundle \art{Principal-diffeological-fiber-bundles} with structure group $\G$,
  and let $\E$ be a diffeological space together with a smooth action of $\G$,
  that is,
  a smooth homomorphism $g \mapsto g_\E$ from $\G$ to $\Diff(\E)$.
  Let $\X = \T \times_\G \E$ be the quotient of $\T \times \E$ by the diagonal action of $\G$,
  $g_{\T \times \E} : (t,e) \mapsto (g_\T(t),g_\E(e))$.
  Let $p : \X \to \B$ be the projection $\class(t,e) \mapsto \pi(t)$.
  Then the projection $p$ is a diffeological fiber bundle,
  with fiber $\E$.
  We call it the fiber bundle {\em associated with $\pi$ by the action of $\G$ on $\E$}.
  We remark that if the action of $\G$ on $\E$ is trivial,
  then the associated fiber bundle is also trivial.
  The converse is given by the following proposition.
  \begin{itemize}
    \item[($\clubsuit$)] Every diffeological fiber bundle $p : \X \to \B$ is associated with its principal structure bundles \xart{Principal-bundle-attached-to-a-fibrating-groupoid}{Note 2}.
  \end{itemize}
  Precisely,
  let $\KK$ be the structure groupoid of the projection $p$.
  Let $\pi : \T \to \B$ be the principal fiber bundle attached to $\KK$ at some point $\eo \in \B$,
  and let $\E = p^{-1}(\eo)$ be the fiber.
  Let $\G = \pi^{-1}(\eo) = \Diff(\E)$ be the structure group,
  acting on $\T$ by right action,
  $(g,f) \mapsto f \circ g^{-1}$, $(g,f) \in \G \times \T$,
  and acting on $\E$ naturally by $(g,e) \mapsto g(e)$,
  $(g,e) \in \G \times \E$.
  Therefore,
  $p : \X \to \B$ is equivalent to the associated fiber bundle $\T \times_\G \E \to \B$.
  %
  \begin{center}
    \begin{tikzcd}[column sep=large, row sep=large, every label/.append style = {font = \small}]
      \T \times \E \arrow[r,"\class"] \arrow[d, swap, "\pr_1"] & \X = \T \times_\G \E \arrow[d, "p"]  \\
      \T \arrow[r, swap, "\pi"] & \B
    \end{tikzcd}
  \end{center}
  %
  \Note{1} If a principal fiber bundle $\pi : \T \to \B$ is trivial,
  then the associated fiber bundle $p : \T \times_\G \E \to \B$ is also trivial.
  Conversely,
  if a diffeological fiber bundle $p : \X \to \B$ is trivial,
  then its principal structure bundles,
  those characterizing its smooth structure,
  are trivial.
  
  \Note{2} If $f : \B' \to \B$ is a smooth map,
  then $f^*(\T \times_\G \E)$ is equivalent  to $f^*(\T) \times_\G \E$,
  where the action of $\G$ on $f^*(\T)$ is on the second factor.
  
  \Note{3} Every smooth section $\sigma$ of the associated bundle $p : \T \times_\G \E$ corresponds to a smooth equivariant map $s : \T \to \E$,
  that is,
  $s(g_\X(t)) = g_\E(s(t))$, with
  $\sigma(b) = \class(t, s(t))$,
  for any $t$ such that $\pi(t) = b$.
  This is summarized by
  $$%
  \Sec^\infty(p : \T \times_\G \E \to \B) \simeq \Eq^\infty(\T, \E),
  $$%
  with a clear meaning of the notations.
\end{article} %% Associated-fiber-bundles

\begin{proof}
  We prove the first note independently of the main proposition.
  Let $\pi = \pr_1 : \T = \Q \times \G \to \Q$.
  The diagonal action of $\G$ on $\T \times \E$ writes $g_{\T \times \E}(q,k,e) = (q,gk,g_\E(e))$,
  and the map $ \psi : \class(q,k,e) \mapsto (q,k_\E^{-1}(e))$ from $\T \times_\Q \E$ to $\Q \times \E$ is a realization of the quotient space $\X = \T \times_\G \E$.
  
  For the second note,
  the pullback $f^*(\T \times_\G \E)$ and $f^*(\T) \times_\G \E$ are equivalent to the quotient of $\{ (b', t, e)  \in \B' \times \T \times \E \mid f(b') = \pi(t) \}$ by the diagonal action of $\G$ on the last two variables.
  
  Let us check now that the associated bundle $p : \T \times_\G \E \to \B$ is a diffeological fiber bundle.
  Let $\P$ be a plot of $\B$.
  Thanks to Note 2,
  the pullback $\P^*(\T \times_\G \E)$ is equivalent to $\P^*(\T) \times_\G \E$,
  but $\P^*(\T)$ is locally trivial \art{Equivariant-plot-trivializations},
  thus $\P^*(\T) \times_\G \E$ is also locally trivial,
  thanks to Note 1.
  Therefore,
  $\P^*(\T \times_\G \E)$ is locally trivial,
  and $p : \T \times_\G \E \to \B$ is a diffeological fiber bundle \art{Fibrations-and-local-triviality-along-the-plots}.
  
  Conversely,
  let $p : \X \to \B$ be a diffeological fiber bundle.
  Let $\pi : \T \to \B$ be a principal bundle attached to its structure groupoid $\KK$,
  at some point $\eo \in \B$ \art{Principal-bundle-attached-to-a-fibrating-groupoid},
  \art{Structure-groupoid-of-a-smooth-surjection},
  and let $\E = \X_\eo$.
  Thus, $\T = \{ f \in \Diff(\E, \T_b) \mid b \in \B \}$ and $\G = \T_\eo = \Diff(\E)$ is the structure group.
  Now,
  let $\phi : \T \times \E \to \X$ be defined by $\phi(f,e) = f(e)$.
  Since $p$ is a diffeological fibration,
  $p$ is surjective and all the fibers are equivalent.
  Now,
  $f' = f \circ g^{-1}$ and $e' = g(e)$ if and only if $\phi(f,e) = \phi(f',e')$.
  Then,
  $\phi$ is the composite $\varphi \circ \class$,
  where $\class : \T \times \E \to \T \times_\G \E$ is the projection,
  and $\varphi : \T \times_\G \E \to \X$ is a bijection.
  But $\phi$ is a subduction.
  Indeed,
  let $\P$ be a plot of $\X$ and $\beta = p \circ \P$,
  since $\beta$ is a plot of $\B$ and $\pi : \T \to \B$ is a subduction,
  $\beta$ lifts locally in a plot $\tau$ of $\T$,
  in particular $\tau(r) \in \Diff(\E,\X_{\beta(r)})$.
  Then,
  since $\P(r) \in \X_{\beta(r)}$,
  and since $\tau$ and $\P$ are smooth,
  $r \mapsto e(r) = \tau(r)^{-1}(\P(r))$ is smooth.
  Hence, $r \mapsto (\tau(r),e(r))$ is a plot,
  lifting locally $\P$  along $\phi$.
  Therefore,
  $\phi$ is a subduction.
  Next,
  since $\phi$ and $\class$ are subductions,
  $\varphi$ is a diffeomorphism and it projects onto the identity of $\B$.
  Therefore,
  $\X$ is equivalent to the associated fiber bundle $\T \times_\G \E \to \B$.
  
  For the third note,
  if $s$ is an equivariant map,
  it is clear that the map $\bar\sigma : t \mapsto \class(t,s(t))$ from $\T$ to $\T \times_\G \E$,
  is invariant by the action of $\G$,
  that is,
  $\bar\sigma(g_\T(t)) = \bar\sigma(t)$.
  And since $\pi : \T \to \B$ is a subduction, there exists a smooth map,
  which is clearly a section,
  $\sigma : \B \to \T \times_\G \E$ such that $\sigma \circ \pi = \bar\sigma$.
  Conversely,
  let $\sigma$ be a smooth section,
  since the action of $\G$ on $\T$ is free,
  there exists a unique $s(t) \in \E$ such that $\sigma(\pi(t)) = \class(t,s(t))$,
  and the map $s$ is clearly equivariant.
  Now,
  let us show that $s$ is smooth.
  Let $r \mapsto t_r$ be a plot of $\T$,
  since $\class$ is a subduction,
  there exists a local lift $r \mapsto (t'_r,e'_r)$ such that $\bar\sigma(t_r) = \class(t'_r,e'_r)$,
  but $\pi(t'_r) = \pi(t_r)$,
  then $t'_r = (g_r)_\X(t_r)$,
  where $r \mapsto g_r$ is a plot of $\G$,
  because $\pi$ is locally trivial along the plots.
  Now,
  let
  $e_r = (g_r^{-1})_\E(e'_r)$,
  then $r \mapsto (t_r,e_r)$ is another smooth local lift of $\bar\sigma$,
  and by unicity $e_r = s(t_r)$,
  hence $s$ is smooth.
\end{proof}

\begin{article}\artlabel{Structures on fiber bundles}
  \addcontentsline{toc}{section}{\small\hspace{10pt} Structures on fiber bundles}
  \label{Structures-on-fiber-bundles}
  Let $p : \X \to \B$ be a diffeological fiber bundle.
  Let $\KK$ be the associated structure groupoid \art{Structure-groupoid-of-a-smooth-surjection}.
  We call the {\em structure}\index{Structure of fiber bundle} of the fiber bundle any fibrating subgroupoid $\Gamma$ of $\KK$.
  
  1. Every principal fiber bundle $\pi : \T \to \B$,
  attached to $\Gamma$ at some point $\eo \in \B$ \art{Principal-bundle-attached-to-a-fibrating-groupoid},
  is a {\em principal fiber bundle structure} for the $\Gamma$-structure,
  and the structure group $\G$ of $\pi$ is the {\em structure group}\index{Structure group} of the {\em structure groupoid}\index{Structure groupoid} $\Gamma$.
  
  2. Together with a structure $\Gamma$,
  a fiber bundle $p : \X \to \B$ is called a {\em structured fiber bundle}.
  The structured fiber bundles form a category,
  let $p': \X' \to \B'$ be a fiber bundle structured by $\Gamma'$.
  A {\em structured morphism} $\Phi : \X \to \X'$ is a fiber bundle morphism \art{Diffeological-fiber-bundles-category} which defines a functor from $\Gamma$ to $\Gamma'$ by
  \begin{equation}
    \renewcommand{\theequation}{$\diamondsuit$}
    f \mapsto f' = (\Phi \restriction \X_{b_2}) \circ f \circ (\Phi \restriction \X_{b_1})^{-1},
    \text{ for all } f \in \Mor_\Gamma(b_1,b_2).
  \end{equation}
  For every morphism $f \in \Mor_{\Gamma}(b_1,b_2)$,
  $f'$ belongs to $\Mor_{\Gamma'}(\phi(b_1),\phi(b_2))$,
  where $\phi : \B \to \B'$ is the factorization of $\Phi$, $p'\circ \Phi = \phi \circ p$.
  
  \Example~The main example of structured bundle,
  except the structure of principal fiber bundle that we have already seen,
  is the structure of vector fiber bundle for which the structure consists in a groupoid of linear maps.
  
  \Note{1} The notion of structure introduced in this paragraph is the diffeological version of F-structure,
  for topological fiber bundles,
  one can find in \cite{FR81}.
  The structure groupoid $\Gamma$ replaces the family of ``marked'' homeomorphisms.
  
  \Note{2} Every diffeological fiber bundle has a favorite structure,
  the {\em smooth structure},
  defined by the groupoid $\KK$ itself.
  The structured smooth morphisms ($\diamondsuit$) coincide with the fiber bundles morphisms defined in \art{Diffeological-fiber-bundles-category}.
  
  \Note{3} Principal fiber bundles deserve a special attention.
  We define the {\em principal fiber bundle structure} of any principal fiber bundle $p : \X \to \B$,
  with structure group $\G$,
  as the groupoid $\KK_\G$ defined in \art{Principal-diffeological-fiber-bundles}.
  So,
  every principal structure fiber bundle $\pi : \T \to \B$,
  attached at any point $\eo \in \B$ \art{Principal-bundle-attached-to-a-fibrating-groupoid},
  is equivalent to $p: \X \to \B$,
  and the principal structure morphisms ($\diamondsuit$) coincide with the morphisms of principal fiber bundles.
  
  \Note{4} Let $f : \B' \to \B$ be a smooth map.
  There exists,
  on the pullback $p' : f^*(\X) \to \B'$ \art{Diffeological-fiber-bundles-category},
  a natural structure $\Gamma'$,
  induced by $\Gamma$,
  defined by
  $$%
  \Obj(\Gamma') = \B', \text{ and }, \Mor_{\Gamma'}(b'_1,b'_2) = \Mor_{\Gamma}(f(b'_1),f(b'_2)).
  $$%
  
  \Note{5} Let $p : \X \to \B$ be a fiber bundle,
  let $b$ be a point of $\B$,
  let $\F = p^{-1}(b)$,
  and let $\Gamma$ be a subgroupoid of its smooth structure groupoid $\KK$.
  The groupoid $\Gamma$ defines a structure on $p$ if and only if,
  for any plot $\P : \U \to \B$,
  there exists a system of local trivializations $\{\Psi_i\}_{i \in \cI}$ of the pullback $\pr_1 : \P^*(\X) \to \U$ such that,
  for all $(r,u) \in \U_i \times \F = \Def(\Psi_i)$,
  $$%
  \Psi_i(r,\xi) = (r, \psi_i(r)(\xi)), \text{ with }
  \left\{ \begin{array}{l}
    \psi_i \in \Cinfty(\U_i,\Mor(\Gamma)), \\[.8ex]
    \psi_i(r) \in \Mor_\Gamma(\F, \X_{\P(r)}).
  \end{array}
  \right.
  $$%
  This is the specialization,
  in the presence of a $\Gamma$-structure,
  of the criterion of local triviality along the plots for diffeological fiber bundles \art{Fibrations-and-local-triviality-along-the-plots}.
  
  \Note{6} As we have seen,
  every diffeological fiber bundle is associated with any one of its smooth structure principal fiber bundles \xart{Associated-fiber-bundles}{($\clubsuit$)},
  by the same process by which every structured diffeological fiber bundle is associated with any one of its structure principal bundles.
\end{article} %% Structures on fiber bundles

\begin{proof}
  The assertion of the second note is clear,
  the restriction of every global diffeomorphism to a subspace is a diffeomorphism from the subspace to its image.
  Now,
  let $p : \X \to \B$ be a principal fiber bundle with group $\G$.
  Let $\car : \X \times_\G \X \to \B \times \B$ be the characteristic map of the groupoid $p \times_\G p$,
  equivalent to $\KK_\G$ \xart{Principal-diffeological-fiber-bundles}{Note 1}.
  The total space of the principal structure bundle,
  attached at $\eo \in \B$,
  is $\T = \car^{-1}(\{\eo\} \times \B)$,
  that is,
  the subspace $\T = \{ \class(x,x') \mid x,x' \in \X, \text{ and } p(x) = \eo\}$,
  thus $\T = \X_\eo \times_\G \X$.
  But,
  by definition of a principal fiber bundle,
  the action map $\eF : \X \times \G \to \X \times \X$ is an induction \art{Principal-diffeological-fiber-bundles} and $\X_\eo \times \X$ is equivalent to $\G \times \X$,
  where the diagonal action of $\G$ transmutes into $g_{\G \times \X}(g',x') = (gg',g_\X(x'))$.
  Hence,
  the quotient $\T$ is equivalent to $\X$,
  thanks to the subduction $(g',x') \mapsto {g'}_\X^{-1}(x')$.
  That proves the first part.
  
  Next,
  let us consider a structure morphism $\Phi$,
  from $p : \X \to \B$ to $p' : \X' \to \B'$.
  Since,
  by definition,
  $\Phi$ maps fibers into fibers and since the action map $\eF' : \X' \times \G' \to \X' \times \X'$ is an induction \art{Principal-diffeological-fiber-bundles},
  there exists a smooth map $h : \G \times \X \to \G'$ such that $\Phi(g_\X(x)) = h(g,x)_{\X'}(\Phi(x))$.
  Thus,
  for every $f \in \Mor(\KK_\G)$ with $f : \X_{b_1} \to \X_{b_2}$,
  $\Phi(g_\X(f(x))) = h(g,f(x))_{\X'}(\Phi(f(x)))$.
  But $g_\X \circ f = f \circ g_\X$,
  then $\Phi(f(g_\X(x))) = h(g,f(x))_{\X'}(\Phi(f(x)))$.
  Now,
  by hypothesis,
  there exists $f' \in \Mor(\KK_{\G'})$ such that $\Phi_{b_2} \circ f = f' \circ \Phi_{b_1}$,
  where $\Phi_{b_i} = \Phi \restriction \X_{b_i}$,
  thus $f'(\Phi(g_\X(x))) = h(g,f(x))_{\X'}(f'(\phi(x))) = f'(h(g,f(x))_{\X'}(\phi(x)))$.
  Then,
  since $f'$ is a diffeomorphism,
  $\Phi(g_\X(x)) = h(g,f(x))_{\X'}(\phi(x))$,
  but $\Phi(g_\X(x)) = h(g,x)_{\X'}(\phi(x))$,
  therefore $h(g,f(x))_{\X'}(\phi(x)) = h(g,x)_{\X'}(\phi(x))$.
  But the action of $\G'$ is free,
  then $h(g,f(x)) = h(g,x)$.
  Now,
  for every pair of points $x,y$ in $\X$,
  there exists a morphism $f$ such that $f(x) = y$,
  thus $h(g,x)$ does not depend on $x$,
  and $\Phi(g_\X(x)) = h(g)_{\X'}(\Phi(x))$.
  It is then clear that $h$ is a homomorphism.
\end{proof}

\begin{article}\artlabel{Space of structures of a diffeological fiber bundle}
  \addcontentsline{toc}{section}{\small\hspace{10pt} Space of structures of a diffeological fiber bundle}
  \label{Space-of-structures-of-a-diffeological-fiber-bundle}
  We know that every diffeological fiber bundle $p: \X \to \B$,
  with some fiber $\F$,
  is associated with a diffeological principal fiber bundle $\pi : \T \to \B$,
  with some group $\G$ as structure group \xart{Structures-on-fiber-bundles}{Note 6},
  and with structure groupoid $\Gamma$ equivalent to $\pi \times_\G \pi$ \xart{Principal-diffeological-fiber-bundles}{Note 1}.
  For the smooth structure,
  the group $\G$ is the full group $\Diff(\F)$.
  So,
  equipping the fiber bundle $p$ with a special structure consists in reducing the principal bundle $\pi$ to a principal subbundle for some subgroup $\H \subset \G$.
  
  \Definition A {\em reduction} of a principal fiber bundle $\pi : \T \to \B$,
  with structure group $\G$,
  to a subgroup $\H \subset \G$ is any $\H$-principal subbundle $\pi_\S : \S \to \B$,
  where $\S \subset \T$.
  
  Let us consider the quotient $\pi/\H : \T/\H \to \B$,
  where $\T/\H$ is the quotient of $\T$ by the action of $\H$,
  induced from the action of $\G$.
%
  \begin{center}
    \begin{tikzcd}[column sep=large, row sep=large, every label/.append style = {font = \small}]
      \T \arrow[dr, swap, "\pi"] \arrow[rr,"q"] & {} & \T/\H \arrow[dl, "\pi/\H"]  \\
      {} & \B  & {}
    \end{tikzcd}
  \end{center}
%
  1. Every reduction $\S$ of $\pi$ to $\H$ corresponds to a smooth section $\sigma : \B \to \T/\H$ and conversely,
  $\S = q^{-1}(\Val(\sigma))$.
  
  2. The projection $\pi/\H : \T/\H \to \B$ is a diffeological fiber bundle with fiber $\G/\H$.
  It is equivalent to the associated bundle $\pi_\H : \T \times_\G (\G/\H) \to \B$,
  where $\G$ acts on $\G/\H$ by $g_{\G/\H}(\H g' ) = \H g'g^{-1}$.
  Hence,
  the space of reductions of $\pi$ to $\H$ is in bijection with the space $\Eq^\infty(\T, \G/\H)$ \xart{Associated-fiber-bundles}{Note 3}.
\end{article} %% Space-of-structures-of-a-diffeological-fiber-bundle

\begin{proof}
  First of all,
  let us check that $\T/\H$ is a diffeological bundle.
  Let $\P : \U \to \B$ be a plot,
  the pullback $\P^*(\T/\H)$ is equivalent to $\P^*(\T)/\H$.
  Now let $\Phi : (r,g) \mapsto (r, \phi_r(g))$ be a local trivialization of $\pr_1 : \P^*(\T) \to \U$,
  thus the map $\Psi : (r,\H g) \mapsto (r, \H_\T(\phi_r(g)))$ is a local trivialization of $\P^*(\T)/\H \sim \P^*(\T/\H)$.
  Therefore,
  $\pi/\H$ satisfies the local triviality condition along the plots \art{Fibrations-and-local-triviality-along-the-plots}.
  
  Now,
  let $q$ be the projection from $\T$ to $\T/\H$,
  and let $\sigma$ be a smooth section of $\pi/\H$.
  On the one hand,
  it is clear that,
  since $q(h_\T(t)) = q(t)$,
  the preimage $\S = q^{-1}(\Val(\sigma))$ is invariant by $\H$.
  Next,
  let $t$ and $t'$ be two points of $\S$ such that $\pi(t) = \pi(t')$,
  since $\pi/\H \circ q = \pi$, $\pi/\H(q(t)) = \pi/\H(q(t'))$,
  but $q(t), q(t') \in \sigma(\B)$ and $\sigma$ is a section,
  then $q(t)= q(t')$.
  Thus,
  there exists $h \in \H$ such that $t' = h_\T(t)$,
  and $\S$ is a $\H$-principal subbundle,
  that is,
  a reduction of $\pi$ to $\H$.
  
  Conversely,
  let $\S \subset \T$ be a $\H$-subbundle.
  Because $\S$ is $\H$-invariant,
  there exists a map $\sigma : \B \to \T/\H$ such that $\sigma \circ (\pi \restriction \S) = q \restriction \S$.
  This map is smooth because $\pi$ is a subduction.
  But since,
  for all $b \in \B$, $\pi^{-1}(b) \cap \S$ is only one $\H$-orbit,
  $\sigma$ is a section of $\pi/\H$.
  
  For the second note,
  let us remark that $\T$ is also equivalent to $\T \times_\G \G$,
  where $\G$ acts on the product by $g'_{\T \times \G} (t,g) = (g'_\T(t), g{g'}^{-1})$,
  thanks to the factorization of the map $\phi : (t,g) \mapsto g_\T(t)$.
  \begin{center}
    \begin{tikzcd}[column sep=large, row sep=large, every label/.append style = {font = \small}]
      \T \times \G \arrow[r,"\phi"] \arrow[d, swap, ""] & \T \times_\G \G \sim \T \arrow[d, ""]  \\
      \T \times \G/\H \arrow[r, swap, "\varphi"] & (\T \times_\G \G)/\H \sim \T \times_\G (\G/\H) \sim \T/\H
    \end{tikzcd}
  \end{center}
  Then,
  taking on both sides of the quotient by the left action of $\H$,
  on the second factor,
  we get a factorization $\varphi$ of $\phi$ which realizes the quotient $\T \times_\G (\G/\H) \sim (\T \times_\G \G)/\H$ as the quotient $\T/\H$.
\end{proof}

%%%%%%%%%%%%%%%%%%%%%%%%%%%%%%%%%%%%%%%%%%%%%%%%%%%%%%%%%%
%
%   Exercises
%
%%%%%%%%%%%%%%%%%%%%%%%%%%%%%%%%%%%%%%%%%%%%%%%%%%%%%%%%%%

\Exercises

\begin{exercise}[Polarized smooth functions]
  \label{Polarized-smooth-functions}
  Let $\P^1(\RR)$ be the space of all lines in $\RR^2$ passing through the origin.
  Let
  $$%
  \T = \{ (\DD, f) \in \P^1(\RR) \times \Cinfty(\RR^2,\CC) \mid f(x + v) = f(x), \ \forall x \in \RR^2, \ \forall v \in \DD \}.
  $$%
  In other words,
  if $(\DD, f) \in \T$,
  then $f$ is constant on all affine lines parallel to $\DD$.
  First,
  describe a good diffeology for $\P^1(\RR)$.
  Then,
  show that $\pi : \T \to \P^1(\RR)$,
  with $\pi(\DD,f) = \DD$,
  is a nontrivial fiber bundle with fiber $\E = \Cinfty(\RR,\CC)$.
\end{exercise} %% Polarized-smooth-functions

\begin{exercise}[Playing with $\SO(3)$]
  \label{Playing-with-SO(3)}
  Let $\SO(3)$ be the group of direct rotations of the space $\RR^3$,
  that is,
  the group of real $3 \times 3$ matrices $\A$ such that $\bar\A \A = \id_{\RR^3}$ and $\det(\A) = +1$,
  where the bar denotes the transposition.
  Let $\ee_1 \in \RR^3$ be the vector of coordinates $(1,0,0)$,
  and let $\SO(2,\ee_1)$ be the stabilizer of $\ee_1$,
  that is,
  the subgroup of elements $k \in \SO(3)$ such that $k\ee_1 = \ee_1$.
  
  \Question{1)} Show that $p : \SO(3) \to \S^2$ defined by $\A \mapsto \A\ee_1$ is a principal fibration with structure group $\SO(2,\ee_1)$,
  for the action $(\A,k) \mapsto \A k^{-1}$,
  where $(\A,k) \in \SO(3) \times \SO(2,\ee_1)$.
  
  \Question{2)} Show that the tangent space $\T\S^2$ of the 2-sphere,
  that is,
  $\T\S^2 = \{ (u,v) \in \S^2 \times \RR^3 \mid \bar u v = 0 \}$,
  is a fiber bundle with fiber $\RR^2$,
  for the first projection $\pr_1 : (u,v) \mapsto u$.
  Use the first question,
  and consider the associated fiber bundle $\SO(3) \times_{\SO(2,\ee_1)} \ee_1^\perp$,
  where $\SO(2,\ee_1)$ acts naturally on the subspace $\ee_1^\perp$ of $\ee_1$-orthogonal vectors.
\end{exercise} %% Playing-with-SO(3)

\begin{exercise}[Homogeneity of manifolds]
  \label{Homogeneity-of-manifolds}
  Consider $\RR^n \times \RR$,
  let $\eC$ be a cylinder $\Ball \times \closedinterval{-\delta,1+\delta} \subset \RR^n \times \RR$,
  where $\Ball$ is a ball,
  centered at $0_n \in \RR^n$,
  of radius $r>0$,
  and $\delta>0$.
  Build a diffeomorphism of $\RR^n \times \RR$ which coincides with the identity out of $\eC$ and maps $(0_n,0)$ to $(0_n,1)$.
  Next,
  let $\M$ be a connected Hausdorff manifold,
  deduce first that for any two points $x$ and $x'$ belonging to the values of a chart $\F : \cB \to \M$,
  where $\cB$ is some open ball,
  there exists a compactly supported diffeomorphism mapping $x$ to $x'$,
  and extend this result for any two points of $\M$.
  Let $\CSDiff(\M)$ be the group of compactly supported diffeomorphisms,
  equipped with the functional diffeology \art{Functional-diffeology-on-groups-of-diffeomorphisms}.
  Let $\hat x_0 : \CSDiff(\M) \to \M$ be the map $\hat x_0(\varphi) = \varphi(x_0)$, $x_0 \in \M$.
  Prove that $\hat x_0$ is a principal diffeological fibration with structure group $\CSDiff(\M,x_0)$,
  the subgroup of diffeomorphisms preserving $x_0$, and thus that $\M$ is a homogeneous space of $\CSDiff(\M,x_0)$.
  Remark that $\M$ is by consequence a homogeneous space of its whole group of diffeomorphisms,
  as well.
  Can you imagine the tangent space of $\M$ as an associated fiber bundle?
\end{exercise} %% Homogeneity-of-manifolds

%%%%%%%%%%%%%%%%%%%%%%%%%%%%%%%%%%%%%%%%%%%%%%%%%%%%%%%%%%
%% MARK: Homotopy of Diffeological Fiber Bundles
%%%%%%%%%%%%%%%%%%%%%%%%%%%%%%%%%%%%%%%%%%%%%%%%%%%%%%%%%%

\section*{Homotopy of Diffeological Fiber Bundles}
\label{Homotopy-of-diffeological-fiber-bundles}

\begin{sechead}
  The main result concerning homotopy and diffeological fiber bundles is the exact homotopy sequence which links the homotopy of the base,
  and the homotopy of the fiber,
  with the homotopy of the total space.
  For the notations used thereafter,
  let us recall that $\Paths(\X) = \Cinfty(\RR,\X)$ denotes the set of paths in $\X$,
  with $\0(\gamma) = \gamma(0)$, $\1(\gamma) = \gamma(1)$,
  and $\ends(\gamma) = (\gamma(0), \gamma(1))$,
  for all paths $\gamma$.
  Let us also recall that $\X$ is said to be connected if $\ends$ is surjective on $\X \times \X$,
  and that $\ends$ is then a subduction \art{Pathwise-connectedness}.
  Let us finally recall that $\Loops(\X)$ denotes the subspace of  $\ell \in \Paths(\X)$ such that $\0(\ell) = \1(\ell)$,
  and that for every point $x$ in $\X$,
  $\Loops(\X,x)$ denotes the subspace of $\ell \in \Paths(\X)$ such that $\0(\ell) = \1(\ell) = x$,
  and more generally that $\Paths(\X,x,\star)$ denotes the subspace of $\gamma \in \Paths(\X)$ such that $\0(\gamma) = x$.
  We end with the following inclusions:
  $\Loops(\X,x) \subset \Paths(\X,x,\star) \subset \Paths(\X)$ and  $\Loops(\X,x) \subset \Loops(\X) \subset \Paths(\X)$.
\end{sechead}

\begin{article}\artlabel{Triviality over global plots}
  \addcontentsline{toc}{section}{\small\hspace{10pt} Triviality over global plots}
  \label{Triviality-over-global-plots}
  Every diffeological fibration with base space $\RR^n$ is trivial,
  for all $n \in \NN$.
  Equivalently,
  let $\pi : \T \to \X$ be some principal fiber bundle.
  Every global plot $f : \RR^n \to \X$ admits a global smooth lift $\F : \RR^n \to \T$,
  that is,
  $\F \in \Cinfty(\RR^n,\T)$ and $\pi \circ \F = f$.
  
  \Note{1} This property is closely related to what is called,
  in some other contexts and for some other kind of objects,
  the {\em homotopy lifting property}.
  This is a crucial property in the establishment of the exact homotopy sequence of the diffeological fiber bundles \art{Exact-homotopy-sequence-of-a-diffeological-fibration}.
  
  \Note{2} Actually the proof shows that any diffeological fiber bundle,
  over an open ball of a real vector space,
  is trivial.
\end{article} %% Triviality over global plots

\begin{proof}
  We shall prove this proposition in a few steps.
  
  \Lemma{1} Let $\pi : \T \to \X \times \openinterval{a,b}$ be a principal fibration with structure group $\G$.
  Let $a < b' < a' < b$ such that $\pi$ is trivial over $\X \times \openinterval{a,a'}$ and trivial over $\X \times \openinterval{b',b}$.
  Then,
  $\pi$ is trivial over $\X \times \openinterval{a,b}$.
  
  {\em Proof of Lemma 1.} Let us denote by $[g \mapsto g_\T] \in \DHom(\G, \Diff(\T))$ the action of $\G$ on $\T$.
  Thanks to \art{Principal-diffeological-fiber-bundles} there exist two sections $s_1 : \X \times \openinterval{a,a'} \to \T$ and $s_2 : \X \times \openinterval{b',b} \to \T$,
  and the isomorphisms (equivariant diffeomorphisms) associated write $\Phi_i(y,g) = g_\T(s_i(y))$,
  where $y=(x,t)$ is in the domain of definition of the section $s_i$.
  Now,
  thanks to the equivariance of the trivializations $\Phi_1$ and $\Phi_2$,
  $\Phi_1^{-1} \circ \Phi_2 : \X \times \openinterval{b',a'} \times \G \to \X \times \openinterval{b',a'} \times \G$  writes $\Phi_1^{-1} \circ \Phi_2(y,g) = (y, g \gamma(y))$,
  and
  $$%
  \gamma : \X \times \openinterval{b',a'} \to \G,
  \text{ defined by } \gamma(y) = \pr_2 \circ \Phi_1^{-1} \circ \Phi_2(y,\id_\G)
  $$%
  is clearly smooth.
  Then,
  for all $y=(x,t) \in \X \times \openinterval{b',a'}$, $\Phi_2(y,g) = \Phi_1(y,g\gamma(y))$,
  that is,
  $g_\T(s_2(y)) = g_T(\gamma(y)_\T(s_1(y)))$,
  thus $s_2(y) = \gamma(y)_\T(s_1(y))$.
  Now,
  let $c \in \openinterval{b',a'}$,
  and let $\mu$ be a real smooth function, described in Figure \ref{fig-extension-function},
  satisfying
  $$%
  \mu : \openinterval{a - \varepsilon,a' + \varepsilon} \to \openinterval{b' - \varepsilon , a' + \varepsilon}, \text{ with }
  \left\{
  \begin{array}{l}
    \mu(a) = b', \text{ and } \mu(a') = a', \\
    \mu \restriction {\openinterval{c,a'}} =
    \id_{\openinterval{c,a'}},
  \end{array}
  \right.
  $$%
  where $\varepsilon$ is some positive number.
  %%###########
  \begin{figure}[tb]
    \centerline{\includegraphics{Figures-PDF/fig-extension-function}}
    \caption{The extension function.}
    \label{fig-extension-function}
  \end{figure}
  %%###########
  Now,
  since for all $(x,t) \in \X \times \openinterval{a,a'}$,
  $(x, \mu(t)) \in \X \times \openinterval{b',a'} = \Dom(\gamma)$,
  the map $(x,t) \mapsto \gamma(x, \mu(t))$ is defined on $\X \times \openinterval{a,a'}$,
  and since $s_1$ is also defined on $\X \times \openinterval{a,a'}$,
  we have a new section given by
  $$%
  s'_1 : \X \times \openinterval{a,a'} \to \T, \text{ with } s'_1(x,t) = \gamma(x,\mu(t))_\T(s_1(x,t)).
  $$%
  Then,
  let us define
  $$%
  s'_2 : \X \times \openinterval{c,b} \to \T, \text{ with } s'_2 = s_2 \restriction \X \times \openinterval{c,b}.
  $$%
  The common domain of $s'_1$ and $s'_2$ is $\X \times \openinterval{c,a'}$.
  Now,
  for every $(x,t) \in \X \times \openinterval{c,a'}$,
  $s'_1(x,t) = \gamma(x,\mu(t))_\T(s_1(x,t)) = \gamma(x,t)_\T(s_1(x,t)) = s_2(x,t) = s'_2(x,t)$,
  and the sections $s'_1$ and $s'_2$ coincide on the intersection (which is D-open) of their domains.
  Hence,
  they are the restriction,
  to their domains,
  of a smooth section $s : \X \times \openinterval{a,b} \to \T$.
  Therefore, the principal fibration $\pi$ is trivial.
  
  \Lemma{2} Every principal fibration $\pi : \T \to \RR^n$,
  for any structure group $\G$,
  is trivial over every paracompact open subset of $\RR^n$.
  In particular,
  it is trivial over every ball $\cB$ of any radius,
  centered at the origin $0 \in \RR^n$.
  
  {\em Proof of the Lemma 2.} Since the identity of $\RR^n$ is a plot,
  the fibration $\pi : \T \to \RR^n$ is locally trivial \art{Fibrations-and-local-triviality-along-the-plots}.
  Thus,
  there exists a family  $\{s_i : \U_i \to \T \}_{i \in \cI}$ of smooth sections of $\pi$ whose domains are a cover of $\RR^n$,
  that is,
  $$%
  \Dom(s_i) = \U_i, \  \bigcup_{i \in \cI} \U_i = \RR^n, \  s_i \in \Cinfty(\U_i,\T), \text{ and } \pi \circ s_i = \id_{\U_i}.
  $$%
  Let $\K_{\L}$ be the cube with side $2\L$ centered at the origin $0 \in \RR^n$, $\K_{\L} = \closedinterval{-\L,\L}^n$.
  Let us consider the subcover $\{\U_a\}_{a \in \cA}$,
  $\cA \subset \cI$,
  of $\K_{\L}$ defined by $\U_a \cap \K_{\L} \neq \varnothing$.
  Since $\K_{\L}$ is compact,
  there exists a finer cover of $\K_{\L}$ made of small cubes of side $\ell$ arranged along a lattice directed by the canonical basis,
  such that each cube meets only its closest neighbours,
  and such that their union is a cube $\K'_{\L}$ containing $\K_{\L}$;
  see Figure \ref{fig-Cubes-filling-cubes}.
  Such a cover can be built using the Lebesgue number of the cover \cite{EOM02}.
  Indeed,
  for any compact metric space (here $\K_{\L}$) and an open cover (of $\K_{\L}$) is given,
  there exists a number $\ell$ such that every subset of $\K_{\L}$ of diameter $\ell$ is contained in some member of the cover.
  Let us denote this cover of $\K_{\L}$ by $\{\C_{i_1, \ldots, i_n}\}_{i_k = 1}^\N$,
  then
  %%###########
  \begin{figure}[tb]
    \centerline{\includegraphics{Figures-PDF/fig-bundle-local-covering}}
    \caption{Cubes filling cubes.}
    \label{fig-Cubes-filling-cubes}
  \end{figure}
  %%###########
  $$%
  \K'_{\L} = \bigcup_{i_1, \ldots, i_n = 1}^\N \C_{i_1, \ldots, i_n}, \ \text{and} \ \forall i_1, \ldots, i_n, \ \exists a \in \cA, \ \C_{i_1 \cdots i_n} \in \U_a.
  $$%
  The restriction of the fiber bundle $\pi : \T \to \RR^n$ over each of these cubes is trivial.
  Thus,
  we can apply recursively the previous lemma to the first two cubes of a row $\C_{1,1,\ldots,1}$ and $\C_{2,1,\ldots,1}$,
  then to the $\C_{1,1,\ldots,1} \cup \C_{2,1,\ldots,1}$ and $\C_{3,1,\ldots,1}$ and so on,
  until $\C_{\N,1,\ldots,1}$,
  we construct a section of $\pi$ over the union $\bigcup_{i=1}^\N \C_{i,1, \ldots, 1}$.
  Repeating this process on the first index $i_1= 1,\ldots,\N$,
  we construct a smooth section over every domain of this type:
  $\C_{\star,i_2,\ldots,i_n} = \bigcup_{j=1}^\N \C_{j,i_2,\ldots,i_n}$.
  Then,
  by recurrence,
  we construct a smooth section of the fibration $\pi : \T \to \RR^n$ over the cube $\K'_{\L}$.
  Thus,
  the fibration is trivial over $\K'_{\L}$ and therefore on every relatively compact open subset.
  We also say that $\pi$ is trivial over the closed cube $\K_{\L}$,
  and this is what we mean when we say in general that the fibration is trivial over a closed subset $\K \subset \RR^n$.
  In particular,
  $\pi$ is trivial over every ball of any radius, centered at the origin.
  
  \Lemma{3} Let $\pi : \T \to \RR^n$ be a principal fibration with structure group $\G$.
  There exists a family of smooth sections $s_k : \cB_k \to \T$ indexed by $\NN$,
  where the $\cB_k$ are open balls,
  such that  $s_k \prec s_{k+1}$,
  that is,
  $s_{k+1} \restriction \cB_k = s_k$,
  for all $k$.
  
  {\em Proof of Lemma 3.} Thanks to the previous lemma,
  the fibration $\pi : \T \to \RR^n$ is trivial over every open ball $\cB_k$.
  Let us choose once and for all a family of smooth sections $s'_k : \cB_k \to \T$,
  for every positive integer $k$,
  and let us introduce two series of maps,
  $\varepsilon_k : \closedinterval{0,k+1} \to \closedinterval{0,k}$ and $\lambda_k : \cB_{k+1} \to \cB_k$,
  for $k$ integer and $k \geq 2$.
  The function $\varepsilon_k$,
  described in Figure \ref{fig-Smashing-balls-function},
  is the restriction of an increasing smooth map satisfying the following conditions,
  $$%
  \varepsilon_k \restriction {\closedinterval{0,k-1}} = \id_{\closedinterval{0,k-1}}, \text{ and } \varepsilon_k(k+1) = k.
  $$%
  The maps $\lambda_k$ are then defined by
  $$%
  \lambda_k(x) = \varepsilon_k(\norm{x}) \times {x \over \norm{x}}, \ \text{for all} \ x \in \cB_{k+1}.
  $$%
  Since $\varepsilon_k \restriction {\closedinterval{0,k-1}} = \id_{\closedinterval{0,k-1}}$,
  $\lambda_k \restriction \cB_{k-1} = \id_{\cB_{k-1}}$,
  and the map $\lambda_k$ is smooth.
  Now,
  $s'_2$ is given,
  let us define $s_1$ and $s''_2$ by
  $$%
  s_1 = s''_2 \restriction \cB_1, \text{ with } s''_2 = s'_2.
  $$%
  Let us then consider the series of sections $s_1, s''_2, s'_3, s'_4, \ldots$,
  and let us focus on $s''_2$ and $s'_3$.
  There exists a smooth map
  $$%
  \gamma_{2,3} : \cB_2 \to \G, \text{ such that } s''_2(x) = \gamma_{2,3}(x)_\T(s'_3(x)), \ \text{for all} \ x \in \cB_2.
  $$%
  Let
  $$%
  s_2 = s''_3 \restriction \cB_2, \text{ with } s''_3 = \gamma_{2,3} \circ \lambda_2(x)_\T(s'_3(x)).
  $$%
  The function $\lambda_2$ maps $\cB_3$ onto $\cB_2$ and is the identity on $\cB_1$.
  So,
  since $\gamma_{2,3}$ is defined on $\cB_2$,
  $\gamma_{2,3} \circ \lambda_2 : \cB_3 \to \cB_2 \to \G$ and coincides with $\gamma_{2,3}$ on $\cB_1$.
  Thus,
  for all $x \in \cB_1$,
  $s_2(x) = s''_3(x) = \gamma_{2,3}(x)_\T(s'_3(x)) = s''_2(x) = s_1(x)$.
  Therefore,
  $s_1 = s_2 \restriction \cB_1$,
  and we have the series of sections,
  $s_1, s_2, s_3'', s'_4,s'_5, \ldots$,
  for which $s_1 \prec s_2 \prec s''_3$.
  %
  The recurrence is now clear,
  for $k \geq 3$ let $s_1 \prec \cdots \prec s_{k-1} \prec s''_k$ being a series of sections over the balls $\cB_1 \subset \cdots \subset \cB_{k-1} \subset \cB_k$.
  So,
  there exist two sections $s_k$ and $s''_{k+1}$ over the balls $\cB_k$ and $\cB_{k+1}$ such that $s_{k-1} \prec s_k \prec s''_{k+1}$.
  Indeed,
  let
  $$%
  \gamma_{k,k+1} : \cB_k \to \G, \text{ such that } s''_k(x) = \gamma_{k,k+1}(x)_\T(s'_{k+1}(x)), \ \text{for all} \ x \in \cB_k.
  $$%
  The map $\gamma_{k,k+1}$ is smooth,
  and we define
  $$%
  s_k = s''_{k+1} \restriction \cB_k, \text{ with } s''_{k+1} = \gamma_{k,k+1} \circ \lambda_k(x)_\T(s'_{k+1}(x)).
  $$%
  The function $\lambda_k$ maps $\cB_{k+1}$ onto $\cB_k$ and is the identity on $\cB_{k-1}$.
  Since $\gamma_{k,k+1}$ is defined on $\cB_k$,
  $\gamma_{k,k+1} \circ \lambda_k : \cB_{k+1} \to \cB_k \to \G$ and coincides with $\gamma_{k,k+1}$ on $\cB_{k-1}$.
  Thus,
  for all $x \in \cB_{k-1}$, $s''_{k+1}(x) = \gamma_{k,k+1}(x)_\T(s'_{k+1}(x)) = s''_k(x) = s_{k-1}(x)$,
  that is to say,
  $s_{k-1} \prec s_k$,
  and by construction, $s_k \prec s''_{k+1}$.
  Lemma 3 is thus proved.
  
  %%###########
  \begin{figure}[tb]
    \centerline{\includegraphics{Figures-PDF/fig-Smashing-balls-function}}
    \caption{The smashing balls function.}
    \label{fig-Smashing-balls-function}
  \end{figure}
  %%###########
  
  Now,
  the rest of the proposition is clear,
  once we have an exhaustion of $\RR^n$ by the open balls of radius the integers,
  and a series of compatible sections of the $\G$-principal fibration $\pi : \T \to \RR^n$ over these balls,
  we have a global section,
  global by compatibility and smooth by the locality axiom of diffeology.
  This completes the proof of the theorem.
  
  Note that the triviality of fiber bundles in diffeology is reduced to the triviality of principal fiber bundles \art{Associated-fiber-bundles}.
  Thus every diffeological fiber bundle over $\RR^n$ is trivial.
  
  Finally,
  considering a smooth map $f : \RR^n \to \X$,
  a smooth lifting $\F$ of $f$ in $\T$ is equivalent to a smooth section of $\pr_1 : f^*(\T) \to \RR^n$.
  We just built one for every principal bundle over $\RR^n$.
\end{proof}

\begin{article}\artlabel{Associated loops fiber bundles}
  \addcontentsline{toc}{section}{\small\hspace{10pt} Associated loops fiber bundles}
  \label{Associated-loops-fiber-bundles}
  Let $\pi : \X \to \B$ be a diffeological fiber bundle.
  Let $x \in \X$, $b = \pi(x)$, $\F = \pi^{-1}(b)$ and
  $$%
  \pi_* : \Loops(\X,x) \to \Loops(\B,b), \text{ with } \pi_*(\ell) = \pi \circ \ell.
  $$%
  Let $\Loops^\bullet(\B,b) = \comp(\bmb) \subset \Loops(\B,\bmb)$ be the connected component of the constant loop $\bmb = [t \mapsto b]$,
  and let us introduce
  $$%
  \Loops^*(\X,x) = (\pi_*)^{-1}(\Loops^\bullet(\B,b)), \text{ and } \pi_*^\bullet = \pi_* \restriction \Loops^*(\X,x).
  $$%
  The map $\pi_*$ satisfies the following properties.
  \begin{itemize}
    \item[($\diamondsuit$)] The map $\pi_*$ is onto,
    that is,
    every loop in $\B$,
    based at $b$ and null-homotopic,
    can be lifted along $\pi_*$ by a loop in $\X$,
    based at $x$.
    \item[($\heartsuit$)] The projection $\pi_*^\bullet : \Loops^*(\X,x) \to \Loops^\bullet(\B,b)$ is a diffeological fibration with fiber $\Loops(\F,x)$.
  \end{itemize}
  We call this diffeological fiber bundle, the {\em Loops bundle}\index{Loops bundle} associated with $\pi$ at the point $x$.
  This situation is summarized by the short fiber bundle sequence,
  $$%
  \bmx \longrightarrow \Loops(\F,x) \rfl{j_*} \Loops^*(\X,x) \rfl{\pi_*^\bullet} \Loops^\bullet(\B,b) \longrightarrow \bmb.
  $$%
\end{article} %% Associated-loops-fiber-bundles

\begin{proof}
  Let us prove ($\diamondsuit$).
  Let $\varphi$ be a homotopy from $\ell \in \Loops^\bullet(\B,b)$ to the constant loop $\bmb$.
  Thus,
  $\varphi \in \Paths(\Loops(\B,b))$ with $\varphi(0) = \ell$ and $\varphi(1) = \bmb$.
  We identify $\varphi$ with $\bar\varphi \in \Cinfty(\RR^2,\B)$ such that $\bar\varphi(t,s) = \varphi(t)(s)$ \art{Iterating-paths},
  we have
  $$%
  \bar\varphi \restriction_{s=0} = \bmb, \
  \bar\varphi \restriction_{s=1} = \bmb, \
  \bar\varphi \restriction_{t=0} = \ell, \
  \bar\varphi \restriction_{t=1} = \bmb.
  $$%
  The pullback $\pi_\varphi$ of $\pi$ by $\varphi$ is a diffeological fiber bundle over $\RR^2$,
  thus it is trivial \art{Triviality-over-global-plots}.
  \begin{equation}
    \renewcommand{\theequation}{$\spadesuit$}
    \begin{tikzcd}[column sep=large, row sep=large, every label/.append style = {font = \small}]
      \RR^2 \times \F \arrow[rr,"\Psi"] \arrow[dr,swap,"\pr_1" {pos=0.5}] & {} & \P^*(\X) \arrow[r,"\pr_2"] \arrow[dl, "\pr_1"] & \X \arrow[d,"\pi"]  \\
      {} & \RR^2 \arrow[rr, swap, "\bar\varphi"] & {} & \B
    \end{tikzcd}
  \end{equation}
  %
  Let $\Psi$ be a trivialization and let $\psi$ be defined,
  for all $(t,s;\xi) \in \RR^2 \times \F$,
  by
  $$%
  \Psi(t,s;\xi) = (t,s;\psi(t)(s)(\xi)), \text{ thus } \psi(t)(s) \in \Diff(\F, \X_{\varphi(t)(s)}).
  $$%
  By definition of the involved diffeologies \art{Functional-diffeologies} \art{Principal-bundle-attached-to-a-fibrating-groupoid},
  $\psi$ is a smooth lift of the plot $(t,s) \mapsto (b,\varphi(t)(s))$ of $\B \times \B$,
  in the total space
  $$%
  \T = \{f \in \Diff(\F,\X_b) \mid b \in \B \}
  $$%
  of the principal fiber bundle associated with the smooth structure groupoid of $\pi$,
  attached at the point $b$.
  Now,
  since
  $\varphi(t)(s) = b$ when $t=1$ or $s=0$ or $s=1$,
  $\psi(t)(s) \in \Diff(\F)$ for $t=1$ or $s=0$ or $s=1$.
  Thus,
  $\psi(0)(0)$ and $\psi(1)(0)$ are two diffeomorphisms of $\F$,
  and since they can be connected by the family of smooth paths $\bar\psi \restriction_{s=0}$,
  $\bar\psi \restriction_{s=1}$,
  and $\bar\psi \restriction_{t=1}$ (with the same convention for the over bar as previously for $\varphi$),
  they belong to the same component.
  Then,
  there exists a smooth path $\sigma$ in $\Diff(\F)$ such that $\sigma(0) = \psi(0)(0)$ and $\sigma(1) = \psi(1)(0)$ \art{Connected-components}.
  Let us consider the restriction $\Psi_0$ of $\Psi$ to $\{0\} \times \RR \times \F$,
  then for all $(s,\xi) \in \RR \times \F$,
  $$%
  \Psi_0(s,\xi) = (s,\psi_0(s)(\xi)), \text{ with } \psi_0(s) = \psi(0)(s), \text{ thus } \psi_0(s) \in \Diff(\F,\X_{\ell(s)}).
  $$%
  Actually,
  the path $\psi_0$ is a smooth lift in $\T$ of the loop $\ell$,
  $\Psi_0$ is a trivialization of the pullback of $\pi : \X \to \B$ by $\ell$.
  Let us introduce then $\Psi'$,
  for all $(s,\xi) \in \RR \times \F$,
  $$%
  \Psi'(s,\xi) = (s, \psi'(s)(\xi)), \text{ with } \psi'(s) = \psi_0(s) \circ \sigma(s)^{-1}.
  $$%
  The path $\psi'$ is again a smooth path in $\T$,
  and $\Psi'$ is a trivialization of $\pr_1 : \ell^*(\X) \to \RR$,
  but $\psi'$ satisfies now $\psi'(0) = \psi'(1) = \id_\F$.
  Finally,
  mapping the $x$-constant section $s \mapsto (s,x)$ to $\X$,
  by the trivialization $\Psi'$,
  we get a path in $\X$,
  that is,
  $\ell' : s \mapsto \psi'(x)$,
  which is a loop because $\psi'(0) = \psi'(1) = \id_\F$,
  and which lifts $\ell$ because $\psi' : \F \to \X_{\ell(s)}$.
  
  Now,
  let us prove ($\heartsuit$).
  Let $\Phi$ be a global $n$-plot in $\Loops^\bullet(\B,b)$,
  and let $\bar\Phi(r,t) = \Phi(r)(t)$,
  for all $(r,t) \in \RR^n \times \RR$,
  then $\bar\Phi \in \Cinfty(\RR^n \times \RR, \B)$ and $\bar\Phi$ satisfies $\bar\Phi(r,0) = \bar\Phi(r,1) = b$ for all $r\in \RR^n$.
  Since the pullback of $\pi : \X \to \B$ by $\bar\Phi$ is trivial \art{Triviality-over-global-plots},
  let $\psi$ be a global smooth lift of $\bar\Phi$ in the principal bundle $\T$ associated with $\pi$ (see above),
  and let $\Psi$ be the associated trivialization,
  $$%
  \Psi(r,t;\xi) = (r,t; \psi(r,t)(\xi)), \text{ with } \psi(r,t) \in \Diff(\F, \X_{\Phi(r)(t)}),
  $$%
  for all $(r,t;\xi) \in \RR^n \times \RR \times \F$.
  Let $\psi_0 = \psi \restriction \RR^n \times \{0\}$ and $\psi_1 = \psi \restriction \RR^n \times \{1\}$,
  $\psi_0$ and $\psi_1$ belong to $\Cinfty(\RR^n,\Diff(\F))$ because $\Phi(r)(0) = \Phi(r)(1) = \F$.
  Since $\RR^n$ is connected,
  $\Val(\psi_0)$,
  and $\Val(\psi_1)$,
  are contained in some connected components of $\Diff(\F)$.
  But $\psi_0(r)$ and $\psi_1(r)$ are contained in a same connected component of $\Diff(\F)$,
  for all $r \in \RR^n$,
  because $\Phi(r) \in \Loops^\bullet(\B,b)$,
  \ie,
  is null-homotopic,
  see above the proof of ($\diamondsuit$).
  Thus,
  $\Val(\psi_0)$ and $\Val(\psi_1)$ are contained in a same connected component of $\Diff(\F)$.
  But,
  because $\RR^n$ is contractible and because $\psi_0$ and $\psi_1$ take their values in the same connected component,
  there exists a homotopy in $\Cinfty(\RR^n, \Diff(\F))$ connecting $\psi_0$ with $\psi_1$ \xart{Contractible-diffeological-spaces}{Note 3},
  let us denote it by $\sigma$,
  that is,
  $\sigma \in \Paths(\Cinfty(\RR^n, \Diff(\F)))$ with $\sigma(0) = \psi_0$ and $\sigma(1) = \psi_1$.
  We define now $\psi'(r)(t) = \psi(r)(t) \circ \sigma(t)(r)^{-1}$ which satisfies $\psi' \in \Cinfty(\RR^n \times \RR,\T)$ with $\psi'(r)(0) = \psi'(r)(1) = \id_\F$ for all $r \in \RR^n$.
  The map $\Xi$ from $\RR^n \times \Loops(\F,x)$ to $\RR^n \times \Paths(\X)$,
  defined by
  $$%
  \Xi(r,\ell) = (r,[t \mapsto \psi'(r)(t)(\ell(t))]),
  $$%
  takes its values in $\Phi^*(\Loops^*(\X,x))$ and is a trivialization of the pullback by $\Phi$ of the smooth surjection $\pi_*^\bullet$.
  Therefore,
  according to \art{Fibrations-and-local-triviality-along-the-plots},
  $\pi_*^\bullet$ is a diffeological fibration with fiber $\Loops(\F,x)$.
\end{proof}

\begin{article}\artlabel{Exact homotopy sequence of a diffeological fibration}
  \addcontentsline{toc}{section}{\small\hspace{10pt} Exact homotopy sequence of a diffeological fibration}
  \label{Exact-homotopy-sequence-of-a-diffeological-fibration}
  Let $p : \X \to \B$ be a diffeological fiber bundle.
  Let $x \in \X$, $b = p(x)$ and $\F = p^{-1}(b)$.
  We consider the relative homotopy of the pair $(\X,\F)$,
  at the point $x$,
  as it is described in \art{The-short-homotopy-sequence-of-a-pair} and \art{The-long-homotopy-sequence-of-a-pair} and we use the same notations.
  %
  Let $i_{k\#}$ and $p_{k\#}$ be the maps in homotopy,
  induced by the inclusion $i : \F \to \X$ \art{Higher-homotopy-groups} and by the projection $p$,
  $$%
  i_{k\#} : \pi_k(\F,x) \to \pi_k(\X,x), \text{ and } p_{k\#} : \pi_k(\X,x) \to \pi_k(\B,b).
  $$%
  Now,
  since the projection $p$ maps $\F$ to the point $b$,
  $p_k : \Loops_k(\X,x) \to \Loops_k(\B,b)$ \art{Higher-homotopy-groups} maps $\Loops_k(\F,x)$ to $\bmb_k$,
  then every element of
  $
  \Paths_k(\X,\F,x)
  $
  is mapped into an element of $\Loops_k(\B,b)$.
  We still denote by $p_{k\#}$ its induced map in homotopy for all $k \geq 1$,
  $$%
  p_{k\#} : \pi_k(\X,\F,x) \to  \pi_k(\B,b).
  $$%
  Then, $p_{k\#}$ satisfies the following property.
  \begin{itemize}
    \item[($\clubsuit$)]
    The map $p_{k\#}$ is an isomorphism of group for $k>1$ and an isomorphism of pointed spaces for $k=1$.
  \end{itemize}
  Let us define then,
  for all integers $k$,
  the {\em connector} $\Delta_k$ by
  $$%
  \Delta_{k+1} : \pi_{k+1}(\B,b) \to \pi_k(\F,x), \text{ with } \Delta_k = \0_\# \circ p_{(k+1)\#}^{-1},
  $$%
  where $\0_\# : \pi_{k+1}(\X,\F,x) \to \pi_k(\F,x)$ is the connector defined in \art{The-short-homotopy-sequence-of-a-pair}.
  By connecting then each degree $k+1$ of the exact sequence of the pair $(\X,\F)$,
  at the point $x$, to the degree $k$,
  with the connector $\Delta_{k+1}$,
  we get the {\em exact homotopy sequence of the fiber bundle}\index{Exact homotopy sequence} $p : \X \to \B$.
  Omitting the indices of the maps since they are clearly defined by their source and target,
  the sequence writes
  \begin{equation}
    \renewcommand{\theequation}{$\spadesuit$}
    \left.
    \begin{array}{c@{\hspace{-1pt}}c@{\hspace{-1pt}}c@{\hspace{-1pt}}c@{\hspace{-1pt}}c@{\hspace{-1pt}}c@{\hspace{-1pt}}c@{\hspace{-1pt}}c@{\hspace{-1pt}}c@{\hspace{-1pt}}c@{\hspace{-1pt}}c}
      \cdots \
      & \rfl{p_\#}
      & \pi_{k+1}(\B,b)
      & \rfl{\Delta}
      & \pi_{k}(\F,x)
      & \rfl{i_\#}
      & \pi_{k}(\X,x)
      & \rfl{p_\#}
      & \pi_k(\B,b)
      & \rfl{\Delta}
      & \cdots
      \\
      \vspace{-5pt}
      \\
      \cdots\
      & \rfl{p_\#}
      &  \pi_1(\B,b)
      & \rfl{\Delta}
      & \pi_0(\F,x)
      & \rfl{i_\#}
      & \pi_0(\X,x)
      & \rfl{p_\#}
      & \pi_0(\B,b).
      &
      &
    \end{array}
    \right\}
  \end{equation}
  
  \Note~The following propositions are immediate consequences of the exact homotopy sequence.
  
  (a) If the homotopy of the fiber $\F$ vanishes,
  in particular if $\F$ is contractible,
  then $\pi_k(\X,x) = \pi_k(\B,b)$ for all $k$.
  
  (b) If the fiber $\F$ is a deformation retract of $\X$,
  then $\pi_k(\B,b) = \{0\}$ for all $k$.
  
  (c) If $p : \X \to \B$ is a principal fibration with group $\G$,
  then we can replace the fiber $\F$ pointed at $x$ by the group $\G$ pointed at $\id_\G$ and the inclusion $i : \F \to \X$ by the orbit map $\orb{x} : \G \to \X$.
  The exact sequence writes then
  $$%
  \cdots
  \xrfl{\Delta_{k+1}}{15mm} \pi_k(\G,\id_\G)
  \xrfl{\orb{x}_{k\#}}{15mm} \pi_k(\X,x)
  \xrfl{p_{k\#}}{15mm} \pi_k(\B,b)
  \xrfl{\Delta_k}{15mm} \cdots\,.
  $$%
  This replacement is not really necessary but makes the sequence looks prettier.
  This can be used to compute the homotopy of some quotients which are not manifolds and cannot be computed otherwise,
  such as the irrational torus.
\end{article} %% Exact homotopy sequence of a diffeological fibration

\begin{proof}
  Let us prove first ($\clubsuit$) for $k=1$,
  that is, for $p_1 : \Paths(\X,\F,x) \to \Loops(\B,b)$.
  Let $\ell \in \Loops(\B,b)$,
  since the pullback $\pr_1 : \ell^*(\X) \to \RR$ is trivial,
  we can lift $\ell$ by a path $\bar\ell$ by mapping any section of the pullback to $\X$.
  We can even choose $\bar\ell$ such that $\bar\ell(1) = x$ and,
  since $\ell(0) = \ell(1) = b$, $\bar\ell(0) \in p^{-1}(b) = \F$.
  This shows that $p_1$ is surjective,
  and therefore $p_{1\#}$.
  Now let us prove the injectivity.
  Since $p_{1\#}$ is a group morphism,
  we shall prove that its kernel is reduced to $\class(\bmx) \in \pi_1(\X,\F,x)$,
  that is,
  every path $\gamma' \in \Paths(\X,\F,x)$ which projects to a null-homotopic loop $\ell \in \Loops(\B,b)$ is relatively homotopic to $\bmx$ and,
  which is sufficient,
  relatively homotopic to a path in $\F$.
  So,
  let $\Phi$ be a homotopy connecting $\gamma'$ to $\bmb$,
  $\Phi(0) = \gamma'$ and $\Phi(1)= \bmb$.
  As usual we identify $\Phi$ with a map,
  denoted here by the same letter,
  $\Phi \in \Cinfty(\RR^2,\B)$,
  that is,
  $\Phi(s,t) = \Phi(s)(t)$.
  Let $\Psi$ be a trivialization of the pullback $\pr_1 : \Phi^*(\X) \to \RR^2$ \art{Triviality-over-global-plots},
  and let $\Psi(s,t;\xi) = (s,t;\psi(s)(t)(\xi))$,
  where $(s,t) \mapsto \psi(s)(t)$ is a plot of the principal smooth structure bundle $\pi : \T \to \B$ associated with $p : \X \to \B$,
  attached at the point $b$ \art{Space-of-structures-of-a-diffeological-fiber-bundle}.
  In particular $\psi(s)(t) \in \Diff(\F,\X_{\Phi(s)(t)})$.
  We can even choose $\psi(s)(1) = \id_\F$ by just replacing $\psi(s)(t)$ by $\psi(s)(t) \circ \psi(s)(1)^{-1}$.
  Now,
  let
  $$%
  \gamma(t) = \psi(0)(t)^{-1}(\gamma'(t)).
  $$%
  Since $\psi(0)(t) \in \Diff(\F, \X_{\Phi(0)(t)})$ and $\Phi(0)(t) = \gamma'(t)$, $\psi(0)(t)^{-1}(\gamma'(t)) \in \F$ and therefore,
  $\gamma \in \Paths(\F)$.
  Next, let us define
  $$%
  \Phi' = [ s \mapsto [t \mapsto \psi(s)(t)(\gamma(t))]],
  $$%
  which satisfies
  $$%
  \Phi'(0) = \gamma', \ \Phi'(1) \in \Paths(\F), \text{ and } \Phi'(s)(1) = x \text{ for all } s \in \RR.
  $$%
  So,
  $\Phi'$ is a fixed-ends homotopy connecting $\gamma'$ to a path $\gamma'' \in \Paths(\F,\star,x)$.
  But,
  every path in $\Paths(\F,\star, x)$ is relatively homotopic to the constant loop $\bmx$,
  consider for example the homotopy $s \mapsto \gamma''_s = [t \mapsto(\gamma''((1-t)s+t))]$.
  Thus,
  $p_{1\#}$ is injective and therefore bijective.
  %
  Now, by recurrence over $k$,
  on the one hand we have
  \renewcommand{\arraystretch}{1.5}
  $$%
  \begin{array}{c}
    \pi_{k+1}(\X,\F,x) = \pi_1(\Loops_{k}(\X,x),\Loops_{k}(\F,x),\bmx_{k}), \\
    \pi_{k+1}(\X,x) = \pi_1(\Loops_{k}(\X,x),\bmb_{k}) \ \text{and} \ \pi_{k+1}(\B,b) = \pi_1(\Loops_{k}(\B,b),\bmb_{k}).
  \end{array}
  $$%
  \renewcommand{\arraystretch}{1}
  On the other hand,
  by definition,
  $$%
  \left\{
  \begin{array}{l}
    \Loops_{k+1}(\X,x) = \Loops(\Loops_k(\X,x),\bmx_k), \\[.8ex]
    \Loops_{k+1}(\B,b) =\Loops(\Loops_k(\B,b),\bmb_k),
  \end{array}
  \right.
  $$%
  and the projection $p_k : \Loops_{k+1}(\X,x) \to \Loops_{k+1}(\B,b)$ is the associated loops fiber bundle \art{Associated-loops-fiber-bundles} of the fiber bundle $p_k : \Loops_k(\X,x) \to \Loops_k(\B,b)$,
  with fiber $\Loops(\Loops_{k}(\F,x),\bmx_{k}) = \Loops_{k+1}(\F,x)$.
  By denoting $\X_k=\Loops_k(\X,x)$, $\B_k=\Loops_k(\B,b)$,
  $\F_k=\Loops_k(\F,x)$,
  we are in the same situation as previously,
  replacing $\X$ by $\X_k$,
  $\B$ by $\B_k$,
  $\F$ by $\F_k$,
  $x$ by $x_k$ and $b$ by $b_k$,
  that is,
  $$%
  p_{(k+1)\#} = (p_k)_{1\#} : \pi_1(\X_k, \F_k, \bmx_x) \to \pi_1(\B_k,\bmb_k).
  $$%
  Then, because $p_{1\#}$ is an isomorphism,
  $p_{k\#}$ is an isomorphism for all $k$.
  
  Now, to complete the proof about the exactness of the homotopy sequence of diffeological fiber bundles,
  we need to check only the case not covered by the exact sequence of a pair,
  that is,
  $$%
  \pi_0(\F,x) \xrfl{i_\#}{15mm} \pi_0(\X,x) \xrfl{p_\#}{15mm} \pi_0(\B,b) \xrfl{}{15mm} 0.
  $$%
  But $\Val(p_\#) = \pi_0(\B,b)$ is a simple consequence of the surjectivity of $p$.
  Then,
  $$%
  \Val(i_\#) = \ker(p_\#) \ \text{means} \ p^{-1}(\comp_\B(b)) = \bigcup_{y \in \F} \comp_\X(y).
  $$%
  This means firstly that connected points are mapped to connected points by $p$,
  which is clear,
  and secondly that,
  for every point $b'$ connected to $b$,
  if a point $x'$ in $\X$ is mapped to $b'$ by $p$,
  then it is connected to $\F$,
  which is a direct consequence of the global triviality of the pullback of $p : \X \to \B$ by any path in $\B$.
\end{proof}

%%%%%%%%%%%%%%%%%%%%%%%%%%%%%%%%%%%%%%%%%%%%%%%%%%%%%%%%%%
%% MARK: Coverings of Diffeological Spaces
%%%%%%%%%%%%%%%%%%%%%%%%%%%%%%%%%%%%%%%%%%%%%%%%%%%%%%%%%%

\section*{Coverings of Diffeological Spaces}
\label{Coverings-of-diffeological-spaces}

\begin{sechead}
  The definition of diffeological discrete spaces \art{Discrete-diffeology} leads naturally to the definition of diffeological coverings.
  In this section we shall see,
  first of all,
  that every diffeological space has a simply connected universal covering,
  unique up to equivalence,
  and then that the various coverings are classified,
  up to equivalence,
  by the conjugacy classes of the subgroups of the fundamental group.
  The universal covering is built actually thanks to the Poincar{\'e} groupoid,
  introduced previously \art{The-Poincare-groupoid-and-fundamental-group}.
  Moreover,
  we shall establish the important monodromy theorem concerning the lifting of maps from simply connected spaces to coverings.
  It has already been applied for lifting actions of groups on coverings \art{Covering-smooth-actions}.
  Also note that,
  applied to the category of manifolds,
  these constructions and theorems give again the classical results,
  and that the covering of a manifold as a diffeological space is a covering as a manifold, which is satisfactory.
\end{sechead}

\begin{article}\artlabel{What is a covering?}
  \addcontentsline{toc}{section}{\small\hspace{10pt} What is a covering}
  \label{What-is-a-covering}
  Let $\X$ be a connected diffeological space,
  a {\em covering}\index{Covering} of $\X$ is any fiber bundle $p : \Y \to \X$ with a discrete fiber $\F$.
  The space $\Y$ is called the {\em covering space}.
  \begin{itemize}
    \item[1.] If $\Y$ is connected,
    we say that $p$ is a {\em connected covering}.
    \item[2.] If the fiber $\F$ is made of $\N$ points,
    we say that $p$ is an {\em $N$-folds covering}.
    \item[3.] If $p$ is a principal bundle,
    we say that $p$ is a {\em  Galoisian covering}\index{Galoisian covering}.
  \end{itemize}
  Coverings over a diffeological space $\X$ form naturally a full subcategory of the category of fiber bundles \art{Diffeological-fiber-bundles-category} with base $\X$.
  
  \Note~As a first example of covering we have all the manifolds coverings,
  but also the quotients of any diffeological group $\G$ by any discrete (not necessarily closed) subgroup $\Gamma$,
  since a quotient of a group by a subgroup is always a diffeological fiber bundle \art{Fibrations-of-groups-by-subgroups}.
  If $\Gamma$ is normal,
  these coverings are Galoisian.
  For example,
  $\RR \to \RR/\QQ$,
  $\RR \to \RR/(\ZZ + \alpha \ZZ)$,
  etc.
  Every strict subgroup $\Gamma \subset \RR$ defines a Galoisian covering from $\RR$ to $\RR/\Gamma$.
\end{article} %% What-is-a-covering

\begin{article}\artlabel{Lifting global plots on coverings}
  \addcontentsline{toc}{section}{\small\hspace{10pt} Lifting global plots on coverings}
  \label{Lifting-global-plots-on-coverings}
  Let $p : \Y \to \X$ be a covering with fiber $\F = p^{-1}(x)$.
  Let $\varphi : \RR^n \to \X$ be a plot centered at $x$,
  that is,
  $\varphi(0)=x$.
  There exists a trivialization $\Psi : (r,y) \mapsto (r, \psi(r)(y))$,
  $(r,y) \in \RR^n \times \F$,
  of the pullback $\pr_1 : \varphi^*(\Y) \to \RR$ such that $\Psi \restriction \{0\} \times \F = \id_\F$.
  Moreover,
  for all $y \in \F$ there exists a unique lift $\bar\varphi : \RR^n \to \Y$,
  $\bar\varphi : r \mapsto \psi(r)(y)$,
  such that $\bar\varphi(0) = y$.
  
  \Note~We get the same conclusion {\em mutatis mutandis},
  by choosing any basepoint $r_0 \in \RR^n$ instead of the origin $0 \in \RR^n$.
\end{article} %% Lifting-global-plots-on-coverings

\begin{proof}
  Thanks to \art{Triviality-over-global-plots},
  the pullback $\pr_1 : \varphi^*(\Y) \to \RR$ is trivial.
  Let $\Psi : \RR^n \times \F \to \varphi^*(\Y)$ be a trivialization.
  Then,
  for all $(r,\xi) \in \RR^n \times \F$,
  $$%
  \Psi(r,\xi) = (r, \psi(r)(\xi)), \text{ with } \psi \in \Cinfty(\RR^n, \T), \ \psi(r) \in \Diff(\F, \Y_{\varphi(r)}),
  $$%
  where $\pi : \T \to \X$ is the principal structure bundle associated with $p$,
  attached at the point $x$ \art{Associated-fiber-bundles}.
  Now since $\F = \Y_{\varphi(0)}$,
  replacing $\psi(r)$ by $\psi(r) \circ \psi(0)^{-1}$,
  we get $\psi(0) = \id_\F$ and an isomorphism $\Psi$ satisfying the condition $\Psi \restriction \{0\} \times \F = \id_\F$.
  Next,
  the map $\bar\varphi : r \mapsto \psi(r)(y)$ is a lift of $\varphi$ satisfying $\bar\varphi(0) = y$.
  Next,
  let $\sigma$ be another lift of $\varphi$ such that $\sigma(0) = y$,
  then $\Psi^{-1}(r,\sigma(r)) = (r, f(r))$ where $f \in \Cinfty(\RR^n, \F)$ and $f(0) = y$.
  But $\F$ is discrete,
  thus $f$ is constant,
  $f(r) = y$ for all $r$,
  and $\sigma = \bar\varphi$.
\end{proof}

\begin{article}\artlabel{Fundamental group acting on coverings}
  \addcontentsline{toc}{section}{\small\hspace{10pt} Fundamental group acting on coverings}
  \label{Fundamental-group-acting-on-coverings}
  Let $\X$ be a connected diffeological space,
  $\pi_0(\X) = \{\X\}$.
  Let $p : \Y \to \X$ be a covering with fiber $\F = p^{-1}(x)$.
  Let $\gamma \in \Paths(\X,\star, x)$,
  and let $y$ be some point in $\F$,
  there exists a unique path $\lift_y(\gamma) \in \Paths(\Y,\F,y)$ lifting $\gamma$,
  that is,
  $$%
  p \circ \lift_y(\gamma) = \gamma, \text{ and } \lift_y(\gamma)(1) = y.
  $$%
  The map $\ell \mapsto [y \mapsto \lift_y(\ell)(0)]$,
  where $\ell \in \Loops(\X,x)$ and $y \in \F$ is smooth and depends only on the homotopy class $\tau$ of $\ell$.
  This defines an action of $\pi_1(\X,x)$ on $\F$ denoted by
  \begin{equation}
    \renewcommand{\theequation}{$\clubsuit$}
    \vartau_\F(y) = \lift_y(\ell)(0), \text{ with } \tau = \class(\ell).
  \end{equation}
  Since $\F$ is discrete there is a natural identification between $\F$ and $\pi_0(\F)$ or between $\pi_0(\F,y)$ and $(\F,y)$.
  So,
  the exact homotopy sequence of the fiber bundle $p$ \art{Exact-homotopy-sequence-of-a-diffeological-fibration} splits into
  $$%
  \left.
  \begin{array}{c}
    0 \rfl{} \pi_k(\Y,y)
    \xrfl{p_{k\#}}{13mm} \pi_k(\X,x)
    \rfl{} 0 \quad \text{for } k \geq 2, \\
    \vspace{-2mm} \\
    0 \rfl{} \pi_1(\Y,y)
    \xrfl{p_{1\#}}{13mm} \pi_1(\X,x)
    \xrfl{\orb{y}}{13mm} (\F,y)
    \xrfl{i_\#}{13mm}
    \pi_0(\Y,y)
    \rfl{} \{\X\},
  \end{array}
  \right\}
  $$%
  where $\orb{y}$ denotes the orbit map for the $\pi_1(\X,x)$ action above ($\clubsuit$).
  \begin{itemize}
    \item[1.] The map $p_{k\#}$ is an isomorphism for $k \geq 2$ and a monomorphism for $k = 1$.
    
    \item[2.] The image of $\pi_1(\Y,y)$ by $p_{1\#}$ is the stabilizer of $y$ for the action of $\pi_1(\X,x)$.
    The space $\Y$ is connected if and only if $\pi_1(\X,x)$ is transitive on $\F$.
    
    \item[3.] If $\Y$ is connected and $p_{1\#}$ surjective,
    thus bijective,
    then $p$ is a diffeomorphism. If $\X$ is simply connected,
    then all of its coverings are trivial.
    
    \item[4.] The space $\Y$ is connected and simply connected if and only if the action of $\pi_1(\X,x)$ is free and transitive on $\F$.
  \end{itemize}
\end{article} %% Fundamental-group-acting-on-coverings

\begin{proof}
  The path $\lift_y(\gamma)$ is explicitly given by $\lift_y(\gamma) : t \mapsto \psi(t)(y)$,
  where $(t,y) \mapsto (t,\psi(t)(y))$, $(t,y) \in \RR \times \F$ is a trivialization of $\pr_1 : \ell^*(\Y) \to \RR$ such that $\psi(1) = \id_\F$,
  and this lift is unique;
  see \art{Lifting-global-plots-on-coverings}.
  Actually,
  $\psi$ is a smooth lift of $\ell$ in the principal structure bundle of $\pr_1 : \Y \to \X$.
  Then,
  since the fiber $\F$ is discrete,
  any plot in $\F$ is locally constant and the composite with $y \mapsto \vartau_\F(y) = \lift_y(\ell)(0) = \psi(0)(y)$ is locally constant,
  thus smooth.
  Therefore,
  $\vartau_\F$ is smooth.
  Let $\bar\ell$ be the reverse of $\ell$,
  we get a trivialization of $\pr_1 : \bar\ell^*(\Y) \to \RR$,
  by $(t,y) \mapsto (t,\bar\psi(t)(y))$ with $\bar\psi(t) = \psi(1-t)\circ\psi(0)^{-1}$,
  and $\bar\psi(1) = \id_\F$,
  and $\lift_y(\bar\ell)(t) = \bar\psi(t)(y)$.
  So,
  let $\bar\vartau_\F(y) = \lift_y(\bar\ell)(0) = \psi(1) \circ \psi(0)^{-1}(y)$,
  we get then $\bar\vartau_\F \circ \vartau_\F(y) = \psi(1) \circ \psi(0) \circ \psi(0)^{-1}(y) = y$,
  and $\vartau_\F \circ \bar\vartau_\F(y) = \psi(0) \circ \psi(1) \circ \psi(0)^{-1}(y) = y$.
  Thus,
  $\bar\vartau_\F = \tau_\F^{-1}$,
  and $\vartau_\F$ is a diffeomorphism of $\F$.
  Now let us consider two elements $\tau = \class(\ell)$ and $\tau' = \class(\ell')$ of $\pi_1(\X,x)$,
  let $\psi$ and $\psi'$ be the lifts of $\ell$ and $\ell'$ in the principal structure bundle of $\pi : \Y \to \X$.
  On one hand we have $\vartau_\F(\tau'_\F(y)) = \psi(0)(\psi'(0)(y))$.
  On the other hand,
  assuming as usual,
  that the loops $\ell$ and $\ell'$ are stationary,
  the lifts $\psi$ and $\psi'$ can also be chosen stationary,
  thus the concatenation $(\psi \circ \psi'(0)) \vee \psi'$ is well defined.
  This is a lift of $\ell \vee \ell'$,
  in the principal structure bundle of $\pi : \Y \to \X$,
  satisfying $((\psi \circ \psi'(0)) \vee \psi')(1)= \id_\F$.
  Next,
  $\tau \cdot \tau'= \class(\ell \vee \ell')$,
  thus $(\tau \cdot \tau')_\F(y) = [(\psi \circ \psi'(0)) \vee \psi'](0)(y) = \psi(0)(\psi'(0)(y))$.
  Hence,
  $\tau \mapsto \vartau_\F$ is a homomorphism from $\pi_1(\X,x)$ to $\Diff(\F)$,
  that is,
  an action of $\pi_1(\X,x)$ on $\F$.
  
  Now,
  let us check that the connecting homomorphism $\Delta_0 : \pi_1(\X,x) \to \pi_0(\F,y)$,
  of the exact homotopy sequence \art{Exact-homotopy-sequence-of-a-diffeological-fibration},
  coincides with the orbit map $\eR(y)$ of the action of $\pi_1(\X,x)$ on $\F$.
  Remember that $\Delta_0 = \0_\# \circ p_{1\#}^{-1}$,
  where $p_{1\#} : \pi_1(\Y,\F,y) \to \pi_1(\X,x)$.
  Then,
  the map $\Delta_0$ is exactly the map associating the homotopy class of every loop $\ell \in \Loops(\X,x)$ with the connected component of the origin of the lift $\bar\ell$ ending at $y$,
  that is,
  $\lift_y(\ell)(0)$.
  Thus,
  the connecting homomorphism from $\pi_1(\X,x)$ to $\pi_0(\F,y)$,
  identified to the pointed space $(\F,y)$, is just the orbit map $\eR(y)$ of the action of $\pi_1(\X,x)$ on the fiber $\F$.
  Then,
  
  1. Reading the homotopy exact sequence of $\pi : \Y \to \X$ gives immediately that $p_{k\#}$ is an isomorphism for $k \geq 2$,
  and a monomorphism for $k=1$.
  
  2. We read from the exact homotopy sequence that the kernel of $\eR(y)$,
  which is the stabilizer of $y$ in $\pi_1(\X,x)$,
  is the image of $\pi_1(\Y,y)$ by $p_{1\#}$.
  Then,
  $\Y$ is connected,
  that is,
  $\pi_0(\Y,y) = \{ \Y \}$, or $i_\#(\F,y) = \{ \Y \}$,
  if and only if $\pi_1(\X,x)(\F)= \eR(y)(\pi_1(\X,x)) = \F$,
  that is,
  if and only if $\pi_1(\Y,y)$ acts transitively on $\F$.
  
  3. If $p_{1\#}$ is surjective,
  then $\ker(\eR(y)) = \pi_1(\X,x)$ and then $\Val(\eR(y)) = \ker(i_\#) = \{y\}$,
  but if $\Y$ is connected,
  then $\ker(i_\#) = (\F,y)$,
  thus $(\F,y) = \{y\}$,
  that is,
  $\F = \{y\}$.
  The fiber $\F$ is reduced to a point,
  and $p$ is an injective subduction,
  hence a diffeomorphism \art{Injective-subductions}.
  Now,
  if $\X$ is simply connected,
  then $(\F,y) = \pi_0(\Y,y)$,
  that is,
  $\F$ intersects each component of $\Y$ in one and only one point.
  Thus,
  the restriction of $\pi$ to each component is an injective subduction,
  that is, a diffeomorphism.
  Hence,
  $p$ is equivalent to the direct product $\pr_1 : \X \times \F \to \X$.
  
  4. If $\pi_1(\X,x)$ acts freely on $\F$,
  then $\ker(\eR(y)) = \id_{\pi_1(\X,x)}$,
  and $\pi_1(\X,x) = \Val(\eR(y)) = \F$.
  Thus,
  $\pi_1(\Y,y) = 0$, $\pi_0(\Y,y) = \{\Y\}$,
  and conversely.
\end{proof}

\begin{article}\artlabel{Monodromy theorem}
  \addcontentsline{toc}{section}{\small\hspace{10pt} Monodromy theorem}
  \label{Monodromy-theorem}\index{Monodromy theorem}
  Let $\X$ and $\X'$ be two connected diffeological spaces,
  let $p : \Y \to \X$ be a covering, and let $f : \X' \to \X$ be a smooth map.
  \begin{itemize}
    \item[1.] Let $\varphi_1$ and $\varphi_2$ be two smooth lifts of $f$ in $\Y$.
    If they coincide somewhere they coincide everywhere.
    \item[2.] If $\X'$ is simply connected,
    then for all smooth maps $f : \X' \to \X$,
    for every $x' \in \X'$ and every $y \in \Y_x = p^{-1}(x)$ such that $x = f(x')$,
    there exists one and only one smooth lift $\varphi$ such that $\varphi(x') = y$.
  \end{itemize}
\end{article} %% Monodromy-theorem

\begin{proof}
  Let $x' \in \X'$ such that $\varphi_1(x') = \varphi_2(x')$,
  and let $x'' \in \X'$.
  Since $\X'$ is connected,
  there exists a path
  $\gamma$ in $\X'$ connecting $x'$ to $x''$.
  Thus,
  the paths $\varphi_1 \circ \gamma$ and $\varphi_2 \circ \gamma$ are two lifts of $f \circ \gamma$ coinciding in $0$,
  thanks to \art{Lifting-global-plots-on-coverings},
  they coincide everywhere.
  Therefore $\varphi_1(x'') = \varphi_1(\gamma(1)) =  \varphi_2(\gamma(1)) = \varphi_2(x'')$,
  that is,
  $\varphi_1 = \varphi_2$.
  %
  Now let us assume that $\X'$ is simply connected,
  the pullback $\pr_1 : f^*(\Y) \to \X'$ is a covering,
  it is trivial because $\pi_1(\X') = 0$ \xart{Fundamental-group-acting-on-coverings}{point 3}.
  Then,
  the composite of any section of $\pr_1$ with $\pr_2 : \pr_1^*(\Y) \to \Y$ gives a lift $\varphi$ of $f$ in $\Y$.
  Choosing a fold of the covering,
  we can map $x'$ wherever we want over $f(x')$.
  Thanks to the first item this choice adjusts the lift.
\end{proof}

\begin{article}\artlabel{The universal covering}
  \addcontentsline{toc}{section}{\small\hspace{10pt} The universal covering}
  \label{The-universal-covering}
  Let $\X$ be a connected diffeological space.
  Let $\XX$ be the Poincar\'e groupoid defined in \art{The-Poincare-groupoid-and-fundamental-group}.
  Remember that $\Obj(\XX) = \X$ and $\Mor(\XX) = \Pi(\X)$ is the space of fixed-ends homotopy classes of paths in $\X$.
  Let $x$ be any point of $\X$,
  and let us denote
  $$%
  \tilde\X = \Mor_\XX(\X,x,\star), \text{ and } \pi : \tilde\X \to \X, \text{ with } \pi = \1 \restriction \Mor_\XX(\X,x,\star).
  $$%
  Then,
  $\tilde \X$ satisfies the following properties:
  \begin{itemize}
    \item[1.] The space $\tilde\X$ is connected and simply connected.
    The projection $\pi : \tilde\X \to \X$ is a Galoisian covering,
    with structure group $\pi_1(\X,x)$.
    \item[2.] Every other connected and simply connected covering is isomorphic to the covering $\pi : \tilde\X \to \X$.
    \item[3.] Every connected covering $p : \Y \to \X$ is a quotient of $\pi : \tilde\X \to \X$ by the action of a subgroup $\H \subset \pi_1(\X,x)$.
  \end{itemize}
  Thanks to these universal properties,
  the covering $\pi : \tilde\X \to \X$ is called {\em universal covering}\index{Universal covering} of $\X$.
  
  \Note{1} To paraphrase the above proposition:
   Every connected diffeological space has a connected and simply connected covering,
  unique up to isomorphism.
  Any other covering is isomorphic to one of its quotients.
  
  \Note{2} The set of equivalence classes of coverings of $\X$ is in a one-to-one correspondence with the conjugacy classes of the subgroups of $\pi_1(\X,x)$.
  
  \Note{3} The universal covering of a product is the product of the
  universal coverings of the factors.
\end{article} %% The-universal-covering

\begin{proof}
  First of all,
  the structure group of the Poincar\'e groupoid at a point $x \in \X$ is,
  by definition,
  the fundamental group $\pi_1(\X,x)$.
  We have seen that the concatenation of paths is smooth,
  as well as the inversion.
  The map $x \mapsto \class(\bmx)$,
  where $\bmx$ is the constant loop $t \mapsto x$ is an induction,
  indeed a plot in the units of $\Pi(\X)$ lifts locally as a plot of $\Paths(\X)$,
  then taking the value for $t=0$,
  which is a smooth map,
  we get a plot of $\X$.
  So, $\Pi(\X)$ is a diffeological groupoid according to \art{Diffeological-groupoids}.
  Moreover, since the characteristic map $\ends : \Pi(\X) \to \X$ is a subduction \art{The-Poincare-groupoid-and-fundamental-group},
  the Poincar\'e groupoid is a fibrating groupoid \art{Fibrating-groupoids}.
  Then,
  the structure bundle attached at a point $x$ \art{Principal-bundle-attached-to-a-fibrating-groupoid} is a principal fibration with structure group $\pi_1(\X,x)$.
  %
  Next,
  since the fundamental group $\pi_1(\X,x)$ acts freely and transitively on the fiber $\F = \pi^{-1}(x)$,
  and thanks to \xart{Fundamental-group-acting-on-coverings}{point 4},
  we conclude that $\tilde\X$ is connected and simply connected.
  %
  Now let $p : \Y \to \X$ be another covering,
  and let $\F = p^{-1}(x)$.
  The pullback $\pr_1 : \pi^*(\Y) \to \X$ is trivial because $\tilde\X$ is simply connected \xart{Fundamental-group-acting-on-coverings}{point 3}.
  Let $\Psi : \tilde\X \times \F \to \pi^*(\Y)$ be a trivialization such that,
  for all $(\xi,y) \in \tilde\X \times \F$,
  $$%
  \Psi(\xi,y) = (\xi, \psi(\xi)(y)), \text{ with }
  \left\{
  \begin{array}{l}
    \psi(\xi) \in \Diff(\F,\Y_{\pi(\xi)}), \\
    \vspace{-10pt} \\
    \psi \restriction \pi^{-1}(x) = \id_\F.
  \end{array}
  \right.
  $$%
  The map $\psi$ is a lift of $\pi$ in the principal structure bundle associated with the fibration $p$.
  The commutative diagram $(\clubsuit)$ describes the situation.
  Let $\xi$ be the homotopy class of a path $\gamma$ in $\X$,
  with origin $x$.
  The lift $\tilde\gamma$ of $\gamma$ in $\tilde\X$ is given by $\tilde\gamma : s \mapsto \class(t \mapsto \gamma(st))$,
  and $\gamma_y : t \mapsto \psi(\tilde\gamma(t))(y)$ is the lift of the path $\gamma$ in $\Y$,
  with origin $y \in \F$.
  We deduce then the fundamental property of $\psi$,
  for all $\tau \in \pi_1(\X,x)$ and all $\xi \in \tilde \X$,
  for all $y \in \F$,
  $\psi(\tau \cdot \xi) (y) = \psi(\xi)(\tau^{-1}_\F(y))$ and then $\psi(\tau)(y) = \tau^{-1}_\F(y)$,
  where the action of $\pi_1(\X,x)$ on $\F$ has been described in \art{Fundamental-group-acting-on-coverings}.
  Next,
  let us choose a basepoint $y \in p^{-1}(x)$,
  the map $\xi \mapsto \psi(\xi)(y)$ is a smooth lift of $\pi : \tilde\X \to \X$.
  This lift is the composite of the section $\xi \mapsto (\xi, \psi(\xi)(y))$ of $\pi^*(\Y)$ with $\pr_2$.
  Note first that it is surjective in particular because the action of $\pi_1(\X,x)$ is transitive on the fiber $\F$,
  thanks to its connectedness \xart{Fundamental-group-acting-on-coverings}{point 2}.
  Moreover this lift is a subduction because $p$ and $\pi$ are two subductions.
  \begin{equation}
    \renewcommand{\theequation}{$\clubsuit$}
    \begin{tikzcd}[column sep=large, row sep=large, every label/.append style = {font = \small}]
      \tilde\X \times \F \arrow[rr,"\Psi"] \arrow[dr, swap, "\pr_1"] & {} & \pi^*(\Y) \arrow[r,"\pr_2"] \arrow[dl, "\pr_1"] & \Y \arrow[d,"p"]  \\
      {}  & \tilde\X \arrow[rr, swap, "\pi"] & {} & \X
    \end{tikzcd}
  \end{equation}

  Now,
  if two points $\xi$ and $\xi'$
  have the same image $\psi(\xi)(y) = \psi(\xi')(y)$,
  then $\pi(\xi) = \pi(x')$,
  and there exists $\tau \in \pi_1(\X,x)$ such that $\xi'= \tau \cdot \xi$,
  thus $\psi(\xi')(y) = \psi(\tau \cdot \xi)(y) = \psi(\xi)(\tau^{-1}_\F(y))$,
  and therefore $\vartau_\F(y) = y$.
  Conversely,
  if $\tau \in \pi_1(\X,x)$ stabilizes $y$,
  then $\psi(\tau \cdot \xi)(y) =  \psi(\xi)(y)$.
  Let $\H = \{ \tau \in \pi_1(\X,x) \mid \vartau_\F(y) = y\}$ be the stabilizer of $y$,
  the space $\Y$ identifies now naturally with the quotient $\tilde \X/\H$.
  This proves that every covering is the quotient of the universal covering $\pi : \tilde \X \to \X$.
  This also proves,
  in particular,
  that there exists only one connected and simply connected covering,
  up to isomorphism.
  
  For Note 2,
  it is purely algebraic and easy to prove that two conjugate
  subgroups $\H$ and $\H'$ will give two equivalent coverings,
  and conversely two equivalent connected coverings will give two conjugate stabilizers in $\pi_1(\X,x)$.
  
  For Note 3, the product of the universal coverings is still a covering and still simply connected,
  thus it is the universal covering of the product.
\end{proof}

\begin{article}\artlabel{Coverings and differential forms}
  \addcontentsline{toc}{section}{\small\hspace{10pt} Coverings and differential forms}
  \label{Coverings-and-differential-forms}
  Let $\X$ be a connected diffeological space,
  and let $\pi : \tilde\X \to \X$ be a universal covering.
  %
  \begin{itemize}
    \item[1.]  A differential $k$-form $\alpha$ on $\tilde\X$ is the pullback of a $k$-form $\beta$ of $\X$ if and only if $\alpha$ is invariant by $\pi_1(\X)$.
    In this case $\alpha$ is said to be {\em basic}\index{Basic differential form},
    and $\beta$ is the pushforward of $\alpha$,
    also denoted by $\pi_*(\alpha)$.
    \item[2.] Let $p : \Y \to \X$ be a covering,
    a differential $k$-form $\bar\alpha$ on $\Y$ is the pullback of a $k$-form $\beta$ of $\X$ if and only if $\bar\alpha$ is the pushforward on $\Y$ of a $\pi_1(\X)$-invariant $k$-form $\alpha$ of $\tilde\X$.
  \end{itemize}
  %
  \Note~If the covering $p: \Y \to \X$ is Galoisian with structure group $\H$,
  the differential form $\alpha$ is basic if and only if $\alpha$ is invariant by $\H$.
\end{article} %% Coverings-and-differential-forms

\begin{proof}
  Let $x \in \X$,
  and let us identify $\tilde\X$ with the principal bundle,
  associated with the Poincar\'e groupoid,
  attached at the point $x$ \art{The-universal-covering}.
  Since $\pi \circ \tau = \pi$ for all $\tau \in \pi_1(\X,x)$,
  it is clear that if $\alpha = \pi^*(\beta)$,
  then $\alpha$ is invariant under the action of $\pi_1(\X,x)$.
  Now,
  let us assume that $\alpha$ is invariant under $\pi_1(\X,x)$.
  Let us apply the criterion of \art{Pushing-forms-onto-quotients}.
  Let $\P : \U \to \tilde\X$ and $\P' : \U \to \tilde\X$ be two plots of $\tilde\X$ such that $\pi \circ \P = \pi \circ \P'$.
  Because $\pi$ is a principal fibration,
  for all $r_0 \in \U$,
  there exist an open ball $\cB$ centered at $r_0$ and a plot $r \mapsto \tau_r$ in the structure group $\pi_1(\X,x)$ such that,
  for all $r \in \cB$,
  $\P'(r) = \tau_r \cdot \P(r)$.
  But $\pi_1(\X,x) = \pi^{-1}(x)$ is discrete and $\cB$ is connected,
  then $\tau_r$ is constant and $\P'(r) = \tau \cdot \P(r)$ on $\cB$,
  thus $\alpha(\P') = \alpha(\tau \cdot \P) = \tau^*(\alpha(\P)) = \alpha(\P)$.
  Therefore,
  $\alpha(\P) = \alpha(\P')$ on $\cB$ and hence on $\U$.
  Thanks to the criterion cited above,
  there exists a $k$-form $\beta$ on $\X$ such that $\alpha = \pi^*(\beta)$.
  %
  The second assertion is a consequence of the first item,
  and because every connected covering is a quotient of the universal covering by a subgroup of $\pi_1(\X)$ \art{The-universal-covering}.
\end{proof}

%%%%%%%%%%%%%%%%%%%%%%%%%%%%%%%%%%%%%%%%%%%%%%%%%%%%%%%%%%
%
%   Exercises
%
%%%%%%%%%%%%%%%%%%%%%%%%%%%%%%%%%%%%%%%%%%%%%%%%%%%%%%%%%%

\Exercise

\begin{exercise}[Covering tori]
  \label{Covering-tori}
  Let $\Torus_{\Gamma} = \RR^n/\Gamma$ be a torus,
  that is, $\Gamma \subset \RR^n$ is any discrete subgroup generating $\RR^n$,
  dense or not.
  Give a canonical representative for every class of homotopic loops of the torus $\Torus_{\Gamma}$.
  Show,
  in particular,
  that,
  for the circle $\S^1$,
  every loop based at a point $x$ is homotopic to the loop $t \mapsto \cR(2\pi k t)(x)$,
  where $\cR(\theta)$ is the rotation of angle $\theta$,
  and $k \in \ZZ$.
  Use the diffeomorphism $\S^1 \simeq \RR/2\pi \ZZ$ built in \exref{Subduction-onto-the-circle}.
\end{exercise} %% Covering-tori

%%%%%%%%%%%%%%%%%%%%%%%%%%%%%%%%%%%%%%%%%%%%%%%%%%%%%%%%%%
%% MARK: Integration Bundles of Closed 1-Forms
%%%%%%%%%%%%%%%%%%%%%%%%%%%%%%%%%%%%%%%%%%%%%%%%%%%%%%%%%%

\enlargethispage{\baselineskip}

\section*{Integration Bundles of Closed 1-Forms}
\label{Section-Integration-bundles-of-closed-1-forms}

\begin{sechead}
  How are differential forms connected to fiber bundles?
  The first case comes with closed $1$-forms.
  To be a closed differential form is not far from being exact.
  Where can we read the lack of exactness of a closed form?
  How can we represent it?
  We have seen that a closed $1$-form $\alpha$,
  on a connected diffeological space $\X$,
  is exact if and only if its integral vanishes on every loop \art{Closed-1-forms-vanishing-on-loops}.
  We can do more,
  we can construct,
  explicitly,
  the minimal space $\X_\alpha$ over $\X$,
  on which the pullback of $\alpha$ is exact.
  This space is a covering over $\X$,
  and a Galoisian covering.
  The group of the covering is the {\em group of periods of $\alpha$},
  $\Periods_\alpha\subset (\RR,+)$,
  made with the integrals of $\alpha$ on the loops in $\X$.
  The idea is that,
  thanks to the proposition in \art{Closed-forms-on-centered-paths-spaces},
  the pullback of $\alpha$ on the space of pointed paths in $\X$, $\Paths(\X,x,\star)$,
  where $x \in \X$ is any base point, is exact \art{Closed-forms-on-centered-paths-spaces}.
  So,
  we shall find the smallest quotient of $\Paths(\X,x,\star)$,
  covering $\X$,
  on which the pullback of $\alpha$ can be pushed forward,
  and remains exact.
  Conversely,
  cohomology classes of closed $1$-forms can be regarded as {\em characteristic classes} of such coverings.
\end{sechead}

\begin{article}\artlabel{Periods of closed 1-forms}
  \addcontentsline{toc}{section}{\small\hspace{10pt} Periods of closed $1$-forms}
  \label{Periods-of-closed-1-forms}
  Let $\X$ be a diffeological space and $\alpha$ be a closed $1$-form,
  $\alpha \in \DForms^1(\X)$ and $d\alpha = 0$.
  Let $\CHK$ be the Chain-Homotopy operator \art{The-Chain-Homotopy-operator-K} and consider $\CHK\!\alpha$,
  the smooth real function,
  defined on $\Paths(\X)$ by
  $$%
  \CHK\!\alpha(\gamma) = \int_\gamma \alpha = \int_0^1 \alpha(\gamma)_t(1)\ dt, \text{ for all } \gamma \in \Paths(\X).
  $$%
  
  1. The restriction of $\CHK\!\alpha$ to the subspace $\Loops(\X)$ is closed,
  $$%
  d\left[\CHK\!\alpha \restriction \Loops(\X)\right] = 0.
  $$%
  Thus, $\CHK\!\alpha \restriction \Loops(\X)$ is
  constant on the connected components of
  $\Loops(\X)$.
  
  2. The group of periods $\Periods_\alpha$ of $\alpha$ \xart{The-De-Rham-homomorphism}{($\clubsuit$)} coincides with the group of periods of $\CHK\!\alpha \restriction \Loops(\X)$ and it is generated by the following set of periods
  \begin{equation}
    \renewcommand{\theequation}{$\clubsuit$}
    \PeriodsOf(\alpha) = \bigg\{ \CHK\!\alpha(\ell) = \int_\ell \alpha \biggm\vert \ell \in \Loops(\X) \bigg\}.
  \end{equation}
  
  3. If $\X$ is connected,
  $\pi_0(\X) = \{\X\}$ \art{Pathwise-connectedness},
  then $\Periods_\alpha = \PeriodsOf(\alpha)$,
  and the periods can be computed on $\Loops(\X,x)$,
  the subspace of loops based at some point $x$ of $\X$.
  The result does not depend on $x$ and the group of periods $\Periods_\alpha \subset \RR$ is a homomorphic image of the fundamental group,
  and obviously its {\em Abelianized}\index{Abelianized} $\Ab{\pi_1}(\X)$.
  
  4. If $\X$ is not connected,
  a $1$-form $\alpha$ on $\X$ is any family of $1$-forms $\alpha_i = \alpha \restriction \X_i$ defined on the component $\X_i \in \pi_0(\X)$.
  The form $\alpha$ is closed if and only if each $\alpha_i$ is closed.
  We can,
  if necessary,
  contiue to talk about the group of periods of $\alpha$.
  It will be generated by the union of the groups of periods on all components
  $$%
  \PeriodsOf(\alpha) =
  \bigcup_{\X_i \in \pi_0(\X)} \PeriodsOf(\alpha\restriction \X_i).
  $$%
  For each component $\X_i$ the periods $\Periods_{\alpha_i} = \PeriodsOf(\alpha \restriction \X_i)$ are a subgroup of $\RR$,
  a homomorphic image of the Abelianized fundamental group of the component $\X_i$.
\end{article} %% Periods-of-closed-1-forms

\begin{proof}
  The space $\X$ is assumed to be connected.
  Thanks to the identity $d \circ \CHK + \CHK \circ d = \1^* - \0^*$,
  applied to $\alpha$,
  and restricted to the subspace $\Loops(\X)$,
  for which $\1^* = \0^*$,
  and given that $\alpha$ is closed,
  we get $d[\CHK\!\alpha \restriction \Loops(\X)] = 0$.
  Now,
  let us prove that the set $\PeriodsOf(\alpha)$ is a subgroup of $(\RR,+)$.
  Let $\ell$ and $\ell'$ be two loops,
  based in $x$ and $x'$,
  that is,
  $\ell(0) = \ell(1) = x$, $\ell'(0) = \ell'(1) = x'$.
  Let $a = \int_\ell \alpha$ and $a' = \int_{\ell'}\alpha$,
  we want to find a loop $\ell''$ such that $\int_{\ell''}\alpha = a+a'$.
  Since $\X$ is connected,
  there exists a path $\gamma$ connecting $x$ to $x'$,
  that is,
  $\gamma(0) = x$ and $\gamma(1) = x'$.
  Now let us consider the smashed concatenation \xart{Homotopy-of-paths}{Note} $\ell'' = \ell\star(\gamma\star(\ell\star \bar\gamma))$,
  where $\bar\gamma(t) = \gamma(1-t)$,
  $\ell''$ is a loop based at $x$ and $\int_{\ell''} \alpha = \int_\ell\alpha + \int_\gamma\alpha + \int_{\ell'} \alpha + \int_{\bar\gamma} \alpha $,
  but since $\int_{ \bar\gamma} \alpha = - \int_\gamma\alpha$,
  $\int_{\ell''} = a + a'$.
  Now,
  if $a = \int_\ell\alpha$,
  then $\int_{ \bar\ell}\alpha = -a$.
  Hence,
  $\PeriodsOf(\alpha)$ is a subgroup of $\RR$.
  Next,
  let $\ell$ and $\ell'$ be two homotopic (free or not) loops,
  that is,
  such that there exists a path $[s \mapsto \ell_s] \in \Cinfty(\RR,\Loops(\X))$ with $\ell_0 = \ell$ and $\ell_1 = \ell'$.
  Thanks to the homotopic invariance of the De Rham cohomology \art{Homotopic-invariance-of-the-De-Rham-cohomology},
  we get $\ell_1^*\alpha = \ell_0^*\alpha + d\beta$,
  then $\int_{\ell'}\alpha = \int_\ell\alpha + \int_\ell d\beta$,
  but thanks to Stokes' theorem \art{The-Stokes-theorem} $\int_\ell d\beta = \int_{\Der \ell}\beta = 0$ since $\Der \ell = 0$.
  Hence,
  $\int_{\ell'}\alpha = \int_\ell\alpha$.
  Therefore,
  the
  group of periods $\Periods_\alpha = \PeriodsOf(\alpha)$ is the homomorphic image of $\pi_1(\X,x)$ by the integration morphism.
\end{proof}

\begin{article}\artlabel{Integrating closed 1-forms}
  \addcontentsline{toc}{section}{\small\hspace{10pt} Integrating closed 1-forms}
  \label{Integrating-closed-1-forms}
  Let $\X$ be a connected diffeological space,
  $\pi_0(\X) = \{\X\}$ \art{Pathwise-connectedness}.
  Let $\alpha$ be a closed $1$-form,
  $\alpha \in \DForms^1(\X)$ and $d\alpha=0$.
  Let $\CHK\!\alpha \in \DForms^0(\Paths(\X)) = \Cinfty(\Paths(\X),\RR)$,
  where $\CHK$ is the Chain-Homotopy operator \art{The-Chain-Homotopy-operator-K}.
  Restricted to the subspace $\Paths(\X,\xo,\star)$ of the paths in $\X$ with origin $\xo \in \X$,
  it satisfies $d\left[\CHK\!\alpha \restriction \Paths(\X,\xo,\star)\right] = \1^*(\alpha)$.
  Let $\X_\alpha$ be the quotient of $\Paths(\X,\xo,\star)$ by the equivalence relation
  \renewcommand{\theequation}{$\diamondsuit$}
  \begin{equation}
    \gamma \sim \gamma', \text{ if } \1(\gamma) = \1(\gamma'), \text{ and } \CHK\!\alpha(\gamma) = \CHK\!\alpha(\gamma').
  \end{equation}
  Let $\class : \Paths(\X,\xo,\star) \to \X_\alpha$ be the projection,
  and $\pr_\alpha : \X_\alpha \to \X$ be the factor defined by $\pr_\alpha \circ \class = \1$.
  Let $\F_\alpha : \X_\alpha \to \RR$ be defined by $\F_\alpha \circ \class = \CHK\!\alpha$,
  then
  $$%
  \F_\alpha(\hat x) = \int_\gamma \alpha = \int_0^1 \alpha(\gamma)_t(1) dt, \text{ with } \hat x = \class(\gamma),
  $$%
  and $\F_\alpha \in \Cinfty(\X_\alpha,\RR)$.
  The commutative diagram ($\spadesuit$) expresses the relationship between all these maps.
  
  1. The projection $\pr_\alpha : \X_\alpha \to \X$ is a Galoisian covering\index{Galoisian covering} \art{What-is-a-covering} with structure group $\Periods_\alpha$ and the function $\F_\alpha$ is a primitive of the pullback $\pr_\alpha^*(\alpha)$,
  that is,
  $d\F_\alpha = \pr_\alpha^*(\alpha)$.
  The bundle $\pr_\alpha$ will be called the {\em integration covering} of $\alpha$.
  
  Let $\Periods_\alpha$ be the group of periods of $\alpha$ \art{Periods-of-closed-1-forms},
  and let us define the {\em torus of periods} of $\alpha$ as the quotient group \art{Diffeological-groups}
  $$%
  \Torus_\alpha = \RR/\Periods_\alpha, \text{ with projection } \class_\alpha : \RR \to \Torus_\alpha.
  $$%
  Then,
  the map $\F_\alpha : \X_\alpha \to \RR$ projects onto a smooth map $f_\alpha : \X \to \Torus_\alpha$,
  that is,
  $\class_\alpha \circ \F_\alpha = f_\alpha \circ \pr_\alpha$,
  which is summarized by the following diagram.
  \renewcommand{\theequation}{$\spadesuit$}
  \begin{equation}
    \begin{tikzcd}[column sep=large, row sep=large, every label/.append style = {font = \small}]
      \Paths(\X,x_0,\star) \arrow[rr,"\class"] \arrow[dr, swap, "\1"] & {} & \X_\alpha \arrow[r,"\F_\alpha"] \arrow[dl, "\pr_\alpha"] & \RR \arrow[d,"\class_\alpha"]  \\
      {} & \X \arrow[rr, swap, "f_\alpha"] & {} & \Torus_\alpha
    \end{tikzcd}
  \end{equation}
  
  2. If the group of periods $\Periods_\alpha$ is a discrete diffeological subgroup of $\RR$,%
  \footnote{That is always the case when $\X$ is a second-countable manifold.}
  %
  which is equivalent to $\Periods_\alpha \neq \RR$,
  then there exists a unique $1$-form \renewcommand{\theequation}{$\clubsuit$}
  \begin{equation}
    \theta_\alpha \in \Omega^1(\Torus_\alpha), \text{ such that } dt = \class_\alpha^*(\theta_\alpha).
  \end{equation}
  The form $\theta_\alpha$ is the {\em canonical $1$-form} on $\Torus_\alpha$.
  Moreover,
  the form $\alpha$ coincides with the pullback of $\theta_\alpha$ by $f_\alpha$,
  that is,
  \renewcommand{\theequation}{$\heartsuit$}
  \begin{equation}
    \alpha = f_\alpha^*(\theta_\alpha)
    \text{ with }
    f_\alpha(x) = \class_\alpha \bigg(\int_\xo^x \alpha \bigg),
  \end{equation}
  where the integral is computed on any path $\gamma$ connecting $\xo$ to $x$.
  We shall say, in this case, that $f_\alpha$ {\em integrates} $\alpha$,
  or that $f_\alpha$ is an {\em integration function}\index{Integration function} of $\alpha$.
  
  This whole construction,
  integration bundle and integration function,
  is illustrated in Figure \ref{fig-Integration-of-a-closed-1-form}.%
  \footnote{This drawing was partly borrowed from \emph{Structure des syst\`emes dynamiques} de J.-M. Souriau.}
  %%###########
  \begin{figure}[tb]
    \centerline{\includegraphics{Figures-PDF/fig-integration-1-forms}}
    \caption{Integration of a closed $1$-form.}
    \label{fig-Integration-of-a-closed-1-form}
  \end{figure}
  %%###########
  
  \Note{1} The proposition in \art{Closed-1-forms-vanishing-on-loops} follows directly from this construction.
  If $\Periods_\alpha = \{0\}$,
  then $\Torus_\alpha = \RR$ and $\alpha = f_\alpha^*(dt) = df_\alpha$.
  In this case,
  the integration function is a primitive of $\alpha$.
  
  \Note{2} Another important case occurs in differential geometry,
  when the group of periods $\Periods_\alpha$ is generated  by one number $a \in \RR$,
  the form is said to be {\em integral}.
  In this case, $\Torus_\alpha \sim \S^1_a = \RR/a\ZZ$ is a manifold,
  the circle of perimeter $a$,
  and $f_\alpha$ is a smooth map from $\X$ to $\S^1_a$.
  
  \Note{3} The projection $\class_\alpha : \X_\alpha \to \X$ is the smallest diffeological covering over $\X$,
  for which the pullback of $\alpha$ is exact on the total space.
  It is equivalent to the pullback $f_\alpha^*[\class_\alpha : \RR \to \Torus_\alpha] = \{(x,t) \in \X \times \RR \mid f_\alpha(x) = \class_\alpha(t) \}$.
  
  \Note{4} When the group of periods $\Periods_\alpha$ is discrete,
  the solution of the equation $\alpha = f_\alpha^*(\theta_\alpha)$ ($\clubsuit$) is unique up to a constant.
  Every solution $f'_\alpha$ writes $f'_\alpha = f_\alpha + \tau$ with $\tau \in \Torus_\alpha$,
  where the group operation on $\Torus_\alpha$ has been denoted additively,
  and $f_\alpha$ has been defined above ($\heartsuit$) by the choice of the base point $\xo$.
  
  \Note{5} It may happen that the group of periods $\Periods_\alpha$ is not discrete,
  that is,
  $\Periods_\alpha = \RR$,
  then the torus of periods is reduced to $\{0\}$.
  The map $\pr_\alpha : \X_\alpha \to \X$ defined above ($\spadesuit$) continues to be a covering of $\X$,
  with a fiber equivalent to the set of real numbers equipped with the discrete diffeology.
  But there is no longer an integration function,
  for the form $\alpha$,
  because the projection from $\RR$ to $\{0\}$ is no more a covering;
  see also \exref{The-fiber-of-the-integration bundle}.
  
  \Note{6} It is worthy to note that this integration applies to all diffeological situations,
  whether the space is infinite dimensional,
  as in the case of group of diffeomorphisms \exref{Is-the-group-Diff-S-1-simply-connected},
  or singular as in the case of irrational torus \exref{Forms-on-irrational-tori-are-closed},
  or both as in the case of the singular symplectic reduction in \cite{PIZ15}.
\end{article} %% Integrating-closed-1-forms

\begin{proof}
  Let us check that $\pr_\alpha : \X_\alpha \to \X$ is a principal covering with $\Periods_\alpha$ as structure group.
  Since $\alpha$ is closed,
  the integral $\CHK\!\alpha(\gamma) = \int_\gamma \alpha$ depends only on the fixed-ends homotopy class \art{Homotopic-invariance-of-the-De-Rham-cohomology}.
  Thus,
  the map $\CHK\!\alpha$ factorizes through the space of fixed-end homotopy classes of paths with origin $\xo$,
  that is,
  the universal covering $\pi : \tilde \X \to \X$ \art{The-universal-covering}.
  Let $\tilde \F_\alpha$ be defined by $\tilde\F_\alpha(\class(\gamma)) = \CHK\!\alpha(\gamma)$.
  Since the projection $\class : \Paths(\X,\xo,\star) \to \tilde \X$ is a subduction,
  $\tilde\F_\alpha \in \Cinfty(\tilde \X, \RR)$.
  Now,
  let $\eta_\alpha : \pi_1(\X,\xo) \to \RR$ be the homomorphism defined by the integration,
  $\eta_\alpha(\class(\ell)) = \int_\ell \alpha$,
  then $\X_\alpha = \tilde\X / \ker(\eta_\alpha)$.
  Indeed,
  let $k = \class(\ell) \in \pi_1(\X,\xo)$,
  and $\tilde x = \class(\gamma) \in \tilde \X$,
  so $\tilde\F_\alpha(k(\tilde x)) = \int_{\ell \star \gamma} \alpha = \int_\ell \alpha + \int_\gamma \alpha$,
  where $\star$ is the smashed concatenation of paths.
  Hence,
  if $k \in \ker(\eta_\alpha)$,
  that is,
  if $\int_\ell\alpha = 0$,
  then $\tilde\F_\alpha(k(\tilde x)) = \F_\alpha(\tilde x)$.
  Now,
  if $\tilde\F_\alpha(\tilde x) = \tilde\F_\alpha(\tilde x')$,
  with $\tilde x' = \class(\gamma')$ and $\1(\gamma') = \1(\gamma)$,
  then $\int_\gamma\alpha = \int_{\gamma'}\alpha$,
  but $\int_{\gamma'}\alpha - \int_{\gamma}\alpha = \int_\ell \alpha$ with $\ell = \gamma' \star \bar\gamma \in \Loops(\X,\xo)$.
  Then,
  $\tilde x' = k(\tilde x)$ with $k = \class(\ell)$ and $\eta_\alpha(k) = 0$.
  Now,
  since $\eta_\alpha$ is a homomorphism,
  the projection $\pr_\alpha : \X_\alpha = \tilde \X/\ker(\eta_\alpha) \to \X$ is a covering \art{The-universal-covering},
  and since $\eta_\alpha$ is a homomorphism to an Abelian group,
  $\ker(\eta_\alpha)$ is an invariant subgroup of $\pi_1(\X,\xo)$ and the covering $\pr_\alpha : \X_\alpha \to \X$ is a principal covering,
  with fiber $\pi_1(\X)/\ker(\eta_\alpha) = \Val(\eta_\alpha) = \Periods_\alpha$.
  Now,
  $d[\CHK\!\alpha] = \1^*(\alpha)$ writes $d[\F_\alpha \circ \class] = (\pr_\alpha \circ \class)^*(\alpha)$,
  that is, $\class^*[d\F_\alpha)] = \class^*[\pr_\alpha^*(\alpha)]$,
  and since $\class$ is a subduction,
  $\pr_\alpha^*(\alpha) = d\F_\alpha$ \art{Vanishing-forms-on-quotients}.
  Finally,
  if $\Periods_\alpha$ is a discrete diffeological subgroup of $\RR$,
  then $\class_\alpha : \RR \to \Torus_\alpha$ is a covering,
  and since the $1$-form $dt \in \Omega^1(\RR)$ is invariant by $\Periods_\alpha$,
  there exists a unique $1$-form $\theta_\alpha \in \Omega^1(\Torus_\alpha)$ such that $\class_\alpha^*(\theta_\alpha) = dt$ \art{Coverings-and-differential-forms}.
  Now,
  $(\class_\alpha \circ \F_\alpha)^*(\theta_\alpha) = (f_\alpha \circ \pr_\alpha)^*(\theta_\alpha)$,
  thus:
  $\F_\alpha^*(dt) = \F_\alpha^*(\class_\alpha^*(\theta_\alpha)) = \pr_\alpha^*(f_\alpha^*(\theta_\alpha))$,
  but $\F_\alpha^*(dt) = d\F_\alpha = \pr_\alpha^*(\alpha)$,
  hence $\pr_\alpha^*(\alpha) = \pr_\alpha^*(f_\alpha^*(\theta_\alpha))$.
  Then,
  since $\pr_\alpha$ is a subduction, $f_\alpha^*(\theta_\alpha) = \alpha$ \art{Vanishing-forms-on-quotients}.
\end{proof}

\begin{article}\artlabel{The cokernel of the first De Rham homomorphism}\index{De Rham homomorphism}
  \addcontentsline{toc}{section}{\small\hspace{10pt} The cokernel of the first De Rham homomorphism}
  \label{The-cokernel-of-the-first-De-Rham-homomorphism}
  Let $\X$ be a connected diffeological space and $\tilde \X$ be its universal covering \art{The-universal-covering}.
  The injectivity of the first De Rham homomorphism $\eta$ \xart{Closed-1-forms-vanishing-on-loops}{Note 2} can be reformulated as the short sequence
  $$%
  0 \Rfl{10mm}{} \HDR^1(\X) \Rfl{10mm}{\eta} \Hom(\pi_1(\X),\RR).
  $$%
  We would close now this sequence,
  on the right,
  by interpreting geometrically the cokernel
  $$%
  \coker(\eta) \simeq \Hom(\pi_1(\X),\RR)/\HDR^1(\X).
  $$%
  Let $\rho$ be any homomorphism $\rho \in \Hom(\pi_1(\X),\RR)$.
  Let us consider the diagonal action of $\pi_1(\X)$ on the direct product $\tilde \X \times \RR$,
  defined by
  $$%
  k(\tilde x, t) = (k(\tilde x), t+\rho(k)),
  $$%
  for all $k \in \pi_1(\X)$ and $(\tilde x, t) \in \tilde \X \times \RR$,
  where we have denoted by $(k,\tilde x) \mapsto k(\tilde x)$ the natural action of $\pi_1(\X)$ on $\tilde \X$.
  The diagonal action of $\pi_1(\X)$ on $\tilde \X \times \RR$ is free and we shall denote the quotient by
  $$%
  \X_\rho = (\tilde \X \times \RR)/\pi_1(\X).
  $$%
  Let $\bmp$ be the projection from $\tilde \X \times \RR$ to $\tilde \X_\rho$.
  The diffeological space $\X_\rho$,
  quotient of the product $\tilde \X \times \RR$ by $\rho$,
  is the total space of the associated fibration \art{Associated-fiber-bundles}
  $$%
  p: \X_\rho \to \X, \text{ with } p: \bmp(\tilde x,t) \mapsto \pi(\tilde x).
  $$%
  This construction is summarized by the next commutative diagram.
  The projection $p: \X_\rho \to \X$ is a principal bundle with structure group $(\RR,+)$ \art{Principal-diffeological-fiber-bundles},
  whose action is given by
  $$%
  s(\bmp(\tilde x,t)) = \bmp(x,t+s),
  $$%
  for all $s \in \RR$ and all $\bmp(\tilde x,t) \in \X_\rho$.
  \begin{center}
    \begin{tikzcd}[column sep=large, row sep=large, every label/.append style = {font = \small}]
      \tilde \X \times \RR \arrow[r,"\bmp"] \arrow[d, swap, "\pr_1"] & \X_\rho \arrow[d, "p"]  \\
      \tilde \X \arrow[r, swap, "\pi"] & \X
    \end{tikzcd}
  \end{center}
%
  Moreover,
  this fiber bundle $p: \X_\rho \to \X$ is trivial if and only if the homomorphism $\rho$ is the image of a closed $1$-form by the De Rham homomorphism,
  that is,
  if and only if there exists a closed $1$-form $\alpha \in \ZDR^1(\X)$ such that
  $$%
  \rho(k) = \int_\ell \alpha\,,
  $$%
  for all $k \in \pi_1(\X)$ with $k = \class(\ell)$ and $\ell \in \Loops(\X)$.
  Therefore,
  $\coker(\eta)$ can be interpreted as the group of equivalence classes of principal flat $(\RR,+)$-bundles over $\X$,
  that is,
  the ones whose pullback on the universal covering $\tilde \X$ is trivial.
  
  \Note~This result also appeared in the comparison between De Rham and \v{C}ech cohomologies \cite{PIZ21b}.
\end{article} %% The-cokernel-of-the-first-De-Rham-homomorphism

\begin{proof}
  Let us begin by assuming that $\rho$ is associated with the closed $1$-form $\alpha$ on $\X$.
  The pullback $\pi^*(\alpha) \in \DForms^1(\tilde \X)$,
  of $\alpha$ by the projection $\pi$,
  is closed.
  But since $\tilde \X$ is simply connected, $\pi^*(\alpha)$ is exact \art{Vanishing-De-Rham-1-cohomology},
  let $f \in \Cinfty(\tilde \X,\RR)$ be a primitive of $\pi^*(\alpha)$.
  Since $k^*(df) = df$, for all $k \in \pi_1(\X)$,
  $f$ satisfies
  $$%
  f(k(\tilde x)) = f(\tilde x) + \rho(k).
  $$%
  Then,
  the smooth map $\Sigma$,
  defined by
  $$%
  \Sigma: \tilde x \mapsto (\tilde x, f(\tilde x)) \in \Cinfty(\tilde \X,\tilde \X\times \RR), \text{ satisfies } \Sigma(k(\tilde x)) =  k(\Sigma(\tilde x)).
  $$%
  Therefore,  $\Sigma$ projects on a map $\sigma: \X \to \tilde \X_\rho$ such that $\sigma \circ \pi = \bmp \circ \Sigma$.
  Since $\pi$ and $\bmp$ in the diagram ($\clubsuit$) are subductions,
  and since $\Sigma$ is smooth,
  the map $\sigma$ is smooth.
  Now,
  the map $\sigma$ is a smooth section of a principal fiber bundle,
  $p \circ \sigma = \id_\X$,
  and the bundle $p: \X_\rho \to \X$ is trivial \art{Principal-diffeological-fiber-bundles}.
  \begin{equation}
    \renewcommand{\theequation}{$\clubsuit$}
    \begin{tikzcd}[column sep=large, row sep=large, every label/.append style = {font = \small}]
      \tilde \X \times \RR \arrow[r,"\bmp"]  & \X_\rho   \\
      \tilde \X \arrow[r, swap, "\pi"] \arrow[u, "\Sigma"] & \X \arrow[u, swap, "\sigma"]
    \end{tikzcd}
  \end{equation}
  %
  Conversely,
  let us assume that the bundle $p: \X_\rho \to \X$ is trivial.
  Then,
  there exists a smooth section $\sigma: \X \to \tilde \X_\rho$.
  But the map $\sigma \circ \pi$ is smooth and $\tilde \X$ is simply connected,
  thus $\bmp$ is the universal covering of $\X_\rho$.
  Now,
  thanks to the monodromy theorem \art{Monodromy-theorem},
  there exists a smooth map $\Sigma:  \tilde \X \to \tilde \X \times \RR$ lifting $\sigma \circ \pi$,
  that is,
  such that $\sigma \circ \pi = \bmp \circ \Sigma$.
  This lift is unique with the condition $\sigma(\xo) = [\tilde \xo, 0]$,
  where $\xo$ is some point in $\X$ and $\tilde \xo$ is any point of $\tilde \X$ above $\xo$.
  The map $\Sigma$ is a smooth section of $\pr_1: \tilde \X \times \RR \to \tilde \X$.
  Thus, $\bmp \circ \Sigma = \sigma \circ \pi$ implies $p \circ \bmp \circ \Sigma = p \circ \sigma \circ \pi$,
  but $p \circ \sigma = \id_\X$ and $p \circ \bmp = \pi \circ \pr_1$,
  hence $\pi \circ \pr_1 \circ \Sigma = \pi$.
  Therefore,
  for any $\tilde x \in \tilde \X$ there exists some $\kappa(\tilde x) \in \pi_1(\X)$ such that $\pr_1 \circ \Sigma(\tilde x) =  \kappa(\tilde x)(\tilde x)$.
  But since the maps $\pr_1$ and $\Sigma$ are smooth,
  the map $\kappa$ is smooth,
  and since $\pi_1(\X)$ is discrete and $\tilde \X$ connected,
  $\kappa$ is constant.
  By the choice made above,
  $\sigma(\tilde \xo) = [\tilde \xo, 0]$,
  we get that $\kappa(\tilde x)$ is the identity for all $\tilde x$.
  Now,
  since the map $\Sigma$ is a section,
  there exists a smooth map $f \in \Cinfty(\tilde \X,\RR)$ such that $\Sigma(\tilde x) = (\tilde x, f(\tilde x))$.
  Let us consider $k \in \pi_1(\X)$,
  and let us apply $\bmp \circ \Sigma = \sigma \circ p$ to $k(\tilde x)$,
  we get $\bmp \circ \Sigma(k(\tilde x)) = \sigma \circ p(k(\tilde x)) = \sigma \circ p(x) = \bmp \circ \Sigma(\tilde x)$.
  Then,
  $\Sigma(k(\tilde x))$ and $\Sigma(\tilde x)$ are on the same orbit of $\pi_1(\X)$.
  Thus,
  there exists $h \in \pi_1(\X)$ such that $\Sigma(k(\tilde x)) = h(\Sigma(\tilde x))$,
  that is,
  $(k(\tilde x), f(k(\tilde x))) = (h(\tilde x), f(\tilde x) + \rho(h))$.
  But since the action of $\pi_1(\X)$ on $\tilde \X$ is free,
  $h = k$ and $f(k(\tilde x)) = f(\tilde x) + \rho(k)$.
  Hence,
  the differential $df$ is invariant by the action of $\pi_1(\X)$,
  $d[f \circ k] = d[f + \rho(k)] = df$.
  Therefore, there exists a closed $1$-form $\alpha$ on $\X$ such that $\pi^*(\alpha) = df$ \art{Pushing-forms-onto-quotients}.
  Finally,
  the fiber bundle $p: \tilde \X_\rho \to \X$ is associated with the $1$-form $\alpha$.
\end{proof}

\begin{article}\artlabel{The first De Rham homomorphism for manifolds}\index{De Rham homomorphism}
  \addcontentsline{toc}{section}{\small\hspace{10pt} The first De Rham homomorphism for manifolds}
  \label{The-first-De-Rham-homomorphism-for-manifolds}
  It is known classically that every bundle with a contractible fiber over a manifold $\M$ admits a smooth section \cite{Die70c}.
  Then, any $(\RR,+)$ principal bundle over a manifold admits a section and is thus trivial.
  Therefore,
  the cokernel of the first De Rham homomorphism is trivial,
  and the De Rham homomorphism $\eta$ is an isomorphism.
  This is a part of the De Rham theorem for manifolds,
  which says the general De Rham homomorphism is an isomorphism in each degree.
  As we can see in \exref{De-Rham-homomorphism-and-irrational-tori},
  this is not always the case in diffeology.
  But the construction above \art{The-cokernel-of-the-first-De-Rham-homomorphism} shows that the direct proof of the De Rham isomorphism in degree 1,
  at least,
  is reduced to the existence of a section on fiber bundles with contractible fiber,
  which is by itself an interesting point.
\end{article} %% The-first-De-Rham-homomorphism-for-manifolds

%%%%%%%%%%%%%%%%%%%%%%%%%%%%%%%%%%%%%%%%%%%%%%%%%%%%%%%%%%
%
%   Exercises
%
%%%%%%%%%%%%%%%%%%%%%%%%%%%%%%%%%%%%%%%%%%%%%%%%%%%%%%%%%%

\Exercises

\begin{exercise}[De Rham homomorphism and irrational tori]\index{De Rham homomorphism}
  \label{De-Rham-homomorphism-and-irrational-tori}
  Let ${\T}_{\alpha} =$
  \linebreak
  $\RR/(\ZZ+\alpha\ZZ)$ be the irrational torus with slope $\alpha \in \RR-\QQ$.
  We know that $\HDR^1({\T}_{\alpha}) = \RR$,
  every $1$-form on ${\T}_{\alpha}$ is closed and proportional to $\theta = \pi_*(dt)$,
  where $\pi: \RR \to {\T}_{\alpha}$ is the projection;
  see \exref{Forms-on-irrational-tori-are-closed}.
  We shall admit that $\pi_1({\T}_{\alpha}) = \ZZ\times \ZZ$ is included in the universal covering $\tilde {\T}_{\alpha} = \RR$ as the numbers $\ZZ+\alpha \ZZ$.
  Then,
  any morphism $\rho \in \Hom(\pi_1({\T}_{\alpha}),\RR)$ writes $\rho: (n,m) \mapsto r n + s m$ with $(r,s) \in \RR^2$.
  Now,
  for any closed $1$-form $c \times \theta$ of ${\T}_{\alpha}$,
  the homomorphism $\rho_c: \class(\ell) \mapsto \int_\ell c \times \theta$ is given by $\rho_c(n,m) = c(n +\alpha m)$.
  
  Question: Describe the space $\RR\times_\rho\RR = [\RR\times \RR]/\rho_c$ and its fibration on ${\T}_{\alpha}$.
\end{exercise} %% De-Rham-homomorphism-and-irrational-tori

\begin{exercise}[The fiber of the integration bundle]
  \label{The-fiber-of-the-integration bundle}
  Prove directly that the fiber of the projection $\class_\alpha : \X_\alpha \to \X$ \art{Integrating-closed-1-forms} is discrete.
  Use the formula of the variation of the integral of a differential form \art{Variation-of-the-integral-of-a-form-on-a-cube}.
\end{exercise} %% The-fiber-of-the-integration bundle

%%%%%%%%%%%%%%%%%%%%%%%%%%%%%%%%%%%%%%%%%%%%%%%%%%%%%%%%%%
%% MARK: Connections on Diffeological Fiber Bundles
%%%%%%%%%%%%%%%%%%%%%%%%%%%%%%%%%%%%%%%%%%%%%%%%%%%%%%%%%%

\section*{Connections on Diffeological Fiber Bundles}
\label{Connections-on-diffeological-fiber-bundles}

\begin{sechead}
  In this section we suggest what should be a connection in diffeology.
  This is more a program than a strict immutable definition.
  The general idea behind the concept of connection on a principal fiber bundle is to give a way for lifting paths from the base space to the total space,
  in an equivariant way, and such that,
  once a source point of the lift is given,
  the lift is unique.
  The main and simplest example of a connection is the lifting of paths in coverings.
  Once a connection is chosen in a principal bundle,
  it gives a way for lifting paths in every associated fiber bundle.
  Naturally,
  this lifting process must satisfy some locality and compatibility properties.
  We shall suggest a series of conditions,
  fulfilled by the standard connections,
  which seem appropriate to be required in a generalization to diffeology.
  We treat in particular the case of principal bundles with generalized tori $\RR/\Gamma$ as structure groups,
  where $\Gamma$ is any subgroup of $\RR$ and for which the connection is defined by a $1$-form.
  These principal bundles play a major role in the geometric integration of closed $2$-forms \art{Integration-bundles-of-closed-2-forms}.
  Also,
  an important property is that the pullbacks by homotopic smooth maps of a principal bundle,
  equipped with a connection, are equivalent \art{Connections-and-equivalence-of-pullbacks}.
  This property,
  always satisfied in the geometry of manifolds,
  is related to the existence of connections on ordinary principal fiber bundles.
  
  This section does not intend to exhaust all the consequences of such a definition and leaves the field open for future  investigations.
  This approach of connections in diffeology follows the classical theory presented in \cite{KoNo63} and what I began in \cite{Igl87}.
  It is possible to give a more synthetic approach,
  by reduction of fibrating groupoids, which I leave for later.
\end{sechead}

\begin{article}\artlabel{Connections on principal fiber bundles}
  \addcontentsline{toc}{section}{\small\hspace{10pt} Connections on principal fiber bundles}
  \label{Connections-on-principal-fiber-bundles}
  Let $\Y$ be a diffeological space.
  We denote by  $\Paths_\loc(\Y)$ the space of {\em local paths}\index{Local path} in $\Y$,
  that is,
  the set of $1$-plots of $\Y$ defined on open intervals,
  $$%
  \Paths_\loc(\Y) = \{ \tilde c \in \Cinfty(\openinterval{a,b},\Y) \mid a,b \in \RR \},
  $$%
  equipped with the functional diffeology induced by the functional diffeology of the $1$-plots of $\Y$ \art{Functional-diffeology-of-a-diffeology}.
  Let us denote then by $\tbPaths_\loc(\Y)$ the {\em tautological bundle}\index{Tautological bundle},
  equipped naturally with the subset diffeology of the product,
  $$%
  \tbPaths_\loc(\Y) = \{ (\tilde c,t) \in \Paths_\loc(\Y) \times \RR \mid t \in \Dom(\tilde c) \}.
  $$%
  Next,
  let $\pi : \Y \to \X$ be a principal diffeological fibration with structure group $\G$,
  and let $(g,y) \mapsto g_\Y(y)$ denote the action of $\G$ on $\Y$.
  
  \Definition \begin{em}
    We shall call \emph{connection}\index{Connection} on the $\G$-principal fiber bundle $\pi : \Y \to \X$ any smooth map
  \begin{equation}
    \renewcommand{\theequation}{$\spadesuit$}
    \Theta : \tbPaths_\loc(\Y) \to \Paths_\loc(\Y)
  \end{equation}
  satisfying the following series of conditions.
  
  1. {\em Domain}. $\Dom(\Theta(\tilde c,t)) = \Dom(\tilde c)$.
  
  2. {\em Lifting}. $\pi \circ \Theta(\tilde c,t) = \pi \circ \tilde c$.
  
  3. {\em Basepoint}. $\Theta(\tilde c,t)(t) = \tilde c(t)$.
  
  4. {\em Reduction}.  $\Theta(\gamma \cdot \tilde c,t) = \gamma(t)_\Y \circ \Theta(\tilde c,t)$,
  where $\gamma : \Dom(\tilde c) \to \G$ is any smooth path and $\gamma \cdot \tilde c = [s \mapsto \gamma(s)_\Y(\tilde c(s))]$.
  
  5. {\em Locality}. $\Theta(\tilde c \circ f,s) = \Theta(\tilde c,f(s)) \circ f$,
  where $f$ is any smooth local path defined on an open domain with values in $\Dom(\tilde c)$.
  
  6. {\em Projector}. $\Theta(\Theta(\tilde c,t),t) = \Theta(\tilde c,t)$.
  \end{em}
  
  The local path $\Theta(\tilde c,t)$ is the {\em horizontal projection} of $\tilde c$ pointed at $t$;
  it is a {\em horizontal path}\index{Horizontal path} for the connection $\Theta$.
  The set of horizontal local paths will be denoted by
  $$%
  \HorPaths_\loc(\Y,\Theta) = \{ \Theta(\tilde c,t) \mid \tilde c \in \Paths_\loc(\Y), \text{ and } t \in \Dom(\tilde c) \}.
  $$%
  We may also denote by $\Theta_t(\tilde c)$ or $\Theta(\tilde c)(t)$ the horizontal path $\Theta(\tilde c, t)$.
  Then, these diffeological connections satisfy the following series of properties.
  
  (a)  Let $\bar c$ and $\bar c'$ be two horizontal paths in $\Y$ defined on the same domain,
  such that $\pi \circ \bar c = \pi \circ \bar c'$.
  If they have one value in common,
  then they coincide.
  
  (b)  Let $\tilde c$ be a local path in $\Y$, and let $s,t$ be two points in $\Dom(\tilde c)$.
  If $\tilde c(t) = g_\Y(\Theta(\tilde c, s)(t))$,
  for some $g$ in $\G$,
  then $\Theta(\tilde c,t) = g_\Y \circ \Theta(\tilde c,s)$.
  
  (c) Let $\tilde c$ and $\tilde c'$ be two local paths in $\Y$,
  and let us assume that $\tilde c$ and $\tilde c'$ are compatible,
  that is,
  $\tilde c \restriction \Dom(\tilde c) \cap \Dom(\tilde c') = \tilde c' \restriction \Dom(\tilde c) \cap \Dom(\tilde c')$.
  Let $\tilde c \cup \tilde c'$ be the smallest common extension.
  Let $t \in \Dom(\tilde c)$ and $s \in \Dom(\tilde c) \cap \Dom(\tilde c')$,
  there exists a unique $g \in \G$ such that $\Theta(\tilde c,t)(s) = g_\Y(\tilde c'(s))$,
  then $\Theta(\tilde c,t)$ and $g_\Y \circ \Theta(\tilde c',s)$ are compatible and
  $$%
  \Theta(\tilde c \cup \tilde c',t) = \Theta(\tilde c,t) \cup g_\Y \circ \Theta(\tilde c',s).
  $$%
  
  (d) The constant paths are fixed by the connection.
  For all $y \in \Y$ and all $t$,
  $$%
  \Theta([s \mapsto y],t) = [s \mapsto y].
  $$%
  
  (e) For any path $c \in \Paths_\loc(\X)$,
  any $t \in \Dom(c)$,
  any $y \in \pi^{-1}(c(t))$,
  there exists a unique path $\bar c$ in $\Y$,
  such that
  $$%
  \bar c \in \HorPaths_\loc(\Y, \Theta)\text{ with } \pi \circ \bar c = c, \text{ and } \bar c(t) = y.
  $$%
  The path $\bar c$ is called the {\em horizontal lift}\index{Horizontal lift} of $c$,
  pointed at $y$ at time $t$.
  Therefore, we get a map
  \begin{equation}
    \renewcommand{\theequation}{$\clubsuit$}
    \hor_\Theta : (c,t,y) \mapsto \bar c \in \HorPaths_\loc(\Y,\Theta) \subset \Paths(\Y),
  \end{equation}
  defined on
  $$%
  \ev^*(\Y) = \{ (c,t,y) \in \Paths_\loc(\X) \times \RR \times \Y \mid t \in \Dom(c), \text{ and } c(t) = \pi(y) \},
  $$%
  where $\ev : \tbPaths_\loc(\X) \to \X$ is the evaluation $\ev(c,t) = c(t)$.
  The {\em horizontal lifting} is smooth and $\G$-equivariant,
  $$%
  \hor_\Theta(c,t)(g_\Y(y)) = g_\Y \circ \hor_\Theta(c,t)(y), \text{ for all } g \in \G.
  $$%
  Thanks to the axiom of locality,
  the horizontal lift of a composite is given by
  $$%
  \hor_\Theta(c \circ f, s, y) = \hor_\Theta(c, f(s),y)\circ f.
  $$%
  We also remark that,
  thanks to (d),
  the horizontal lift of a constant path is a constant path,
  $\hor_\Theta([s \mapsto x],t)(y) = [s \mapsto y]$.
  
  (f) Let $c$ and $c'$ be two paths in $\X$ such that the concatenation $c\vee c'$ \art{Concatenation-of-paths} is smooth.
  Let $x = c(0)$,
  $y \in \pi^{-1}(x)$,
  and $y' = \hor_\Theta(c,0,y)(1)$,
  then the horizontal lifts $\hor_\Theta(c,0,y)$ and $\hor_\Theta(c',0,y')$ can be concatenated and the result is the horizontal lift of the concatenation $c \vee c'$ starting at $y$,
  that is,
  $$%
  \hor_\Theta(c \vee c',0,y) =  \hor_\Theta(c,0,y) \vee \hor_\Theta(c',0, \hor_\Theta(c,0,y)(1)).
  $$%
  In particular, this applies for all pairs of stationary paths $c$ and $c'$ \art{Stationary-paths} such
  that $c(1) = c'(0)$.
  
  \Note{1} Let $\E$ be a diffeological space with a smooth action of $\G$ denoted by $(g,v) \mapsto g_\E(v)$.
  Let $c$ be a local path in $\X$, $t \in \Dom(c)$,
  $(y,v) \in \Y \times \E$ such that $y \in \pi^{-1}(c(t))$.
  Let $\pr : \T = \Y \times_\G \E \to \X$ be the associated fiber bundle \art{Associated-fiber-bundles},
  and let $\class : \Y \times \E \to \T$ be the projection.
  Let $\bar c$ be the horizontal lift of $c$ in $\Y$ such that $\bar c(t) = y$,
  so the local path $s \mapsto \class(\bar c(s),v)$ in $\Y \times_\G \E$ covers $c$,
  it is the {\em horizontal lifting} of $c$ associated with the chosen initial conditions.
  It is also called the {\em parallel transport}\index{Parallel transport} of $\tau = \class(y,e)$ along $c$.
  
  \Note{2} We denote by $\HorPaths(\Y,\Theta)$ the set of global horizontal paths,
  that is,
  the set of all $\Theta(\tilde c,t)$ such that $\tilde c \in \Paths(\Y)$ and $t \in \RR$.
  Note that, thanks to the point (b),
  it is sufficient to consider $t=0$,
  $$%
  \HorPaths(\Y,\Theta) = \{ \Theta(\tilde c,0) \mid \tilde c \in \Paths(\Y) \}.
  $$%
  Now,
  if $\X$ is connected,
  then the projection $\chi_{\Y,\Theta} = \pi \times \pi \circ \ends_\Y$,
  from $\HorPaths(\Y,\Theta)$ to $\X \times \X$,
  which associates with a horizontal path $\bar c$ the projections  $\pi \circ \bar c (0)$ and $\pi \circ \bar c (1)$ of its ends,
  is a subduction.
  Actually,
  the projection $\pi_*: \Paths(\Y) \to \Paths(\X)$ is a principal fibration with structure group $\Paths(\G)$.
  The restriction $\pi_* \restriction \HorPaths(\Y, \Theta)$ is a reduction of $\pi_*$ to a $\G \subset \Paths(\G)$ principal subbundle \art{Principal-diffeological-fiber-bundles},
  where $\G$ injects itself in $\Paths(\G)$ as the subgroup of constant paths.
  And the map, which associates the points of $\Y$ and $\X$ with the associated constant paths,
  is a strict  morphism of $\G$-principal fiber bundles from $\pi$ to the restriction $\pi_* \restriction \HorPaths(\Y, \Theta)$.
\end{article} %% Connections-on-principal-fiber-bundles

\begin{proof}
  (a) Let $\cI = \Dom(\bar c) = \Dom(\bar c')$,
  the paths $\bar c$ and $\bar c'$ are horizontal,
  so $\Theta(\bar c,t) = \bar c$ and $\Theta(\bar c',t) = \bar c'$.
  Next,
  since $\pi \circ \bar c = \pi \circ \bar c'$ there exists a local smooth path $\gamma : \cI \to \G$ such that $\bar c'(t) = \gamma(t)_\Y(\bar c(t))$ for all $t \in \cI$ \art{Triviality-over-global-plots},
  thus $\bar c' = \gamma(t)_\Y \circ \Theta(\bar c,t) = \gamma(t)_\Y \circ \bar c$ (reduction and projector axioms).
  Hence,
  we choose $t = s$ for which $\bar c(s) = \bar c'(s)$,
  then $\gamma(s) = \id_\G$ and $\bar c = \bar c'$.
  
  (b) By the reduction axiom,
  $g_\Y(\Theta(\tilde c,s)(t)) = \Theta(g_\Y \circ \tilde c,s)(t)$,
  and by assumption $\tilde c(t) = g_\Y(\Theta(\tilde c,s)(t))$,
  then,
  since $\tilde c(t) = \Theta(\tilde c,t)(t)$,
  $\Theta(\tilde c,t)(t) = \Theta(g_\Y \circ \tilde c,s)(t)$.
  Now,
  the paths $\Theta(\tilde c,t)$ and $\Theta(g_\Y \circ \tilde c,s)$ are horizontal,
  they project on the same path $\pi \circ \tilde c$ and they coincide at one point,
  thus they coincide (a).
  Therefore,
  $\Theta(\tilde c,t) = \Theta(g_\Y \circ \tilde c,s) = g_\Y \circ \Theta(\tilde c,s)$.
  
  (c) For all $t \in \Dom(\tilde c) \cap \Dom(\tilde c')$,
  $\tilde c(t) = \tilde c'(t)$ and  $\pi(\tilde c'(t)) = \pi(\tilde c(t)) = \pi(\Theta(\tilde c,s)(t))$ (lifting axiom),
  then there exists a unique $g \in \G$ such that $\Theta(\tilde c,s)(t) = g_\Y(\tilde c'(t)) = g_\Y(\tilde c(t))$.
  Now,
  on the one hand $\Theta(\tilde c,s) \restriction \Dom(\tilde c) \cap \Dom(\tilde c') = \Theta(\tilde c \restriction \Dom(\tilde c) \cap \Dom(\tilde c'),s)$ (axiom of locality),
  and on the other hand,
  $g_\Y \circ \Theta(\tilde c',t) \restriction \Dom(\tilde c) \cap \Dom(\tilde c') = \Theta(g_\Y \circ \tilde c' \restriction \Dom(\tilde c) \cap \Dom(\tilde c'),t)$ (axiom of locality and reduction).
  Thus,
  these two restrictions are horizontal paths.
  But we have $\Theta(\tilde c,s)(t) = g_\Y(\tilde c'(t)) = \Theta(g_\Y \circ \tilde c',t)(t) = g_\Y \circ \Theta(\tilde c',t)(t)$ (basepoint axiom),
  hence $\Theta(\tilde c,s) \restriction \Dom(\tilde c) \cap \Dom(\tilde c')$ and $g_\Y \circ \Theta(\tilde c',t) \restriction \Dom(\tilde c) \cap \Dom(\tilde c')$ are two horizontal paths having a common value at $t$,
  thanks to point (a) they coincide,
  and the local paths $\Theta(\tilde c,s)$ and $g_\Y \circ \Theta(\tilde c',t)$ are compatible.
  Now, thanks again to point (a),
  $\Theta(\tilde c,s) = g_\Y \circ \Theta(\tilde c,t)$ and $g_\Y \circ \Theta(\tilde c \cup \tilde c',t) = \Theta(\tilde c \cup \tilde c',s)$.
  Thus,
  $\Theta(\tilde c,s) \cup g_\Y \circ \Theta(\tilde c',t) = g_\Y \circ \Theta(\tilde c,t) \cup g_\Y \circ \Theta(\tilde c',t) = g_\Y \circ \Theta(\tilde c \cup \tilde c',t) = \Theta(\tilde c \cup \tilde c',s)$.
  
  (d) Let $c : \openinterval{a,b} \to \X$,
  there exists a lift $\tilde c : \openinterval{a,b} \to \Y$ covering $c$ \art{Triviality-over-global-plots},
  and we can choose $\tilde c(t) = y$.
  Now,
  $\bar c = \Theta(\tilde c,t)$ satisfies the condition.
  Thanks to point (a),
  this lift is unique.
  
  (e) If $\tilde c$ covers $c$ with $c(t) = y$,
  then $\tilde c \circ f$ covers $c \circ f$ with $t = f(s)$ and $\tilde c \circ f(s) = y$,
  thus $\hor_\Theta(c \circ f,s,y) = \Theta(\tilde c \circ f, s) = \Theta(\tilde c, f(s)) \circ f = \hor_\Theta(c,f(s),y) \circ f$,
  thanks to the axiom of locality.
  Now,
  let $\bmZero : s \mapsto 0$,
  and $\bmy : s \mapsto y$,
  by the locality axiom we have $\Theta(\bmy \circ \bmZero,s)(t) = \Theta(\bmy,\bmZero(s))(\bmZero(t))$ for all $t$,
  that is,
  $\Theta(\bmy,s)(t) = \Theta(\bmy,0)(0) = \bmy(0) = y$,
  \ie\ $\Theta(\bmy,s) = \bmy$.
  
  (f) The paths $\hor_\Theta(c,0,y)$ and $\hor_\Theta(c',0, \hor_\Theta(c,0,y)(1))$ are smooth and can be concatenated since $\hor_\Theta(c,0,y)(1) = \hor_\Theta(c',0, \hor_\Theta(c,0,y)(1))(0)$.
  Then,
  thanks to point (e) and to the unicity of horizontal lift given a starting point,
  $\hor_\Theta(c \vee c',0,y)(s) = \hor_\Theta(c,0,y)(2s)$ for $s <1/2$,
  as well $\hor_\Theta(c \vee c',0,y)(s) = \hor_\Theta(c',0, \hor_\Theta(c,0,y)(1))(2s - 1)$ for $s > 1/2$.
  Now,
  $$%
  \hor_\Theta(c,0,y) \vee \hor_\Theta(c',0, \hor_\Theta(c,0,y)(1))(1/2) = y' = \hor_\Theta(c \vee c',0,y)(1/2),
  $$%
  thus concatenation and horizontal lifts commute.
  
  For Note 2, $\chi_{\Y,\Theta}$ is surjective since for every pair of points $(x,x') \in \X \times \X$ there exists a path $c$ connecting $x$ to $x'$,
  and any horizontal lift $\bar c$ satisfies $\pi \circ \bar c(0) = x$ and $\pi \circ \bar c(1) = x'$.
  Then,
  since $\ends_\Y : \Paths(\Y) \to \Y \times \Y$ is a subduction and $\pi \times \pi : \Y \times \Y \to \X \times \X$ is also a subduction,
  the composite $\pi \times \pi \circ \ends_\Y$ is a subduction.
  Now,
  the connection $\Theta : \Paths(\Y) \to \HorPaths(\Y, \Theta)$ is smooth (see point (d)) thus every plot of $\X \times \X$ lifts locally to $\Paths(\Y)$,
  and composed with $\Theta$ lifts locally in $\HorPaths(\Y, \Theta)$,
  thus $\chi_{\Y, \Theta}$ is a subduction.
\end{proof}

\begin{article}\artlabel{Pullback of a connection}
  \addcontentsline{toc}{section}{\small\hspace{10pt} Pullback of a connection}
  \label{Pullback-of-a-connection}
  Let $\pi : \Y \to \X$ be a principal fiber bundle with structure group $\G$,
  and let $\Theta$ be a connection.
  Let $\X'$ be a diffeological space and let $f : \X' \to \X$ be a smooth map.
  Let $\pr_1 : \Y' \to \X'$ be the pullback of $\pi$,
  where $\Y' = f^*(\Y)$ \art{Category-of-principal-fiber-bundles}.
  This pullback is a principal bundle with structure group $\G$ acting naturally by $g_{\Y'}(x',y) = (x',g_\Y(y))$,
  where $g_\Y$ denotes the action of $\G$ on $\Y$ and $(x',y) \in \Y'$.
  The map $\Theta':\tbPaths_\loc(\Y') \to \Paths_\loc(\Y')$,
  defined by
  \begin{equation}
    \renewcommand{\theequation}{$\clubsuit$}
    \Theta'(\tilde c',t) = [s \mapsto (\xi'(s), \Theta(\tilde c,t)(s))], \text{ with } \tilde c' : s \mapsto (\xi'(s),\tilde c(s)),
  \end{equation}
  is a connection on the pullback $\pr_1 : \Y' \to \X'$,
  it will be called the pullback\index{Pullback} of the connection $\Theta$ by $f$ and denoted also by $f^*(\Theta)$.
\end{article} %% Pullback of a connection

\begin{proof}
  The verification of the conditions 1--6 in \art{Connections-on-principal-fiber-bundles} is not difficult.
\end{proof}

\begin{article}\artlabel{Connections and equivalence of pullbacks}
  \addcontentsline{toc}{section}{\small\hspace{10pt} Connections and equivalence of pullbacks}
  \label{Connections-and-equivalence-of-pullbacks}
  Let $\X$ and $\X'$ be two connected diffeological spaces.
  Let $\pi : \Y \to \X$ be a principal fiber bundle,
  and let $t \mapsto f_t$ be a path in $\Cinfty(\X',\X)$.
  If $\pi$ admits a connection,
  then the pullbacks $\pi'_0 : f_0^*(\Y) \to \X'$ and $\pi'_1 : f_0^*(\Y) \to \X'$ are isomorphic \xart{Category-of-principal-fiber-bundles}{Note 2}.
  In particular,
  if $\X$ is contractible, then $\pi$ is trivial.
  
  \Note{1} Any fiber bundle $p : \T \to \X$ such that $\X$ is contractible \art{Contractible-diffeological-spaces},
  and such that its associated principal fiber bundle $\pi : \Y \to \X$ \art{Associated-fiber-bundles} admits a connection,
  is trivial.
  In particular,
  in the subcategory of finite-dimensional manifolds every principal fiber bundle admits a connection \cite{KoNo63}.
  Thus,
  in the classical category of manifolds,
  any fiber bundle with contractible base space is trivial.
  
  \Note{2} Let $\pi : \Y \to \X$ be a principal fiber bundle equipped with a connection $\Theta$.
  Let $\E$ be a contractible space,
  any smooth map $f : \E \to \X$ admits a smooth lift $\psi$ along $\pi$,
  that is,
  $\pi \circ \psi = f$.
\end{article} %% Connections-and-equivalence-of-pullbacks

\begin{proof}
  Let us recall that the pullback of $\pi$ by $f_t$ has total spaces
  $$%
  f_t^*(\Y) = \{ (x',y) \in \X \times \Y \mid f_t(x') = \pi(y) \}.
  $$%
  The map $x' \mapsto c_{x'} = [t \mapsto f_t(x')]$ from $\X'$ to $\Paths(\X)$ is smooth.
  For all $(x',y) \in f_0^*(\Y)$,
  let $\bar c_{x'}(y) = \hor_\Theta(c_{x'},0,y)$ be the horizontal lift of $c_{x'}$ such that $\bar c_{x'}(y)(0) = y$ \xart{Connections-on-principal-fiber-bundles}{e}.
  By construction,
  the endpoint $(x',\bar c_{x'}(y)(1))$ belongs to $f_1^*(\Y)$.
  Now,
  the map
  $$%
  \psi : f_0^*(\Y) \to f_1^*(\Y), \ \text{defined by} \ \psi(x',y) = (x',\bar c_{x'}(y)(1)),
  $$%
  satisfies $\pr_1 \circ \psi = \pr_1$,
  is equivariant with respect to the action of the structure group,
  and is smooth because $\hor_\Theta$ is smooth.
  Next,
  let us consider $x' \mapsto c'_{x'} = [t \mapsto f_{1-t}(x')]$,
  and let $\bar c'_{x'}(y) = \hor_\Theta(c'_{x'},0,y')$ be the horizontal lift of $c'_{x'}$ such that $\bar c'_{x'}(y')(0) = y'$,
  for all $(x',y') \in f_1^*(\Y)$.
  It is then clear that the map
  $$%
  \psi' : f_1^*(\Y) \to f_0^*(\Y), \ \text{defined by} \ \psi'(x',y') = (x',\bar c'_{x'}(y')(1)),
  $$%
  is inverse of $\psi$.
  Therefore,
  $\psi$ is an isomorphism from $f_0^*(\Y)$ to $ f_1^*(\Y)$.
  
  For Note 1, $\X$ is contractible means that the identity $\id_\X$ is homotopic to the constant map $x \mapsto \xo$,
  for a given $\xo$.
  Thus,
  the pullback of $p : \Y \to \X$ by the identity,
  which is equivalent to $p$ itself,
  and the pullback by the constant map,
  which is the trivial bundle $\pr_1 : \X \times p^{-1}(\xo) \to \X$,
  are isomorphic.
  
  For Note 2, the pullback of $\pi$ by $f$ is a principal fiber bundle,
  and the pullback of the connection $\Theta$ is a connection on $\pr_1 : f^*(\Y) \to \E$ \art{Pullback-of-a-connection}.
  Thus,
  since $\E$ is contractible,
  $\pr_1 : f^*(\Y) \to \E$ is trivial,
  and then admits a smooth section $\sigma : \E \to f^*(\Y)$ \art{Category-of-principal-fiber-bundles}.
  Now,
  the section $\sigma$ writes necessarily $\sigma(e) = (e,\psi(e))$,
  where $\psi$ is a smooth map from $\E$ to $\Y$,
  $\psi$ is a smooth lift of $f$.
\end{proof}

\begin{article}\artlabel{The holonomy of a connection}
  \addcontentsline{toc}{section}{\small\hspace{10pt} The holonomy of a connection}
  \label{The-holonomy-of-a-connection}
  Let $\pi : \Y \to  \X$ be a principal fibration with structure group $\G$,
  equipped with a connection $\Theta$.
  Let us choose two points,
  $x \in \X$ and $y \in \Y_x = \pi^{-1}(x)$,
  and let $\Psi$ be the map associating with every path $c$ in $\X$,
  pointed at $x$,
  the end of the horizontal lift of $c$,
  pointed at $y$,
  $$%
  \Psi : \Paths(\X,x) \to \Y, \ \text{such that} \ \Psi : c \mapsto \bar c(1), \ \text{with} \ \bar c = \hor_\Theta(c,0,y).
  $$%
  The map $\Psi$ is smooth and covers a smooth map $\varphi : \X \to \Y/\G_y$ according to the following commutative diagram,
  \begin{equation}
    \renewcommand{\theequation}{$\clubsuit$}
    \begin{tikzcd}[column sep=large, row sep=large, every label/.append style = {font = \small}]
      \Paths(\X,x) \arrow[r,"\Psi"] \arrow[d, swap, "\1"] & \Y \arrow[d, "\pr"]  \\
      \X \arrow[r, swap, "\varphi"] & \Y/\G_y
    \end{tikzcd}
  \end{equation}
  %
  where $\G_y$ is roughly speaking the image by $\Psi$ of the fiber of the subduction $\1$ over $x$,
  that is,
  $\1^{-1}(x) = \Loops(\X,x)$.
  Precisely,
  $$%
  \G_y = \orb{y}^{-1}\bigg(\Psi\big(\Loops(\X,x)\big)\bigg),
  $$%
  where $\orb{y} : g \mapsto g_\Y(y)$ is the orbit map,
  identifying $\G$ with $\Y_x$.
  We have denoted by $g_\Y$ the action of $g \in \G$ on $\Y$.
  Formally,
  $$%
  \G_y = \{ g \in \G \mid \exists \ell \in \Loops(\X,x), \hor_\Theta(\ell,0,y)(1) = g_\Y(y) \}.
  $$%
  The set $\G_y$ is a subgroup of $\G$,
  depending on the choice of the point $y$ by conjugation,
  it is called the {\em holonomy group}\index{Holonomy group} of $\Theta$ at the point $y$.
  Its conjugacy class,
  regarded as a group of automorphisms of the fiber $\Y_x$,
  is called the holonomy group at the point $x$.
  
  The image $\Y_\Theta(y) = \Psi(\Paths(\X,x)) \subset \Y$,
  which is a connected subspace of $\Y$,
  since $\Psi$ is smooth and $\Paths(\X,x)$ is connected,
  is precisely made of the points of $\Y$ which can be connected to $y$ by a horizontal path, that is,
  \begin{equation}
    \renewcommand{\theequation}{$\spadesuit$}
    \Y_\Theta(y) = \{\bar c(1) \in \Y \mid \bar c \in \HorPaths(\Y,\Theta), \text{ and } \bar c(0) = y \}.
  \end{equation}
  The restriction $\pi_\Theta = \pi \restriction \Y_\Theta(y)$ is a principal fiber bundle over the connected component $\comp(x) \subset \X$ with structure group $\G_y$.
  This is the {\em holonomy bundle}\index{Holonomy bundle} of the connection $\Theta$ with reference point $y$.
  Said a little bit differently,
  if $\X$ is connected,
  then the holonomy principal bundle $\pi_\Theta : \Y_\Theta(y) \to \X$,
  with structure group $\G_y$, is a reduction  of the $\G$-principal fiber bundle $\pi : \Y \to \X$.
  
  \Note{1} If the holonomy $\G_y$ is discrete,
  then the connection $\Theta$ is said to be {\em flat}.
  Assuming $\X$ is connected,
  if $\Theta$ is flat,
  then $\pi_\Theta : \Y_\Theta(y) \to \X$ is a connected covering onto $\X$.
  The $\G$-principal fiber bundle $\pi : \Y \to \X$ is equivalent to the associated bundle $\pr : \Y_\Theta \times_{\G_y} \G \to \X$,
  where the holonomy $\G_y$ acts by $g : (y',g') \mapsto (g_\Y(y'), g'\cdot g^{-1})$,
  with $g \in \G_y$ and $(y',g') \in \Y_\Theta \times \G$.
  Then,
  $\Theta$ is equivalent to the connection associated with the natural lifting of paths in a covering;
  see below \art{The-natural-connection-on-a-covering}.
  That said,
  every principal bundle equipped with a flat connection is equivalent to such an associated bundle,
  which can be built using the universal covering $\tilde \X$ of $\X$.
  Precisely,
  $\Y \simeq \tilde \X \times_{\pi_1(\X,x)} \G$ where $\pi_1(\X,x)$ acts by multiplication on $\G$ through a homomorphism called the {\em monodromy representation}\index{Monodromy representation}.
  In particular,
  every principal bundle equipped with a flat connection on a simply connected space is necessarily trivial,
  but also note that there exist non-flat connections on trivial principal bundles.
  
  \Note{2} The subset $\G_y^\circ \subset \G_y$,
  defined by the horizontal lifts of loops in $\X$ homotopic to $\bmx = [t \mapsto x]$,
  is an invariant subgroup and is called the {\em reduced holonomy group}\index{Reduced holonomy group}.
  This is actually the holonomy of the pullback of the connection $\Theta$ \art{Pullback-of-a-connection} on the universal covering $\pr : \tilde \X \to \X$.
  The quotient $\G_y/\G_y^\circ$ is called the {\em monodromy}\index{Monodromy} of the connection $\Theta$,
  it is a homomorphic image of $\pi_1(\X,x)$.
\end{article} %% The-holonomy-of-a-connection

\begin{proof}
  First of all,
  let us note that any loop in $\X$ based at $x$ lifts in a unique path  $\bar c$ of $\Y$ such that $y = \bar c(0)$ and $\bar c(1) \in \Y_x$.
  Indeed, we use first any lift,
  thanks to \art{Triviality-over-global-plots},
  then we apply $\Theta$.
  Then,
  it is also clear that every horizontal path $\bar c$,
  such that $\bar c(0) = y$ and $\bar c(1) \in \Y_x$,
  projects to a loop $\ell$ in $\X$ based at $x$.
  Thus,
  there is a one-to-one correspondence between the loops in $\X$ based at $x$ and the horizontal paths in $\Y$,
  starting at $y$ and ending in $\Y_x$.
  %
  Now,
  let us prove that $\G_y \subset \G$ is a subgroup,
  let $\bar c$ and $\bar c'$ be the horizontal lifts,
  starting at $y$,
  of two loops $\ell$ and $\ell'$ based at $x$.
  Let $g,g' \in \G_y$ such that $\bar c(1) = g_\Y(y)$ and $\bar c'(1) = g_\Y(y)$.
  Let us assume that $\bar c$ and $g_\Y \circ \bar c'$ can be concatenated into a smooth path.
  Thus,
  the path $\bar c \vee (g_\Y \circ \bar c')$ is horizontal and satisfies $[\bar c \vee (g_\Y \circ \bar c')](1) = (g_\Y \circ \bar c')(1) = g_\Y(g'_\Y(y)) = (gg')_\Y(y)$ \xart{Connections-on-principal-fiber-bundles}{f}.
  Therefore,
  $gg'$ belongs to $\G_y$.
  If $c$ and $c'$ cannot be concatenated,
  because $c \vee c'$ is not smooth,
  then we first smash them,
  $\bar c^\star = \bar c \circ \lambda$ and ${\bar c'}{}^\star = \bar c' \circ \lambda$ \art{Homotopy-of-paths}.
  Thanks to the locality axiom,
  these paths are still horizontal,
  indeed $\Theta(\bar c^\star,0) = \Theta(\bar c \circ \lambda,0) = \Theta(\bar c, \lambda(0))\circ \lambda =  \Theta(\bar c,0) \circ \lambda = \bar c \circ \lambda = \bar c^\star$,
  they have the same ends $\bar c^\star(0) = \bar c(0)$ and $\bar c^\star(1) = \bar c(1)$,
  and since the same holds for $\bar c'$,
  $\bar c^\star$ and $g_\Y \circ {\bar c'}{}^\star$ can be concatenated.
  For the inverse,
  if $g \in \G_y$,
  then let $\bar c$ be such that $g_\Y(y) = \bar c(1)$,
  let $\rev(\bar c) : t \mapsto \bar c(1-t)$.
  Thanks to the axioms of reduction and locality,
  the path $\bar c' = g^{-1}_\Y \circ \rev(\bar c)$ is still horizontal,
  it satisfies $\bar c'(0) = y$ and $\bar c'(1) = g^{-1}_\Y(y)$.
  Thus,
  if $g \in \G_y$,
  then $g^{-1} \in \G_y$.
  Therefore,
  $\G_y$ is a subgroup of $\G$.
  
  Next,
  let $c$ and $c'$ be two paths in $\X$ starting at $x$ and ending at $x'$,
  that is,
  $\0(c)=\0(c') = x$ and $\1(c) = \1(c') = x'$.
  We assume $c$ and $c'$ stationary,
  if it is not the case we smash them as above.
  Let $\bar c$ and $\bar c'$ be their horizontal lifts starting at $y$,
  thus there exists $g \in \G$ such that $g_Y(\bar c'(1)) = \bar c(1)$,
  since $\pi(\bar c(1)) = \pi(\bar c'(1))$.
  The path $g_\Y \circ \bar c'$ is horizontal,
  stationary,
  and the concatenation $\bar c \vee \rev(g_\Y \circ \bar c')$ is smooth,
  horizontal,
  starts at $y$ and ends at $\rev(g_\Y \circ \bar c')(1) = g_\Y \circ \bar c'(0) = g_\Y(y)$.
  Thus,
  $\bar c \vee \rev(g_\Y \circ \bar c')$ is the horizontal lift of a loop based at $x$ and $g \in \G_y$.
  Therefore,
  $\pr$ denoting the projection from $\Y$ to $\Y/\G_y$ (see ($\clubsuit$)),
  the map
  $$%
  \varphi : x' \mapsto \pr(\bar c(1)), \ \text{with} \ c \in \Paths(\X,x,x') \ \text{and} \ \bar c = \hor(c,0,y),
  $$%
  is well defined,
  and smooth since $\Psi$ is smooth and $\1$ is a subduction \art{Pathwise-connectedness}. Consider the next diagram,
  the map $\varphi$ is not just a smooth map but also a section of the quotient fiber bundle $\pi_y = \pi / \G_y$.
  Then,
  according to \art{Space-of-structures-of-a-diffeological-fiber-bundle} the principal fiber bundle $\pi : \Y \to \X$ is reduced to the subbundle $\pi_\Theta : \Y_\Theta(y) \to \X$ with structure group $\G_y$.
  And by construction,
  the connection $\Theta$ follows and induces a connection on $\pi_\Theta$.
  The dependency of this construction with respect to the basepoints $x$ and $y$ is clear,
  an easy computation shows that if we change the basepoint $y$ into $g_\Y(y)$,
  then $\G_{g_\Y(y)} = g \cdot \G_y \cdot g^{-1}$.
  \begin{center}
    \begin{tikzcd}[column sep=large, row sep=large, every label/.append style = {font = \small}]
      \Y \arrow[dr, swap, "\pi"] \arrow[rr,"\pr"] & {} & \Y/\G_y \arrow[dl, "\pi_y"]  \\
      {} & \X  & {}
    \end{tikzcd}
  \end{center}
  %
  For Note 1. If the holonomy $\G_y$ is discrete,
  then $\pi_\Theta : \Y_\Theta(y) \to \X$ is a fibration with discrete fiber,
  that is,
  a covering \art{What-is-a-covering}.
  Moreover,
  since $\pi_\Theta$ is a principal bundle,
  it is a Galoisian covering,
  and we have seen that the total space $\Y_\Theta(y)$ is connected.
  Then,
  the pullback $\pr_1 : \pi_\Theta^*(\Y) \to \X$ is a $\G$ principal bundle with
  $$%
  \pi_\Theta^*(\Y) = \{ (y',y'') \in \Y \times \Y \mid y' \in \Y_\Theta(y), \text{ and } \pi(y') = \pi(y'') \},
  $$%
  where $\G$ acts trivially on the first factor and naturally on the second.
  
  Now,
  since $\sigma : y' \mapsto (y',y')$ is a smooth section,
  this pullback is trivial,
  and the map $\Phi : \Y_\Theta(y) \times \G \to \pi_\Theta^*(\Y)$ defined by $\Phi(y',g') = (y',g'_\Y(y'))$ is an isomorphism of $\G$-principal bundle \art{Category-of-principal-fiber-bundles}.
  Then,
  since $\pi : \Y \to \X$ is a subduction,
  $\pr_2 : \pi_\Theta^*(\Y) \to \Y$ is a subduction and the composite $\pr_2 \circ \Phi$ is also a subduction.
  Thus, $\Y$ is equivalent to the quotient of the direct product $\Y_\Theta(y) \times \G$ by the lift $g : (y',g') \mapsto (g_\Y(y), g' \cdot g^{-1})$ of the action of $g \in \G_y$ on $\Y_\Theta(y) \times \G$,
  associated with the equivalence $g'_\Y(y') = g''_\Y(y'')$.
  It is then clear that,
  conversely,
  every $\G$-principal fiber bundle associated this way with a Galoisian covering
  ---~and thus with the universal covering~---
  through a morphism from the covering group to $\G$,
  inherits a flat connection thanks to the natural lift of paths in coverings;
  see \xart{Connections-on-principal-fiber-bundles}{Note 1} and \art{The-natural-connection-on-a-covering}.
  \begin{center}
    \begin{tikzcd}[column sep=large, row sep=large, every label/.append style = {font = \small}]
      \Y_\Theta(y) \times \G \arrow[rr,"\Phi"] \arrow[dr, swap, "\pr_1"] & {} & \pi_\Theta^*(\Y) \arrow[r,"\pr_2"] \arrow[dl, "\pr_1"] & \Y \arrow[d,"\pi"]  \\
      {}  & Y_\Theta(y) \arrow[rr, swap, "\pi_\Theta"] & {} & \X
    \end{tikzcd}
  \end{center}
  %
  For Note 2. The map $\ell \mapsto \hor_\Theta(\ell,0,y)(1)$ is smooth \xart{Connections-on-principal-fiber-bundles}{e});
  thus,
  the map $\varphi : \ell \mapsto \orb{y}^{-1}(\hor_\Theta(\ell,0,y)(1))$,
  from $\Loops(\X,x)$ to $\G_y$ is smooth too.
  We have seen in the previous paragraph that $\varphi(\ell \vee \ell') = \varphi(\ell) \cdot \varphi(\ell')$,
  so,
  the quotient map from $\pi_1(\X,x) = \Loops(\X,x)/\comp(\bmx)$ to $\G_y/\G_y^\circ$ is a homomorphism.
  Note that $\varphi$ maps connected components in connected components,
  in particular $\G_y^\circ$ is connected since it is the image of the connected component $\comp(\bmx)$.
  Now,
  for any two loops $\ell$ and $\ell'$ homotopic to $\bmx$,
  the loop $\ell \vee \ell' \vee \rev(\ell)$ is homotopic to $\bmx$,
  thus $\G_y^\circ$ is invariant,
  $g \cdot \G_y^\circ \cdot g^{-1} \subset \G_y^\circ$.
  %
  Now, let us consider the pullback $\pr_1 : \Y' = \pr^*(\Y) \to \X$,
  and let us recall that $\Y'$ is the subspace of pairs $y' = (\tilde x, y) \in \tilde\X \times \Y$ such that $\pr(\tilde x) = \pi(y)$.
  Let us choose $\tilde x \in \pr^{-1}(x)$ and $y' = (\tilde x, y)$ as a basepoint in $\Y'$.
  The holonomy $\G_{y'}$ of $\Theta' = \pr^*(\Theta)$ is the image of $\varphi' : \tilde\ell \mapsto \orb{y'}^{-1}(\hor_{\Theta'}(\tilde\ell,0,y')(1))$,
  where $\tilde \ell \in \Loops(\tilde \X,\tilde x)$.
  But $\hor_{\Theta'}(\tilde\ell,0,y')(t) = (\tilde\ell(t), \hor_\Theta(\ell,y,0)(t))$ with $\ell = \pr \circ \tilde \ell$ and $\hor_\Theta(\ell,y,0)(1) = \varphi(\ell)_\Y(y)$.
  Thus,
  $\hor_{\Theta'}(\tilde\ell,0,y')(1) = (\tilde x, \varphi(\ell)_\Y(y)) = \varphi(\ell)_{\Y'}(\tilde x,y) = \varphi(\ell)_{\Y'}(y')$ and $\varphi'(\tilde \ell) = \varphi(\ell)$.
  Now,
  since the loops in $\tilde \X$ based at $\tilde x$ are in one-to-one correspondence with the loops in $\X$ based at $x$ and homotopic to the constant path $\bmx : t \mapsto x$,
  we get $\G_{y'} = \G_y^\circ$.
\end{proof}

\begin{article}\artlabel{The natural connection on a covering}
  \addcontentsline{toc}{section}{\small\hspace{10pt} The natural connection on a covering}
  \label{The-natural-connection-on-a-covering}
  Let $\X$ be a connected diffeological space,
  and let $\pi : \tilde \X \to \X$
  be its universal covering\index{Universal covering} \art{The-universal-covering}.
  Let $c : \openinterval{a,b} \to \X$ be a local smooth path.
  For all $t \in \openinterval{a,b}$ and $\tilde x \in \pi^{-1}(c(t))$,
  there exists a unique smooth lift $\bar c : \openinterval{a,b} \to \tilde \X$ such that $\bar c(t) = \tilde x$.
  This defines clearly the only possible connection on the covering.
  Indeed,
  the pullback of the covering by $c$ is the product of $\openinterval{a,b}$ by $\pi_1(\X)$,
  and $\pi_1(\X)$ is discrete.
  By quotient,
  or associated bundle,
  this gives again the natural {\em lifting of paths}, on every covering\index{Covering} of $\X$.
\end{article} %% The-natural-connection-on-a-covering

\begin{article}\artlabel{Connections and connection 1-forms on torus bundles}
  \addcontentsline{toc}{section}{\small\hspace{10pt} Connections and connection 1-forms on torus bundles}
  \label{Connections-and-connection-1-forms-on-torus-bundles}
  A {\em diffeological $1$-torus}\index{Diffeological $1$-torus} is any quotient $\Torus = \RR/\Gamma$ where $\Gamma \subset \RR$ is a strict subgroup,
  $\Gamma \neq \RR$.
  In this case $\Gamma$ is discrete and the projection $\class : \RR \to \Torus$ is the universal covering \art{The-universal-covering}.
  These tori are Abelian diffeological groups of dimension 1;
  see \art{Dimension-of-a-diffeological-space} and \exref{Dimension-of-tori}.
  %
  The differential $1$-form $\dt$ defined on $\RR$ is invariant by translation,
  {\em a fortiori\/} by the group $\Gamma$.
  Thus,
  according to \art{Coverings-and-differential-forms},
  there exists a unique $1$-form $\theta \in \DForms^{1}(\Torus)$,
  which is actually a volume form \art{Volumes-on-manifolds-and-diffeological-spaces},
  defined by
  $$%
  dt = \class^*(\theta), \text{ with } \class : \RR \to \Torus.
  $$%
  Now,
  let $\pi : \Y \to \X$ be a diffeological  $\Torus$-principal fiber bundle.
  Let $\tau_\Y$ denote the action of $\tau \in \Torus$ on $\Y$,
  and let $\orb{y} : \Torus \to \Y$ be the orbit map $\orb{y}(\tau) = \tau_\Y(y)$.
  
  \Definition
  \begin{em} We call {\em connection form}\index{Connection form} on the $\Torus$-principal bundle
  $\pi : \Y \to \X$ any differential $1$-form $\lambda$, defined on
  the total space $\Y$, such that
  \begin{itemize}
    \item[(a)] the form $\lambda$ is {\em invariant} ---
    for all $\tau \in \Torus$,
    $\tau_\Y^*(\lambda) = \lambda$,
    \item[(b)] the form $\lambda$ is {\em calibrated} ---
    for all $y \in \Y$, $\orb{y}^*(\lambda) = \theta$.
  \end{itemize}
  \end{em}
  
  1. First of all let us note a useful formula related to such connection forms.
  For any integers $n$ and $m$, any $n$-plot $\btau : \U \to \Torus$ and any $m$-plot $\P : \V \to \Y$,
  let $\btau \cdot \P$ be the plot of $\Y$ defined by
  $$%
  \btau \cdot \P : \U \times \V \to \Y, \text{ with } \btau \cdot \P : (r,s) \mapsto \btau(r)_\Y(\P(s)).
  $$%
  Let $(r,s) \in \U \times \V$,
  $\delta r \in \RR^n$,
  $\delta s \in \RR^m$,
  the value of $\lambda$ on $\btau \cdot \P$ is given by
  \begin{equation}
    \renewcommand{\theequation}{$\clubsuit$}
    \lambda(\btau \cdot \P)_{r \choose s} \vect{\delta r \\ \delta s} = \btau^*(\theta)_r(\delta r) + \lambda(\P)_s(\delta s).
  \end{equation}
  
  2. There exists a unique $2$-form $\omega \in \DForms^2(\X)$,
  called the {\em curvature}\index{Curvature} of the connection form $\lambda$,
  such that
  \begin{equation}
    \renewcommand{\theequation}{$\spadesuit$}
    \pi^*(\omega) = d\lambda, \text{ and } d\omega = 0.
  \end{equation}
  
  3. There exists a connection $\Theta$ associated with $\lambda$ such that the horizontal paths are the ones on which the connection form vanishes,
  \begin{equation}
    \renewcommand{\theequation}{$\diamondsuit$}
    \HorPaths_\loc(\Y,\Theta) = \{ \bar c \in \Paths_\loc(\Y) \mid \lambda(\bar c) = 0 \}.
  \end{equation}
  Note that $\lambda(\bar c)=0$ means that for all $t \in \Dom(\bar c)$,
  $\lambda(\bar c)_t(1) = 0$.
  The connection $\Theta$ is explicitly given by
  \begin{equation}
    \renewcommand{\theequation}{$\heartsuit$}
    \Theta(\tilde c,t) : t' \mapsto \class\bigg(-\int_{t}^{t'} \lambda(\tilde c)_s(1)\, ds\bigg)_\Y \bigg(\tilde c(t')\bigg).
  \end{equation}
  
  4. If the curvature $\omega$ defined above ($\spadesuit$) is zero,
  then the connection $\Theta$ is flat,
  that is,
  the holonomy \art{The-holonomy-of-a-connection} is discrete.
  
  \Note~If $\lambda$ is a connection form,
  then for any $1$-form $\alpha$ on $\X$,
  $\lambda' = \lambda + \pi^*(\alpha)$ is again a connection form.
  Conversely,
  if $\lambda$ and $\lambda'$ are two connection forms, then their difference $\lambda' - \lambda$ is the pullback of a $1$-form $\alpha$ on $\X$.
  The space of connection forms of a $\Torus$-principal bundle,
  if it is not empty,
  is an affine space of $\DForms^{1}(\X)$.
\end{article} %% Connections-and-connection-1-forms-on-torus-bundles

\begin{proof}
  1. Let us develop the $1$-form $\lambda(\btau \cdot \P)$ defined on $\U \times \V$,
  \begin{eqnarray*}
    \lambda(\btau \cdot \P)_{r \choose s} \vect{\delta r \\ \delta s} & = & \lambda[(r,s) \mapsto (\btau(r),\P(s)) \mapsto \tau(r)_\Y (\P(s))]_{r \choose s}\vect{\delta r \\ \delta s} \\
    & = & [\lambda_{\U,s}(r) \quad \lambda_{\V,r}(s)] \vect{\delta r \\ \delta s},
  \end{eqnarray*}
  because every $1$-form on $\U \times \V$ at a point $(r,s) \in \U \times \V$ writes $[\lambda_{\U,s}(r) \ \lambda_{\V,r}(s)]$,
  where $\lambda_{\U,s}$ is a $1$-form of $\U$ depending on $s$,
  and $\lambda_{\V,r}$ is a $1$-form of $\V$ depending on $r$.
  Let $y_s = \P(s)$,
  we have
  \begin{eqnarray*}
    \lambda(\btau \cdot \P)_{r \choose s} \vect{ \delta r \\ \delta s} & = & \lambda_{\U,s}(r) (\delta r) + \lambda_{\V,r}(s)(\delta s) \\
    & = & \lambda[r \mapsto \btau(r)_\Y(\P(s))]_r(\delta r) + \lambda[s \mapsto \btau(r)_\Y(\P(s))]_s(\delta s) \\
    & = & \lambda(\orb{y_s} \circ \btau)_r(\delta r) + \btau(r)_\Y^*(\lambda)(\P)_s(\delta s) \\
    & = & \btau^*[\orb{y_s}^*(\lambda)]_r(\delta r) + \lambda(\P)_s(\delta s) \\
    & = & \btau^*(\theta)_r(\delta r) + \lambda(\P)_s(\delta s).
  \end{eqnarray*}
  
  2. Let $\P : \U \to \Y$ and $\Q: \U \to \Y$ be two plots such that $\pi \circ \P = \pi \circ \Q$.
  Since $\pi$ is a principal fiber bundle with structural group $\T$,
  for every point $r_0 \in \U$ there exists at least locally,
  on some open neighborhood of $r_0$,
  a smooth parametrization $\btau$ of $\T$ such that $\Q(r) = \btau(r)_\Y(\P(r))$ \art{Principal-diffeological-fiber-bundles}.
  Applying proposition 1) to $\Q$,
  we get $\lambda(\Q) = \btau^*(\theta) + \lambda(\P)$,
  and since $\theta$ is a $1$-form on $\T$ which is $1$-dimensional (see \exref{Dimension-of-tori}),
  $d\theta = 0$ and $d\lambda(\Q) = d\lambda(\P)$.
  Then, thanks to \art{Pushing-forms-onto-quotients},
  there exists a $2$-form $\omega$ on $\X$ such that $\pi^*(\omega) = d\lambda$,
  then $\pi^*(d\omega) = d[d\lambda] = 0$,
  and since $\pi$ is a subduction,
  the $2$-form $\omega$ is closed \art{Vanishing-forms-on-quotients}.
  
  3. Let us consider the connection as a lifting of paths.
  Let $c$ be a local path in $\X$,
  let $x = c(t)$ for some $t \in \Dom(c)$ and $y \in \Y_x$.
  Then,
  let $\tilde c$ be a global lift of $c$,
  which exists always thanks to \art{Fibrations-and-local-triviality-along-the-plots}.
  We can even choose $\tilde c(t) = y$.
  Now,
  let $f \in \Cinfty(\Dom(c),\RR)$,
  and let us consider the new path $s \mapsto \class(f(s))_\Y(\tilde c(s))$.
  Using the point 1) and that $\class^*(\theta) = dt$ means $\theta[t \mapsto \class(t)] = dt$,
  what is equivalent to $\theta[r \mapsto \class(\F(r))] = \F^*(dt) = d\F$,
  for all smooth real parametrizations $\F$,
  we get
  \begin{eqnarray*}
    \lambda\bigg[ s \mapsto \class(f(s))_\Y\bigg(\tilde c(s)\bigg)\bigg]_s(1) & = & [s \mapsto \class(f(s))]^*(\theta)_s(1) + \lambda(\tilde c)_s(1) \\
    & = & {df(s) \over ds} + \lambda(\tilde c)_s(1).
  \end{eqnarray*}
  Thus,
  the local path
  $$%
  \bar c (t') = \class(f(t'))_\Y\bigg(\tilde c(t')\bigg), \text{ with } f(t') = - \int_t^{t'} \lambda(\tilde c)_s(1) \, ds,
  $$%
  satisfies $\bar c(t) = \tilde c(t) = y$ and $\lambda(\bar c)_s = 0$ for all $s$.
  Now,
  we define $\Theta(\tilde c, t)$ by ($\heartsuit$) and ($\diamondsuit$) follows.
  To check that $\Theta$ satisfies the six axioms of connections \art{Connections-on-principal-fiber-bundles},
  it suffices to check the unicity of horizontal lift with a given starting point,
  since the other axioms are satisfied by construction or thanks to the locality of the integration of real function.
  So,
  let $\bar c'$ be an another horizontal lift of $c$ starting at $y$ at time $t$,
  we have $\bar c'(s) = \class(g(s))_\Y(\bar c(s))$,
  where $g$ is a smooth real function defined on $\Dom(c)$.
  Applying the same computation to $\bar c'$,
  since $\lambda(\bar c')_s = \lambda(\bar c)_s = 0$ for all $s$,
  we get that $dg(s)/ds = 0$,
  thus $g$ is constant and $g(s) = g(t)$,
  but since $\bar c'(t) = \bar c(t) = y$,
  we get finally that $g(t) = 0$ and hence $g(s) = 0$ for all $s$.
  Therefore,
  the horizontal lift of $c$ starting at $y$,
  as defined above,
  is unique,
  and $\Theta$ defined by ($\heartsuit$) is a connection on $\pi : \Y \to \X$.
  
  4. Let $x \in \X$ and $y \in \Y_x = \pi^{-1}(x)$,
  the holonomy at the point $y$ is the subgroup $\Torus_y$ of the elements $\tau \in \Torus$ for which there exists a loop $\ell$ based at $x$ such that the horizontal lift $\bar\ell$ starting at $y$ ends at $\tau_\Y(y)$,
  that is,
  $\bar\ell(0) = y$ and $\bar\ell(1) = \tau_\Y(y)$.
  Then,
  the map $\ell \mapsto \bar\ell$,
  from $\Loops(\X,x)$ to $\Paths(\Y,y,\Y_x)$ is smooth,
  thus the map $\ell \mapsto \bar\ell(1)$ also is smooth,
  and also the map $\ell \mapsto \orb{y}^{-1}(\bar\ell(1))$,
  that is,
  the map $\ell \mapsto \tau$ such that $\tau_\Y(y) = \bar\ell(1)$.
  Thus,
  for any 1-plot $s \mapsto \ell_s$,
  $s \mapsto \bar\ell_s$ is a 1-plot of $\Paths(\Y,y,\Y_x)$ and $\btau : s \mapsto \tau_s$,
  such that $\bar\ell_s(1) = \btau(s)_\Y(y)$, is a 1-plot of $\Torus_y$.
  Now,
  since $\bar\ell_s$ is horizontal for all $s$,
  $\lambda(\bar\ell_s) = 0$ ($\diamondsuit$),
  thus
  $$%
  \int_{\bar\ell_s} \lambda = \int_0^1 \lambda(\bar\ell_s)_t(1) \, dt =0.
  $$%
  Then,
  the variation of this identity, using the formula \xart{Variation-of-the-integral-of-a-form-on-a-cube}{Note 2},
  gives for $\delta = \partial/\partial s$
  $$%
  \delta \int_{\bar\ell_s} \lambda = 0, \ \text{that is,} \ \int_0^1 d\lambda(\delta \bar\ell) + \limitbrackets{\lambda(\delta\bar\ell)}{0}{1} = 0.
  $$%
  But
  \begin{eqnarray*}
    \limitbrackets{\lambda(\delta\bar\ell)}{0}{1} & = & \lambda(\delta\bar\ell)(1) - \lambda(\delta\bar\ell)(0) \\
    & = &
    \lambda\left(\vect{s \\ t} \mapsto \bar\ell_s(t)\right)_{s \choose 1}\vect{1 \\ 0}
    - \lambda\left(\vect{s \\ t} \mapsto \bar\ell_s(t)\right)_{s \choose 0}\vect{1 \\ 0} \\
    & = &
    \lambda\left(s \mapsto \bar\ell_s(1)\right)_{s}(1)
    - \lambda\left(s \mapsto \bar\ell_s(0)\right)_{s}(1) \\
    & = &
    \lambda\left(s \mapsto \btau(s)_\Y(y)\right)_{s}(1)
    - \lambda\left(s \mapsto y \right)_{s}(1) \\
    & = &
    \lambda\left(s \mapsto \orb{y}(\btau(s)) \right)_{s}(1)
    - 0 \\
    & = & \orb{y}^*(\lambda)(\btau)_{s}(1)\\
    & = & \theta(\btau)_{s}(1).
  \end{eqnarray*}
  Then,
  since $d\lambda = \pi^*(\omega)$ and $\pi\circ\bar\ell_s = \ell_s$,
  the identity above writes
  $$%
  \int_0^1 \omega(\delta \ell) + \theta(\btau)_{s}(1) = 0.
  $$%
  It is then clear that if $\omega = 0$,
  then $\theta(\btau)_{s} = 0$,
  which means that $\btau$ is locally constant,
  that is,
  the holonomy $\Torus_y$ is discrete.
\end{proof}

\begin{article}\artlabel{The Kronecker flows as diffeological fibrations}%
  \addcontentsline{toc}{section}{\small\hspace{10pt} The Kronecker flows as diffeological fibrations}%
  \label{The-Kronecker-flows-as-diffeological-fibrations}~
  The Kronecker flow has been the first example of application of diffeological methods on geometrical constructions regarded by classical differential geometry as {\em singular} \cite{DI83}.
  %
  We consider the 2-dimensional torus $\Torus^2$.
  Equipped with the quotient diffeology,
  $\Torus = \RR/\ZZ$ is a manifold,
  as well as the square $\Torus^2 = \Torus \times \Torus$ for the product diffeology.
  It is also a group and a diffeological group (actually a Lie group) \art{Diffeological-groups}.
  %
  Then,
  we pick a $1$-parameter subgroup $\A = \{ \left[t,\alpha t\right] \in \Torus^2 \mid t \in \RR \}$,
  for some real number $\alpha$,
  where the brackets denote the class modulo $\ZZ^2$.
  The subgroup $\A$ is just the projection of the line $\Delta \subset \RR^2$ with equation $y = \alpha x$.
  The nature of the quotient $\Torus_\alpha = \Torus^2/\A$ depends on the rationality of $\alpha$:
  
  (a) If $\alpha \in \QQ$,
  then $\Torus_\alpha$ is a Lie group diffeomorphic to $\Torus$.
  
  (b) If $\alpha \in \RR - \QQ$,
  then $\Torus_\alpha$ is still a $1$-dimensional diffeological group but not anymore a manifold,
  we say that $\Torus_\alpha$ is an {\em irrational torus}\index{Irrational torus}.
  
  We have already seen many aspects of this irrational torus in \exref{Diffeomorphisms-between-irrational-tori};
  \exref{The-pierced-irrational-torus};
  \exref{The-irrational-solenoid};
  \exref{The-irrational-torus-is-not-a-manifold}.
  As a direct application of the theory of diffeological fiber bundles developed in this part,
  we get the following properties.
  
  1. Considering the homotopy of $\Torus_\alpha$,
  since the fiber $\RR$ of the fibration $\pi : \Torus^2 \to \Torus_\alpha$ is contractible,
  the homotopy sequence \art{Exact-homotopy-sequence-of-a-diffeological-fibration} tells us that its homotopy coincides with the homotopy of $\Torus^2$,
  that is,
  $\pi_0(\Torus_\alpha) = \{\Torus_\alpha\}$,
  $\pi_1(\Torus_\alpha) \simeq \ZZ^2$,
  and $\pi_k(\Torus_\alpha) = \{0\}$ for all $k > 1$.
  
  2. The 1-form $\Lambda = dy/\alpha$ defined on $\RR^2$ is invariant by $\ZZ^2$ and defines a $1$-form $\lambda$ on $\Torus^2$ \art{Coverings-and-differential-forms}.
  The $1$-form $\lambda$ satisfies the conditions to be a connection form for the $\RR$-principal fibration $\pi : \Torus^2 \to \Torus_\alpha$;
  it is invariant by the Kronecker flow and calibrated \art{Connections-and-connection-1-forms-on-torus-bundles} with $\Gamma = \{0\}$.
  Then,
  since $\dim(\Torus_\alpha) = 1$,
  the curvature of $\lambda$ vanishes and the connection defined by $\lambda$ is flat \xart{Connections-and-connection-1-forms-on-torus-bundles}{point 4}.
  Every path in $\Torus^2$ writes locally $t\mapsto \left[x(t),y(t)\right]$,
  the path is horizontal if and only if $d(y(t))/dt = 0$,
  that is,
  if and only if $y(t)$ is locally constant.
  Therefore,
  the holonomy bundle total space,
  with reference to the point $\left[0,0\right]$ \xart{The-holonomy-of-a-connection}{($\spadesuit$)},
  is the torus $\Torus \times \{0\} = \{ \left[x,0\right] \mid x \in \RR \} \subset \Torus^2$.
  Now,
  the holonomy is the subgroup $\H \subset \RR$ made of $t$ such that $t_{\Torus^2}\left[x,0\right] = \left[x + t, 0 + \alpha t\right] = \left[x',0\right]$,
  that is,
  if and only if $t = m/\alpha$, with $m \in \ZZ$,
  \ie,
  $\H \simeq \ZZ$.
  Applying \xart{The-holonomy-of-a-connection}{Note 1},
  we must find the torus $\Torus^2$ again by quotient of the space of pairs $(\left[x,0\right],t) \simeq (\left[x\right],t)$,
  where $\left[x\right] \in \Torus$ and $t \in \RR$,
  by the action of $\ZZ$ given by
  $$%
  m(\left[x\right],t) = \left(\left[x + {m \over \alpha} \right], t - {m \over \alpha} \right),
  $$%
  where the brackets denote the class modulo $\ZZ$.
  The quotient of the product of the holonomy fiber bundle by the structure group $\RR$,
  under the action of the holonomy,
  is then given by the projection $(\left[x\right],t) \mapsto (\left[x+t\right], \left[\alpha t\right])$,
  and we find the torus $\Torus^2$ again as claimed.
  
  \Note~To end this discussion,
  note that,
  as a group $\Torus_\alpha$ is far from being singular as it is usually considered.
  On the contrary there is nothing more homogeneous/regular than a group,
  and nothing smoother than a diffeological group,
  whether it is a Lie group or not.
  Diffeology does justice to this intuition.
\end{article} %% The-Kronecker-flows-as-diffeological-fibrations

\begin{article}\artlabel{$\RR$-bundles over irrational tori and small divisors}
  \addcontentsline{toc}{section}{\small\hspace{10pt} $\RR$-bundles over irrational tori and small divisors}
  \label{R-bundles-over-irrational-tori-and-small-divisors}
  In the classical category of differential manifolds,
  every fiber bundle with contractible fiber has a global smooth section;
  see for example \cite{Die70c}.
  Therefore,
  every principal bundle over a manifold,
  with structure group the additive group of real numbers $\RR$, is trivial.
  But this is not true anymore in diffeology,
  as we have seen with the example of the irrational torus,
  the fibration $\pi : \Torus^2 \to \Torus_\alpha$,
  is a nontrivial $\RR$-principal fibration.
  This remark suggests,
  for every diffeological space $\X$,
  a new class of invariant:
  the set $\Fb(\X,\RR)$ of (classes of) $\RR$-principal bundles with base $\X$.
  Such a bundle is also called a {\em flow}\index{Flow} over $\X$,
  since it is defined by a smooth action of $\RR$.
  This question has been raised first in \cite{Igl85} and \cite{Igl86},
  let us summarize its main aspects.
  
  First of all,
  the set $\Fb(\X,\RR)$ has a structure of an Abelian group.
  Let us describe the group operation.
  Let $\pi : \Y \to \X$ and $\pi' : \Y' \to \X$ be two $\RR$-principal bundles.
  Let $\pi \otimes \pi' : \Y \otimes \Y' \to \X$ be the fiber product defined by
  \begin{multline*}
    \Y \otimes \Y' = \pi^*(\Y') = \{ (y,y') \in \Y \times \Y' \mid \pi(y) = \pi'(y') \} \\
    \text{with }
    \pi \otimes \pi'(y,y') = \pi(y) = \pi'(y').
  \end{multline*}
  But $\pi \otimes \pi'$ is an $\RR^2$-principal bundle for the product of the action on each factor,
  $(t,t')_{\Y \otimes \Y'}(y,y') = (t_\Y(y), t'_{\Y'}(y'))$.
  Let us denote by $\overline \RR$ the antidiagonal,
  that is, the subgroup of pairs $(t,-t) \in \RR^2$.
  Now,
  the quotient $\pi'' : \Y'' \to \X$,
  where $\Y'' = (\Y \otimes \Y')/\overline \RR$ and $\pi''(y'') = \pi(y) = \pi'(y')$,
  with $y'' = \cl{y,y'}$,
  is again an $\RR$-principal bundle on $\X$,
  the brackets denoting the orbits of the action of $\overline \RR$.
  The map $(\pi,\pi') \mapsto \pi''$ passes to the equivalence classes of the $\RR$-principal bundles over $\X$.
  
  (a) Let $p = \class(\pi)$,
  $p' = \class(\pi')$,
  and $p'' = \class(\pi'')$.
  Then the operation $(p,p') \mapsto p'' = p + p'$ is an Abelian group operation on $\Fb(\X,\RR)$.
  
  (b) The identity element $0 \in \Fb(\X,\RR)$ is represented by the trivial bundle $\pr_1 : \X \times \RR \to \X$.
  
  (c) The inverse $-p$ of $p = \class(\pi)$, $\pi : \Y \to \X$,
  is the class of the principal bundle $\bar \pi : \overline \Y \to \X$,
  with same total space $\overline \Y = \Y$ and same projection,
  but with the inverse $\RR$-action,
  $t_{\overline \Y}(y) = (-t)_\Y(y)$.
  
  Let us exemplify this construction by making explicit the group $\Fb(\Torus,\RR)$ for $\Torus = \RR/\Gamma$,
  where $\Gamma$ is a strict subgroup of $\RR$.
  Let $\pi : \Y \to \Torus$ be a $\RR$-principal bundle,
  its pullback by the projection $\pr : \RR \to \Torus$ is trivial,
  there exists actually a smooth lift $\varphi : \RR \to \Y$ of $\pr$,
  that is,
  $\pi \circ \varphi = \pr$.
  The map $\Phi : (x,t) \mapsto (x, t_\Y(\varphi(x)))$ is an isomorphism from $\RR \times \RR$ to $\pr^*(\Y)$.
  Then,
  since $\pi$ and $\pr$ are two subductions,
  the second projection $\pr_2 : \pr^*(\Y) \to \Y$ is a subduction,
  and $\Y$ is equivalent to the quotient of $\RR \times \RR$ by the relation $(x,t) \sim (x',t')$,
  if and only if $\pr_2 \circ \Phi(x,t) = \pr_2 \circ \Phi(x',t')$,
  that is,
  $t_\Y(\varphi(x)) =  t'_\Y(\varphi(x'))$.
  This implies in particular that $x' = x + k$,
  for some $k \in \Gamma$,
  and there exists $\tau(k)(x) \in \RR$ such that $\varphi(x) = \tau(k)(x)_\Y(\varphi(x+k))$.
  We then get a map $\tau : \Gamma \mapsto \Cinfty(\RR)$ such that
  \begin{equation}
    \renewcommand{\theequation}{$\diamondsuit$}
    \tau(k+k')(x) = \tau(k)(x+k') + \tau(k')(x)
  \end{equation}
  and $(x',t')$ is equivalent to $(x,t)$ if and only if $x' = x + k$ and $t' = t + \tau(k)(x)$.
  Hence,
  the classes of the equivalence relation $\sim$ are the orbits of the group $\Gamma$ acting on the product $\RR \times \RR$ by
  $$%
  k : (x,t) \mapsto (x + k, t + \tau(k)(x)),
  $$%
  and $\Y$ is equivalent to $\RR \times_\Gamma \RR$,
  the quotient of $\RR \times \RR$ by this action of $\Gamma$.
  
  Conversely,
  for all $\tau$ satisfying ($\diamondsuit$),
  let us define $\Y$ as the quotient $\RR \times_\Gamma \RR$ and $\pi : \Y \to \Torus$ by $\pi(\cl{x,t}) = \cl{x}$,
  where the brackets denote the class under the action of $\Gamma$.
  We get a natural free action of $\RR$ on $\Y$ by $t'_\Y(\cl{x,t}) = \cl{x,t + t'}$.
  This projection $\pi$,
  together with this action of $\RR$,
  is a principal fibration.
  
  Next,
  $\tau$ defines a trivial principal bundle if and only if there is a global smooth section $\bsigma : \Torus \to \Y$.
  Then,
  $\pr \circ \bsigma : \RR \to \Y$ has a smooth lift $x \mapsto (x,\sigma(x))$ in $\RR \times \RR$ such that $\bsigma(\cl{x}) = \cl{x,\sigma(x)}$.
  This can be regarded as a consequence of the monodromy theorem \art{Monodromy-theorem},
  since the projection from $\RR \times \RR$ to its quotient $\Y$ is a covering,
  actually the universal covering.
  Now,
  for all $k \in \Gamma$,
  on the one hand $\bsigma(\cl{x}) = \bsigma(\cl{x + k})$ gives $\cl{x,\sigma(x)} = \cl{x + k,\sigma(x + k)}$,
  and on the other hand $\cl{x,\sigma(x)} = \cl{x + k,\sigma(x) + \tau(k)(x)}$.
  Thus,
  $\sigma(x + k) = \sigma(x) + \tau(k)(x)$,
  that is,
  $\tau(k)(x) = \sigma(x + k) - \sigma(x)$.
  So,
  two maps $\tau$ and $\tau'$ satisfying ($\diamondsuit$) define equivalent $\RR$-principal bundles if and only if there exists a map $\sigma \in \Cinfty(\RR)$ such that
  \begin{equation}
    \renewcommand{\theequation}{$\heartsuit$}
    \tau'(k)(x) = \tau(k)(x) + \sigma(x+k) - \sigma(x).
  \end{equation}
  Thus,
  the group $\Fb(\X,\RR)$ of classes of $\RR$-principal bundles over $\Torus$ is equivalent to the set of maps $\tau$ defined by ($\diamondsuit$) modulo the equivalence defined by ($\heartsuit$).
  This can be interpreted in terms of group cohomology.
  
  The map $\tau$ is actually a $1$-cocycle of $\Gamma$ with coefficients in $\Cinfty(\RR)$,
  where $\Gamma$ acts on $\Cinfty(\RR)$ by translations $(k,f) \mapsto \T_k^*(f) = f \circ \T_k$,
  with $(k,f) \in \Gamma \times \Cinfty(\RR)$ and $\T_k(x) = x + k$.
  Indeed,
  the identity ($\diamondsuit$) writes equivalently
  $$%
  \tau(k + k') = \T_{k'}^*(\tau(k)) + \tau(k').
  $$%
  Next,
  the cocycle $\tau$ defines a trivial bundle if $\tau(k)(x) = \sigma(x+k) - \sigma(x)$,
  that is,
  if $\tau$ is a coboundary,
  $\tau(k) = \T_k^*(\sigma) - \sigma$.
  Thus the group $\Fb(\X,\RR)$ is equivalent to the first cohomology group
  $$%
  \Fb(\X,\RR) \simeq \HG^1(\Gamma, \Cinfty(\RR)).
  $$%
  Now,
  let us come to the particular example of $\Gamma = \ZZ + \alpha \ZZ$,
  where $\alpha \in \RR - \QQ$,
  $\RR/\Gamma = \Torus_\alpha$.
  The space $\Torus_\alpha$ is equivalent to the quotient $\Torus/\ZZ$,
  where $\Torus = \RR/\ZZ$,
  with $n(x) = x + n$, $x \in \RR$ and $n \in \ZZ$,
  and then $\ZZ$ acts on $\Torus$ by $n \cl{x} = \cl{ x + n \alpha}$,
  where $\cl{x} \in \Torus$ denotes the class of $x$ modulo $\ZZ$.
  Let us denote by $\pr : \Torus \to \Torus_\alpha$ the projection,
  and let $\pi : \Y \to \Torus_\alpha$ be an $\RR$-principal fiber bundle.
  The pullback $\pr_1 : \pr^*(\Y) \to \Torus$ is an $\RR$-principal fiber bundle over a manifold $\Torus \simeq \S^1$,
  thus it is trivial $\pr^*(\Y) \simeq \Torus \times \RR$.
  And the previous reconstruction of $\Y$,
  for $\Gamma$ acting on $\RR \times \RR$,
  can be mimicked in this situation with $\ZZ$ acting on $\Torus \times \RR$ through a cocycle $\tau : \ZZ \to \Cinfty(\Torus,\RR)$ satisfying
  $$%
  \tau(n + n')(\cl{x}) = \tau(n)(\cl{x + n'\alpha}) + \tau(n')(\cl{x}).
  $$%
  Let $f = \tau(1)$,
  then the cocycle $\tau$ satisfies
  $$%
  \tau(n)(\cl{x}) = \sum_{k=0}^{n-1} f(\cl{x+k\alpha}), \text{ and } \tau(-n)(\cl{x}) = - \sum_{k=0}^{n-1} f(\cl{x+(k-n)\alpha}),
  $$%
  for all integers $n > 0$ and $\tau(0) = 0$.
  Then,
  $f$ defines a coboundary if and only if there exists a function $g \in \Cinfty(\Torus,\RR)$ such that
  $$%
  f(\cl{x}) = g(\cl{x+\alpha}) - g(\cl{x}), \text{ or } \F(x) = \G(x+\alpha) - \G(x),
  $$%
  where $\F$ and $\G$  are two $1$-periodic smooth real functions on $\RR$,
  representing respectively $f$ and $g$.
  This equation in $\G$ is known as the {\em cohomological equation}
  and has been studied by many authors,
  beginning by \cite{Arn65}, \cite{Arn80}, \cite{Mos66}, \cite{Her79}.
  The space $\Fb(\Torus_\alpha,\RR)$ we are looking for,
  $\Cinfty(\Torus,\RR)$ modulo coboundaries,
  has a factor $\RR$ given by the invariant $\nu = \int_0^1\F(x) dx$.
  The second factor depends radically on the arithmetic nature of $\alpha$.
  In particular, if $\alpha$ is a diophantine,
  that is,
  if there exist an integer $k>0$ and a real number $c >0$ such that
  $$%
  \left|n\alpha-m\right| \geq {c \over n^k} \text{ for all } n \in \NN-\{0\},\ m \in \NN,
  $$%
  then the equation $\F(x) = \G(x+\alpha) - \G(x)$,
  where $\F$ is normalized by $\int_0^1\F(x)dx = 0$,
  has a (unique) solution $\G$.
  Thus,
  for these diophantine numbers,
  the space
  $\Fb(\Torus_\alpha,\RR)$ is equivalent to $\RR$,
  every element is then characterized by its invariant $\nu$.
  Geometrically,
  every $\RR$-principal bundle over the irrational torus $\Torus_\alpha$,
  with $\alpha$ satisfying the diophantine condition above,
  is either trivial or isomorphic to the 2-torus $\Torus^2$ equipped with the $\alpha$-Kronecker flow,
  but for different $\RR$-action speeds.
  But this is not the case if $\alpha$ is not diophantine,
  since there exist functions $\F$ for which the cohomological equation above has no solution.
  So,
  the diffeology of the irrational torus captures not only the equivalence by $\GL(2,\ZZ)$ or the possible quadratic character of $\alpha$,
  as we have seen before, but also its fine arithmetic.
  
  \Note{1} The space $\HG^1(\Gamma, \Cinfty(\RR))$ is actually a real vector space.
  In particular,
  for every cocycle $\tau$ and real number $s$,
  $s\cdot\tau$ is again a cocycle.
  Geometrically speaking,
  if $s \neq 0$,
  then $s\cdot\tau$ represents the same total space with a $\RR$-action dilated by $s$.
  
  \Note{2} Every $\RR$-principal bundle $\pi : \Y \to \Torus$ over a torus $\Torus = \RR / \Gamma$,
  defined by a cocycle $\tau$,
  can be naturally equipped
  with a connection associated with the connection of the covering $\pr : \RR \to \Torus$ \art{The-natural-connection-on-a-covering}.
  However,
  not all these bundles support a connection form $\lambda$ \art{Connections-and-connection-1-forms-on-torus-bundles},
  only those whose cocycle $\tau$ defining $\pi$ is equivalent to a homomorphism from $\Gamma$ to $\RR$;
  see \exref{Connection-forms-over-tori}.
  In other words,
  if the cocycle $\tau$ is not cohomologous to a homomorphism,
  then there is no connection which can be defined by a connection form.
  In particular,
  for $\Gamma = \ZZ + \alpha \ZZ$,
  the only $\RR$-principal bundles equipped with connections defined by a connection form are the Kronecker flows on $\Torus^2$.
\end{article} %% R-bundles-over-irrational-tori-and-small-divisors

\begin{proof}
  The above construction of $\pi : \Y = \RR \times_\Gamma \RR \to \Torus$,
  together with its action of $\RR$,
  is not exactly an associated bundle to the covering $\pr : \RR \to \Torus$ as described in \art{Associated-fiber-bundles}.
  Let us prove directly that it is indeed an $\RR$-principal bundle.
  Since $\pr$ is a generating function of the quotient diffeology,
  it is sufficient to check that the pullback of $\pi$ by $\pr$ is a (trivial) $\RR$-principal bundle,
  or,
  which is equivalent,
  that the map $\Phi : \RR \times \RR \to \pr^*(\Y)$ defined by $\Phi(x,t) = (x,\cl{x,t})$ is a (clearly equivariant) diffeomorphism.
  The map $\Phi$ is obviously smooth and injective.
  It is also surjective.
  Let $(x,\cl{x',t'}) \in \pr^*(\Y)$,
  then there exists $k \in \Gamma$ such that $x' = x +k$,
  thus $\cl{x',t'} = \cl{x, t = t' - \tau(k)(x)}$ and $(x,\cl{x',t'}) = \Phi(x,t)$.
  Let us check that $\Phi^{-1}$ is smooth.
  A plot of $\pr^*(\Y)$ writes locally $r \mapsto (x(r), \cl{x'(r),t'(r)})$,
  with $\cl{x(r)} = \cl{x'(r)}$.
  Thus,
  there exists a plot $r \mapsto k(r)$ of $\Gamma$ such that $x'(r) = x(r) + k(r)$,
  but since $\Gamma$ is discrete,
  $k(r)$ is locally constant,
  that is,
  $k(r)=_\loc k$.
  Hence,
  $r \mapsto t(r) = t'(r) - \tau(k)(x(r))$ is a plot of $\RR$ and $\Phi(x(r),t(r)) = (x(r), \cl{x'(r),t'(r)})$.
  Therefore,
  $\Phi^{-1}$ is smooth and $\Phi$ is a diffeomorphism.
  
  Now,
  let us check that the projection $(x,t) \mapsto \cl{x,t}$ is a covering as claimed.
  Let $\P : r \mapsto y_r$ be a plot of $\Y$,
  since $(x,t) \mapsto \cl{x,t}$ is a subduction the plot $\P$ has a local lift $r \mapsto (x_r,t_r)$.
  Then,
  $\Psi : (r,k) \mapsto (r,(x_r + k, t_r + \tau(k)(x_r)))$ is a smooth bijection from $\Dom(\P) \times \Gamma$ to $\P^*(\RR \times \RR) = \{(r,(x,t)) \in \Dom(\P) \times \RR \times \RR \mid \P(r) = \cl{x,t} \}$.
  Its inverse is given by $\Psi^{-1}(r,(x,t)) = (r,x-x_r)$ which is clearly smooth.
  Thus,
  $(x,t) \mapsto \cl{x,t}$ is a covering with group $\Gamma$,
  and since $\RR \times \RR$ is simply connected,
  it is the universal covering.
\end{proof}

%%%%%%%%%%%%%%%%%%%%%%%%%%%%%%%%%%%%%%%%%%%%%%%%%%%%%%%%%%
%
%   Exercises
%
%%%%%%%%%%%%%%%%%%%%%%%%%%%%%%%%%%%%%%%%%%%%%%%%%%%%%%%%%%

\Exercises

\begin{exercise}[Spheric periods on toric bundles]
  \label{Spheric-periods-on-toric-bundles}
  Let $\pi : \Y \to \X$ be a principal fiber bundle with structure group a torus $\Torus = \RR/\Gamma$ \art{Connections-and-connection-1-forms-on-torus-bundles}.
  Let $x \in \X$ and $\sigma$ be any homotopy of the constant loop $\bmx : t \mapsto x$ to itself,
  that is,
  $\sigma \in \Loops(\Loops(\X,x), \bmx)$.
  Show that $\int_{\sigma} \omega \in \Gamma$.
\end{exercise} %% Spheric-periods-on-toric-bundles

\begin{exercise}[Fiber bundles over tori]
  \label{Fiber-bundles-over-tori}
  Let $\Torus_\Gamma = \RR^n/\Gamma$ be a torus as defined in \exref{Irrational-tori-are-orientable}.
  Show that every fiber bundle over $\Torus_\Gamma$ is ``associated'' with a covering,
  as in \art{R-bundles-over-irrational-tori-and-small-divisors}.
\end{exercise} %% Fiber-bundles-over-tori

\begin{exercise}[Flat connections on toric bundles]
  \label{Flat-connections-on-toric-bundles}
  Show the converse of
  \linebreak
  \xart{Connections-and-connection-1-forms-on-torus-bundles}{point 4},
  that is,
  if a connection on a principal bundle,
  with structure group a 1-dimensional torus,
  is flat,
  then the curvature of the connection vanishes.
\end{exercise} %% Flat-connections-on-toric-bundles

\begin{exercise}[Connection forms over tori]
  \label{Connection-forms-over-tori}
  Let $\tau$ be a cocycle as defined in \xart{R-bundles-over-irrational-tori-and-small-divisors}{($\diamondsuit$)}.
  Show that there exists a connection $1$-form on the $\RR$-principal fiber bundle $\pi : \Y \to \Torus$, with $\Torus = \RR/\Gamma$ and $\Y = \RR \times_\Gamma \RR$,
  if and only if $\tau$ is equivalent to a homomorphism from $\Gamma$ to $\RR$.
  Make explicit the subspace of $\Fb(\Torus_\alpha,\RR)$ made of classes of such bundles.
\end{exercise} %% Connection-forms-over-tori

%%%%%%%%%%%%%%%%%%%%%%%%%%%%%%%%%%%%%%%%%%%%%%%%%%%%%%%%%%
%% MARK: Integration Bundles of Closed 2-Forms
%%%%%%%%%%%%%%%%%%%%%%%%%%%%%%%%%%%%%%%%%%%%%%%%%%%%%%%%%%

\section*{Integration Bundles of Closed 2-Forms}
\label{sec-Integration-bundles-of-closed-2-forms}

\begin{sechead}
  Integral closed $2$-forms play an important role in theory of fiber bundles,
  because they are used to classify complex line bundles \cite{Mil74}.
  They are also crucial in symplectic mechanics because of geometric quantization;
  see for example \cite{Sou70}, \cite{Bry93}, \cite{BatWei97}.
  We shall see in this section a universal construction,
  associating every closed $2$-form with some {\em integration fiber bundle},
  with structure group the {\em torus of periods} of the $2$-form,
  whether the $2$-form is integral or not.
  The $2$-form will be the curvature of a connection $1$-form on this bundle.
  This construction generalizes for diffeological spaces a previous construction,
  on manifolds, described in \cite{Igl95}.
  We only address in this section the case of simply connected diffeological spaces,
  and show that every closed $2$-form  is the curvature of a connection.
  This is the basis of the general case which will be the subject of a separate publication.
\end{sechead}

\begin{article}\artlabel{Torus bundles over diffeological spaces}
  \addcontentsline{toc}{section}{\small\hspace{10pt} Torus bundles over diffeological spaces}
  \label{Torus-bundles-over-diffeological-spaces}
  Let $\X$ be a connected diffeological space,
  $\pi_0(\X) = \{\X\}$.
  Let $\Torus = \RR/\Gamma$,
  where $\Gamma$ is a strict subgroup of $\RR$,
  be a 1-dimensional diffeological torus.
  We denote by $\theta$ the canonical $1$-form defined on $\Torus$ by pushing forward the canonical $1$-form $dt$ of $\RR$.
  Let $\pi : \Y \to \X$ be a $\Torus$-principal bundle,
  and let $\lambda$ be a connection form on $\Y$ with curvature $\omega \in \DForms^2(\X)$ \art{Connections-and-connection-1-forms-on-torus-bundles}.
  The action of $\Torus$ on $\Y$ is denoted as usual by $\tau_\Y(y)$ for $\tau \in \Torus$ and $y \in \Y$.
  Let $x \in \X$ be a basepoint in $\X$,
  and let $\Paths(\X,x)$ be the space of paths in $\X$ based at $x$.
  Let us consider the $\Torus$-principal bundle,
  pullback of $\pi$ by the target map $\1 : \Paths(\X,x) \to \X$,
  with total space
  $$%
  \1^*(\Y) = \{(\gamma,y) \in \Paths(\X) \times \Y \mid \gamma(0) = x, \text{ and } \pi(y) = \gamma(1) \}.
  $$%
  Since $\lambda$ is a connection form on $\Y$,
  $\pr_1 : \1^*(\Y) \to \Paths(\X,x)$ can be equipped by pullback with a connection,
  and since $\Paths(\X,x)$ is contractible (\exref{Contractible-space-of-paths}),
  $\pr_1 : \1^*(\Y) \to \Paths(\X,x)$ is a trivial $\Torus$-principal bundle \art{Connections-and-equivalence-of-pullbacks}.
  
  1. There exists an equivariant diffeomorphism
  $$%
  \Psi : \Paths(\X,x) \times \Torus \to \1^*(\Y) \text{ such that }
  (\pr_2 \circ \Psi)^*(\lambda) = \CHK\!\omega \oplus \theta,
  $$%
  where $\CHK$ is the Chain-Homotopy operator \art{The-Chain-Homotopy-operator-K}.
  Note that $\CHK\!\omega \oplus \theta$ is a connection form on the trivial $\Torus$-principal fiber bundle $\pr_1 : \Paths(\X,x) \times \Torus \to \Paths(\X,x)$,
  with curvature $\1^*(\omega)$.
  
  2. The equivariant diffeomorphism $\Psi$ writes $\Psi(\gamma,\tau) = (\gamma, \tau_\Y(\psi(\gamma)))$,
  where $\psi \in \Cinfty(\Paths(\X,x), \Y)$ is a lift of $\1$ along $\pi$,
  that is,
  $\pi \circ \phi = \1$.
  The projection $\pr_2 \circ \Psi$ is a subduction.
  This means that the total space $\Y$ of the $\Torus$-principal bundle $\pi$ can be regarded as a quotient of the product $\Paths(\X,x) \times \Torus$,
  onto which the connection form $\CHK\!\omega + \theta$ descends.
  \begin{center}
    \begin{tikzcd}[column sep=large, row sep=large, every label/.append style = {font = \small}]
      \Paths(\X,x) \times \Torus \arrow[rr,"\Psi"] \arrow[dr, swap, "\pr_1"] & {} & \1^*(\Y) \arrow[r,"\pr_2"] \arrow[dl, "\pr_1"] & \Y \arrow[d,"\pi"]  \\
      {} & \Paths(\X,x) \arrow[rr, swap, "\1"] & {} & \X
    \end{tikzcd}
  \end{center}

  3. There exists a smooth map $\Phi$ defined on the tautological pullback
  $$%
  \1^*(\Paths(\X,x)) = \{ (\gamma,\gamma') \in \Paths(\X,x)^2 \mid \gamma(1) = \gamma'(1) \}
  $$%
  with values in $\Torus$,
  $\Phi \in \Cinfty(\1^*(\Paths(\X,x)),\Torus)$,
  satisfying the cocycle condition
  \begin{equation}
    \renewcommand{\theequation}{$\diamondsuit$}
    \Phi(\gamma,\gamma') + \Phi(\gamma',\gamma'') = \Phi(\gamma,\gamma''),
  \end{equation}
  for every couple of pairs of paths $(\gamma,\gamma')$  and $(\gamma',\gamma'')$ in $\1^*(\Paths(\X,x))$,
  and such that
  \begin{equation}
    \renewcommand{\theequation}{$\heartsuit$}
    \psi(\gamma) = \psi(\gamma'), \text{ if and only if } \Phi(\gamma,\gamma') = 0.
  \end{equation}
  Note that $\psi$ takes its values in $\Y$ but $\Phi$ takes its values in $\Torus$.
  Then,
  the equivalence relation $\phi$ identifying $\Y$ as a quotient of $\Paths(\X,x) \times \Torus$ writes
  \begin{equation}
    \renewcommand{\theequation}{$\clubsuit$}
    (\gamma,\tau) \mathrel{\phi} (\gamma',\tau'),  \text{ if }  \1(\gamma) = \1(\gamma'),  \text{ and }  \tau' - \tau = \Phi(\gamma,\gamma'),
  \end{equation}
  with $(\gamma,\tau)$ and $(\gamma',\tau')$ in $\Paths(\X,x) \times \Torus$.
  The function $\Phi$ above ($\diamondsuit$) will be called a {\em paths $1$-cocycle}\index{Paths $1$-cocycle}.
  It characterizes the class of the $\Torus$-principal bundle $\pi : \Y \to \X$,
  together with its connection $\lambda$.
  Modulo coboundaries
  $\Delta\F(\gamma,\gamma') = \F(\gamma') - \F(\gamma)$,
  the cocycle $\Phi$ characterizes the class of the $\Torus$-principal bundle $\pi : \Y \to \X$ only.
  %
  \begin{center}
    \begin{tikzcd}[column sep=large, row sep=large, every label/.append style = {font = \small}]
      \Paths(\X,x)\times \Torus \arrow[r,"\pr_2 \circ \Psi"] \arrow[d, swap, "\pr_1"] & \Y \simeq \Paths(\X,x) \times_\phi \Torus \arrow[d, "\pi"]  \\
      \Paths(\X,x) \arrow[r, swap, "\1"] & \X
    \end{tikzcd}
  \end{center}
  %
  \Note~The map $\Phi$ above is the restriction to $\1^*(\Paths(\X,x))$ of a smooth function from $\ends^*(\Paths(\X)) = \{ (\gamma,\gamma') \in \Paths(\X)^2 \mid \ends(\gamma) = \ends(\gamma') \}$ to $\Torus$,
  satisfying the same cocycle property ($\diamondsuit$).
  The cohomology group defined by these paths cocycles,
  modulo coboundaries,
  may be denoted by $\HG^1(\Paths(\X),\Torus)$.
  Thus,
  every $\Torus$-principal bundle,
  which can be equipped with a connection $1$-form,
  defines a unique class in $\HG^1(\Paths(\X),\Torus)$ which can be regarded as its {\em characteristic class}\index{Characteristic class}.
\end{article} %% Torus-bundles-over-diffeological-spaces

\begin{proof}
  As it is claimed,
  the pullback of $\lambda$ by $\pr_2$ on $\1^*(\Y)$ is a connection form on a $\Torus$-principal bundle over a contractible base,
  thus the fiber bundle is trivial \art{Connections-and-equivalence-of-pullbacks}.
  Let $\Psi : \Paths(\X,x) \to \1^*(\Y)$ be a trivialization of $\Torus$-principal bundle,
  that is,
  an equivariant diffeomorphism.
  Thus,
  $\Psi(\gamma,\tau) = (\gamma, \tau_\Y(\psi(\gamma)))$,
  where $\psi : \Paths(\X,x) \to \Y$ is a lift of $\1$ along $\pi$,
  $\pi \circ \psi = \1$.
  Then,
  the pullback of $\lambda$ by $\pr_2 \circ \Psi$ is a connection form of the trivial bundle $\pr_1 : \Paths(\X,x) \times \Torus$,
  with curvature $\1^*(\omega) = d[\CHK\!\omega]$.
  But the $1$-form $\CHK\!\omega \oplus \theta$ also is a connection form on $\Paths(\X,x) \times \Torus$ with curvature $\1^*(\omega)$,
  thus the difference of these two connections is the pullback of a closed $1$-form of $\Paths(\X,x)$ \xart{Connections-and-connection-1-forms-on-torus-bundles}{Note},
  but since $\Paths(\X,x)$ is contractible,
  this closed $1$-form is exact.
  Therefore,
  there exists a smooth function $f : \Paths(\X,x) \to \RR$ such that
  $$%
  (\pr_2 \circ \Psi)^*(\lambda) = (\CHK\!\omega + df) \oplus \theta, \text{ and } \psi^*(\lambda) = \CHK\!\omega + df.
  $$%
  Now,
  let us denote by $\cl{t} \in \Torus = \RR/\Gamma$ the class of $t \in \RR$,
  and let us define
  $$%
  \psi'(\gamma) = \cl{-f(\gamma)}_\Y(\psi(\gamma)).
  $$%
  The map $\psi' : \Paths(\X,x) \to \Y$ is smooth and satisfies
  $$%
  {\psi'}^*(\lambda) = \CHK\!\omega.
  $$%
  Indeed,
  let $\P : r \mapsto \gamma_r$ be a plot of $\Paths(\X,x)$ and $\btau : r \mapsto \cl{-f(\gamma_r)}$.
  Thus,
  according to \xart{Connections-and-connection-1-forms-on-torus-bundles}{($\clubsuit$)},
  and since $\btau^*(\theta) = - df$,
  we get
  \begin{eqnarray*}
    {\psi'}^*(\lambda)(\P) & = & \lambda( \psi' \circ \P) \\
    & = & \lambda\left(r \mapsto \cl{-f(\gamma_r)}_\Y(\psi(\gamma_r))\right) \\
    & = & \lambda\left(r \mapsto \btau(r)_\Y(\psi \circ \P(r))\right)\\
    & = & \btau^*(\theta) + \lambda(\psi \circ \P) \\
    & = & \btau^*(\theta) + \psi^*(\lambda)(\P) \\
    & = & -df + (\CHK\!\omega + df) \\
    & = & \CHK\!\omega.
  \end{eqnarray*}
  Hence,
  $\Psi'(\gamma, \tau) = (\gamma, \tau_\Y(\psi'(\gamma)))$ is a trivialization of the principal bundle%
  \linebreak
  $\pr_1 : \1^*(\Y) \to \Paths(\X,x)$ such that $(\pr_2 \circ \Psi')^*(\lambda) = \CHK\!\omega \oplus \theta$.
  Therefore,
  we can rename $\psi'$ by $\psi$,
  $\Psi'$ by $\Psi$,
  and we get what is claimed.
  
  Next,
  let $r \mapsto y_r$ be a plot of $\Y$ and $x_r = \pi(y_r)$.
  Since $\pi$ is a subduction,
  there exists locally $r \mapsto \gamma_r$,
  a plot such that $\gamma_r(1) = x_r$ \xart{Pathwise-connectedness}{($\heartsuit$)}.
  Since $\pi(y_r) = \pi \circ \psi(\gamma_r)$ and since $\pi$ is a diffeological bundle,
  that is,
  locally trivial along the plots,
  there exists a plot $r \mapsto \tau_r$ of $\Torus$ such that,
  locally,
  $y_r = \tau_r(\psi(\gamma_r))= \pr_2 \circ \Psi(\gamma_r,\tau_r)$.
  That proves also that $\pr_2 \circ \Psi$ is surjective,
  thus $\pr_2 \circ \Psi$ is a subduction.
  
  Now,
  let $(\gamma,\tau)$ and $(\gamma',\tau')$ in $\Paths(\X,x) \times \Torus$ projecting on the same point by $\pr_2 \circ \Psi$,
  that is,
  $\tau_\Y(\psi(\gamma)) = \tau'_\Y(\psi(\gamma'))$.
  This implies,
  first of all,
  that $\pi(\psi(\gamma)) = \pi(\psi(\gamma'))$,
  there exists then a unique $\Phi(\gamma,\gamma')\! \in\! \Torus$ such that $\psi(\gamma)\! =\! \Phi(\gamma,\gamma')_\Y(\psi(\gamma'))$,
  and we get $\tau' = \tau + \Phi(\gamma,\gamma')$.
  Therefore,
  $\Y$ is equivalent to the quotient defined by the relation ($\clubsuit$).
  Let us check that $\Phi$ is smooth.
  Let $r \mapsto (\gamma_r,\gamma'_r)$ be a plot of $\1^*(\Paths(\X,x))$.
  Then,
  $r \mapsto x_r = \pi(\psi(\gamma_r))= \pi(\psi(\gamma'_r))$ is a plot of $\X$,
  but the pullback of $\Y$ by this plot is locally trivial,
  thus there exists locally everywhere a plot $r \mapsto \tau_r$ of $\Torus$ such that $\psi(\gamma_r)= {\tau_r}_\Y(\psi(\gamma'_r))$.
  Hence,
  locally,
  $\Phi(\gamma_r,\gamma'_r) = \tau_r$,
  and thus $\Phi$ is smooth.
  
%  To complete the equivalence,\n
%  we have to check that $\pr : \Paths(\X,x)\times_\phi \Torus \to \X$ is a fiber bundle.\n
%  That is,\n
%  its pullback by a plot of $\X$ is locally trivial.\n
%  %Note that $\pr : \class(\gamma,\tau) \mapsto \gamma(1)$ is a subduction because $\1 : \gamma \mapsto \gamma(1)$ is a subduction.\n
%  Let $\P : r \mapsto x_r$ be a plot,\n
%  there exists a smooth lifting $r \mapsto \gamma_r$,\n
%  defined on some domain $\V$,\n
%  such that $\gamma_r(1) = x_r$.\n
%  Then,\n
%  let $\Psi_\V : (r,\tau) \mapsto (r, \class(\gamma_r,\tau))$ from $\V \times \Torus$ to $(\P \restriction \V)^*(\Paths(\X,x)\times_\phi \Torus)$,\n
%  $\Psi_\V$ is smooth and injective.\n
%  Now,\n
%  Let $r \mapsto (r, y_r)$ be a plot in $(\P \restriction \V)^*(\Paths(\X,x)\times_\phi \Torus)$,\n
%  then there exists locally a smooth plot $r \mapsto (\gamma'_r, \tau'_r)$ such that $y_r = \class(\gamma'_r, \tau'_r)$.\n
%  Let $\Psi_\V(r,\tau) = (r, \class(\gamma'_r, \tau'_r))$,\n
%  then $\class(\gamma_r,\tau) =  \class(\gamma'_r, \tau'_r)$,\n
%  that is,\n
%  $\tau'_r - \tau = \Phi(\gamma'_r,\gamma_r)$.\n
%  Thus $\tau = \tau'_r - \Phi(\gamma'_r,\gamma_r)$,\n
%  $\Psi_\V^{-1}$ is smooth and $\pr$ is a fiber bundle,\n
%  equivalent to $\pi$.\n
  
  Let us assume now that $\Psi$ and $\Psi'$ are two isomorphisms from $\Paths(\X,x) \times \Torus$ to $\1^*(\Y)$.
  They define an automorphism of $\pr_1 : \Paths(\X,x) \times \Torus \to \Paths(\X,x)$,
  that is,
  a map $\eF : (\gamma,\tau) \mapsto (\gamma, \tau + \F(\gamma))$ such that $\Psi = \Psi' \circ \eF$.
  Then,
  from $\Psi(\gamma,\tau) = \Psi'(\gamma,\tau + \F(\gamma))$,
  and the definition of $\Phi$ and $\Phi'$,
  we get $\Phi' = \Phi + \F(\gamma') - \F(\gamma)$.
  Therefore,
  the class of $\Psi$,
  modulo the coboundaries $\Delta\F(\gamma,\gamma') = \F(\gamma') - \F(\gamma)$,
  characterizes the class of the $\Torus$-principal bundle $\pi$.
  Now,
  if we also consider the connection form $\lambda$,
  the map $\eF$ must preserve $\CHK\!\omega \oplus \theta$,
  and that implies $\F^*(\theta) = 0$,
  that is,
  $\F(\gamma)$ is constant.
  In this case $\Phi'= \Phi$,
  and the cocycle $\Phi$ itself characterizes the class of the $\Torus$-principal bundle $\pi$ together with the connection form $\lambda$.
\end{proof}

\begin{article}\artlabel{Integrating concatenations over homotopies}
  \addcontentsline{toc}{section}{\small\hspace{10pt} Integrating concatenations over homotopies}
  \label{Integrating-concatenations-over-homotopies}
  Let $\X$ be a connected diffeological space,
  and let $\omega$ be a closed $2$-form on $\X$.
  All paths in the following are assumed to be stationary \art{Stationary-paths}.
  Let $\gamma_0$ and $\gamma'_0$ be two paths such that $\gamma_0(1) = \gamma'_0(0)$.
  Then let
  $$%
  \sigma : s \mapsto \gamma_s, \ \sigma' : s \mapsto \gamma'_s, \text{ and }, \sigma*\sigma' : s \mapsto \gamma_s \vee \gamma'_s,
  $$%
  where $\sigma$ is a homotopy from $\gamma_0$ to $\gamma_1$ and $\sigma'$ a homotopy from $\gamma'_0$ to $\gamma'_1$,
  such that $\gamma_s(1) = \gamma'_s(0)$ for all $s$.
  The homotopy $\sigma*\sigma'$ is the resultant homotopy from $\gamma_0 \vee \gamma'_0$ to $\gamma_1 \vee \gamma'_1$.
  Now,
  let $\CHK$ be the Chain-Homotopy operator \art{The-Chain-Homotopy-operator-K},
  then
  \begin{equation}
    \renewcommand{\theequation}{$\clubsuit$}
    \int_{\sigma * \sigma'} \CHK\!\omega = \int_\sigma \CHK\!\omega + \int_{\sigma'} \CHK\!\omega.
  \end{equation}
\end{article} %% Integrating-concatenations-over-homotopies

\begin{proof}
  By definition \art{Integrating-forms-on-chains},
  $\int_{\sigma * \sigma'} \CHK\!\omega = \int_0^1 \CHK\!\omega(\sigma * \sigma')_t(1) \ dt$.
  Let us show the additivity
  \begin{equation}
    \renewcommand{\theequation}{$\star$}
    \CHK\!\omega(\sigma * \sigma')_t(1) = \CHK\!\omega(\sigma)_t(1) + \CHK\!\omega(\sigma')_t(1),
  \end{equation}
  which will give the identity ($\clubsuit$).
  From the definition \xart{The-Chain-Homotopy-operator-K}{($\diamondsuit$)},
  we have
  \begin{eqnarray*}
    \CHK\!\omega(\sigma * \sigma')_t(1)
    & =& \int_0^1 \omega \left(\vect{s \\ t} \mapsto (\sigma * \sigma')(t) (s) \right)_{{s \choose t}}\vect{1 \\ 0} \vect{0 \\ 1} \ ds \\
    & =& \int_0^1 \omega \left(\vect{s \\ t} \mapsto [\gamma_t \vee \gamma'_t] (s) \right)_{{s \choose t}}\vect{1 \\ 0} \vect{0 \\ 1} \ ds \\
    & =& \int_0^{1/2} \omega \left(\vect{s \\ t} \mapsto \gamma_t(2s) \right)_{{s \choose t}}\vect{1 \\ 0} \vect{0 \\ 1} \ ds \\
    & + & \int_{1/2}^1 \omega \left(\vect{s \\ t} \mapsto \gamma'_t(2s-1) \right)_{{s \choose t}}\vect{1 \\ 0} \vect{0 \\ 1} \ ds,
  \end{eqnarray*}
  and after a change of parameters $s' = 2s$ and $s'' = 2s-1$, we get
  \begin{eqnarray*}
    \CHK\!\omega(\sigma * \sigma')_t(1)
    & =& \int_0^1 \omega \left(\vect{s' \\ t} \mapsto \gamma_t(s') \right)_{{s' \choose t}}\vect{1 \\ 0} \vect{0 \\ 1} \ ds' \\
    & + & \int_0^1 \omega \left(\vect{s'' \\ t} \mapsto \gamma'_t(s'') \right)_{{s'' \choose t}}\vect{1 \\ 0} \vect{0 \\ 1} \ ds'' \\
    & = & \CHK\!\omega(\sigma)_t(1) + \CHK\!\omega(\sigma')_t(1).
  \end{eqnarray*}
  We proved $(\star)$ which, by integration, gives $(\clubsuit)$.
\end{proof}

\begin{article}\artlabel{Integration bundles of closed $2$-forms}%
  \addcontentsline{toc}{section}{\small\hspace{10pt} Integration bundles of closed $2$-forms}%
  \label{Integration-bundles-of-closed-2-forms}~
  We have seen that every $\Torus$-principal fiber bundle
  ---~$\Torus$ being an irrational torus~---
  equipped with a connection form $\lambda$,
  with curvature $\omega$,
  defines a unique paths cocycle $\Phi$ which permits us to reconstruct the bundle by quotient \art{Torus-bundles-over-diffeological-spaces}.
  Conversely,
  we shall see in this paragraph how to associate,
  with any nonzero closed $2$-form $\omega$ on a diffeological space $\X$,
  a paths cocycle $\Phi$ and construct a $\Torus$-principal bundle equipped with a connection $\lambda$ whose curvature is $\omega$.
  We shall however restrict ourselves to simply connected spaces,%
  \footnote{The general case will be the subject of a separate publication.}
  %
  We assume now $\X$ connected,
  simply connected,
  here are the data:
  $$%
  \pi_0(\X) = \{\X\}, \ \pi_1(\X) = \{0\}, \ \omega \in \DForms^2(\X), \ \omega \neq 0, \text{ and } d\omega = 0.
  $$%
  Thanks to the fundamental property of the Chain-Homotopy operator $\CHK$ \art{The-Chain-Homotopy-operator-K},
  that is,
  $d \circ \CHK + \CHK \circ d = \1^* - \0^*$,
  the restriction of the $1$-form $\CHK\!\omega \in \DForms^1(\Paths(\X))$ to the subspace $\Loops(\X)$ is closed,
  $d[\CHK\!\omega \restriction \Loops(\X)] = 0$.
  Then,
  we define $\Periods_\omega \subset \RR$ as the {\em group of periods}\index{Group of periods} of $\CHK\!\omega \restriction \Loops(\X)$,
  that is,
  $$%
  \Periods_\omega = \left\{ \int_\sigma \CHK\!\omega \mid \sigma \in \Loops(\Loops(\X)) \right\}.
  $$%
  We shall assume now that the group $\Periods_\omega$ is (diffeologically) discrete,
  or which is equivalent $\Periods_\omega \neq \RR$.
  We define the {\em torus of periods}\index{Torus of periods} $\Torus_\omega$ as the quotient
  \begin{equation}
    \renewcommand{\theequation}{$\clubsuit$}
    \Torus_\omega = \RR /\Periods_\omega.
  \end{equation}
  Thus,
  $\Torus_\omega$ is a 1-dimensional diffeological torus,
  see \exref{Dimension-of-tori}.
  We denote by $\class : t \mapsto \cl{t}$ the canonical projection from $\RR$ to $\Torus_\omega$,
  and by $\theta$ the canonical $1$-form on $\Torus_\omega$,
  pushforward of $dt$ by the projection $\class$ \art{Pushing-forms-onto-quotients},
  $$%
  \class : \RR \to \Torus_\omega, \text{ and } \class^*(\theta) = dt.
  $$%
  Now,
  consider the tautological pullback of
  $\ends : \Paths(\X) \to \X \times \X$,
  $$%
  \ends^*(\Paths(\X)) = \{ (\gamma,\gamma') \in \Paths(\X)^2 \mid \ends(\gamma) = \ends(\gamma') \}.
  $$%
  Let $\pr_1$ and $\pr_2$ be the projections from $\ends^*(\Paths(\X))$ onto each factor.
  \begin{center}
    \begin{tikzcd}[column sep=large, row sep=large, every label/.append style = {font = \small}]
      {\ends^*(\Paths(\X))} \arrow[r,"\pr_2"] \arrow[d, swap, "\pr_1"] & {\Paths(\X)} \arrow[d, "\ends"]  \\
      {\Paths(\X)} \arrow[r, swap, "\ends"] & \X \times \X
    \end{tikzcd}
  \end{center}

  1. The space $\ends^*(\Paths(\X))$ is homotopic to $\Loops(\X)$,
  and $\Loops(\X)$ is connected,
  thus $\ends^*(\Paths(\X))$ is connected.
  
  2. The following $1$-form $\CHK\!\omega \ominus \CHK\!\omega$,
  defined on $\ends^*(\Paths(\X))$,
  is closed,
  $$%
  \CHK\!\omega \ominus \CHK\!\omega = \pr_1^*(\CHK\!\omega) - \pr_2^*(\CHK\!\omega), \text{ and } d\left[ \CHK\!\omega \ominus \CHK\!\omega \right] = 0.
  $$%
  
  3. There exists,
  up to a constant,
  a unique smooth integration function $\Phi$
  $$%
  \Phi : \ends^*(\Paths(\X)) \to \Torus_\omega, \text{ with }
  \left\{
  \begin{array}{l}
    \CHK\!\omega \ominus \CHK\!\omega = \Phi^*(\theta), \\
    \vspace{-7pt} \\
    \Phi(\gamma,\gamma') + \Phi(\gamma',\gamma'') = \Phi(\gamma,\gamma'').
  \end{array}
  \right.
  $$%
  
  4. Let $\phi$ be the following equivalence relation,
  defined on $\Paths(\X)$ and associated with
  the paths cocycle $\Phi$:
  \begin{equation}
    \renewcommand{\theequation}{$\heartsuit$}
    \gamma \mathrel{\phi} \gamma', \text{ if } \ends(\gamma) = \ends(\gamma'), \text{ and } \Phi(\gamma,\gamma') = 0,
  \end{equation}
  Let $\Y$ be the quotient of the space $\Paths(\X,x)$ by the relation $\phi$,
  $$%
  \Y = \Paths(\X,x)/\phi.
  $$%
  Then,
  $\Y$ is a $\Torus_\omega$-principal bundle over $\X$ for the projection $\pi : \Y \to \X$ defined by $\pi(\class_\omega(\gamma)) = \gamma(1)$,
  where $\gamma \in \Paths(\X,x)$ and $\class_\omega : \Paths(\X,x) \to \Y$ is the projection associated with $\phi$.
  Moreover,
  the $1$-form $\CHK\!\omega$ passes to the quotient $\Y$ into a connection form $\lambda$ with curvature $\omega$,
  that is,
  $$%
  \class_\omega^*(\lambda) = \CHK\!\omega, \text{ and } d\lambda = \pi^*(\omega).
  $$%
  
  5. The $\Torus_\omega$-principal fiber bundle $\pi : \Y \to \X$,
  together with the connection form $\lambda$,
  is unique up to isomorphism.
  Such a pair $(\pi,\lambda)$ will be called an {\em integration structure} of the closed $2$-form  $\omega$.
  \begin{center}
    \begin{tikzcd}[column sep=large, row sep=large, every label/.append style = {font = \small}]
      \Paths(\X,x) \arrow[dr, swap, "\1"] \arrow[rr,"\class_\omega"] & {} & \Y \arrow[dl, "\pi"]  \\
      {} & \X  & {}
    \end{tikzcd}
  \end{center}
  %
  \Note{1} The relation $\phi$ defines a groupoid $\KK$,
  with structure group $\Torus_\omega$,
  by
  $$%
  \Obj(\KK) = \X, \text{ and } \Mor(\KK) = \cY, \text{ with } \cY = \Paths(\X)/\phi.
  $$%
  For all $x$ and $x'$ in $\X$,
  $$%
  \Mor_{\KK}(x,x') = \{ \class_\omega(\gamma) \mid \ends(\gamma) = (x,x') \},
  $$%
  and the groupoid composition is the factorization of the concatenation of paths,
  $$%
  \class_\omega(\gamma) \cdot \class_\omega(\gamma') = \class_\omega(\gamma \vee \gamma').
  $$%
  Moreover,
  there exists a differential $1$-form
  $$%
  \blambda \in \DForms^1(\cY), \text{ such that } \class_\omega^*(\blambda) = \CHK\!\omega.
  $$%
  This $1$-form is invariant by precomposition and postcomposition in $\cY$,
  that is,
  $$%
  \left\{\begin{array}{lcl}
    \eL(\tau)^*(\blambda \restriction \Mor_\KK(x,\star)) & = & \blambda \restriction \Mor_\KK(x',\star), \\
    \eR(\tau)^*(\blambda \restriction \Mor_\KK(\star,x')) & = & \blambda \restriction \Mor_\KK(\star,x),
  \end{array}\right.
  $$%
  where $\tau \in \Mor_\KK(x',x)$ acts by $\eL(\tau) : y \mapsto \tau \cdot y$ on the subspace $\Mor_\KK(x,\star)$ of morphisms with source $x$ (precomposition),
  and by  $\eR(\tau) : y' \mapsto y' \cdot \tau$ on the subspace $\Mor_\KK(\star,x')$ of morphisms with target $x'$ (postcomposition).
  The groupoid $\KK$ is fibrating \art{Fibrating-groupoids},
  and this is the characteristic groupoid of the integration bundle $\pi : \Y \to \X$ \art{Principal-bundle-attached-to-a-fibrating-groupoid},
  $\Y$ is the subspace $\cY_x = \Mor_\KK(x,\star)$.
  The fact that $\lambda$ is a connection form on $\Y$ is a direct consequence of the invariance of $\blambda$ by precomposition and postcomposition in $\cY$.
  %
  The group $\Diff(\X,\omega)$ of diffeomorphisms of $\X$,
  preserving $\omega$, acts naturally on the groupoid $\KK$,
  by precomposition.
  An element $\varphi \in \Diff(\X,\omega)$ maps $x \in \Obj(\KK)$ naturally to $\varphi(x)$ and $\class_\omega(\gamma)$ to $\class_\omega(\varphi \circ \gamma)$.
  This action preserves the $1$-form $\blambda$.
  Hence,
  the group of automorphisms of the structure $(\X,\omega)$ has a natural representation in the group of automorphisms of the structure $(\KK,\blambda)$.
  
  \Note{2} If $\omega = d\alpha$, then $\Periods_\omega = \{0\}$ and $\Torus_\omega = \RR$.
  The integration structure is made up of the trivial bundle $\pr_1 : \Y = \X \times \RR \to \X$ and of the connection form $\lambda = \alpha \oplus dt$.
  If $\omega$ is not exact,
  then the principal bundle $\pi : \Y \to \X$ is not trivial.
\end{article} %% Integration-bundles-of-closed-2-forms

\begin{proof}
  Let us check first that $\ends^*\!(\Paths(\X))$ is homotopy equivalent to $\Loops(\X)$.
  Let us recall that $\Paths(\X)$ and $\Loops(\X)$ are homotopy  equivalent,
  respectively,
  to the stationary paths $\stPaths(\X)$ and $\stLoops(\X)$;
  see \art{Stationary-paths} and \exref{Deformation-onto-stationary-paths}.
  Thus,
  we shall work with stationary paths.
  We consider the subspace $\stLoops_{1/2}(\X)$ of stationary loops that are stationary at $t = 1/2$,
  that is,
  constant on an open interval $\openinterval{1/2 -\varepsilon,1/2+\varepsilon}$.
  The proof of \exref{Deformation-onto-stationary-paths},
  can be adapted to show that $\stLoops(\X)$ and $\stLoops_{1/2}(\X)$ are homotopy equivalent.
  Now,
  let $f : \ends^*(\stPaths(\X)) \to \stLoops_{1/2}(\X)$ defined by
  \begin{equation}
    \renewcommand{\theequation}{$\diamondsuit$}
    f(\gamma,\gamma') = \gamma \vee \bar\gamma',
  \end{equation}
  where $\bar\gamma'$ is the reverse path $\bar\gamma'(t) = \gamma'(1-t)$.
  Next,
  let $g : \stLoops_{1/2}(\X) \to \ends^*(\stPaths(\X))$ defined by $\gamma(t) =\ell(t/2)$,
  if $0 \leq t \leq 1$,
  $\gamma(t) = \gamma(0)$ if $t\leq 0$ and $\gamma(t) = \ell(1/2)$ if $t\geq 1$;
  and $\gamma'(t) = \ell(1-t/2)$,
  if $0 \leq t \leq 1$,
  $\gamma'(t) = \ell(1)$ if $t\leq 0$ and $\gamma'(t) = \ell(1/2)$ if $t\geq 1$.
  These two maps,
  $f$ and $g$,
  are homotopic inverse to each other,
  thus $\ends^*(\Paths(\X))$ is homotopy equivalent to $\Loops(\X)$.
  Now,
  since $\pi_1(\X,x) = \pi_0(\Loops(\X,x)) = \{0\}$ and $\pi_0(\Loops(\X))$ is a quotient of $\pi_0(\Loops(\X,x))$ (\exref{Homotopy-of-loops-spaces}),
  then $\Loops(\X)$ is connected and so is $\ends^*(\Paths(\X))$.
  %
  Moreover,
  the map $f$ defined by ($\diamondsuit$) satisfies the identity
  \begin{equation}
    \renewcommand{\theequation}{$\star$}
    \pr_1^*(\CHK\!\omega) - \pr_2^*(\CHK\!\omega) = f^*(\CHK\!\omega).
  \end{equation}
  Indeed,
  let $(\P,\P')$ be a plot of $\ends^*(\stPaths(\X))$,
  then $f^*(\CHK\!\omega)(\P,\P') =$%
  \linebreak%
  $\CHK\!\omega(f\circ(\P,\P')) = \CHK\!\omega(\P * \bar\P')$,
  where the operation $*$ has been defined in \art{Integrating-concatenations-over-homotopies} and $\bar\P'(r)(t) = \P'(r)(1-t)$.
  Thanks to \xart{Integrating-concatenations-over-homotopies}{($\clubsuit$)},
  $\CHK\!\omega(\P * \bar\P') = \CHK\!\omega(\P) + \CHK\!\omega(\bar\P') = \CHK\!\omega(\P) - \CHK\!\omega(\P')$,
  that is,
  $f^*(\CHK\!\omega) = \pr_1^*(\CHK\!\omega) - \pr_2^*(\CHK\!\omega)$.
  Therefore,
  $\CHK\!\omega \ominus \CHK\!\omega$ is closed and has the same periods as $\CHK\!\omega \restriction \Loops(\X)$,
  as a consequence of the homotopic invariance of the De Rham cohomology \art{Homotopic-invariance-of-the-De-Rham-cohomology}.
  %
  Now,
  %we assumed that the periods $\Periods_\omega$ of the $1$-form $\CHK\!\omega \restriction \ends^*(\Paths(\X))$ are discrete.
  we assumed that the periods $\Periods_\omega$ of the $1$-form $\CHK\!\omega \restriction \Loops(\X)$ are discrete.
  Then,
  since $\ends^*(\Paths(\X))$ is connected,
  there exists an integration function $\Phi : \ends^*(\Paths(\X)) \to \T_\omega$ \art{Integrating-closed-1-forms},
  unique up to a constant,
  such that
  $$%
  \left[\pr_1^*(\CHK\!\omega) - \pr_2^*(\CHK\!\omega)\right] = \Phi^*(\theta).
  $$%
  This function is explicitly given by
  \begin{equation}
    \renewcommand{\theequation}{$\spadesuit$}
    \Phi(\gamma,\gamma') = \class\bigg(\int_{\gamma'}^{\gamma} \CHK\!\omega\bigg) \in \Torus_\omega,
  \end{equation}
  where the integral is computed along a path $\sigma$ in $\Paths(\X,x,x')$,
  connecting $\gamma$ to $\gamma'$, with $(x,x') = \ends(\gamma) = \ends(\gamma')$.
  Indeed, we know that
  $$%
  \Phi(\gamma,\gamma') =
  \class\bigg(\int_{(\bmx, \bmx)}^{(\gamma, \gamma')}
  \pr_1^*(\CHK\!\omega) - \pr_2^*(\CHK\!\omega)\bigg)
  $$%
  is a solution of $\pr_1^*(\CHK\!\omega) - \pr_2^*(\CHK\!\omega) = \Phi^*(\theta)$
  \art{Integrating-closed-1-forms},
  where $\bmx : t \mapsto x$ and $x \in \X$ is a fixed basepoint,
  and the integral is computed along any path in $\ends^*(\stPaths(\X))$ connecting $(\bmx, \bmx)$ to $(\gamma,\gamma')$.
  But,
  thanks to $(\diamondsuit)$ and ($\star$),
  $$%
  \class\bigg(\int_{(\bmx, \bmx)}^{(\gamma, \gamma')} \pr_1^*(\CHK\!\omega) - \pr_2^*(\CHK\!\omega)\bigg)
  = \class\bigg(\int_\bmx^{\gamma \vee \bar\gamma'} \CHK\!\omega\bigg),
  $$%
  where the integral is computed along a path in $\stLoops(\X,x)$,
  connecting $\bmx$ to $\gamma \vee \bar\gamma'$.
  Now,
  let us consider the map $\nu : \gamma' \mapsto \gamma \vee \bar\gamma'$ from $\Paths(\X,x,x')$ to $\Loops(\X,x)$.
  Thanks to the invariance of $\CHK\!\omega$ by concatenation,
  $\nu^*(\CHK\!\omega) = - \CHK\!\omega$,
  the ``$-$'' sign is due to reversing path.
  Thus,
  $$%
  \class\bigg(\int_\bmx^{\gamma \vee \bar\gamma'} \CHK\!\omega\bigg) = \class\bigg(\int_{\gamma'}^{\gamma} \CHK\!\omega\bigg),
  $$%
  and this confirms the expression of $\Phi$ given above.
  Then,
  $\Phi$ is clearly additive and $\Phi(\gamma,\gamma') + \Phi(\gamma',\gamma'') = \Phi(\gamma,\gamma'')$
  for any triple of paths $\gamma$, $\gamma'$ and $\gamma''$ such that $\ends(\gamma) = \ends(\gamma') = \ends(\gamma'')$.
  
  Next,
  let us check that the $1$-form $\CHK\!\omega \in \DForms^1(\Paths(\X))$ passes to the quotient $\cY = \Paths(\X)/\phi$,
  where $\phi$ is the equivalence relation defined by ($\heartsuit$).
  According to \xart{Pushing-forms-onto-quotients}{Note 2},
  we need only check that $\pr_1^*(\CHK\!\omega) - \pr_2^*(\CHK\!\omega)$ vanishes on the total space of the tautological pullback of the projection $\class_\omega : \Paths(\X) \to \cY$,
  that is,
  $$%
  \class_\omega^*(\Paths(\X)) =\{(\gamma,\gamma') \in \ends^*(\Paths(\X)) \mid \Phi(\gamma,\gamma') = 0 \}.
  $$%
  But $\class_\omega^*(\Paths(\X)) = \Phi^{-1}(0)) \subset \ends^*(\Paths(\X))$ and $\pr_1^*(\CHK\!\omega) - \pr_2^*(\CHK\!\omega) = \Phi^*(\theta)$ implies immediately that $\pr_1^*(\CHK\!\omega) - \pr_2^*(\CHK\!\omega) \restriction \class_\omega^*(\Paths(\X)) = 0$.
  Therefore,
  there exists $\blambda \in \DForms^1(\cY)$ such that $\class_\omega^*(\blambda) = \CHK\!\omega$.
  
  Now,
  $\Y = \Paths(\X,x)/\phi$ is the subset $\cY_x \subset \cY$ made of classes of paths based at $x \in \X$,
  thus $\lambda = \blambda \restriction \cY_x$.
  %
  To prove that $\lambda$ is a connection form on a $\Torus_\omega$-principal bundle \art{Connections-and-connection-1-forms-on-torus-bundles},
  we shall prove first that $\KK$,
  defined in the Note 1,
  is a groupoid with structure group $\Torus_\omega$,
  objects $\X$,
  and morphisms $\cY$.
  
  {\em Concatenation.} Let $\gamma_0$ and $\gamma'_0$ be two stationary paths such that $\gamma_0(1) = \gamma'_0(0)$.
  Then,
  let $\gamma_1$ and $\gamma'_1$ be two other paths such that $\class_\omega(\gamma_0) = \class_\omega(\gamma_1)$ and $\class_\omega(\gamma'_0) = \class_\omega(\gamma'_1)$.
  Because $\X$ is simply connected,
  there exists a fixed-ends homotopy $\sigma$ connecting $\gamma_0$ to $\gamma_1$,
  and another one $\sigma'$ connecting $\gamma'_0$ to $\gamma'_1$,
  that is,
  $\int_\sigma \CHK\!\omega \in \Periods_\omega$ and $\int_{\sigma'} \CHK\!\omega \in \Periods_\omega$.
  Thanks to \art{Integrating-concatenations-over-homotopies},
  $\int_{\sigma*\sigma'} \CHK\!\omega = \int_{\sigma} \CHK\!\omega + \int_{\sigma'} \CHK\!\omega$,
  then $\int_{\sigma*\sigma'} \CHK\!\omega \in \Periods_\omega$,
  where $\sigma*\sigma'$ is a homotopy from $\gamma_0 \vee \gamma'_0$ to $\gamma_1 \vee \gamma'_1$.
  Thus $\class_\omega(\gamma_0 \vee \gamma'_0) = \class_\omega(\gamma_1 \vee \gamma'_1)$,
  and hence the composition $\class_\omega(\gamma) \cdot \class_\omega(\gamma')$ is well defined on $\cY$.
  
  {\em Associativity, identities, and inverses.}  The associativity of the concatenation,
  the fact that $\id_x = \class_\omega(\bmx)$ and $\class_\omega(\gamma)^{-1} = \class_\omega(\bar\gamma)$,
  where $\bar\gamma$ is the reverse of $\gamma$,
  are all based on the homotopies described in \xart{The-Poincare-groupoid-and-fundamental-group}{Proof},
  connecting $(\gamma_1 \vee \gamma_2) \vee \gamma_3$ to $\gamma_1 \vee (\gamma_2 \vee \gamma_3)$,
  $\gamma$ to $\bmx \vee \gamma$ and $\gamma \vee \bar\gamma$ to $\bmx$.
  The integral of $\CHK\!\omega$ on these homotopies vanishes.
  Indeed,
  the integrand of
  $$%
  \CHK\!\omega(\sigma)_t(1) = \int_0^1 \omega\left(\vect{s \\ t} \mapsto \sigma(t)(s)\right)_{s \choose t} \vect{1 \\ 0}\vect{0 \\ 1} \ ds
  $$%
  itself vanishes,
  because the homotopy $\sigma$ factorizes through a path,
  that is,
  $\sigma(t)(s) = \gamma'(\varphi(t,s))$ for some path $\gamma'$ and some real function $\varphi$,
  but the pullback of a $2$-form on $\RR$ vanishes.
  Thus,
  $\KK$ is a groupoid.
  Then,
  since the concatenation is smooth,
  as well as the inversion \xart{Stationary-paths}{2},
  and since $x \mapsto \class_\omega(\bmx)$ is clearly an induction,
  $\KK$ is a diffeological groupoid.
  Moreover,
  since $\ends : \Paths(\X) \to \X \times \X$ is a subduction and $\class_\omega : \Paths(\X) \to \cY$ is smooth,
  the factorization $\bpi : \cY_\omega \to \X \times \X$ is a subduction \art{Smooth-maps-from-quotients}.
  Therefore,
  $\KK$ is a fibrating groupoid \art{Fibrating-groupoids},
  and then $\Y= \cY_x$ is the principal fiber bundle attached at the point $x$ \art{Principal-bundle-attached-to-a-fibrating-groupoid}.
  
  Let us examine now the structure group of $\KK$ at a point $x \in \X$.
  By construction,
  $\KK_x = \Mor_{\KK}(x,x)$ is the quotient $\Loops(\X,x)/\phi$,
  but the restricted cocycle $\Phi \restriction \Loops(\X,x)$ is trivial.
  Indeed,
  for any pair of loops $\ell$ and $\ell'$ of $\X$ based at $x$,
  $$%
  \Phi(\ell,\ell') = \F(\ell) - \F(\ell'), \text{ with } \F(\ell) = \class\bigg(\int_{\sbmx}^{\ell} \CHK\!\omega\bigg) \in \Torus_\omega.
  $$%
  Therefore,
  $\Phi(\ell,\ell') = 0$ if and only if $\F(\ell) = \F(\ell')$.
  Hence,
  set theoretically,
  $\F : \Loops(\X,x) \to \Torus_\omega$ identifies $\Val(\F)$ with $\KK_x$.
  Let us show first that $\Val(\F)$ can be {\em a priori\/} either $\{0\}$ or $\Torus_\omega$,
  but since $\omega \neq 0$,
  then $\Val(\F) = \Torus_\omega$.
  %
  The map $\F$ is  the projection,
  on $\Loops(\X,x)$,
  of the smooth map
  $$%
  \eF : \Paths(\Loops(\X,x),\bmx,\star) \to \RR, \text{ with } \eF(\sigma) = \int_\sigma \CHK\!\omega.
  $$%
  Let $\sigma$ and $\sigma'$ be two elements of $\Paths(\Loops(\X,x),x,\star)$,
  connecting $\bmx$ to $\ell$ and $\bmx$ to $\ell'$.
  Let $\hat\sigma$ defined by $\hat\sigma_t(s) =  \sigma_t(1-s)$,
  hence $\hat\sigma_t(s)$ is a path in $\Loops(\X,x)$ connecting $\bmx$ to the reverse $\bar\ell$.
  Thanks to \art{Integrating-concatenations-over-homotopies},
  and to the fact that $\CHK\!\omega(\hat\sigma) = - \CHK\!\omega(\sigma)$,
  we get
  \begin{equation}
    \renewcommand{\theequation}{$\ast$}
    \eF(\sigma*\sigma') = \eF(\sigma) + \eF(\sigma')
    \text{ and }
    \eF(\hat\sigma) = - \eF(\sigma).
  \end{equation}
  Thus,
  $\Val(\eF)$ is a subgroup of $\RR$,
  but $\Val(\eF)$ is connected because $\Paths(\Loops(\X,x),\bmx,\star)$ is connected (actually contractible) and $\eF$ is smooth.
  Hence,
  either $\Val(\eF) = \{0\}$ or $\Val(\eF) = \RR$,
  that gives $\Val(\F) = \{0\}$ or $\Val(\F) = \Torus_\omega$.
  Let us examine the two different cases.
  
  {\em The zero case,} $\Val(\F) = \{0\}$.
  Then, $\KK$ is trivial and $\omega =0$.
  Indeed,
  in this condition the groupoid $\KK_x$ is reduced to the identity $\{1_x\}$,
  and the principal fiber bundle attached to the point $x$ \art{Principal-bundle-attached-to-a-fibrating-groupoid} is just the identity map,
  $\Y = \X$ and $\pi = \id_\X : \X \to \X$.
  Thus,
  $\omega$ is exact $\omega = d\lambda$.
  But returning to $\CHK\!\omega$,
  we have $\CHK\!\omega = \CHK(d \lambda) = \1^*(\lambda) -\0^*(\lambda) - d(\CHK\lambda)$,
  then  $\CHK\!\omega \restriction \Loops(\X,x) = d(\CHK\lambda)$ and the condition $\eF = 0$ writes,
  for all $\ell \in \Loops(\X,x)$,
  $$%
  \int_\sbmx^\ell d\CHK\lambda = \CHK\lambda(\ell) = \int_\ell \lambda = 0.
  $$%
  Therefore,
  $\lambda$ is a $1$-form vanishing on every loop,
  thus $\lambda$ is closed and $\omega = 0$;
  see \exref{1-forms-vanishing-on-loops},
  and this is not permitted by the hypothesis.
  
  {\em The full case,} $\Val(\F) = \Torus_\omega$.
  Then,
  $\omega \neq 0$ and $\F : \Loops(\X,x) \to \Torus_\omega$ is a subduction projecting to a smooth isomorphism $j : \KK_x \to \Torus_\omega$.
  Indeed,
  $\eF$ is already a smooth surjection,
  let us check that $\eF$ is a subduction,
  what will imply that $\F$ itself is a subduction,
  and therefore that its projection $j : \KK_x \to \Torus_\omega$ is a smooth isomorphism.
  Let us choose any $\sigma \in \Paths(\Loops(\X,x),\bmx,\star)$ such that $\eF(\sigma) \neq 0$,
  and let $\sigma_s(t) = \sigma(st)$.
  Then,
  $\varphi : s \mapsto \eF(\sigma_s)$ is a smooth parametrization such that $\varphi(0) = \eF(\bmx) = 0$ and $\varphi(1) = \eF(\sigma) \neq 0$.
  Thus,
  there exists $s_0 \in \left]0,1\right[$ such that $\varphi'(0) \neq 0$.
  Let $\fS(s) = \hat\sigma_{s_0} * \sigma_{s+s_0}$,
  where $\hat\sigma_{s_0} = [t \mapsto [ t' \mapsto \sigma_{s_0}(t)(1-t')]$,
  and $\psi(s) = \eF(\fS(s))$,
  we have $\eF(\hat\sigma_{s_0} * \sigma_{s+s_0}) = \eF(\sigma_{s+s_0}) - \eF(\sigma_{s_0})$ (see above),
  and then $\psi(s) = \varphi(s + s_0) - \varphi(s_0)$.
  Thus,
  $\psi(0) = 0$ and $\psi'(0) \neq 0$.
  Therefore,
  there exists $\varepsilon > 0$ such that $\psi \restriction \left]- \varepsilon,+ \varepsilon \right[$ is a local diffeomorphism by the implicit function theorem.
  Hence,
  $\fS \circ \psi^{-1} : \left]- \varepsilon,+ \varepsilon \right[ \to \Paths(\Loops(\X,x),\bmx,\star)$ is a smooth local section of $\eF : \Paths(\Loops(\X,x),\bmx,\star) \to \RR$.
  By additivity with respect to the $*$ operation, there exists a local smooth section of $\eF$ everywhere and therefore $\eF$ is a subduction.
  Then,
  by projection, $\F$ is itself a subduction, and
  the projection $j : \KK_x \to \Torus_\omega$,
  which is injective by construction,
  is a diffeomorphism and therefore a smooth isomorphism from $\KK_x$ to $\Torus_\omega$.
  Therefore,
  $\KK$ is a fibrating groupoid with structure group $\Torus_\omega$.
  
  Next,
  the invariance of $\blambda$,
  with respect to precomposition and postcomposition,
  is the translation on $\cY_\omega = \Paths(\X)/\phi$ of the invariance of $\CHK\!\omega$ by preconcatenation and postconcatenation,
  proved in \art{Chain-Homotopy-and-paths-concatenation}.
  Considering the $1$-form $\lambda = \blambda \restriction \Y$,
  the invariance of $\blambda$ by precomposition and postcomposition writes in particular for $x' = x$,
  $\tau_\Y^*(\lambda) = \lambda$,
  $\orb{y}^*(\lambda) = \theta$ and $\pi^*(\omega) = d\lambda$,
  for all $\tau$ in $\Torus_\omega$,
  where $\tau^{\phantom{*}}_\Y$ denotes the action of $\tau$ on $\Y$,
  that is, $\tau^{\phantom{*}}_\Y(y) = j^{-1}(\tau) \cdot y$.
  Therefore,
  the $1$-form $\lambda$ is precisely a connection form on $\Y$ with curvature $\omega$.
  
  Let us consider now the special case $\omega = d\alpha$.
  First of all,
  $\Periods_\omega = \{0\}$ and $\Torus_\omega = \RR$.
  Now let $\gamma$ and $\gamma'$ be two paths such that $\ends(\gamma) = \ends(\gamma')$,
  and $\sigma$ a homotopy connecting $\gamma'$ to $\gamma$ in $\Paths(\X,x,x')$.
  Then,
  $$%
  \Phi(\gamma,\gamma') = \int_\sigma\CHK\!\omega
  =  \int_\sigma \CHK\left [d\alpha\right]
  =  \int_\sigma (\1^* - \0^*)(\alpha) - \int_\sigma d\left[\CHK\!\alpha\right]
  =  \CHK\!\alpha(\gamma') - \CHK\!\alpha(\gamma).
  $$%
  Thus,
  the cocycle $\Phi$ is trivial,
  and hence the principal bundle $\pi : \Y \to \X$ also is trivial \art{Torus-bundles-over-diffeological-spaces}.
  Therefore,
  $\pr_1 : \X \times \RR \to \X$,
  equipped with the connection $\lambda = \alpha \oplus dt$,
  is a representative of the integration structure.
  
  Let us prove that this construction is essentially unique.
  Let $\pi' : \Y' \to \X$ be some $\Torus_\omega$-principal fiber bundle,
  equipped with a connection $\lambda'$ of curvature $\omega \neq 0$.
  In \art{Torus-bundles-over-diffeological-spaces},
  we have seen that there exists a smooth map $\psi : \Paths(\X,x) \to \Y'$ such that $\psi^*(\lambda') = \CHK\!\omega$,
  and that $\Y'$ can be then reconstructed by quotient.
  %
  Let us begin by proving that there exists a smooth injection $\J : \Y = \Paths(\X,x)/\phi \to \Y'$ such that $\J \circ \class_\omega = \psi$,
  that is,
  $\psi(\gamma) = \psi(\gamma')$ if and only if $\class_\omega(\gamma) = \class_\omega(\gamma')$.
  First of all,
  $\psi(\gamma) = \psi(\gamma')$ implies $\gamma(1) = \gamma'(1)$,
  let  $y' = \psi(\gamma) = \psi(\gamma')$  and $x' = \pi(y') = \gamma(1) = \gamma'(1)$.
  Then,
  since by hypothesis $\X$ is simply connected,
  there exists a smooth path $\sigma$ in $\Paths(\X,x,x')$ connecting $\gamma$ to $\gamma'$.
  Hence,
  $$%
  \int_\sigma \CHK\!\omega = \int_\sigma \psi^*(\lambda) = \int_{\psi \circ \sigma} \lambda.
  $$%
  Since $\psi(\sigma(0)) = \psi(\gamma) = \psi(\gamma') = \psi(\sigma(1))$,
  the path $\psi \circ \sigma : t \mapsto y'_t$ is a loop in $\Y_{x'} = {\pi'}^{-1}(x')$,
  based at $y'$.
  But,
  since $\pi$ is a principal fibration,
  there exists a smooth loop $\tau$ in $\Torus_\omega$,
  based at the origin, such that $y'_t = \tau(t)_\Y(y') = \orb{y'}(\tau (t))$,
  where $\orb{y'}$ denotes the orbit map of the point $y'$.
  But $\lambda$ is a connection form,
  so $\orb{y'}^*(\lambda) = \theta$,
  where $\theta$ is the canonical $1$-form on $\Torus_\omega$,
  thus
  $$%
  \int_{\psi \circ \sigma} \lambda = \int_{\orb{y'} \circ \tau} \lambda = \int_\tau \orb{y'}^*(\lambda) = \int_\tau \theta = \int_{0}^{p} {d\over ds}(t_s)\, ds = p,
  $$%
  where $s \mapsto t_s$ is a lift in $\RR$ of $\tau$,
  with $t_0 = 0$ and $t_1 = p$.
  But,
  since $\tau$ is a loop,
  $p \in \Periods_\omega$,
  thus $\Phi(\gamma,\gamma') = 0$,
  that is,
  by definition,
  $\class_\omega(\gamma) = \class_\omega(\gamma')$.
  Conversely,
  if $\class_\omega(\gamma) = \class_\omega(\gamma')$,
  then the same computation shows that $\psi(\gamma) = \psi(\gamma')$.
  Therefore,
  $\J$ is well defined and is a smooth injection.
  
  Now,
  let us consider the restriction of $\psi$ to $\Loops(\X,x)$ with values in $\Y'_x = {\pi'}^{-1}(x)$.
  Let us identify the fiber $\Y'_x$ with $\Torus_\omega$, $\psi(\bmx)$ then becoming the identity.
  By a computation analogous to the previous one and thanks to \art{Integrating-concatenations-over-homotopies},
  we get $\psi(\ell \vee \ell') = \psi(\ell) + \psi(\ell')$ and $\psi(\bar\ell) = - \psi(\ell)$,
  where $\bar\ell$ is the reverse of $\ell$.
  Thus, $\psi(\Loops(\X,x))$ is a connected subgroup of $\Torus_\omega$,
  that is,
  either $\Torus_\omega$ or $\{0\}$.
  If $\psi(\Loops(\X,x)) = \Torus_\omega$,
  then the injection $\J$ is surjective,
  thus bijective,
  and moreover an equivariant diffeomorphism.
  The integration structures $(\pi,\lambda)$ and $(\pi',\lambda')$ are equivalent.
  Now, if $\psi(\Loops(\X,x)) = \{0\}$,
  we note first that $\psi(\Loops(\X,x))$ gives exactly the holonomy of the connection $\lambda'$ defined in \art{The-holonomy-of-a-connection},
  and we know that the fibration is reducible to its holonomy.
  Thus,
  the fibration is reduced to a covering,
  but since $\X$ is simply connected the covering is trivial.
  Hence,
  $\omega = d\alpha$ $\Periods_\omega = \{0\}$, $\Torus_\omega = \RR$ and $\pi' : \Y' \mapsto \X$,
  equipped with the connection $\lambda'$,
  is equivalent to $\pr_1 : \X \times \RR \to \X$ equipped with the connection $\lambda = \alpha \oplus dt$.
  But this is also a representative of the standard integration structure for an exact $2$-form.
  Therefore,
  there exists essentially only one integration structure for the closed $2$-form  $\omega$.
\end{proof}

%%%%%%%%%%%%%%%%%%%%%%%%%%%%%%%%%%%%%%%%%%%%%%%%%%%%%%%%%%
%
%   Exercises
%
%%%%%%%%%%%%%%%%%%%%%%%%%%%%%%%%%%%%%%%%%%%%%%%%%%%%%%%%%%

\Exercises

\begin{exercise}[Loops in the torus]
  \label{Loops-in-the-torus}
  Consider the torus $\Torus^2 = \RR^2/\ZZ^2$,
  and let $\class : \RR^2 \to \Torus^2$ be the projection.
  Let $\omega = \class_*(\ee^1 \wedge \ee^2)$ be the canonical volume on $\Torus^2$.
  Let $\ell : t \mapsto \class(t,0) \in \Loops(\Torus^2)$,
  use the $1$-form $\CHK\!\omega$ to show that the connected component $\comp(\ell) \subset \Loops(\Torus^2)$ is not simply connected.
\end{exercise} %% Loops-in-the-torus

\begin{exercise}[Periods of a surface]
  \label{Periods-of-a-surface}
  Consider the torus $\Torus^2 = \RR^2/\ZZ^2$.
  Justify the fact that every connected component of $\Loops(\Torus^2)$ is the component of a loop $\ell_{n,m} : t \mapsto \class(nt,mt)$,
  for some $(n,m) \in \ZZ^2$.
  Show that every loop $\sigma : s \mapsto \sigma_s$ in $\comp(\ell_{n,m})$,
  based at $\ell_{n,m}$,
  is fixed-ends homotopic to a loop $s \mapsto \sigma'_s =$
  \linebreak
  $[t \mapsto \class(nt+ks,mt+k's)]$, for some $(k,k') \in \ZZ^2$.
  Let $\omega$ be the $2$-form defined on $\Torus^2$  in \exref{Loops-in-the-torus},
  show that the periods of $\CHK\!\omega$ on the component of $\ell_{n,m}$ are the group
  $$%
  \PeriodsOf(\CHK\!\omega \restriction \comp(\ell_{n,m})) = \{nk'-mk \mid k,k' \in \ZZ\}.
  $$%
  Comment on the result.
\end{exercise} %% Periods-of-a-surface
