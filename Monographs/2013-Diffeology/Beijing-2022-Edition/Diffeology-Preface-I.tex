\chapter*{Preface}
%\addcontentsline{toc}{chapter}{Preface I}

\begin{chaphead}
  At the end of the last century,
  differential geometry was challenged by theoretical physics:
  new objects were displaced from the periphery of the classical theories to the center of attention of the geometers.
  These are the irrational tori,
  quotients of the 2-dimensional torus by irrational lines,
  with the problem of quasi-periodic potentials,
  or orbifolds with the problem of singular symplectic reduction,
  or spaces of connections on principal bundles in Yang-Mills field theory,
  also groups and subgroups of symplectomorphisms in symplectic geometry and in geometric quantization,
  or coadjoint orbits of groups of diffeomorphisms,
  the orbits of the famous Virasoro group for example.
  All these objects,
  belonging to the outskirts of the realm of differential geometry,
  claimed their place inside the theory, as full members.
  Diffeology gives them satisfaction in a unified framework,
  bringing simple answers to simple problems,
  by being the right balance between rigor and simplicity,
  and pushing off the boundary of classical geometry to include seamlessly these objects in the heart of its concerns.
  
  However,
  diffeology did not spring up on an empty battlefield.
  Many solutions have been already proposed to these questions,
  from functional analysis to noncommutative geometry,
  via smooth structures \`a la Sikorski or \`a la Fr\"olicher.
  For what concerns us,
  each of these attempts is unsatisfactory:
  functional analysis is often an overkilling heavy machinery.
  Physicists run fast;
  if we want to stay close to them we need to jog lightly.
  Noncommutative geometry is uncomfortable for the geometer who is not familiar enough with the $\CC^*$-algebra world,
  where he loses intuition and sensibility.
  Sikorski or Fr\"olicher spaces miss the singular quotients.
  Perhaps most frustrating,
  none of these approaches embraces the variety of situations at the same time.
  
  So,
  what's it all about? Roughly,
  a diffeology on an arbitrary set $\X$ declares,
  which of the maps from $\RR^n$ to $\X$ are {\em smooth},
  for all integers $n$.
  This idea,
  refined and structured by three natural axioms,
  extends the scope of classical differential geometry far beyond its usual targets.
  The smooth structure on $\X$ is then defined by all these {\em smooth parametrizations},
  which are not required to be injective.
  This is what gives plenty of room for new objects,
  the quotients of manifolds for example,
  even when the resulting topology is vague.
  The examples detailed in the book prove that diffeology captures remarkably well the smooth structure of singular objects.
  But quotients of manifolds are not the sole target of diffeology,
  actually they were not even the first target,
  which was spaces of smooth functions,
  groups of diffeomorphisms.
  Indeed,
  these spaces have a natural {\em functional diffeology},
  which makes the category Cartesian closed.
  But also,
  the theory is closed under almost all set-theoretic operations:
  products,
  sums,
  quotients,
  subsets etc.
  Thanks to these nice properties,
  diffeology provides a fair amount of applications and examples and offers finally a renewed perspective on differential geometry.
  
  Also note the existence of a convenient {\em powerset diffeology},
  defined on the set of all the subsets of a diffeological space.
  Thanks to this original diffeology,
  we get a clear notion of what is a smooth family of subsets of a diffeological space,
  without needing any model for the elements of the family.
  This powerset diffeology \guillemots{encodes genetically} the smooth structure of many classical constructions without any exterior help.
  The set of the lines of an affine space,
  for example,
  inherits a diffeology from the powerset diffeology of the ambient space,
  and this diffeology coincides with its ordinary manifold diffeology,
  which is remarkable.
  
  Moreover,
  every structural construction
  (homotopy, Cartan calculus, De Rham cohomology, fiber bundles etc.)
  renewed for this category,
  applies to all these derived spaces
  (smooth functions, differential forms, smooth paths etc.)
  since they are diffeological spaces too.
  This unifies the discourse in differential geometry and makes it more consistent,
  some constructions become more natural and some proofs are shortened.
  For example,
  since the space of smooth paths is itself a diffeological space,
  the Cartan calculus naturally follows and then gives a nice shortcut in the proof of homotopic invariance of the De Rham cohomology.
  
  What about standard manifolds? Fortunately,
  they become a full subcategory.
  Then,
  considering manifolds and traditional differential geometry,
  diffeology does not subtract anything nor add anything alien in the landscape.
  About the natural question,
  ``Why is such a generalization of differential geometry necessary,
  or for what is it useful?''
  The answer is multiple.
  First of all,
  let us note that differential geometry is already a generalization of traditional Greek Euclidean geometry,
  and the question could also be raised at this level.
  More seriously,
  on a purely technical level,
  considering many of the recent heuristic constructions coming from physics,
  diffeology provides a light formal rigorous framework,
  and that is already a good reason.
  Two examples:
  
  {\it Example 1}. For a space equipped with a closed $2$-form,
  diffeology gives a rigorous meaning to the moment maps associated with every smooth group action by automorphisms.
  It applies to every kind of diffeological space,
  it can be a manifold,
  a space of smooth functions,
  a space of connection forms,
  an orbifold or even an irrational torus.
  It works that way because the theory provides a unified coherent notion of differential forms,
  on all these kinds of spaces,
  and the tools to deal with them.
  In particular,
  such a general diffeological construction clearly reveals that the status of moment maps is high in the hierarchy of differential geometry.
  It is clearly a categorical construction which exceeds the ordinary framework of the geometry of manifolds:
  every closed $2$-form  on a diffeological space gets naturally a {\em universal moment map}  associated with its group of automorphisms.
  
  {\it Example 2}. Every closed $2$-form  on a simply connected diffeological space%
  \footnote{The general case will be the subject of a separate publication.}
  %
  is the curvature of a connection form on some diffeological principal bundle.
  The structure group of this bundle is the diffeological {\em torus of periods} of the $2$-form,
  \ie\ the quotient of the real line by the group of periods of the $2$-form.
  This construction is completely universal and applies to every diffeological space and to every closed $2$-form,
  whether the form is integral or not.
  The only condition is that the group of periods is diffeologically discrete,
  that is,
  a strict subgroup of the real numbers.
  The construction of a prequantization bundle corresponds to the special case when the periods are a subgroup of the group generated by the Planck constant $h$ or,
  if we prefer,
  when the group of periods is generated by an integer multiple of the Planck constant.
  
  The crucial point in these two constructions is that the quotient of a diffeological group
  ---~the group of momenta of the symmetry group by the holonomy for the first example,
  and the group of real numbers by the group of periods for the second~---
  is naturally a nontrivial diffeological group whose structure is rich enough to make these generalizations possible.
  In this regard,
  the {\em contravariant approaches}
  ---~Sikorski or Fr\"olicher differentiable spaces~---
  are globally helpless because these crucial quotients are trivial,
  and this is irremedible.
  By respecting the internal (nontrivial) structure of these quotients,
  diffeology leads one to a good level of generality for such general constructions and statements.
  The reason is actually quite simple,
  the contravariant approaches define smooth structures by declaring which maps from $\X$ to $\RR$ are smooth.
  Doing so,
  they capture only what looks like $\RR$
  ---~or a power of $\RR$~---
  in $\X$,
  killing everything else.
  The quotient of a manifold may not resemble $\RR$ at all,
  if we wanted to capture its singularity,
  we would have to compare it with all kinds of standard quotients.
  {\em A contrario\/},
  diffeology as a {\em covariant approach} assumes nothing  about the resemblance of the diffeological space to some Euclidean space.
  It just declares what are the smooth families of elements of the set,
  and this is enough to retrieve the local aspect of the singularity,
  if it is it what we are interested in.
  
  Another strong point is that diffeology treats simply and rigorously infinite-dimensional spaces without involving heavy functional analysis,
  where obviously it is not needed.
  Why would we involve deep functional analysis to show, for example,
  that every symplectic manifold is a coadjoint orbit of its group of automorphisms?
  It is so clear when we know that it is what happens when a Lie group acts transitively,
  and the group of symplectomorphisms acts transitively.
  In this case,
  and maybe others, diffeology does the job easily,
  and seems to be,
  here again,
  the right balance between rigor and simplicity.
  Recently A.~Weinstein et al.~wrote ``For our purposes,
  spaces of functions,
  vector fields,
  metrics,
  and other geometric objects are best treated as diffeological spaces rather than as manifolds modeled  on infinite-dimensional topological vector spaces'' \cite{BFW10}.
  
  \smallskip\noindent{\sc Note A}.
  The axiomatics of {\em Groupes diff\'erentiels} were introduced by J.-M. Souriau in the beginning of the eighties \cite{Sou80}.
  They then mutated into {\em Espaces diff\'erentiels} and eventually into diffeological groups and spaces.
  Diffeology is a variant of the theory of {\em differentiable spaces},
  introduced and developed a few years before by K.T. Chen \cite{Che77}.
  The main difference between these two theories is that Souriau's diffeology is  more differential geometry oriented,
  whereas Chen's theory of differentiable spaces is driven by algebraic geometry considerations.
  
  \smallskip\noindent{\sc Note B}.
  I began to write this textbook in June 2005.
  My goal was,
  first of all,
  to describe the basics of diffeology,
  but also to improve the theory by opening new fields inside,
  and by giving many examples of applications and exercises.
  If the basics of diffeology and a few developments have been published a long time ago now \cite{Sou80} \cite{DI83} \cite{Sou84} \cite{Don84} \cite{Igl85},
  many of the constructions appearing in this book are original and have been worked out during its redaction.
  This is what also explains why it took so long to complete.
  I chose to introduce the various concepts and constructions involved in diffeology from the simple to the complex,
  or from the particular to the more general.
  This is why there are repetitions,
  and some constructions,
  or proofs,
  can be shortened,
  or simplified.
  I included sometimes these simplifications as exercises at the end of the sections.
  In the examples treated,
  I tried to learly separate what is the responsibility of the category and what is specific.
  I hope this will help for a smooth progression in the reading of this text.
  
  \smallskip\noindent{\sc Note C}.
  By the time I wrote these words,
  and seven years after I began this project,
  a few physicists or mathematicians have shown some interest in diffeology,
  enough to write a few papers \cite{BaeHof09} \cite{Sta10} \cite{Sch11}.
  The point of view adopted in these papers is strongly categorical.
  Diffeology is a Cartesian closed category,
  complete and cocomplete.
  Thus,
  diffeology is an ``interesting beast'' from a pure categorical point of view.
  However,
  if I understand and appreciate the categorical point of view,
  it does not correspond to the way I apprehended this theory.
  I may not have commented clearly enough,
  or exhaustively,
  on the categorical aspects of the constructions and objects appearing there because my approach has been guided by my habits in classical differential geometry.
  I made an effort to introduce a minimum of new vocabulary or notation,
  to give the feeling that studying the geometry of a torus or of its group of diffeomorphisms,
  or the geometry of its quotient by an irrational line,
  is the same exercise,
  involving the same concepts and ideas,
  the same tools and intuition.
  I believe that the role of diffeology is to bring closer the objects involved in differential geometry,
  to treat them on an equal footing,
  respecting the ordinary intuition of the geometer.
  All in all,
  I no longer see diffeology as a replacement theory,
  but as the natural field of application of traditional differential geometry.
  But I judged,
  at the moment when I began this textbook,
  that diffeology was far enough from the main road to avoid moving too far away.
  Maybe it is not true anymore,
  and it is possible that,
  in a future revision of this book,
  I shall insist,
  or write a special chapter,
  on the categorical aspects of diffeology.
  
  \medskip
  \centerline{\sc Contents of the book}\smallskip
  
  Throughout its nine chapters, the contents of the book try to cover,
  from the point of view of diffeology,
  the main fields of differential geometry used in theoretical physics:
  differentiability,
  groups of diffeomorphisms,
  homotopy,
  homology and cohomology,
  Cartan differential calculus,
  fiber bundles,
  connections,
  and eventually some comments and constructions on what wants to be {\em symplectic diffeology}.
  
  
  Chapter 1 presents the abstract constructions and definitions related to diffeology:
  objects are diffeologies,
  or diffeological spaces,
  and morphisms are {\em smooth maps}.
  This part contains all the categorical constructions:
  sums,
  products,
  subset diffeology,
  quotient diffeology,
  functional diffeology.
  
  In Chapter 2 we discuss the local properties and related constructions,
  in particular:
  D-topology,
  generating families,
  local inductions or subductions,
  dimension map,
  modeling diffeology,
  in brief:
  everything related to local properties and constructions.
  
  In Chapters 3 and 4,
  we introduce the notion of {\em diffeological vector spaces},
  which leads to the definition of {\em diffeological manifolds}.
  Each construction is illustrated with several examples,
  not all of them coming from traditional differential geometry.
  In particular the examples of the infinite-dimensional sphere and the infinite-projective space are treated in detail.
  
  Chapter 5 describes the diffeological theory of homotopy.
  It presents the definitions of connectedness,
  Poincar\'e's groupoid and fundamental groups,
  the definition of higher homotopy groups and relative homotopy.
  The exact sequence of the relative homotopy of a pair is established.
  Everything relating to functional diffeology of iterated spaces of paths or loops finds its place in this chapter.
  
  Chapter 6 is about Cartan calculus:
  exterior differential forms and De Rham constructions,
  their generalization to the context of diffeology.
  Differential forms are defined and presented first on open subsets of real vector spaces,
  where everything is clearly explicit,
  and then carried over to diffeologies.
  Then,
  we talk about exterior derivative,
  exterior product,
  generalized Lie derivative,
  generalized Cartan formula,
  integration on chain,
  De Rham cohomology on diffeology,
  chain-homotopy operator and obstructions to exactness of differential forms.
  We shall also introduce a very useful formula for the variation of the integral of differential forms on smooth chains.
  In particular,
  the generalization of Stokes' theorem;
  the homotopic invariance of De Rham cohomology,
  and the generalized Cartan formula are established by application of this formula.
  
  Chapter 7 talks about {\em diffeological groups} and gives some constructions relative to objects associated with diffeological groups,
  for instance the space of its momenta,
  equivalence between right and left momenta, etc.
  Smooth actions of diffeological groups and natural coadjoint actions of diffeological groups on their spaces of momenta are defined.
  
  Chapter 8 presents the theory of {\em diffeological fiber bundles\/},
  defined by {\em local triviality along the plots\/} of the base space
  (not to be confused with the local triviality of topological bundles).
  It is more or less a rewriting of my thesis \cite{Igl85}.
  We define principal and associated bundles,
  and establish the exact homotopy sequence of a diffeological fiber bundle.
  The construction of the universal covering and the construction of coverings by quotient is also a part of the theory,
  as well as the generalization of the monodromy theorem in the diffeological context.
  We also see in this general framework how we can understand connections,
  reductions,
  construction of the holonomy bundle and group.
  In the same vein,
  we represent any closed $1$-form or $2$-form on a diffeological space by a special structured fiber bundle,
  a groupoid.
  
  In Chapter 9 we discuss {\em symplectic diffeology}.
  It is an attempt to generalize to diffeological spaces the usual constructions in symplectic geometry.
  This construction will use an essential tool,
  the {\em moment map},
  or more precisely its generalization in diffeology.
  %
  We have to note first that,
  if diffeology is perfectly adapted to describe covariant geometry,
  \ie,
  the geometry of differential forms,
  pullbacks etc.,
  it needs more work when it comes to dealing with contravariant objects,
  for example vectors.
  This is why it is better to introduce directly the space of momenta of a diffeological group,
  the diffeological equivalent of the dual of the Lie algebra,
  without referring to some putative Lie algebra.
  Then,
  we generalize the moment map relative to the action of a diffeological group on a diffeological space preserving a closed $2$-form.
  This generalization also extends slightly the classical moment map for manifolds.
  %
  Thanks to these constructions,
  we get the complete characterization of homogeneous diffeological spaces equipped with a closed $2$-form $\omega$.
  This theorem is an extension of the well-known Kirillov-Kostant-Souriau theorem.
  It applies to every kind of diffeological spaces,
  the ones regarded as singular by traditional differential geometry,
  as well as spaces of infinite dimensions.
  It applies to the exact/equivariant case as well as the not-exact/not-equivariant case,
  where exact here means {\em Hamiltonian}.
  In fact,
  the natural framework for these constructions is some equivariant cohomology,
  generalized to diffeology.
  This theory locates pretty well all the questions related to exactness versus nonexactness,
  equivariance versus nonequivariance,
  as well as the so-called Souriau {\em symplectic cohomology} \cite{Sou70}.
  %
  Incidentally,
  this definition of the moment map for diffeology gives a way for defining {\em symplectic diffeology},
  without considering the {\em kernel} of a $2$-form for a diffeological space,
  what can be problematic because of the contravariant nature of the kernel of a form.
  They are defined as diffeological spaces $\X$,
  equipped with a closed $2$-form $\omega$ which are homogeneous under some subgroup of the whole group of diffeomorphisms preserving $\omega$,
  and such that the moment map is a covering.
  This definition can be considered as strong,
  but it includes a lot of various situations.%
  \footnote{I have often discussed the question of symplectic diffeological spaces with J.-M. Souriau.
  The definition given here seems to be a good answer to the question,
  because this moment map and the related construction of {\em elementary spaces} include the complete case of symplectic manifolds even if the action of the group of symplectomorphisms is not Hamiltonian,
  or the orbit is not linear but affine. But the few discussions I have had with Yael Karshon about the status of {\em symplectic orbifolds} will maybe lead to a refinement of the concept of symplectic diffeological space.
  It is however too early to conclude.}
  For example every connected symplectic manifold is symplectic in this meaning.
  Some refinements are needed to deal with some nonhomogeneous singular spaces like orbifolds for example,
  but this is still in discussion.
  Many questions are still open in this new framework of symplectic diffeology.
  I discuss some of them when they appear throughut the book.
  
  \medskip\centerline{\sc On the structure of the book}\smallskip
  
  The book is made up of numbered chapters,
  each chapter is made of unnumbered sections.
  Each section is made of a series of numbered paragraphs,
  with a title which summarizes the content.
  Throughout the book,
  we refer to the numbered paragraphs as (art.\ X).
  Paragraphs may be followed by notes,
  examples,
  or a proof if the content needs one.
  This structure makes the reading of the book easy,
  one can decide to skip some proofs,
  and the title of each paragraph gives an idea about what the paragraph is about.
  Moreover,
  at the end of most of the  sections there are one or more exercises related to their content.
  These exercises are here to familiarize the reader with the specific techniques and methods introduced by diffeology.
  We are forced,
  sometimes,
  to reconsider the way we think about things and change our methods accordingly.
  The solutions of the exercises are given at the end of the book in a special chapter.
  Also,
  at the end of the book there is a list of the main notations used.
  There is now an index and a table of contents which includes the title of each paragraph,
  so it is easy to find the subject in which one is interested in,
  if it exists.
\end{chaphead}
\bigskip
