  \chapter{Diffeology and Diffeological Spaces}
  
  \label{Chapter-Diffeology-and-Diffeological-Spaces}
  \newcommand{\ChapterDDS}{Diffeology and Diffeological Spaces}
  
  \begin{chaphead}
    A {\em diffeology} on a set $\X$ declares, for all $n \in \NN$ and for all
    open sets $\U \subset \RR^n$, which maps, from $\U$ to $\X$, are {\em smooth}.
    A {\em diffeological space} is a set equipped with a diffeology.
    The elements of a diffeology are called the {\em plots} of 
    the diffeological space.
    To suit the consensual meaning of the
    word \guillemots{smooth}, the plots of a diffeological space
    satisfy three natural axioms which are the basis of the theory.
    These are the axioms of {\em covering\/}, {\em
    locality\/} and {\em smooth compatibility\/}
    \art{Diffeologies-and-diffeological-spaces}.
    
    Relationships between diffeological spaces are defined
    through {\em smooth maps}. A smooth map
    from a diffeological space to another is a map
    exchanging the plots of the spaces
    \art{Smooth-maps}. Diffeological spaces together 
    with smooth maps form the category
    $\Diffeology$, whose isomorphisms are
    called {\em diffeomorphisms}. This category generalizes
    the ordinary category $\Manifolds$ insofar as $\Manifolds$ is a full
    subcategory of $\Diffeology$. 
    But this is not its specificity, since there exist other extensions, 
    and this is not its sole interest.
    
    One of the most striking properties of the category 
    $\Diffeology$\ is its stability
    under almost all natural set-theoretic
    constructions. Every subset of a diffeological space
    carries a natural {\em subset diffeology}, induced by the ambient
    space \art{Subspaces-and-subset-diffeology}, and defined
    by the {\em pullback of diffeologies}
    \art{Pullback-of-diffeologies}. As well, every 
    quotient of a diffeological space carries a natural {\em
    quotient diffeology}
    \art{Quotient-and-quotient-diffeology}, defined by the
    {\em pushforward of diffeologies}
    \art{What-is-a-subduction}. There is also a natural
    diffeology on every  product of
    diffeological spaces
    \art{Building-products-with-spaces}, 
    and on every sum (coproduct) of diffeological spaces
    \art{Building-sums-with-spaces}.
    
    Another important diffeological construction is the {\em
    functional diffeology}, defined on the set of
    smooth maps between diffeological spaces
    \art{Functional-diffeologies}. This diffeology is a key
    for many other diffeological constructions, for example, in the
    theory of homotopy of diffeological spaces, but also in
    the definition of differential forms and bundles,
    diffeological groups etc. A {\em powerset diffeology},
    defined on the powerset $\Powerset(\X)$ of every
    diffeological space $\X$, is
    another example of a general and important construction
    (\exref{The-powerset-diffeology-of-the-set-of-lines}). 
    But it is more an attempt than a final conclusion. 
  \end{chaphead}
  
  %%%%%%%%%%%%%%%%%%%%%%%%%%%%%%%%%%%%%%%%%%%%%%%%%%%%%%%%%%
  
  \section*{Linguistic Preliminaries}
  \label{Section-Linguistic-preliminaries}
  
  \begin{sechead}
    Every theory needs a precise vocabulary, a fixed set of
    notations and conventions to be transmitted. In this
    section, we introduce the basic vocabulary and objects used in diffeology:
    finite dimensional real vector
    spaces, domains and parametrizations.
  \end{sechead}
  
  \begin{article}\artlabel{Real vector spaces and domains}
    \addcontentsline{toc}{section}{\small\hspace{10pt} Euclidean spaces and domains} 
    \label{Real-vector-spaces-and-domains}
    We denote by $\RR$ the field of real numbers. 
    We call {\em $n$-domain} every subset of $\RR^n$, 
    open for the standard topology. 
    Let us recall that a set $\U \subset \RR^n$ is open, for the standard 
    topology, if there exist a set $\cI$ of indices (possibly empty) and a family
    $\{\Ball(x_i,r_i)\}_{i \in \cI}$ of open balls in $\RR^n$,
    such that $\U = \cup_{i \in \cI} \B(x_i,r_i)$. 
    We will denote generally by 
    $$
     \Ball(x,r) = \{ x' \in \RR^n \mid \norm{x'-x} < r, 
     \mbox{ with } x \in \RR^n \mbox{ and } r>0 \},
    $$ 
    the open ball centered at $x$ with radius $r$. 
    The set of all $n$-domains will be denoted by $\Domains(\RR^n)$.
    We shall also denote by $\Domains$
    the sum of all the $\Domains(\RR^n)$, when $n$ runs over the integers.
    An element of $\Domains$ will be called a {\em real domain}, 
    or simply a {\em domain}, when there is no possible confusion. 
    But note that the real vector space of which the domain is a subset 
    is implicit.
  \end{article} %% Real-vector-spaces-and-domains
  
  \begin{article}\artlabel{Sets, subsets, maps etc.}
    \addcontentsline{toc}{section}{\small\hspace{10pt} Sets, subsets, maps etc.}
    \label{Sets-subsets-maps-etc}
    Let $\X$ be some set. As usual we call {\em subset} of
    $\X$ any set $\A$ whose elements are elements of
    $\X$, and we denote $\A \subset \X$. We denote by
    $\Powerset(\X)$ the powerset of $\X$, that is, the set of
    all the subsets of $\X$,
    $$
      \Powerset(\X) = \{ \A \mid \A \subset \X \}.
    $$
    Now, let $\A$ be a subset of $\X$, every subset $\B \subset \X$ 
    containing $\A$ is called a {\em superset} of $\A$.    
    By extension, a {\em superset of a point} $x \in \X$ is any superset of the singleton
    $\{x\}$. For any topological space $\X$, an {\em
    open superset} is a superset which is open
    for the topology of $\X$. A {\em neighborhood} of $x \in \X$ 
    is a superset of an open superset of $x$. An 
    {\em open superset} of $x$ or an {\em open neighborhood} of $x$ are the 
    same thing. 
    %
    For a map $f$, its domain of definition is denoted by $\Def(f)$, 
    and the set of its values by $\Val(f)$. 
    Thus, if $f : \X \to \Y$, $\Def(f) = \X$ and $\Val(f) = f(\X) \subset \Y$.
    The set of all maps from $\X$ to $\Y$ will be denoted by $\Maps(\X,\Y)$. 
    Also, for every set $\X$ the identity map will be denoted by $\id_\X$. 
  \end{article} %% Sets-subsets-maps-etc
    
  \begin{article}\artlabel{Parametrizations in sets}
    \addcontentsline{toc}{section}{\small\hspace{10pt} Parametrizations in sets} 
    \label{Parametrizations-in-sets}
    Let $\X$ be a non empty set, we call {\em parametrization in
    $\X$} every map $\P : \U \to \X$, where $\U$ is a domain. 
    If $\U$ is an $n$-domain, we also say that $\P$ is 
    an {\em $n$-parametrization}. 
    %
    If $\U$ is a non empty $n$-domain we say that the {\em dimension} of 
    $\P$ is $n$ and we denote $\dim(\P) = n$. 
    The empty parametrization has, by convention, dimension zero.
    %
    We shall denote by $\Param(\U,\X)$ the set of all the parametrizations 
    in $\X$ defined on $\U$,
    and by $\Param(\X)$ the set of all parametrizations in $\X$. If necessary, 
    we shall denote with a star, that is, $\Param^\star(\U,\X)$ or 
    $\Param^\star(\X)$, the subset of non empty parametrizations.
    To avoid useless discussions, let us note that the
    parametrizations in $\X$ form a well defined set.
    Note first that, for every integer $n$, for every $n$-domain $\U$,
    $\Param(\U,\X)$ is a set. Indeed, $\Param(\U,\X)$
    is a subset of the powerset $\Powerset(\U \times \X)$, 
    since maps are just some special relations. Now, let us denote
    $$
    \Param_{n}(\X) = \bigcup_{\U \in
      \Domains(\RR^{n})} \Param(\U,\X).
      $$
    But since the domains of $\RR^{n}$ are a subset of
    $\Powerset(\RR^{n})$, $\Param_{n}(\X)$ is a family of
    sets, indexed by a set. Therefore, $\Param_{n}(\X)$ is
    itself a set. Then, 
    $$
      \Param(\X) = \bigcup_{n \in \NN} \Param_{n}(\X)
      $$ 
    is a family of sets indexed by
    the integers and thus a set.
    %
    Also note that the non empty 0-parametrizations in $\X$ are
    naturally identified with the elements of $\X$. In
    other words, $\Param_0^\star(\X)\simeq \X$, 
    where $\Param_n^\star(\X)$ denotes the set of non empty 
    $n$-parametrizations in $\X$. 
    For every element $x$ in $\X$, we denote by the bold letter
    $$\bmx: \{0\} \to \X, \qmbox{with} \bmx(0) = x,$$ 
    the unique $0$-parametrization with value $x$.
    
    \Note~It happens that we write this kind of sentence: ``Let
    $\P : \U \to \X$ be a parametrization, let $\V$
    be a superset of $r \in \U$ and $\P \restriction \V$ be a
    parametrization satisfying such condition \ldots''. This implies that
    $\V$ is an open neighborhood of $r$, since a parametrization is always
    defined on a domain.
  \end{article} %% Parametrizations-in-sets
    
  \begin{article}\artlabel{Smooth parametrizations in domains}
    \addcontentsline{toc}{section}{\small\hspace{10pt} Smooth parametrizations in domains} 
    \label{Smooth-parametrizations-in-domains}
    Let us recall some basic material on which diffeology
    is founded. Let $\U$ be an $n$-domain and $\V$ be 
    an $m$-domain. Let $f : \U \to \V$ be a map. Let $f$ be
    continuous, $f$ is said to be {\em differentiable} if
    there exists a map $\D(f) : \U \to \Lin(\RR^n,\RR^m)$
    such that
    $$ 
    \mbox{for all $x \in \U$ and all   $u \in \RR^n,$}
    \quad  \D(f)(x)(u) = \lim_{\varepsilon \mathop{\rightarrow} 0} 
    {f(x + \varepsilon u) - f(x) \over \varepsilon}.
    $$ 
    The map $\D(f)$ is called {\em the derivative} of $f$.
    For all $x \in \U$, the map $\D(f)(x)$ is called the {\em
    tangent linear map} or simply the {\em tangent map} of
    $f$ at $x$. It belongs to the space $\Lin(\RR^n,\RR^m)$ of
    linear maps from $\RR^n$ to $\RR^m$. The tangent map is
    made of the partial derivatives of $f$, and is
    represented by the $n \times m$
    matrix
    $$
    \D(f)(x) = 
    \begin{pmatrix} 
    \displaystyle{\partial y_1 \over \partial x_1} & \cdots & \displaystyle{\partial y_1 \over \partial x_{n}} \\ 
    \vdots & \ddots & \vdots \\
    \displaystyle{\partial y_m \over \partial x_1}  & \cdots & \displaystyle{\partial y_m \over \partial x_n} 
    \end{pmatrix}, 
    \mbox{ where } f : x = 
    \begin{pmatrix} 
    x_1 \\
    \vdots \\
    x_n
    \end{pmatrix} 
    \mapsto y =
    \begin{pmatrix} 
    y_1 \cr \vdots \cr y_m
    \end{pmatrix}.
    $$ 
    The partial derivatives are a notation for
    $$
    {\partial y_i \over \partial x_j} = \D(x_j \mapsto f_i(x_1,\ldots,x_j,\ldots,x_n))(x_j) 
    \qmbox{where}
    y_i = f_i(x_1,\ldots,x_n).
    $$
    The tangent map $\D(f)(x)$ is denoted
    sometimes by $\D(f)_x$ or $\D f_x$ or even $df_x$. If we
    use the variable mapping notation $f : x \mapsto y$, we
    may also write $\D(x \mapsto
    y)(x)$ for $\D(f)(x)$. We may also use the
    partial derivative shortcut,
    $$
    {\partial y \over \partial x} 
    \qmbox{or}
    {\partial f(x) \over \partial x}
    \qmbox{for} 
    \D(f)(x).
    $$
    The map $f$ is said to be {\em of class} $\cC^k$ 
    if it satisfies the following conditions.
    
    \begin{itemize}
      \item[a)] $k=0$ --- $f$ is continuous.
      \item[b)] $0 < k < \infty$ --- $f$ is continuous, differentiable 
      and its derivative
      $$\D(f) : \U \to \Lin(\RR^n,\RR^m) \simeq \RR^{n \times m}$$
      %$\D(f) : \U \to \Lin(\RR^n,\RR^m) \simeq \RR^{n \times m}$,
      is of class $\cC^{k-1}$.
      \item[c)] $k = \infty$ --- $f$ is
      of class $\cC^n$ for every $n \in \NN$.
    \end{itemize}
    %
    The set of $\cC^k$ mappings from $\U$ to $\V$ is denoted by $\cC^k(\U,\V)$. 
    For $k=\infty$, we say that $f$ is {\em infinitely
    differentiable} or {\em smooth}. 
    The map $f$ is smooth if and only if all its partial derivatives are smooth.
    Let $\V$ be a domain, we call {\em smooth parametrization in} $\V$ 
    every smooth (infinitely differentiable) map $f : \U \to \V$, where $\U$ is some
    domain. We denote by $\Cinfty_n(\V)$ the set of all smooth
    $n$-parametrizations in $\V$, and sometimes by $\Cinfty_\star(\V)$ the set of all smooth
    parametrizations in $\V$.
  \end{article} %% Smooth-parametrizations-in-domains
    
  %%%%%%%%%%%%%%%%%%%%%%%%%%%%%%%%%%%%%%%%%%%%%%%%%%%%%%%%%%
  
  \section*{Axioms of Diffeology}
  \label{section-Axioms-of-diffeology}
  
  \begin{sechead}
    A {\em diffeology} on an arbitrary set is defined by declaring which 
    parametrizations in the set are smooth. 
    These parametrizations must satisfy three axioms:
    {\em covering}, {\em locality} and
    {\em smooth compatibility}. They are required to ensure the coherence 
    with the usual smooth parametrizations in real domains.
    A set equipped with a diffeology becomes then a {\em diffelogical space}.
    After the introduction of the
    axiomatics, we give a few simple examples to familiarize
    the reader with the basics of this theory. 
  \end{sechead}

  \begin{article}\artlabel{Diffeologies and diffeological spaces}
    \addcontentsline{toc}{section}{\small\hspace{10pt} Diffeologies and diffeological spaces} 
    \label{Diffeologies-and-diffeological-spaces} 
    Let $\X$ be a non empty set, a {\em diffeology} of $\X$
    is any set $\cD$ of parametrizations of $\X$, 
    such that the three following axioms are satisfied. 
    \begin{enumerate}
      \item[D1.] {\em Covering} --- The set $\cD$ contains the
      constant parametrizations $\bmx: r \mapsto x$ defined on $\RR^n$, for all 
      $x \in \X$ and all $n \in \NN$. 
      \item[D2.] {\em Locality} --- Let $\P : \U \to \X$ be a parametrization. 
      If for every point $r$ of $\U$
      there exists an open neighborhood $\V$ of $r$ such that $\P \restriction \V$
      belongs to $\cD$, then the parametrization $\P$
      belongs to $\cD$.
      \item[D3.] {\em Smooth compatibility} --- For every
      element $\P : \U \to \X$ of $\cD$, for every real domain $\V$, 
      for every $\F$ in $\Cinfty(\V, \U)$, $\P \circ \F$ belongs to $\cD$.
    \end{enumerate}
    A {\em diffeological space} is a non empty set equipped 
    with a diffeology.
    Formally, a diffeological space is
    a pair $(\X, \cD)$ where $\X$ is the {\em underlying
    set} and $\cD$ its {\em diffeology}, but 
    we shall denote the most often the diffeological space by one letter,
    $\X$ for example. Some spaces have a natural implicit diffeology
    and are just denoted by the letter denoting the underlying set, for example $\RR$, $\CC$,
    $\Cinfty(\RR)$ etc. It will be specified when it will not be the case.
  \end{article} %% Diffeologies and diffeological spaces
    
  \begin{article}\artlabel{Plots of a diffeological space}
    \addcontentsline{toc}{section}{\small\hspace{10pt} Plots of a
    diffeological space} 
    \label{Plots-of-a-diffeological-space}
    Let $\X$ be a diffeological space. The elements of the
    diffeology $\cD$ of $\X$ are called the {\em plots of the
    space $\X$\/}. In other words, to be a plot of a
    diffeological space means to be an element of its
    diffeology. 
    We shall say an {\em $n$-plot} of $\X$ for a plot defined on 
    an $n$-domain. We shall denote by
    $\cD(\U,\X)$ or $\Plots(\U,\X)$ the set of
    the plots of $\X$ defined on the domain $\U$, 
    by $\cD_n(\X)$ or $\Plots_n(\X)$ the set of the
    $n$-plots of $\X$, and sometimes by
    $\cD_\star(\X)$ or $\Plots_\star(\X)$ the set
    $\cD$ of all the plots of $\X$. 
    We call {\em global plot} of
    $\X$ every plot defined on a whole space
    $\RR^n$. The set of global $n$-plots of $\X$ is just
    $\cD(\RR^n,\X)$, it will also be denoted by
    $\Paths_n(\X)$ \art{Iterating-paths}.
  \end{article} %% Plots-of-a-diffeological-space

  \begin{article}\artlabel{Diffeology or diffeological space?}
    \addcontentsline{toc}{section}{\small\hspace{10pt} Diffeology or diffeological space?} 
    \label{Diffeology-or-diffeological-space}
    The distinction between a dif\-fe\-o\-logy, as a
    structure, and a diffeological space, as  a set equipped
    with a diffeology, is purely formal. Every diffeology
    $\cD$ of a set $\X$ contains the underlying set as the
    set of non empty $0$-plots \art{Parametrizations-in-sets}. The
    difference between a diffeological space and
    its diffeology is psychological. Sometimes it is more convenient to
    think in terms of diffeology rather than in terms of
    diffeological space, and vice versa. 
  \end{article} %% Diffeology-or-diffeological-space ?
  
  \begin{article}\artlabel{The set of diffeologies of a set}
    \addcontentsline{toc}{section}{\small\hspace{10pt} The set of diffeologies of a set} 
    \label{The-set-of-diffeologies-of-a-set} 
    Since a diffeology $\cD$ of a set $\X$ is a subset of $\Param(\X)$
    \art{Parametrizations-in-sets},
    the set of the diffeologies of a set $\X$, that is, 
    $$
    \Diffeoset(\X) = \{\cD\subset\Param(\X) \mid \mbox{$\cD$
    satisfies D1, D2 and D3} \},
    $$ 
    is a subset of the powerset of $\Param(\X)$, 
    $$
    \Diffeoset(\X)\subset\Powerset(\Param(\X)). 
    $$
  \end{article} %% The-set-of-diffeologies-of-a-set

  \begin{article}\artlabel{Real domains as diffeological spaces}
    \addcontentsline{toc}{section}{\small\hspace{10pt} Real domains as diffeological
    spaces} 
    \label{Real-domains-as-diffeological-spaces} 
    The set of all smooth parametrizations
    $\Cinfty_\star(\U)$ \art{Smooth-parametrizations-in-domains}
    in a domain $\U$ is a diffeology. It 
    will be called the {\em
    usual diffeology}, or the {\em standard diffeology}, or
    the {\em smooth diffeology} of the domain $\U$.
    The three axioms
    \art{Diffeologies-and-diffeological-spaces} are obviously
    satisfied. These are precisely the three properties of
    smooth parametrizations of real domains which have
    been chosen to define, by extension, diffeologies.
    Real domains, equipped with the standard diffeology,
    are the first and basic examples of diffeological spaces.
  \end{article} %% Real-domains-as-diffeological-spaces
  
  \begin{article}\artlabel{The \guillemots{wire diffeology}}
    \addcontentsline{toc}{section}{\small\hspace{10pt} The wire diffeology} 
    \label{The-wire-diffeology}
    This is an example of an unexpected diffeology. Let us
    consider the smooth parametrizations
    ${\P}: \U \to \RR^n$  which factorize locally through
    $\RR$, that is, for every $r$ in $\U$, there exist an
    open neighborhood $\V$ of $r$, a smooth parametrization 
    $\Q : \V \to \RR$ and a smooth $1$-parametrization $\F$ in $\RR^n$
    such that $\P \restriction \V = \F \circ \Q$. We check 
    immediately that these parametrizations form a diffeology.
    This diffeology, we call it the {\em wire diffeology}, is
    characterized by the 1-plots and does not coincide
    with the standard diffeology of $\RR^n$. Indeed, the
    identity map of $\RR^n$ is a plot for the standard
    diffeology but not a plot for the wire diffeology\footnote{This diffeology 
    has been introduced, as a significant example, by J.-M. Souriau. 
    He named it the {\em spaghetti diffeology}. 
    But because we use it afterwards, we prefer a more neutral, less culinary, name.}. We
    can imagine, of course, other diffeologies of this
    type: those defined by plots which factorize
    through $\RR^2$, $\RR^3$ etc. These
    examples are related to the construction of diffeologies
    by means of generating families, see
    \art{Generating-diffeology}.
  \end{article} %% The-wire-diffeology
  
  \begin{article}\artlabel{A diffeology for the circle}
    \addcontentsline{toc}{section}{\small\hspace{10pt} A diffeology for the circle} 
    \label{A-diffeology-for-the-circle}
    Let us consider the circle $\S^1$ as the subset of complex numbers,
    $$
    \S^1 = \{ z \in \CC \mid \bar z z =1 \}.
    $$
    The parametrizations $\P : \U \to \S^1$
    satisfying the following condition are a diffeology.
    \begin{itemize}
      \item[($\spadesuit$)] For all $r_0$ in $\U$,
      there exist an open neighborhood $\V$
      of $r_0$ and a smooth parametrization 
      $\varphi : \V \to \RR$ such that 
      $\P \restriction \V : r \mapsto \exp(2 i \pi \varphi(r))$. 
    \end{itemize}
    This diffeology is the quotient diffeology of $\RR$
    by the exponential \art{Quotient-and-quotient-diffeology}.
  \end{article} %% A-diffeology-for-the-circle
  
  \begin{proof}
    Let us check the three axioms of a diffeology
    \art{Diffeologies-and-diffeological-spaces}. Axiom D1: 
    let $z$ be a point of $\S^1$, and $\bmz : r \mapsto z$ be the
    associated global constant $n$-parametrization. Since the
    exponential is surjective from $\RR$ onto $\S^1$, 
    there exists a real number
    $\theta$ such that $\exp(2 i \pi \theta) = z$. 
    Then, for every $r_0$ we
    choose $\V = \RR^n$ and $\varphi(r) = \theta$. 
    Axiom D2: by definition, a parametrization satisfying the
    condition ($\spadesuit$) is local. Axiom D3: let $\F : \W \to \U$ 
    be a smooth parametrization.
    Let $s_0$ be a point of $\W$, let $r_0 = \F(s_0)$, let
    $\V$ and $\varphi$ as described in the condition
    ($\spadesuit$). Since $\varphi$ is a parametrization, $\V$
    is a domain. Since $\F$ is a smooth parametrization, it
    is continuous, hence $\V' = \F^{-1}(\V)$ is a domain and
    $\varphi \circ \F : \V' \to \RR$ is a smooth
    parametrization. Now, by construction, $\V'$ contains
    $s_0$. Thus, for every point $s_0$ of $\W$ there exist an 
    open neighborhood $\V'$ of $s_0$ and a smooth parametrization
    $\varphi' = \varphi \circ \F : \V' \to \RR$
    such that 
    $(\P \circ \F) \restriction \V' : s \mapsto \exp(2 i \pi \varphi'(s))$. 
    Therefore, the parametrization
    $\P \circ \F$ satisfies the condition ($\spadesuit$).
  \end{proof}
  
  \begin{article}\artlabel{A diffeology for the square}
    \addcontentsline{toc}{section}{\small\hspace{10pt} A diffeology for the square} 
    \label{A-diffeology-for-the-square}
    Let us consider the square 
    $\Sq = [0,1] \times \{0,1\} \cup \{0,1\} \times [0,1] \subset \RR \times \RR$. 
    The set of the parametrizations of the square
    which, regarded as parametrizations of $\RR \times \RR$,
    are smooth, is a diffeology. This diffeology is the subset diffeology,
    inherited from $\RR^2$ \art{Subspaces-and-subset-diffeology}.
  \end{article} %% A-diffeology-for-the-square
  
  \begin{proof}
    Let us check the three axioms of a diffeology
    \art{Diffeologies-and-diffeological-spaces}. Axiom D1: every constant 
    parametrization, regarded as a
    parametrization in $\RR \times \RR$, is smooth. Axiom D2: a parametrization 
    in the square which, regarded as
    a parametrization in $\RR \times \RR$, is locally
    smooth at each point of its domain, is smooth. Axiom D3: the composite 
    of a plot of the square with any
    smooth parametrization in the source of the plot does not
    change its set of values and, regarded as a
    parametrization in $\RR \times \RR$, is smooth.
  \end{proof}
  
  \begin{article}\artlabel{A diffeology for the sets of smooth maps}
    \addcontentsline{toc}{section}{\small\hspace{10pt} A diffeology for the sets 
    of smooth maps} 
    \label{A-diffeology-for-the-sets-of-smooth-maps}
    Let $\A$ and $\B$ be two, non empty, real domains. Let us consider the set
    $\Cinfty(\A,\B)$ of all the smooth maps from $\A$ to
    $\B$. The set $\cD$ of parametrizations $\P : \U \to \Cinfty(\A,\B)$ 
    satisfying the following condition is a diffeology. 
    \begin{itemize}
      \item[($\diamondsuit$)] $\bmP : (r,s) \mapsto \P(r)(s) \in \Cinfty(\U \times \A,\B)$. 
    \end{itemize}
    This diffeology is the functional diffeology of
    $\Cinfty(\A,\B)$ \art{Functional-diffeologies}.
  \end{article} %% A-diffeology-for-the-sets-of-smooth-maps
  
  \begin{proof}
    First of all, note that $\U \times \A$ is a
    domain. Thus, it makes sense to consider the set of
    smooth parametrizations $\Cinfty(\U \times \A,{\B})$.
    Let us check now the three axioms of a diffeology
    \art{Diffeologies-and-diffeological-spaces}. Axiom D1: every constant 
    parametrization ${\P}: r \mapsto f$,
    with $f \in \Cinfty(\A,{\B})$, satisfies
    $$
    \bmP = [(r,s) \mapsto {\P}(r)(s) = f(s)] = f \circ \pr_2 \in
    \Cinfty(\U \times \A,{\B}), 
    $$ 
    where $\pr_2 : \U \times
    \A \to \A$ is the projection onto the second factor.
    Then, $\bmP$ is the composite of two smooth maps,
    therefore $\bmP$ is smooth. Axiom D2: let us consider a parametrization 
    $\P : \U \to \Cinfty(\A,{\B})$ such that for all $r_0 \in \U$
    there exists an open neighborhood $\V$ of $r_0$ such that 
    $\P \restriction \V$ is a parametrization in $\Cinfty(\A,\B)$
    satisfying ($\diamondsuit$). Since $\P \restriction \V$
    is a parametrization, $\V$ is a domain and 
    $\bmP \restriction \V \times \A$ belongs to 
    $\Cinfty(\V \times \A,\B)$. Hence, for all $(r_0,s_0)$ in 
    $\U \times \A$ there exists an open neighborhood $\W = \V \times \A$
    such that $\bmP \restriction \W$ is a smooth
    parametrization in $\B$. Therefore $\bmP$ is a smooth
    parametrization in $\B$, that is, satisfies ($\diamondsuit$).
    Axiom D3: let $\P : \U \to \Cinfty(\A,{\B})$ be a parametrization 
    satisfying the condition ($\diamondsuit$) and $\F: \V
    \to \U$ be a smooth parametrization. Let $\P' = \P \circ \F$, the
    parametrization $\bmP' = [(t,s) \mapsto
    \P(\F(t),s)] = [(t,s) \mapsto (r = \F(t),s) \mapsto \P(r)(s)]$, 
    as the composite of two smooth
    parametrizations, is a smooth parametrization. Thus,
    $\P \circ \F$ satisfies the condition ($\diamondsuit$).
    Therefore, $\cD$ is a diffeology of $\Cinfty(\A,\B)$.
  \end{proof}
  
  %%%%%%%%%%%%%%%%%%%%%%%%%%%%%%%%%%%%%%%%%%%%%%%%%%%%%%%%%%
  %
  %   Exercises  
  %
  %%%%%%%%%%%%%%%%%%%%%%%%%%%%%%%%%%%%%%%%%%%%%%%%%%%%%%%%%%
  
  \Exercises
  
  \begin{exercise}[Equivalent axiom of covering] 
    \label{Equivalent-axiom-of-covering}
    Check that the first axiom of diffeology
    \art{Diffeologies-and-diffeological-spaces} can be replaced by the
    following.
    \alinea{D1'.} The values of the elements of $\cD$ cover $\X$, that is,
      $\bigcup_{\P \in \cD} \Val(\P) = \X.$
  \end{exercise} %% Equivalent-axiom-of-covering
  
  \begin{exercise}[Equivalent axiom of locality]
    \label{Equivalent-axiom-of-locality}
    Let $\X$ be a set, and $\cD \subset \Param(\X)$.  Show
    that the axiom D2 of diffeology can be replaced by the following.
    \alinea{D2'.} For all integers $n$, for all families $\{\P_i :
      \U_i \to \X\}_{i \in \cI}$ of
      $n$-parametriza\-tions such that 
      $r \in \U_i \cap \U_j$ implies $\P_i(r) = \P_j(r)$,
      if all the $\P_i$ belong to $\cD$, then the following
      parametrization $\P$ belongs to $\cD$,
      $$ 
      \P: \U = \cup_{i \in \cI} \U_i \to \X,
      \qmbox{with} {\P}(r) = {\P_i}(r) \qmbox{if} r \in \U_i.
      $$ 
      
    \Note~The family $\{\P_i : \U_i \to \X\}_{i \in \cI}$ 
    is said to be a {\em compatible family} of parametrizations,
    and $\P$ is called the {\em supremum} of the family. The axiom
    D2' expresses itself this way: {\em the supremum of a compatible family
    of elements of $\cD$ belongs to $\cD$}.
  \end{exercise} %% Equivalent-axiom-of-locality
  
  \begin{exercise}[Global plots and diffeology]
    \label{Global-plots-and-diffeology}
    Let $\X$ be a set, $\cD$ and $\cD'$ be two diffeologies of $\X$.
    Show that if for every integer $n$, $\cD(\RR^n, \X) =
    \cD'(\RR^n,\X)$ \art{Plots-of-a-diffeological-space}, then $\cD$ and $\cD'$
    coincide. In other words, prove that the global plots characterize
    the diffeology. 
  \end{exercise} %% Global-plots-and-diffeology
    
  %%%%%%%%%%%%%%%%%%%%%%%%%%%%%%%%%%%%%%%%%%%%%%%%%%%%%%%%%%
  
  \section*{Smooth Maps and the Category Diffeology}
  \label{section-Smooth-maps-and-category-tDiffeology}
  
  \begin{sechead}
    We consider now the notion of {\em
    smooth maps} between
    diffeological spaces. They are maps which transform, by
    composition, the plots of the source into plots of the
    target. Diffeological spaces, together with
    smooth maps, define a category denoted by
    $\Diffeology$. The isomorphisms of this category are
    called {\em diffeomorphisms}. 
  \end{sechead} %% Header
  
  \begin{article}\artlabel{Smooth maps}
    \addcontentsline{toc}{section}{\small\hspace{10pt} Smooth maps} 
    \label{Smooth-maps} 
    Let $\X$ and $\X'$ be two diffeological spaces. A map
    $f : \X \to \X'$ is said to be {\em smooth} if for
    each plot ${\P}$ of $\X$, $f \circ {\P}$ is a plot of
    $\X'$ \fig{fig-smooth-map}. 
    \begin{figure}[tb]
      \centerline{\includegraphics{Figures-PDF/fig-smooth-map.pdf}}
      \caption{A smooth map $f \in \cD(\X,\X')$.} 
      \label{fig-smooth-map}
    \end{figure}
    The set of smooth maps from $\X$ to $\X'$ is denoted by
    $\cD(\X,\X')$. In other words, denoting by $\cD$ and
    $\cD'$  the diffeologies of $\X$ and $\X'$, 
    $$
    \cD(\X,\X') = \{ f \in \Maps(\X,\X') \mid  f \circ \cD
    \subset \cD'\}. 
    $$
  \end{article} %% Smooth-maps
  
  \begin{article}\artlabel{Composition of smooth maps}
    \addcontentsline{toc}{section}{\small\hspace{10pt} Composition of smooth
    maps}  \label{composition-of-smooth-maps} 
    The composition of two smooth maps is a smooth map.
    Diffeological spaces together with the smooth maps
    define a category denoted by
    $\Diffeology$.
  \end{article} %% composition-of-smooth-maps
  
  \begin{proof}
    The fact that the composition of smooth maps is smooth
    results directly from the associativity of the
    composition of maps. Let $\X$, $\X'$ and $\X''$ be three
    diffeological spaces whose diffeologies are denoted by
    $\cD$, $\cD'$ and $\cD''$. Let $f: \X \to \X'$ and
    $g: \X' \to \X''$ be two smooth maps, that is, 
    $f \circ \cD\subset\cD'$ and $g \circ \cD'\subset
    \cD''$. Then, $(g \circ f) \circ \cD = g \circ f \circ
    \cD \subset g \circ \cD' \subset\cD''$. Therefore, $g
    \circ f$ is smooth.  
  \end{proof}
  
  \begin{article}\artlabel{Plots are smooth}
    \addcontentsline{toc}{section}{\small\hspace{10pt} Plots are smooth} 
    \label{Plots-are-smooth} 
    Let $\X$ be a diffeological space and $\U$ be a non empty real domain
    \art{Real-domains-as-diffeological-spaces}. The
    smooth maps from $\U$ to $\X$ are exactly the
    plots of $\X$ defined on $\U$, and only them. This
    property, which writes $\cD(\U,\X) = \cD(\U)$
    \art{Plots-of-a-diffeological-space}, is the very
    principle of diffeology. The plots of a space $\X$ are
    the parametrizations of $\X$ we want to regard as
    smooth.
    Now, let $\X = \V$ be another real domain, then
    $\cD(\U,\V) = \Cinfty(\U,\V)$. The smooth maps from
    $\U$ to $\V$, from the diffeological point of view, are
    exactly the usual smooth maps. This justifies the
    equivalent notation $\Cinfty(\U,\X)$ for
    $\cD(\U,\X)$.  By extension, we also denote,
    for every pair $\X$
    and $\X'$ of diffeological spaces, 
    $$\Cinfty(\X,\X') = \cD(\X, \X').$$
  \end{article} %% Plots-are-smooth
  
  \begin{proof}
    Let $f : \U \to \X$ be a smooth parametrization. Thanks
    to the axiom D3
    \art{Diffeologies-and-diffeological-spaces}, the
    composite $f \circ \id_\U$, where $\id_\U$ is the
    identity of $\U$, is a plot of $\X$. Thus $\cD(\U,\X)
    \subset \cD(\U)$. Now, let $\P : \U \to \X$ be a plot.
    Thanks again to the axiom D3, for every smooth
    parametrization $\F$ in $\U$, the parametrization $\P
    \circ \F$ belongs to $\cD$. Hence, $\P$ is smooth
    \art{Smooth-maps}. Thus, $\cD(\U) \subset
    \cD(\U,\X)$. Therefore, $\cD(\U) = \cD(\U,\X)$.
  \end{proof}
  
  \begin{article}\artlabel{Diffeomorphisms}
    \addcontentsline{toc}{section}{\small\hspace{10pt} Diffeomorphisms} 
    \label{Diffeomorphisms}
    Let $\X$ and $\X'$ be two diffeological spaces. 
    A map $f : \X \to \X'$ is called a {\em diffeomorphism} if
    $f$ is bijective and if both $f$ and $f^{-1}$ are
    smooth. Diffeomorphisms are the isomorphisms of the
    category $\Diffeology$. The set of diffeomorphisms from $\X$ to $\X'$ is
    denoted by $\Diff(\X,\X')$. Note that for $\X = \X'$ we denote by $\Diff(\X)$, 
    instead of $\Diff(\X,\X)$, the diffeomorphisms of $\X$. It is a group for the composition.
  \end{article} %% Diffeomorphisms
  
  \begin{proof}
    By definition, the isomorphisms of the category
    $\Diffeology$ are the invertible morphisms, that is,
    invertible smooth maps whose inverses also are smooth.
    This is what we called diffeomorphisms.
  \end{proof}
  
  %%%%%%%%%%%%%%%%%%%%%%%%%%%%%%%%%%%%%%%%%%%%%%%%%%%%%%%%%%
  %
  %   Exercises  
  %
  %%%%%%%%%%%%%%%%%%%%%%%%%%%%%%%%%%%%%%%%%%%%%%%%%%%%%%%%%%
  
  \Exercises
  
  \begin{exercise}[Diffeomorphisms between irrational tori]  
    \label{Diffeomorphisms-between-irrational-tori}
    Let $\alpha$ be some irrational number, $\alpha \in \RR
    - \QQ$. Let $\Torus_\alpha$ be the quotient set $\RR/(\ZZ +
    \alpha \ZZ)$, that is, the quotient of $\RR$ by the
    equivalence relation: $x \sim x'$ if and only if there
    exist $n, m \in \ZZ$ such that $x' = x + n + \alpha m$.
    The set $\Torus_\alpha$ is an Abelian group. We say that
    $\Torus_\alpha$ is an {\em irrational torus}. 
    %
    Let $\pi_\alpha : \RR \to \Torus_\alpha$ be the canonical
    projection. Let $\cD$ be the set of parametrizations $\P
    : \U \to \Torus_\alpha$ such that 
    \begin{itemize}
      \item[(\blacksnow)] for all $r_0 \in \U$ there exist an open neighborhood $\V$
      of $r_0$ and a smooth parametrization $\Q : \V \to \RR$ such that
      $\pi_\alpha \circ \Q = \P \restriction \V$.
    \end{itemize}
    
    \Question{1)} Check that $\cD$ is a diffeology of
    $\Torus_\alpha$, see \art{A-diffeology-for-the-circle}. 
    
    \Question{2)} Check that $\Cinfty(\Torus_\alpha,\RR)$ is reduced to the constants,
    that is, $\Cinfty(\Torus_\alpha,\RR) \simeq \RR$.
    
    \Question{3)} Let $\alpha$ and $\beta$ be two irrational numbers.
    Let $f : \Torus_\alpha \to \T_\beta$ be a smooth
    map. Show that there exist an interval $\cJ$ of $\RR$
    and some affine map $\F : \cJ \to \RR$, such that
    $\pi_\beta \circ \F = f \circ \pi_\alpha \restriction
    \cJ$, use the fact that $\ZZ + \alpha \ZZ$ is dense in
    $\RR$. Then, show that $\F$ can be extended to the whole
    $\RR$ in an affine map. Deduce that
    $\Cinfty(\Torus_\alpha,\T_\beta)$ does not reduce to the
    constant maps if and only if there exist four
    integers $a$, $b$, $c$ and $d$ such that 
    $$
     \alpha = {a + \beta b \over c + \beta d} \,\cdot
    $$
    
    \Question{4)} Show that $\Torus_\alpha$ and $\T_\beta$
    are diffeomorphic if and only if $\alpha$ and $\beta$ are {\em
    conjugate\/} modulo $\GL(2,\ZZ)$, that is, the four integers $a, b,
    c, d$ of the question 1) satisfy $ad - bc = \pm 1$. 
  \end{exercise} %% Diffeomorphisms-between-irrational-tori
  
  \begin{exercise}[Smooth maps on $\RR/\QQ$] 
    \label{Smooth-maps-on-R/Q}
    Let $\E_\QQ$ be the quotient of $\RR$ by $\QQ$.
    Let $\pi : \RR \to \E_\QQ$ be the projection. Since the set $\RR$ is
    a vector space over $\QQ$, and $\QQ$ is a $\QQ$-vector subspace of
    $\RR$, the quotient $\E_\QQ$ inherits the structure of quotient 
    vector space over $\QQ$\,: 
    \begin{itemize} \item  for all
      $\tau, \tau' \in \E_\QQ$, if $\tau = \pi(x)$
      and $\tau' = \pi(x')$, then 
      $\tau + \tau' = \pi(x)  + \pi(x') = \pi(x + x')$,  
      \item for all $q \in \QQ$ and all $\tau \in \E_\QQ$, 
      if $\tau = \pi(x)$ then $q \cdot \tau = \pi(q x)$. 
    \end{itemize}
    Let us equip the quotient vector space $\E_\QQ$ with the diffeology
    defined, {\em mutatis mutandis}, by the condition (\blacksnow) of 
    \exref{Diffeomorphisms-between-irrational-tori}.
    
    \Question{1)} Show that the only smooth maps from $\E_\QQ$ to
    $\E_\QQ$ are the $\QQ$-affine maps:
    $$
    \Cinfty(\E_\QQ) = \{ \tau \mapsto q \cdot \tau + \tau' \mid q \in \QQ
    \mbox{ and } \tau' \in \E_\QQ \}. 
    $$
    
    \Question{2)} Show that the only smooth maps from any irrational
    torus $\Torus_\alpha$ (\exref{Diffeomorphisms-between-irrational-tori})
    to $\E_\QQ$ are the constant maps, $\Cinfty( \Torus_\alpha, \E_\QQ) =
    \E_\QQ$. 
    
    \Question{3)} Also show that $\Cinfty( \E_\QQ, \RR) = \RR$ and 
    $\Cinfty( \E_\QQ, \Torus_\alpha ) = \Torus_\alpha$. 
  \end{exercise} %% Smooth-maps-on-R/Q
  
  \begin{exercise}[Smooth maps on spaces of maps] 
    \label{Smooth-maps-on-spaces-of-maps}
    Let  $\Cinfty(\RR)= \Cinfty(\RR,\RR)$, equipped with the functional 
    diffeology defined in \art{A-diffeology-for-the-sets-of-smooth-maps}.
    
    \Question{1)} Let $f$ be an element of $\Cinfty(\RR)$, and let $f^{(k)}$ denote
    its $k$-th derivative. Show that, for every integer $k$, the following 
    map is smooth. 
    $$
    { d^k \over dx^k} : \Cinfty(\RR) \to \Cinfty(\RR) 
    \qmbox{defined by} { d^k \over dx^k}(f) = f^{(k)}. 
    $$ 
    
    \Question{2)} Show that, for every real number $x$, the map 
    $\hat x : f \mapsto f(x)$, from $\Cinfty(\RR)$ to $\RR$, is smooth.
    Deduce that, for every real number $x$, for every integer
    $k$, the following map, called the {\em $k$-jet} or the 
    {\em jet of order $k$}, is smooth.
    $$ 
    \D_x^k : \Cinfty(\RR) \to \RR^{k+1}
    \qmbox{defined by} \D_x^k(f) = (f(x),f'(x),\ldots,f^{(k)}(x)). 
    $$ 
    
    \Question{3)} Show that, for any pair of real numbers $a$ and $b$, the following
    map $\I_{a,b}$ is smooth, where the sign $\int$ denotes the Riemann integral.
    $$
    \I_{a,b} : \Cinfty(\RR) \to \RR \qmbox{with} \I_{a,b}(f) =
    \int_a^b f(t) \dt.
    $$
    
    \Question{4)} Let $\Cinfty_0(\RR)$ be the space of smooth real maps $f$ such that
    $f(0)=0$. Check that the parametrizations of $\Cinfty_0(\RR)$
    satisfying the condition ($\diamondsuit$) of
    \art{A-diffeology-for-the-sets-of-smooth-maps} are still a diffeology.
    Then, using the question 3, deduce that the derivative 
    $d/dx : f \mapsto f'$, defined from $\Cinfty_0(\RR)$ to $\Cinfty(\RR)$, is a
    diffeomorphism.
  \end{exercise} %% Smooth-maps-on-spaces-of-maps

  %%%%%%%%%%%%%%%%%%%%%%%%%%%%%%%%%%%%%%%%%%%%%%%%%%%%%%%%%%
  
  \section*{Comparing Diffeologies}
  \label{section-Comparing-diffeologies}
  
  \begin{sechead}
    The inclusion of diffeologies, regarded as subsets of the powerset of
    all the parametrizations of a set, is a partial ordering. This relation,
    called {\em fineness}, is a key for the categorical constructions of
    the theory.
  \end{sechead} %% Header
  
  \begin{article}\artlabel{Fineness of diffeologies}
    \addcontentsline{toc}{section}{\small\hspace{10pt} Fineness of diffeologies} 
    \label{Fineness-of-diffeologies}
    Let $\X$ be a set, a diffeology $\cD$ of $\X$
    is said to be {\em finer\/} than another $\cD'$ if 
    $\cD\subset \cD'$.
    This relation, called {\em fineness\/}, is a partial order. We say
    equivalently that $\cD$ is finer than $\cD'$ or $\cD'$ {\em coarser\/}
    than $\cD$. In other words, $\cD$ is finer than $\cD'$ if $\cD$ has
    less elements than $\cD'$. If $\X'$ denotes the set $\X$ equipped with
    the diffeology $\cD'$ we also denote $\X \preceq \X'$ to mean that
    $\cD$ is finer than $\cD'$.
  \end{article} %% Fineness-of-diffeologies
  
  \begin{article}\artlabel{Fineness via the identity map}
    \addcontentsline{toc}{section}{\small\hspace{10pt} Fineness via the identity map}  
    \label{fineness-via-the-identity-map} 
    Let us denote by $\X_1$ and $\X_2$ the same set
    $\X$ equipped with two diffeologies $\cD_1$ and
    $\cD_2$. The diffeology $\cD_1$ is finer than $\cD_2$ if
    and only if the identity map $\id_\X: \X_1 \to
    \X_2$ is smooth.
  \end{article} %% fineness-via-the-identity-map
  
  \clearpage 
  
  \begin{article}\artlabel{Discrete diffeology}
    \addcontentsline{toc}{section}{\small\hspace{10pt} Discrete diffeology} 
    \label{Discrete-diffeology}
    Let $\X$ be a set. The {\em locally constant
    parametrizations\/} of $\X$ are defined as follows.
    \begin{itemize}
      \item[($\clubsuit$)] A parametrization $\P : \U \to \X$ 
      is said to be {\em locally constant} if for all $r \in \U$ 
      there exists an open neighborhood $\V$ of $r$ such that
      ${\P}\restriction \V$ is constant. 
    \end{itemize}
    The locally constant parametrizations in $\X$ form a
    diffeology called the {\em discrete diffeology\/}. The
    set $\X$, equipped with the discrete diffeology, will be
    denoted by $\discrete{\X}$, and the discrete diffeology
    itself by $\discrete{\cD}(\X)$. A set equipped with the
    discrete diffeology will be called a {\em discrete
    diffeological space\/}.
    
    \Note For every set $\X$, the discrete diffeology is the finest diffeology of
    $\X$, that is, finer than any other diffeology of $\X$. Every diffeology
    contains the discrete diffeology.
    Moreover, let $\X$ and $\X'$ be any two sets equipped
    with the discrete diffeology. Every map $f$ from $\X$ to
    $\X'$ is smooth, that is, $\Cinfty(\discrete{\X},
    \discrete{\X'}) = \Maps(\X,\X')$. In other words, the
    correspondence $\discretefunctor$ defined from the
    category $\Set$ into the category $\Diffeology$ by
    $\discretefunctor(\X) = \discrete{\X}$, and
    $\discretefunctor(f) = f$, is a full faithful functor
    called the {\em discrete functor}. 
  \end{article} %% Discrete-diffeology
  
  \begin{proof}
    The axioms of diffeology are satisfied by the locally constant
    parametrizations. For the covering axiom: every constant
    parametrization is locally constant. For the locality
    axiom: it is satisfied by the very definition of locally
    constant parametrizations. For the smooth compatibility:
    the composite of a locally constant parametrization by
    a smooth parametrization in its domain
    is again locally constant.
    %
    Now, let $\cD$ be any other diffeology of $\X$. Let $n$ be any integer,
    and $\P : \U \to \X$ be a locally constant $n$-parametrization. Let 
    $r \in \U$ and $x = \P(r)$. Let $\bmx : \RR^n \to \X$ be the constant
    $n$-parametrization with value $x$. By the axiom D1 of diffeology, the
    parametrization $\bmx$ is a plot for the diffeology $\cD$. Since $\P$ is
    locally constant, there exists an open neighborhood $\V$ of $r$ such that 
    $\P \restriction \V$ is a constant parametrization. Let us denote by 
    $j_\V : \V \to \RR^n$ the natural inclusion, it is a smooth parametrization
    in $\RR^n$. By axiom D2, 
    $\bmx \circ j_\V = \P \restriction \V$ belongs to $\cD$. Hence,
    $\P$ belongs locally to $\cD$ everywhere. Thus, by axiom D3, $\P$ belongs
    to $\cD$. Therefore, the discrete diffeology
    $\discrete{\cD}(\X)$ is contained in every diffeology
    $\cD$ of $\X$. It is the finest diffeology of $\X$.
    %
    Next, consider a map $f : \X \to \X'$ where $\X$
    and $\X'$ are equipped with the discrete diffeology. 
    Let us recall that $f$ is said to be smooth if the
    composite of $f$ with any plot of $\X$ is  a plot of
    $\X'$ \art{Smooth-maps}. The composite of any
    map with any locally constant parametrization is locally
    constant. Indeed, let $\P : \U \to \X$ be a locally
    constant parametrization. Let $r \in \U$ and $x = \P(r)$.
    Let $\V$ be an open neighborhood of $r$ such that $\P \restriction \V$
    is the constant parametrization defined on $\V$ with
    value $x$. Hence, $f \circ \P \restriction \V$ is the
    constant parametrization in $\X'$, defined on $\V$, with
    value $f(x)$. Therefore $f \circ \P$ is locally constant
    and $f \in \Cinfty(\discrete{\X}, \discrete{\X'})$.
  \end{proof}
  
  \begin{article}\artlabel{Coarse diffeology}
    \addcontentsline{toc}{section}{\small\hspace{10pt} Coarse diffeology} 
    \label{Coarse-diffeology}
    Let $\X$ be a set. The set of all the parametrizations
    in $\X$ is a diffeology. This diffeology, coarser than every
    other one, is called the {\em coarse diffeology}. The set
    $\X$ equipped with the coarse diffeology will be denoted
    by $\coarse{\X}$, and the coarse diffeology itself by
    $\coarse{\cD}(\X)$, that is, $\coarse{\cD}(\X) =
    \Param(\X)$. A set equipped with the coarse diffeology
    will be called a {\em coarse diffeological space\/}.
    Moreover, let $\X$ and $\X'$ be two sets equipped
    with the coarse diffeology. Every map $f$ from $\X$ to
    $\X'$ is smooth, that is, $\Cinfty(\coarse{\X},
    \coarse{\X'}) = \Maps(\X,\X')$. In other words, the
    correspondence $\coarsefunctor$ from the
    category $\Set$ to the category $\Diffeology$ by
    $\coarsefunctor(\X) = \coarse{\X}$ and
    $\coarsefunctor(f) = f$, is a full faithful functor
    called the {\em coarse functor}. 
    
    \Note Combining the existence of the coarse diffeology on every set $\X$
    with the existence of the discrete diffeology
    \art{Discrete-diffeology}, we conclude that every diffeology $\cD$ of
    $\X$ is somewhere in between,  
    $$
    \discrete{\cD}(\X) \subset \cD \subset \coarse{\cD}(\X).
    $$
  \end{article} %% Coarse-diffeology
  
  \begin{proof}
    For the set $\Param(\X)$ the two first axioms of
    diffeology are obviously satisfied. For the third one,
    just remark that the composite of a parametrization in
    $\X$ with a smooth parametrization in its source is still a
    parametrization, since to be a parametrization is just to
    be defined on a real domain.  
    Now, the composite of any map $f : \X \to
    \X'$ with any parametrization in $\X$ is a parametrization
    in $\X'$, for the same reason as above. Hence, every map
    $f : \X \to \X'$ belongs to $\Cinfty(\coarse{\X}, \coarse{\X'})$, 
    therefore $\Cinfty(\coarse{\X}, \coarse{\X'}) = \Maps(\X,\X')$.
  \end{proof}
  
  \begin{article}\artlabel{Intersecting diffeologies}
    \addcontentsline{toc}{section}{\small\hspace{10pt} Intersecting diffeologies} 
    \label{Intersecting-diffeologies} 
    Let $\X$ be any set. Let
    $\DD = \{\cD_i\}_{i \in {\cI}}$ be any family of diffeologies of
    $\X$, indexed by some set ${\cI}$. The intersection of the
    elements of the family, denoted by $$\cD =  \bigcap_{i \in {\cI}}
    \cD_i,$$ is still a diffeology of $\X$, finer than each
    element of the family $\DD$. Note that we may also denote just by
    $\cap \DD$ the intersection $\cap_{i \in {\cI}} \cD_i$.
  \end{article} %% Intersecting-diffeologies
  
  \begin{proof}
    Let us first check the three axioms of diffeology
    \art{Diffeologies-and-diffeological-spaces}.
    Axiom D1: since every diffeology contains
    the constant parametrizations, they are contained in each element
    $\cD_i$ of $\DD$. Hence, the constant parametrizations also are 
    contained in the intersection $\cap_{i \in {\cI}}\cD_i$. 
    Axiom D2: let $ \P : \U \to \X$ be a parametrization in
    $\X$ such that for every point $r$ in $\U$ there exists
    an open neighborhood $\V$ of $r$ for which $\P \restriction \V$ 
    is a parametrization belonging to the intersection
    $\cap_{i \in {\cI}}\cD_i$, that is, for each $i$ in
    ${\cI}$, $\P \restriction \V \in \cD_i$. Thus, by
    application of the axiom D2 to the diffeology $\cD_i$,
    $\P$ is a plot for $\cD_i$. Therefore, for all $i \in
    \cI$, $\P \in \cD_i$, that is, $\P \in \cap_{i \in cI}\cD_i$.
    Axiom D3: let ${\P} \in
    \cap_{i \in {\cI}}\cD_i$ and let $\F$ be a smooth
    parametrization in the domain of ${\P}$.
    For each $i$ in ${\cI}$, ${\P} \in \cD_i$. Since each
    $\cD_i$ is a diffeology: ${\P} \circ \F$ belongs to
    $\cD_i$, hence ${\P} \circ \F$ belongs to the
    intersection $\cap_{i \in {\cI}}\cD_i$.
    Therefore, $\cD = \cap_{i \in {\cI}}\cD_i$ is a diffeology of $\X$. By
    construction, this diffeology is contained in every $\cD_i$, that is,
    $\cD$ is finer than every element of the family $\DD$.
  \end{proof}
  
  \begin{article}\artlabel{Infimum of a family of diffeologies}
    \addcontentsline{toc}{section}{\small\hspace{10pt} Infimum of a family of diffeologies} 
    \label{infimum-of-a-family-of-diffeologies} 
    Let $\X$ be a set and $\DD$ be some family of diffeologies of $\X$. 
    The family $\DD$ has an infimum for the fineness partial ordering
    \art{Fineness-of-diffeologies}, this is the intersection of the elements of the family
    $\DD$ \art{Intersecting-diffeologies}. 
    $$
    \inf(\DD) = \bigcap_{\cD \in \DD} \cD = \{ \P \in
    \Param(\X) \mid \mbox{For all } \cD \in \DD, \P \in \cD\}.
    $$
    It is the coarsest diffeology contained in every element of the family $\DD$,
  \end{article} %% infimum-of-a-family-of-diffeologies
  
  \begin{proof}
    Let us recall that the infimum of $\DD$, 
    if it exists, is the greatest lower bound of $\DD$. 
    A diffeology $\cD'$ is 
    a lower bound of $\DD$ if and only if $\cD' \subset \cD$ for all $\cD \in \DD$.
    The set of lower bounds of $\DD$ is not empty, 
    since it contains the discrete diffeology \art{Coarse-diffeology}. 
    Since $\cap \DD = \cap_{\cD \in \DD} \cD$ is a diffeology 
    \art{Intersecting-diffeologies} and since
    for all $\cD' \in \DD$, $\cap_{\cD \in \DD} \cD \subset \cD'$, 
    $\cap \DD$ is a lower bound of $\DD$. Now, let $\cD'$ be a lower bound of 
    $\DD$, \ie\/, $\cD' \subset \cD$ for all $\cD \in \DD$. Thus, 
    $\cD' \subset \cap_{\cD \in \DD} \DD = \cap \DD$. 
    Therefore, $\cap \DD$ is the greatest lower bound of $\DD$, 
    that is, its infimum. 
  \end{proof}
  
  \begin{article}\artlabel{Supremum of a family of diffeologies}
    \addcontentsline{toc}{section}{\small\hspace{10pt} Supremum of a family of diffeologies}
    \label{Supremum-of-a-family-of-diffeologies} 
    Let $\X$ be a set and $\DD$ be some family of diffeologies of
    $\X$. The family $\DD$ has a supremum for the fineness partial ordering
    \art{Fineness-of-diffeologies}, this is the infimum of the 
    diffeologies of $\X$ containing
    all the elements of $\DD$ \art{infimum-of-a-family-of-diffeologies}. 
    $$
    \sup(\DD) = \inf\{\cD' \in \Diffeoset(\X) \mid \mbox{For
    all } {\cD} \in \DD, \cD \subset \cD'\}.
    $$
    It is the finest diffeology 
    containing every element of the family $\DD$,
  \end{article} %% Supremum-of-a-family-of-diffeologies
  
  \begin{proof}
    Let us recall that the supremum of $\DD$, 
    if it exists, is the smallest upper bound of $\DD$. 
    A diffeology $\cD'$ is an upper bound of $\DD$ if and only if, 
    for all $\cD \in \DD$,
    $\cD \subset \cD'$. The set of the upper bounds of $\DD$ is not empty, since
    it contains the coarsest diffeology \art{Coarse-diffeology}. 
    Let us consider the intersection of all the upper bounds of 
    $\DD$. We know that it is a diffeology \art{Intersecting-diffeologies}. 
    Next, since every $\cD \in \DD$ is contained in every upper bound, it is also
    contained in their intersection. Thus, this intersection is an upper bound of 
    $\DD$. Then, since the intersection of the upper bounds is the infimum 
    of the family \art{infimum-of-a-family-of-diffeologies}, the intersection
    of all the upper bounds of the family $\DD$ is its supremum.
  \end{proof}
  
  \begin{article}\artlabel{Playing with bounds}
    \addcontentsline{toc}{section}{\small\hspace{10pt} Playing with bounds}
    \label{playing-with-bounds}
    It's a nice property that any family of diffeologies of a set
    $\X$ has an infimum and a supremum. This property has
    even got a name by Bourbaki, the set $\Diffeoset(\X)$
    is said to be \guillemots{r\'eticul\'e achev\'e} \cite{Bou72}. In english, it
    is said to be a {\em complete lattice}. 
    This property will be heavily used in order to introduce
    many diffeologies. It is often uncertain to comment what
    is still not defined, but sometimes it can help the
    reader to guess what is coming. 
    Many of the cases which we shall meet follow the same pattern.
    We have a set $\X$
    and some property $\cP$ relating to the diffeologies of
    $\X$ --- we can understand $\cP$ as a function from
    $\Diffeoset(\X)$ to the set $\{\true,\false\}$. The diffeologies for
    which $\cP$ is satisfied constitute some set 
    $\DD = \cP^{-1}(\true)$. This set
    defines two distinguished diffeologies, its infimum $\inf(\DD)$ and
    its supremum $\sup(\DD)$. 
    In most cases, one or the other of these bounds satisfies the property $\cP$.
    Then, we get one (or two) distinguished
    diffeologies satisfying $\cP$, this deserves to be
    noticed. Translated in terms of diffeology this gives
    ``the finest diffeology such that\ldots'' or ``the
    coarsest diffeology such that\ldots''. Infimum and
    supremum change then into minimum and maximum. Here are
    two examples we shall find again later.
    
    \Example{1}~---~The diffeology generated by a set $\cF$
    of parametrizations \art{Generating-diffeology}  is the minimum of
    the set of all the diffeologies containing $\cF$.
    
    \Example{2}~---~Let ${\Y}$ be a diffeological space and
    $\X$ be a set. Let $f: \X \to {\Y}$ be some map. The
    supremum of the diffeologies of $\X$ such that $f$ is
    smooth is a maximum, and it is called the {\em
    pullback} of the diffeology of ${\Y}$
    \art{Pullback-of-diffeologies}. In other words, it is
    the coarsest diffeology such that $f$ is smooth.
    
    \alinea{}These two examples illustrate a frequent phenomenon in
    diffeology: 

    %%###########
    \begin{figure}[ht]
    \centerline{\includegraphics{Figures-PDF/fig-schema-finest.pdf}}
    \caption{Case 1 -- The finest diffeology such that...} 
    \label{fig-Case-1-The-finest-diffeology-such-that}
    \end{figure}
    %%###########

    \alinea{\sc Case 1.}~A family of diffeologies clearly contains
    the coarse diffeology, the right black point of the figure
    \ref{fig-Case-1-The-finest-diffeology-such-that}, 
    then the distinguished diffeology is the lower bound, the finest
    diffeology such that... 
    This is the case of example 1.
    The set of diffeologies containing $\cF$ clearly contains
    the coarse diffeology.
    
    \alinea{\sc Case 2.}~A family of diffeologies clearly contains 
    the discrete diffeology, the left white point of the
    figure \ref{fig-Case-2-The-coarsest-diffeology-such-that}, 
    %%###########
    \begin{figure}[ht]
    \centerline{\includegraphics{Figures-PDF/fig-schema-coarse.pdf}}
    \caption{Case 2 -- The coarsest diffeology such
    that...} 
    \label{fig-Case-2-The-coarsest-diffeology-such-that}
    \end{figure}
    %%###########
    then the distinguished diffeology is the upper bound, the
    coarsest diffeology such that... 
    This is the case in the second
    example. The set of all the diffeologies such that $f$ is
    smooth clearly contains the discrete diffeology, because
    the left composite of any map with a locally constant
    parametrization is locally constant, then a plot of every
    diffeology.

    \alinea{}But the figures
    \ref{fig-Case-1-The-finest-diffeology-such-that} and
    \ref{fig-Case-2-The-coarsest-diffeology-such-that}
    suggest wrongly that fineness is a total order, which is
    obviously not the case. The figure
    \ref{fig-Diffeologies-and-bounds-of-families-inf-sup-min-max}
    is more representative of the real situation. 
    %%###########
    \begin{figure}[ht]
    \centerline{\includegraphics{Figures-PDF/fig-comparing-diffeologies.pdf}}
    \caption{Diffeologies and bounds of families: $\inf$, $\sup$, $\min$,
    $\max$.} 
    \label{fig-Diffeologies-and-bounds-of-families-inf-sup-min-max}
    \end{figure}
    %%###########
    In this figure, the gray domains represent families of
    diffeologies, and the dots their infimum and supremum. The
    far left dot represents the discrete diffeology and the
    far right dot, the coarse diffeology.
    One can argue, of course, that every diffeology is always
    finer than any other coarser diffeology, or conversely coarser 
    than any other finer.
    Then, any diffeology is distinguished, and hence, all this
    discussion does not make sense. But it does not matter, we shall see how
    this discussion applies in concrete situations.
  \end{article} %% playing-with-bounds
    
  %%%%%%%%%%%%%%%%%%%%%%%%%%%%%%%%%%%%%%%%%%%%%%%%%%%%%%%%%%
  %
  %   Exercises  
  %
  %%%%%%%%%%%%%%%%%%%%%%%%%%%%%%%%%%%%%%%%%%%%%%%%%%%%%%%%%%

  \clearpage
  \Exercises  
  
  \begin{exercise}[Locally constant parametrizations] 
    \label{Locally-constant-parametrizations}
    Let ${\P}: \U \to \X$ be a parametrization in some
    set $\X$. Show that $\P$ is locally constant
    \art{Discrete-diffeology} if and only if ${\P}$ is
    constant on each {\em pathwise-connected component\/} of
    its domain $\U$, that is, if and only if, for every pair
    of points $r$ and $r'$ in $\U$, connected by a smooth
    path $\gamma \in \Cinfty(\RR, \U)$, $\gamma(0) =
    r$ and $\gamma(1) = r'$, then $\P(r) = \P(r')$.
  \end{exercise} %% Locally-constant-parametrizations
  
  \begin{exercise}[Diffeology of $\QQ \subset \RR$] 
    \label{Diffeology-of-Q-in-R}
    Check that the set of smooth parametrizations in $\RR$
    with values in $\QQ$ is a diffeology of $\QQ$. Show
    that this diffeology is discrete. Be aware that, in
    the category $\Topology$, $\QQ$ is not discrete in $\RR$.
    More generally, show that every countable subset of every real domain
    (and by consequence, of every manifold) is diffeologically discrete.
  \end{exercise} %% Diffeology-of-Q-in-R
  
  \begin{exercise}[Smooth maps from discrete spaces] 
    \label{Smooth-maps-from-discrete-spaces}
    Show that every map defined on a discrete
    diffeological space to any other
    diffeological space is smooth, that is,
    $\Cinfty(\discrete{\X},\X') = \Maps(\X, \X')$.
  \end{exercise} %% Smooth-maps-from-discrete-spaces
  
  \begin{exercise}[Smooth maps to coarse spaces] 
    \label{Smooth-maps-to-coarse-spaces}
    Show that every map defined from any diffeological
    space to a coarse diffeological space is smooth, 
    that is, $\Cinfty(\X,\coarse{\X'}) = \Maps(\X, \X')$.
  \end{exercise} %% Smooth-maps-to-coarse-spaces

  %%%%%%%%%%%%%%%%%%%%%%%%%%%%%%%%%%%%%%%%%%%%%%%%%%%%%%%%%%
  
  \section*{Pulling Back Diffeologies}
  \label{section-Pulling-backs-diffeologies}
  
  \begin{sechead}
    In this section we describe the transfer of diffeology by
    {\em pullback}. Given any set $\X$ and any map $f$ from $\X$ to a
    diffeological space $\X'$, the set $\X$ inherits a natural
    diffeology by pullback of the diffeology of $\X'$.
  \end{sechead}    
  
  \begin{article}\artlabel{Pullbacks of diffeologies}
    \addcontentsline{toc}{section}{\small\hspace{10pt} Pullbacks of diffeologies}
    \label{Pullback-of-diffeologies} Let $\X$ be a set. Let
    $\X'$ be a diffeological space and $\cD'$ its diffeology.
    Let $f: \X \to \X'$ be some map. There exists a
    coarsest diffeology of $\X$ such that the map $f$ is
    smooth. This diffeology is called the {\em pullback} of
    the diffeology $\cD'$ by $f$, and it is denoted by $f^*(\cD')$.
    A parametrization $\P$ in $\X$ belongs to
    $f^*(\cD')$ if and only if $f \circ \P$ belongs to
    $\cD'$, which writes $$ f^*{(\cD')} = \{ \P \in \Param(\X) \mid f
    \circ {\P} \in \cD' \}. 
    $$ 
  \end{article} %% Pullback-of-diffeologies
  
  \begin{proof}
    Let us consider the family $\DD$ of all the diffeologies
    of $\X$ such that $f$ is smooth. A diffeology
    of $\X$ belongs to $\DD$ if and only if the left
    composition of each of its element by $f$ is an element of
    $\cD'$. By the way, note that the discrete diffeology of
    $\X$ belongs to $\DD$
    (\exref{Smooth-maps-from-discrete-spaces}). We
    know that the family $\DD$ has a supremum
    \art{Supremum-of-a-family-of-diffeologies}. It is the
    intersection of all the diffeologies of $\X$ containing
    the set of all the  parametrizations $\P$ of $\X$ such
    that $f \circ \P$ belongs to $\cD'$. But this set, let us
    call it $\cD$, is already a diffeology, let us check it.
    
    \alinea{D1.} The composition of every constant
    map by any function$f$ is constant. Hence, the constant parametrizations
    belong to $\cD$.
    
    \alinea{D2.} Let $\P : \U \to \X$ be a parametrization such that,
    locally at each point $r$ of $\U$, $f \circ \P \in \cD'$. 
    Since $\cD'$ is a diffeology, $f \circ \P$ is
    a plot for $\cD'$. Hence, $\P$ belongs to $\cD$.
    
    \alinea{D3.} let $\P$ be a parametrization in $\X$. If $f \circ
    {\P}$ is a plot of $\X'$, for every smooth parametrization
    $\F$ in the domain of ${\P}$, 
    $f \circ ({\P} \circ \F) = (f \circ {\P}) \circ \F$ is a plot of $\X'$.
    Hence, $\P \circ \F$ belongs to $\cD$. 
    
    \alinea{}Therefore, 
    $\cD = \{ \P \in \Param(\X) \mid f \circ \P \in \cD' \}$ is the maximum of $\DD$, that is, the
    coarsest diffeology such that $f$ is smooth. 
  \end{proof}
  
  \begin{article}\artlabel{Smoothness and pullbacks}
    \addcontentsline{toc}{section}{\small\hspace{10pt} Smoothness and
    pullbacks} \label{Smoothness-and-pullbacks}
    Let $\X$ and $\X'$ be two diffeological spaces, whose
    diffeologies are denoted by $\cD$ and $\cD'$. The notion
    of pullback of diffeologies gives a new interpretation
    of the notion of smooth maps: a map $f$ from $\X$
    to $\X'$ is smooth if and only if $\cD \subset
    f^*(\cD')$. In other words,
    $$ \Cinfty(\X, \X') = \{ f \in
    \Maps(\X, \X') \mid \cD \subset f^*(\cD') \}.
    $$
  \end{article} %% Smoothness-and-pullbacks
  
  \begin{proof}
  If $f$ is smooth then, for all $\P \in \cD$, $f\circ \P \in \cD'$, by definition. 
  But $f^*(\cD')$ is the set of parametrizations $\P$ in $\X$ 
  such that $f\circ \P \in \cD'$
  \art{Pullback-of-diffeologies}.
  Thus, $\P \in f^*(\cD')$, that is, $\cD \subset f^*(\cD')$. Conversely, if 
  $\cD \subset f^*(\cD')$ then, for all $\P \in \cD$, $f \circ \P \in \cD'$, thus
  $f$ is smooth. Hence, $f$ is smooth if and only if $\cD \subset f^*(\cD')$.
  \end{proof}
  
  \begin{article}\artlabel{Composition of pullbacks}
    \addcontentsline{toc}{section}{\small\hspace{10pt} Composition of pullbacks}
    \label{Composition-of-pullbacks}
    Let $\X$ and $\X'$ be two sets, and
    $\X''$ be a diffeological space.
    Let $f: \X \to \X'$ and $g: \X' \to \X''$ be two
    maps. The pullback by $f$ of the pullback by $g$ of the
    diffeology $\cD''$ of $\X''$ is equal to the pullback 
    of $\cD''$ by $g \circ f$,
      $$
      \X \rfl{f} \X' \rfl{g} (\X'',\cD'') \qmbox{and}
      f^*(g^*(\cD'')) = (g \circ f)^*(\cD'').
      $$
    In other words, the pullback of
    diffeologies is contravariant.
  \end{article} %% Composition-of-pullbacks
  
  \begin{proof}
    This is a direct consequence of the associativity of
    the composition of maps, and of the characterization of the plots
    of the pullback diffeology \art{Pullback-of-diffeologies}.
    Let $ \cD' = g^*(\cD'') = \{ {\P} \in \Param(\X') \mid
    g \circ {\P} \in \cD'' \}$. Then,
    $f^*(g^*(\cD'')) = \{ {\P} \in \Param(\X) \mid f \circ {\P} \in \cD' \} 
      = \{ {\P} \in \Param(\X) \mid g \circ (f \circ {\P}) \in \cD'' \}
      =\{ {\P} \in \Param(\X) \mid (g \circ f) \circ {\P} \in \cD'' \}
      = (g \circ f)^*(\cD'')$. 
  \end{proof}
  
  %%%%%%%%%%%%%%%%%%%%%%%%%%%%%%%%%%%%%%%%%%%%%%%%%%%%%%%%%%
  %
  %   Exercises  
  %
  %%%%%%%%%%%%%%%%%%%%%%%%%%%%%%%%%%%%%%%%%%%%%%%%%%%%%%%%%%

  \Exercise

  \begin{exercise}[Square root of the smooth diffeology] 
    \label{Square-root-of-the-smooth-diffeology}
    Let $\cD$ be the pullback, by the
    {\em square map} $\sq : x \mapsto x^2$, of the smooth
    diffeology $\Cinfty_\star(\RR)$ of $\RR$
    \art{Real-domains-as-diffeological-spaces}. Is $\cD$ finer
    or coarser than $\Cinfty_\star(\RR)$? Check that the map
    $\modulus{\cdot} : x \mapsto \modulus{x}$, defined on
    $\RR$, is a plot for the diffeology $\cD$.  
    Conclude that all the parametrizations of the kind $\P : r
    \mapsto \modulus{\Q(r)}$, where $\Q \in
    \Cinfty_\star(\RR)$, belong to $\cD$.  
  \end{exercise} %% Square-root-of-the-smooth-diffeology
  
  %%%%%%%%%%%%%%%%%%%%%%%%%%%%%%%%%%%%%%%%%%%%%%%%%%%%%%%%%%
  
  \section*{Inductions}
  \label{section-Inductions}
  
  \begin{sechead}
    {\em Inductions} are injections between diffeological
    spaces, identifying the source space with the pullback
    \art{Pullback-of-diffeologies} of the target. They are a
    key categorical construction, used in particular in the definition of diffeological
    subspaces. 
  \end{sechead}
  
  \begin{article}\artlabel{What is an induction?}
    \addcontentsline{toc}{section}{\small\hspace{10pt} What is an induction?}
    \label{What-is-an-induction}
    Let $\X$ and $\X'$ be two diffeological spaces and let $f: \X \to \X'$ be some map. 
    The map $f$ is said to be {\em inductive} or to be an {\em
    induction} if the following two conditions are satisfied.
    %
    \begin{itemize}
      \item[1.] The map $f$ is injective.
      \item[2.] The diffeology $\cD$ of $\X$ is the pullback by $f$
      of the diffeology $\cD'$ of $\X'$, that is,
      with the notation introduced above, $f^*(\cD') = \cD$.
    \end{itemize}
    %
    \Note~The second condition also writes $\cD
    \subset f^*(\cD')$ and  $f^*(\cD')\subset \cD$, where the first
    inclusion $\cD \subset f^*(\cD')$ just says that $f$ is smooth
    \art{Smoothness-and-pullbacks}. 
  \end{article} %% What-is-an-induction
  
  \begin{article}\artlabel{Composition of inductions}
    \addcontentsline{toc}{section}{\small\hspace{10pt} Composition of inductions}
    \label{Composition-of-inductions}
    The composite of two inductions is an induction.
    Inductions form a subcategory of the category
    $\Diffeology$, which can be naturally denoted by $\Inductions$.
  \end{article} %% Composition-of-inductions
  
  \begin{proof}
    Let $\X$, $\X'$ and $\X''$ be three diffeological spaces, and let us
    denote by $\cD$, $\cD'$ and $\cD''$ their diffeologies. Let $f:
    \X \to \X'$ and $g: \X' \to \X''$ be two
    inductions, that is, $g^*(\cD'') = \cD'$ and $f^*(\cD') =
    \cD$. Since pullbacks are contravariant
    \art{Composition-of-pullbacks}, $(g \circ f)^*(\cD'') =
    f^*(g^*(\cD'')) = f^*(\cD') = \cD$.  Since the composite
    of two injections is an injection, $g \circ f$ is
    injective.  Therefore, $g \circ f$ is an induction.
  \end{proof}
  
  \begin{article}\artlabel{Criterion for being an induction}
    \addcontentsline{toc}{section}{\small\hspace{10pt} Criterion for being an induction} 
    \label{Criterion-for-being-an-induction} 
    Let $\X$ and $\X'$ be two diffeological
    spaces. 
    %%###########
    \begin{figure}[tb]
      \centerline{\includegraphics{Figures-PDF/fig-induction.pdf}}
      \caption{An induction.}
      \label{fig-induction}
    \end{figure}
    %%###########
    A map $f: \X \to \X'$ is an
    induction \art{What-is-an-induction} 
    if and only if the following two conditions,
    illustrated by the figure \ref{fig-induction}, are satisfied.
    \begin{itemize}
      \item[1.] The map $f$ is a smooth injection.
      \item[2.] For every plot $\P$ of $\X'$ with
      values in $f(\X)$, the parametrization $f^{-1} \circ \P$
      is a plot of $\X$. 
    \end{itemize}
  \end{article} %% Criterion-for-being-an-induction
  
  \begin{proof}
    Let us recall that the map $f: \X \to \X'$ is an
    induction if and only if it is
    injective and  $\cD = f^*(\cD')$. 
    Let us assume first that $f$ is an induction. Hence $f$
    is injective and smooth
    \art{What-is-an-induction}. Now, let 
    ${\P}: \U \to \X'$ be a plot with values in $f(\X)$. Hence,
    $\Q = f^{-1} \circ {\P}: \U \to \X$ is a
    parametrization in $\X$ such that $\P = f \circ \Q$ is a plot of $\X'$. 
    Thus, by definition of
    induction, $\Q$ is a plot of $\X$ 
    \art{What-is-an-induction}. Therefore, for any plot
    ${\P}$ of $\X'$ with values in $f(\X)$, the map
    $f^{-1} \circ {\P}$ is a plot of $\X$.
    Conversely, let $f$ be a smooth injection such
    that, for any plot $\P$ of $\X'$ with values in $f(\X)$,
    the map $f^{-1} \circ \P$ is a plot of $\X$. Let
    $\P$ be a parametrization in $\X$ such that $f \circ \P$ 
    is a plot of $\X'$,  that is, $\P \in f^*(\cD')$.
    The parametrization $\Q = f \circ \P$ is a plot of
    $\X'$ with values in $f(\X)$. By hypothesis
    $f^{-1} \circ \Q = f^{-1} \circ f \circ {\P} = \P$ is
    a plot of $\X$, hence $f^*(\cD')\subset\cD$. Therefore,
    since $f$ is smooth and injective, $f$ is an induction. 
  \end{proof}
  
  \begin{article}\artlabel{Surjective inductions}
    \addcontentsline{toc}{section}{\small\hspace{10pt} Surjective inductions}
    \label{Surjective-inductions}
    Let $\X$ and $\X'$ be two diffeological spaces. Let $f :
    \X \to \X'$ be an induction. If $f$ is surjective, then
    $f$ is a diffeomorphism. Thus, surjective
    inductions are diffeomorphisms. Conversely,
    diffeomorphisms are surjective inductions.
    This behavior expresses the strong nature of inductions.
    Compare, for example, with the different behavior of
    local inductions \art{Local-inductions}.
  \end{article} %% Surjective-inductions
  
  \begin{proof}
    An induction is, by definition, smooth and
    injective \art{What-is-an-induction}. If moreover, $f$ is
    surjective, then $f$ is a bijection, a smooth bijection.
    By application of the characterization of induction
    above \art{Criterion-for-being-an-induction}, the map
    $f^{-1}$ also is smooth. Therefore, $f$ is a bijection,
    smooth as well as its inverse, that is, a diffeomorphism.
    Conversely, let us denote by $\cD$ and $\cD'$ the diffeologies of
    $\X$ and $\X'$. If $f$ is a diffeomorphism, then it is
    smooth, thus $f \circ \cD \subset \cD'$ or $\cD
    \subset f^*(\cD')$ \art{Smoothness-and-pullbacks}.
    Now its inverse also is smooth, thus $f^{-1}
    \circ \cD' \subset \cD$, that is, $\cD' \subset f \circ
    \cD$ or $\cD \subset f^*(\cD')$. Therefore $f^*(\cD') =
    \cD$, $f$ is an induction, and surjective. 
  \end{proof}
  
  %%%%%%%%%%%%%%%%%%%%%%%%%%%%%%%%%%%%%%%%%%%%%%%%%%%%%%%%%%
  %
  %   Exercises 
  %
  %%%%%%%%%%%%%%%%%%%%%%%%%%%%%%%%%%%%%%%%%%%%%%%%%%%%%%%%%%
  \Exercises
  
  \begin{exercise}[Immersions of real domains] 
    \label{Immersions-of-real-domains}
    Let $\U$ be an $n$-domain,
    regarded as a diffeological space
    \art{Real-domains-as-diffeological-spaces}. Let
    $f : \U \to \RR^m$ be a smooth map, and $r \in \U$.
    Show that, if $f$ is an {\em immersion} at the point $r$,
    that is, if the tangent linear map $\D(f)(r) : \RR^n \to
    \RR^m$ is injective, 
    then there exists an open neighborhood
    $\cO$ of $r$ such that $f \restriction \cO$ is an
    induction. Apply the implicit function theorem, or the
    rank theorem, for smooth parametrizations, see for example in \cite{Die70a}.
  \end{exercise} %% Immersions-of-real-domains
  
  \begin{exercise}[Flat points of smooth paths] 
    \label{Flat-points-of-smooth-paths}
    Let $\gamma$ be a smooth parametrization
    defined from an interval $\openinterval{-\varepsilon, +
    \varepsilon}$ to $\RR^n$. Let $\gamma(0) = 0$. We say
    that $\gamma$ is {\em flat} at $0$ if all the
    derivatives of $\gamma$ vanish at $0$, that is, if 
    $\gamma^{(k)}(0) = 0$, for all $k>0$. 
    Show that, if there exists a sequence 
    of numbers $(t_n)_{n=1}^\infty$
    converging to $0$, such that $t_n < t_{n+1}$ and
    $\gamma(t_n) = 0$, for all $n$, then $\gamma(0) =
    0$ and $\gamma$ is flat at $0$.
  \end{exercise} %% Flat-points-of-smooth-paths
  
  \begin{exercise}[Induction of intervals into domains] 
    \label{Induction-of-intervals-into-domains}
    Let $\varepsilon >0$, let $f$ be a smooth injection from
    an interval $\openinterval{-\varepsilon,+\varepsilon}$ to
    $\RR^n$, such that $f(0) = 0$.
    We say that $f$ is {\em flat} at the point $0$ if, for
    all positive integers $p$, the $p$-th derivative of $f$ at
    $0$ vanishes, which is denoted by $f^{(p)}(0) = 0$ (see
    \exref{Flat-points-of-smooth-paths}). 

    \Question{1)} Show that, if
      $f$ is flat at the point $0$, then $f$ is not an
      induction. Thus, an induction from
      $\openinterval{-\varepsilon,+\varepsilon}$ to $\RR^n$ is
      nowhere flat.
      
    \Question{2)} Show that, if $f$ is an induction,  there
      exist a smallest integer $p \geq 0$ and a smooth map
      $\varphi : \openinterval{-\varepsilon,+\varepsilon} \to
      \RR^n$, such that
      $$
      \mbox{for all $t \in 
      \openinterval{-\varepsilon,+\varepsilon}$,} \quad f(t) = t^p
      \times \varphi(t) \qmbox{and} \varphi'(t) \neq 0.
      $$
    \Note An injective immersion from an interval to $\RR^n$, 
    that is, a smooth injection 
    whose first derivative never vanishes, 
    is not necessarily an induction. 
    A counterexample is treated in the 
    \exref{The-infinite-symbol}. At the moment I write this exercise
    I still do not know if there exist inductions of domains in $\RR^n$ that are not
    immersions.
  \end{exercise} %% Induction-of-intervals-into-domains
  
  \begin{exercise}[Smooth injection in the corner]
    \label{Smooth-injection-in-the-corner}
    Let \guillemots{the corner} $\K$ be the subset of $\RR^2$ defined
    by
    $$
    \K =  \set{ 
    \begin{pmatrix}
    x \\ 0
    \end{pmatrix}
    \mbox{ with } 0 \leq x < 1 } \cup  
    \set{ 
    \begin{pmatrix}
    0 \\ y
    \end{pmatrix}
    \mbox{ with } 0 \leq y < 1 }.
    $$
    \Question{1)} Show that if a smooth injection $\gamma :
      \openinterval{-\varepsilon,+\varepsilon} \to \RR^2$ takes its
      values in the corner $\K$, and if $\gamma(0) = (0,0)$, then
      $\gamma$ is flat (see definition in
      \exref{Flat-points-of-smooth-paths}) at the point
      $0$. Exhibit a smooth parametrization  $\gamma$ of the corner, such that
      $\gamma(0) = (0,0)$ and $\gamma$ is not flat at $0$.
      
    \Question{2)} Check that  $j : \RR
      \to \RR^2$, defined as follows, is a smooth injection
      whose values are the corner $\K$.
      $$
      j(t) = 
      \begin{pmatrix}
      e^{1 \over t} \\ 0
      \end{pmatrix}
      \mbox{ if  } t  < 0, 
      \quad j(0) = 
      \begin{pmatrix}
      0 \cr 0
      \end{pmatrix}, 
      \quad j(t) =
      \begin{pmatrix}
      0 \cr e^{-{1 \over t}}
      \end{pmatrix}
      \mbox{ if } t  > 0.
      $$
      
    \Question{3)} Exhibit a 
      parametrization $c : \RR \to \RR$ such that $j \circ c$
      is smooth, but not $c$ (see
      \exref{Induction-of-intervals-into-domains}). Is $j$ an
      induction? 
  \end{exercise} %% Smooth-injection-in-the-corner

  \begin{exercise}[Induction into smooth maps] 
    \label{Induction-into-smooth-maps} Let $n$ be some integer
    and equip $\Cinfty(\RR,\RR^n)$ with the functional diffeology defined in
    \art{A-diffeology-for-the-sets-of-smooth-maps}. Show that
    the map $f : \RR^n \times \RR^n \to \Cinfty(\RR,\RR^n)$
    defined by $f(x,v) = [t \mapsto x + t v]$ is an induction.
  \end{exercise} %% Induction-into-smooth-maps

  %%%%%%%%%%%%%%%%%%%%%%%%%%%%%%%%%%%%%%%%%%%%%%%%%%%%%%%%%%
  
  \section*{Subspaces of Diffeological Spaces}
  \label{Subspaces-of-diffeological-spaces}
  
  \begin{sechead}
    Every subset of a diffeological space inherits the {\em subset diffeology}
    defined as the pullback of the ambient diffeology by the
    natural inclusion. This construction is related to the notion of 
    induction discussed above \art{What-is-an-induction}.
  \end{sechead}
  
  \begin{article}\artlabel{Subspaces and subset diffeology}
    \addcontentsline{toc}{section}{\small\hspace{10pt} Subspaces and subset diffeology}
    \label{Subspaces-and-subset-diffeology}
    Let $\X$ be a diffeological space and $\cD$ be its diffeology. Pick any
    subset $\A$ of $\X$ and let $j_\A : \A \to \X$ be the
    inclusion. The subset $\A$ carries naturally the diffeology
    $j_\A^*(\cD)$, pullback of the diffeology $\cD$ by the
    inclusion $j_\A$ \art{Pullback-of-diffeologies}. This
    diffeology is said to be {\em inherited} from $\X$ or
    {\em induced} by the inclusion, or by the ambient space.
    It is called the {\em
    subset diffeology} of $\A$. 
    The plots of the subset
    diffeology are just the plots of $\X$ with values in
    $\A$, that is, 
    $$
    j_\A^*(\cD) = \{ {\P} \in \cD \mid \Val({\P})\subset
    \A \}. 
    $$
    Equipped with the subset diffeology, the subset
    $\A$ is said to be a {\em diffeological subspace} of
    $\X$. Thus, a diffeological subspace of $\X$ is any
    subset of $\X$ equipped with the subset diffeology.
    
    \Note~If $\X'$ is a diffeological space and $f : \X \to \X'$ 
    is a smooth map, then just  because $j_\A$ is smooth, 
    $f \circ j_\A$ is smooth, in other words, 
    {\em the restriction of a smooth map to any subspace is smooth}.
  \end{article} %% Subspaces-and-subset-diffeology
  
  \begin{article}\artlabel{Smooth maps to subspaces}
    \addcontentsline{toc}{section}{\small\hspace{10pt} Smooth maps to
    subspaces} \label{Smooth-maps-to-subspaces}
    Let $\X$, $\X'$ and $\X''$ be three
    diffeological spaces. Let $f : \X \to \X'$ be some map and 
    $j : \X' \to \X''$ be an induction.  
    \begin{center}
    \begin{tikzpicture}[auto]
      \node (A)  at (0,0) {$\X$};
      \node (B)  at (0,-2) {$\X'$};
      \node (C)  at (2,-2) {$\X''$};
      \draw[->] (A) to node {$f$} (B);
      \draw[->] (B) to node {$j$} (C);
    \end{tikzpicture}
    \end{center}
    The map $f$ is smooth if and only if $j \circ f$
    is smooth. Moreover, the map $f$ is an induction if and
    only if $j \circ f$ is an induction.
  \end{article} %% Smooth-maps-to-subspaces
  
  \begin{proof}
    Let $\cD$, $\cD'$ and $\cD''$ be the diffeologies of
    $\X$, $\X'$ and $\X''$. The map $j
    \circ f$ is smooth if and only if $\cD \subset (j \circ
    f)^*(\cD'')$ \art{Smoothness-and-pullbacks}. But $(j \circ
    f)^*(\cD'') = f^*(j^*(\cD''))$ \art{Composition-of-pullbacks} and, since
    $j$ is an induction, $\cD' = j^*(\cD'')$, thus $f^*(j^*(\cD'')) =
    f^*(\cD')$. Hence, $\cD \subset (j \circ f)^*(\cD'')$ is equivalent to
    $\cD \subset f^*(\cD')$, which means that $f$ is smooth.
    Note that if $j \circ f$ is injective, then
    $f$ is injective. Next, replacing above the inclusion by an equality  
    proves the second assertion. 
  \end{proof}
  
  \begin{article}\artlabel{Subspaces subsubspaces etc.}
    \addcontentsline{toc}{section}{\small\hspace{10pt} Subspaces subsubspaces etc.}
    \label{Subspaces-subsubspaces-etc.} Let $\X$ be a diffeological space. Let
    $\A$ and $\B$ be two subspaces of $\X$ such that
    $\A \subset \B \subset \X$. It is equivalent to
    consider $\A$ as a subspace of $\B$, regarded as a subspace of $\X$, or
    to consider $\A$ as a subspace of $\X$.  
  \end{article} %% Subspaces-subsubspaces-etc.
  
  \begin{proof}
    The inclusion
    $j_\A$ of $\A$ into $\X$ is the composite $ j_\A = j_\B
    \circ j_{\A\B}$, where $j_\B$ is the inclusion of $\B$
    into $\X$ and $j_{{\A}{\B}}$ the inclusion of $\A$ into
    $\B$. Then, it is just a direct consequence of  the
    transitivity of inductions
    \art{Composition-of-inductions}.  
  \end{proof} 
  
  \begin{article}\artlabel{Inductions identify source and image}
    \addcontentsline{toc}{section}{\small\hspace{10pt} Inductions identify source and
    image}  \label{Inductions-identify-source-and-image} Let
    $\X'$ be a diffeological space and $\A \subset \X'$. For the subset
    diffeology \art{Subspaces-and-subset-diffeology}, the inclusion
    $j_\A: \A \to \X'$ is an induction. 
    More generally, let $\X$ and $\X'$ be two diffeological spaces. An
    induction $f : \X \to \X'$ \art{What-is-an-induction} identifies the
    source $\X$ with its image. In other words, $f$ is a
    diffeomorphism from $\X$ onto $f(\X)$, where $f(\X)$ is equipped with the
    subset diffeology. 
  \end{article} %% Inductions-identify-source-and-image
  
  \begin{proof}
    The first part of the proposition is clear. Now, if $f$ is
    an induction, then it is a smooth bijection onto its image
    $\A = f(\X)$. Let us consider the image $\A$,
    equipped with the subset diffeology, and let $f^{-1}: \A
    \to \X$ be the inverse map. Let ${\P}: \U \to \A$
    be a plot for the subset diffeology, that is, a plot of
    $\X'$ with values in $\A$
    \art{Subspaces-and-subset-diffeology}. By criterion
    \art{Criterion-for-being-an-induction},  $f^{-1} \circ
    {\P}$ is a plot of $\X$. Thus, $f^{-1}$ is
    smooth and $f$ is a diffeomorphism onto its
    image, equipped with the subset diffeology. 
  \end{proof}
  
  \begin{article}\artlabel{Restricting inductions to subspaces}
    \addcontentsline{toc}{section}{\small\hspace{10pt} Restricting inductions to subspaces}
    \label{Restricting-inductions-to-subspaces} Let $\X$ and $\X'$ be
    two diffeological spaces. Let $\F: \X \to \X'$ be a
    induction and $\A$ be some subspace of $\X$. The
    restriction $f $ of $\F$ to $\A$ is a diffeomorphism from
    $\A$ onto its image. In particular restrictions of
    diffeomorphisms are diffeomorphisms from the source to
    the image. 
  \end{article} %% Restricting-inductions-to-subspaces
  
  \begin{proof}
    Let $j_\A$ be the inclusion from $\A$ into $\X$. Since diffeomorphisms
    are inductions \art{Surjective-inductions}, the map $f = \F \circ
    j_\A$ is the composite of two inductions, thus an
    induction \art{Composition-of-inductions}. Now, thanks to
    \art{Inductions-identify-source-and-image}, the map $f$ is a
    diffeomorphism onto its image, equipped with the subset diffeology. 
  \end{proof}
  
  \begin{article}\artlabel{Discrete subsets of a diffeological space}
    \addcontentsline{toc}{section}{\small\hspace{10pt} Discrete subsets of a diffeological
    space} 
    \label{Discrete-subsets-of-a-diffeological-space}
    Let $\X$ be a diffeological space. A subset $\A \subset \X$
    is said to be {\em discrete} if its subset diffeology
    is discrete \art{Discrete-diffeology}, that is, if $\A$,
    as a subspace of $\X$, is diffeologically discrete. 
    See for example \exref{Diffeology-of-Q-in-R} and
    \exref{The-pierced-irrational-torus}. 
  \end{article} %% Discrete-subsets-of-a-diffeological-space
  
  %%%%%%%%%%%%%%%%%%%%%%%%%%%%%%%%%%%%%%%%%%%%%%%%%%%%%%%%%%
  %
  %   Exercises  
  %
  %%%%%%%%%%%%%%%%%%%%%%%%%%%%%%%%%%%%%%%%%%%%%%%%%%%%%%%%%%
  
  \Exercises
  
  \begin{exercise}[Vector subspaces of real vector spaces] 
    \label{Vector-subspaces-of-real-vector-spaces}
    Let $\RR^n$ be equipped with the standard diffeology. Let
    $(b_1,\ldots,b_k)$ be $k$ independent vectors of $\RR^n$. Let $\cB$ be
    the linear map from $\RR^k$ to $\RR^n$ defined by $\cB(x_1,\ldots,x_k)
    =  \sum_{i=1}^k x_i b_i$. Show that $\cB$ is an induction. Deduce that
    any vector subspace $\E$ of $\RR^n$, equipped with the
    subset diffeology, is diffeomorphic to $\RR^k$, with $k = \dim(\E)$.
  \end{exercise} %% Vector-subspaces-of-real-vector-spaces
  
  \begin{exercise}[The sphere as diffeological subspace]
    \label{The-sphere-as-diffeological-subspace}
    Let $\S^n$ denote the unit $n$-sphere in $\RR^{n+1}$,
    $$
    \S^n = \{ x \in \RR^{n+1} \mid \norm{x} = 1 \}.
    $$
    We equip $\S^n$ with the subset diffeology. Let $x$ be a
    point of the sphere $\S^n$, and $\E \subset\RR^{n+1}$
    be the subspace orthogonal to $x$. Let $f$ be the map
    defined, on the vectors of $\E$ whose norm is strictly
    less than 1 \fig{fig-circle}, by
    \begin{equation}
      \renewcommand{\theequation}{$\diamondsuit$}
      f : t \mapsto t + \sqrt{1-\norm{t}^2}\ x, \qmbox{with} t \in
      \E \qmbox{and} \norm{t} < 1. 
    \end{equation}
    Show that $f$ is an induction from  $\B = \{ t \in \E
    \mid \norm{t} < 1 \}$ to $\S^n$. 
    %%###########
    \begin{figure}[tb]
      \centerline{\includegraphics{Figures-PDF/fig-circle.pdf}}
      \caption{The projection $f$.}\label{fig-circle}
    \end{figure}
    %%###########
    Thus, for every point
    $x$ of $\S^n$, there exists an induction from the open
    unit ball of $\RR^n$ into $\S^n$ with values the
    hemisphere made up of the points $x' \in \S^n$ such that the scalar
    product $x\cdot x'$ is strictly positive.
  \end{exercise} %% The-sphere-as-diffeological-subspace
  
  \begin{exercise}[The pierced irrational torus] 
    \label{The-pierced-irrational-torus}
    Let $\Torus_\alpha$ be the irrational torus, described in
    \exref{Diffeomorphisms-between-irrational-tori}. Let $\tau$ be some
    element of $\Torus_\alpha$. Check that $\Torus_\alpha$ is not discrete
    but show that the subspace $\Torus_\alpha - \{ \tau \}$ is discrete. 
  \end{exercise} %% The-pierced-irrational-torus
  
  \begin{exercise}[A discrete image of $\RR$] 
    \label{A-discrete-image-of-R}
    Let us consider the set $\Cinfty(\RR, \RR)$ of smooth maps 
    from $\RR$ to $\RR$.
    
    \Question{1)} Show that the set of parametrizations $\P : \U \to \Cinfty(\RR, \RR)$
    satisfying the following two conditions is a diffeology.
    
    \begin{itemize}
      \item[($\clubsuit$)] The parametrization $\PP : (r,s)
      \mapsto \P(r)(s)$ belongs to $\Cinfty(\U \times \RR,\RR)$.
      \item[($\spadesuit$)] For all $r_0 \in \U$ there exists an
      open ball $\eB$ centered at
      $r_0$, and for all $r \in \eB$ there exists a closed 
      interval $[a,b] \subset \RR$ such that $\P(r)$ and
      $\P(r_0)$ coincide outside of $[a,b]$.  
    \end{itemize} 
    Note that the
    condition ($\clubsuit$) says that $\P$ is a plot for the
    diffeology defined in
    \art{A-diffeology-for-the-sets-of-smooth-maps}. The
    second condition ($\spadesuit$) says that the
    plots are locally compactly supported variations of smooth maps.
    
    \Question{2)} Show that the image of the injective map $f : \RR \to
    \Cinfty(\RR, \RR)$, which associates with each real $a$ the
    linear map $[x \mapsto a x]$, is discrete. Is the map
    $f$ smooth?  
    
    \Question{3)} Generalize this construction to $\Cinfty(\RR^n, \RR^n)$. 
    What can you say about the injection of $\GL(n,\RR)$ 
    into $\Cinfty(\RR^n)$? 
  \end{exercise}  %% A-discrete-image-of-R

  
  %%%%%%%%%%%%%%%%%%%%%%%%%%%%%%%%%%%%%%%%%%%%%%%%%%%%%%%%%%
  
  \section*{Sums of Diffeological Spaces}
  \label{section-Sums-of-diffeological-spaces}
  
  \begin{sechead}
    The category $\Diffeology$ is closed for 
    coproducts --- also called sums or disjoint unions --- of
    diffeological spaces. There exists a distinguished diffeology,
    called the {\em sum diffeology}, 
    on the sum of any family of diffeological spaces. 
  \end{sechead}
    
  \begin{reminder}  
    Let us recall the formal construction 
    of the sum (also called coproduct)
    of any family of sets \cite{Bou72}. Let $\cE =
    \{\E_i\}_{i \in \cI}$ be an arbitrary family of sets,
    $\cI$ being some set of indices. The set $\cI$ can be the
    set $\cE$ itself, in this case the family is said to be
    self-indexed. The sum of the elements of $\cE$ is defined
    as follows,
    $$
    \coprod_{i \in \cI} \E_i = \{(i,e) \mid i \in \cI \mbox{ and } e \in \E_i\}.
    $$
    The sets $\E_i$ are called the {\em components} of the sum 
    $\coprod_{i \in \cI} \E_i$. 
    For every index $i \in \cI$, we denote by $j_i$ the natural injection 
    $$
    j_i : \E_i \to \coprod_{i \in \cI} \E_i \qmbox{ defined
    by }  j_i(e) = (i,e). 
    $$ 
    The sum is sometimes simply denoted by $\coprod \cE$.
    \end{reminder}

  \begin{article}\artlabel{Building sums with spaces} 
  \addcontentsline{toc}{section}{\small\hspace{10pt} Building sums with spaces}
    \label{Building-sums-with-spaces}
    Let $\{\X_i\}_{i \in \cI}$ be a family of
    diffeological spaces, for an arbitrary set of indices $\cI$. 
    There exists, on the direct sum  
    $$ 
    \X =  \coprod_{i \in \cI}\X_i,  
    $$  
    a finest diffeology such that the
    natural injections $j_i: \X_i \to \coprod_{i \in \cI}\X_i$ 
    are smooth, for each index $i \in \cI$. This diffeology is called
    the  {\em sum diffeology}. The set $\X$, 
    equipped with the sum diffeology,
    is called the {\em diffeological sum}, or simply the
    {\em sum}, of the spaces $\X_i$.
    The plots for this diffeology are the parametrizations
    $\PP: \U \to \coprod_{i \in \cI} \X_i$ which are
    locally plots of one of the components of the sum.
    Precisely, a parametrization $\PP: r \to (i_r,\P(r))$ 
    is a plot of the diffeological sum $\X$ if and only if 
    it satisfies the following condition.
    \begin{itemize} 
    \item[($\clubsuit$)] For every $r \in \U$ 
    there exist an index $i$ and an open neighborhood $\V$ of $r$ such that 
    $i_{r'} = i$ for all $r' \in \V$, and $\P \restriction \V$ is a plot of
    $\X_i$. 
    \end{itemize}
    
    \Note~For the sum diffeology,
    every natural injection $j_i: \X_i \to \X$ is an
    induction \art{What-is-an-induction}. Each
    component $\X_i$ equipped with the subset diffeology induced by the sum 
    is diffeomorphic to the original $\X_i$.
    
  \end{article} %% Building-sums-with-spaces
  
  \begin{proof}
    Let us check that ($\clubsuit$) defines a diffeology. 
    Axiom D1, the constant maps take their values in one of the $\X_i$, with
    $i \in \cI$. The locality axiom D2 is satisfied by definition.
    The smooth compatibility axiom D3 is satisfied
    thanks to the continuity of smooth parametrizations in real domains,
    and because each $\X_i$ is itself a diffeological space.
    Therefore, ($\clubsuit$) defines a diffeology of the sum 
    $\X = \coprod_{i \in \cI} \X_i$, for which every injection $j_i$
    is obviously smooth. 
    
    Conversely, let us consider a diffeology $\cD$ on the 
    sum $\X = \coprod_{i \in \cI}\X_i$ such that every
    injection $j_i$, $i \in \cI$, is smooth. Let
    $\PP : r \mapsto (i_r,\P(r)) \in \coprod_{i \in \cI}\X_i$
    be a parametrization satisfying the condition ($\clubsuit$). 
    Thus, for all $r \in \U$, there exist an index $i$ and
    an open neighborhood $\V$ of $r$ such that $i_{r'} = i$  
    for all $r' \in \V$, 
    and $\P \restriction \V$ is a plot of $\X_i$.
    Since $j_i: \X_i \to \X$ is
    smooth for each $i \in \cI$, the composite 
    $j_i \circ (\P \restriction \V) : r \mapsto (i,\P(r))$
    belongs to $\cD$. But
    $j_i \circ (\P \restriction \V) = \PP \restriction \V$,
    thus $\PP$ is the supremum of a family of elements of $\cD$, 
    and thanks to the axiom of locality, $\PP$ itself belongs to $\cD$. 
    Hence, the elements of the diffeology
    defined by the condition ($\clubsuit$) are elements
    of every diffeology such that the natural injections
    $j_i$, $i \in \cI$, are smooth. Therefore the diffeology defined by 
    ($\clubsuit$) is the finest diffeology for which every injection $j_i$ is
    smooth. The fact that the injections $j_i$ are inductions is obvious.
  \end{proof}
  
  \begin{article}\artlabel{Refining a sum}
    \addcontentsline{toc}{section}{\small\hspace{10pt} Refining a sum}
    \label{Refining-a-sum}
    Let $\X = \coprod_{i \in \cI} \X_i$ be a sum of diffeological spaces. 
    Let, for some $j \in \cI$, $\X_j = \coprod_{k \in \cK} \X_j^k$. 
    Let us consider the new family of indices 
    $$
     \cI' = \{ \cI - \{j\} \} \coprod \cK = \{ (\cI,i) \mid i \in \cI \mbox{ and } i \neq j \}
     \cup \{ (\cK,k) \mid k \in \cK \}.
    $$
    For all $i' \in \cI'$, let $\X_{i'} = \X_i$ if $i' = (\cI,i)$ 
    and $i \neq j$, and let
    $\X_{i'} = \X_j^k$ if $i' = (\cK,k)$, with $k \in \cK$.
    Then, there exists a natural equivalence between the two decompositions,
    
    $$
    \X = \coprod_{i \in \cI} \X_i \simeq \coprod_{i' \in \cI'} \X_{i'}.
    $$
    %
    This decomposition applies recursively to any subfamily of 
    components which are themselves the sum of diffeological spaces. 
    That leads to a finest decomposition of every diffeological space into
    the sum of its components \art{The-sum-of-its-components}.
  \end{article} %% Refining-a-sum
    
  \begin{article}\artlabel{Foliated diffeology}
    \addcontentsline{toc}{section}{\small\hspace{10pt} Foliated diffeology}
    \label{Foliated-diffeology}
    Let $\X$ be a diffeological space $\X$, and $\cR$ be an equivalence
    relation defined on $\X$. Let $\cQ$ denote the quotient
    of $\X$ by the relation $\cR$, that is, $\cQ$ is the set
    of the equivalence classes $q = \class(x) \subset \X$, $x \in \X$. As a
    set, $\X$ is equivalent to the sum of its classes: 
    $$
    \X \simeq \coprod_{q \in \cQ} q,
    $$  
    thanks to the bijection $ \sigma : x \mapsto (\class(x),x)$.  The pullback
    of the sum diffeology of the
    family $\cQ$ \art{Building-sums-with-spaces}, 
    by $\sigma$ \art{Pullback-of-diffeologies}, where each class is
    equipped with the subset diffeology of $\X$
    \art{Subspaces-and-subset-diffeology}, is called the {\em foliated
    diffeology} of $\X$ (associated with the relation $\cR$).
    The elements of the foliated diffeology are the plots of
    $\X$, which take locally their values in some
    equivalence class. More precisely, a parametrization
    $\P: \U \to \X$ is a plot for the foliated diffeology
    if and only if the following condition is satisfied. 
    \begin{itemize}  
      \item[($\clubsuit$)] For
      all $r \in \U$ there exists an open neighborhood $\V$ of $r$,
      such that $\P \restriction \V$ is a plot of $\X$ with
      values in $\class(\P(r))$. 
    \end{itemize}  
    The foliated diffeology is obviously finer than the original diffeology. 
  \end{article} %% Foliated-diffeology
  
  \begin{article}\artlabel{Klein structure and singularities of a diffeological space}
    \addcontentsline{toc}{section}{\small\hspace{10pt} Klein structure and singularities of a diffeological space} 
    \label{Klein-structure-and-singularities-of-a-diffeological-space}
    The notion of singularity takes a special meaning in the diffeological context. 
    For example, the irrational torus 
    $\Torus_\alpha = \RR/(\ZZ + \alpha\ZZ)$
    described in \exref{Diffeomorphisms-between-irrational-tori}, 
    often viewed as singular is --- as a group --- extremely regular.
    Indeed, the regularity of a diffeological space $\X$ lies
    in its homogeneity, and conversely, its inhomogeneity expresses
    its singularities. Precisely,  
    the action of the group of diffeomorphisms $\Diff(\X)$
    \art{Diffeomorphisms} distinguishes between subsets of points of the same kind. 
    A space transitive under the action of its diffeomorphisms ---
    all its points are of the same kind --- may be
    regarded as regular, otherwise it can be split into orbits, 
    $$ 
    \X = \cup_{x \in \X} \cO_x \qmbox{with} \cO_x = \{ g(x) \mid g \in \Diff(\X)
    \}. 
    $$ 
    Each orbit represents a certain degree of singularity. Note that
    each orbit $\cO$ may be equipped with the subset
    diffeology, and the sum diffeology of the orbits of
    $\Diff(\X)$ may be called the {\em Klein diffeology} of $\X$.
    Equipped with the Klein diffeology, the set may be
    denoted by $\X_\Klein$, that is,
    $$ 
    \X_\Klein = \coprod_{\cO \in (\X/\!\Diff(\X))} \cO.  
    $$
    The space $\X_\Klein$ reflects the singular spliting of $\X$.
    %
    If $\X$ is transitive under $\Diff(\X)$, then $\X$ is said to be
    a {\em Klein space}\footnote{In reference 
    to Felix Klein's Erlangen program, where he exposed his ideas about what a geometry is. 
    Here, we describe the geometry defined by a diffeology.}. 
    The orbits of $\Diff(\X)$, equipped with
    the subset diffeology, may be called the {\em Klein strata} of $\X$. 
    The way the Klein strata are glued together, to build the space $\X$, is 
    encoded by the quotient $\eS(\X) = \X/\Diff(\X)$, equipped with the quotient 
    diffeology, as it is described further \art{Quotient-and-quotient-diffeology}.
    These are perhaps the tools to explore the notion of singularity 
    from the diffeological viewpoint. We could refine this analysis
    by considering the orbits of the groupoid of germs of local diffeomorphisms
    \art{Local-diffeomorphisms}. 
    
    \Example~Let us consider the half line 
    $\Delta_\infty = [0,\infty[ \subset \RR$, 
    equipped with the subset diffeology, see 
    \exref{Klein-strata-of-0-infty} and
    \exref{The-diffeomorphisms-of-the-half-line}. The group
    $\Diff(\Delta_\infty)$ has 2 orbits: the point $\bullet =\{0\}$ and the open
    subset $\circ = \openinterval{0,\infty}$. Thus, $\eS(\Delta_\infty)$ is the pair
    $\{\bullet, \circ\}$. But the diffeology of $\eS(\Delta_\infty)$ is
    not the discrete diffeology, not all plots are locally constant 
    (see also \exref{Has-the-set-0-1-dimension-1}
    for an example of a $2$-points non discrete space). The
    diffeology of $\eS(\Delta_\infty)$ encodes the relationship between the
    {\em singularity} $\bullet = \{0\}$ and the {\em regular stratum} 
    $\circ = \openinterval{0,\infty}$. Informally, the quotient space $\eS(\Delta_\infty)$
    represents what is called generally the {\em transverse structure}, here 
    the transverse structure of  the partition in Klein strata.
    \end{article} %% Klein-structure-and-singularities-of-a-diffeological-space
  
  %%%%%%%%%%%%%%%%%%%%%%%%%%%%%%%%%%%%%%%%%%%%%%%%%%%%%%%%%%
  %
  %   Exercises  
  %
  %%%%%%%%%%%%%%%%%%%%%%%%%%%%%%%%%%%%%%%%%%%%%%%%%%%%%%%%%%
  
  \Exercises
  
  \begin{exercise}[Sum of discrete or coarse spaces] 
    \label{Sum-of-discrete-or-coarse-spaces}
    Check that the sum of discrete spaces is discrete. Show
    that a diffeological space $\X$ is discrete if and only
    if it is the sum of its elements, that is, if and only
    if $\X = \coprod_{x \in \X} \{x\}$. What
    about the sum of coarse spaces?
  \end{exercise} %% Sum-of-discrete-or-coarse-spaces
  
  \begin{exercise}[Plots of the sum diffeology] 
    \label{Plots-of-the-sum-diffeology}
    Let $\X = \coprod_{i \in \cI} \X_i$ be a sum of
    diffeological spaces. Show that a parametrization $\P :
    \U \to \X$ is a plot if and only if there exists a
    partition $\{\U_i\}_{i \in \cI}$ of $\U$, that is, $\U = \cup_{i
    \in \cI} \U_i$ and $\U_i \cap \U_j = \varnothing$ if
    $i\neq j$, such that, for every non empty $\U_i$, $\P_i =
    \P \restriction \U_i$ is a plot of $\X_i$. We say that
    the plot $\P$ is the sum of the $\P_i$. 
    \end{exercise} %% Plots-of-the-sum-diffeology
  
  \begin{exercise}[Diffeology of $\RR - \{0\}$]  
  \label{Diffeology-of-R---0}
    Check that $\RR - \{0\}$, equipped with the subset
    diffeology of $\RR$, is equal to the sum $\left]-\infty
    , 0 \right[ \coprod \left]0 , \infty\right[$.
  \end{exercise} %% Diffeology-of-R---0
  
  \begin{exercise}[Klein strata of $\left[0,\infty\right[$]  
  \label{Klein-strata-of-0-infty}
    Let $\X$ be the segment $\left[0,\infty\right[$ equipped
    with the subset diffeology of $\RR$. Show that every diffeomorphism
    $\varphi$ of $\X$ fixes $0$, that is, $\varphi(0) = 0$. Justify the fact 
    that the segment has two Klein strata: the singleton $\{0\}$ and the
    open interval $\left]0,\infty\right[$. 
  \end{exercise} %% Klein-strata-of-0-infty
  
  \begin{exercise}[Compact diffeology] 
    \label{Compact-diffeology}
    Check that the diffeology of $\Cinfty(\RR)$ defined
    in \exref{A-discrete-image-of-R} is the foliated diffeology of
    $\Cinfty(\RR)$ for the following equivalence relation.
    \begin{itemize}
      \item[($\diamondsuit$)] Two functions $f$ and $g$ are equivalent if there
      exists a compact $\K$ of $\RR$ such that $f$ and $g$
      coincide outside $\K$. 
    \end{itemize}
    This diffeology is called the {\em compact diffeology} of $\Cinfty(\RR)$
    \cite{Igl87}.
    The third part of the \exref{A-discrete-image-of-R} can be
    rephrased as follows: the group $\GL(n,\RR)$ inherits the
    discrete diffeology from the compact diffeology of $\Cinfty(\RR^n)$.
  \end{exercise} %% Compact-diffeology

  %%%%%%%%%%%%%%%%%%%%%%%%%%%%%%%%%%%%%%%%%%%%%%%%%%%%%%%%%%
  
  \section*{Pushing Forward Diffeologies}
  \label{section-Pushing-forward-diffeologies}
  
  \begin{sechead}
    In this section we describe the transfer of diffeology by
    {\em pushforward}. Given a diffeological space $\X$ and a map $f$
    from $\X$ to some set $\X'$, the set $\X'$ inherits a
    natural diffeology by pushing forward the diffeology of $\X$ 
    to $\X'$.
  \end{sechead}
  
  \begin{article}\artlabel{Pushforward of diffeologies}
    \addcontentsline{toc}{section}{\small\hspace{10pt} Pushforward of diffeologies}
    \label{Pushforward-of-diffeologies} 
    Let $\X$ be
    a diffeological space, and $\X'$ be a set. Let $f: \X
    \to \X'$ be a map. There exists a finest diffeology of
    $\X'$ such that the map $f$ is smooth. This diffeology
    is called the {\em image}, or the {\em pushforward}, by $f$ of
    the diffeology $\cD$ of $\X$, and will be denoted by
    $f_*(\cD)$.  
    A parametrization $\P : \U \to \X'$ is a plot for $f_*(\cD)$ if and only
    if it satisfies the following condition.
    \begin{itemize}
      \item[($\clubsuit$)] For every $r \in \U$, there exists an 
      open neighborhood $\V$ of $r$ such that, either ${\P}\restriction
      \V$ is a constant parametrization, or there exists a plot
      $\Q: \V \to \X$ such that ${\P}\restriction \V = f \circ \Q$.  
    \end{itemize}  
    In the case where $\P \restriction
    \V = f \circ \Q$, we say that $\P$ {\em lifts locally,
    at the point $r$, along the map $f$}. We also say that
    $\Q$ is a {\em local lift of $\P$, over $\V$, along
    $f$}.  
  \end{article} %% Pushforward-of-diffeologies
  
  \begin{proof}
    The set $\DD$ of all the diffeologies of $\X'$
    such that $f$ is smooth is non-empty, since it contains
    the coarse diffeology.  Now, this family $\DD$ has an infimum, let
    us denote it by $\cD'$, it is the intersection of all the elements of
    $\DD$ \art{infimum-of-a-family-of-diffeologies}. We have now to check
    that this infimum is actually a minimum, that is, $f$ is
    smooth for the diffeology $\cD'$. But by the
    very definition of smooth maps, for any plot
    ${\P}$ belonging to $\cD$, $f \circ {\P}$ belongs to
    every element of $\DD$. Thus, ${\P} \circ f$ belongs to
    their intersection $\cD'$. 
    Next, let us check that the set of parametrizations satisfying
    ($\clubsuit$) is a diffeology. The axioms D1 and D2 are satisfied by
    definition. Let us consider then a parametrization $\P : \U \to \X'$
    satisfying ($\clubsuit$) and let $\F : \U' \to \U$ be a smooth
    parametrization. Let $\P' = \P \circ \F$, $r' \in \U'$, $r = \F(r')$,
    $\V' = \F^{-1}(\V)$. Since $\F$ is continuous, $\V'$ is a domain, and
    by construction an open neighborhood of $r'$. Then, let $\Q' = \Q \circ
    \F$, then $\P \restriction \V = f \circ \Q$ implies $\P'
    \restriction \V' = f \circ \Q'$. Therefore, the axiom D3 is satisfied.
    Now, let $\cD'$ be a diffeology such that the map $f$
    is smooth, that is, for every plot $\Q$ of
    $\X$, $f \circ \Q$ is a plot of $\X'$. Hence, the
    diffeology $\cD'$ contains the set of all the
    parametrizations of the type $f \circ \Q$. By
    restriction to any subdomain of the plots $\Q$ (allowed
    by the axiom D3), $\cD'$ contains the elements of the
    diffeology described by ($\clubsuit$). It also contains
    the locally constant parametrizations
    \art{Discrete-diffeology}. Therefore, this diffeology is the
    finest diffeology such that $f$ is smooth.
  \end{proof}
  
  \begin{article}\artlabel{Smoothness and pushforwards}
    \addcontentsline{toc}{section}{\small\hspace{10pt} Smoothness and pushforwards} 
    \label{Smoothness-and-pushforwards}
    Let $\X$ and $\X'$ be two diffeological spaces, whose
    diffeologies are denoted by $\cD$ and $\cD'$. The notion
    of pushforward of diffeologies gives a new
    interpretation of the notion of differentiability: a map
    $f : \X \to \X'$ is smooth if and only if
    $f_*(\cD) \subset \cD'$. In other words, 
    $$ 
    \Cinfty(\X, \X') = \{ f \in \Maps(\X, \X') \mid f_*(\cD) \subset \cD' \}. 
    $$
  \end{article} %% Smoothness-and-pushforwards
  
  \begin{proof}
    Let us assume first that $f$ is smooth. Said differently, $\cD'$ is a diffeology of 
    $\X'$ such that $f$ is smooth. But $f_*(\cD)$ is the intersection of all the
    diffeologies of $\X'$ for which $f$ is smooth
    \art{Pushforward-of-diffeologies}, thus $f_*(\cD) \subset \cD'$.
    Now, let us assume that $f_*(\cD) \subset \cD'$. Let $\P$ be a plot of 
    $\X$ and $\P' = f \circ \P$. Thanks to
    \xart{Pushforward-of-diffeologies}{($\clubsuit$)}, $\P'$ is a plot of $f_*(\cD)$. 
    Thus, by hypothesis, $\P'$ is a plot of $\cD'$. Therefore, $f$ is smooth.
  \end{proof} 
  
  \begin{article}\artlabel{Composition of pushforwards}
    \addcontentsline{toc}{section}{\small\hspace{10pt} Composition of pushforwards}
    \label{Composition-of-pushforwards}
    Let $\X$ be a diffeological space, let $\X'$ and
    $\X''$ be two sets. Let $f: \X \to \X'$ and $g: \X'
    \to \X''$ be two maps. The pushforward by $g$ of the
    pushforward by $f$ of the diffeology $\cD$ of $\X$ is equal to
    the pushforward of $\cD$ by $g \circ f$,
    $$
      (\X,\cD) \rfl{f} \X' \rfl{g} \X'' \qmbox{and}
      g_*(f_*(\cD)) = (g \circ f)_*(\cD).
      $$
      In other words, the pushforward of diffeologies is associative.
  \end{article} %% Composition-of-pushforwards
  
  \begin{proof}
    Let us prove first that  $(g \circ f)_*(\cD) \subset g_*(f_*(\cD))$. Let
    $\P : \U \to \X''$ be a plot for $(g \circ f)_*(\cD)$. By
    \art{Pushforward-of-diffeologies} $\P$ is either locally constant or
    writes locally $\P \restriction \V = (g \circ f) \circ \Q$, where $\Q$
    is a plot of $\X$. If $\P$ is locally constant, every local constant
    plot is included in every diffeology, {\em a fortiori\/} in $g_*(f_*(\cD))$. Now,
    $(g \circ f) \circ \Q = g \circ (f \circ \Q)$. But $f \circ \Q$ belongs
    to $f_*(\cD)$, hence $\P \restriction \V$ belongs to $g_*(f_*(\cD))$.
    Therefore $(g \circ f)_*(\cD) \subset g_*(f_*(\cD))$. 
    Next, let us check
    that $g_*(f_*(\cD)) \subset (g \circ f)_*(\cD)$. Let $\P : \U \to \X''$
    be an element of $g_*(f_*(\cD))$. The plot $\P$ is either locally constant or
    writes locally $\P \restriction \V' = g \circ \Q'$, where $\Q'$ is a
    plot for $f_*(\cD)$. Locally constant parametrizations belong to
    every diffeology, {\em a fortiori\/} to $(g \circ f)_*(\cD)$. Then, let $\Q'$ write
    locally $ \Q' \restriction \V = f \circ \Q$ where $\Q$ is a plot of
    $\X$ defined on $\V$. Thus,
    $\P \restriction \V = (g \circ f) \circ \Q$, that is, $\P$ belongs to
    $(g \circ f)_*(\cD)$. 
  \end{proof}
  
  %%%%%%%%%%%%%%%%%%%%%%%%%%%%%%%%%%%%%%%%%%%%%%%%%%%%%%%%%%
  %
  %   Exercises  
  %
  %%%%%%%%%%%%%%%%%%%%%%%%%%%%%%%%%%%%%%%%%%%%%%%%%%%%%%%%%%

  \Exercise

  \begin{exercise}[Square of the smooth diffeology] 
    \label{Square-of-the-smooth-diffeology}
    Let $\cD$ be the pushforward of the smooth diffeology
    of $\RR$ by the {\em square map} $\sq = [x \mapsto x^2]$,
    that is, $\cD = \sq_*(\Cinfty_\star(\RR))$
    \art{Real-domains-as-diffeological-spaces}. Is $\cD$
    finer or coarser than $\Cinfty_\star(\RR)$? Let $\P : \U
    \to \RR$ be an $n$-plot for $\cD$, $n >0$, and let $r \in
    \U$. 
    
    \Question{1)} Check that if $\P(r) < 0$, then there exists an open ball $\Ball$
    centered at $r$ such that $\P \restriction \Ball$ is constant. 
    
    \Question{2)} Check that if $\P(r) > 0$, then there exists an open
    ball $\Ball$ centered at $r$ such that the parametrization
    $\sqrt{\P}$ is smooth.
    
    \Question{3)} Check that if $\P(r_0) = 0$, then the tangent linear
    map of $\P$ at $r_0$, denoted by $\D(\P)(r_0)$, vanishes.
    Check that the Hessian $\H$ of $\P$ at the point $r_0$,
    that is, $\H = \D^2(\P)(r_0) = \D(r \mapsto
    \D(\P)(r))(r_0)$, regarded as a bilinear map on $\RR^n$,
    is positive, \ie, for all $v \in \RR^n$,
    $\H(v)(v) \geq 0$.
    
    \Question{4)} Is the real function $f : x \mapsto x^2 - x^3$,
    defined on $]-\infty, 1[$, a plot of $\cD$?
  \end{exercise} %% Square-of-the-smooth-diffeology
  
  %%%%%%%%%%%%%%%%%%%%%%%%%%%%%%%%%%%%%%%%%%%%%%%%%%%%%%%%%%
  
  \section*{Subductions}
  \label{section-Subductions}
  
  \begin{sechead}
    Subductions are surjections between diffeological
    spaces, identifying the target
    with the pushforward \art{Pushforward-of-diffeologies}
    of the source. They are a categorical key construction, used in particular in the
    definition of the diffeological quotients
    \art{Quotient-and-quotient-diffeology}. 
  \end{sechead}
  
  \begin{article}\artlabel{What is a subduction?}
    \addcontentsline{toc}{section}{\small\hspace{10pt} What is a subduction?}
    \label{What-is-a-subduction}
    Let $\X$ and $\X'$ be two diffeological spaces and let
    $f: \X \to \X'$ be some map. The map $f$ is said to be a 
    {\em subduction} if it satisfies the following condition.  
    \begin{itemize}
      \item[1.] The map $f$ is surjective.
      \item[2.] The diffeology $\cD'$ of $\X'$ is the pushforward of the
      diffeology $\cD$ of $\X$, that is, with the notation introduced above,
      $f_*(\cD) = \cD'$.
    \end{itemize}
    
    \Note~The second condition also writes $f_*(\cD) \subset \cD'$ and $\cD'
    \subset f_*(\cD)$, where the first inclusion just says that $f$ is
    smooth \art{Smoothness-and-pushforwards}.
  \end{article} %% What-is-a-subduction
  
  \begin{article}\artlabel{Compositions of subductions}
    \addcontentsline{toc}{section}{\small\hspace{10pt} Compositions of subductions}
    \label{Composition-of-subductions}
    The composite of two subductions is a subduction.
    Subductions form a subcategory of the category
    $\Diffeology$, which may be naturally denoted by $\Subductions$. 
  \end{article} %% Composition-of-subductions
  
  \begin{proof}
    Let $\X$, $\X'$ and $\X''$ be three diffeological spaces, and let
    $\cD$, $\cD'$ and $\cD''$ be their diffeologies. Let $f:
    \X \to \X'$ and $g: \X' \to \X''$ be two
    subductions, that is, $f_*(\cD) = \cD'$ and $g_*(\cD') =
    \cD''$. Since the pushforward is associative
    \art{Composition-of-pushforwards}, $(g \circ f)_*(\cD) =
    g_*(f_*(\cD)) = g_*(\cD') = \cD''$. Therefore, $g \circ
    f$ is a subduction.  
  \end{proof}
  
  \begin{article}\artlabel{Criterion for being a subduction}
    \addcontentsline{toc}{section}{\small\hspace{10pt} Criterion for being a subduction}
    \label{Criterion-for-being-a-subduction}
    Let $\X$ and $\X'$ be two diffeological spaces and
    $f : \X \to \X'$ be some map. The map $f$ is a subduction if 
    and only the following conditions, illustrated by figure \ref{fig-subduction},
    are satisfied.
    \begin{itemize}
      \item[1.] The map $f$ is a smooth surjection.
      \item[2.] For every plot
      $\P : \U \to \X'$, for every $r \in \U$,
      there exist an open neighborhood $\V$ of $r$ and a plot 
      $\Q : \V \to \X$ such that 
      ${\P}\restriction \V = f \circ \Q$.  
    \end{itemize}
    %%###########
    \begin{figure}[tb]
      \centerline{\includegraphics{Figures-PDF/fig-subduction.pdf}}
      \caption{A subduction.}
      \label{fig-subduction}
    \end{figure}
    %%########### 
    Put differently, using the vocabulary introduced in
    \art{Pushforward-of-diffeologies}, the map $f$ is a subduction if and only if
    $f$ is smooth and every plot of $\X'$ lifts locally along
    $f$, at each point of its domain. 
  \end{article} %% Criterion-for-being-a-subduction
  
  \begin{proof}
    The first condition is equivalent to $f_*(\cD) \subset \cD'$
    \art{Smoothness-and-pushforwards}. The second condition is the
    reduction of the condition $\cD'
    \subset f_*(\cD)$ \art{Pushforward-of-diffeologies}, in the case of a
    surjection.
  \end{proof}
  
  \begin{article}\artlabel{Injective subductions}
    \addcontentsline{toc}{section}{\small\hspace{10pt} Injective subductions}
    \label{Injective-subductions}
    Let $\X$ and $\X'$ be two diffeological spaces and let $f : \X \to \X'$ be
    a subduction. If $f$ is injective, then $f$ is a diffeomorphism.
    Conversely, diffeomorphisms are injective subductions.
  \end{article} %% Injective-subductions
  
  \begin{proof}
    A subduction is, by definition, smooth and surjective
    \art{What-is-a-subduction}. If moreover, $f$ is
    injective, then $f$ is a smooth
    bijection. Now, let $\P : \U \to \X'$ be a plot, for every
    $r \in \U$ there exist an open neighborhood $\V$ of $r$ and a plot
    $\Q$ of $\X$ such that $f \circ \Q = \P \restriction \V$
    \art{Criterion-for-being-a-subduction}. But since $f$
    is bijective, $\Q$ is unique and $\Q = f^{-1} \circ (\P
    \restriction \V)$, that is, $f^{-1} \circ \P$ is locally
    a plot of $\X$ at each point $r$ of $\U$. Hence, $f^{-1}
    \circ \P$ is a plot of $\X$ (axiom D2) and $f^{-1}$ is
    smooth. Thus, $f$ is a diffeomorphism.
    Conversely, let $\cD$ and $\cD'$ denote the diffeologies
    of $\X$ and $\X'$. If $f$ is a diffeomorphism, then it
    is smooth and $f_*(\cD) \subset \cD'$
    \art{Smoothness-and-pushforwards}. Now, its
    inverse also is smooth and $f^{-1}_*(\cD')
    \subset \cD$. But, thanks to
    \art{Composition-of-subductions}, this is equivalent to
    $\cD' \subset f_*(\cD)$. Therefore, $f_*(\cD) = \cD'$.
  \end{proof}
  
  %%%%%%%%%%%%%%%%%%%%%%%%%%%%%%%%%%%%%%%%%%%%%%%%%%%%%%%%%%
  %
  %   Exercises  
  %
  %%%%%%%%%%%%%%%%%%%%%%%%%%%%%%%%%%%%%%%%%%%%%%%%%%%%%%%%%%
  
  \Exercises
  
  \begin{exercise}[Subduction onto the circle] 
    \label{Subduction-onto-the-circle}
    Equip $\RR$ with the standard diffeology, and the circle $\S^1
    \subset \RR^2$ with the subset diffeology. Show that the  map
    $\Pi : t \mapsto (\cos(t), \sin(t))$ from $\RR$ to
    $\S^1$ is a subduction. 
  \end{exercise} %% Subduction-onto-the-circle
  
  \begin{exercise}[Subduction onto diffeomorphisms] 
    \label{Subduction-onto-diffeomorphisms} 
    Let $\G$ be the set of smooth maps defined by
    $$
    \G = \{ f \in \Cinfty(\RR) \mid \mbox{For all } x \in \RR,\ f'(x) \neq 0 
    \mbox{ and } f(x + 2 \pi) = f(x) \pm 2\pi \}, 
    $$
    where $f'$ denotes the first derivative. 
        
    \Question{1)} Check that $\G$ is a group, actually a subgroup of 
     $\Diff(\RR)$.
    
    \Question{2)} Let $\S^1 \subset \RR^2$ be equipped with the subset diffeology,
    and $\Diff(\S^1)$ its group of diffeomorphisms. 
    Show that the set of parametrizations $\P : \U \to \Diff(\S^1)$
    satisfying the following condition is a diffeology of $\Diff(\S^1)$.
    \begin{itemize} 
    \item[($\diamondsuit$)]
      For all plots $\Q : \V \to \S^1$, the
      parametrization $(r,s) \mapsto \P(r)(\Q(s))$, defined on $\U \times
      \V$, is a plot of $\S^1$. 
    \end{itemize}

    \Question{3)} Show that, for every $f \in \G$, there exists a unique diffeomorphism
    $\varphi$ of $\S^1$ such that $\Pi \circ f = \varphi \circ \Pi$, 
    where $\Pi$ is defined in \exref{Subduction-onto-the-circle}. 
    Show that the map $\Phi : f \mapsto \varphi$ is a homomorphism.
    
    \Question{4)} Let $\G$ be equipped with the subset diffeology of the diffeology of 
    $\Cinfty(\RR)$ defined in \exref{Smooth-maps-on-spaces-of-maps}.
    We shall admit that any smooth map, from an open 
    ball of a real domain to $\S^1 \subset \RR^2$, 
    has a global smooth lift to $\RR$ along the subduction $\Pi$.
    Show that the map $\Phi$ is a subduction from $\G$ onto
    $\Diff(\S^1)$. Describe the preimage of any $\varphi \in \Diff(\S^1)$.
    
    \Question{5)} Show that the subgroup of translations $x \mapsto x + a$, where $a$
    is any real number, is a subgroup of $\G$, diffeomorphic to $\RR$. Show
    that its image by $\Phi$ is a subgroup of $\Diff(\S^1)$ diffeomorphic to
    the circle $\S^1$.
  \end{exercise} %% Subduction-onto-diffeomorphisms
  
  %%%%%%%%%%%%%%%%%%%%%%%%%%%%%%%%%%%%%%%%%%%%%%%%%%%%%%%%%%
  
  \section*{Quotients of Diffeological Spaces}
  \label{section-Quotients-of-diffeological-spaces}
  
  \begin{sechead}
    Any quotient of a diffeological space inherits the {\em quotient
    diffeology} defined as the pushforward of the diffeology of the source 
    space to the quotient. This construction is
    related to the notion of subduction discussed above in 
    \art{What-is-a-subduction}. We say that the category $\Diffeology$ is 
    closed by quotient.
    
    I remember in highschool, I have always been uneasy with the notion of 
    quotient  of a set $\X$ by an equivalence relation $\sim$, 
    even though I understood pretty well
    that the class of an element $x$ was the set of all its equivalent elements. 
    Only later did I realize my confusion: it  were the various 
    representations of the quotients, natural or not, used by my teachers, 
    like magic tricks. I decided then to always 
    regard a class for what it is: a subset of $\X$,
    independently of any representation, and I decided to regard every quotient,
    the set of all the equivalence classes, 
    as a subset of the powerset $\Powerset(\X)$. 
    Embedding the quotient into the powerset comforted me.  
    This is the way I shall understand thereafter the notion of quotient. 
    I shall also often use the generic notation $\class(x)$ to denote the 
    class of the element $x$, \ie, $\class(x) = \{ x' \in \X \mid x' \sim x \}$. 
    So, $\class : \X \to \Powerset(\X)$ is an application, 
    the {\em natural projection} or the {\em canonical projection} from $\X$ 
    onto its quotient, and by definition  
    $\quotient{\X}{\sim} = \class(\X) \subset \Powerset(\X)$.
  \end{sechead}
  
  \begin{article}\artlabel{Quotient and quotient diffeology}
    \addcontentsline{toc}{section}{\small\hspace{10pt} Quotient and quotient diffeology}
    \label{Quotient-and-quotient-diffeology}
    Let $\X$ be a diffeological space and $\cD$ be its diffeology. Pick an
    equivalence relation $\cR$ on $\X$. The quotient set 
    $\X/\cR$  carries a natural diffeology $\class_*(\cD)$, 
    pushforward of the diffeology 
    $\cD$ by the natural projection $\class : \X \to \X/\cR$    
    \art{Pushforward-of-diffeologies}. This diffeology is called the {\em
    quotient diffeology}. For the quotient diffeology, the projection $\class$
    is a subduction \art{What-is-a-subduction}. The set $\X/\cR$    
    equipped with the  quotient diffeology is called the {\em quotient space} 
    or the {\em  diffeological quotient} of $\X$ by the relation $\cR$. 
    For example, let $\Q$ be a diffeological space. Every surjection 
    $\pi : \X \to \Q$ defines a natural equivalence relation $\cR$, that is, $x\, \cR \, x'$ 
    if $\pi(x) = \pi(x')$. The quotient space $\X/\cR$ is 
    often also denoted by $\X/\pi$. We shall see in \art{Uniqueness-of-quotient} 
    that, if $\pi$ is a subduction, then the bijection 
    $\class(x) \mapsto \pi(x)$, from $\X/\cR$ to $\Q$, is a  diffeomorphism.
  \end{article} %% Quotient-and-quotient-diffeology
  
  \begin{article}\artlabel{Smooth maps from quotients}
    \addcontentsline{toc}{section}{\small\hspace{10pt} Smooth maps from quotients} 
    \label{Smooth-maps-from-quotients}
    Let $\X$, $\X'$ and $\X''$ be three diffeological spaces.
    Let $\pi: \X \to \X'$ be a subduction.  A map $f: \X' \to \X''$ 
    is smooth if and only if $f \circ \pi$ is smooth. 
    \begin{center}
    \begin{tikzpicture}[auto]
      \node (A)  at (0,0) {$\X$};
      \node (B)  at (-2,-2) {$\X'$};
      \node (C)  at (2,-2) {$\X''$};
      \draw[->] (A) to node [swap] {$\pi$} (B);
      \draw[->] (A) to node {$f \circ \pi$} (C);
      \draw[->] (B) to node [swap] {$f$} (C);
    \end{tikzpicture}
    \end{center}
    Moreover, the map $f$ is
    a subduction if and only if $f \circ \pi$ is a subduction.
  \end{article} %% Smooth-maps-from-quotients
  
  \begin{proof}
    Let $\cD$, $\cD'$ and $\cD''$ be the diffeologies of $\X$, $\X'$ and
    $\X''$. 
    The map $f \circ \pi$ is smooth if and only if
    $(f \circ \pi)_*(\cD) \subset \cD''$
    \art{Smoothness-and-pushforwards}. But 
    $(f \circ \pi)_*(\cD) = f_*(\pi_*(\cD))$
    \art{Composition-of-pushforwards}. 
    Since $\pi$ is a subduction, $\pi_*(\cD) = \cD'$ and $f_*(\pi_*(\cD)) =
    f_*(\cD')$. Hence,  $(f \circ \pi)_*(\cD) \subset \cD''$
    is equivalent to $f_*(\cD') \subset \cD''$ which means
    that $f$ is smooth. 
    Note that if $\pi \circ f$ is surjective, then
    $f$ is surjective. Next, replacing above the inclusion by an equality  
    proves the second assertion. 
  \end{proof}
  
  \begin{article}\artlabel{Uniqueness of quotient}
    \addcontentsline{toc}{section}{\small\hspace{10pt} Uniqueness of quotient}
    \label{Uniqueness-of-quotient}
    Let $\X$, $\X'$ and $\X''$ be three diffeological spaces. Let $\pi':
    \X \to \X'$ and $\pi'': \X \to \X''$ be two
    subductions. If $f: \X' \to \X''$ is a bijection such
    that $f \circ \pi' = \pi''$, then $f$ is a
    diffeomorphism.
    In particular, let $\X$ be a diffeological space and $\cR$ be
    an equivalence relation on $\X$. Let $\X/\cR$ be the diffeological
    quotient \art{Quotient-and-quotient-diffeology}. Let $\Q$ be a set and
    $p : \X \to \Q$ be a surjection such that $p(x) = p(x')$
    if and only if $x\, \cR\, x'$. 
    Therefore, the map $f : \class(x) \mapsto p(x)$, is a diffeomorphism
    from $\X/\cR$ to $\Q$, where $\Q$ is equipped with the pushforward
    diffeology of $\X$ \art{Pushforward-of-diffeologies}.
   \begin{center} 
     \begin{tikzpicture} 
     \matrix[matrix of math nodes, row sep=1.2cm, column sep=1.2cm]
     {
       |(A)|  \X  \\ 
       |(B)|  \X/\cR & |(C)| \Q \\
     }; 
     \begin{scope}[every node/.style={pos=0.5,font=\normalsize}]
       \draw[->] (A) -- (B) node[auto=right] {$\pi$}; 
       \draw[->] (B) -- (C) node[auto=right] {$f$} ;
       \draw[->] (A) -- (C) node[auto=left] {$p$} ;
     \end{scope} 
     \end{tikzpicture}
   \end{center}    We say that the space $\Q$ is
    a {\em representation of the quotient} $\X/\cR$. 
  \end{article} %% Uniqueness-of-quotient
  
  \begin{proof}
    Since $f \circ \pi' = \pi''$ and $\pi''$ is a
    subduction, $f$ is a subduction 
    \art{Smooth-maps-from-quotients}. 
    Since $f$ is bijective,
    $f$ is a diffeomorphism \art{Injective-subductions}.
  \end{proof}
  
  \begin{article}\artlabel{Sections of a quotient}
    \addcontentsline{toc}{section}{\small\hspace{10pt} Sections of a quotient}
    \label{Section-of-a-quotient}
    It is sometimes convenient to characterize --- or compute ---
    the quotient $\X/\cR$, of a diffeological space $\X$ 
    by an equivalence relation $\cR$,
    by restricting the equivalence relation to a subspace $\X'$. 
    In other words, giving $\X' \subset \X$ and $\cR'$, 
    the restriction of $\cR$ to $\X'$, what are the conditions for which
    the natural map
    $$
     \Phi : \X'/\cR' \to \X/\cR \qmbox{ with} 
     \Phi : \class'(x') \mapsto \class(x')
    $$
    is a diffeomorphism? Remark that since $\cR'$ is the restriction of $\cR$ to 
    $\X'$, then $\class'(x') = \class(x') \cap \X' \subset \class(x')$. 
    Note that $\Phi$ is necessarily smooth,
    since $\class$ and $\class'$ are two subductions, and moreover injective, 
    its inverse is given by,
    $$
     \Phi^{-1} : \X/\cR \to \X'/\cR' \qmbox{ with} 
     \Phi^{-1} : \class(x) \mapsto \class(x) \cap \X'.
    $$  
    The first condition, that $\Phi$ is surjective, writes
    $\X' \cap \class(x) \neq \varnothing$, for all $x \in \X$. 
    If this condition is fulfilled, then $\Phi$ is a smooth bijection. 
    Now, if $\Phi$ is surjective, then $\Phi^{-1}$ 
    is smooth if and only if for all plots $\P : \U \to \X$, for all $r \in \U$,
    there exist an open neighborhood $\V$ of $r$ and a plot $\P' : \V \to \X'$ 
    such that $\class(\P'(r')) = \class(\P(r'))$ for all $r' \in \V$. 
    If theses conditions are fulfilled then
    we shall say that $(\X',\cR')$ is a {\em reduction} of $(\X,\cR)$.     
    If moreover, $\X'$ is such that $\class(x) \cap \X'$ is reduced to one point,
    then we shall say that $\X'$ is a {\em smooth section} of the projection 
    $\class : \X \to \cQ = \X/\cR$. Indeed, the map $\sigma : \cQ \to \X$ defined 
    by $\sigma(q) = q \cap \X'$ is a diffeomorphism and satisfies 
    $\class \circ \sigma = \id_\cQ$. In other words, $\X'$ 
    represents $\cQ$.
    
    \Note~If there exists a smooth projector $\rho: \X \to \X$,  
    that is, $\rho \circ \rho = \rho$, 
    which preserves the equivalence relation: 
    $\class \circ \rho = \class$,
    then $\X' = \rho(\X)$, together with the restriction $\cR'$, is a reduction
    of $(\X,\cR)$. Moreover, if $\class'$ 
    is injective, then $\X'$ is a 
    smooth section of $\class$.
  \end{article} %% Section-of-a-quotient
  
  \begin{proof}
    The map $\Phi$ is smooth because for every plot $r \mapsto q'_r$ of 
    $\cQ' = \X'/\cR'$, there exists, locally everywhere, a smooth paramatrization 
    $r \mapsto x'_r$ such that $q'_r = \class'(x'_r)$. Then, 
    $\Phi \circ \left[r \mapsto q'_r\right] = \class \circ \left[r \mapsto x'_r\right]$ 
    is smooth. Next, $\Phi$ is clearly injective, let 
    $\Phi(\class'(x')) = \Phi(\class'(x''))$, that is, $\class(x') = \class(x'')$.
    Thus,  $\class'(x') = \class(x') \cap \X' = \class(x'') \cap \X' = \class'(x'')$.
    Now, let us assume that $\Phi$ is surjective, 
    let $\P : r \mapsto q_r$ be a plot of $\cQ$, 
    by definition, there exists, locally everywhere, a plot 
    $r \mapsto x_r$ such that $q_r = \class(x_r)$. Let us assume next that,
    after possibly shrinking the domain of $r \mapsto x_r$, there exists a plot
    $r \mapsto x'_r$ of $\X'$ such that $\class(x'_r) = \class(x_r)$. Hence,
    locally, $\Phi^{-1}(q_r) = \Phi^{-1}(\class(x_r)) = \class(x_r) \cap \X'
    = \class(x'_r) \cap \X' = \class'(x'_r)$. 
    Therefore, $\Phi^{-1} \circ \left[r \mapsto q_r\right]$ is locally smooth,
    thus $\Phi^{-1}$ is smooth, and $\Phi$ is a diffeomorphism. Conversely, 
    let us assume that $\Phi$ is a diffeomorphism, that is, $\Phi^{-1}$ is smooth. Let 
    $\P : \U \to \X$ be a plot, thus $\class \circ \P$ is a plot of 
    $\cQ$, and $\Phi^{-1} \circ \class \circ \P$ is a plot of $\cQ'$. Hence,
    there exists locally everywhere a plot $\P'$ of $\X'$ such that
    $\class'(\P'(r)) = \Phi^{-1}(\class(\P(r)))$, or 
    $\Phi(\class'(\P'(r))) = \class(\P(r))$, that is, $\class(\P'(r)) = \class(\P(r))$.
    Now, if $\X' \cap \class(x)$ is reduced to one point, 
    then $\class'$ is an injective subduction, that is, a diffeomorphism. 
    Let us assume next that there exists a smooth projector $\rho$ satisfying the conditions 
    of the note. The subspace $\X' = \rho(\X)$ intersects every orbit. Indeed, 
    let $q \in \cQ$ and $x \in q$, thus $x' = \rho(x) \in \X'$ and  
    $\class(x') = q$. Thus, $\Phi$ is surjective. Next, let $\P$ be a plot of
    $\X$, the parametrization $\P' = \rho \circ \P$ is a plot of $\X'$, and
    since $\class \circ \operatorname{\rho} = \class$, the 
    condition of the proposition is satisfied. 
  \end{proof}  
  
  \begin{article}\artlabel{Strict maps, between quotients and subspaces} 
  \addcontentsline{toc}{section}{\small\hspace{10pt} Strict maps, between
    quotients and subspaces}
    \label{Strict-maps-between-quotients-and-subspaces}
    Let $\X$ and $\X'$ be two diffeological
    spaces.  
    We say that a map $f: \X \to \X'$ is {\em strict} if $f$ is a 
    subduction onto its image, when its image is 
    equipped with the subset diffeology.
    We may also use the following criterion.
    \begin{itemize} 
      \item[($\diamondsuit$)] A map $f :
      \X \to \X'$ is strict if and only if $f$ is smooth and for all plots $\P :
      \U \to \X$ such that $\Val(\P) \subset f(\X)$, for
      all $r \in \U$, there exist an open neighborhood $\V$ of $r$ and a
      plot $\Q : \V \to \X$ such that $f \circ \Q = \P
      \restriction \V$. 
    \end{itemize}
    \Note~There exists a
    universal factorization described by the following
    diagram,
    where $\X/f$ is the quotient by the equivalence
    relation defined by $f$ \art{Quotient-and-quotient-diffeology},
    $p$ is the natural projection, and $j$ is the inclusion
    $f(\X) \subset \X'$. 
    \begin{center}
    \begin{tikzpicture}[ auto]
      \node (A)  at (0,0) {$\X$};
      \node (B)  at (3,0) {$\X'$};
      \node (C)  at (0,-2) {$\X/f$};
      \node (D)  at (3,-2) {$f(\X)$};
      \draw[->] (A) to node {$f$} (B);
      \draw[->] (A) to node [swap] {$p$} (C);
      \draw[->] (C) to node [swap] {$\varphi$} (D);
      \draw[->] (D) to node [swap] {$j$} (B);
    \end{tikzpicture}
    \end{center}
    The map $\varphi$, called the {\em
    factorization} of $f$,
    is always a smooth bijection from
    $\X/f$, equipped with the quotient diffeology, to
    $f(\X)$, equipped with the subset diffeology.
    Then, saying that $f$ is  strict is equivalent to say that $\varphi$ is
    a diffeomorphism. In other words, a strict smooth map
    identifies the quotient $\X/f$ with $f(\X)$.
  \end{article} %% Strict-maps-between-quotients-and-subspaces
  
  \begin{proof}
    Since $f$ is smooth and $p$ is a subduction, 
    we know that $\varphi$ is smooth and, by construction, injective. We have
    just to express that $\varphi^{-1}$ is smooth. Then,
    let $\P : \U \to f(\X)$ be a plot, that is, a plot $\P$
    of $\X$ with values in $f(\X)$
    \art{Subspaces-and-subset-diffeology}. The
    parametrization $\varphi^{-1} \circ \P$ is a plot of
    $\X/f$ if and only if $\varphi^{-1} \circ \P$ lifts
    locally along the projection $p$, at every point $r$ of
    $\U$ \art{What-is-a-subduction}. Thus, $\varphi^{-1}$ is
    smooth if and only if for all $r$ in $\U$ there
    exist an open neighborhood $\V$ of $r$ and a plot 
    $\Q : \V \to \X$ such that 
    $p \circ \Q = \varphi^{-1} \circ \P \restriction \V$, \ie, 
    $\varphi \circ p \circ \Q = \P \restriction \V$, 
    that is, $f \circ \Q = \P \restriction \V$.
  \end{proof}
  
  %%%%%%%%%%%%%%%%%%%%%%%%%%%%%%%%%%%%%%%%%%%%%%%%%%%%%%%%%%
  %
  %   Exercises  
  %
  %%%%%%%%%%%%%%%%%%%%%%%%%%%%%%%%%%%%%%%%%%%%%%%%%%%%%%%%%%
  
  \Exercises
  
  \begin{exercise}[Quotients of discrete or coarse spaces] 
    \label{Quotients-of-discrete-or-coarse-spaces}
    Check that any quotient of any discrete diffeological space
    \art{Discrete-diffeology} is discrete. As well, any quotient of
    any coarse space is coarse.
  \end{exercise} %% Quotients-of-discrete-or-coarse-spaces
  
  \begin{exercise}[Examples of quotients] 
    \label{Examples-of-quotients}
    Give a few examples, met in the 
    previous exercises, of diffeological quotients.
  \end{exercise} %% Examples-of-quotients
  
  \begin{exercise}[The irrational solenoid]
    \label{The-irrational-solenoid}
    The following example of the {\em irrational solenoid}, or {\em
    Kronecker flow}, mixes inductions and subductions. Let $\Torus^2$ be
    the 2-torus, defined as the quotient space of $\RR^2$ by the subgroup
    $\ZZ^2$. In other words, two points $(x,y)$ and $(x',y')$ of $\RR^2$ are
    equivalent if there exist two integers $n$ and $m$ such
    that $x' = x+n$ and $y' = y+m$. 
    
    \Question{1)} Check that the map $q : (x,y) \mapsto (p(x), p(y))$, with
    $p(t) = (\cos(2\pi t),\sin(2\pi t))$, from $\RR^2$ to $\RR^2 \times \RR^2$,
    is strict \art{Strict-maps-between-quotients-and-subspaces}, and identifies
    $\RR^2/\ZZ^2$ with $\S^1 \times \S^1 \subset \RR^2 \times \RR^2$.
    
    \Question{2)} Let $\Delta_\alpha \subset \RR^2$ be the line
    $\{(x,\alpha x) \mid x \in \RR \}$, with $\alpha \in \RR - \QQ$.
    Let $\cS_\alpha = q(\Delta_\alpha) \subset \S^1
    \times \S^1$. The subspace $\cS_\alpha$
    is the {\em irrational solenoid} with slope $\alpha$.
    Show that $q_\alpha = q \restriction \Delta_\alpha$ is an induction,
    and thus that $\cS_\alpha \subset \S^1 \times \S^1$ is
    diffeomorphic to $\RR$.
    
    \Question{3)} Consider the space $\S^1 \times \S^1$ as a group, after identification
    with $\RR^2/\ZZ^2$, for the operation inherited from the addition in $\RR^2$. 
    Note that  $\cS_\alpha$ is a subgroup.
    Show that the quotient $(\S^1 \times
    \S^1)/\cS_\alpha$ is diffeomorphic to the irrational torus $\Torus_\alpha$
    defined in \exref{Diffeomorphisms-between-irrational-tori}. 
  \end{exercise} %% The-irrational-solenoid
  
  \begin{exercise}[A minimal powerset diffeology]
    \label{A-minimal-powerset-diffeology}
    Exhibit, for all diffeological spaces $\X$, 
    a diffeology of the powerset $\Powerset(\X)$, 
    that induces
    for any equivalence relation $\sim$ defined on $\X$,
   the quotient diffeology on $\quotient{\X}{\sim}$.
    Let us recall that the canonical projection $\class : \X \to \Powerset(\X)$ 
    is defined by $\class(x) = \{ x' \in \X \mid x' \sim x \}$, and that 
    $\quotient{\X}{\sim} = \class(\X) \subset \Powerset(\X)$.
  \end{exercise} %% A-minimal-powerset-diffeology
  
  \begin{exercise}[Universal construction]
    \label{Universal-construction}
    Let $\X$ be a diffeological space, and $\cD$ be its diffeology. Let
    $\cN$ be the diffeological sum
    \art{Building-sums-with-spaces}
    $$
    \cN = \coprod_{\P \in \cD} \Dom(\P) 
    = \{ (\P, r) \mid \P \in \cD \mbox{ and } r \in \Dom(\P) \},
    $$
    where the domains of the plots are
    equipped with the standard diffeology. 
    
    \Question{1)} Check that the map $\ev : \cN \to \X$ defined 
    by $\ev(\P,r) = \P(r)$ is a subduction. In other
    words, show that $\X$ is equivalent to the quotient $\cN/\ev$. 
    
    \Question{2)} Is the subset of $\cN$ made up of the $0$-plots a smooth section?
    
    \Note~In other words, every diffeological space
    is the quotient of the union of some real domains by
    some equivalence relation. The set $\cN$ defined above is
    called the {\em Nebula} of the diffeology $\cD$, see
    \art{Nebula-of-a-generating-family}. This exercise shows
    the important role played by the two operations: sum and
    quotient, in diffeology.
  \end{exercise} %% Universal-construction
  
  \begin{exercise}[Strict action of $\SO(3)$ on $\RR^3$]
    \label{Strict-action-of-SO(3)-on-R3}
    Let us consider the  space $\RR^3$, equipped with smooth diffeology. Let
    $\SO(3)$ be the group of all $3\times 3$ real matrices
    ${\M}$ such that ${\M}^t\M = \id$, where ${\M}^t$ is the transposed
    matrix of ${\M}$, and $\det(\M) = +1$. 
    The group $\SO(3)$ is a part of the real space $\RR^9$,
    each matrix ${\M}$ is defined by its 9 components $\M_{ij}$, $\M =
    \sum_{i,j = 1}^3 \M_{ij} \ee_{ij}$, where $\ee_{ij}$ is the matrix with
    a 1 at the line $i$ and the column $j$ and 0 elsewhere. The group
    $\SO(3)$ is equipped with the subset diffeology of $\RR^9$. 
    Let $\X$ be some point of $\RR^3$, show that the {\em orbit map} 
    $\orb{\X} : \SO(3) \to \RR^3$, with $\orb{\X}(\M) = \M\X$,
    is strict \art{Strict-maps-between-quotients-and-subspaces}.
  \end{exercise} %% Strict-action-of-SO(3)-on-R3
  
  %%%%%%%%%%%%%%%%%%%%%%%%%%%%%%%%%%%%%%%%%%%%%%%%%%%%%%%%%%
  
  \section*{Products of Diffeological Spaces}
  \label{section-Products-of-diffeological-spaces}
  
  \begin{sechead}
    As well as for direct sums of spaces \art{Building-sums-with-spaces},
    the category $\Diffeology$ is closed for products. In other words,
    there exists a natural diffeology, called the {\em product
    diffeology},  on the product of every family of diffeological spaces. We
    give here its definition \art{Building-products-with-spaces}, and we
    describe then some related constructions. 
  \end{sechead}
    
  \begin{reminder} First of all, let us recall the formal construction 
    of the product for any family of sets \cite{Bou72}. Let 
    $\cE = \{\E_i\}_{i \in \cI}$ be an arbitrary family of sets,
    $\cI$ being any set of indices. The set $\cI$ can be the
    set $\cE$ itself, in this case one says that the family is
    self-indexed. The product of the elements of the family
    $\cE$ is the set of all the maps $\sigma$ which
    associate with each index $i \in \cI$  some element
    $\sigma(i) \in \E_i$. In other words, let $\coprod_{i
    \in \cI} \E_i$ be the sum of the members of $\E$, then
    $$
    \coprod_{i \in \cI} \E_i = \{(i,e) \in \cI\times
    \cup_{i \in \cI} \E_i \mid e \in \E_i\}.
    $$
    Let us denote by $\pr_1$ the projection from $\coprod_{i \in \cI}
    \E_i$ onto the first factor: 
    $$
    \pr_1 : \coprod_{i \in \cI} \E_i \to \cI \qmbox{with} \pr_1(i,e) =
    i. $$
    The product of the
    members of the family $\cE = \{\E_i\}_{i \in \cI}$ is the
    set of all the sections of the projection $\pr_1$, that
    is,  
    $$
    \prod_{i \in \cI}\E_i = \{ s: \cI \to \coprod_{i \in
    \cI}\E_i \mid \pr_1 \circ s = \id_\cI \}.
    $$
    For any index $i \in \cI$, we denote by $\pi_i$ the
    natural projection
    $$
    \pi_i : \prod_{i \in \cI} \E_i \to \E_i \qmbox{with}
    \pi_i(s) = \pr_2(s(i)), $$
    where $\pr_2 : (i,e) \mapsto e$ is the projection from $\coprod_{i \in
    \cI}\E_i$ onto the second factor. In other words,
    $\pi_i(s) = e$ if and only if $s(i) = (i,e)$. 
    For the sake of simplicity, we shall denote
    sometimes the product $\prod_{i \in \cI}\E_i$ by
    $\prod\cE$. Note that, if the set of indices $\cI$ is
    finite, $\cI = \{1,\ldots,{\N}\}$, then the product
    $\prod_{i \in \cI}\E_i$ is denoted by
    $\E_1\times\ldots\times \E_{\N}$.
  \end{reminder}
  
  \begin{article}\artlabel{Building products with spaces}
    \addcontentsline{toc}{section}{\small\hspace{10pt} Building products with spaces}
    \label{Building-products-with-spaces}
    Let $\{\X_i\}_{i \in \cI}$ be a family of diffeological
    spaces, indexed by some set $\cI$. 
    Let us denote by $\cD_i$, $i \in \cI$, their diffeologies. Then, there exists
    on the product 
    $$
    \X = \prod_{i \in \cI} \X_i
    $$ 
    a coarsest diffeology $\cD$ such that, for each index $i \in \cI$, the
    projection $\pi_i: \prod_{i \in \cI} \X_i \to \X_i$
    is smooth. This diffeology is called the {\em product
    diffeology} of the family $\{\cD_i\}_{i \in \cI}$. The
    set $\X$, equipped with the product diffeology, is called
    the {\em diffeological product}, or simply the {\em
    product}, of the $\X_i$.
    
    \Note~The product diffeology $\cD$ is the intersection, over all
    indices $i \in \cI$, of the pullbacks  by the
    projection $\pi_i$ of the diffeologies $\cD_i$.
    The plots for the product diffeology are the
    parametrizations $\P: \U \to \prod_{i \in \cI} \X_i$
    such that, for each $i \in \cI$, the parametrization
    $\pi_i \circ \P$ is a plot of $\X_i$. 
    In other words,
    a plot $\P : \U \to \prod_{i \in \cI} \X_i$ is a family
    $\{\P_i : \U \to \X_i\}_{i \in \cI}$ such that $\P_i \in \cD_i$, for all $i \in \cI$.
    $$
    \cD = \bigcap_{i \in \cI} \pi_i^*(\cD_i).
    $$ 
    In particular, if $\cI$ is a finite
    set of indices, $\cI = \{1,\ldots,{\N}\}$, then the plot $\P$ is an 
    $\N$-uple $(\P_1,\ldots, \P_\N)$, with $\P_i \in \cD_i$. 
  \end{article} %% Building-products-with-spaces
  
  \begin{proof}
    For each $i \in \cI$, there exists a coarsest diffeology
    of $\prod_{i \in \cI} \X_i$ such that the projection
    $\pi_i$ is smooth, it is the pullback $\pi_i^*(\cD_i)$
    \art{Pullback-of-diffeologies}. Then, the coarsest
    diffeology such that $\pi_i$ is smooth, for any
    $i \in \cI$, contains at least the intersection of all
    these pullbacks. But the intersection of any family of
    diffeologies is a diffeology
    \art{Intersecting-diffeologies}, and this intersection
    is the infimum of all the diffeologies containing the 
    $\pi_i^*(\cD_i)$. This infimum is clearly a minimum for
    the property in question. Therefore, the diffeology
    product is just the intersection  $\cD = \cap_{i \in \cI}
    \pi_i^*(\cD_i)$. 
    
    Now, the condition for $\P$ to be a plot of $\X$, stated in the
    proposition, is clearly necessary. It just means that each projection is
    smooth. The three axioms of diffeology
    \art{Diffeologies-and-diffeological-spaces} for this set of
    parametrizations are inherited from the diffeology of each
    element  of the family $\{\X_i\}_{i \in \cI}$. Thus, this condition
    defines a diffeology $\cD$ on $\X$ such that each projection $\pi_i$ is
    smooth. On the other hand, every other diffeology for
    which each projection $\pi_i$ is smooth
    satisfies this condition, so it is contained in $\cD$.
    Therefore, this diffeology $\cD$ is indeed the product
    diffeology. \end{proof}
  
  \begin{article}\artlabel{Projections on factors are subductions}
    \addcontentsline{toc}{section}{\small\hspace{10pt} Projections on factors are subductions}
    \label{Projections-on-factors-are-subductions}
    Let  $\X = \prod_{i \in \cI} \X_i$ be the diffeological product of some
    family $\set{\X_i}_{i \in \cI}$ of diffeological spaces. Every
    projection $\pi_i :  \X \to \X_i$ \art{Building-products-with-spaces}
    onto each factor is a subduction \art{What-is-a-subduction}.
  \end{article} %% Projections-on-factors-are-subductions
  
  \begin{proof}
    Let ${\P}: \U \to \X_k$ be a plot. 
    For every index $i \neq k$, let $x_i$ be a point of $\X_i$. The
    parametrization $\PP : \U \to \X$ defined by $\PP(r) = [i
    \mapsto (i,x_i) \mbox{ if } i \neq k \mbox{ and }
    (k,\P(r)) \mbox{ for } i = k]$ is a lift of $\P$, that is, 
    $\pi_k \circ \PP = \P$. Hence $\pi_k$ is a subduction,
    according to the characterization of the plots of a
    subduction \art{What-is-a-subduction}.
    Remark that the construction of $\PP$ uses the axiom of choice of the
    theory of sets. But, from the very beginning of the diffeology theory, this
    axiom has been implicit. 
  \end{proof}
  
  %%%%%%%%%%%%%%%%%%%%%%%%%%%%%%%%%%%%%%%%%%%%%%%%%%%%%%%%%%
  %
  %   Exercises  
  %
  %%%%%%%%%%%%%%%%%%%%%%%%%%%%%%%%%%%%%%%%%%%%%%%%%%%%%%%%%%
  
  \Exercises
  
  \begin{exercise}[Products and discrete diffeology] 
    \label{Products-and-discrete-diffeology} Check that for the discrete
    diffeology on the product $\prod_{i \in \cI} \X_i$
    \art{Building-products-with-spaces}, the
    projections $\pi_i$ are smooth. Link that remark to the
    discussion of \art{playing-with-bounds}. In the same spirit,
    comment the definition of the sum diffeology
    \art{Building-sums-with-spaces}.
  \end{exercise} %% Products-and-discrete-diffeology
  
  \begin{exercise}[Products of coarse or discrete spaces] 
    \label{Product-of-coarse-or-discrete-spaces} Check that any
    product of coarse spaces is coarse. 
    Show directly that finite products of discrete spaces are discrete.
    Using the result of the \exref{Locally-constant-parametrizations}, 
    show that actually every product of discrete spaces is discrete.
  \end{exercise} %% Product-of-coarse-or-discrete-spaces
  
  \begin{exercise}[Infinite product of $\RR$ over $\RR$] 
    \label{Infinite-product-of-R-over-R}
    Describe the sum diffeology of an infinite number of copies of $\RR$
    indexed by $\RR$. Describe then the product diffeology of an
    infinite number of copies of $\RR$ indexed by $\RR$.
  \end{exercise} %% Infinite-product-of-R-over-R
  
  \begin{exercise}[Graphs of smooth maps]
    \label{Graph-of-smooth-maps}  Let $\X$ and $\X'$
    be two diffeological spaces and $f: \X \to \X'$
    be any map. Let us denote by $\graph(f) \subset \X
    \times \X'$ the graph of $f$, that is, the subset of
    pairs $(x,f(x)) \in \X \times \X'$, where $x$ runs over $\X$, 
    equipped with
    the subset diffeology of the product diffeology. Show
    that $f$ is smooth if and only if the first projection 
    $\pr_\X: (x,x') \mapsto x$, restricted to
    $\graph(f)$, is a subduction \art{What-is-a-subduction}. 
  \end{exercise} %% Graph-of-smooth-maps
  
  \begin{exercise}[The 2-torus]
    \label{The-2-torus}
    Check that the diffeology of $\S^1 \times \S^1$
    described in the \exref{The-irrational-solenoid} is the
    product diffeology of $\S^1$ by $\S^1$.  
  \end{exercise} %% The-2-torus
  
  %%%%%%%%%%%%%%%%%%%%%%%%%%%%%%%%%%%%%%%%%%%%%%%%%%%%%%%%%%
  
  \section*{Functional Diffeology}
  \label{section-Functional-diffeology}
  
  \begin{sechead}
    A special and remarkable feature of
    diffeology is that the set of the smooth maps between
    two diffeological spaces $\X$ and $\X'$ 
    carries a natural diffeology, called the
    {\em functional diffeology}. This diffeology is used intensively
    everywhere, in the theory of diffeological spaces: 
    in homotopy theory, for fiber bundles, in Cartan's differential calculus
    etc. To be more precise, $\Cinfty(\X,\X')$ carries
    a whole family of functional diffeologies, all those
    such that the {\em evaluation map} $\ev : (f,x) \mapsto
    f(x)$ is smooth 
    \art{Functional-diffeologies}.
    But if some of them are however interesting,
    for example the compact controlled diffeology
    \art{Compact-controlled-diffeology}, the supremum of this
    family, that is, the {\em standard functional diffeology}, is the most used.
    Equipped with this functional diffeology, 
    the category of diffeological spaces is Cartesian closed.
  \end{sechead}
  
  \begin{article}\artlabel{Functional diffeologies}
    \addcontentsline{toc}{section}{\small\hspace{10pt} Functional diffeologies}
    \label{Functional-diffeologies} 
    Let $\X$ and $\X'$ be two diffeological spaces and let $\Cinfty(\X,\X')$ be the set
    of smooth maps from $\X$ to $\X'$
    \art{Smooth-maps}. Let $\ev$ be the {\em evaluation map}, defined by
    $$
    \ev :  \Cinfty(\X,\X') \times \X \to \X', \qmbox{and}
    \ev(f,x) = f(x).
    $$ 
    We shall call {\em functional diffeology} any
    diffeology of $\Cinfty(\X,\Y)$ such that the map $\ev$ is
    smooth. Note that the discrete diffeology is a
    functional diffeology. 
    But there exists a coarsest functional
    diffeology on $\Cinfty(\X,\X')$, we shall call it the
    {\em standard functional diffeology}, or simply {\em the
    functional diffeology}, when there will be no risk of
    confusion. Actually, the plots of this diffeology are explicitely 
    given by the following condition.
    \begin{itemize}
      \item[($\clubsuit$)] A parametrization $\P: \U \to
      \Cinfty(\X,\X')$ is a plot for the standard functional
      diffeology if and only if the map $(r,x) \mapsto \P(r)(x)$
      is smooth. That means that, for every plot $\Q: \V \to \X$, 
      $\P \cdot \Q: (r,s) \mapsto \P(r)(\Q(s))$ is a plot of $\X'$.
    \end{itemize}
  \end{article} %% Functional-diffeologies
  
  \begin{proof}
  First of all, let us consider the set
    $\Cinfty(\X,\X')$ equipped with the discrete diffeology.
    A plot of the product $\Cinfty(\X,\X') \times \X$ is some
    parametrization $r \mapsto (f,x)$ such that $r \mapsto
    f$ is locally constant \art{Discrete-diffeology} and $r
    \mapsto x$ is a plot of $\X$. Hence, the parametrization
    $r \mapsto f(x)$ is a plot of $\X'$ by the very
    definition of smooth maps
    \art{Smooth-maps}. Therefore, the set of
    functional diffeologies is not empty, and the discrete
    diffeology is a functional diffeology.
    
    Let us check the equivalence of conditions in ($\clubsuit$). 
    Let us assume that $(r,x) \mapsto \P(r)(x)$ is smooth, for every plot 
    $s \mapsto \Q(s)$ in $\X$, then the parametrization 
    $(r,s) \mapsto (r,\Q(s)) \mapsto \P(r)(\Q(s))$ is the composite of 
    smooth maps, thus a plot of $\X'$. Conversely, let us
    assume $\P \times \Q$ smooth for every plot $\Q$ of $\X$.
    Let $s \mapsto (r(s),\Q(s))$ be a plot of $\U \times \X$, then
    $[s \mapsto \P(r(s))(\Q(s))] = [s \mapsto (r(s),s) \mapsto \P(r(s))(\Q(s))]$ 
    is the composite of smooth maps, thus 
    $(r,x) \mapsto \P(r)(x)$ is smooth. 
    
    \alinea{D1.} This is equivalent to the fact that the discrete diffeology of
    $\Cinfty(\X,\X')$ is a functional diffeology
    \art{Functional-diffeologies}.
    
    \alinea{D2.} Let $\P: \U \to \Cinfty(\X,\X')$ be a
    parametrization satisfying locally ($\clubsuit$), that is,
    for every $r \in \U$, there exists an open neighborhood $\W$
    of $r$ such that for every plot $\Q: \V \to \X$, the
    parametrization 
    $(\P \restriction \W) \cdot \Q : \W \times \V \to \X'$ is a plot. 
    Hence, the parametrization $\P \cdot \Q$ is locally a plot of
    $\X'$. Therefore, $\P \cdot \Q$ is a plot of $\X'$ and
    the parametrization $\P$ satisfies ($\clubsuit$).
    
    \alinea{D3.} Let $\P : \U \to \Cinfty(\X,\X')$ be a parametrization
    satisfying ($\clubsuit$), and $\F: {\W} \to \U$ be a
    smooth parametrization. Let $\Q: \V \to \X$ be a plot.
    The parametrization $({\P} \circ \F) \cdot \Q: (t,s)
    \to {\P}(\F(t)(\Q(s))$ decomposes into $(t,s) \mapsto
    (\F(t),s) \mapsto {\P}(\F(t))(\Q(s))$. Now, $(\P \circ \F) \cdot \Q$ 
    is the composite of two smooth maps. Thus, 
    $(\P \circ \F) \cdot \Q$ is smooth. Therefore, $\P \circ \F$ satisfies ($\clubsuit$).
    
    Hence, the condition ($\clubsuit$) defines a diffeology on
    $\Cinfty(\X,\X')$. For this diffeology, the map $\ev$ is
    smooth. Indeed, let $\P \times \Q: r \mapsto
    (\P(r),\Q(r))$ be a plot of $\Cinfty(\X,\X') \times \X$.
    Then, $\ev \circ ({\P} \times \Q) = \P \cdot \Q = [r
    \mapsto {\P}(r)(\Q(r))]$ is a plot of $\X'$, by the
    very definition of the plot $\P$.
    
    Now, let us show that every diffeology of
    $\Cinfty(\X,\X')$ such that $\ev$ is smooth
    satisfies ($\clubsuit$). Since the map $\ev$ is
    smooth, for every plot $\P \times \Q : r \mapsto
    ({\P}(r),\Q(r))$ of $\Cinfty(\X,\X') \times \X$, the
    map $\P \cdot \Q : r \mapsto {\P}(r)(\Q(r))$ is a plot
    of $\X'$. Let ${\P}: \U \to \Cinfty(\X,\X')$ and $\Q:
    \V \to \X$ be two plots. The parametrizations
    ${\P}': (r,s) \mapsto {\P}(r)$ and $\Q': (r,s)
    \mapsto \Q(s)$, defined on $\U \times \V$, are
    two plots. Thus, the parametrization $\P' \cdot \Q' :
    (r,s) \mapsto {\P}'(r,s)(\Q'(r,s))$ is a plot of
    $\X'$. But, ${\P}'(r,s)(\Q'(r,s)) = {\P}(r)(\Q(s))$.
    Hence, ${\P}$ satisfies ($\clubsuit$). Thus, every
    diffeology of $\Cinfty(\X, \X')$ such that $\ev$ is
    smooth is contained in the diffeology defined by
    ($\clubsuit$). Therefore, this is the coarsest
    diffeology such that $\ev$ is smooth. 
  \end{proof}
  
  \begin{article}\artlabel{Restriction of the functional diffeology}
    \addcontentsline{toc}{section}{\small\hspace{10pt} Restriction of the functional diffeology} 
    \label{Restriction-of-the-functional-diffeology}
    Let $\X$ and $\X'$ be two diffeological spaces.
    Let $\Cinfty(\X,\X')$ be the set
    of smooth maps from $\X$ to $\X'$. Every subset
    $\cM \subset \Cinfty(\X,\X')$ inherits the functional
    diffeology \art{Subspaces-and-subset-diffeology}. The
    map $\ev_\cM = \ev \restriction (\cM \times \X)$ remains
    smooth, since it is the composition of a
    smooth map with an induction. The subset
    functional diffeology of $\cM$ is still the coarsest
    diffeology of $\cM$ such that $\ev_\cM$ is
    smooth \art{Functional-diffeologies}. This
    subset diffeology will be called again the {\em
    functional diffeology of $\cM$}. 
    
    \Note~As an example of such restriction, we can think of the
    subset of $2 \pi$-periodical smooth maps from $\RR$ to
    $\RR$, which inherits the functional diffeology.
  \end{article} %% Restriction-of-the-functional-diffeology
  
  \begin{article}\artlabel{The composition is smooth}
    \addcontentsline{toc}{section}{\small\hspace{10pt} The composition is
    smooth} \label{The-composition-is-smooth}
    Let $\X$, $\X'$ and $\X''$ be three diffeological spaces.  Let
    $\Cinfty(\X,\X')$, $\Cinfty(\X',\X'')$ and $\Cinfty(\X, \X'')$ be equipped
    with the standard functional diffeology \art{Functional-diffeologies}
    and $\Cinfty(\X,\X') \times \Cinfty(\X',\X'')$ with the
    product diffeology \art{Building-products-with-spaces}.
    Then, the {\em composition map} is smooth,
    $$
    \circ : \Cinfty(\X,\X') \times \Cinfty(\X',\X'') \to
    \Cinfty(\X, \X'') \qmbox{with} \circ (f,g) = g \circ f.
    $$
    
    \Note~This property, satisfied the smooths maps between manifolds, 
    was at the origin of Souriau's theory of {\em groupes diff\'erentiels} \cite{Sou80}, 
    what then led to the introduction of diffeology. 
  \end{article} %% The-composition-is-smooth
  
  \begin{article}\artlabel{Functional diffeology and products} 
    \addcontentsline{toc}{section}{\small\hspace{10pt} Functional diffeology and products} 
    \label{Functional-diffeology-and-products}
    Let $\X$, $\X'$ and $\X''$
    be three diffeological spaces. Let $\X \times \X'$ be
    equipped with the product diffeology \art{Building-products-with-spaces}. 
    Let $\Cinfty(\X', \X'')$, $\Cinfty(\X\times \X',\X'')$ and 
    $\Cinfty(\X,\Cinfty(\X', \X''))$ be equipped with the functional diffeology. 
    The spaces $\Cinfty(\X,\Cinfty(\X',\X''))$ and
    $\Cinfty(\X\times \X',\X'')$ are diffeomorphic. 
    The diffeomorphism $\phi$ consists in the game of parentheses,
    $$ 
      \phi(f): (x,x') \mapsto  f(x)(x'), \qmbox{for all} f \in
      \Cinfty(\X,\Cinfty(\X',\X'')). 
    $$ 
    We say that the category $\Diffeology$ is {\em Cartesian closed}.
  \end{article} %% Functional-diffeology-and-products
  
  \begin{proof}
    First of all, let us check that $\bmf = \phi(f) \in \Cinfty(\X
    \times \X', \X'')$. Since $f \in \Cinfty(\X, \Cinfty(\X', \X''))$, for
    every plot $\Q$ of $\X$, $r \mapsto f(\Q(r))$ is a plot of
    $\Cinfty(\X', \X'')$, which means that for every plot $\Q'$
    of $\X'$, $(r,s) \mapsto f(\Q(r))(\Q'(s))$ is a plot
    of $\X''$. But this implies that, for every plot $(\Q,\Q')$ of 
    $\X \times \X'$, $r \mapsto 
    \bmf(\Q(r),\Q'(r)) = f(\Q(r))(\Q'(r))$ is a plot
    of $\X''$. Then, $\bmf = \phi(f)$ is smooth, that is,
    belongs to $\Cinfty(\X \times \X', \X'')$. Now let us
    remark that $\phi$ is bijective, since $\bmf(x,x') =
    f(x)(x')$ is just a game of parentheses. Now, let us
    prove that $\phi$ is smooth. Let $\P$ be a plot of
    $\Cinfty(\X, \Cinfty(\X', \X''))$, that is, a parametrization
    such that for every plot $\Q$ of $\X$, the
    parametrization $(r,s) \mapsto {\P}(r)(\Q(s))$ is a
    plot of $\Cinfty(\X,\X')$. This means that for every plot
    $\Q'$ of $\X'$ the parametrization $(r,s,s') \mapsto
    \P(r)(\Q(s))(\Q'(s'))$ is a plot of $\X''$. But
    this is equivalent to the fact that, for every plot $\Q \times \Q'$ of 
    $\X \times \X'$, the parametrization $(r,s,s')
    \mapsto  \phi({\P}(r))(\Q(s),\Q'(s'))$ is a plot of
    $\X''$. Hence, $\phi \circ {\P}$ is a plot of $\Cinfty(\X
    \times \X', \X'')$. Therefore, $\phi$ is smooth. Since
    what we have said is completely reversible, that proves that
    $\phi$ is a diffeomorphism. 
  \end{proof}
  
  \begin{article}\artlabel{Functional diffeology on diffeomorphisms}
    \addcontentsline{toc}{section}{\small\hspace{10pt} Functional diffeology on groups of
    diffeomorphisms}
    \label{Functional-diffeology-on-groups-of-diffeomorphisms}
    Let $\X$ be a diffeological space. The group $\Diff(\X)$,
    as well as any of its subgroups, inherits the functional
    diffeology of $\Cinfty(\X,\X)$. Note that the structure of {\em
    diffeological group} \art{Diffeological-groups} of $\Diff(\X)$ is
    finer. It is the coarsest {\em group diffeology} such
    that the evaluation map is smooth
    \art{Functional-diffeologies}. A parametrization
    $\P$ of $\Diff(\X)$ is a plot of the {\em standard
    diffeology of group of diffeomorphisms} if
    $r \mapsto \P(r)$ and $r \to \P(r)^{-1}$ are plots for the 
    functional diffeology\footnote{For a manifold 
    \art{Manifolds-as-diffeologies}, the second condition,
    $r \to \P(r)^{-1}$ smooth, is the consequence of the first one,
    thanks to the implicit function theorem. For arbitrary
    diffeological spaces, it is not clear, it is why this condition is needed.}.
  \end{article} %% Functional-diffeology-on-groups-of-diffeomorphisms
  
  \begin{article}\artlabel{Slipping $\X$ into $\Cinfty(\Cinfty(\X,\X),\X)$} 
    \addcontentsline{toc}{section}{\small\hspace{10pt} Slipping $\X$ into
    $\Cinfty(\Cinfty(\X,\X),\X)$}   
    \label{Sliping-X-into-Cinf-Cinf-X-X-X} 
    Let $\X$ be a diffeological
    space. Let us associate with every $x \in \X$ the
    {\em $x$-evaluation map}
    $$
    \bmx : \Cinfty(\X,\X) \to \X \qmbox{with} \bmx: f \mapsto f(x).
    $$
    By definition of the functional diffeology
    \art{Functional-diffeologies}, the map
    $\bmx$ is smooth, that is, $\bmx \in
    \Cinfty(\Cinfty(\X,\X),\X)$. Moreover, the map $j: x \mapsto
    \bmx$ is an induction from $\X$ into
    $\Cinfty(\Cinfty(\X,\X),\X)$. We shall see further that $j$
    is stronger than an induction, but it is also an embedding
    \art{Embeddings}.
  \end{article} %% Sliping-X-into-Cinf-Cinf-X-X-X
  
  \begin{proof}
    First of all, applying this definition to the identity
    $f = \id_\X$ we get the injectivity of $j$,
    $\bmx(\id_x) = \bmx'(\id_\X)\Rightarrow x = x'$. Now,
    let us check that $j$ is smooth. Let $\Q$ be a plot of
    $\X$. The parametrization $j \circ \Q$ is a plot of
    $\Cinfty(\Cinfty(\X,\X),\X)$ if and only if, for every plot $\P$
    of $\Cinfty(\X,\X)$, the map $(r,s) \mapsto (j \circ
    \Q(r))(\P(s))$ is a plot of $\X$. But, by definition, 
    $(j \circ \Q(r))(\P(s)) = \P(s)(\Q(r))$, and to be a plot of
    $\Cinfty(\X,\X)$ means precisely that, for every plot $\Q$ of
    $\X$, the parametrization $(r,s) \mapsto \P(r)(\Q(s))$ is
    a plot of $\X$ \art{Functional-diffeologies}. Thus, $j$
    is smooth. Finally let us check that the pullback of the
    diffeology of $\Cinfty(\Cinfty(\X,\X),\X)$ by $j$ is finer
    than the diffeology of $\X$. Let $\Q$ be a
    parametrization in $\X$, such that $j \circ \Q$ is a plot
    of $\Cinfty(\Cinfty(\X,\X),\X)$. Then $(r,f) \mapsto
    f(\Q(r))$ is smooth. Restricting to $f = \id_\X$, $r
    \mapsto (r,\id_\X) \mapsto  \id_\X(\Q(r)) = \Q(r)$,
    we conclude that $\Q$ is a plot of $\X$. Therefore, by
    application of \art{Criterion-for-being-an-induction},
    $j$ is an induction. 
  \end{proof}
  
  \begin{article}\artlabel{Functional diffeology of a diffeology} 
    \addcontentsline{toc}{section}{\small\hspace{10pt} Functional diffeology of a diffeology} 
    \label{Functional-diffeology-of-a-diffeology}
    Every diffeology, regarded as a set of smooth maps, carries a
    natural diffeology, which is a specialization of the
    functional diffeology described above
    \art{Functional-diffeologies}. Pick a diffeological
    space $\X$ and let $\cD$ be its diffeology. The set of
    parametrizations $\P : \U \to \cD$ satisfying the
    following condition is a diffeology.
    \begin{itemize}
      \item[($\diamondsuit$)] For all $r_0 \in \U$, for all $s_0
      \in \Dom(\P(r_0))$, 
      there exist an open neighborhood $\V
      \subset \U$ of $r_0$ and an open neighborhood $\W$ of $s_0$
      such that $\W \subset \Dom(\P(r))$
      for all $r \in \V$, and $(r,s) \mapsto \P(r)(s)$,
      defined on $\V \times \W$, is a plot of $\X$.
    \end{itemize}
    Note that a plot $\P$ of $\cD$ takes locally its
    values in the set of plots of $\X$ with same dimension
    \art{Parametrizations-in-sets}. Also note that the last
    condition writes again $[r \mapsto \P(r) \restriction \W]
    \restriction \V \in \Cinfty(\V, \Cinfty(\W,\X))$. This is
    why this diffeology will be called the {\em standard
    functional diffeology} of $\cD$. It will play
    afterwards a role in  some constructions. 
    
    To simplify
    the vocabulary, and to avoid confusing statements, 
    we shall prefer sometimes the wording {\em smooth family of plots}
    instead of a {\em plot of the diffeology}, when the diffeology is
    regarded itself as a diffeological space.
  \end{article} %% Functional-diffeology-of-a-diffeology
  
  \begin{proof}
    For a constant parametrization $\P : s \mapsto \Q$,
    with $\Q  \in \Cinfty(\U,\X)$, we can take $\W = \U$.
    Hence, the axiom D1 of diffeology is satisfied. The axiom
    D2 is satisfied by the very definition. Now, let $\P: \U
    \to \cD$ satisfying ($\diamondsuit$), and let 
    $\F: \U' \to \U$ be a smooth parametrization. Then, let 
    $r'_0 \in \U'$ and $r_0 = \F(r'_0)$. Let $\V$ be an open neighborhood
    of $r_0 \in \U$ satisfying ($\diamondsuit$), and let 
    $\V' = \F^{-1}(\V)$. Then, $\V'$ is a domain satisfying
    ($\diamondsuit$), with the same $\W$. Thus, the axiom D3
    is satisfied and ($\diamondsuit$) defines a diffeology on
    the set $\cD$. 
  \end{proof}
  
  \begin{article}\artlabel{Iterating paths} 
    \addcontentsline{toc}{section}{\small\hspace{10pt} Iterating paths} 
    \label{Iterating-paths}
    Functional diffeology is heavily used in the theory of
    homotopy of diffeological spaces
    \sect{Section-Connectedness-and-Homotopy-category}, in
    particular through the following construction of the
    iterated spaces of paths. Let $\X$ be any diffeological
    space.  The {\em space of paths} in $\X$, denoted by
    $\Paths(\X)$, is defined  by
    $$
    \Paths(\X) = \Cinfty(\RR,\X).
    $$ 
    Now, let us define the following
    recurrence,
    $$ 
    \DPaths{0}(\X) = \X, \mbox{ and }
    \DPaths{p}(\X) =  \Paths(\DPaths{p-1}(\X)), \mbox{ for all } p>0.
    $$
    Said differently,  
    $$ 
    \DPaths{1}(\X) = \Paths(\X), \ \DPaths{2}(\X) =  \Paths(\Paths(\X)), 
    \ \mbox{etc.}
    $$
    The spaces $\DPaths{p}(\X)$ will be called\footnote{In reference to 
    Chen's paper {\em Iterated paths integrals} \cite{Che77}, see also the 
    {\em Afterword}.} the {\em
    iterated spaces of paths} in $\X$. For each integer $p$,
    $\Paths_p(\X)$ is equipped with the functional
    diffeology.  Let $ j_p : \sigma \mapsto \bsigma$ be
    the map defined, for all integers $p \geq 0$, from
    $\DPaths{p}(\X)$ to $\Maps(\RR^p,\X)$,
    by
    \begin{equation}
      \renewcommand{\theequation}{$\diamondsuit$}
      \bsigma: (x_1,\ldots,x_p)
      \mapsto \sigma(x_1)\cdots(x_p), \quad \mbox{for all } \sigma \in \DPaths{p}(\X).
    \end{equation}
    Then, the map $j_p$ takes its values in $\Cinfty(\RR^p,\X)$, and is a
    diffeomorphism from $\DPaths{p}(\X)$ onto $\Cinfty(\RR^p,\X)$.  
    
    \Note~The injection $j_\X : \X \to \Paths(\X)$
    which associates with each point $x$ the constant path
    $\bmx : t \mapsto x$ is not just smooth but it is also
    an induction. Indeed, for any plot $\P : \U \to
    \Paths(\X)$ such that $\P(\U) \subset j_\X(\X)$,
    $j_\X^{-1} \circ \P = [r \mapsto \P(r)(0)]$ is
    smooth. This is also clear for the iterated space of
    paths.
  \end{article} %% Iterating-paths
  
  \begin{proof}
    Let us prove the proposition for $p=0$. In this
    case, the map $j_0$ is given by $x \mapsto \bmx = [0 \mapsto x]$,
    and maps $\Paths_0(\X) = \X$ in $\Maps(\RR^0,\X)$. First of all,
    let us remark that every map from $\RR^0 = \{0\}$ to $\X$ is constant,
    hence smooth. Thus $\Maps(\RR^0,\X) = \Cinfty(\RR^0,\X)$,
    and $j_0 : \X \to \Cinfty(\RR^0,\X)$. Now, $j_0$ is
    bijective, let us show that $j_0^{-1}$ is
    smooth. Let $\P : \U \to \X$ be a plot, $j_0
    \circ \P : r \mapsto [0 \mapsto \P(r)]$ and $\Q : \V \to
    \RR^0$ be some plot. But $\Q$ is constant, $\Q(s) = 0$,
    and $\P(r)(\Q(s))= \P(r)(0) = \P(r)$. Thus, $(r,s)
    \mapsto \P(r)(\Q(s)) = \P(r)$ is a plot of $\X$ and
    $j_0$ is smooth \art{Functional-diffeologies}.
    
    Conversely, let $\P : \U \to \Cinfty(\RR^0,\X)$ be a plot
    of the functional diffeology. Since the evaluation map
    is smooth \art{Functional-diffeologies}, the map
    $j_p^{-1} \circ \P = [r \mapsto \P(r)(0)]$ is
    smooth, and hence is a plot of $\X$. Thus,
    $j_0^{-1}$ is smooth and $j_0$ is a
    diffeomorphism from $\X$ to $\Cinfty(\RR^0,\XX)$, equipped
    with the functional diffeology. 
    
    Now, let us complete the
    proof by a recurrence. Let us assume the proposition is
    true for some $p \geq 0$. Now $\Paths_{p+1}(\X) = \Paths(\Paths_p(\X)) =
    \Cinfty(\RR, \Paths_p(\X)) \simeq \Cinfty(\RR, \Cinfty(\RR^p, \X))$. But
    thanks to  \art{Functional-diffeology-and-products}, $\Cinfty(\RR,
    \Cinfty(\RR^p, \X)) \simeq \Cinfty(\RR \times \RR^p, \X)$. Hence,
    $\Paths_{p+1}(\X) \simeq \Cinfty(\RR^{p+1}, \X)$. Thus,
    the proposition is still true for $p+1$.
  \end{proof}
  
  \begin{article}\artlabel{Compact controlled diffeology}
    \addcontentsline{toc}{section}{\small\hspace{10pt} Compact controlled diffeology}
    \label{Compact-controlled-diffeology}
    Functional diffeology can be used as a formal framework for the
    {\em variational calculus}. Let us exemplify this claim by
    the simple classical problem  of the 
    extremals of the \guillemots{energy functional}
    $$
    \E(\gamma) = \undemi \int \norm{\dot \gamma(t)}^2 \ dt, 
    \qmbox{with} \gamma \in \Paths(\RR^n) 
    \qmbox{and} \dot \gamma(t) = {d \gamma(t) \over dt}.
    $$
    As we know, this integral does not converge in general, 
    but we can avoid this difficulty by just changing the
    diffeology of the space of paths. Let us begin 
    by computing informally the variation of this integral 
    for a smooth $1$-parameter 
    family of paths $s \mapsto \gamma_s$, and let $\gamma = \gamma_0$. 
    The variable $s$ is supposed to belong to some open neighborhood of
    $0 \in \RR$. We get
    $$
    \left.{\partial \E(\gamma_s) \over \partial s}\right|_{s=0} = 
    \int \bigg\langle \dot \gamma(t) \operatorname{,} \left.{\partial \dot \gamma_s(t) 
    \over \partial s}\right|_{s=0} \bigg\rangle \ dt = 
    \int \bigg\langle \dot \gamma(t) \operatorname{,} {d \over dt} 
    \left\{{\partial \gamma_s(t) \over \partial s}
    \right\}_{s=0} \bigg\rangle \ dt,
    $$
    Which we can write also, informally
    \begin{equation}\renewcommand{\theequation}{$\bullet$}
    d\E(\delta\gamma) = 
    \int \bigg\langle \dot \gamma(t) \operatorname{,} {d \over dt} 
   \delta \gamma(t) \bigg\rangle \ dt, \qmbox{with} 
   \delta\gamma(t) = \left. {\partial \gamma_s(t) \over \partial s}
    \right|_{s=0}.
    \end{equation}
    As usual in variational calculus, if we assume that the variation 
    $s \mapsto \gamma_s$ is compact supported, that is, if $\gamma_s$ coincides
    with $\gamma$ outside an interval $[a,b]$, then the integral will converge, 
    since the variation $\delta\gamma$ vanishes outside $[a,b]$. This consideration suggest 
    to equip the space $\Paths(\RR^n)$ with the {\em compact diffeology}, 
    defined in \exref{Compact-diffeology}. But we need to differentiate 
    under the integral sign, and with boundaries moving with the parameter $s$, 
    this is a little bit uncomfortable.
    To secure the place we introduce a new diffeology, finer than the 
    compact diffeology. 
    Let $\X$ and $\Y$ be two diffeological spaces. The parametrizations
    of the functional diffeology
    $\P : \U \to \Cinfty(\X,\Y)$, satisfying the following condition, 
    form the {\em compact controlled diffeology} \cite{Igl87}.
    \begin{itemize} 
      \item[($\spadesuit$)] For all $r_0 \in \U$ there exist an
      open neighborhood $\V$ of $r_0$ and a D-compact $\K$ of $\X$ such that,
      for all $r \in \V$, $\P(r)$ and $\P(r_0)$ coincide outside $\K$. 
    \end{itemize} 
    A D-compact means a compact for the D-topology 
    \art{The-D-Topology-of-diffeological-spaces}. 
    The difference with \exref{Compact-diffeology} is subtle, when $r$ 
    runs around $r_0$ the compact diffeology implies that there exists a 
    compact $\K_r$, {\em a priori\/} depending on $r$, 
    such that $\P(r)$ and $\P(r_0)$ coincide outside of $\K_r$. 
    The compact controlled diffeology, as far as it is concerned,
    implies that we can find one large compact $\K$ matching the 
    condition for all $r$ belonging to some small open ball 
    centered at $r_0$. 
    
    Coming back to our problem, these are exactly the conditions of the
    variational calculus. Let 
    $\P : \U \to \Paths(\RR^n)$ be an $m$-plot for 
    the compact controlled diffeology. Adjusting the above
    expression of the variation of the energy, denoted by 
    $d\E(\P)_r(\delta r)$, where $r$ belongs to $\U$ 
    and $\delta r$ is any vector of $\RR^m$,
    and after a classical integration by parts, we get 
    $$
     d\E(\P)_r(\delta r) = - 
      \int_a^b \bigg\langle {d^2 \over dt^2} \bigg\{\P(r)(t)\bigg\} 
      \operatorname{,} 
      {\partial \P(r)(t) \over \partial r}(\delta r) \bigg\rangle \ dt,
      $$
    where $[a,b]$ is some interval satsfying the
    condition ($\spadesuit$) for $\P$ in an open neighborhood of $r$. 
    Now, this last expression of $d\E$,
    mapping every plot $\P : \U \to \Paths(\RR^n)$ 
    to the $1$-form $d\E(\P)$ of $\U$, is a well defined 
    (closed) differential $1$-form of $\Paths(\RR^n)$,
    for the compact controlled diffeology 
    \art{Differential-forms-on-diffeological-spaces} and gives to ($\bullet$) 
    its formal status. 
    The critical points of the energy are the zeros of $d\E$. 
    In our very simple case, this $1$-form is indeed exact, 
    but the choice of a primitive depends on an 
    arbitrary constant on which $d\E$ does not depend.
  \end{article} %% Compact-controlled-diffeology 
  
  \begin{proof}
    The proof that ($\spadesuit$) defines a diffeology is a simple adaptation 
    of the proof of \exref{A-discrete-image-of-R}.
  \end{proof}
  
  %%%%%%%%%%%%%%%%%%%%%%%%%%%%%%%%%%%%%%%%%%%%%%%%%%%%%%%%%%
  %
  %   Exercises  
  %
  %%%%%%%%%%%%%%%%%%%%%%%%%%%%%%%%%%%%%%%%%%%%%%%%%%%%%%%%%%
  
  \Exercises
  
  \begin{exercise}[The space of polynomials]
    \label{The-space-of-polynomials}
    Let us consider the space of polynomials with coefficients in $\RR^m$,
    and degree less or equal than $n$,
    $$ \Pol_n(\RR, \RR^m) = \{
    t \mapsto x_0 + t x_1 + \ldots + t^n x_n \mid t \in \RR, \mbox{ and }
    x_0,\ldots,x_n \in \RR^m \}.
    $$ 
    
    \Question{1)} Show that, the map $j_n$, defined from $(\RR^m)^{n+1}$ to
    $\Cinfty(\RR, \RR^m)$  by
    $$
    j_n(x_0,x_1,\ldots,x_n) = [t \mapsto x_0 + t x_1 + \ldots + t^n x_n],
    $$
     where $\Cinfty(\RR, \RR^m)$ is equipped with the functional diffeology
    \art{Functional-diffeologies}, is an induction. Conclude that $\Pol_n(\RR, \RR^m)$,
    equipped with the functional diffeology, inherited from
    $\Cinfty(\RR,\RR^m)$, is diffeomorphic to $(\RR^m)^{n+1}$.
    
    \Question{2)} Let $\omega \subset (\RR^m)^{n+1}$ be any domain.
    Show that there exists a subset $\Omega$ of $\Cinfty(\RR,\RR^m)$ such that: 
    \begin{itemize}
      \item[($\diamondsuit$)] $\Omega \cap \Pol_n(\RR, \RR^m) =
      j_n(\omega)$,
      \item[($\heartsuit$)] for every $n$-plot
      $\P$ of $\Cinfty(\RR,\RR^m)$, $\P^{-1}(\Omega)$ is an $n$-domain.
    \end{itemize} 
    We say that the subset $\Pol_n(\RR, \RR^m)$ is {\em embedded} in
    $\Cinfty(\RR,\RR^m)$ \art{Embeddings}. 
    
    \Note~Polynomials are just defined by their
    coefficients. Since the coefficients belong to some
    vector space, we decide usually to carry arbitrarily
    the structure of (a power of) the vector space to the set of
    polynomials. But polynomials are, above all, maps between
    vector spaces. This exercise shows how diffeology
    can give its smooth structure considering only 
    the functional diffeology of polynomials. 
  \end{exercise} %% The-space-of-polynomials
  
  \begin{exercise}[A diffeology for the space of lines]
    \label{A-diffeology-for-the-space-of-lines}
    Let $\PL(\RR^n)$ be the set of {\em parametrized
    lines} in $\RR^n$, defined as the set of all polynomials of $\RR^n$ of
    degree strictly equal to 1  (see
    \exref{The-space-of-polynomials}). 
    
    \Question{1)} Show that two lines $f$ and $g$ have
    the same images in $\RR^n$ if and only if $g(t) = f(at + b)$, where $a$
    and $b$ are two real numbers, and $a \neq 0$. 
    
    \Question{2)} Check that 1) defines an action of the affine group
    of $\RR$, denoted by $\Aff(\RR)$, on $\PL(\RR^n)$, that is, $(a,b) : f
    \mapsto [t \mapsto f(at +b)]$. 
        
    \Question{3)}     Let $\Aff_+(\RR) \subset \Aff(\RR)$ be the group
    of {\em direct affine transformations}, the ones for which $a>0$.
    Let then $\UL_+(\RR^n) = \PL(\RR^n)/\Aff_+(\RR)$ be the set of {\em oriented non parametrized
    lines}, equipped with the quotient diffeology.  
    Show that the space $\PL(\RR^n)$ is diffeomorphic to $\RR^n \times
    (\RR^n - \{0\})$, use the result of \exref{The-space-of-polynomials}.
        
    \Question{4)}     Let $\rho$ be the map $(x_0,x_1) \mapsto (r,u)$ defined
    by 
    $$
    r = x_0 - (x_0 \cdot u) \times
    u \qmbox{with}  u = {x_1 \over \norm{x_1}}, 
    $$
    where the $\cdot$ denotes the ordinary scalar product in $\RR^m$, and
    $\norm{*}$ denotes the associated norm. 
    Show that $\rho$ is a smooth
    section of the projection $\pi : \PL(\RR^n) \to \UL_+(\RR^n)$
    \art{Section-of-a-quotient}. Conclude that the
    space of oriented non parametrized lines $\UL_+(\RR^n)$, equipped with
    the diffeology specified in 2), is diffeomorphic to the subspace
    $\T\S^{n-1} \subset \RR^n \times \RR^n$ defined by
    $$
    \T\S^{n-1} = \{ (u,r) \in \RR^n \times \RR^n \mid \norm{u} = 1 \mbox{
    and } u \cdot r = 0 \}. 
    $$
    Deduce that the space $\Lines(\RR^2)$ of unparametrized and 
    non oriented lines of $\RR^2$ is diffeomorphic to the M\"obius strip, 
    that is, the quotient of $\S^1 \times \RR$ by the equivalence relation 
    $(u,r) \sim (-u,-r)$. 
  \end{exercise} %% A-diffeology-for-the-space-of-lines
  
  \begin{exercise}[A diffeology for the space of circles] 
    \label{A-diffeology-for-the-space-of-circles}
    Write down an exercise of the same type as
    \exref{A-diffeology-for-the-space-of-lines}, for
    circles instead of lines. 
  \end{exercise} %% A-diffeology-for-the-space-of-circles
  
  %%%%%%%%%%%%%%%%%%%%%%%%%%%%%%%%%%%%%%%%%%%%%%%%%%%%%%%%%%
  
  \section*{Generating Families}
  \label{Section-Generating-families}
  
  \begin{sechead}
    The diffeologies can be built by {\em generating
    families}. Any family of parametrizations of a set
    generates a diffeology. Conversely, any diffeology is
    generated by some set of parametrizations
    \art{Generating-diffeology}. This mode of construction of
    diffeologies is very useful since it can reduce the analysis
    of the properties of a diffeological space to a subset of
    its plots, sometimes smaller than the whole diffeology.
    This definition leads to the definition of the dimension
    of a diffeological space
    \art{Dimension-of-a-diffeological-space}, which
    is a first global invariant of the category
    $\Diffeology$. But this construction also leads to the
    introduction of important subcategories of diffeological
    spaces, for example the category of manifolds
    \art{Diffeological-Manifolds}, or
    the category of orbifolds
    \art{Orbifolds-as-diffeologies} etc. 
  \end{sechead}
  
  \begin{article}\artlabel{Generating diffeology}
    \addcontentsline{toc}{section}{\small\hspace{10pt} Generating diffeology}
    \label{Generating-diffeology} 
    Let $\X$ be a set. Pick a set $\cF$ of parametrizations
    of $\X$, that is, $\cF \subset \Param(\X)$. There exists
    a finest diffeology containing $\cF$, this diffeology
    will be called the  {\em diffeology generated} by $\cF$
    and will be denoted by $\gen{\cF}$. 
    Conversely, let $\X$ be a diffeological space and $\cD$ be its
    diffeology. A family $\cF$ of plots of $\X$, which generates the
    diffeology $\cD$, will be called a {\em generating family of the
    diffeology $\cD$}, or a {\em generating family for the space $\X$}. The
    set of all the generating families of the diffeology $\cD$ will be
    denoted by $\Gen(\mathop\cD)$, or by $\Gen(\X)$. 
    $$
    \Gen(\mathop\cD) = \{ \cF \subset \cD \mid \gen{\cF} = \cD \}, \
    \Gen(\X) = \Gen(\cD). $$
  \end{article} %% Generating-diffeology
  
  \begin{proof}
    The finest diffeology of a set $\X$ containing a family $\cF$ of
    parametrizations is the intersection of all diffeologies
    containing $\cF$ \art{Intersecting-diffeologies}.
    \begin{equation}
      \renewcommand{\theequation}{$\diamondsuit$}
      \gen{\cF} = \bigcap_{\cD \in \DD}\cD \qmbox{with} \DD = \{\cD \in \Diffeoset(\X) \mid \cF \subset \cD\}.
    \end{equation}
    Note that the set $\DD$
    of all diffeologies containing $\cF$ is not empty since it contains, at
    least, the coarse diffeology.
  \end{proof}
  
  \begin{article}\artlabel{Generated by the empty family}
    \addcontentsline{toc}{section}{\small\hspace{10pt} Generated by the empty family}
    \label{generated-by-the-empty-family} 
    Let $\X$ be a set, and let us consider the empty family
    $\cF = \varnothing$ as a subset of $\Param(\X)$. 
    The diffeology of $\X$ generated by the 
    empty family \art{Generating-diffeology} is the discrete diffeology 
    \art{Discrete-diffeology},
    that is, the intersection of all diffeologies of $\X$.
    Hence, considering the set $\X$,
    $\gen{\varnothing} = \discrete{{\cD}}(\X)$.
  \end{article} %% generated-by-the-empty-family
  
  \begin{article}\artlabel{Criterion of generation}
    \addcontentsline{toc}{section}{\small\hspace{10pt} Criterion of generation}
    \label{Criterion-of-generation} 
    Let $\X$ be a set, and $\cF$ be a family of
    parametrizations of $\X$. The plots of the
    diffeology generated by
    $\cF$ are characterized by the following property
    \fig{fig-Smooth-local-lift-of-a-parametrization}.
    %%###########
    \begin{figure}[tb]
      \centerline{\includegraphics{Figures-PDF/fig-local-lifting.pdf}}
      \caption{Smooth local lift of a parametrization.}
      \label{fig-Smooth-local-lift-of-a-parametrization}
    \end{figure}
    %%###########
    \begin{itemize} 
      \item[($\clubsuit$)]  A parametrization
      $\P : \U \to \X$ is a plot for the diffeology
      generated by a family $\cF$ if and only if, 
      for all $r \in \U$, there exists an open neighborhood $\V$ of $r$ such
      that either $\P \restriction \V$ is a constant
      parametrization, or there exist a parametrization  
      $\F : \W \to \X$ belonging to $\cF$, and a smooth
      parametrization $\Q: \V \to \W$ such that 
      $\P \restriction \V = \F \circ \Q$.  
    \end{itemize}
    Now, if the union of the images of the elements $\F$ of $\cF$ covers $\X$,
    that is, if $\X = \cup_{\F \in \cF} \Val(\F)$ (we shall say that $\cF$ is
    a {\em parametrized cover} of $\X$), the criterion above is reduced to the
    following.
    \begin{itemize}
      \item[($\spadesuit$)]
      A parametrization $\P : \U \to \X$ is a plot for the
      diffeology generated by a parametrized cover $\cF$ if and
      only if, for every point $r$ in $\U$, there exist an 
      open neighborhood $\V$ of $r$, a parametrization  $\F : {\W} \to
      \X$ belonging to $\cF$ and a smooth parametrization
      $\Q : \V \to {\W}$ such that ${\P} \restriction \V = \F
    \circ \Q$. \end{itemize}
    A generating family $\cF$ which is a parametrized cover will
    be called a {\em covering generating family}. The criterion above can
    be rephrased as follows: let $\cF$ be a covering generating family for
    $\X$, a parametrization $\P$ is a plot of $\X$ if and only if it {\em
    lifts locally}, at each point, along an element of $\cF$. 
  \end{article} %% Criterion-of-generation
    
  \begin{proof}
    Let us check first, for the second case ($\spadesuit$), that the defined
    set of parametrizations of $\X$ is a diffeology
    \art{Diffeologies-and-diffeological-spaces}. Since $\cF$ is a parametrized cover,
    the plots satisfying ($\spadesuit$) cover $\X$ 
    (see \exref{Equivalent-axiom-of-covering}), and the covering axiom D1 is satisfied.
    The locality axiom D2 is satisfied by the very definition. Then, let ${\P}: \U \to \X$
    be a parametrization satisfying the condition
    ($\spadesuit$). Let $\phi: \U' \to \U$ be a smooth
    parametrization, where $\U'$ is some domain. Let 
    $\P' = \P \circ \phi$, and $r' \in \U'$, let $r = \phi(r')$.
    Let $\F : {\W} \to \X$ belonging to $\cF$, and $\Q :
    \V \to {\W}$ such that $\F \circ \Q = {\P} \restriction
    \V$, according to ($\spadesuit$). Now, $\F \circ \Q \circ
    \phi =  (\P \restriction \V) \circ \phi = (\P \circ \phi)
    \restriction \V'$, where $\V' = \phi^{-1}(\V)$. Thus,
    $\P' \restriction \V' = \F \circ \Q'$, with $\Q' = \Q
    \circ \phi$. Therefore, $\P'$ satisfies ($\spadesuit$) 
    and the locality axiom D3 is satisfied.
    
    In the general case described by 
    ($\clubsuit$) everything works as well as for
    ($\spadesuit$), except for D3, where there is something to check. If $\P$
    is a constant parametrization defined on an open neighborhood $\V$
    of $r$ then, $\phi$ being continuous, $\V' =
    \Q^{-1}(\V)$ is an open neighborhood of $r'$, and ${\P}
    \circ \phi \restriction \V' = {\P}\restriction \V$ is
    constant. The axiom D3 is checked.
    
    Now, since ($\spadesuit$) is a special case of ($\clubsuit$), we
    consider only the case ($\clubsuit$). This property
    defines a diffeology containing $\cF$. By
    application of axioms D1 and D3 of diffeology, any
    other diffeology containing $\cF$ contains either the
    constant parametrizations or the plots of the form $\F \circ \Q$ 
    where $\F \in \cF$ and $\Q$ is a smooth
    parametrization in the domain of $\F$. Thus, the
    diffeology defined by ($\clubsuit$) is contained in any
    diffeology containing $\cF$. Hence, it is the finest
    diffeology containing $\cF$, that is, the diffeology
    generated by $\cF$.
  \end{proof}
  
  \begin{article}\artlabel{Generating diffeology as projector}
    \addcontentsline{toc}{section}{\small\hspace{10pt} Generating diffeology as projector}
    \label{Generating-diffeology-as-projector} 
    Let
    $\X$ be a set and $\cD$ be a diffeology of $\X$. Since
    the finest diffeology of $\X$ containing $\cD$ is $\cD$
    itself, the diffeology generated by $\cD$
    \art{Generating-diffeology} is $\cD$. In other words, for
    all $\cD \in \Diffeoset(\X)$, $\gen{\cD} = \cD$. More
    precisely, the construction of diffeologies by means of
    generating families can be interpreted as a {\em
    projector} from the set of all the subsets of
    $\Param(\X)$ onto the set of all the diffeologies of
    $\X$, which we can write
    $$\gen{\cdot} : \Powerset(\Param(\X)) \to
    \Diffeoset(\X) \qmbox{and} \gen{\gen{\cdot}} =
    \gen{\cdot}.
    $$ 
  \end{article} %% Generating-diffeology-as-projector
  
  \begin{article}\artlabel{Fineness and generating families}
    \addcontentsline{toc}{section}{\small\hspace{10pt} Fineness and generating families}
    \label{fineness-and-generating-families}
    Let $\X$ be a set, let $\cF$ and $\cF'$ be two families of
    parametrizations of $\X$ such that $\cF\subset \cF'$.
    The diffeology generated by
    ${\cF}$ is finer than the diffeology generated by ${\cF}'$,
    $$
    \cF\subset \cF' \mbox{ implies } \gen{\cF} \subset \gen{\cF'}.
    $$
    In other words, generating diffeologies by means of families of
    parametrizations is monotonic, relating to the partial ordering
    defined by inclusion.
    In particular, let $\X$ be a diffeological space. Let
    $\cF$ be any family of plots of $\X$, that is, a subset of
    the diffeology $\cD$ of $\X$. The diffeology generated by
    $\cF$ is finer than the original $\cD$, 
    $$
    \cF\subset \cD \mbox{ implies } \gen{\cF} \subset \cD.
    $$
  \end{article} %% fineness-and-generating-families
  
  \begin{proof}
      The diffeology generated by $\cF$ is the intersection of all
      the diffeologies $\cD$ containing $\cF$, that is,
      $$
      \gen{\cF} = \bigcap_{\cD \in \DD} \cD \qmbox{with} \DD = \{ \cD \in
      \Diffeoset(\X) \mid \cF \subset \cD \}. $$
      Let us split the set $\DD$ into the following two disjoint sets $\cA$
      and $\cB$: 
      \begin{eqnarray*} \cA & = & \{ \cD \in
        \Diffeoset(\X) \mid \cF \subset \cD \mbox{ and } \cF' \subset \cD \}, \\
        \cB & = & \{ \cD \in \Diffeoset(\X) \mid \cF \subset \cD 
        \mbox{ and } \cF' \not \subset \cD \}.
      \end{eqnarray*}
      Thus, $\DD = \cA \cup \cB$, and $\cA \cap \cB = \varnothing$.  Then, by
      associativity of the intersection, $\gen{\cF}$ is the intersection
      of the diffeologies belonging to ${\cA}$ and the diffeologies belonging
      to ${\cB}$, that is, 
      $$
      \gen{\cF}  = \bigcap_{\cD \in \DD} \cD =  \bigcap_{\cD \in \cA \cup
      \cB} \cD =  \bigg( \bigcap_{\cD \in \cA} \cD \bigg) \bigcap \bigg(
      \bigcap_{\cD \in \cB} \cD \bigg). 
      $$
      But since $\cF \subset \cF'$, any diffeology containing $\cF'$ contains
      necessarily $\cF$, and the set ${\cA}$ is just the set of all the
      diffeologies containing $\cF'$,
      $$
      \cA = \{ \cD \in
      \Diffeoset(\X) \mid \cF' \subset \cD \}.
      $$
      Hence, 
      $$\bigcap_{\cD \in \cA} \cD = \gen{\cF'} \ \Rightarrow \
      \gen{\cF} =
      \gen{\cF'} \bigcap \bigg( \bigcap_{\cD \in \cB}
      \cD \bigg).$$
      Therefore, $\gen{\cF} \subset \gen{\cF'}$, and the first part of the
      proposition is complete. Now, since $\gen{\cD} = \cD$
      \art{Generating-diffeology-as-projector}, if $\cF' = \cD$ 
      is a diffeology, then $\gen{\cF}\subset \cD$. 
    \end{proof}
    
  \begin{article}\artlabel{Adding and intersecting families}
    \addcontentsline{toc}{section}{\small\hspace{10pt} Adding and intersecting families} 
    \label{Adding-and-intersecting-families}
    Let $\X$ be a set. Let $\cF$ and $\cF'$ be two families of
    parametrizations of $\X$, then
    $$
    \gen{\cF \cup \cF'} = \gen{\gen{\cF} \cup \gen{\cF'}} \mbox{
    and } \gen{\cF \cap \cF'} \subset \gen{\cF} \cap \gen{\cF'}. 
    $$
  \end{article} %% Adding-and-intersecting-families
  
  \begin{proof}
    Let us prove the first assertion. Let us check first that $\gen{\cF
    \cup \cF'} \subset \gen{\gen{\cF}\cup\gen{\cF'}}$. Since
    $\cF\subset\gen{\cF}$ and $\cF'\subset\gen{\cF'}$, $\cF
    \cup \cF'\subset \gen{\cF}\cup\gen{\cF'}$. By application of
    \art{fineness-and-generating-families} we get $\gen{\cF \cup \cF'}
    \subset \gen{\gen{\cF}\cup\gen{\cF'}}$. Now, let us check that
    $\gen{\gen{\cF}\cup\gen{\cF'}} \subset \gen{\cF \cup
    \cF'}$. Since $\cF \subset \cF \cup \cF'$, $\gen{\cF} \subset
    \gen{\cF \cup \cF'}$ \art{fineness-and-generating-families}, as
    well $\gen{\cF'} \subset \gen{\cF \cup \cF'}$. Thus,
    $\gen{\cF} \cup \gen{\cF'} \subset \gen{\cF \cup \cF'}$.
    Hence $\gen{\gen{\cF} \cup \gen{\cF'}} \subset
    \gen{\gen{\cF \cup \cF'}} = \gen{\cF \cup \cF'}$
    \art{Generating-diffeology-as-projector}. Therefore, $ \gen{\cF \cup
    \cF'} = \gen{\gen{\cF}\cup\gen{\cF'}}$. 
    %
    Now concerning the second   assertion, since $\cF \cap \cF' \subset
    \cF$, $\gen{\cF \cap \cF'} \subset \gen{\cF}$
    \art{fineness-and-generating-families}. As well, $\cF \cap \cF'
    \subset \cF$, and then $\gen{\cF \cap \cF'} \subset \gen{\cF'}$.
    Hence, $\gen{\cF \cap \cF'} \subset
    \gen{\cF} \cap \gen{\cF'}$.
  \end{proof}
  
  \begin{article}\artlabel{Adding constants to generating family}
    \addcontentsline{toc}{section}{\small\hspace{10pt} Adding constants to generating
    family} 
    \label{Adding-constants-to-generating-family}
    Let $\X$ be a set, and $\cF$ be a family of parametrizations of $\X$.
    Let $\XX$ be some family of parametrizations generating the discrete
    diffeology, that is, $\gen{\XX} = \discrete{\cD}(\X)$, then
    $\gen{\cF} = \gen{\XX \cup \cF}$. 
    Thus, it is always possible to
    add a family of parametrizations, generating the discrete diffeology,
    to a generating family, without altering the
    generated diffeology. For example, considering the criterion
    \art{Criterion-of-generation}, extending the family $\cF$ by 
    the constant parametrizations $\XX = \Cinfty(\RR^0,\X)$ reduces
    the case ($\clubsuit$)  to the case ($\spadesuit$). We may
    denote by $\overline \cF = \Cinfty(\RR^0,\X) \cup \cF$ the extended
    family.
  \end{article} %% Adding-constants-to-generating-family
  
  \begin{proof}
    First of all, by  \art{fineness-and-generating-families} we have
    $\gen{\cF} \subset \gen{\XX \cup \cF}$. Now, since the diffeology
    generated by $\XX$ is the discrete diffeology, by
    \art{Adding-and-intersecting-families} we have $\gen{\XX \cup \cF} =
    \gen{\discrete{\cD}(\X) \cup \gen{\cF}}$. But, since every
    diffeology contains the discrete diffeology, 
    $\discrete{\cD}(\X) \cup \gen{\cF} = \gen{\cF}$, and $
    \gen{\XX \cup \cF} = \gen{\cF}$. 
  \end{proof}
  
  \begin{article}\artlabel{Lifting smooth maps along generating families}
    \addcontentsline{toc}{section}{\small\hspace{10pt} Lifting smooth maps along generating families} 
    \label{Lifting-smooth-maps-along-generating-families}
    Let $\X$ and and $\X'$ be two diffeological spaces.
    Let $\X$ be generated by a family
    $\cF$, and $\X'$ by a
    family $\cF'$. A map $f: \X \to \X'$ is
    smooth if and only if, for every element 
    $\F : \U \to \X$ of $\cF$, for every $r \in \U$ 
    there exists an open neighborhood
    $\V \subset \U$ of $r$, such that either 
    $f \circ \F \restriction \V$ is constant, or 
    there exist a parametrization $\F' : \U' \to \X'$ belonging
    to $\cF'$ and a smooth parametrization 
    $\phi : \V \to \U'$ such that 
    $f \circ \F \restriction \V = \F' \circ \phi$.
   The following commutative diagram summarizes this criterion of smoothness for diffeologies defined through
   generating families.
    \begin{center}
    \begin{tikzpicture}[auto]
      \node (A)  at (0,0) {$\U\supset\V$};
      \node (B)  at (3,0) {$\U'$};
      \node (C)  at (0,-2) {$\X$};
      \node (D)  at (3,-2) {$\X'$};
      \draw[->] (A) to node {$\phi$} (B);
      \draw[->] (A) to node [swap] {$\F$} (C);
      \draw[->] (B) to node {$\F'$} (D);
      \draw[->] (C) to node [swap] {$f$} (D);
    \end{tikzpicture}
    \end{center}
  \end{article} %% Lifting-smooth-maps-along-generating-families
  
  \begin{proof}
    Let us assume that $f$ is smooth, for every plot
    ${\P}: \U \rightarrow \X$ the map $f \circ {\P}$ is a
    plot of $\X'$, in particular
    for $\P = \F \in \cF$. Then, according to  
    \art{Criterion-of-generation}, since  $\cF'$ is a generating family of
    $\X'$, for every point $r$ of $\U$ there exists an open neighborhood 
    $\V$ of $r$ such that either the parametrization 
    $f\circ \P \restriction \V$ is
    constant or there exist an element $\F' : \U' \to \X'$ of $\cF'$, and
    a smooth parametrization $\phi : \V \to \U'$, such that $\F' \circ \phi
    = f \circ \F \restriction \V$.
    
    Conversely, let $\P: \cO \to \X$ be a plot and $s$ be a
    point in $\cO$. Since $\cF$ is a generating family for
    $\X$, there exists an open neighborhood $\W$ of $r$ such that
    either $\P \restriction \W$ is constant
    --- thus $f \circ \P \restriction \W$ is
    a constant parametrization, hence a plot of $\X'$ --- or
    there exist a plot $\F : \U \to \X$ belonging to $\cF$
    and a smooth parametrization $\Q : \W \to \U$ such that
    $\P \restriction \W = \F \circ \Q$. In the second case,
    since $\F$ is an element of $\cF$, by assumption there
    exists an open neighborhood $\V$ of $r = \F(s)$ such that either
    $f \circ \F \restriction \V$ is constant, 
    or there exist an element $\F' : \U' \to \X'$ of $\cF'$ 
    and a smooth parametrization $\phi : \V \to \U'$ 
    such that 
    $f \circ \F \restriction \V = \F' \circ \phi$. 
    Now, let $\W' = \Q^{-1}(\V) \cap \W$, the
    point $s$ belongs to $\W'$, and since $\Q$ is a smooth
    parametrization $\Q^{-1}(\V)$ is a domain, but $\W$ is
    also a domain, hence $\V' = \Q^{-1}(\W) \cap \V$ is a
    domain. Thus, $\P \restriction \W'$ is a parametrization
    in $\X$, defined on a neighborhood of $s$. Now, in the first
    case, $f \circ \P \restriction \W' = f \circ \F \circ \Q
    \restriction \W'$ is a constant parametrization, hence a
    plot of $\X'$. In the second case, $f \circ \P
    \restriction \W' = f \circ \F \circ \Q \restriction \W'
    = \F' \circ \phi \circ \Q \restriction \W'$. But, since
    $\F'$ is a plot of $\X'$ and $\phi$ and $\Q$ are smooth
    parametrizations,  $\F' \circ \phi \circ \Q$ is a plot
    of $\X'$. Finally, $f \circ \P$ is locally a plot of
    $\X'$ at each point $s$ of $\cO$. Therefore, $f \circ \P$
    is a plot of $\X'$ and then $f$ is smooth. 
  \end{proof}
  
  \begin{article}\artlabel{Pushing forward families}
    \addcontentsline{toc}{section}{\small\hspace{10pt} Pushing forward families}  
    \label{Pushing-forward-families}
    Let $\X$ and $\X'$ be two sets. Let $\cF$ be a family of
    parametrizations of $\X$ and $f : \X \to \X'$ be a map. We call {\em
    pushforward of the family $\cF$ by $f$}, the family  $f_*(\cF)$ of
    parametrizations of $\X'$, defined by
    $$
    f_*(\cF) = \left\{ f \circ \F \mid \F \in \cF \right\}.
    $$ 
    Then, the diffeology generated by the pushforward of the family
    $\cF$ by $f$
    is the pushforward by $f$ of the diffeology generated by $\cF$
    \art{Pushforward-of-diffeologies}, that is, 
    $$
    \gen{f_*(\cF)} = f_*(\gen{\cF}).
    $$
    In particular, let $\X$ and $\X'$ be two diffeological spaces and $f :
    \X \to \X'$ be a subduction \art{What-is-a-subduction}. The pushforward
    $f_*(\cF)$ of any generating family $\cF$ for $\X$ is a generating
    family for $\X'$.   
  \end{article} %% Pushing-forward-families
  
  \begin{proof}
    Let us denote $\cD = \gen{\cF}$, $\cF' = f_*(\cF)$ and
    $\cD' = \gen{\cF'}$. We want to show that $f_*(\cD) =
    \cD'$. First of all, let us remark that, for $\X$
    equipped with $\cD$ and $\X'$ equipped with $\cD'$, the
    map $f$ is smooth. Indeed, let $\P : \cO \to \X$
    be a plot of $\X$ and $r$ be a point of $\cO$. Then,
    according to \art{Criterion-of-generation}, at least one of the two following
    possibilities occurs. Either there exists an open neighborhood $\V$ of 
    $r$ such that the parametrization $\P \restriction \V$ is constant, 
    then $f \circ \P \restriction \V$ is constant, and 
    $f \circ \P \restriction \V$ belongs to $\cD'$. 
    Or there exists an open neighborhood $\V$ of $r$,
    an element $\F : \U \to \X$ of $\cF$ and a smooth
    parametrization $\Q : \V \to \U$ such that 
    $\P \restriction \V = \F \circ \Q$. Thus, 
    $f \circ \P \restriction \V = f \circ \F \circ \Q$, 
    but $\F' = f \circ \F \in \cF'$, hence 
    $f \circ \P \restriction \V = \F' \circ \Q$ belongs to $\cD'$.
    In the two cases, $f \circ \P$ is locally a plot of $\X'$ at each
    point of $\cO$, thus $f \circ \P$ is a plot of $\X'$.
    Therefore, $f$ is smooth, and since $f$ is
    smooth, $f_*(\cD) \subset \cD'$
    \art{Smoothness-and-pushforwards}.
    
    Let us show now that $ \cD' \subset f_*(\cD)$. Let $\P' : \cO' \to \X'$ 
    be a plot of $\X'$ and $r'$ be a point of
    $\cO'$. Then,
    according to \art{Criterion-of-generation}, at least one of the two following
    possibilities occurs. Either there exists an open neighborhood $\V'$ of $r'$ such that the
    parametrization $\P' \restriction \V'$ is constant, and then
    $\P' \restriction \V'$ belongs to  $f_*(\cD)$. 
    Or there exists an open neighborhood $\V'$ of $r'$, 
    an element $\F' : \U' \to \X'$ of $\cF'$
    and a smooth parametrization $\Q' : \V' \to \U'$ 
    such that $\P' \restriction \V' = \F' \circ \Q'$. But $\F' =
    f \circ \F$, for some $\F \in \cF$, thus 
    $\P' \restriction \V' = f \circ \F \circ \Q' = f \circ \Q$,
    with $\Q = \F \circ \Q'$. Next, since $\Q \in \cD$
    \art{Criterion-of-generation}, $\P' \restriction \V'$
    belongs to $f_*(\cD)$.  In the two cases $\P'$ is locally an element 
    of $f_*(\cD)$ at each point of $\cO'$, thus $\P'$ belongs to $f_*(\cD)$.
    Therefore,  $\cD' \subset f_*(\cD)$.
  \end{proof}
  
  \begin{article}\artlabel{Pulling back families}
    \addcontentsline{toc}{section}{\small\hspace{10pt} Pulling back families}  
    \label{Pulling-back-families}
    Let $\X$ and $\X'$ be two sets. Let $\cF'$ be a family of
    parametrizations of $\X'$ and let $f : \X \to \X'$ be any map. Let us define the {\em
    pullback of the family $\cF'$ by $f$} as the
    family $f^*(\cF)$ of parametrizations $\F : \U \to \X$ 
    satisfying the following property.
    \begin{itemize}
      \item[($\diamondsuit$)] Either $f\circ \F$ is constant or there exist an
      element $\F' : \U' \to \X'$ of $\cF'$ and a smooth
      parametrization $\phi : \U \to \U'$ such that $\F' \circ \phi = f \circ
      \F$.
    \end{itemize}
    Then, the diffeology generated by the pullback $f^*(\cF')$
    is the pullback by $f$ of the diffeology generated by $\cF$, that is, 
    $$
      \gen{f^*(\cF')} = f^*(\gen{\cF'}).
    $$
    In particular, let $\X$ and $\X'$ be two diffeological spaces and $f :
    \X \to \X'$ be an induction \art{What-is-an-induction}. The pullback
    $f^*(\cF')$ of any generating family $\cF'$ for $\X'$ is a generating
    family for $\X$. Unfortunately, compared with the pushforward of a family,
    pulling back a small generating 
    family may lead to a
    huge family, almost as big as the diffeology itself, see
    \exref{Generating-the-half-line}. This is what shows the
    \exref{Dimension-of-the-half-line}. 
    
    \Note~Concerning diffeologies, the choice of
    a generating family is relatively arbitrary. For example, 
    the empty family is equivalent to the 
    family of constant parametrizations.
    If the family $\cF'$ is empty, its pullback is not empty, but is the set
    of the parametrizations of $\X$ with values in the preimages of points $f^{-1}(x')$,
    $x' \in \X'$. This must not surprise us since the pullback of the discrete diffeology is
    the sum of the preimages of points, equipped with the coarse diffeology. 
  \end{article} %% Pulling-back-families
  
  \begin{proof}
    Let $\cF = f^*(\cF')$, 
    $\cD = \gen{\cF}$, and $\cD'= \gen{\cF'}$. We want to check that
    $\cD = f^*(\cD')$. First of all, let us remark that, for
    $\X$ equipped with $\cD$ and $\X'$ equipped with $\cD'$,
    the map $f$ is smooth. Indeed, let $\P : \cO \to \X$ be
    a plot and $r_0 \in \cO$. According to \art{Criterion-of-generation},
    there exists an open neighborhood $\V$ of $r_0$ such that either 
    $\P \restriction \V$ is constant or there exist $\F \in \cF$ and
    $\psi \in \Cinfty(\V, \Dom(\F))$ such that $\F \circ \psi = \P \restriction \V$.
    In the first case $f \circ (\P \restriction \V)$ is constant, thus 
    $f \circ (\P \restriction \V)$ is a plot of $\X'$.
    in the second case, according to the definition of $\cF = f^*(\cF')$, either 
    $f \circ \F$ is constant or there exist $\F' \in \cF'$ and 
    $\phi \in \Cinfty(\Dom(\F),\Dom(\F'))$ such that
    $\F' \circ \phi = f \circ \F$. If  $f \circ \F$ is constant, so is 
    $f \circ \F \circ \psi = f \circ (\P \restriction \V)$, 
    then $f \circ (\P \restriction \V)$ is a plot of $\X'$. If $\F' \circ \phi = f \circ \F$, 
    then $f \circ (\P \restriction \V) = f \circ \F \circ \psi = \F' \circ \phi \circ \psi$,
    and $f \circ (\P \restriction \V)$ is a plot of $\X'$.
    Therefore, $f \circ \P$ is a plot of $\X'$ (axiom of locality)
    and $f$ is smooth, that gives $\cD \subset f^*(\cD')$ \art{Smoothness-and-pullbacks}
    and thus $\gen{f^*(\cF')} \subset f^*(\gen{\cF'})$.
    
    Let us show next that $f^*(\cD') \subset \cD$. Let $\P : \cO \to \X$ 
    be an element of $f^*(\cD')$, that is,  $f \circ \P \in \cD'$. 
    Let $r_0 \in \cO$, according to \art{Criterion-of-generation} there
    exists an open neighborhood $\V$ of $r_0$ such that either $f \circ
      \P \restriction \V$ is constant or
    there exist a parametrization $\cF' : \U' \to \X'$
    belonging to $\cF'$, and a smooth parametrization $\phi :
      \V \to \U'$ such that $f \circ \P \restriction \V = \F'
        \circ \phi$. Let $\F = \P \restriction \V$, 
    we have just said that, either $f
      \circ \F$ is constant or $f \circ \F = \F' \circ \phi$.
    But this is exactly the definition ($\diamondsuit$) of the
    elements of the pullback $f^*(\cF') = \cF$. Hence, $\P$
    is locally an element of $\cF$ at each point of $\cO$.
    Hence, $\P$ is an element of $\cD$. Therefore 
    $f^*(\cD') \subset \cD$.
  \end{proof}
  
  \begin{article}\artlabel{Nebula of a generating family}
    \addcontentsline{toc}{section}{\small\hspace{10pt} Nebula of a generating family}
    \label{Nebula-of-a-generating-family}
    Let $\X$ be a diffeological space, and $\cF$ be a generating family for
    $\X$ \art{Generating-diffeology}. Let us assume that $\cF$ is a
    parametrized cover of $\X$ \art{Criterion-of-generation}. Note that, if
    it's not the case we can always add the constants
    $\Cinfty(\RR^0, \X)$ to $\cF$ and then consider the extended
    generating family $\overline \cF = \Cinfty(\RR^0, \X) \cup
    \cF$ \art{Adding-constants-to-generating-family}.
    We call {\em nebula of the generating family $\cF$} the diffeological
    sum \art{Building-sums-with-spaces} of the domains of the elements of
    $\cF$, where the domains of the elements of $\cF$ are
    equipped with the standard diffeology, that is,
    $$ \Nebula(\cF) =
    \coprod_{ \F \in \cF} \Dom(\F) = \left\{ (\F, r) \mid
    \F \in \cF \mbox{ and } r \in \Dom(\F) \right\}. $$
    Then, the natural evaluation map, denoted by $\pi_\cF$ and defined by 
    $$ 
    \pi_\cF : \Nebula(\cF) \to \X \qmbox{with} \pi_\cF(\F,r) = \F(r), 
    $$ 
    is a subduction. Therefore, the space $\X$ can be regarded as the
    quotient space $\Nebula(\cF)/\pi_\cF$ 
    \art{Quotient-and-quotient-diffeology}. 
    Now, let us define a {\em nebula over $\X$} as any diffeological
    sum $\cN$ of real domains together with a subduction $\pi : \cN \to
    \X$. Then, the nebula of any generating family $\cF$ of $\X$, together with
    $\pi_\cF$, is a nebula over $\X$. Conversely, every nebula over $\X$ is the
    nebula of a generating family of $\X$. The nebula associated with the
    diffeology $\cD$ of $\X$ is the maximal nebula of $\X$, every
    other nebula is some of its subspaces. 
    
    \Note~This construction shows that every 
    finite dimensional diffeological space $\X$ \art{Dimension-of-a-diffeological-space}
    is equivalent to the quotient of a manifold, maybe with variable dimension, see
    \art{Manifolds-as-diffeologies} and \xart{Diffeological-Manifolds}{note 2}. 
    Indeed, if $\dim(\X) = n < \infty$, 
    there exists a generating family $\cF$ made up
    of plots with dimensions less or equal to $n$, and 
    $\Nebula(\cF)$ is clearly a manifold, in a broad sense. 
  \end{article} %% Nebula-of-a-generating-family
  
  \begin{proof}
    Let us show that $\pi_\cF$ is
    smooth. Let $\Q : \cO \to \coprod_{ \F \in \cF} \Dom(\F)$ 
    be a plot of the nebula. By definition
    of the sum diffeology \art{Building-sums-with-spaces},
    for every $r$ in $\cO$, there exist an open neighborhood $\V$ of
    $r$ and an element $\F : \U \to \X$ of $\cF$, such that 
    $\Q \restriction \V$ is a smooth parametrization in $\U$.
    Now, $\pi_\cF \circ \Q \restriction \V = \F \circ \Q
    \restriction \V$, but since $\F$ is a plot of $\X$ and
    $\Q$ is a smooth parametrization in its domain,  $\F
      \circ \Q \restriction \V$ is a plot of $\X$. Thus,
    $\pi_\cF \circ \Q$ is locally, at each point of $\cO$, a
    plot of $\X$, that is, $\pi \circ \Q$ is a plot of $\X$,
    and $\pi_\cF$ is smooth. Now, let $\P : \cO \to
      \X$ be a plot of $\X$ and $r$ be a point of $\cO$. Since
    $\cF$ is a generating family for $\X$, there exist an open neighborhood
    $\V$ of $r$, an element $\F : \U \to \X$ of
    $\cF$, and a smooth parametrization $\Q : \V \to \U$
    such that $\P \restriction \V = \F \circ \Q$ 
    \art{Criterion-of-generation}. But
    $\Q' : r \mapsto (\F, \Q(r))$, defined on $\V$, is a plot of
    $\coprod_{ \F \in \cF} \Dom(\F)$, and 
    $\P \restriction \V = \pi_\cF \circ \Q'$. Thus, the
    plot $\P$ lifts locally at each point of its domain
    along $\pi_\cF$. Therefore, $\pi_\cF$ is a subduction
    \art{Criterion-for-being-a-subduction}. 
    
    Now, let $\cN$ be a nebula over $\X$. By
    definition $\cN = \coprod _{i \in \cI} \U_i$, where
    $\set{\U_i}_{i \in \cI}$ is a family of domains, and
    $\pi : \cN \to \X$ is some subduction. Then, let us define
    the family of parametrizations $\cF = \set{\F_i = \pi
      \restriction \U_i}_{i \in \cI}$. Since $\pi$ is
    smooth, the parametrizations are plots of $\X$,
    and since $\pi$ is a subduction, $\cF$ is a generating
    family for $\X$ (\cf\ the first part of the proof), thus
    $\cN = \Nebula(\cF)$. Therefore, every nebula over $\X$ is
    the nebula of a generating family.
  \end{proof}
    
  %%%%%%%%%%%%%%%%%%%%%%%%%%%%%%%%%%%%%%%%%%%%%%%%%%%%%%%%%%
  %
  %   Exercises  
  %
  %%%%%%%%%%%%%%%%%%%%%%%%%%%%%%%%%%%%%%%%%%%%%%%%%%%%%%%%%%
  
  \Exercises
  
  \begin{exercise}[Generating tori]
    \label{Generating-tori}
    Exhibit a minimal generating family for $\S^1$
    defined in \art{A-diffeology-for-the-circle}, for the irrational torus 
    $\Torus_\alpha$ defined in \exref{Diffeomorphisms-between-irrational-tori}, 
    and for $\RR/\QQ$, \exref{Smooth-maps-on-R/Q}. 
  \end{exercise} %% Generating-tori
  
  \begin{exercise}[Global plots as generating families]
    \label{Global-plots-as-generating-families}
    Let $\X$ be a diffeological space. Show that the set of global plots
    $\cP = \cup_{n \in \NN} \Cinfty(\RR^n,\X)$ is a generating family for
    $\X$.
  \end{exercise} %% Global-plots-as-generating-families
  
  \begin{exercise}[Generating the half-line]
    \label{Generating-the-half-line}
    Consider $\left[ 0, \infty \right[ \subset \RR$ 
    equipped with the subset diffeology
    inherited from the standard diffeology of $\RR$ 
    \art{Subspaces-and-subset-diffeology}. 
    Let $\cF$ be the generating
    family of $\RR$ reduced to the identity, $\cF =
    \set{\id_\RR}$. Show that the pullback of the generating
    family $\cF$ by the inclusion $j: \left[ 0, \infty
    \right[ \to \RR$ \art{Pulling-back-families} is the whole
    diffeology of $\left[ 0, \infty \right[$. 
  \end{exercise} %% Generating-the-half-line
  
  \begin{exercise}[Generating the sphere]
    \label{Generating-the-sphere}
    Let us consider the constructions and notations of
    \exref{The-sphere-as-diffeological-subspace}. For each point $x$ of
    $\S^n$, let us choose, once and for all, an orthonormal basis $(u_1,\ldots,u_n)$
    of $\E = x^\perp$. Let $\Ball$ be the open unit ball
    centered at $0 \in \RR^n$. Let $\F : \Ball \to \S^n$ be the
    parametrization defined by $\F(s_1,\ldots,s_n) = f(\sum_{i=1}^n s_i u_i)$ where
    $f$ is defined in that exercise by $(\diamondsuit)$. Show that the set of
    such $\F$, when $x$ runs over $\S^n$, is a generating
    family for $\S^n$. 
  \end{exercise} %% Generating-the-sphere
  
  \begin{exercise}[When the intersection is empty]
    \label{When-the-intersection-is-empty}
    Let us consider the two families $\cF = \left\{ x \mapsto x
    \right\}$ and $\cF' = \left\{ x \mapsto 2x \right\}$ of
    parametrizations in $\RR$. Check that $\gen{\cF \cap \cF'}
    = \discrete{\cD}(\RR)$ and $\gen{\cF} \cap \gen{\cF'} =
    \Cinfty_\star(\RR)$. Conclude that if  generally $\gen{\cF \cap \cF'}
    \subset \gen{\cF} \cap \gen{\cF'}$, it is not true that
    $\gen{\cF \cap \cF'} = \gen{\cF} \cap \gen{\cF'}$.
  \end{exercise} %% When-the-intersection-is-empty
  
  %%%%%%%%%%%%%%%%%%%%%%%%%%%%%%%%%%%%%%%%%%%%%%%%%%%%%%%%%%
  
  \section*{Dimension of Diffeological Spaces}
  \label{Section-Dimension-of-diffeological-spaces}
  
  \begin{sechead}
    The first numerical invariant of a
    diffeological space is its dimension. The {\em global dimension}
    of a diffeological space, introduced in this section, has a refinement
    given further \art{The-dimension-map}, 
    the {\em dimension map} of a diffeological space.
  \end{sechead}
  
  \begin{article}\artlabel{Dimension of a family of parametrizations}
    \label{Dimension-of-a-family-of-parametrizations} 
    Let $\X$ be a set and $\cF$ be some non empty family of
    parametrizations of $\X$. We define the {\em dimension}
    of $\cF$ as the supremum of the dimensions
    of its elements \art{Parametrizations-in-sets}, that is,
    $$
    \dim(\cF) = \sup \{ \dim(\F) \mid \F \in  \cF
    \},
    $$
    By convention, we shall admit that for the empty
    family $\dim(\varnothing) = 0$. On the other hand,
    if the dimension of the elements of $\cF$ is unbounded,
    that is, if for all $n \in \NN$ there exists an element
    $\F$ of $\cF$ such that $\dim(\F) = n$, we shall agree
    that the dimension of $\cF$ is infinite and we shall
    denote $\dim(\cF) = \infty$. In other words,
    $\dim(\cF) \in \NN \cup \set{\infty}$.
  \end{article} %% Dimension-of-a-family-of-parametrizations
  
  \begin{article}\artlabel{Dimension of a diffeological space} 
    \addcontentsline{toc}{section}{\small\hspace{10pt} Dimension of a
    diffeological space}
    \label{Dimension-of-a-diffeological-space} Let $\X$ be a
    diffeological space and let $\cD$ be its
    diffeology. We shall call {\em dimension of $\X$} the infimum
    of the dimensions of the generating families of $\cD$,
    $$
    \dim(\X) = \inf_{\gen{\cF} = \cD} \dim(\cF) = \inf
    \set{ \dim(\cF) \mid \cF \subset \cD \mbox{ and }
    \langle \cF \,\rangle = \cD }.
    $$  
    If the diffeology
    $\cD$ has no generating family with finite dimension, the
    dimension of $\X$ will be said infinite and we
    shall denote $\dim(\X) = \infty$.
  \end{article} %% Dimension-of-a-diffeological-space
  
  \begin{article}\artlabel{The dimension is a diffeological invariant}
    \addcontentsline{toc}{section}{\small\hspace{10pt} The dimension is a diffeological
    invariant} 
    \label{The-dimension-is-a-diffeological-invariant}
    Let $\X$ and $\X'$ be two diffeological spaces. If they  are
    diffeomorphic then they have the same dimension.
  \end{article} %% The-dimension-is-a-diffeological-invariant
  
  \begin{proof}
    Let $f: \X \to \X'$ be a diffeomorphism. Let
    $\cF$ be a generating family for $\X$. Clearly, 
    $f \circ \cF = \set{f \circ \F \mid \F \in \cF}$ 
    is a generating family for $\X'$
    \art{Pushing-forward-families}. Conversely, let $\cF'$ be
    a generating family of $\X'$, then $f^{-1} \circ \cF'$ is
    a generating family of $\X$. 
    There is a one-to-one correspondence between the generating families
    of $\X$ and $\X'$, therefore $\dim(\X) = \dim(\X')$. 
  \end{proof}
  
  \begin{article}\artlabel{Dimension of real domains}
    \addcontentsline{toc}{section}{\small\hspace{10pt} Dimension of real domains} 
    \label{dimension-of-real-domains} 
    The diffeological dimension of a non empty $n$-domain $\U$, 
    equipped with the standard diffeology
    \art{Real-domains-as-diffeological-spaces}, is $n$.
    In other words, for real domains, the
    diffeological dimension coincides with the usual dimension. 
    $$
    \mbox{If } \U \in \Domains(\RR^n), \mbox{ then } \dim(\U) = n.
    $$
  \end{article} %% dimension-of-real-domains
  
  \begin{proof}
    The singleton $\{\id_\U : \U \to \U\}$ is
    a generating family of $\U$, hence $\dim(\id_\U) = \dim(\U)
    \leq n$. Now, let us assume that 
    there exists a generating
    family $\cF$ of $\U$ such that $\dim(\cF) < n$. Then,
    for every $r \in \U$ there exist an open neighborhood $\V$ of $r$, an
    element $\F : \W \to \U$ of $\cF$, that is, $\F \in
    \Cinfty(\W,\U)$ ---~since $\F$ needs to factorize through the
    identity $\id_\U$~--- and a smooth map $\Q : \V \to \W$
    such that $\id_\U \restriction \V = \id_\V = \F \circ \Q$.
    But $\dim(\cF) < n$ implies that $\dim(\F)= \dim(\W) <n$.
    Now, the rank of the linear tangent map $\D(\F \circ \Q)$
    is less or equal to $\dim(\W) <n$, but $\D(\F \circ \Q) =
    \D(\id_\V) = \id_{\RR^n}$, thus $\rank(\D(\F \circ \Q)) =
    \rank(\id_{\RR^n}) = n$. Therefore, there is no
    generating family $\cF$ of $\U$ with $\dim(\cF)
    <n$, and $\dim(\U) = n$. \end{proof}
  
  \begin{article}\artlabel{Dimension zero spaces are discrete}
    \addcontentsline{toc}{section}{\small\hspace{10pt} Dimension zero spaces are discrete} 
    \label{dimension-zero-spaces-are-discrete} 
    The dimension of a
    diffeological space is zero if and only if it is 
    discrete. But note that a diffeological space may consist 
    of a finite number of points and have a
    non zero dimension,
    \exref{Has-the-set-0-1-dimension-1}. 
  \end{article} %% dimension-zero-spaces-are-discrete
  
  \begin{proof}
    Let $\discrete{\X}$ be a set $\X$, equipped with the
    discrete diffeology \art{Discrete-diffeology}. Every plot
    ${\P}: \U \to \X$, being locally constant, it lifts
    locally along some 0-plot $\bmx : 0 \to x$, where $x =
      {\P}(r)$ \art{Parametrizations-in-sets}. Hence,
    $\dim(\discrete{\X}) = 0$. Conversely, let $\X$ be a
    diffeological space such that $\dim(\X) = 0$. Then, by
    application of \art{Criterion-of-generation}, the 0-plots
    generate the diffeology $\cD$ of $\X$. Every plot lifting
    locally along a 0-plot is locally constant. Therefore,
    $\X$ is discrete. 
  \end{proof}
  
  \begin{article}\artlabel{Dimensions and quotients of diffeologies}
  \addcontentsline{toc}{section}{\small\hspace{10pt} Dimensions and quotients of diffeologies} 
    \label{Dimensions-of-quotients} Let $\X$ and $\X'$ be two
    diffeological spaces. If $\pi: \X \to \X'$ is a
    subduction, then $\dim(\X') \leq \dim(\X)$. 
    Put differently, for any equivalence relation $\cR$ defined on
    $\X$,  $$\dim(\X/\cR) \leq \dim(\X).$$ 
  \end{article} %% Dimensions-of-quotients
  
  \begin{proof}
    Let $\cD$ and $\cD'$ be the diffeologies of $\X$ and
    $\X'$. Let $\Gen(\cD)$ and $\Gen(\cD')$ denote the sets
    of all the generating families of $\cD$ and $\cD'$.  We
    know that for every generating family $\cF$ of $\cD$, $\pi
    \circ \cF$ is a generating family of $\cD'$, that is,
    $\pi \circ \Gen(\cD) \subset \Gen(\cD')$
    \art{Pushing-forward-families}.
    Now, for every plot $\P : \U \to \X$, $\dim(\P) =
    \dim(\U) = \dim(\pi \circ \P)$. Hence, for every
    generating family $\cF \in \Gen(\cD)$, $\dim(\cF) =
    \dim(\pi \circ \cF)$. It follows the series of inequalities,
    $$ 
    \begin{array}{lclcr}
      \dim(\X') & =    & \inf_{\cF' \in \Gen(\cD') }\dim(\cF')        && \mbox{by definition}  \\ 
                & \leq & \inf_{\cF \in \Gen(\cD) }\dim(\pi \circ \cF) &&  \mbox{since }  \pi \circ \Gen(\cD) \subset \Gen(\cD') \\ 
                & \leq & \inf_{\cF \in \Gen(\cD) }\dim(\cF) && \mbox{since }\dim(\pi \circ \cF) = \dim(\cF) \\ 
                &\leq & \dim(\X) && \mbox{by definition.} 
    \end{array} 
    $$    
    Therefore, $\dim(\X')\leq \dim(\X)$.
  \end{proof}
  
  \begin{article}\artlabel{Dimensions of a product}
    \label{Dimensions-of-a-product} 
    Let $\X$ and $\X'$ be two
    diffeological spaces. The dimension of the product $\X \times \X'$
    satisfies the following inequality\footnote{Thanks to the referee 
    of the manuscript who pointed out to me that the proof contained a 
    better lower bound than the one I gave in the statement.},
    $$
    \max(\dim(\X),\dim(\X')) \leq \dim(\X
    \times \X') \leq \dim(\X) + \dim(\X').
    $$ 
    Note that I still don't know at this moment, if we generally have the equality
    $\dim(\X \times \X') = \dim(\X) + \dim(\X')$, 
    or if there are counterexamples.
  \end{article} %% Dimensions-of-a-product

  \begin{proof}
    Let $\cF$ and $\cF'$ be two generating families, for $\X$
    and $\X'$. Since every plot of the product $\X \times \X'$ is a 
    pair $(\P, \P') \in \cD \times \cD'$, 
    where $\cD$ and $\cD'$ are the
    diffeologies of $\X$ and $\X'$, the product $\cF \times \cF'$ is a
    generating family for $\X \times \X'$. The elements 
    $(\F,\F') \in \cF \times \cF'$ define a parametrization
     $\F \times \F' : \U \times \U' \to \X
    \times \X'$ by $\F \times \F' (r,r') = (\F(r),\F'(r'))$.
    Then, 
    \begin{eqnarray*}
      \dim(\X \times \X') & = & \inf \set{ \dim(\F'') \mid \F'' \in \Gen(\X \times \X')} \\
      & \leq &  \inf \set{ \dim(\F \times \F') \mid (\F,\F') \in \Gen(\X) \times \Gen(\X')}  \\
      & \leq & \inf \set { \dim(\F) + \dim (\F')  \mid (\F,\F') \in \Gen(\X) \times \Gen(\X')} \\ 
      & \leq & \inf\set {\dim(\F) \mid \F \in \Gen(\X)} + \inf\set {\dim(\F') \mid \F' \in \Gen(\X')}  
      \\ & \leq & \dim(\X) + \dim(\X').
    \end{eqnarray*}
    Now, let $\cF''$ be a generating family for $\X \times
    \X'$. The set of parametrizations $\cF = \pr_\X
    \circ \cF''$, where $\pr_\X$ is the projection onto the first factor,
    is a generating family for $\X$. Indeed, let $x'_0 \in \X'$, every plot
    $\P : \U \to \X$ extends into a plot $\bar \P : r \mapsto (\P(r), x'_0)$
    of $\X \times \X'$. Thus, the plot $\bar \P$ lifts at each point along
    some element $\F''$ of $\cF''$, by composition with
    $\pr_\X$ we get a local lift of $\P$ along $\pr_\X
    \circ \F''$. The same holds for $\X'$, and $\cF' =
    \pr_{\X'} \circ \cF''$ is a generating family for $\X'$.
    For these generating families we have $\dim(\cF) =
    \dim(\cF'')$ and  $\dim(\cF') = \dim(\cF'')$. Thus, 
    \begin{eqnarray*} 
    \dim(\X \times \X') & = & \inf \set{\dim(\cF'') \mid \ \cF'' \in \Gen(\X \times \X') } \\
      & = & \inf \set{ \dim(\cF) \mid \ \cF = \pr_\X \circ \cF'', \cF'' \in \Gen(\X \times \X') } \\
      & \geq & \inf \set{ \dim(\cF) \mid  \cF \in \Gen(\X) } \\
      & \geq & \dim(\X).
    \end{eqnarray*}
    Exchanging $\X$ and $\X'$ 
    gives then $\dim(\X \times \X') \geq \max(\dim(\X),\dim(\X'))$. 
    The inequality is then complete.
  \end{proof}
  
  \begin{article}\artlabel{Dimension of a subspace?}
    \label{Dimension-of-a-subspace}
    There is no simple relation between the dimension of a diffeological
    space $\X$ and the dimension of its subspaces.
    The dimension of a subspace $\A \subset \X$ can be less
    than the dimension of $\X$, equal or even greater. The
    \exref{Dimension-of-the-half-line} is the simple example
    of the subspace $\left[0,\infty\right[ \subset \RR$,
    having infinite dimension, while
    $\RR$ has dimension 1.
  \end{article} %% Dimension-of-a-subspace
  
  %%%%%%%%%%%%%%%%%%%%%%%%%%%%%%%%%%%%%%%%%%%%%%%%%%%%%%%%%%
  %
  %   Exercises  
  %
  %%%%%%%%%%%%%%%%%%%%%%%%%%%%%%%%%%%%%%%%%%%%%%%%%%%%%%%%%%
  
  \Exercises
  
  \begin{exercise}[Has the set $\{0,1\}$ dimension 1?\/]
    \label{Has-the-set-0-1-dimension-1}
    Let $\pi: \RR \to \{0,1\}$ be the parametrization
    defined by $\pi(x) = 0$ if $x \in \QQ$ and $\pi(x) = 1$
    otherwise. Let $\set{0,1}_\pi$ be the set $\set{0,1}$
    representing the quotient $\RR/\pi$. Show
    that $\dim(\set{0,1}_\pi) =1$.  
  \end{exercise} %% Has-the-set-0-1-dimension-1
  
  \begin{exercise}[Dimension of tori]
    \label{Dimension-of-tori} 
    Let $\Gamma \subset \RR$ be any strict subgroup of
    $(\RR,+)$ and let $\Torus_\Gamma$ be the quotient
    $\RR/\Gamma$. Show that $\dim(\Torus_\Gamma) = 1$. 
    Note that this applies in particular both to the circles
    \art{A-diffeology-for-the-circle} and to the
    irrational tori, see
    \exref{Diffeomorphisms-between-irrational-tori} and
    \exref{Generating-tori}.
  \end{exercise} %% Dimension-of-tori
  
  \begin{exercise}[Dimension of $\RR^n/ \OG(n,\RR)$]
    \label{Dimension-of-Rn/On}
    Let $\Delta_n$ be the quotient
    space of $\RR^n$ by the action of
    the orthogonal group $\OG(n,\RR)$ 
    \art{Quotient-and-quotient-diffeology}, that is, two points $x$ and
    $x'$ of $\RR^n$ are equivalent if there exists an element
    $\A$ of $\OG(n,\RR)$ such that $x' = \A x$. 
    
    \Question{1)} Show that $\Delta_n$ is equivalent to the set $\left[0,\infty\right[$
    equipped with the pushforward of the standard diffeology of $\RR^n$ by
    the map $\nu_n : x \mapsto \norm{x}^2$. 
    
    \Question{2)} Show that the plot $\nu_n$ cannot be lifted locally at the point $0$
    along a $p$-plot, with $p < n$. Consider the tangent map of $\nu_n$ at
    the point $0$.
    
    \Question{3)} Deduce that $\dim(\Delta_n) = n$ and that $\RR^n/\OG(n,\RR)$ and
    $\RR^m/\OG(m, \RR)$, $n \neq m$, are not diffeomorphic.
    
    \Note~This exercise shows how a diffeology encodes a smooth structure
    and not just a set, or even a topology of a set. These quotients
    $\RR^n/\OG(n,\RR)$ are homeomorphic two to two, but indeed not 
    diffeomorphic if they are not identical.
  \end{exercise} %% Dimension-of-Rn/On
  
  \begin{exercise}[Dimension of the half-line]
    \label{Dimension-of-the-half-line}
    Let $\Delta_\infty$ be $\left[0,\infty\right[ \subset
    \RR$, equipped with the subset diffeology
    \art{Subspaces-and-subset-diffeology}. Show that
    $\dim(\Delta_\infty) = \infty$. Use a similar
    development than in exercise \exref{Dimension-of-Rn/On}
    and the fact that for any integer $n$, the map $\nu_n$
    is a plot of $\Delta_\infty$.
    
    \Note~This exercise, and the previous one, show in particular that the quotient 
    $\Delta_1 = \RR/\{\pm 1\}$, which has a structure of an orbifold
    \art{Orbifolds-as-diffeologies}, is not a manifold 
    with boundary \art{Classical-manifolds-with-boundary}, 
    which is the case of the half-line 
    $\Delta_\infty = \left[0,\infty\right[ \subset \RR$.
  \end{exercise} %% Dimension-of-the-half-line
  