  %%%%%%%%%%%%%%%%%%%%%%%%%%%%%%%%%%%%%%%%%%%%%%%%%%%%%%%%%%
  %% 
  %% Locality and Diffeologies MARK: -
  %% 
  %%%%%%%%%%%%%%%%%%%%%%%%%%%%%%%%%%%%%%%%%%%%%%%%%%%%%%%%%%
  
  \chapter{Locality and Diffeologies}
  
  \label{Chapter-Locality-and-diffeologies}
  \newcommand{\ChapterLAD}{Locality and diffeologies}
  
  \begin{chaphead}
    Diffeology is built on purpose on a dry set, without structure.   
    This may appear unusual, when most of the traditional     
    constructions in differential geometry are built on top of 
    topological spaces.  
    However, that does not mean that topology is completely 
    absent from the theory, it appears actually as a byproduct 
    of local smoothness. This is the way that I chose to 
    introduce these notions, following the logic of diffeology. 
    Indeed, a smooth map $f$ from $\X$ to $\X'$ transforms, 
    by composition, the plots of $\X$ into plots of $\X'$. Thus, a local 
    smooth map $f$ from $\X$ to $\X'$ will be a map, 
    defined on a subset $\A \subset \X$, to $\X'$, 
    which transforms every plot $\P$ of $\X$ into a plot of $\X'$. 
    But that raises the question of the domain of definition. 
    Naturally $f \circ \P$ is defined on $\P^{-1}(\A)$, 
    which needs therefore to be a domain (maybe empty). 
    Thus, a local smooth map is defined on a 
    special kind of subsets $\A$, the ones for which $\P^{-1}(\A)$ is open 
    for all plots $\P$ of $\X$, and these are the open sets of 
    a special topology, the D-topology of X \art{The-D-Topology-of-diffeological-spaces}. 
    I could have introduced first the D-topology of
    the diffeological spaces and then the local smooth maps, but I think that it 
    breaks the logic, or self-consistency, of the theory. Moreover, diffeology
    applies on numerous spaces for which topology is helpless, 
    irrational tori for example. Emphasizing D-topology
    could have been misinterpreted.
    However, the reader who is 
    uncomfortable with my choice can begin by the section 
    on D-topology 
    and come back afterwards to local smoothness.
    That said, locality in diffeology will allow us to
    refine some diffeological concepts like
    subduction or induction. 
    It will suggest the notions of embeddings and embedded
    subsets. It will give us the notion of local
    diffeomorphisms which is associated with the definition
    of manifolds, or more generally, with the concept of modeling
    diffeological spaces.
  \end{chaphead}
  
  %%%%%%%%%%%%%%%%%%%%%%%%%%%%%%%%%%%%%%%%%%%%%%%%%%%%%%%%%%
  
  \section*{To be Locally Smooth}
  \label{To-be-locally-smooth}
  
  \begin{sechead}
    If we can define local smoothness for diffeological spaces, 
    we must  keep in mind that for many singular spaces, like irrational tori, 
    this notion is helpless. For this kind of spaces, 
    there are no local smooth maps but only global. Nevertheless, local smoothness 
    will be essential to
    include properly the category of manifolds in the category $\Diffeology$,
    or the category of orbifolds, or some infinite dimensional spaces, 
    and others.
  \end{sechead}
  
  \begin{article}\artlabel{Local smooth maps} 
    \addcontentsline{toc}{section}{\small\hspace{10pt} Local smooth maps} 
    \label{Local-smooth-maps}
    Let $\X$ and $\X'$ be two diffeological spaces. Let
    $\A \subset \X$ be some subset. We shall say that a map $f$ from 
    $\A \subset \X$ to $\X'$ is a {\em local smooth map} 
    if, for every plot $\P$ of
    $\X$, the parametrization 
    $f \circ \P$ defined on $\P^{-1}(\A)$ is a plot of $\X'$.
    That implies in particular that $\P^{-1}(\A)$ is a domain.  
    We shall say that a
    map $f: \X \to \X'$ is {\em locally smooth} at the
    point $x \in \X$ if there exists a
    superset $\A$ of $x$ such that the map $f$, restricted 
    to $\A$, is a local smooth map. 
    We shall admit that the empty map is a local smooth map.
    
    \Note{1} Since $\P^{-1}(\A)$ must be open for each plot $\P$ of $\X$, 
    the domain of definition $\A \subset \X$ of the local smooth map $f$ 
    is not any subset, but an open subset for the D-topology of $\X$, see
    \art{The-D-Topology-of-diffeological-spaces}.
    
    \Note{2} To clearly express that the property of 
    {\em local smoothness} of $f$
    involves the whole set of plots of $\X$, 
    and not just the ones with values in $\A$,
    we shall often use this notation  $f : \X \supset \A \to \X'$.
        
    \Note{3} When considering local properties we may use the subscript ${\star}_\loc$,
    for instance to denote space of local smooth maps, or whatever else.
  \end{article} %% Local-smooth-maps
  
  \begin{article}\artlabel{Composition of local smooth maps} 
    \addcontentsline{toc}{section}{\small\hspace{10pt} Composition of local
    smooth maps} 
    \label{Composition-of-local-smooth-maps}
    The composite of two local smooth maps is
    again a local smooth map. More precisely, let $\X$, $\Y$
    and $\Z$ be three diffeological spaces. Let $f : \X
    \supset \A \to \Y$ and $g : \Y \supset \B \to \Z$ be two
    local smooth maps. The map $ g \circ f : \X
    \supset f^{-1}(\B) \to \Z$ is a local smooth map.
  \end{article} %% Composition-of-local-smooth-maps
  
  \begin{proof}
    Let $\P : \U \to \X$ be a plot of $\X$, since $f$ is
    locally smooth: $\P^{-1}(\A)$ is a domain, and 
    $\Q = f \circ \P : \P^{-1}(\A) \to \X$ is a plot of $\Y$.
    Now, since $g$ is a local smooth map and $\Q$
    is a plot of $\Y$, $\Q^{-1}(\B) = (f \circ \P)^{-1}(\B)
    = \P^{-1}(f^{-1}(\B))$ is a domain and $g \circ \Q = g
    \circ (f \circ \P) = (g \circ f) \circ \P :
    \P^{-1}(f^{-1}(\B)) \to \Z$ is a plot of $\Z$. Hence,
    for every plot $\P$ of $\X$, the map $(g \circ f) \circ
    \P$ is a plot of $\Z$. Therefore $g \circ f$ is a local
    smooth map. \end{proof}
  
  \begin{article}\artlabel{To be smooth or locally smooth}
    \addcontentsline{toc}{section}{\small\hspace{10pt} To be smooth or locally smooth} 
    \label{To-be-smooth-or-locally-smooth} 
    Let $\X$ and $\X'$ be two diffeological spaces. A map 
    $f: \X \to \X'$ is smooth if and only if $f$ is locally
    smooth at every point $x$ of $\X$. 
  \end{article} %% To-be-smooth-or-locally-smooth
  
  \begin{proof}
    Obviously, if $f$ is smooth it is locally
    smooth at every point $x$ of $\X$, just choose $\X$ as
    a superset of $x$ \art{Local-smooth-maps}. 
    Conversely, let us assume that $f$ is
    locally smooth at every point $x$ of $\X$. Let 
    $\P : \U \rightarrow \X$ be a plot of $\X$. For all
    $r \in \U$ there exists, by assumption, a superset
    $\A$ of $x = \P(r)$ such that 
    $f \restriction \A : \X \supset \A \to \X'$ is a local smooth map
    \art{Local-smooth-maps}. This implies in
    particular that $(f \restriction \A) \circ \P$, defined
    on $\V = \P^{-1}(\A)$, is a plot of $\X'$. Thus, $\V \subset \U$ 
    is a domain and $(f \circ \P) \restriction \V$ 
    is a plot of $\X'$. Hence, $f \circ {\P}$ is locally
    a plot of $\X'$ at every point $r \in \U$, that is, by
    axiom {\D}2 of diffeology
    \art{Diffeologies-and-diffeological-spaces}, $f \circ \P$ 
    is a plot of $\X'$. Therefore, for any plot $\P$
    of $\X$, $f \circ \P$ is a plot of $\X'$, and $f$ is
    smooth. 
  \end{proof}
  
  \begin{article}\artlabel{Germs of local smooth maps} 
    \addcontentsline{toc}{section}{\small\hspace{10pt} Germs of local
    smooth maps}
    \label{Germs-of-local-smooth-maps} Let $\X$ and
    $\Y$ be two diffeological spaces, and let $x$ be a point
    of $\X$. We say that two local smooth maps $f:
    \X \supset \A \to \Y$ and $f' : \X \supset \A' \to \Y$
    {\em have the same germ at the point $x$} if and only if
    there exists a superset $\A'' \subset \A \cap \A'$ of
    $x$ such that: $f \restriction \A'' : \X \supset \A'' \to \Y$ and 
    $f' \restriction \A'' : \X \supset \A'' \to \Y$ 
    are still two local smooth maps,
    and $f \restriction \A'' = f' \restriction \A''$.
    
    \Note{1} Having the same germ at the point $x$ is
    an equivalence relation whose classes are called {\em
    germs}. The germ of the local smooth map $f$ at
    the point $x$ will be denoted by $\germ(f)(x)$.
    If $\phi$ is the germ of $f$ at the point $x$, the
    value $y = f(x)$ is well defined and is called the value
    of the germ of $f$ at $x$, and we denote $\phi(x)
    = y$. 
    
    \Note{2} Let $\Z$ be a third diffeological space,
    let  $f : \X \supset \A \to \Y$ and $g : \Y \supset \B
    \to \Z$ be two local smooth maps. Let $x \in \A$ and $y =
    f(x) \in \B$. The germ of $g \circ f : \X \supset
    f^{-1}(\B) \to \Z$ at $x$ depends only on the germ of $f$
    at $x$ and on the germ of $g$ at $y$. We shall denote
    $\germ(g)(y) \cdot \germ(f)(x) = \germ(g \circ f)(x)$.
    This defines the {\em composition of germs}. Like
    composition of functions, composition of germs is
    associative.
  \end{article} %% Germs-of-local-smooth-maps
  
  \begin{proof}
  For the first point, having the same germ at some point is
    obviously an equivalence relation. Now, for every local
    smooth map $f' : \X \supset \A' \to \Y$, having
    the same germ as $f$ at $x$ implies that the point $x$
    belongs to $\A'$ and $f'(x) = f(x)$. Hence $y = f(x)$
    depends only on the germ of $f$. 
    For the second point,  let $f' : \X \supset \A' \to \Y$ be a local
    smooth map such that $\germ(f')(x) =
    \germ(f)(x)$. Then, let  $g' : \Y \supset \B' \to \Z$ be a
    local smooth map such that $\germ(g')(y) =
    \germ(g)(y)$. Thus, there exists a 
    superset $\A'' \subset
    \A \cap \A'$ of $x$ such that $f \restriction \A''$ and
    $f' \restriction \A''$ are two equal local
    smooth maps.  There exists, as well, a superset
    $\B'' \subset \B \cap \B'$ of $y$ such that $g
    \restriction \B''$ and $g' \restriction \B''$ are two
    equal local smooth maps. Hence, $g \circ f
    \restriction f^{-1}(\B'')  = g' \circ f' \restriction
    f^{-1}(\B'')$, $x \in f^{-1}(\B'')$, $g \circ f$ and $g'
    \circ f'$ are local smooth maps
    \art{Composition-of-local-smooth-maps}. Hence,
    $\germ(g' \circ f')(x) = \germ(g \circ f)(x)$ and
    $\germ(g \circ f)(x)$ depends only on the germ of $f$ at
    $x$ and the germ of $g$ at $y$. Associativity of 
    composition of germs is a direct consequence of the
    associativity of the composition of maps. 
  \end{proof}
  
  \begin{article}\artlabel{Local diffeomorphisms and \'etale maps} 
    \addcontentsline{toc}{section}{\small\hspace{10pt} Local diffeomorphisms} 
    \label{Local-diffeomorphisms}
    Let $\X$ and $\X'$ be two diffeological spaces. We say that a map $f:
    \X \supset \A
    \to \X'$ is a {\em
    local diffeomorphism} if and only if $f$ is injective and $f$ is a
    local smooth map as well as its inverse $f^{-1} : \X'
    \supset f(\A)
    \to
    \X$.  The set of local
    diffeomorphisms from
    $\X$ to
    $\X'$ will be denoted by $\Diffloc(\X,\X')$ and $\Diffloc(\X)$ when 
    $\X = \X'$.
    
    Let $x$ be a point of $\X$, we say that a map $f: \X \to
    \X'$ is a {\em local diffeomorphism at $x$}, or is
    {\em \'etale at $x$}, if there exists a superset $\A$ of $x$
    such that $f\restriction \A : \X \supset \A \to \X'$ 
    is a local diffeomorphism.
    The map $f$ is said to be {\em \'etale} if it is \'etale
    at each point. 
  \end{article} %% Local-diffeomorphisms
  
  \begin{article}\artlabel{Etale maps and diffeomorphisms} 
    \addcontentsline{toc}{section}{\small\hspace{10pt} Etale maps and diffeomorphisms} 
    \label{Etale-maps-and-diffeomorphisms}
    An \'etale map $f: \X \to \X'$
    \art{Local-diffeomorphisms}, where $\X$ and $\X'$ are two diffeological
    spaces, is not necessarily a diffeomorphism, because $f$ is not
    necessarily bijective. But, if $f$ is bijective and \'etale, then $f$
    is a diffeomorphism. This is a direct consequence of
    \art{To-be-smooth-or-locally-smooth}. 
  \end{article} %% Etale-maps-and-diffeomorphisms
  
  \begin{article}\artlabel{Germs of local diffeomorphisms as groupoid} 
    \addcontentsline{toc}{section}{\small\hspace{10pt} Germs of local diffeomorphisms as
    groupoid} 
    \label{Germs-of-local-diffeomorphisms-as-groupoid}
    The set $\Diffloc(\X)$ of local diffeomorphisms of $\X$ is
    no longer a group, since the domains and the sets of values of local
    diffeomorphisms do not coincide necessarily. But let
    $\phi$ be the germ of a local diffeomorphism $f$ at a
    point $x \in \X$, let $\gamma$ be the germ of a local
    diffeomorphism $g$ at a point $y = f(x)$. Then, the germ
    $\gamma \cdot \phi$ of  $g \circ f$ at the point $x$ is
    well defined \art{Germs-of-local-smooth-maps} and maps
    the point $x$ to the point $z = g(y)$. This construction
    gives to the set of germs of local diffeomorphisms of
    $\X$ the structure of a groupoid \cite{McL71}, see also
    \art{Diffeological-groupoids}. The objects of this
    groupoid are the points of $\X$ and the morphisms from
    $x$ to $x'$ are the germs, at the point $x$, of the local
    diffeomorphisms of $\X$ mapping $x$ to $x'$. This groupoid may 
    be useful to explore the local structure of a diffeological space
    and its singularities, in relation to
    \art{Klein-structure-and-singularities-of-a-diffeological-space}.
  \end{article} %% Germs-of-local-diffeomorphisms-as-groupoid
  

  %%%%%%%%%%%%%%%%%%%%%%%%%%%%%%%%%%%%%%%%%%%%%%%%%%%%%%%%%%
  %
  %   Exercises  
  %
  %%%%%%%%%%%%%%%%%%%%%%%%%%%%%%%%%%%%%%%%%%%%%%%%%%%%%%%%%%
  
  \Exercise
  
  \begin{exercise}[To be a locally constant map]
    \label{To-be-a-locally-constant-map}
    Let $\X$ and $\X'$ be two diffeological spaces. Let $f:
    \X \to \X'$ be a {\em locally constant map}, that is, for each
    point $x_0 \in \X$ there exists a superset $\V$ of
    $x_0$ such that $f\restriction \V$ is a local smooth map
    and constant: $f \restriction \V : x \to f(x_0)$. Show
    that $f$ is constant on the {\em connected components} of
    $\X$ \art{Pathwise-connectedness}, that is, if $x_0$ and
    $x_1$ are two points of $\X$ such that there exists a
    $1$-plot $\gamma: \RR \to \X$ connecting $x_0$ to
    $x_1$, \ie, $\gamma(0) = x_0$ and $\gamma(1) = x_1$, then
    $f(x_0) = f(x_1)$. 
  \end{exercise}
  
  %%%%%%%%%%%%%%%%%%%%%%%%%%%%%%%%%%%%%%%%%%%%%%%%%%%%%%%%%%
  
  \section*{D-topology and Local Smoothness}
  \label{Section-D-topology-and-local-smoothness}
  
  \begin{sechead}
    There are several topologies on a diffeological space,
    compatible with the diffeology, that is, such that smooth
    maps are continuous, for example, the coarse topology. But one of
    them plays a particular role, it is the finest
    topology such that plots are continuous. This topology is
    called the {\em D-topology} of the space. In particular, local smooth maps 
    are necessarily defined on {\em D-open} subsets, that is, open subsets for the D-topology.
    The D-topology defines a faithful functor 
    from the category
    $\Diffeology$ of diffeological spaces to the category $\Topology$ of
    topological spaces. 
  \end{sechead}
  
  \begin{article}\artlabel{The D-Topology of diffeological spaces}
    \addcontentsline{toc}{section}{\small\hspace{10pt} The D-Topology of
    diffeological spaces} 
    \label{The-D-Topology-of-diffeological-spaces}
    There exists, on every diffeological space $\X$, a finest topology such
    that the plots are continuous. This topology is called the {\em D-topology}
    of $\X$ \cite{Igl85}. 
    The open sets for the D-topology are called {\em D-open sets}, they
    are characterized by the following property.
    \begin{itemize}
    \item[($\clubsuit$)]
    A subset $\A \subset \X$ is open for the D-topology 
    if and only if, for every plot $\P$ of $\X$,
    $\P^{-1}(\A)$ is open.
    \end{itemize}
  \end{article} %% The-D-Topology-of-diffeological-spaces
  
  \begin{proof}
    Let us consider the set $\cT$ of all the subsets $\A$ of
    $\X$ such that for any plot $\P$ of $\X$,
    $\P^{-1}(\A)$ is open. This set is a topology of
    $\X$, indeed: 

      \alinea{1.} The empty set belongs to $\cT$, since
      $\P^{-1}(\varnothing) = \varnothing$ is open.
      
      \alinea{2.} The set $\X$ belongs to
      $\cT$, since $\P^{-1}(\X) = \Dom(\P)$ is open by
      assumption. 
      
      \alinea{3.} Let $\{\A_i\}_{i \in \cI}$ be any finite
      family, with $\A_i \in \cT$, and let $\A = \cap_i \A_i$. We have
      $\P^{-1}(\A) = \P^{-1}(\cap_i \A_i) = \cap_i
        \P^{-1}(\A_i)$. But every $\P^{-1}(\A_i)$ is open. 
      Since the family is finite, the intersection $\cap_i
        \P^{-1}(\A_i)$ is open. Hence, $\P^{-1}(\A)$ is open.
      
      \alinea{4.} Let $\{\A_i\}_{i \in \cI}$ be any
      family, with $\A_i \in \cT$, and let
      $\A = \cup_i \A_i$. We have $\P^{-1}(\A) =
        \P^{-1}(\cup_i \A_i) = \cup_i \P^{-1}(\A_i)$. Since
      every $\P^{-1}(\A_i)$ is open the union $\cup_i
        \P^{-1}(\A_i)$ is open. Hence, $\P^{-1}(\A)$
      is open. 

    \alinea{}Now, let $\cT'$ be another topology for which the plots are continuous. 
    Let us recall that $\cT$ is finer than $\cT'$ if $\cT$ contains $\cT'$,
    that is, $\cT' \subset \cT$.
    Let $\A \subset \X$ be open for the
    topology $\cT'$, that is, $\A \in \cT'$. 
    Because the plots are $\cT'$-continuous, $\P^{-1}(\A)$ is open for all plots
    $\P$ of $\X$, thus $\A$ is
    D-open, that is, $\A \in \cT$, and thus $\cT' \subset \cT$. 
    Therefore, the D-topology is
    finer than $\cT'$. Thus, the D-topology is the finest
    topology such that the plots are continuous. 
  \end{proof}
  
  \begin{article}\artlabel{Smooth maps are D-continuous}
    \addcontentsline{toc}{section}{\small\hspace{10pt} Smooth maps are D-continuous} 
    \label{Smooth-maps-are-D-continuous} Let $\X$
    and $\X'$ be two diffeological spaces.  
      \begin{itemize}
      \item[1.]  Every smooth
      map $f$ from $\X$ to $\X'$ is {\em D-continuous}, that is, continuous
      for the D-topology.  
      \item[2.] Every diffeomorphism $f$ from $\X$ to $\X'$ is a
      {\em D-homeomorphism}, that is, a homeomorphism for the
      D-topology.
      \end{itemize}
    Associating the underlying D-topological space with a diffeology
    defines a faithful functor from
    the category $\Diffeology$ to the category
    $\Topology$. 
  \end{article} %% Smooth-maps-are-D-continuous
  
  \begin{proof}
    Let us denote by $\cD$ and $\cD'$ the diffeologies
    of $\X$ and $\X'$. By the very definition of smooth maps
    \art{Smooth-maps}, $f$ sends every plot
    $\P \in \cD$ into a plot
    $f \circ \P \in \cD'$. Let $\A' \subset \X'$ be
    D-open, and $\A = f^{-1}(\A')$. For every plot
    $\P \in \cD$, $\P^{-1}(\A) =
    \P^{-1}(f^{-1}(\A')) = (f \circ \P)^{-1}(\A')$.
    But $f \circ \P$ is a plot of $\cD'$ and $\A'$
    is D-open, thus $(f \circ \P)^{-1}(\A')$ is open.
    Hence, $\P^{-1}(\A)$ is open for every plot $\P \in
    \cD$. Therefore $\A = f^{-1}(\A')$ is D-open and $f$
    is continuous. 
    Next, a diffeomorphism is a bijective map, smooth as well as its inverse.
    Thus, a diffeomorphism is a bijective bi-continuous map, that is, 
    a homeomorphism. 
  \end{proof}
  
  \begin{article}\artlabel{Local smooth maps are defined on D-opens}  
  \addcontentsline{toc}{section}{\small\hspace{10pt} Local smooth maps are defined on D-opens} 
    \label{Local-smooth-maps-are-defined-on-D-opens}
    Let $f$ be a map defined on a subset $\A$ of a diffeological space $\X$ 
    with values in another one $\X'$.
    \begin{itemize}
      \item[($\clubsuit$)]  The map $f : \X \supset \A
      \to \X'$ is a local smooth map
      \art{Local-smooth-maps} if and only if $\A$ is D-open and
      $f$ is smooth as a map from $\A$ to $\X'$, where $\A$ is
      equipped with the subset diffeology. 
    \end{itemize}
    In particular, if $f : \X \supset \A \to \X'$ is a local
    diffeomorphism, then both $\A$ and $f(\A)$ are D-open,
    $f$ is a local homeomorphism for the D-topology, and
    $f \restriction \A$ is a diffeomorphism onto $f(\A)$. 
  \end{article} %% Local-smooth-maps-are-defined-on-D-opens
  
  \begin{proof}
    Let us assume first that $f$ is a local smooth map. By
    definition, for every plot $\P : \U \to \X$, the parametrization $f
    \circ \P : \P^{-1}(\A) \to \X'$ is a plot of $\X'$. Hence,
    since plots are defined on domains,
    $\P^{-1}(\A)$ is a domain.
    Thus, since for any plot $\P$ of $\X$ the set 
    $\P^{-1}(\A)$ is a domain, $\A$ is D-open. Now let us
    consider $\A \subset \X$ equipped with the subset
    diffeology \art{Subspaces-and-subset-diffeology} and 
    $f : \A \to \X$. Let $\P : \U \to \A$ be a plot, since the injection 
    $j_\A : \A \to  \X$ is smooth,
    $\P$ also is a plot of $\X$. Now, since $f$ is a local smooth map, 
    $f \circ \P$ is a plot of $\X'$. 
    Therefore, $f : \A \to \X'$, where $\A$ is equipped
    with the subset diffeology, is smooth. 
    
    Conversely, let us assume that $\A$ is D-open and that
    $f : \A \to \X'$, where $\A$ is equipped with
    the subset diffeology, is smooth. Let $\P : \U \to \X$ be a plot.
    since $\A$ is D-open, $\V = \P^{-1}(\A)$ is a domain.
    Since the inclusion map $j_\V : \V \to \U$ is a smooth
    parametrization, $\P \restriction \V = \P \circ j_\V$ is
    a plot of $\X$ with values  in $\A$, thus $\P
    \restriction \V$ is a plot of $\A$ for the subset
    diffeology. Now, since $f : \A \to \X$ is smooth, $f \circ
    (\P \restriction \V)$ is a plot of $\X'$. Then,
    since $f \circ \P = f \circ (\P \restriction \V)$, the
    parametrization $f \circ \P$ is a plot of $\X'$.
    Therefore, $f$ is a local smooth map from $\A \subset \X$
    to $\X'$.
    
    Now, if $f$ is a local diffeomorphism, $f^{-1}$, defined
    on $f(\A)$, is a local smooth map, thus both $\A$
    and $f(\A)$ are D-open. The restrictions $f \restriction \A : \A \to f(\A)$
    and $f^{-1} \restriction f(\A) : f(A) \to \A$, 
    where $\A$ and $f(\A)$ are equipped with 
    the subset diffeology, are smooth, thus $f \restriction \A$ a diffeomorphism onto $f(\A)$. 
    Then, $f \restriction \A$ is a homeomorphism for the D-topology
    \art{Smooth-maps-are-D-continuous}, that is, $f$
    is a local homeomorphism.
  \end{proof}
  
  \begin{article}\artlabel{D-topology on discrete and coarse spaces}
    \addcontentsline{toc}{section}{\small\hspace{10pt} {\D}-topology on discrete and coarse
    diffeological spaces} 
    \label{D-topology-on-discrete-and-coarse-diffeological-spaces}
    The D-topology of discrete diffeological spaces is 
    discrete. The D-topology of coarse diffeological spaces 
    is coarse. But note that a non coarse diffeological space can
    inherit the coarse D-topology 
    (see \exref{D-topology-of-irrational-tori}). 
  \end{article} %% D-topology-on-discrete-and-coarse-diffeological-spaces
  
  \begin{proof}
    Let us consider a set $\X$ equipped with the
    discrete diffeology. Let $x \in \X$, let ${\P}: \U \to
    \X$ be a plot of the discrete diffeology, if $x\notin
    {\P}(\U)$ then ${\P}^{-1}(x) = \varnothing$ is open.
    If $x = {\P}(r)$, $r \in \U$, there exists an open neighborhood
    $\V$ of $r$ such that ${\P}\restriction \V$ is
    constant and equal to $x$. Then, since $\P^{-1}(x)$ is a union 
    of open subsets of $\U$, it is open. Thus,
    every point of $\X$ is D-open, and the D-topology of the
    discrete diffeology is discrete. On the other hand, let
    $\X$ be equipped with the coarse diffeology. Let us assume
    that $\X$ is not reduced to a point. Let $\Omega\subset
    \X$ such that $\Omega\neq\varnothing$ and $\Omega\neq
    \X$. 
    Then, there exist a point $x \in \X-\Omega$ and
    a point $y \in \Omega$. Let us define ${\P}: \RR \to
    \X$ by ${\P}(t) = x$ if $t \in \RR-\{0\}$ and ${\P}(0)
    = y$. 
    Since every parametrization in $\X$ is a plot,
    ${\P}$ is a plot. But, ${\P}^{-1}(\Omega) = \{0\}$ is
    closed in $\RR$, then $\Omega$ is not D-open. Therefore,
    the only D-open subsets of $\X$ are the empty set and
    $\X$, the D-topology associated with the coarse
    diffeology is coarse. \end{proof}
  
  \begin{article}\artlabel{Quotients and D-topology}
    \addcontentsline{toc}{section}{\small\hspace{10pt} Quotients and D-topology}
    \label{Quotients-and-D-topology}
    Let $\X$ and $\X'$ be two diffeological
    spaces. Let $\pi : \X \to \X'$ be a subduction
    \art{What-is-a-subduction}. Then, the D-topology of $\X'$ is the
    quotient of the D-topology of $\X$ by $\pi$. In other
    words, a subset $\A \subset \X'$ is D-open if and only if
    $\pi^{-1}(\A)$ is D-open. This applies in particular to the quotient of
    $\X$ by any equivalence relation
    \art{Quotient-and-quotient-diffeology}.
  \end{article} %% Quotients-and-D-topology
  
  \begin{proof}
    Let $\A'$ be a subset of $\X'$. Let us assume that
    $\pi^{-1}(\A')$ is D-open. Let
    $\P': \U \to \X'$ be a plot. For all $r \in \U$,
    there exist an open neighborhood $\V$ of $r$, and a plot 
    $\P : \V \to \X$, such that
    $\pi \circ \P = \P' \restriction \V$. Hence, there
    exists an open covering
    $\{\U_i\}_{i \in \cI}$ of $\U$, indexed by
    some set
    $\cI$ ($\cI$ can be $\U$ itself), and for each 
    $i \in \cI$ there exists a plot $\P_i: \U_i \to \X$, such that
    $\P' = \sup\{\pi \circ \P_i\}_{i \in \cI}$ (where $\sup$
    denotes the smallest common extension). Then,
    ${\P'}^{-1}(\A') = (\sup\{\pi \circ \P_i\}_{i \in \cI})^{-1}(\A') = 
    \cup_{i \in \cI}(\pi \circ \P_i)^{-1}(\A') = 
    \cup_{i \in \cI}\P_i^{-1}(\pi^{-1}(\A'))$. But $\pi^{-1}(\A')$
    is assumed to be D-open, and the $\P_i$ are plots of $\X$. Thus, for all $i \in \cI$, 
    $\P_i^{-1}(\pi^{-1}(\A'))$ are domains, and their union is a domain.
    Hence, ${\P'}^{-1}(\A')$ is a domain for all plots $\P'$ of $\X'$. 
    Therefore, $\A'$ is D-open. 
    Conversely, let us assume that $\A'$ is D-open. Let $\P:
    \U \to \X$ be a plot. Then, $\P^{-1}(\pi^{-1}(\A'))
    = (\pi \circ \P)^{-1}(\A')$. Since $\pi$ is smooth,
    $\pi \circ \P$ is a plot of
    $\X'$,
    and $(\pi \circ \P)^{-1}(\A')$ is a domain. Therefore
    $\pi^{-1}(\A')$ is D-open.
  \end{proof}
  

  %%%%%%%%%%%%%%%%%%%%%%%%%%%%%%%%%%%%%%%%%%%%%%%%%%%%%%%%%%
  %
  %   Exercises  
  %
  %%%%%%%%%%%%%%%%%%%%%%%%%%%%%%%%%%%%%%%%%%%%%%%%%%%%%%%%%%
  
  \Exercises
  
  \begin{exercise}[Diffeomorphisms of the square]
    \label{Diffeomorphisms-of-the-square} 
    Let $\Sq$ be the square $[0,1] \times
    \{0,1\} \cup \{0,1\} \times [0,1]$, equipped with the
    subset diffeology  \art{Subspaces-and-subset-diffeology}
    of the smooth diffeology of $\RR^2$. Justify that the
    restriction of the diffeomorphisms of $\Sq$ to the four corners
    defines a homomorphism $h$ from $\Diff(\Sq)$ to the
    group of permutations $\Perms(4)$. Describe
    the image of $h$. 
  \end{exercise} %% Diffeomorphisms-of-the-square
  
  \begin{exercise}[Smooth D-topology] 
  \label{Smooth-D-topology}
    Show that the D-topology of a real domain,
    equipped with the smooth diffeology
    \art{Real-domains-as-diffeological-spaces}, coincides with its
    standard topology.  
  \end{exercise} %% Smooth-D-topology
  
  \begin{exercise}[D-topology of irrational tori]
    \label{D-topology-of-irrational-tori}
    Let $\T_\Gamma$ be the quotient of $\RR$ by a strict dense subgroup
    (for example $\QQ$ or $\ZZ + \alpha \ZZ$ where $\alpha \notin \QQ$).
    Show that the D-topology of $\T_\Gamma$ is coarse. Deduce that the
    D-topology is not a full functor.
  \end{exercise} %% D-topology-of-irrational-tori
  
  %%%%%%%%%%%%%%%%%%%%%%%%%%%%%%%%%%%%%%%%%%%%%%%%%%%%%%%%%%
  
  \section*{Embeddings and Embedded Subsets}
  \label{Section-Embeddings-and-embedded-parts}
  
  \begin{sechead}
    Combining smoothness and D-topology leads to a refinement of
    the notion of induction and introduces the notion of diffeological
    embedding. Moreover, since every subset of a
    diffeological space has a natural subset diffeology,
    the comparison between the D-topology induced by the
    ambient space and the D-topology of the induced
    diffeology distinguishes between embedded subsets of a
    diffeological space and the others, without referring to
    anything else than the ambient diffeology.
  \end{sechead}
  
  \begin{article}\artlabel{Embeddings}
    \addcontentsline{toc}{section}{\small\hspace{10pt} Embeddings}
    \label{Embeddings}
    Let $\X$ and $\X'$ be two diffeological spaces. Let 
    $f: \X \to \X'$ be an induction \art{What-is-an-induction}. We
    shall say that $f$ is an {\em embedding} if
    the pullback by $f$ of the D-topology of $\X'$ coincides
    with the D-topology of $\X$, that is, an induction $f$ is an
    embedding if and only if, for every D-open subset $\A \subset \X$, 
    there exists a D-open subset 
    $\A' \subset \X'$ such that $f^{-1}(\A') = \A$. 
  \end{article} %% Embeddings
  
  \begin{article}\artlabel{Embedded subsets of a diffeological space}
    \addcontentsline{toc}{section}{\small\hspace{10pt} Embedded subsets of a diffeological space} 
    \label{Embedded-subsets-of-a-diffeological-space}
    Let $\X$ be a diffeological space and $\A \subset \X$
    be some subset. We shall say that $\A$ is {\em embedded}
    if the canonical induction $j_\A: \A \to \X$, 
    where $\A$ is equipped with the
    subset diffeology, is an embedding. 
    Every subspace $\A$ of $\X$ carries two natural
    topologies: the D-topology given by the induced diffeology of $\X$,
    and the induced topology of the ambient D-topology of
    $\X$. The set $\A$ is {\em embedded} if these
    two topologies coincide. In other words, $\A \subset \X$
    is embedded if every $\U \subset \A$, open for the
    D-topology of the induced diffeology, is the imprint of some
    D-Open $\V$ of $\X$, that is, $\U = \A \cap \V$. 
    
    \Note~For a subset of a diffeological space, to be embedded
    depends only on the diffeology of the ambient space $\X$,
    and does not involve any extra structure. 
  \end{article} %% Embedded-subsets-of-a-diffeological-space
  
  %%%%%%%%%%%%%%%%%%%%%%%%%%%%%%%%%%%%%%%%%%%%%%%%%%%%%%%%%%
  %
  %   Exercises  
  %
  %%%%%%%%%%%%%%%%%%%%%%%%%%%%%%%%%%%%%%%%%%%%%%%%%%%%%%%%%%
  \Exercises
  
  \begin{exercise}[$\QQ$ is discrete but not embedded in
    $\RR$] \label{Q-is-discrete-but-not-embedded-in-R} 
    Show that $\QQ \subset \RR$, where $\RR$ is equipped
    with the smooth diffeology, is not embedded.
  \end{exercise} %% Q-is-discrete-but-not-embedded-in-R
  
  \begin{exercise}[Embedding $\GL(n,\RR)$ in $\Diff(\RR^n)$]
    \label{Embedding-GL-n-R-in-Diff-R-n} 
    Let us consider the linear group $\GL(n,\RR)$ as a group of matrices. The
    components of a matrix $\M$ will be denoted by $\M_{ij}$, $i,j = 1
    \cdots n$. The identification between an element of $\GL(n,\RR)$ and
    the ordered set of its matrix elements (for any given definite ordering)
    identifies $\GL(n,\RR)$ with a domain of $\RR^{n \times n}$. By this
    identification $\GL(n,\RR)$ inherits the diffeology of  $\RR^{n \times
    n}$. This diffeology will be called the {\em standard diffeology} of
    $\GL(n,\RR)$ and will equip $\GL(n,\RR)$. On the other hand,
    $\GL(n,\RR)$ can be regarded as a subgroup of $\Diff(\RR^n)$. We equip
    $\Diff(\RR^n)$ with the functional diffeology
    \art{Functional-diffeology-on-groups-of-diffeomorphisms}.
    
    \Question{1)} Show that the inclusion $\GL(n,\RR) \hookrightarrow \Diff(\RR^n)$ is
    an induction.
    
    \Question{2)} Show that this induction is actually an embedding.
    
    \alinea{Hint:} Consider the following subsets of $\Diff(\RR^n)$
    \begin{displaymath}
      \Omega_{\varepsilon} = \{f\in \Diff(\RR^{n}) \mid \D (f)(0)\in \Ball({\bf
      1}_{n}, \varepsilon) \},
    \end{displaymath}
    where $\D(f)(0)$ is the linear tangent map of $f$ at the point $0$, and
    $\Ball({\bf 1}_{n},
    \varepsilon)$ is the open ball in $\GL(n,\RR)$, 
    centered at the identity ${\bf 1}_{n}$, with radius $\varepsilon$.
  \end{exercise} %% Embedding-GL-n-R-in-Diff-R-n

  \begin{exercise}[The irrational solenoid is not embedded]
    \label{The-irrational-solenoid-is-not-embedded} 
    Show that the irrational solenoid defined in the
    \exref{The-irrational-solenoid} is not embedded.
  \end{exercise} %% The-irrational-solenoid-is-not-embedded
  
  %%%%%%%%%%%%%%%%%%%%%%%%%%%%%%%%%%%%%%%%%%%%%%%%%%%%%%%%%%
  
  \section*{Local or Weak Inductions}
  \label{Section-Local-or-weak-inductions}
  
  \begin{sechead}
    Some smooth injections look like inductions but are not.
    The example of the infinite symbol of the
    \exref{The-infinite-symbol} is a good illustration of
    this situation. This is why we need to weaken, or refine, the
    definition of inductions and also consider maps which are
    only locally inductive. This notion is close to the ordinary concept of
    immersion, see \exref{Immersions-of-real-domains}, 
    but it is still do not know if they coincide.
  \end{sechead}

  \begin{article}\artlabel{Local inductions}
    \addcontentsline{toc}{section}{\small\hspace{10pt} Local inductions} 
    \label{Local-inductions} 
    Let $\X$ and $\X'$ be two diffeological spaces. 
    Let $f: \X \rightarrow \X'$ be a smooth map. We shall say that $f$
    is a {\em local induction at the point $x \in \X$} if there
    exists an open neighborhood $\cO$ of $x$ such that $f \restriction \cO$ 
    is injective and for every plot 
    $\P : \U \to f(\cO)$ centered at $x' = f(x)$, $\P(0)=x'$, $f^{-1} \circ \P'$
    is a plot of $\X$. Put differently, $f$ is an induction at the level of the 
    germ of the diffeology of $\X$ at $x$
    \art{Pointed-plots-of-a-diffeological-space}. Note that an induction is a local
    induction everywhere, but a local induction everywhere is not necessarily
    an induction, see \exref{The-infinite-symbol}. 
    It is not enough for a local induction to be injective for being an induction.
    
    \Note~For a smooth injection $j$ from an $n$-domain $\U$ into
    an $m$-domain $\V$, if $j$ is an immersion at a point $x$, that is, if its
    tangent map $\D(j)(x)$ is injective, then $j$ is a local induction at $x$, see
    \exref{Immersions-of-real-domains}. 
    I am still not sure if the converse is true for $n=1$ and $m>1$. 
    Actually, it could be weird if
    local inductions between real domains were not immersions\footnote{When J.~M. 
    Souriau wrote one of his first papers on diffeology he named immersion what is
    called induction now, but after a remark from J. Pradines he changed his mind.
    He named also submersions what is now subductions, but for that question things are 
    different, see \art{Local-subductions}.}. For $n = m = 1$, $j$ is a local diffeomorphism,
    thus these notions coincide.
  \end{article} %% Local-inductions

  %%%%%%%%%%%%%%%%%%%%%%%%%%%%%%%%%%%%%%%%%%%%%%%%%%%%%%%%%%
  %
  %   Exercises  
  %
  %%%%%%%%%%%%%%%%%%%%%%%%%%%%%%%%%%%%%%%%%%%%%%%%%%%%%%%%%%
  
  \Exercise
  
  \begin{exercise}[The infinite symbol]
    \label{The-infinite-symbol} 
    Let $j$ be the smooth map described by the figure \ref{lehuit} and
    defined by
    $$ j : \ ]-\pi , \pi[\ \to \RR^2 \qquad j(t) = 
    \vect{
    \sin t
    \vspace{1ex}
    \\
    \sin 2t
    }.
    $$
    
    %%###########
    \begin{figure}[tb]
      \centerline{\includegraphics{Figures-PDF/fig-J.pdf}}
      \caption{The infinite symbol.}
      \label{lehuit}
    \end{figure}
    %%###########
    
    \Question{1)} Check that $j$ is injective and that
    $\lim_{t \mathop{\rightarrow} -\pi}j(t) = \lim_{t \mathop{\rightarrow} + \pi}j(t) = (0,0)$.
    
    \Question{2)} Let ${\P}$ be the following plot of $\RR^2$: 
    $$ 
    \P :\ ]-\pi , \pi[\ \to \RR^2 \qquad
    \P(t) = 
    \vect{
    \sin t
    \vspace{1ex}
    \\
    -\sin 2t
    }.
    $$ 
    Check that $\P$ and $j$ have the same image. Draw the image of $\P$,
    similar to the image of $j$ drawn in fig. \ref{lehuit}.
    
    \Question{3)} Show that $j^{-1}(\P(\,]-\pi/4,\pi/4[\,))$ is disconnected. Conclude that
    the injection $j$ is not an induction.
    
    \Question{4)} Check that $j$ is everywhere a local induction.
  \end{exercise} %% The-infinite-symbol
      
  %************************************************
  
  \section*{Local or Strong Subductions}
  \label{Section-Local-or-strong-subductions}
  
  \begin{sechead}
    Subductions \art{What-is-a-subduction} express a global behavior of some
    surjections. As we localized the notion of induction, and obtained then
    the notion of local induction \art{Local-inductions}, the
    notion of subduction can be localized, or refined, as well and leads to
    the notion of {\em local subduction}. The \exref{A-not-so-strong-subduction}
    illustrates the case where a subduction is not everywhere a local subduction.
  \end{sechead}
  
  \begin{article}\artlabel{Local subductions}
    \addcontentsline{toc}{section}{\small\hspace{10pt} Local subductions}
    \label{Local-subductions}
    Let $\X$ and $\X'$ be two diffeological spaces.
    We shall say that a smooth surjection $f: \X \to \X'$
    is a {\em local subduction at the point $x \in \X$} 
    if for every plot $\P : \U \to \X'$ pointed at $x' = f(x)$, $\P'(0) = x'$,
    there exist an open neighborhood $\V$ of
    $0$ and a plot $\Q : \V \to \X$ such that $\Q(0) = x$ and 
    $f \circ \Q = \P \restriction \V$. 
    Said differently, $f$ is a subduction \art{What-is-a-subduction} at the level of the germ of 
    the diffeology of $X$ at the point $x$
    \art{Pointed-plots-of-a-diffeological-space}. 
    We simply shall say local subductions for: everywhere local subductions, 
    they are {\em a fortiori} subductions.
        
    \Note~For a smooth surjection $\pi : \U \to \V$ between real domains, 
    if $\pi$ is a {\em submersion} at a point $x \in \U$, that is, if its
    tangent map $\D(\pi)(x)$ is surjective, then $\pi$ is a local subduction at $x$.
    Conversely, if $\pi$ is a local subduction at $x$, then it is a submersion at this point
    \footnote{It is why we could name local subductions, submersions. 
    But for the balance between local immersion and local subduction,
    we continue with this vocabulary. If we can prove that a local induction in real domains
    is always an immersion, then it will be time to adapt the vocabulary.}. For real domains these 
    notions are equivalent.
  \end{article} %% Local-subductions

  \begin{proof}
    Let $\dim(\U)=n$ and $\dim(\V)=m$.
    If $\pi : \U \to \V$ is a submersion at $x$, 
    thanks to the rank theorem for domains \cite[10.3.1]{Die70a},
    there exists a local diffeomorphism $\phi$ defined on a neighborhood $\cO$ of $x$
    to $\RR^m \times \RR^{n-m}$ such that the projection $\pi$ is equivalent to 
    the projection onto the first factor.
    Therefore, every plot centered at $x'$ can be simply locally lifted at $x$ by 
    choosing a constant in the second factor. Conversely, if $\pi$ is a local subduction
    at $x$, for every vector $v \in \RR^m$, the path 
    $t \mapsto x' + tv$ has a local lift $c : t \to x_t$, $\pi(x_t) = x' +tv$, centered at $x$. 
    Thus, $\D(\pi)(x)(\dot c(0)) = v$, with $\dot c(t) = d c(t)/dt$.
    Therefore, $\D(\pi)(x)$ is surjective.
  \end{proof}  
  
  \begin{article}\artlabel{Compositions of local subductions}
    \addcontentsline{toc}{section}{\small\hspace{10pt} Compositions of local
    subductions}
    \label{Compositions-of-local-subductions}
    Let $\X$, $\X'$ and $\X''$ be three diffeological spaces.
    Let $f: \X \to \X'$ and $f': \X' \to \X''$ be
    two local subductions. The composite $f' \circ
    f$ is still a local subduction. 
  \end{article} %% Compositions-of-local-subductions
  
  \begin{proof}
    First of all, the composition of two smooth surjections is
    still a smooth surjection. Let $\P : \U \to \X''$ be a plot centered at $x''$.
    Let $x \in \X$ such that $f' \circ f(x) = x''$. Let $x' = f(x)$,
    since $f'$ is a local
    subduction there exist an open neighborhood $\V'$
    of $0$, and a plot $\Q' : \V' \to \X'$, 
    such that $\Q'(0) = x'$ and 
    $\P \restriction \V' = f' \circ \Q'$. 
    Since $\Q'$ is a plot of $\X'$ and $f$ is a local subduction,
    there exist an open neighborhood
    $\V$ of $0$, and a plot $\Q : \V \to \X$, such that $\Q(0) = x$ and 
    $\Q' \restriction \V = f \circ \Q$. Hence, the plot $\Q$ satisfies 
    $\Q(0) = x$ and $(f' \circ f) \circ \Q = \P \restriction \V$.
  \end{proof}
    
  \begin{article}\artlabel{Local subductions are D-open maps}
    \addcontentsline{toc}{section}{\small\hspace{10pt} Local subductions are D-open maps}
    \label{Local-subductions-are-D-open-maps} 
    Let $\X$ and $\X'$ be two diffeological spaces. Let $f : \X
    \to \X'$ be a local subduction. If $\A \subset \X$  is a D-open
    set, then $f(\A)$ also is D-open. Local
    subductions are D-open maps.
  \end{article} %% Local-subductions-are-D-open-maps
  
  \begin{proof}
    First of all, let us remark that, composing with a translation, 
    we can replace $0$ by any $r$ such that $\P(r)=x$ in the definition of local subduction 
    \art{Local-subductions}. Now, let $\A' = f(\A)$ and $\P : \U \to \X'$ be a
    plot. We want to show that $\P^{-1}(\A')$ is a domain.
    For every $r \in \P^{-1}(\A')$ let us choose $x \in \A$
    such that $f(x) = \P(r)$. Now, since $f$ is a local
    subduction, there exist an open neighborhood $\V$ of $r$ and a plot
    $\Q : \V \to \X$ such that $\Q(r) = x$ and $f \circ \Q =
    \P \restriction \V$. But since $\A$ is open,
    $\Q^{-1}(\A)$ is a domain. Let us define $\W =
    \Q^{-1}(\A) \cap \V$, $\W$ is still a domain, still
    containing r. Then, let us define $\bar \Q = \Q \restriction \W$, 
    $\bar \Q$ is still a plot of $\X$. But $\bar \Q (\W) = \Q(\W) \subset \A$, 
    by construction, thus
    $\P(\W) = f \circ \bar \Q (\W) \subset f(\A) = \A'$, that
    is, $\W \subset \P^{-1}(\A')$. Thus, for every $r \in \U$ we
    found a domain $\W$ such that $\W$ is an open neighborhood of $r$
    and $\W \subset \P^{-1}(\A')$. Hence, since
    $\P^{-1}(\A')$ is the union of all the domains $\W$, when
    $r$ runs over $\P^{-1}(\A')$, $\P^{-1}(\A')$ is itself a
    domain. Hence, $\P^{-1}(\A')$ is a domain for every plot $\P$
    of $\X'$. Therefore $\A'$ is D-open.
  \end{proof}
  
  %%%%%%%%%%%%%%%%%%%%%%%%%%%%%%%%%%%%%%%%%%%%%%%%%%%%%%%%%%
  %
  %   Exercises  
  %
  %%%%%%%%%%%%%%%%%%%%%%%%%%%%%%%%%%%%%%%%%%%%%%%%%%%%%%%%%%
  
  \Exercises
  
  \begin{exercise}[Quotient by a group of diffeomorphisms]
    \label{Quotient-by-a-group-of-diffeomorphisms} 
    Let $\X$ be a diffeological space and
    $\G \subset \Diff(\X)$ be a group of diffeomorphisms,
    equipped with the functional diffeology
    \art{Functional-diffeology-on-groups-of-diffeomorphisms}. 
    Choose a point $x \in \X$ and denote by $\cO$ the orbit 
    $\G(x) = \{ g(x) \mid g \in \G \}$,
    equipped with the subset diffeology. Let
    $\cQ = \G/\pi$ denote the same orbit, but equipped with the
    pushforward of the functional diffeology of $\G$ by the projection 
    $\pi : g \mapsto g(x)$. Let $j : \cQ \to \cO$ be the identity map.
    
    \Question{1)} Check that $j$ is smooth, that is, the quotient diffeology of $\G$
    by $\pi$ is finer than the subset diffeology of $\G(x)$.
    
    \Question{2)} Show that $\pi$ is a local subduction from $\G$ to $\cQ$. 
        
    \Question{3)}     Let us denote by $\cT$ the space $\RR^2$, equipped with the sub-diffeology of the wire
    diffeology \art{The-wire-diffeology} generated by the vertical
    and horizontal lines\footnote{We denote this space by $\cT$ for
    {\em Tahar rug}, because it has been imagined by Guillaume 
    Tahar during a private conversation.}, that is, the parametrizations $t \mapsto (c,t)$ and 
    $t \mapsto (t,c)$, where $c$ runs over $\RR$. 
    Check that for all $u \in \RR^2$ the translation 
    $\T_u : \cT \to \cT$, $\T_u(\tau) = \tau + u$
    is a diffeomorphism. Show that the subgroup $\G \subset \Diff(\cT)$ of 
    these translations, equipped with the functional diffeology, is discrete.
    For $x =(0,0)$, note that $\G(x) = \cT$, and that the quotient 
    diffeology of $\G$ by $\pi$ from the first question (which coincides with $\G$), 
    is strictly finer than the diffeology of the orbit $\cT$ of $x$. 
    In other words, the translations, equipped with the functional diffeology, are transitive on $\cT$ but do not generate
    its diffeology. 
  \end{exercise} %% Quotient-by-a-group-of-diffeomorphisms
  
  \begin{exercise}[A not so strong subduction]
    \label{A-not-so-strong-subduction} \
    Let  $\X$ be the  diffeological sum
    $\RR \coprod
    \RR^2$ \art{Building-sums-with-spaces},
    where the real spaces
    $\RR$ and $\RR^2$ are equipped with the smooth diffeology. Let
    $\norm{\cdot} : \X \to [0,\infty[$ be the modulus 
    $\norm{x} =  \vert x \vert$ if $\ x \in \RR$, and the 
    usual norm when $x \in \RR^2$.
    Let $\cQ$ be the quotient $ \X/\norm{\cdot}$
    \art{Quotient-and-quotient-diffeology}, that is, the quotient of $\X$ by
    the equivalence relation $x \sim x'$ if $\norm{x} =
    \norm{x'}$.
    Consider the quotient $\cQ$ as the half-line $[0, \infty[$,
    equipped with the pushforward of the diffeology of $\X$ by the map
    $\norm{\cdot}^2$. Show that the plot $\P : \RR^2 \to
    \cQ$, defined by $\P(r) = \norm{r}^2$, cannot be lifted locally at
    $(0,0) \in \RR^2$ by a plot $\Q$ of $\X$, such that 
    $\Q(0,0) = 0 \in \RR \subset \X$. 
    Conclude that $\norm{\cdot}$, which is by construction a subduction, 
    is not a local subduction at the point $0 \in \RR \subset \X$.
  \end{exercise} %% A-not-so-strong-subduction
  
  \begin{exercise}[A powerset diffeology]
    \label{A-powerset-diffeology}
    Let $\X$ be a diffeological space, and $\cD$ be its
    diffeology. Let $\Powerset(\X)$ be the powerset of $\X$, that is, 
    $\Powerset(\X) = \{ \A \mid \A \subset \X \}$. We have seen in the 
    \exref{A-minimal-powerset-diffeology} a diffeology on this set, which induces
    on every quotient of $\X$ the quotient diffeology. 
    But this diffeology is too weak to be a good diffeology. 
    For example, consider the set of the affine lines in $\RR^2$,
    and choose the parametrization which associates with each rational number the 
    $x$-axis, and with each irrational number the $y$-axis. This parametrization is a 
    plot for this weak diffeology, since we can lift it into the point $(0,0)$ which
    belongs to each of these lines for any number. This is indeed not satisfactory, 
    since the line jumps from the $x$-axis
    to the $y$-axis, when the parameter crosses each number. 
    This is not a behavior we  want to regard as smooth. 
    The point that we miss with this diffeology is 
    that we cannot follow smoothly the plots of the line, 
    defined by a value of the parameter, 
    when the parameter moves smoothly in its domain. 
    It is why we shall try to fix this diffeology in the following way.
    Let us recall, first of all, that the diffeology $\cD$ is itself a
    diffeological space, and thus, we know what a 
    {\em smooth family of plots of $\X$} is \art{Functional-diffeology-of-a-diffeology}. 
   
    \Question{1)} Show that the parametrizations $\P : \U \to \Powerset(\X)$ 
    defined by the following condition 
    form a diffeology. We shall call it the {\em powerset diffeology}. 
    \begin{itemize}
      \item[($\clubsuit$)] For every $r_0 \in \U$ and for
      every $\Q_0 \in \cD$ such that 
      $\Val(\Q_0) \subset \P(r_0)$, there exist an open neighborhood $\V$ of $r_0$ 
      and a smooth family of plots $\Q : \V \to \cD$, such that 
      $\Q(r_0) = \Q_0$ and $\Val(\Q(r)) \subset \P(r)$, for all $r \in \V$.
    \end{itemize}
    
    \Question{2)} Give a more conceptual
    definition involving local subductions. Compare with being a smooth map
    between diffeological spaces.
    
   \Question{3)} Check that the map $j : x \mapsto \set{x}$, from $\X$
    to $\Powerset(\X)$, is an induction. 
    
    \Question{4)} Show that the {\em Tzim-Tzum}, the parametrization $\cT$ 
    defined by
    $$
    \cT(t) = \{ x \in \RR^2 \mid \norm{x} > t\},
    \ \mbox{for all} \ t \in \RR,   
    $$
    is a plot of the powerset diffeology of $\Powerset(\RR^2)$. 
    Notice how the topology of the subset $\cT(t)$ 
    changes when $t$ passes through $0$.
    
    \Note~This powerset diffeology avoids 
    the phenomenon described  at the beginning. 
    The next exercise shows that it induces, on the set of lines
    of an affine space, a diffeology of manifold, the one we are used to, 
    which is satisfactory. 
    Also note that this construction is presented as an exercise and
    not as an independent section because, as it is an attempt to bring out the 
    right concept of a diffeology on $\Powerset(\X)$, 
    it may not be in its final shape.
  \end{exercise} %% A-powerset-diffeology
  
  \begin{exercise}[The powerset diffeology of the set of lines]
    \label{The-powerset-diffeology-of-the-set-of-lines}
    Let us denote by $\Lines(\RR^n)$ the set of all the (affine) lines of $\RR^n$. 
    Let us recall that an affine line $\DD$ is a part of $\RR^n$ for 
    which there exists a non-zero vector $v \in \RR^n$ 
    such that for any two points $r$ and $r'$
    of $\DD$ there exists $t \in \RR$ such that $r' = r + t v$.
    Let $\T\S^{n-1} \subset \RR^n \times \RR^n$ be the
    {\em tangent space of the $(n-1)$-sphere\/} defined by
    $$
    \T\S^{n-1} = \{ (u,x) \in \RR^n \times \RR^n \mid \norm{u}
    = 1 \mbox{ and } u \cdot x = 0 \},
    $$
    where the dot denotes the usual scalar product on $\RR^n$.
    Let $\Powerset(\RR^n)$ be the powerset of $\RR^n$. Let
    $$
    j : \T\S^{n-1} \to \Lines(\RR^n) \subset \Powerset(\RR^n) 
    \qmbox{with} j(u,x) = \{ x +tu \mid t \in \RR \}.
    $$
    Equip $\T\S^{n-1}$ with the subset diffeology of
    $\RR^n\times \RR^n$ and $\Lines(\RR^n) \subset \Powerset(\RR^n)$
    with the subset diffeology of the powerset diffeology defined 
    in the \exref{A-powerset-diffeology}. Show that $j$ is a subduction onto its image. 
    Check that the pullback of a line $\DD \in \Lines(\RR^n)$ by $j$ 
    is just the pair $(\pm u, x)$, such that 
    $\DD = \set{x + t u \mid t \in \RR}$. 
    Deduce that the set $\Lines(\RR^n)$, equipped with the
    powerset diffeology, is diffeomorphic to the quotient 
    $\T\S^{n-1}/\set{\pm 1}$.
  \end{exercise} %% The-powerset-diffeology-of-the-set-of-lines

  %************************************************
  
  \section*{The Dimension Map of Diffeological Spaces}
  \label{secThe-dimension-map-of-diffeological-spaces}
  
  \begin{sechead}
    Because  diffeological spaces are neither necessarily homogeneous 
    nor transitive \art{Transitive-and-locally-transitive-spaces}, it is
    necessary to refine the notion of (global) dimension of a
    diffeological space \art{Dimension-of-a-diffeological-space} by
    the  {\em dimension map}  \art{The-dimension-map}, which gives the 
    dimension of a diffeological space at each of its points.
  \end{sechead}
  
  \begin{article}\artlabel{Pointed plots and germs of a diffeological space}
    \addcontentsline{toc}{section}{\small\hspace{10pt} Pointed plots and germs
    of a diffeological space}
    \label{Pointed-plots-of-a-diffeological-space}
    Let $\X$ be a diffeological space, $\cD$ be its diffeology, 
    and let $x \in \X$. Let 
    $\P : \U \to \X$ be a plot. We say that $\P$ is {\em centered
    at $x$} if $0 \in \U$ and $\P(0) = x$. We shall agree that
    the set of germs \art{Germs-of-local-smooth-maps} of the
    centered plots of $\X$ at $x$ represents
    the {\em germ of the diffeology} at this point, and we
    shall denote it by $\cD_x$. 
  \end{article} %% Pointed-plots-of-a-diffeological-space
  
  \begin{article}\artlabel{Local generating families}
    \addcontentsline{toc}{section}{\small\hspace{10pt} Local generating
    families}
    \label{Local-generating-families}
    Let $\X$ be a diffeological space and let $x$
    be some point of $\X$. We shall call {\em local generating
    family at $x$}  any family $\cF$ of plots of $\X$ such that the following
    conditions are satisfied.
    \begin{itemize}
      \item[1.] Every element $\P$ of $\cF$ is centered at $x$,
      that is, $0 \in \Dom(\P)$ and $\P(0) =x$.
      \item[2.] For every plot $\P : \U \to \X$
      centered at $x$, there exist an open neighborhood $\V$ of $0 \in
      \U$, a parametrization $\F : \W \to \X$ belonging to
      $\cF$ and  a smooth parametrization $\Q : \V \to \W$,
      centered at $0\in \W$, such that $\F \circ \Q = \P
      \restriction \V$. 
    \end{itemize}
    We shall also say that $\cF$ {\em generates the germ $\cD_x$ of the
    diffeology $\cD$ of
    $\X$ at the point $x$} \art{Pointed-plots-of-a-diffeological-space}. We 
    denote, by analogy with
    \art{Generating-diffeology}, $\cD_x = \langle \cF
    \rangle$.
    
    Note that for all $x \in \X$, the set of local
    generating families at $x$ is
    not empty, since it contains the set of all the
    plots centered at $x$, and this set contains the constant
    parametrizations with value $x$
    \art{Diffeologies-and-diffeological-spaces}.
  \end{article} %% Local-generating-families
  
  \begin{article}\artlabel{Union of local generating families}
    \addcontentsline{toc}{section}{\small\hspace{10pt} Union of local
    generating families}
    \label{Union-of-local-generating-families}
    Let $\X$ be a diffeological space.
    Let us choose, for every $x \in \X$, a local
    generating family $\cF_x$ at $x$. The union
    $\cF$ of all these local generating families, $$\cF =
    \bigcup_{x \in \X}\cF_x,$$ is a generating family of
    the diffeology of $\X$.
  \end{article} %% Union-of-local-generating-families
  
  \begin{proof}
    Let ${\P}: \U \to
    \X$ be a plot, let $r \in \U$ and $x = {\P}(r)$. Let
    $\T_r$ be the translation $\T_r(r')
    = r' + r$. Let $\P'=\P \circ \T_r$ defined on $\U' =
    \T_r^{-1}(\U)$.
    Since the translations are smooth, the parametrization $\P'$ is a
    plot of $\X$. Moreover, $\P'$ is centered at $x$, thus
    $\P'(0) = \P \circ \T_r (0) = \P(r) = x$. 
    By definition of a local generating family \art{Local-generating-families}, 
    there exist an element $\F: \W \to \X$ of $\cF_x$, an open neighborhood
    $\V'$ of $0 \in \U'$ and a smooth parametrization
    $\Q' : \V' \to \W$, centered at $0$, such that 
    $\P' \restriction \V' = \F \circ \Q'$. Thus, 
    $\P \circ \T_r \restriction \V' = \F \circ \Q'$, that is, 
    $\P \restriction \V = \F \circ \Q$, where $\V = \T_r(\V')$ and 
    $\Q = \Q' \circ \T_r^{-1}$. Hence, $\P$ lifts locally, at
    every point of its domain, along an element of $\cF$.
    Thus, $\cF$ is a generating family
    \art{Generating-diffeology} of the diffeology of $\X$.
  \end{proof}
  
  \begin{article}\artlabel{The dimension map}
    \addcontentsline{toc}{section}{\small\hspace{10pt} The dimension map}
    \label{The-dimension-map} 
    Let $\X$ be a diffeological space and $x$ be a point of $\X$. 
    By analogy with the global dimension of $\X$
    \art{Dimension-of-a-diffeological-space}, we define the {\em
    dimension of $\X$ at the point $x$} by 
    $$ 
    \dim_x(\X) = \inf \{ \dim(\cF) \mid  \langle \cF \rangle = \cD_x \},
    $$ 
    where the dimension of
    a family of parametrizations has been defined in
    \art{Dimension-of-a-family-of-parametrizations}. 
    The map $x \mapsto \dim_x(\X)$, with values in 
    $\NN \cup \{\infty\}$, will be called the {\em dimension map} of
    the space $\X$.
  \end{article} %% The-dimension-map
  
  \begin{article}\artlabel{Global dimension and dimension map}
    \addcontentsline{toc}{section}{\small\hspace{10pt} Global dimension and dimension map}
    \label{Global-dimension-and-dimension-map}
    Let $\X$ be a diffeological space. The global
    dimension of $\X$ \art{Dimension-of-a-diffeological-space} is the
    supremum of the dimension map \art{The-dimension-map},
    $$
    \dim(\X) = \sup \set{\dim_x(\X)}_{x \in \X}.
    $$
  \end{article} %% Global-dimension-and-dimension-map
  
  \begin{proof}
    Let $\cD$ be the diffeology of $\X$. 
    Let us prove first that for every $x
    \in \X$, $\dim_x(\X) \leq \dim(\X)$, which implies
    $\sup \set{\dim_x(\X)}_{x \in \X} \leq \dim(\X)$. For that we
    shall prove that for all $x \in \X$ and every generating
    family $\cF$ of $\cD$, $\dim_x(\X) \leq \dim(\cF)$. Then,
    since 
    $\dim(\X) = \inf \{ \dim(\cF) \mid \cF \in \cD \mbox{ and } \langle \cF \rangle = \cD \}$, 
    we shall get $\dim_x(\X) \leq \dim(\X)$. 
    Consider a generating family $\cF$ of $\cD$. 
    For every plot $\P : \U \to \X$, centered at
    $x$, let us choose an element $\F$ of $\cF$ such that there
    exist an open neighborhood $\V$ of $0 \in \U$ and a smooth
    parametrization $\Q : \V \to \Dom(\F)$ such that $\F
    \circ \Q = \P \restriction \V$. Then, let $r = \Q(0)$ and
    $\T_r$ be the translation $\T_r(r') = r' + r$. Let 
    $\F' = \F \circ \T_r$, defined on $\T_r^{-1}(\Dom(\F))$. Thus,
    $\F'(0) = x$, and $\F'$ is a plot of $\X$, centered at
    $x$, such that $\dim(\F') = \dim(\F)$.  Now, 
    $\Q' = \T_r^{-1} \circ \Q$ is smooth and 
    $\P \restriction \V = \F' \circ \Q'$. Hence, the set $\cF'_x$
    of all these plots $\F'$, associated with the plots
    centered at $x$, is a generating family of $\cD_x$, and
    for each of them $\dim(\F') = \dim(\F) \leq \dim(\cF)$.
    Therefore, $\dim(\cF'_x) \leq \dim(\cF)$. But 
    $\dim_x(\X) \leq \dim(\cF'_x)$, thus $\dim_x(\X) \leq \dim(\cF)$.
    We conclude then that $\dim_x(\X) \leq \dim(\X)$, for all
    $x \in \X$, and thus $\sup \set{\dim_x(\X)}_{x \in \X} \leq \dim(\X)$.
    Now, let us prove that $\dim(\X) \leq \sup \set{\dim_x(\X)}_{x \in \X}$. 
    We shall assume that $\sup \set{\dim_x(\X)}_{x \in \X}$ 
    is finite. Otherwise, according to the
    previous part, we have $\sup \set{\dim_x(\X)}_{x \in \X} \leq \dim(\X)$, 
    and then $\dim(\X)$ is
    infinite and $\sup \set{\dim_x(\X)}_{x \in \X} = \dim(\X)$. 
    So, since the sequence of the
    dimensions of the generating families of the $\cD_x$ is lower
    bounded, there exists for every $x$ a generating family
    $\cF_x$ such that $\dim_x(\X) = \dim(\cF_x)$. Then, for every
    $x$ in $\X$ let us choose one of these families. Now, let us define
    $\cF_m$ as the union of all these families we have chosen. Thanks
    to \art{Union-of-local-generating-families}, $\cF_m$ is
    a generating family of $\cD$. Hence, $\dim(\X)
    \leq \dim(\cF_m)$. But, $\dim(\cF_m) = \sup
    \{\dim(\F)\}_{\F \in \cF_m} = \sup \{\sup \{\dim(\F)\}_{\F \in \cF_x}\}_{x \in \X} =
    \sup \{\dim(\cF_x)\}_{x \in \X} = 
    \sup \{\dim_x(\X)\}_{x \in \X}$. Therefore, 
    $\dim(\X) \leq \sup \{\dim_x(\X)\}_{x \in \X}$. 
    We can conclude, from the two parts above,
    that $\dim(\X) = \sup \{\dim_x(\X)\}_{x \in \X}$.
  \end{proof}
  
  \begin{article}\artlabel{The dimension map is a local invariant}
    \addcontentsline{toc}{section}{\small\hspace{10pt} Global The dimension map is a local invariant}
    \label{The-dimension-map-is-a-local-invariant} 
    Let $\X$ and $\X'$ be two diffeological spaces. If $x \in
    \X$ and $x' \in \X'$ are two points  related by a local
    diffeomorphism
    \art{Local-diffeomorphisms}, then 
    $\dim_x(\X) = \dim_{x'}(\X')$. In other words, local diffeomorphisms 
    ({\em a fortiori\/} global diffeomorphisms) can only exchange points where the spaces have the same
    dimension. In particular, for $\X = \X'$, the  dimension map is
    invariant under the local diffeomorphisms of $\X$, that is, constant on
    the orbits of the germs of local diffeomorphisms
    \art{Local-diffeomorphisms}.
  \end{article} %% The-dimension-map-is-a-local-invariant
  
  \begin{proof}
  This proposition is a slight adaptation of
    \art{The-dimension-is-a-diffeological-invariant}.
    Let $f: \X \supset \A \to \X'$ be a local
    diffeomorphism mapping
    $x$ to $x'$. Let
    $\cF$ be a local generating family at $x \in \X$
    \art{Local-generating-families}. Clearly,
    $f
    \circ
    \cF = \set{ f
    \circ \F \mid \F \in \cF}$ is a generating family at
    $x' = f(x) \in \X'$. Conversely, let $\cF'$ be a
    generating family at $x' \in \X'$, then $f^{-1} \circ
    \cF'$ is a generating family at
    $x \in \X$. There is a one-to-one correspondence between the
    local  generating families at $x  \in \X$ and the
    local  generating families at $x' \in \X'$,
    therefore $\dim_x(\X) = \dim_{x'}(\X')$. 
  \end{proof}
  
  \begin{article}\artlabel{Transitive and locally transitive spaces}
    \addcontentsline{toc}{section}{\small\hspace{10pt} Transitive and locally transitive spaces}
    \label{Transitive-and-locally-transitive-spaces} 
    We shall say that a diffeological space $\X$ is 
    {\em transitive} if for any two points $x$ and $x'$ there
    exists a diffeomorphism $\F$ \art{Diffeomorphisms} such that
    $\F(x) = x'$. We shall say that the space is 
    {\em locally transitive} if for any two points
    $x$ and $x'$ there exists a local diffeomorphism $f$
    \art{Local-diffeomorphisms} defined on some superset of $x$ 
    such that $f(x) = x'$. If the space $\X$ is transitive, 
    it is {\em a fortiori\/} locally transitive.
    As a direct consequence of
    \art{The-dimension-map-is-a-local-invariant}, if a
    diffeological space $\X$ is locally transitive, then the
    dimension map \art{The-dimension-map} is constant, 
    {\em a fortiori\/} if the space $\X$ is transitive.
  \end{article} %% Transitive-and-locally-transitive-spaces
  
  \begin{article}\artlabel{Local subduction and dimension}
    \addcontentsline{toc}{section}{\small\hspace{10pt} Local subduction and dimension}
    \label{Local-subduction-and-dimension}
    Let $\X$ and $\X'$ be two diffeological spaces, and let 
    $\pi : \X \to \X'$ be a smooth surjection. Let $x$ be a point of
    $\X$ and $x' = \pi(x)$. If $\pi$ is a local subduction at
    the point $x \in \X$ \art{Local-subductions}, then
    $\dim_{x'}(\X') \leq \dim_x(\X)$.
  \end{article} %% Local-subduction-and-dimension
  
  \begin{proof}
    Let $\cD_x$ and $\cD'_{x'}$ be the germs of the diffeologies of $\X$ at
    the point $x$ and $\X'$ at the point $x' =\pi(x)$. Let
    $\Gen(\cD_x)$ and $\Gen(\cD'_{x'})$ denote the sets of
    all the generating families of $\cD_x$ and $\cD'_{x'}$. 
    We know that for every generating family $\cF$ of $\cD_x$,
    $\pi \circ \cF$ is a generating family of $\cD'_{x'}$,
    that is, $\pi \circ \Gen(\cD_x) \subset \Gen(\cD'_{x'})$.
    Now, for the same reasons as in
    \art{Dimensions-of-quotients}, we get 
    $\dim_{x'}(\X')\leq \dim_x(\X)$. 
  \end{proof} 
    
  %%%%%%%%%%%%%%%%%%%%%%%%%%%%%%%%%%%%%%%%%%%%%%%%%%%%%%%%%%
  %
  %   Exercises  
  %
  %%%%%%%%%%%%%%%%%%%%%%%%%%%%%%%%%%%%%%%%%%%%%%%%%%%%%%%%%%
  
  \Exercise

  \begin{exercise}[The diffeomorphisms of the half line]
    \label{The-diffeomorphisms-of-the-half-line}
    This exercise illustrates how the  invariance of
    the dimension map
    \art{The-dimension-map}, under diffeomorphisms
    \art{The-dimension-map-is-a-local-invariant}, can be used
    to characterize the diffeomorphisms of the half
    line $[0,\infty[ \subset \RR$.
    Show that a
    bijection $f :
    \leftclosedinterval{0,\infty} \to \leftclosedinterval{0,\infty}$ is
    a diffeomorphism, for the subset diffeology induced by
    $\RR$, if and only if the three following conditions are fulfilled. 
    
    \begin{itemize}  
      \item[1)] The origin is fixed, $f(0) = 0$.  
      \item[2)] The restriction of $f$ to the open half line is an
      increasing diffeomorphism of the open half-line, 
      $f \restriction \left]0,\infty\right[ \in {\rm
      Diff}^+(\left]0,\infty\right[)$.  
      \item[3)] The map $f$ is
      infinitely differentiable at the origin and
      its first derivative $f'(0)$ does not vanish.
    \end{itemize} 
    
    \alinea{Hint:} For the first question use the result of
    \exref{Dimension-of-the-half-line} and the invariance of
    the dimension map by diffeomorphisms
    \art{The-dimension-map-is-a-local-invariant}. For the
    second question use the fact that the open interval
    $\openinterval{0,\infty}$ is an orbit of
    $\Diff(\leftclosedinterval{0,\infty})$. For the third
    question use the following Whitney theorem:
    
    \alinea{\sc Theorem} \cite{Whi43} {\em An even function $f(x)
    = f(-x)$, defined on a neighborhood of the origin,  may be
    written as $g(x^2)$. If $f$ is smooth, $g$ may be made smooth.}
  \end{exercise} %% The-diffeomorphisms-of-the-half-line
