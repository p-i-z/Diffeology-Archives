  %%%%%%%%%%%%%%%%%%%%%%%%%%%%%%%%%%%%%%%%%%%%%%%%%%%%%%%%%%
  %% 
  %% Diffeological Vector Spaces MARK: -
  %% 
  %%%%%%%%%%%%%%%%%%%%%%%%%%%%%%%%%%%%%%%%%%%%%%%%%%%%%%%%%%
  
  \chapter{Diffeological Vector Spaces}
  
  \label{chap-Diffeological-vector-spaces}
  \newcommand{\ChapterDVS}{Diffeological vector spaces}
  
  \begin{chaphead}
    Some geometrical objects involve infinite vector spaces,
    for example the construction of the infinite Hilbert
    sphere \art{The-infinite-sphere} or the
    associated infinite projective space
    \art{The-infinite-complex-projective-space}. These
    constructions are perfect examples of how diffeology handles
    infinite dimensional objects. But, in order to describe
    precisely these structures, we need to specialize the
    notion of vector space according to the diffeological
    framework. This chapter is however not devoted to exploring
    all the connections between the notion of diffeological vector space
    and the various kinds of vector spaces we find in the literature, this is 
    still not done and may be a program for a future work. 
    
    In the following chapter, we shall consider the field
    $\RR$ of real numbers equipped with the smooth diffeology
    \art{Smooth-parametrizations-in-domains}. The field $\CC$
    of complex numbers is identified, as real vector space,
    with the product $\RR \times \RR$ by the real isomorphism
    $\fold : (x,y)\mapsto z = x+iy$. A plot of
    $\CC$ is just a parametrization $r\mapsto
    \Z(r) = \P(r)+i\Q(r)$ where $\P$ and $\Q$ are some
    real smooth parametrizations. The letter $\KK$
    will denote $\RR$ or $\CC$.
  \end{chaphead} 
  
  %************************************************
  
  \section*{Basic Constructions and Definitions}
  \label{secBasic-constructions-and-definitions}
  
  \begin{article}\artlabel{Diffeological vector spaces}
    \addcontentsline{toc}{section}{\small\hspace{10pt} Diffeological vector
    spaces} 
    \label{Diffeological-vector-spaces} 
    Let $\E$ be a
    vector space over $\KK$, we shall call {\em vector space
    diffeology} on $\E$ any diffeology of $\E$ such that the
    addition and the multiplication by a scalar
    are smooth, that is,
    $$
    [(u,v)\mapsto u+v] \in \Cinfty(\E \times \E, \E)
    \qmbox{and}  [(\lambda, u) \mapsto \lambda u] \in
    \Cinfty(\KK \times \E, \E),
    $$
    where the spaces $\E \times
    \E$ and $\KK \times \E$ are equipped with the product
    diffeology \art{Building-products-with-spaces}. Every
    vector space $\E$ over $\KK$, equipped with a vector space
    diffeology, is called a {\em diffeological vector space}
    over $\KK$, or a {\em diffeological $\KK$-vector space}.
  \end{article} %% Diffeological-vector-spaces
  
  \begin{article}\artlabel{Standard vector spaces}
    \addcontentsline{toc}{section}{\small\hspace{10pt} Standard vector spaces}
    \label{Standard-vector-spaces}
    Every vector space $\KK^n$, equipped with
    the {\em standard diffeology}, is a diffeological vector
    space. But note that every vector space, equipped with the
    coarse diffeology, is also a diffeological vector space. 
  \end{article} %% Standard-vector-spaces
  
  \begin{article}\artlabel{Smooth linear maps}
    \addcontentsline{toc}{section}{\small\hspace{10pt} Smooth linear maps}
    \label{Smooth-linear-maps}
    Let $\E$ and $\F$ be two diffeological $\KK$-vector
    spaces. We denote by $\DLin(\E,\F)$ the
    space of {\em smooth linear maps} from $\E$ to
    $\F$, $$\DLin(\E,\F) = \Lin(\E,\F)\cap
    \Cinfty(\E,\F).$$ The space $\DLin(\E,\F)$ is a
    $\KK$-vector subspace of $\Lin(\E,\F)$.   
    Since the composite of linear maps is linear and the
    composite of two smooth maps is smooth, diffeological
    vector spaces, together with smooth linear maps, form
    a category which we may denote by $\DLinear$. The isomorphisms of
    this category are the linear isomorphisms, smooth as well as their inverses.
  \end{article} %% Smooth-linear-maps
  
  \begin{proof}
    Let $\A$ and $\B$ belong to $\DLin(\E,\F)$. For any
    plot $\P : \U \to \E$, $(\A + \B) \circ \P = \A \circ \P
    + \B \circ \P$, but $\P_\A = \A \circ \P$ and $\P_\B = \B
    \circ \P$ are plots of $\F$. So, $[r \mapsto (\A + \B)
    \circ \P(r) = \P_\A(r) + \P_\B(r)] = [r \mapsto
    (\P_\A(r), \P_\B(r)) \mapsto \P_\A(r) + \P_\B(r)]$ is the
    composite of two smooth maps. Therefore $\A + \B$ is
    smooth. Let now $\lambda \in \KK$, the map
    $(\lambda \A) \circ \P$ splits into $(\lambda \A)
    \circ \P = [r \mapsto (\lambda, \P_\A(r)) \mapsto \lambda
    \times \P_\A(r)]$. Thus, the addition of two smooth
    linear maps, as well as the multiplication of a smooth
    linear map by a scalar, are smooth linear maps.
    Therefore, $\DLin(\E,\F)$ is a $\KK$-vector subspace of
    $\Lin(\E,\F)$.  \end{proof}
  
  \begin{article}\artlabel{Products of diffeological vectors spaces} 
    \addcontentsline{toc}{section}{\small\hspace{10pt} Products of diffeological vectors spaces} 
    \label{Products-of-diffeological-vectors-spaces}
    Let $\{\E_i\}_{i \in \cI}$ be some family of
    diffeological vector spaces. Let $\E = \prod_{i \in
    \cI} \E_i$ be the product of the family. The
    product $\E$ is naturally a vector space,
    for all $v, v' \in \E$ , for all $\lambda \in \KK$,
    $$ 
    (v+v')(i) = v(i) +v'(i) \qmbox{and} (\lambda v)(i) = \lambda
    \times v(i). 
    $$ 
    The space $\E$, equipped with the
    product diffeology \art{Building-products-with-spaces},
    is a diffeological vector space. 
  \end{article} %% Products-of-diffeological-vectors-spaces
  
  \begin{proof}
    Let us check that the addition and the multiplication by a
    scalar are smooth. Let $\P : r \mapsto v_r$ and 
    $\P' : r \mapsto v'_r$ be two plots of $\E$. So, for all
    $i \in \cI$ the parametrizations $r \mapsto  v_r(i)$ and
    $r \mapsto  v'_r(i)$ are two plots of $\E_i$. Then, $r
    \mapsto (v_r + v'_r)(i) = v_r(i) + v'_r(i)$ is a plot of
    $\E_i$, since the addition is smooth in $\E_i$.
    As well, $(\lambda , r) \mapsto  (\lambda v_r)(i) =
    \lambda \times v_r(i)$ is a plot of $\E_i$. Therefore,
    $\E$ is a diffeological vector space. \end{proof}
  
  \begin{article}\artlabel{Diffeological vector subspaces} 
    \addcontentsline{toc}{section}{\small\hspace{10pt} Diffeological vector subspaces} 
    \label{Diffeological-vector-subspaces}
    Any vector subspace $\F$ of a diffeological vector space
    $\E$, equipped with the subset diffeology, is a
    diffeological vector space.
  \end{article} %% Diffeological-vector-subspaces
  
  \begin{proof}
    Let $r \mapsto v_r$ and $r \mapsto v'_r$ be two plots of
    $\F$ for the subset diffeology, and defined on the same
    domain, that is, two plots of $\E$ with values in $\F$.
    So, the sum $r \mapsto v_r + v'_r$ is a plot of $\E$, and
    since $\F$ is a vector subspace of $\E$, $v_r + v'_r$
    belongs to $\F$ for all $r$. Thus, $r \mapsto v_r + v'_r$
    is a plot of $\F$ for the subset diffeology. Thus, the sum
    is smooth. For analogous reasons, $(\lambda, r) \mapsto
    \lambda \times v_r$ is a plot of $\F$, and the
    multiplication by a scalar is smooth. Therefore $\F$,
    equipped with the subset diffeology, is a diffeological
    vector space.
  \end{proof}
  
  \begin{article}\artlabel{Quotient of diffeological vector spaces} 
    \addcontentsline{toc}{section}{\small\hspace{10pt} Quotient of diffeological vector spaces} 
    \label{Quotient-of-diffeological-vector-spaces}
    Let $\E$ be a diffeological vector space, let
    $\F\subset \E$ be a vector subspace. The quotient vector
    space $\E/\F$ is a diffeological vector space for the
    quotient diffeology
    \art{Quotient-and-quotient-diffeology}.
  \end{article} %% Quotient-of-diffeological-vector-spaces
  
  \begin{proof}
    Let $\E' = \E/\F$ and $r \mapsto v_r$ be a plot of
    the quotient diffeology. So, locally, $v_r = [e_r]$,
    where $r \mapsto e_r$ is a plot of $\E$, and $[e] \in \E'$ means
    the class of $e \in \E$. Now,
    let $r \mapsto v_r$ and $r \mapsto v'_r$ be two plots of
    $\E'$ defined on the same domain $\U$. So, for any
    $r_0 \in \U$ there exist an open neighborhood $\cO$ of $r_0$ and
    two plots $r \mapsto e_r$ and $r \mapsto e'_r$
    defined on $\cO$ such that $v_r = [e_r]$ and $e'_r =
    [v'_r]$. Hence, $v_r + v'_r = [e_r] + [e'_r] = [e_r
    + e'_r]$, and then  $r \mapsto e_r + e'_r$ is a local lift
    of $r \mapsto v_r + v'_r$. Thus, by definition of the
    quotient diffeology, the addition on $\E'$ is
    smooth. As well, $(\lambda, r) \mapsto
    \lambda \times e_r$ lifts $(\lambda , r) \mapsto
    \lambda \times v_r$, where $\lambda \in \KK$. Therefore,
    $\E'$ is a diffeological vector space for the quotient
    diffeology. \end{proof}
  

  %%%%%%%%%%%%%%%%%%%%%%%%%%%%%%%%%%%%%%%%%%%%%%%%%%%%%%%%%%
  %
  %   Exercises  
  %
  %%%%%%%%%%%%%%%%%%%%%%%%%%%%%%%%%%%%%%%%%%%%%%%%%%%%%%%%%%
  
  \Exercise
  
  \begin{exercise}[Vector space of maps into $\KK^n$]
    \label{Vector-space-of-maps-into-Kn}
    Let $\X$ be a diffeological space. Let $\E
    = \Cinfty(\X,\KK^n)$ be the space of smooth maps from
    $\X$ to $\KK^n$. The set $\E$ is a $\KK$-vector space
    for the pointwise addition and multiplication by a
    scalar. Precisely, for all $f$, $f'$ in $\E$ and $\lambda$
    in $\RR$, $f+f' = [x \mapsto f(x) + f'(x)]$ and $\lambda
    f = [x \mapsto \lambda \times f(x)]$. Check that $\E$,
    equipped with the functional diffeology, is a
    diffeological $\KK$-vector space.  
  \end{exercise}
  
  %************************************************
  
  \section*{Fine Diffeology on Vector Spaces}
  \label{secFine-diffeology-on-vector-spaces}
  
  \begin{sechead}
    Every $\KK$-vector space equipped with the coarse
    diffeology is obviously a diffeological vector space,
    what is not really interesting. But every vector space
    has a finest vector space diffeology, which we shall call
    the {\em fine diffeology}. In this section we study
    some aspects of this fine diffeology.
  \end{sechead}
  
  \begin{article}\artlabel{The fine diffeology of vector spaces} 
    \addcontentsline{toc}{section}{\small\hspace{10pt} The fine diffeology of vector spaces}
    \label{The-fine-diffeology-of-vector-spaces} 
    There exists, on every $\KK$-vector space $\E$, a
    finest diffeology of vector space. We shall call it  
    {\em the fine diffeology}. This diffeology is generated
    \art{Generating-diffeology} by the family of
    parametrizations defined by 
    \renewcommand{\theequation}{$\heartsuit$}
    \begin{equation} \P : r \mapsto \sum_{\alpha \in \A}
      \lambda_\alpha(r) v^\alpha,  
    \end{equation} 
    where $\A$ is a finite set of indices, the $\lambda_\alpha$ are smooth
    $\KK$-parametrizations defined on the domain of $\P$, and
    $v^\alpha$ are vectors of $\E$. 
    More precisely, the plots of the fine diffeology are the
    parametrizations $\P : \U \to \E$ such that, for all 
    $r_0 \in \U$ there exist an open neighborhood $\V$ of $r_0$, 
    a family of smooth parametrizations $\lambda_\alpha : \V \to
    \KK$ and a family of vectors $v^\alpha \in \E$, both
    indexed by the same finite set of indices $\A$, such that 
    \renewcommand{\theequation}{$\diamondsuit$}
    \begin{equation}
      (\P \restriction \V) : r \mapsto 
      \sum_{\alpha \in \A}\lambda_\alpha(r) v^\alpha.
    \end{equation}
    A finite family
    $(\lambda_\alpha,v^\alpha)_{\alpha \in \A}$, defined on
    some domain $\V$, and satisfying $(\diamondsuit)$ with
    $\lambda_\alpha \in \cC^{\infty}(\V,\KK)$ and $v^\alpha
    \in \E$, will be called a {\em local family} for the plot
    $\P$. We shall agree that plot properties satisfied only locally, 
    over some non specified domains, 
    will be denoted with the subscript ${}_{\rm loc}$, 
    for example $=_{\rm loc}$ etc. 
  \end{article} %% The-fine-diffeology-of-vector-spaces
  
  \begin{proof}
  Let us prove that the set of parametrizations
    described by $(\diamondsuit)$ is the finest vector
    space diffeology.
    
    \alinea{1.~(Diffeology)} The condition $(\diamondsuit)$ is
    clearly the specialization, for the family defined in
    $(\heartsuit)$, of the criterion
    \art{Criterion-of-generation} for generating families.
    
    \alinea{2.~(Diffeology of vector space)} Let $r\mapsto
    (\P(r),\Q(r))$ be a plot of the product $\E\times \E$. Let
    $(\lambda_\alpha,u^\alpha)_{\alpha \in \A}$ and
    $(\mu_\beta,v^\beta)_{\beta \in \B}$ be two local families
    such that 
    $$
    \P(r) =_{\rm loc} \sum_{\alpha \in \A}
    \lambda_\alpha(r) u^\alpha \qmbox{and} \Q(r) =_{\rm loc}
    \sum_{\beta \in \B} \mu_\beta(r) v^\beta.
    $$ 
    So, the addition $\P + \Q$ writes 
    $$ 
    \P + \Q \vert_{\rm
    loc} : r\mapsto \sum_{\alpha \in \A} \lambda_{\alpha}(r)
    u^{\alpha} + \sum_{\beta \in \B} \mu_{\beta}(r) v^{\beta} =
    \sum_{\sigma \in \C} \nu_{\sigma}(r) w^{\sigma},
    $$
    where $\C$ is just the adjunction of the two sets of
    indices $\A$ and $\B$, and the family
    $(\nu_{\sigma},w^\sigma)_{\sigma \in \C}$ the adjunction
    of the local families $(\lambda_\alpha,u^\alpha)_{\alpha
    \in \A}$ and $(\mu_\beta,v^\beta)_{\beta \in \B}$. Hence,
    the addition is smooth. On the other hand, since the
    multiplication by a scalar is smooth in $\KK$,
    the multiplication by a scalar in $\E$ also is
    smooth. Therefore, this diffeology is a vector
    space diffeology. 
    
    \alinea{3.~(Fineness)} Let us consider $\E$, equipped with
    some other vector space diffeology $\cD$.   
    Since the multiplication by a scalar is smooth, for
    any smooth parametrization $\lambda$ of $\KK$ and any
    vector $u \in \E$, the parametrization $r\mapsto
    \lambda(r)u$ is smooth. Now, since
    the addition is smooth, for any finite local family
    $(\lambda_\alpha,u^\alpha)_{\alpha \in \A}$,  the
    parametrization $r\mapsto \sum_{\alpha \in \A}
    \lambda_{\alpha}(r) u^{\alpha}$ is smooth, that is,
    a plot of the diffeology $\cD$. Thus, the diffeology $\cD$
    is coarser than the fine diffeology defined above. Hence,
    the fine diffeology is the finest vector
    space diffeology of $\E$. 
  \end{proof}
  
  \begin{article}\artlabel{Generating the fine diffeology}
    \addcontentsline{toc}{section}{\small\hspace{10pt} Generating the fine diffeology} 
    \label{Generating-the-fine-diffeology}
    Let $\E$ be a vector space on $\KK$ and let $\Lin(\KK^n,\E)$
    be the set  of all the linear maps from $\KK^n$ into $\E$.
    Let $\Lin^\star(\KK^n,\E)$ be the set of all the injective
    linear maps from $\KK^n$ into $\E$, 
    $$ 
    \Lin^\star(\KK^n,\E)
    = \{ j \in \Lin(\KK^n, \E) \mid \ker(j) = \{0\} \}. 
    $$ 
    The following two families both generate the fine diffeology of $\E$.
    $$
    \cF = \bigcup_{n \in \NN} \Lin(\KK^n,\E) \qmbox{and}
    \cF^\star = \bigcup_{n \in \NN} \Lin^\star(\KK^n,\E).
    $$
    
    \Note~A parametrization $\P:\U\to \E$ is a plot
    for the diffeology generated by $\cF$ if and only if, for
    all $r_0 \in \U$ there exist an open neighborhood $\V$ of $r_0$
    in $\U$, an integer $n$, a smooth parametrization
    $\phi : \V \to \KK^n$, and a linear map $j : \KK^n \to \E$
    such that $\P \restriction \V = j \circ \phi$. In other
    words, locally, $\P$ takes its values in a constant
    finite dimensional subspace $\F  \subset  \E$, such that
    the coordinates of $\P$, for some basis of $\F$, are
    smooth. For the plots generated by $\cF^\star$, $j$ is
    injective.
  \end{article} %% Generating-the-fine-diffeology
  
  \begin{proof}
  Let us prove that $\cF$, as well as
    $\cF^\star$, generates the fine diffeology. 
    Let $\P : \U \to \E$ be a plot of the diffeology
    generated by $\cF$, or by $\cF^\star$. Pick a point $r_0$
    in $\U$. By definition, there exist an open neighborhood $\V$ of
    $r_0$, an integer $n$, a smooth parametrization $\phi :
    \V \to \KK^n$, and a linear map $j : \KK^n \to \E$ such
    that $\P \restriction \V = j \circ \phi$. Thus, for all
    $r$ in $\V$, $\phi(r) =  \sum_{k = 1}^n \phi_k(r)
    \ee_k$, where $(\ee_1,\ldots,\ee_n)$ is the canonical
    basis of $\KK^n$, and $\phi_k \in \cC^{\infty}(\V,\KK)$.
    Now, $\P(r) = j(\sum_{k = 1}^n \phi_k(r) \ee_k) =
    \sum_{k = 1}^n \phi_k(r) j(\ee_k) =  \sum_{k = 1}^n
    \phi_k(r) \ff_k$, where $\ff_k = j(\ee_k)$. Therefore,
    $\P$ is a plot of the fine diffeology of $\E$, and
    $(\phi_k,\ff_k)_{k = 1}^n$ is a local family of the plot
    $\P$. Note that $j$ can be chosen injective.
    
    Conversely, let $\P : \U \to \E$ be a plot of the fine
    diffeology and let $r_0$ be a point of $\U$. There exist 
    an open neighborhood $\V$ of $r_0$ in $\U$, an integer $\N$, 
    a local family $(\lambda_\alpha,v^\alpha)_{\alpha = 1}^\N$,
    with $\lambda_\alpha\in \cC^{\infty}(\V,\KK)$, $v^\alpha
    \in \E$, and such that $\P\restriction \V = \sum_{\alpha
    = 1}^\N \lambda_\alpha(r) v^\alpha$. Let $\F$ be the
    vector space generated by the $v^\alpha$, and let $\ff =
    (\ff_1,\ldots,\ff_n)$ be a basis of $\F$. 
    Let us split the vectors $v^\alpha$ on the basis $\ff$,
    $v^\alpha = \sum_{k = 1}^n v_k^\alpha \ff_k$. Now,
    $\P\restriction \V = \sum_{\alpha = 1}^\N\sum_{k = 1}^n
    \lambda_\alpha(r) v _k^\alpha \ff_k = \sum_{k = 1}^n
    \phi_k(r) \ff_k$, where $\phi_k(r) = \sum_{\alpha = 1}^\N
    \lambda_\alpha(r) v_k^\alpha$. The $\phi_k$ are smooth
    maps defined on $\V$ with values in $\KK$. Now, let $j :
    \KK^n \to \E$ be the linear map defined by $j(\ee_k) =
    \ff_k$ and $\phi : \V \to \KK^n$ defined by $\phi = 
    (\phi_1,\ldots,\phi_n)$. Hence, $\P \restriction \V = j
    \circ \phi$, where $j$ is an injective linear map from
    $\KK^n$ to $\E$ and $\phi$ belongs to
    $\cC^{\infty}(\V,\KK^n)$. Therefore, $\P$ is a plot of
    the diffeology generated by $\cF^\star$, {\em a fortiori\/} by
    $\cF$. Hence, the fine diffeology of $\E$ is generated
    by the set of linear maps, or injective linear maps,
    from $\KK^n$ into $\E$, when $n$ runs over the integers.
  \end{proof}
  
  \begin{article}\artlabel{Linear maps and fine diffeology}
    \addcontentsline{toc}{section}{\small\hspace{10pt} Linear maps and fine diffeology} 
    \label{Linear-maps-and-fine-diffeology} 
    Let $\E$ and $\F$ be two diffeological vector spaces over
    $\KK$. Let $\E$ be equipped with the fine diffeology. 
    Every linear map from $\E$ to $\F$ is smooth. In other
    words, if $\E$ is fine, $\DLin(\E,\F) =
    \L(\E,\F)$.  In particular, if both $\E$ to $\F$ are fine vector spaces, 
    every linear isomorphism from $\E$ to $\F$ is a smooth linear isomorphism. 
  \end{article} %% Linear-maps-and-fine-diffeology
  
  \begin{proof}
    Let $(\P \restriction \V)(r) = \sum_{\alpha = 1}^\N
    \lambda_\alpha(r) v_\alpha$ be a local expression of some
    plot $\P$ of $\E$. Let $\A\in \L(\E,\F)$, thus we have
    $(\A\circ \P \restriction \V)(r) = \sum_{\alpha = 1}^\N
    \lambda_\alpha(r) \A(v_\alpha)$. Since $\A(v_\alpha) \in
    \F$ for each $\alpha$, $\P$ is a plot of the fine
    diffeology of $\F$, thus a plot of any vector space
    diffeology. Hence, the linear map $\A$ is smooth,
    and $\L(\E,\F) \subset \DLin(\E,\F)$. The
    converse inclusion is a part of the definition. 
  \end{proof}
  
  \begin{article}\artlabel{The fine linear category} 
    \addcontentsline{toc}{section}{\small\hspace{10pt} The fine linear category} 
    \label{The-fine-linear-category}
    Thanks to \art{Linear-maps-and-fine-diffeology} the fine
    diffeological spaces define a subcategory of the linear
    diffeological category
    \art{Smooth-linear-maps}, let us denote it by $\FineLinear$.
    The objects of this category are all vector spaces, for
    the field $\KK$. According to the above
    proposition \art{Linear-maps-and-fine-diffeology}, the
    morphisms of this category are just the linear maps. Hence,
    the {\em fine linear category} coincides with the usual
    linear category over $\KK$. In other words,
    the functor from  $\Linear$ to $\FineLinear$, which
    associates with every vector space the same space equipped
    with the fine diffeology, is a full faithful functor.  
  \end{article} %% The-fine-linear-category
  
  \begin{article}\artlabel{Injections of fine diffeological vector spaces} 
    \addcontentsline{toc}{section}{\small\hspace{10pt} Injections of fine diffeological vector spaces}
    \label{Injections-of-fine-diffeological-vector-spaces}
    Let $\E$ and $\E'$ be two $\KK$-vector spaces
    equipped with the fine diffeology. Every linear injection 
    $f : \E \mapsto \E'$ is an induction. In particular, every
    subspace $\F \subset \E$, where $\E$ is a fine
    diffeological vector space, inherits from $\E$ the fine diffeology.
  \end{article} %% Injections-of-fine-diffeological-vector-spaces
  
  \begin{proof}
    The injection $ f : \E \to \E'$ is linear, thus
    it is smooth 
    \art{Linear-maps-and-fine-diffeology}. Let us check now
    that if a parametrization $\P : \U \to \E$ is such that
    $f \circ \P$ is a plot of $\E'$, then $\P$ is a plot of
    $\E$ \art{What-is-an-induction}. Let $r_0 \in \U$ be some
    point, there exist an open neighborhood $\V$ of $r_0$, an
    injection $j:\KK^n\to \E'$, and a smooth parametrization
    $\phi : \V \to \KK^n$ such that $f \circ \P \restriction
    \V = j \circ \phi$ \art{Generating-the-fine-diffeology}.
    Since $f \circ \P \restriction \V$ takes its values in
    $f(\E)$, $j(\Val(\phi)) \subset f(\E)$. Let us denote by
    $\H \subset \KK^n$ the vector space spanned by
    $\Val(\phi)$, that is, the smallest vector subspace of
    $\KK^n$ containing $\Val(\phi)$. Since $j$ is linear,
    $j(\Val(\phi)) \subset f(\E)$ implies $j(\H) \subset
    f(\E)$. Now, let $\cB : \KK^{m} \to \H$ be a linear
    isomorphism (a basis of $\H$), let $\phi' = \cB^{-1}
    \circ \phi$ and $j' = j \circ \cB$. Thus, $f \circ \P
    \restriction \V = j' \circ \phi'$, where $\phi'$ is a
    smooth parametrization in $\KK^{m}$ and $j'$ is a linear
    injection from $\KK^{m}$ in $f(\E)$. Since $f$ is
    injective and $\Val(j') \subset f(\E)$, $j'' = f^{-1}
    \circ j'$ is an injection from $\KK^m$
    into $\E$. Thus, $f \circ \P \restriction \V = j'
    \circ \phi'$ implies $ \P \restriction \V = j''
    \circ \phi'$. Hence, $\P$ is a plot of the fine
    diffeology of $\E$. Therefore, $f$ is an induction.
  \end{proof}
  
  \begin{article}\artlabel{Functional diffeology between fine spaces}
    \addcontentsline{toc}{section}{\small\hspace{10pt} Functional diffeology between fine spaces}
    \label{Functional-diffeology-between-fine-spaces}
    Let $\E$ and $\E'$ be two fine vector spaces over
    $\KK$. The functional diffeology
    \art{Functional-diffeologies} of the space of linear maps
    $\DLin(\E,\E')=\Lin(\E,\E')$ is characterized
    as follows.
    
      \alinea{($\clubsuit$)} If $\E$ and $\F$ are finite
      dimensional spaces, $\dim(\E) = n$ and $\dim(\F) = m$,
      a parametrization $\P : \U \to \Lin(\E,\F)$ is a plot of
      the functional diffeology if and only if the coefficients
      $\P_{i,j}$ of the matrix associated with $\P$, for some
      basis $\cE = \{\ee_i\}_{i=1}^n$ of $\E$ and $\cF =
      \{\ff_j\}_{j = 1}^m$ of $\F$, are smooth parametrizations
      of $\KK$. Briefly, $\P_{i,j} \in \Cinfty(\U,\KK)$, for all
      $i=1\ldots n$ and $j = 1\ldots m$. 
      
      \alinea{($\spadesuit$)} More generally, for two spaces $\E$ and $\F$ 
      of any dimensions, a parametrization $\P:\U \to \Lin(\E,\E')$
      is a plot for the functional diffeology if and only if,
      for all $r_0 \in \U$ and for all vector subspaces $\F \subset \E$ 
      of finite dimension, 
      there exist an open neighborhood $\V$ of $r_0$, 
      and a vector subspace of finite
      dimension $\F' \subset \E'$ such that the two following conditions are
      satisfied.
      
      \begin{itemize}
      \item[1.] For all $r \in \V$, the linear map $\P(r) \restriction \F$ 
      belongs to $\Lin(\F,\F')$.
      
      \item[2.] The parametrization $r\mapsto \P(r)\restriction \F$,
      restricted to $\V$, is a plot of $\Lin(\F,\F')$.
      \end{itemize}

    \Note~Thanks to ($\clubsuit$), considering the second condition of ($\spadesuit$), 
    the parametrization $r \mapsto \P(r) \restriction \F$ is a
    plot of $\Lin(\F,\F')$ if and only if each coefficient of
    the associated matrix is a smooth parametrization in $\KK$,
    for some bases of $\F$ and $\F'$.  
  \end{article} %% Functional-diffeology-between-fine-spaces
  
  \begin{proof}
    ($\clubsuit$) Every basis
    of a finite dimensional fine vector space is a smooth
    isomorphism, see \exref{Finite-dimensional-fine-spaces}.
    Hence, the question is reduced to the functional diffeology
    of $\Lin(\KK^n,\KK^m)$, where $\KK^n$ and $\KK^m$ are
    equipped with the smooth diffeology. Let $\P : \U \to
    \Lin(\KK^n,\KK^m)$ be a plot of the functional diffeology.
    By definition, for any vector
    $\ee_i$, $i = 1 \ldots n$, of the canonical basis of
    $\KK^n$, the parametrization $r \mapsto \P(r)(\ee_i) =
    \sum_{j = 1}^n \P_{i,j}(r) \ff_j$, where
    $\ff_j$ is the $j$-th vector of the canonical basis of
    $\KK^m$, is smooth. But the parametrization $\P_{i,j}$ is the
    composite of $r \mapsto \P(r)(\ee_i)$ with the $j$-th
    projection from $\KK^m$ to $\KK$, which is smooth, by
    definition of the smooth diffeology of $\KK^m$. Hence, the
    parametrizations $\P_{i,j}$ are smooth. 
    
    Conversely, let $\P : \U \to \Lin(\KK^n,\KK^m)$ be a
    parametrization such that all the components $\P_{i,j}$
    of $\P$, for the canonical basis of $\KK^n$ and $\KK^m$,
    are smooth. Let $\Q : \V \to \KK^n$ be a plot of
    $\KK^n$. By definition of the smooth diffeology, $\Q : s
    \mapsto \sum_{i = 1}^n \phi_i(s) \ee_i$, where the
    $\phi_i$ are smooth. Thus, for all $(r,s) \in \U \times
    \V$:
    $$ \renewcommand{\arraystretch}{1.5}
    \begin{array}{lclcl}
      \P(r)(\Q(s)) & = & \P(r) (\sum_{i = 1}^n
      \phi_i(s) \ee_i) & = & \sum_{i = 1}^n \phi_i(s)
      \P(r)(\ee_i) \\
      & = &
      \sum_{i = 1}^n \phi_i(s) \sum_{j = 1}^m \P_{i,j}(r) \ff_j
      & = & \sum_{j = 1}^m \sum_{i = 1}^n \phi_i(s) \P_{i,j}(r) 
      \ff_j \\
      & = & \sum_{j = 1}^m \psi_j(r,s) \ff_j & \& &
      \psi_j(r,s) = \sum_{i = 1}^n \phi_i(s) \P_{i,j}(r). 
    \end{array} $$
    Since the $\psi_j$ are smooth, the parametrization $(r,s)
    \mapsto \P(r)(\Q(s))$ is a plot of $\KK^m$. Therefore
    $\P$ is smooth.
    Let us now consider ($\spadesuit$), let
    $\P : \U \to \Lin(\E,\E')$ be a plot for the
    functional diffeology. Let us show that it satisfies the
    condition of the proposition. Let $\F \subset \E$ be a
    vector subspace of finite dimension. Let
    $(u_1,\ldots,u_m)$ be a basis of $\F$. Let $r_0 \in \U$,
    by definition of the functional diffeology, for every integer $k =
    1\ldots m$, the map $r\mapsto \P(r)(u_k)$ is a plot of
    $\E'$. So, there exist an open neighborhood $\V_k$ of $r_0$, a
    finite set of indices $\A_k$, a family
    $(\lambda_{k,\alpha})_{\alpha \in \A_k}$ of smooth
    parametrizations of $\KK$, a family
    $(w_{k,\alpha})_{\alpha \in \A_k}$ of vectors of $\E$,
    such that, for all $r \in \V_k$: 
    $$ 
    \P(r)(u_k) =
    \sum_{\alpha \in \A_k} \lambda_{k,\alpha}(r)
    w_{k,\alpha}. 
    $$ 
    Hence, for all $u = \sum_{k=1}^m c_k u_k$,
    where $c_k \in \KK^m$, for all $r \in \V =
    \cap_{k = 1}^m \V_k$, we have
    $$
    \P(r)(u) = \P(r)\bigg(\sum_{k = 1}^m c_k u_k\bigg) = 
    \sum_{k = 1}^m c_k \P(r)(u_k)
    = 
    \sum_{k = 1}^m
    \sum_{\alpha \in \A_k} c_k \lambda_{k,\alpha}(r)
    w_{k,\alpha}.
    $$
    Let $\F'$ be the subspace of $\E'$ spanned by the vectors
    $\cup_{k = 1}^m\{w_{k,\alpha}\}_{\alpha \in \A_k}$. Hence,
    for every $u \in \F$, $\P(r)(u) \in \F'$, that is,
    $\P(r)(\F) \subset \F'$, for all $r \in \V$. The first
    condition of ($\spadesuit$) is checked. 
    %
    Now, let $(v_1,\ldots,v_n)$ be a basis of $\F'$, such
    that for every integer $k=1 \ldots m$ and every $\alpha \in \A_k$,
    $w_{k,\alpha} = \sum_{j=1}^n w_{k,\alpha}^j v_j$.
    Replacing $w_{k,\alpha}$ by this expression we get
  \begin{eqnarray*}
    \P(r)\bigg(\sum_{k = 1}^m c_k u_k\bigg)  & = & \sum_{k = 1}^m
    \sum_{\alpha \in \A_k} c_k \lambda_{k,\alpha}(r)
    \sum_{j=1}^n w_{k,\alpha}^j v_j \\
    & = & \sum_{j=1}^n
    \bigg(\sum_{k = 1}^m \sum_{\alpha \in \A_k} c_k
    \lambda_{k,\alpha}(r)
    w_{k,\alpha}^j\bigg) v_j.
  \end{eqnarray*}
    Hence, defining
    $$
    \phi_j(r) = \sum_{k = 1}^m
    \sum_{\alpha \in \A_k} c_k \lambda_{k,\alpha}(r)
    w_{k,\alpha}^j\,, \qmbox{we get} \P(r)(u) = \sum_{j = 1}^n \phi_j(r) v_j,
    $$
    where the $(\phi_j)_{j=1}^n$ are a family of smooth parametrizations
    of $\KK$ defined on $\V$.    
    This expression of $\P(r)$ clearly shows that $r\mapsto
    \P(r)\restriction \F$ is a plot of the functional
    diffeology of $\Lin(\F,\F')$. Indeed, choosing
    for $u$ successively each vector of a basis of $\F$, the
    last expression shows that the components of
    $\P(r)\restriction \F$ are smooth parametrizations of
    $\KK$, which is the condition, for finite dimensional
    fine spaces, to be a plot of the functional diffeology.
    Hence, we proved the direct way of the proposition.
    %
    Conversely, let us assume that the parametrization $\P$
    satisfies the conditions 1 and 2 of ($\spadesuit$), and
    let us show that $\P$ is a plot for the functional
    diffeology. Let us consider a plot $\Q:\V\to \E$. By
    definition of the fine diffeology, for all $s_0 \in \V$
    there exist an open neighborhood $\W$ of $s_0$, a finite set of
    indices $\A$, a family $(\lambda_\alpha)_{\alpha \in \A}$
    of smooth parametrizations of $\KK$ defined on $\V$, and a
    family $(v_\alpha)_{\alpha \in \A}$ of vectors of $\E$,
    such that for every $s \in \V$, $\Q(s) = \sum_{\alpha \in
    \A} \lambda_\alpha(s) v_\alpha$. Let $\F \subset \E$ be
    the vector subspace spanned by the vectors $v_\alpha$.
    Hence, for all $r_0 \in \U$, there exist an open neighborhood
    $\U'$ of $r_0$ and a vector subspace $\F' \subset \E'$ 
    such that $\P(r)(\F) \subset \F'$, for all $r \in \U'$. 
    Thus, for all $(r,s) \in \U'\times \W$,
    $\P(r)(\Q(s)) = \P(r)(\sum_{\alpha \in 
    \A}\lambda_\alpha(s) v_\alpha) = \sum_{\alpha \in 
    \A}\lambda_\alpha(s) \P(r)(v_\alpha) \in \F'$. Then, since
    the parametrization $r\mapsto \P(r)\restriction \F$ is a
    plot of the functional diffeology, the parametrization
    $\P \cdot \Q : (r,s)\mapsto \P(r)(\Q(s))$ is a smooth
    parametrization in $\F' \subset \E'$. Thus, $\P \cdot \Q$
    is a smooth parametrization in $\E'$, because any finite
    subspace is embedded in $\E'$. Therefore, $\P$ is a plot
    of the functional diffeology of $\Lin(\E,\E')$. This
    completes the proof of the proposition. 
  \end{proof}
  
  %%%%%%%%%%%%%%%%%%%%%%%%%%%%%%%%%%%%%%%%%%%%%%%%%%%%%%%%%%
  %
  %   Exercises  
  %
  %%%%%%%%%%%%%%%%%%%%%%%%%%%%%%%%%%%%%%%%%%%%%%%%%%%%%%%%%%
  
  \Exercises
  
  \begin{exercise}[Smooth is fine diffeology]
    \label{Smooth-is-fine-diffeology} 
    Check that the smooth diffeology of $\KK^n$ is the fine
    diffeology.  
  \end{exercise} %% Smooth-is-fine-diffeology
  
  \begin{exercise}[Finite dimensional fine spaces]
    \label{Finite-dimensional-fine-spaces} 
    Let $\E$ be a $\KK$-vector space of dimension $n$,
    equipped with the fine diffeology. Let $\cB =
    \{\ee_i\}_{i=1}^n$ be any basis of $\E$. Show that any
    plot $\P : \U \to \E$ writes $\P : r \mapsto \sum_{i=1}^n
    \phi_i(r) \ee_i$, where $\phi_i \in
    \Cinfty(\U, \KK)$. In other words, a basis $\cB$, regarded
    as the isomorphism $\cB : (u_1,\ldots,u_n) \mapsto
    \sum_{i=1}^n u_i \ee_i$, from $\KK^n$ to $\E$, where $\KK^n$ is equipped with the standard
    smooth diffeology, is a smooth
    isomorphism. Any
    fine vector space of dimension $n$ is isomorphic to the
    standard $\KK^n$. 
  \end{exercise} %% Finite-dimensional-fine-spaces
  
  \begin{exercise}[The fine topology]
    \label{The-fine-topology}
    Let $\E$ be a fine diffeological vector space. Show that
    a subset $\Omega \subset \E$ is D-open (open for the
    D-topology) \art{The-D-Topology-of-diffeological-spaces}
    if and only if its intersection with any finite
    dimensional vector space $\F \subset \E$ is open in
    $\F$. This topology is the so-called {\em finite
    topology}, introduced in functional analysis by Andrei
    Tychonoff \cite{Tyc35}. 
  \end{exercise} %% The-fine-topology
  
  %************************************************
  
  \section*{Euclidean and Hermitian Diffeological Vector Spaces} 
  \label{secEuclidean-and-Hermitian-diffeological-vector-spaces}
  
  \begin{sechead}
    The notions of Euclidean or Hermitian structures on
    diffeological vector spaces are natural extensions of the
    standard definitions. They are used
    in particular in the diffeological descriptions of the
    infinite sphere \art{The-infinite-sphere} and the infinite
    projective space
    \art{The-infinite-complex-projective-space}.
  \end{sechead}
  
  \begin{article}\artlabel{Euclidean diffeological vector spaces}
    \addcontentsline{toc}{section}{\small\hspace{10pt} Euclidean diffeological vector spaces}
    \label{Euclidean-diffeological-vector-spaces}
    Let $\E$ be a real diffeological vector space,
    and let $(\X,{\Y}) \mapsto \X \cdot {\Y}$ be an
    {\em Euclidean product}, that is, a map from $\E \times
    \E$ to $\RR$ which is:
    $$
    \begin{array}{rcl}
      \mbox{Symmetric} &: & \X \cdot {\Y} = {\Y}\cdot \X, \\
      \mbox{Bilinear} &: & \X \cdot (\alpha {\Y} + \alpha' {\Y}') =
      \alpha  \X \cdot {\Y} + \alpha' \X \cdot {\Y}', \\ 
      \mbox{Positive} &: & \X \cdot \X \geq 0, \\ 
      \mbox{Nondegenerate} &: & \X \cdot \X = 0 \Leftrightarrow
      \X = 0,
    \end{array}
    $$
    where $\X$, ${\Y}$ and ${\Y}'$ are
    vectors, and $\alpha$ and $\alpha'$ are real numbers. If
    the Euclidean product is smooth, the pair $(\E,\cdot)$
    is an {\em Euclidean diffeological vector space}. 
  \end{article} %% Euclidean-diffeological-vector-spaces
  
  \begin{article}\artlabel{Hermitian diffeological vector spaces}
    \addcontentsline{toc}{section}{\small\hspace{10pt} Hermitian diffeological vector spaces}
    \label{Hermitian-diffeological-vector-spaces}
    Let ${\H}$ be a complex diffeological vector
    space, and let $(\X,{\Y}) \mapsto \X \cdot {\Y}$ be a
    {\em Hermitian product}, that is, a map from ${\H} \times
    {\H}$ to $\CC$ which is:
    $$
    \begin{array}{rcl}
      \mbox{Sesquilinear} &: & \X \cdot {\Y} = ({\Y}\cdot \X)^*, \\
      \mbox{Bilinear} &: & \X \cdot(\alpha {\Y} + \alpha' {\Y}') =
      \alpha  \X \cdot {\Y} + \alpha' \X \cdot {\Y}', \\ 
      \mbox{Positive} &: & \X \cdot \X \geq 0, \\ 
      \mbox{Nondegenerate} &: & \X \cdot \X = 0 \Leftrightarrow
      \X = 0, 
    \end{array}
    $$
    where $\X$, $\Y$ and $\Y'$ are
    vectors, $\alpha$, $\alpha'$ are complex numbers, and the
    star $*$ denotes the complex conjugation. The Hermitian
    product is also denoted sometimes by 
    $\hermprod{\X}{\Y} =  \X \cdot \Y$. If the  Hermitian product is
    smooth, then $(\H,\cdot)$ is a {\em Hermitian diffeological vector space}.
  \end{article} %% Hermitian-diffeological-vector-spaces
  
  \begin{article}\artlabel{The fine standard Hilbert space}
    \addcontentsline{toc}{section}{\small\hspace{10pt} The fine standard
    Hilbert space}
    \label{The-fine-standard-Hilbert-space} 
    The purpose of this paragraph is to introduce the
    fine diffeology on the standard Hilbert space, and some
    related constructions which will be used further for the
    study of a few diffeological infinite dimensional
    manifolds.
    Let $\cH_\RR$ be the real vector space of square-summable
    real sequences, 
    $$ 
    \cH_\RR = \bigg\{ \X = (\X_k)_{k=1}^\infty \ \bigg\vert\
    \X_k \in \RR, \ k = 1, \ldots, \infty, \mbox{ and } \sum_{k =
    1}^\infty \X_k^2 < \infty \bigg\}.
    $$ 
    The space of real numbers is naturally
    equipped with the smooth diffeology, and the space
    $\cH_\RR$ is equipped with the fine diffeology
    \art{The-fine-diffeology-of-vector-spaces}. The usual
    Euclidean product is defined on $\cH_\RR$ by
    $$
    \X \cdot \X' = \sum_{k = 1}^\infty \X_k \X'_k\,,
    \qmbox{for all} \X, \X' \in \cH_\RR.
    $$
    Hence, the pair $(\cH_\RR, \cdot)$ is a fine
    Euclidean diffeological vector space (see 
    \exref{Fine-Hermitian-vector-spaces}). We shall
    denote as usual, by $\Vert\cdot\Vert$, the norm
    associated with the Euclidean product, that is,
    $$
    \Vert \X \Vert = \sqrt{\X \cdot \X},
    \qmbox{for all} \X \in \cH_\RR. 
    $$
    We shall denote by $\pr_k$ the
    $k$-th projection from $\cH_\RR$ to $\RR$,
    $$ 
    \pr_k(\X) = \X_k \in  \RR,\ k = 1, \ldots, \infty,
    \qmbox{for all} \X = (\X_k)_{k = 1}^\infty \in \cH_\RR.
    $$ 
    Since $\RR$ and $\cH_\RR$ are  fine vector spaces and
    $\pr_k$ is linear, $\pr_k$ is smooth
    \art{Linear-maps-and-fine-diffeology}. We shall denote by
    $\ee_k$ the only vector of $\cH_\RR$ defined by
    $$
    \pr_k(\ee_k) = 1 \quad\mbox{and}\quad \pr_j(\ee_k) = 0
    \quad \mbox{if} \quad j\neq k. $$   We shall call the family
    $\set{\ee_k}_{k=1}^\infty$ the {\em canonical basis} of
    $\cH_\RR$. Also note that $\pr_k = \bar \ee_k : \X
    \mapsto \ee_k \cdot \X$.
    
    Now, what has been said for $\RR$ can be transposed
    to $\CC$. Let $\cH_\CC$ be the set of square-summable
    complex sequences,
    $$ 
    \cH_\CC = \bigg\{ \Z = (\Z_k)_{k=1}^\infty \ \bigg\vert\
    \Z_k \in \CC, k = 1, \ldots, \infty, \mbox{ and } \sum_{k =
    1}^\infty \Z_k^* \Z_k < \infty \bigg\},
    $$ 
    where $z^*$ denotes the complex conjugate of the complex
    number $z$. The space of complex numbers $\CC$ is
    naturally equipped with the standard diffeology, and the
    space $\cH_\CC$ is equipped with the fine diffeology
    \art{The-fine-diffeology-of-vector-spaces}. The usual
    sesqui-linear product is defined on $\cH_\CC$ by
    $$
    \Z \cdot \Z' = \sum_{k = 1}^\infty \Z^*_k \Z'_k,
    \qmbox{for all} \Z, \Z' \in \cH_\CC. 
    $$ 
    The sesqui-linear map $(\Z,\Z') \mapsto \Z \cdot \Z'$ is a
    Hermitian product. Thus, the pair $(\cH, \cdot)$ is a fine
    Hermitian diffeological vector space over $\CC$ (see 
    \exref{Fine-Hermitian-vector-spaces}). The norm
    associated with the Hermitian product is defined by
    $$
    \Vert \Z \Vert = \sqrt{\Z \cdot \Z},
    \qmbox{for all} \Z \in \cH. 
    $$ 
    We preserve the notation $\pr_k$ for the
    $k$-th projection from $\cH_\CC$ to $\CC$,
    $$ 
    \pr_k(\Z) = \Z_k \in  \CC,\ k = 1, \ldots, \infty,
    \qmbox{for all} \Z = (\Z_k)_{k = 1}^\infty \in \cH.
    $$ 
    We also preserve the notation 
    $\ee_k$ for the only vector of $\cH_\CC$ defined by 
    $\pr_k(\ee_k) = 1$ and $\pr_j(\ee_k) = 0$
    if $j\neq k$. Note that $\pr_k(\Z) = \ee_k \cdot \Z$, 
    but $\Z \cdot \ee_k = \Z_k^*$. These definitions and constructions 
    satisfy the following.

    \alinea{1.} The diffeological vector space $\cH_\CC$, regarded
    as a $\RR$ vector space, is still a fine diffeological
    vector space.
    
    \alinea{2.} The map $\fold$ from $\cH_\RR$ to $\cH_\CC$, defined
    by
    $$
    \fold : (\X_k)_{k = 1}^\infty \mapsto (\Z_k)_{k =
    1}^\infty, \ \mbox{with} \ \Z_k = \X_{2k-1} + i  \X_{2k}, \ k = 1, \ldots, \infty, 
    $$
    is a smooth $\RR$-isomorphism, where $\cH_\CC$ is regarded
    as a $\RR$-vector space. Its inverse $\unfold$ is given by 
    $$ \renewcommand{\arraystretch}{1.5} 
    \unfold: (\Z_k)_{k = 1}^\infty \mapsto (\X_k)_{k = 1}^\infty\,, 
    \quad \mbox{with} \quad
    \left\{
    \begin{array}{lll}
      \X_{2k-1} &=& \Re(\Z_k) \\
      \X_{2k} &= &\Im(\Z_k)
    \end{array}\right.,
    $$ 
    where $\Re$ and $\Im$ denote the real and
    imaginary parts. Moreover, the map $\fold$ is an isometry, that is,
    $$ 
    \norm{\fold(\X)}^2 = \norm{\X}^2, \qmbox{for all} \X \in \cH_\RR.
    $$
    
    \alinea{3.} The natural injection $j : \cH_\RR
    \to \cH_\CC$, defined by the inclusion  $\X_k \mapsto
    \Z_k = \X_k + i \times 0$ of $\RR$ into $\CC$, is an
    embedding.
  \end{article} %% The-fine-standard-Hilbert-space
  
  \begin{proof}
  1. Let $\P : \U \to \cH_\CC$ be a plot of the fine diffeology, that is,
    $\P(r) =_{\rm loc} \sum_{\alpha \in \A} 
    \lambda_\alpha(r) \Z^\alpha$,
    where $\A$ is a finite set of indices, the $\Z^\alpha$
    belong to $\cH_\CC$ and the $\lambda_\alpha$ are smooth
    parametrizations in $\CC$. Defining $a_\alpha$ and
    $b_\alpha$ by $\lambda_\alpha(r) = a_\alpha(r) +
    i b_\alpha(r)$, we get $\P(r) =_{\rm loc} \sum_{\alpha \in \A} a_\alpha(r)
    \Z^\alpha + \sum_{\alpha \in \A} b_\alpha(r) 
    \times (i \Z^\alpha)$.
    Since the $i\Z^\alpha$ are still in $\cH_\CC$, the plot
    $\P$ writes locally $\P(r) =_{\rm loc} \sum_{\beta \in \B} 
    \phi_\beta(r) \Z'^\beta$,
    where $\B$ is a finite family of indices, the $\Z'^\beta$
    belong to $\cH_\CC$ and the $\phi_\beta$ are smooth
    parametrizations in $\RR$.  Therefore, as a real vector
    space, $\cH_\CC$ is still a fine diffeological vector
    space.
    
    \alinea{2.} Since $\cH_\CC$ is fine, regarded as a real vector
    space, any linear isomorphism is a smooth isomorphism
    \art{The-fine-linear-category}. Then, it is a simple
    verification to check that $\fold$ and $\unfold$ are
    inverse linear isomorphisms one of each other, and are
    isometries.
    
    \alinea{3.} Let us check that the injection $j :
    \cH_\RR \to \cH_\CC$ is an embedding. First of all, since 
    $j$ is a linear injection it is an induction
    \art{Injections-of-fine-diffeological-vector-spaces}.
    Now, let $\Re : \cH_\CC \to \cH_\RR$ be the map $\Z
    \mapsto \X = \Re(\Z)$ defined by $\Re(\Z)_k = \Re(\Z_k)$.
    Note that the subspace $\cH_\RR$ is just the pointwise
    $\Re$-fixed points set: $\cH_\RR = \set{\Z \in \cH_\CC
    \mid \Re(\Z) = \Z}$. Now, $\Re$ is $\RR$-linear, thus
    smooth \art{Linear-maps-and-fine-diffeology}, thus
    D-continuous \art{Smooth-maps-are-D-continuous}.
    Moreover, $\Re$ is surjective over $\cH_\RR$, so for any
    D-open $\A \in \cH_\RR$, $\A' = \Re^{-1}(\A)$ is D-open
    and since $\Z' \in \cH_\RR$ means $\Re(\Z') = \Z'$, $\A'
    \cap \cH = \set{\Z' \in \cH_\RR \mid \Re(\Z') \in \A} = 
    \A$.
  \end{proof}
  
  %%%%%%%%%%%%%%%%%%%%%%%%%%%%%%%%%%%%%%%%%%%%%%%%%%%%%%%%%%
  %
  %   Exercises  
  %
  %%%%%%%%%%%%%%%%%%%%%%%%%%%%%%%%%%%%%%%%%%%%%%%%%%%%%%%%%%
  
  \Exercises
  
  \begin{exercise}[Fine Hermitian vector spaces]
    \label{Fine-Hermitian-vector-spaces}
    Show that every real (or complex) fine
    diffeological vector space $\E$ 
    \art{The-fine-diffeology-of-vector-spaces}, equipped with any
    Euclidean (or Hermitian) product is an Euclidean
    (or Hermitian) diffeological vector space.
  \end{exercise} %% Fine-Hermitian-vector-spaces
  
  \begin{exercise}[Finite dimensional Hermitian spaces]
    \label{Finite-dimensional-Hermitian-spaces} 
    Show that there exists one and only one
    structure of Euclidean (or Hermitian) diffeological
    vector space of finite dimension, that is, the fine
    structure. Compare with the remark of
    \art{Standard-vector-spaces}.
  \end{exercise} %% Finite-dimensional-Hermitian-spaces
  
  \begin{exercise}[Topology of the norm and D-topology]
    \label{Topology-of-the-norm-and-D-topology}
    The topology of the norm of a Hermitian diffeological
    space $\E$ does not coincide necessarily with the
    D-topology. Show that the topology of the norm is
    finer than the D-topology. 
  \end{exercise} %% Topology-of-the-norm-and-D-topology
  
  \begin{exercise}[Banach's diffeology]
    \label{Banach-s-diffeology}
    Let $\E$ be a Banach space, that is, a complete vector
    space for a given norm $\norm{\cdot}$. Let us recall that
    a continuous map $\phi: \Omega \to \E$, defined on 
    an $n$-domain, is said to be of
    {\em class $\cC^1$} if there exists a continuous linear
    map, denoted by $\D(f): \Omega \to \Lin(\RR^n,\E)$, such
    that,  
    $$ 
    \lim_{t \mathop{\rightarrow} 0} {f(x+tu) - f(x) \over t} =  \D(f)(x)(u),
    \quad \mbox{for all } u \in \RR^n. 
    $$ 
    The map $f$ is said to be of class $\cC^k$, $k>1$, if $f$ is of
    class $\cC^1$ and if $\D(f)$ is of class $\cC^{k-1}$.
    The map $f$ is said to be of class $\Cinfty$ if it is of
    class $\cC^k$ for all $k \in \NN$, see \cite{Die70a}. 
    Check that the set of parametrizations in $\E$ which are
    of class $\Cinfty$ for the norm is a diffeology. We shall
    call it the {\em Banach diffeology} of $(\E, \norm{\cdot})$.

    \Question{1)} Show that, for the Banach diffeology, $\E$ is a
    diffeological vector space. 
    
    \Question{2)} Show that the category of Banach spaces is a 
    full subcategory of diffeological vector spaces. 
    Use the Boman theorem:
    {\em for any two Banach spaces $\E$ and $\F$, a  map $f : \E \to \F$ 
    is Banach-smooth if and only if it takes smooth curves in $\E$ 
    to smooth curves in $\F$} \cite{Bom67}.
  \end{exercise} %% Banach-s-diffeology
  
  \begin{exercise}[$\cH_\CC$ is isomorphic to $\cH_\RR \times \cH_\RR$] 
    \label{HC-is-isomorphic-to-HR-X-HR} 
    With the notations of
    \art{The-fine-standard-Hilbert-space}, check that the map
    $\psi : \cH_\RR \times \cH_\RR \mapsto \cH_\CC$ defined by
    $\psi(\X,\Y) = \X +i\Y$ is a smooth isometry, that is, a
    smooth isomorphism such that $\norm{\psi(\X,\Y)}^2 =
    \norm{\X}^2 + \norm{\Y}^2$. 
  \end{exercise} %% HC-is-isomorphic-to-HR-X-HR
  