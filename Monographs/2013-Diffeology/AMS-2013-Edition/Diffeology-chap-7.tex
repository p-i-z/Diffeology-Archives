  %%%%%%%%%%%%%%%%%%%%%%%%%%%%%%%%%%%%%%%%%%%%%%%%%%%%%%%%%%
  %% 
  %% Diffeological groups MARK: -
  %% 
  %%%%%%%%%%%%%%%%%%%%%%%%%%%%%%%%%%%%%%%%%%%%%%%%%%%%%%%%%%
  
  \chapter{Diffeological Groups}
  
  \label{Chapter-Diffeological-Groups}
  \newcommand{\ChapterDG}{Diffeological Groups}
  
  \begin{chaphead}
    The notion of diffeological group is the natural
    adaptation of the classical notion of Lie
    group to diffeology. 
    Diffeological groups are groups equipped
    with a diffeology compatible with the group multiplication
    and the inversion
    \art{Diffeological-groups}. This definition 
    has been introduced originally
    by J.-M. Souriau \cite{Sou80} as \guillemots{groupes diff\'erentiels}. 
    They appeared with their own axiomatics, and the main
    examples were the groups of
    diffeomorphisms of manifolds. 
    Later they found their place in the general diffeological framework.
    However, the groups of diffeomorphisms of diffeological spaces, equipped with
    their functional diffeology 
    \art{Functional-diffeology-on-groups-of-diffeomorphisms},
    are still the main examples of diffeological groups since 
    every diffeological group is a subgroup of its group of
    diffeomorphisms
    \art{Left-and-right-multiplications-are-embeddings}.
    The introduction of diffeological groups leads naturally 
    to the notion of smooth actions on diffeological spaces as smooth 
    homomorphisms \art{Smooth-action-of-a-diffeological-group}.
    
    After the basic definitions, we focus our attention, in this chapter, 
    on the momenta of diffeological groups, that is, the space of their
    left-invariant $1$-forms
    \art{Momenta-of-a-diffeological-group}. The classical
    way for studying Lie group uses the notion of Lie
    algebra. But, because diffeology is adapted better to
    covariant objects than to contravariant ones, the usual
    point of view is skipped. Lie algebra could be
    possibly defined afterwards, by duality with the space of
    momenta, or as the space of $1$-parameter subgroups, or whatever else 
    if necessary.
    
    The notion of coadjoint orbits persists in the
    diffeological framework and plays an important role. 
    They are introduced directly as
    the orbits of the group acting, by pushforward
    of the adjoint action, on its space of momenta. Because
    there is a natural equivalence between left-invariant and
    right-invariant $1$-forms of diffeological groups 
    \art{Equivalence-between-right-and-left-momenta}, which
    intertwines the coadjoint action, the choice of right
    or left invariant $1$-forms is not relevant. The left momenta
    come naturally in the picture with the study of 
    invariant differential forms, especially in symplectic diffeology
    \art{Definition-of-the-moment-maps}.
  \end{chaphead}
  
  %%%%%%%%%%%%%%%%%%%%%%%%%%%%%%%%%%%%%%%%%%%%%%%%%%%%%%%%%%
  \section*{Basics of Diffeological Groups}
  
  \begin{sechead}
    After their introduction by J.-M. Souriau
    in 1979 as {\em groupes diff\'erentiels} \cite{Sou80}, 
    diffeological groups, and their relations with
    diffeological spaces, have been a little bit developed, in particular in
    \cite{Sou84} \cite{Don84} \cite{DI85} \cite{Igl85}, but they remain
    overall {\em terra incognita\/}. For example, is a finite dimensional
    diffeological group always the quotient of a Lie group? 
  \end{sechead}
  
  \begin{article}\artlabel{Diffeological groups}
    \addcontentsline{toc}{section}{\small\hspace{10pt} Diffeological groups}
    \label{Diffeological-groups}
    A {\em diffeological group} is a group $\G$ equipped with
    a compatible diffeology, that is, such that the
    multiplication and the inversion are smooth:
    $$
    \mul = [(g,g') \mapsto gg'] \in \Cinfty(\G \times \G, \G)
    \mbox{ and } \inv = [g \mapsto g^{-1}] \in \Cinfty(\G,\G).
    $$
    A diffeology $\cD$ on a group $\G$ which satisfies these
    two conditions will be called a {\em group diffeology}.
    Let $\G$ and $\G'$ be two diffeological groups, 
    $\DHom(\G,\G')$ will denote the space of smooth
    homomorphisms from $\G$ to $\G'$, and by
    $\DIsom(\G,\G')$ the space of smooth
    isomorphisms from $\G$ to $\G'$.
    $$%\renewcommand{\arraystretch}{1.2}
    \left\{\begin{array}{rcl}
      \DHom(\G,\G') &=& \Hom(\G,\G') \cap \Cinfty(\G,\G'), \\
      \vspace{-7pt} \\
      \DIsom(\G,\G') &= &\Isom(\G,\G') \cap \Diff(\G,\G').
    \end{array}\right.
    $$
    Diffeological groups, together with smooth
    homomorphisms, form a subcategory of the category
    $\Diffeology$ which may be denoted by $\DGroups$.
    
    Traditional {\em Lie group} are diffeological groups $\G$ which are manifolds, modeled
    on a finite dimensional real vector spaces \art{Manifolds-as-diffeologies}. A large 
    literature exists on Lie groups, which occupy a big
    part of modern mathematics. A diffeological group which
    is a diffeological manifold modeled on a diffeological
    vector space will can be regarded as {\em diffeological Lie
    group}. Very few has been written on diffeological Lie
    groups \cite{Les03}.
  \end{article} %% Diffeological-groups
  
  \begin{article}\artlabel{Subgroups of diffeological groups}
    \addcontentsline{toc}{section}{\small\hspace{10pt} Subgroups of diffeological groups}
    \label{Subgroups-of-diffeological-groups}
    Every subgroup $\H$ of a diffeological group $\G$ is
    canonically a diffeological group for the subset
    diffeology \art{Subspaces-and-subset-diffeology}. When
    we shall refer in the following to a subgroup of a
    diffeological group, it will be always equipped with the
    subset diffeology.   
  \end{article} %% Subgroups-of-diffeological-groups
  
  \begin{proof}
    Let $j: \H \to \G$ be the inclusion. The inclusion $j$ is
    smooth, it is even an induction. Let $\mul_\H$ and
    $\inv_\H$ be the multiplication and the inversion of
    $\H$. Since $\mul_\H = \mul \circ (j \times j)$ and
    $\inv_\H = \inv \circ j$ are composites of smooth maps,
    they are smooth. \end{proof}
  
  \begin{article}\artlabel{Quotients and extensions of diffeological groups}
    \addcontentsline{toc}{section}{\small\hspace{10pt} Quotients of diffeological groups}
    \label{Quotients-and-extensions-of-diffeological-groups}
    Let $\G$ be a diffeological group. Let $\N \subset \G$ be a
    normal subgroup, that is, $g \N g^{-1} = \N$ for all $g \in \G$. 
    Considering indifferently the right or the left coset,
    \begin{itemize}
      \item[($\clubsuit$)] the quotient group $\H = \G/\N$ is
      canonically a diffeological group for the quotient
      diffeology \art{Quotient-and-quotient-diffeology}.
    \end{itemize}
    We represent this situation by the short exact sequence of 
    homomorphisms
    $$
      \id_\N \xrfl{}{15mm} 
      \N \xrfl{j}{15mm} 
      \G \xrfl{\pi}{15mm} 
      \H = \G/\N \xrfl{}{15mm} 
      \id_\H
      $$
    where $j : \N \to \G$ is the inclusion and $\pi : \G \to \H$
    is the projection. By construction, $j$ is an induction and
    $\pi$ is a subduction. Actually, $\pi$ is more than a subduction,
    it is a principal fibration, 
    see \art{Fibrations-of-groups-by-subgroups}.
    %
    Conversely, let $\H$ and $\N$ be just two groups. 
    An {\em extension} of $\H$ by $\N$ is any group $\G$
    with two morphisms $j : \N \to \G$ and $\pi : \G \to \H$ 
    satisfying the diagram above, see for example \cite{Kir76} 
    for a full discussion. Now, if $\N$ and $\H$ are diffeological
    groups, a {\em diffeological} or {\em smooth extension} of $\H$ by $\N$ 
    is an extension such that $j$ is an induction and $\pi$ is 
    a subduction.
    %
    As a special case of diffeological fibrations, 
    {\em diffeological group extensions} form a subcategory of the category of 
    principal fiber bundles
    \art{Category-of-principal-fiber-bundles}. 
    A morphism of this category  
    is a morphism of the category of principal diffeological fiber bundles
    which is also a group morphism (or conversely).
    %
    The simplest case of such extensions is the direct product
    $\G = \H \times \N$, with $(h,n)\cdot (h',n') = (hh',nn')$, 
    $j : n \mapsto (\id_\H,n)$ and $\pi=\pr_1 : (h,n) \mapsto h$. 
    Such an extension is said to be {\em the trivial extension} of $\H$ 
    by $\N$. 
    An extension which is equivalent to a direct product 
    is said to be {\em trivial}. 
    When it comes to this question of trivial extension,
    the following two different points must be inspected. 
    
    \alinea{a)}~The projection 
    $\pi : \G \to \H = \G/\N$, as a principal diffeological fiber bundle,
    must be trivial,
    which is equivalent to the existence of a smooth section 
    $\sigma : \H \to \G$ 
    \xart{Category-of-principal-fiber-bundles}{note 2}. 
    
    \alinea{b)}~This equivalence must be an equivalence of 
    extension, that is, $\sigma$ 
    is a group homomorphism from $\H$ to $\G$, and $\sigma(\H)$ is 
    a subgroup of the centralizer of $\N$, that is,
    $n \sigma(h) = \sigma(h) n$ for all $(h,n) \in \H \times \N$.
    
    \Note~There exist smooth extensions of $\H$
    by $\N$, on the product $\H \times \N$, which 
    are not trivial as group extensions. 
    The first example which comes in mind is the Heisenberg group, 
    in the special case of central extensions. 
    These extensions are given by smooth
    cocycles, the condition a) is fulfilled but not the condition b). 
    It is why it is worth insisting on these two aspects.
  \end{article} %% Quotients-and-extensions-of-diffeological-groups
  
  \begin{proof}
    Let
    $\mul_{\G/\N}$ and $\inv_{\G/\N}$ be the multiplication
    and the inversion of $\G/\N$. Since $\mul_{\G/\N} \circ
    (\pi \times \pi) = \pi \circ \mul$ and since
    $\inv_{\G/\N} \circ \pi = \pi \circ \inv$, and $\pi$ and
    $(\pi \times \pi)$ are subductions, $\mul_{\G/\N}$ and
    $\inv_{\G/\N}$ are smooth
    \art{Smooth-maps-from-quotients}. 
    The other parts of the proposition are simple 
    adaptations of classical results.
  \end{proof}
  
  \begin{article}\artlabel{Smooth action of a diffeological group} 
    \addcontentsline{toc}{section}{\small\hspace{10pt} Smooth action of a diffeological group} 
    \label{Smooth-action-of-a-diffeological-group} 
    Let $\X$ be a diffeological space and $\G$ be a
    diffeological group. A {\em smooth action} of $\G$ on
    $\X$ is any smooth homomorphism $\rho: \G \to
    \Diff(\X)$, where $\Diff(\X)$ is equipped with its
    functional diffeology of group of diffeomorphisms
    \art{Functional-diffeology-on-groups-of-diffeomorphisms}.
    To simplify the notations, and where there is no risk of confusion, 
    we shall sometimes denote $g_\X$ for $\rho(g)$.
    
    \Note{1} A homomorphism $\rho : \G \to
    \Diff(\X)$ is smooth if and only if the evaluation map
    $\ev_\rho : (x,g) \mapsto \rho(g)(x)$ defined on $\X
    \times \G$ into $\X$ is smooth.
    
    \Note{2} Let us recall that the kernel of a group
    action is always a normal subgroup, and that the
    action is said to be {\em effective} if its kernel is
    trivial, $\rho$ is then a monomorphism. 
    For every smooth action $\rho$ of $\G$
    on $\X$, there exists a natural effective smooth action of
    $\K = \G/\ker(\rho)$ on $\X$, where $\ker(\rho)$ 
    is equipped with the subset diffeology and $\K$ 
    with the quotient diffeology.
    
    \Note{3} In the paper \cite{IZK10} we have shown that any 
    smooth action $\rho : \G \to \Diff(\M)$, where $\G$ is a Lie group and 
    $\M$ is a manifold ---~both Hausdorff and second countable~--- is
    actually an induction, that is, a diffeomorphism onto its image.
    In other words, a smooth action, in the category
    of Hausdorff and second countable finite dimensional 
    manifolds, is equivalent to giving a subgroup of $\Diff(\M)$ which,
    equipped with the induced functional diffeology, is a manifold. 
  \end{article} %% Smooth-action-of-a-diffeological-group
  
  \begin{proof}
    The second note is a direct application of
    \art{Quotients-and-extensions-of-diffeological-groups}. 
    So, let us prove
    the first note. Let us assume that $\rho$ is smooth. The
    map $\ev_\rho$ splits into $(x,g) \mapsto
    (x,\rho(g)) \mapsto \rho(g)(x)$ from $\X \times \G$ to
    $\X \times \Diff(\G)$ to $\X$. But, since $\rho$ is
    smooth, $\Ev_\rho : (x,g) \mapsto (x,\rho(g))$ is smooth,
    and the second factor $(x, \phi) \mapsto \phi(x)$ from
    $\X \times \Diff(\G)$ to $\X$ is the evaluation map,
    which is smooth by the very definition of the functional
    diffeology \art{Functional-diffeologies}. Therefore, the map
    $\ev_\rho$ is smooth.
    Conversely, let us assume that the map $\ev_\rho$ is
    smooth. Now, $\rho$ is smooth if and only if for every
    plot $\P': \U' \to \G$ the parametrization $r
    \mapsto \rho(\P'(r'))$ is a plot of $\Diff(\X)$,
    that is, if and only if for every plot $\P'' :
    \U'' \to \X$ the parametrizations $(r',r'') \mapsto
    \rho(\P'(r'))(\P''(r''))$ and $(r',r'') \mapsto
    \rho(\P'(r'))^{-1}(\P''(r''))$ are plots of $\X$.
    By hypothesis, for every plot $\P :
    \U \to \X \times \G$ the parametrization $r
    \mapsto \rho(\P(r))(\Q(r))$ is a plot of $\X$. This,
    applied to $\U = \U' \times \U''$ and to the plot $\P :
    r = (r',r'') \mapsto \P(r) = (\P'(r'),\P''(r''))$, gives
    exactly that $(r',r'') \mapsto
    \rho(\P'(r'))(\P''(r''))$ is a plot of $\X$. Applied to
    the plot $\P : r = (r',r'') \mapsto \P(r) =
    (\P'(r')^{-1},\P''(r''))$  (because the
    inversion in $\G$ is smooth), we get that $(r',r'')
    \mapsto \rho(\P'(r')^{-1})(\P''(r'')) =
    \rho(\P'(r'))^{-1}(\P''(r''))$ is a plot of
    $\X$. Therefore, $\rho$ is smooth.
  \end{proof}
  
  \begin{article}\artlabel{Orbit map} 
    \addcontentsline{toc}{section}{\small\hspace{10pt} Orbit map} 
    \label{Orbit-map}
    Let $\X$ be a diffeological space. Let $\G$ be a
    diffeological group and $g \mapsto g_\X$ be a smooth
    action \art{Smooth-action-of-a-diffeological-group} of
    $\G$ on $\X$. For all points $x \in \X$, we denote by
    $\orb{x}$ the {\em orbit map} 
    $$
      \orb{x} : \G \to \X \qmbox{with} \orb{x}(g) =
      g_\X(x).
      $$
    As an immediate consequence of the definition of smooth actions, 
    the orbit maps are smooth maps from $\G$ to
    $\X$.
    
    \Note~Let $\M$ be a manifold and $\G$ be a Lie group, 
    both Hausdorff and second countable. Let $\rho : \G \to \Diff(\M)$
    be a smooth action and $m \in \M$. Then, the orbit map $\orb{m} : \G \to \M$
    is a strict map \art{Strict-maps-between-quotients-and-subspaces}
    \cite{IZK10}, that is, the projection of $\orb{m}$, from $\G/\G_m$
    to $\M$, where $\G_m$ is the stabilizer of $m$, is an induction.
    Consequently, every orbit of $\G$ in $\M$ 
    is a submanifold, that is, for all
    $m \in \M$, the orbit $\G \cdot m = \{\rho(g)(m) \mid g \in \G\}$,
    equipped with the subset diffeology, is a manifold 
    \art{Submanifolds-of-a-diffeological-space}. 
  \end{article} %% Orbit-map
  
  \begin{article}\artlabel{Left and right multiplications} 
    \addcontentsline{toc}{section}{\small\hspace{10pt} Left and right multiplications} 
    \label{Left-and-right-multiplications}
    Let $\G$ be a diffeological group, and let $g \in \G$.
    We denote by $\eL(g)$ the left multiplication by $g$,
    $$ 
    \eL(g) : k \mapsto gk, \  \eL(g) \in \Cinfty(\G). 
    $$
    Left multiplication is a
    smooth effective action of $\G$ onto itself, that is,
    $\eL \in \DHom(\G, \Diff(\G))$. 
    Then, we denote by $\eR(g)$, the right
    multiplication by $g \in \G$,
    $$
    \eR(g) : k \mapsto kg,  \ \eR(g) \in \Cinfty(\G). 
    $$ 
    The right multiplication is a smooth effective {\em anti-action}
    of $\G$ onto itself, that is, $\eR(gg') = \eR(g') \circ
    \eR(g)$. The map $\eR$ is a smooth anti-homomorphism
    from $\G$ to $\Diff(\G)$, which can also be expressed as
    $\eR \circ \inv \in \DHom(\G, \Diff(\G))$.
    We denote by $\Ad(g)$ the {\em adjoint action}
    of $\G$ onto itself,
    $$
    \Ad(g) : k \mapsto gkg^{-1},\ \Ad(g) \in \Cinfty(\G).
    $$ 
    The adjoint action is smooth, $\Ad \in
    \DHom(\G,\Diff(\G))$. Actually 
    $$\Ad(g) = \eL(g) \circ
    \eR(g^{-1}) = \eR(g^{-1}) \circ \eL(g).
    $$ 
    Note that the adjoint action is not necessarily effective. The
    kernel of the adjoint action is called the {\em center}
    of $\G$, denoted sometimes by $\eZ(\G)$. It is the
    subgroup of $\G$ whose elements commute with every element of $\G$.
  \end{article} %% Left-and-right-multiplications
  
  \begin{article}\artlabel{Left and right multiplications are embeddings}  
    \addcontentsline{toc}{section}{\small\hspace{10pt} Left and right multiplications are inductions} 
    \label{Left-and-right-multiplications-are-embeddings}
    Let $\G$ be a diffeological group. Left and right
    multiplications \art{Left-and-right-multiplications} are
    embeddings from $\G$ to $\Diff(\G)$, where $\Diff(\G)$
    is equipped with the functional diffeology
    \art{Functional-diffeologies}. Explictly, left and right
    multiplications are inductions
    such that the pullback of the D-topology of the functional diffeology
    of $\Diff(\G)$ coincides with the D-topology of $\G$ 
    \art{Embedded-subsets-of-a-diffeological-space}.
    
    \Note~In particular, this shows
    that every diffeological group is (equivalent to) a
    subgroup of a group of diffeomorphisms, for the
    functional diffeology.
  \end{article} %% Left-and-right-multiplications-are-embeddings
  
  \begin{proof}
    Let us prove this proposition for the left
    multiplication. And first, let us prove that $\eL$ is an induction.
     We know already that $\eL : \G \to
    \Diff(\G)$ is injective and smooth
    \art{Left-and-right-multiplications}. So, let $\P : \U
    \to \G$ be a parametrization such that $\eL \circ \P$
    is a plot of $\Diff(\G)$. Thus, for every plot $\Q : \V
    \to \G$, the parametrization $(r,s) \mapsto
    \L(\P(r))(\Q(s)) = \P(r) \Q(s)$ is a plot of $\G$. But
    since the multiplication and the inversion are smooth,
    the parametrization $(r,s) \mapsto (\P(r) \Q(s),
    \Q(s)^{-1}) \mapsto [\P(r) \Q(s)]\Q(s)^{-1} = \P(r)$ is
    a plot of $\G$, that is, $\P$ is a plot of $\G$.
    Therefore, $\eL$ is an induction.
    Next, let $\cO$ be D-open in $\G$, the subset 
    $$
     \Omega = \{ f \in \Diff(\G) \mid f(\id_\G) \in \cO \}
    $$
    is D-open. Indeed, let $\P$ be a plot of $\Diff(\G)$, $\P^{-1}(\Omega)$
    is the subset of $r \in \Dom(\P)$ such that $\P(r)(\id_\G) \in \cO$, 
    that is, $\P^{-1}(\Omega) = (\eR(\id_\G) \circ \P)^{-1}(\cO)$, where 
    $\eR(\id_\G)$ is the orbit map of the point $\id_\G$ for the left 
    multiplication \art{Orbit-map}. Since $\eR(\id_\G)$
    and $\P$ are smooth, thus D-continuous, and since $\cO$ is D-open, 
    $\P^{-1}(\Omega)$ is open and thus $\Omega$ is D-open in $\Diff(\G)$. 
    Now, $\eL^{-1}(\Omega)$ is the set of $g \in \G$ such that
    $\eL(g)(\id_\G) \in \cO$, that is, $\eL^{-1}(\Omega) = \cO$.
  \end{proof}
    
  \begin{article}\artlabel{Transitivity and homogeneity} 
    \addcontentsline{toc}{section}{\small\hspace{10pt} Transitivity and homogeneity}
    \label{Transitivity-and-homogeneity}
    Let $\X$ be a diffeological space and $\G$ be a
    diffeological group. Let $\rho$ be smooth action of
    $\G$ on $\X$, that is, a homomorphism from $\G$ to $\Diff(\X)$. 
    The action of $\G$ is said to be {\em transitive} on $\X$
    if the {\em evaluation map}
    $$
    \Ev_\rho : \X \times \G \to \X \times \X
    \qmbox{defined by} \Ev_\rho(x,g) = (x,\rho(g)(x)) 
    $$
    is surjective, that is, if for any $x, x' \in \X$ there exists
    $g \in \G$ such that $x' = \rho(g)(x)$. 
    We say that $\X$ is {\em homogeneous}
    for the action $\rho$ of $\G$ if the evaluation map $\Ev_\rho$ is
    a subduction. Now, the space $\X$ is a homogeneous space
    of $\G$ (for $\rho$) if and only if the action $\rho$ is smooth 
    and for some $\xo \in \X$ the {\em orbit map}
    $$
    \orb{\xo} : \G \to \X \qmbox{defined by} \orb{\xo}(g) = \rho(g)(\xo), 
    $$
    is a subduction (if it is true for some point, then it
    will be true for every point). Moreover, the subduction
    $\orb{\xo}$ is a local subduction \art{Local-subductions}.
    In this case we also say that $\G$ {\em generates} $\X$, or is
    a {\em generating group} for $\X$.
  \end{article} %% Transitivity-and-homogeneity
  
  \begin{proof}
    First of all, if $\Ev_\rho$ is a subduction, then $\rho$ is smooth, since
    $\Ev_\rho$ is smooth and $\ev_\rho = \pr_2 \circ \Ev_\rho$ 
    \xart{Smooth-action-of-a-diffeological-group}{note 1}.
    Then, $\orb{\xo}$ is the restriction of $\Ev_\rho$ to $\set{\xo} \times \G$
    with values $\set{\xo} \times \X$. Thus, every plot $r \mapsto (\xo, \Q(r))$ 
    lifts locally into a plot $r \mapsto (\xo(r) = \xo,\gamma(r))$ 
    such that $\rho(\gamma(r)(\xo)) = \Q(r)$. 
    Hence, $\orb{\xo}$ is a subduction.
    Now, let us assume that $\rho$ is smooth and $\orb{\xo}$ is a subduction.
    First of all, if $\ev_\rho$ is smooth so is $\Ev_\rho$. 
    Next, let $\P : r \mapsto (\P'(r),\P''(r))$ be a plot of $\X
    \times \X$, defined on a domain $\U$. By hypothesis, for
    every $r_0$ in $\U$, there exist two open neighborhoods $\V'$
    and $\V''$ of $r_0$, there exist two plots $\gamma' :
    \V' \to \G$ and $\gamma'' : \V'' \to \G$ such that
    $\P'(r) = \rho(\gamma'(r))(\xo)$ and $\P''(r) =
    \rho(\gamma''(r))(\xo)$. Thus, on $\V = \V' \cap \V''$,
    we have $\P''(r) =
    \rho(\gamma''(r))[\rho(\gamma(r')^{-1})(\P'(r)]$, that is, 
    $\P''(r) = \rho(\gamma(r))(\P'(r))$, with $\gamma(r)
    = \gamma''(r) \gamma'(r)^{-1}$. Hence, there exists a
    plot $\Q : r \mapsto (\P'(r),\gamma(r))$ of $\X \times
    \G$, defined on $\V$, such that $\P \restriction \V =
    \Ev_\rho \circ \Q$. Therefore, $\Ev_\rho$ is a
    subduction. Then, since $\G$ is transitive, if this is true for
    one point $\xo$, by composition with some element of $\G$, 
    it will be true for every point.
    Finally, pick a plot $\P : \U \to \X$, let $r_0 \in
    \U$, let $g_0 \in \G$ such that $\orb{\xo}(g_0) =
    \P(r_0)$, that is, $\rho(g_0)(\xo) = \P(r_0)$. The plot
    $\P$ lifts locally, on an open neighborhood of $r_0$, into a plot
    $\Q$ of $\G$, $\orb{\xo} \circ \Q =_\loc \P$. Let $g'_0
    = \Q(r_0)$, the parametrization $r \mapsto g_0
    (g'_0)^{-1} \Q(r)$ is a local lift of $\P$ passing
    through $g_0$ at $r_0$. Therefore, $\orb{\xo}$ is a
    local subduction. \end{proof}
  
  \begin{article}\artlabel{Connected homogeneous spaces}
    \addcontentsline{toc}{section}{\small\hspace{10pt} Connected homogeneous spaces} 
    \label{Connected-homogeneous-spaces}
    Let $\X$ be a connected diffeological space. If $\X$ is
    homogeneous for an action $\rho$ of a diffeological group $\G$, 
    then it is homogeneous for its identity component 
    $\G^\circ = \comp(\id_\G)$.
  \end{article} %% Connected-homogeneous-spaces
  
  \begin{proof}
    Let $x_0,x_1 \in \X$ and $t
    \mapsto x_t$ be a smooth path connecting $x_0$ to $x_1$.
    Since $\orb{x_0} : g \mapsto \rho(g)(x_0)$ is a
    subduction from $\G$ onto $\X$, there exist a family of
    intervals $\{\,]a_i,b_i[\,\}_{i \in \eI}$ covering $[0,1]$
    and a family of plots $t \mapsto g_i(t)$ in $\G$, $t \in
    \eI$, such that $\rho(g_i(t))(x_0) = x_t$. Since $[0,1]$
    is compact, we can find a family $\{\,]a'_n,b'_n[\,\}_{n =
    1}^\N$ such that each $]a'_n,b'_n[$ is contained in some
    $]a_i,b_i[$ (chosen once and for all), and such that
    $]a'_n,b'_n[\, \cap\, ]a'_m,b'_m[ \neq \varnothing$, with
    $1 \leq n \leq \N-1$ and  $n > m$, if and
    only if $m = n+1$. We denote by $g_n$ the restriction of
    $g_i$  to  $]a'_n,b'_n[$. Let $t_0 = 0$, $t_\N = 1$, and
    for each $n$ such that $1 \leq n \leq \N-1$, we chose $t_n
    \in ]a'_n,b'_n[ \, \cap \, ]a'_{n+1},b'_{n+1}[$. So, we have
    $\rho(g_n(t_n))(x_0) = \rho(g_{n+1}(t_n))(x_0) =
    x_{t_n}$, that is, $g_{n+1}(t_n)^{-1} g_n(t_n) \in
    \Stab_\G(x_0)$. Now, let us define $ g'_n(t) = g_n(t)
    k_n$ with $k_n = g_n(t_{n-1})^{-1}g_{n-1}(t_{n-1})
    k_{n-1}$, for $2 \leq n
    \leq \N$ and $k_1 = g_1(t_0)^{-1}$. So, all the
    $k_n$ belong to $\Stab_\G(x_0)$, and therefore
    $\rho(g'_n(t))(x_0) = \rho(g_n(t))(x_0) = x_t$. Moreover,
    $g'_n(t_{n-1}) = g_n(t_{n-1})k_n = g_n(t_{n-1})
    g_n(t_{n-1})^{-1}g_{n-1}(t_{n-1}) k_{n-1} =
    g_{n-1}(t_{n-1}) k_{n-1} = g'_{n-1}(t_{n-1})$. Thus, the
    family $\{g'_i\}_{i=1}^\N$ connects $\id_\G = g'_1(0)$ to
    a diffeomorphism $g = g'_\N(1)$ which satisfies
    $\rho(g)(x_0) = x_1$. Therefore, the connected component
    $\G^\circ$ is transitive on $\X$, and since the
    projection $\orb{x_0}$ is a local subduction, its
    restriction to $\G^\circ$ is a local subduction too.
  \end{proof}
  
  \begin{article}\artlabel{Covering diffeological groups}
    \addcontentsline{toc}{section}{\small\hspace{10pt} Covering diffeological groups}  
    \label{Covering-diffeological-groups}
    Let $\pr : \hat \G \to \G$ be a subduction, where $\hat \G$ and
    $\G$ are two diffeological groups. We say that $\pr$ 
    is a {\em group covering} if $\pr$ is a homomorphism and the 
    kernel $\K = \pr^{-1}(\id_\G)$ is discrete. Let us recall
    that {\em discrete} means that the plots ---~here the
    plots for the subset diffeology~--- are locally
    constant \art{Discrete-diffeology}. Let $\G$ be a connected diffeological group,
    its universal covering $\tilde \G$
    \art{The-universal-covering} has a natural structure of
    diffeological group such that the subduction $\pi :
    \tilde \G \to \G$ is a homomorphism. The first
    homotopy group $\pi_1(\G) = \ker(\pi)$
    \art{The-Poincare-groupoid-and-fundamental-group} is a discrete normal 
    subgroup of $\tilde \G$, and $\pi$ is a group
    covering. Any connected covering $\pr : \hat \G
    \to \G$ \art{Fundamental-group-acting-on-coverings} is the
    quotient of the universal covering by a subgroup $\K$ of
    $\pi_1(\G)$. If the subgroup $\K$ is normal then $\pr$ is also
    a group covering.
  \end{article} %% Covering-diffeological-groups
  
  \begin{proof}
    This property has been established originally in \cite{Sou84}
    \cite{Don84}, inside the strict framework of diffeological groups. 
    But let us show how the general
    construction given in \cite{Igl85} or in
    \art{Fundamental-group-acting-on-coverings} applies to the
    special case of diffeological groups. Let $\X$ be a connected
    diffeological space, let $x_0$ be a point of $\X$, chosen
    as a base point. Let $\Paths(\X,x_0,\star)$ be the space of
    paths starting at $x_0$. First of all, the end map $\1
    : p \mapsto p(1)$, defined on  $\Paths(\X,x_0,\star)$, is a
    subduction \art{Pathwise-connectedness}. The quotient of 
    $\Paths(\X,x_0,\star)$ by the fixed
    ends homotopy equivalence relation is exactly the
    universal covering, pointed by the constant map $\hat x_0
    : t \mapsto x_0$, over the pointed space $(\X, x_0)$
    \art{The-universal-covering}. The fiber over $x_0$ is the
    homotopy group $\pi_1(\X, x_0)$. Now, if $\X = \G$ we
    choose the identity $\id_\G$ as base point. Thus, the
    multiplication of paths $(p,p') \mapsto [t \mapsto p(t)
    \cdot p'(t)]$ defines on $\tilde \G$ a group
    multiplication such that the projection $\pi : \tilde
    \G \to \G$, defined by $\pi(\class(p)) = \1(p)$, is a homomorphism. 
    The kernel of this morphism is clearly the
    fiber over $\id_\G$, that is, $\pi_1(\G)$. The kernel
    of a homomorphism is always a normal subgroup. 
    Then, since $\pi$ is a covering, $\pi^{-1}(\id_\G)$ is
    discrete \art{Fundamental-group-acting-on-coverings}.
  \end{proof}
  
  \begin{article}\artlabel{Covering smooth actions}
    \addcontentsline{toc}{section}{\small\hspace{10pt} Covering smooth actions}
    \label{Covering-smooth-actions} 
    Let $\X$ be a connected diffeological space. Let $\G$ be a connected
    diffeological group. Let $\rho : \G \to \Diff(\X)$ be a
    smooth action of $\G$ on $\X$, hence $\rho$  takes its
    values in the identity component of $\Diff(\X)$.
    Then, there exists a
    unique smooth action $\tilde \rho$, of the universal
    covering $\tilde \G$ of $\G$ on the universal covering
    $\tilde \X$ of $\X$, covering $\rho$. Precisely, let us denote briefly 
    by ${\bf D}_\X$ the group $\Diff(\X)$, by ${\bf D}_\X^\bullet$ 
    the identity component and by $\tilde {\bf D}_\X^\bullet$ its universal covering. 
    Let $\pi_\G : \tilde \G \to \G$ and 
    $\pi_{\bf D} : \tilde {\bf D}_\X^\bullet \to {\bf D}_\X^\bullet$ be the projections,
    $\tilde \rho : \tilde \G \to \tilde {\bf D}_\X^\bullet$ is a smooth homomorphism
    such that the following diagram commutes.
    \begin{center}
    \begin{tikzpicture}[ auto]
      \node (A)  at (0,0) {$\tilde\G$};
      \node (B)  at (3,0) {$\tilde{\bf D}_\X^\bullet$};
      \node (C)  at (0,-2) {$\G$};
      \node (D)  at (3,-2) {${\bf D}_\X^\bullet$};
      \draw[->] (A) to node {$\tilde\rho$} (B);
      \draw[->] (A) to node [swap] {$\pi_\G$} (C);
      \draw[->] (B) to node {$\pi_{\bf D}$} (D);
      \draw[->] (C) to node [swap] {$\rho$} (D);
    \end{tikzpicture}
    \end{center} 
  \end{article} %% Covering-smooth-actions
    
  \begin{proof}
    The map $\rho \circ \pi_\G$ is smooth and $\tilde \G$ is
    simply connected. Thanks to the monodromy theorem
    \art{Monodromy-theorem}, there exists a
    unique lift $\tilde \rho$ of $\rho \circ \pi$ mapping
    the identity of $\tilde \G$ to the identity of
    $\tilde {\bf D}_\X^\bullet$. This lift is a
    homomorphism because its restriction to $\ker(\pi_\G)$
    and its projection $\rho$ are both homomorphisms.
  \end{proof}
  
  %%%%%%%%%%%%%%%%%%%%%%%%%%%%%%%%%%%%%%%%%%%%%%%%%%%%%%%%%%
  %
  %   Exercises  
  %
  %%%%%%%%%%%%%%%%%%%%%%%%%%%%%%%%%%%%%%%%%%%%%%%%%%%%%%%%%%
  
  \Exercises
  
  \begin{exercise}[Subgroups of $\RR$]
    \label{Subgroups-of-R}
    Show that every strict subgroup $\K$ of $\RR$, that is,
    $\K \subset \RR$ and $\K \neq \RR$,
    is diffeologically discrete and that the only embedded
    strict subgroups are the groups $a\ZZ \subset \RR$, where
    $a \in \RR$. 
  \end{exercise} %% Subgroups-of-R
  
  \begin{exercise}[Diagonal diffeomorphisms]
    \label{Diagonal-diffeomorphisms}
    Let $\X$ be a diffeological space. Let $\N > 0$ be any
    integer. Let $\Diff(\X)$ and $\Diff(\X^\N)$ be
    equipped with the functional diffeology
    \art{Functional-diffeology-on-groups-of-diffeomorphisms}.
    Show that the diagonal injection $\Delta : \Diff(\X) \to
    \Diff(\X^\N)$, defined by $\Delta(\varphi)(x_1,
    \ldots, x_\N) = (\varphi(x_1), \ldots, \varphi(x_\N))$,
    is an induction. 
  \end{exercise} %% Diagonal-diffeomorphisms
  
  \begin{exercise}[The Hilbert sphere is homogeneous]
    \label{The-Hilbert-sphere-is-homogeneous}
    Let $\cH_\CC$ be the standard Hilbert space,
    described in \art{The-fine-standard-Hilbert-space}. Let
    $\cS_\CC$ be the  Hilbert unit sphere
    \art{The-infinite-sphere} and $\cP_\CC$ be the Hilbert
    projective space, see
    \art{The-infinite-complex-projective-space} and
    \exref{The-Hopf-S1-bundle}. Let $\UU(\cH_\CC)$ be the
    unitary group of $\cH_\CC$, equipped with the functional
    diffeology
    \art{Functional-diffeology-between-fine-spaces}.
    Using the fact that a subspace $\E \subset \cH_\CC$ 
    and its orthogonal $\E^\perp$ are supplementary \cite{Bou55}, 
    show that $\cS_\CC$ is homogeneous under $\UU(\cH_\CC)$. 
    Deduce that $\cP_\CC$ also is homogeneous. 
  \end{exercise} %% The-Hilbert-sphere-is-homogeneous
  
  %%%%%%%%%%%%%%%%%%%%%%%%%%%%%%%%%%%%%%%%%%%%%%%%%%%%%%%%%%
  
  \section*{The Spaces of Momenta}
  
  \begin{sechead}
    Diffeological groups are a huge
    category of groups, not very well known and still to
    explore. Their left (or right) invariant
    differential $1$-forms, called {\em momenta}, 
    and their orbits under the coadjoint action 
    are key concepts, at a classical level 
    in symplectic mechanics, but also for the development of
    \guillemots{symplectic diffeology}. In this section we introduce these objects 
    and we give some of their properties. They will be
    used further, in the chapter \ref{Chapter-Symplectic-Diffeology}. 
    In particular, the notion of coadjoint orbit,
    more precisely $(\Gamma,\theta)$-coadjoint orbit, plays a crucial 
    role in the construction of the moment maps, for an arbitrary 
    action of a diffeological group preserving a closed $2$-form
    \art{Definition-of-the-moment-maps} \art{The-Souriau-cocycle}. 
    It is in this sense, for example, that 
    \guillemots{every symplectic manifold is a coadjoint orbit} 
    \art{Symplectic-manifolds-are-coadjoint-orbits}.
  \end{sechead}
  
  \begin{article}\artlabel{Momenta of a diffeological group}
    \addcontentsline{toc}{section}{\small\hspace{10pt} Momenta of a diffeological group} 
    \label{Momenta-of-a-diffeological-group}
    We call {\em left momentum} ---~or simply {\em momentum}~--- 
    of a diffeological group $\G$ any $1$-form on $\G$,
    invariant by the left action of $\G$ onto itself. We
    denote by $\cG^*$ the {\em space of momenta} of $\G$. The
    space of momenta of a diffeological group is naturally a
    diffeological vector space, equipped with the functional
    diffeology,
    $$
    \cG^* = \{ \alpha \in \Omega^1(\G) \mid \mbox{For all } g \in
    \G, \ \eL(g)^*(\alpha) = \alpha \}.
    $$
    Note that, in spite of what the notation $\cG^*$ suggests,
    the space of momenta of a diffeological group is
    not defined by some duality. This notation is chosen here
    just to remind ourselves the connection with the dual of the Lie
    algebra in the classical case of Lie groups. 
  \end{article} %% Momenta-of-a-diffeological-group
  
  \begin{article}\artlabel{Momenta and connectedness}
    \addcontentsline{toc}{section}{\small\hspace{10pt} Momenta and connectedness} 
    \label{Momenta-and-connectedness} 
    Let $\G$ be a diffeological group. Let $\G^\circ$ be its
    identity component, that is,  $\G^\circ = \comp(\id_\G) \subset \G$. 
    Then, the pullback
    $j^* : \cG^* \to {\cG^\circ}^{\raisebox{-3.2pt}{\scriptsize *}}$ of the
    injection $j : \G^\circ \to \G$ is an isomorphism. This
    property is quite natural but deserves to be checked in
    the context of diffeological groups. 
    
    \Note~Said differently, the space of momenta of a
    connected diffeological group, or any of its extensions 
    by a discrete group, coincide. In particular, the only
    momentum of a discrete group is the momentum zero.
  \end{article} %% Momenta-and-connectedness
  
  \begin{proof}
    Let us check first the injectivity. Let $\alpha \in \cG^*$
    such that $j^*(\alpha) = 0$, and let $\P : \U \to \G$ be a
    plot. Let $r_0 \in \U$ and let $\B \subset \U$ be a small
    open ball centered at $r_0$, and $g_0 = \P(r_0)$. Since
    $\B$ is connected, since $\eL(g_0^{-1}) \circ \P(r_0) =
    \id_\G$, and thanks to the smoothness of group
    operations, the parametrization $\Q = [\eL(g_0^{-1})
    \circ \P] \restriction \B$ is a plot of $\G^\circ$. Since $j^*(\alpha) = 0$,
    $\alpha(\Q) = 0$. But, $\alpha(\Q) = \alpha(\eL(g_0^{-1})
    \circ (\P \restriction \B)) = \eL(g_0^{-1})^*(\alpha)(\P
    \restriction \B) = \alpha(\P \restriction \B)$. Thus,
    $\alpha(\P \restriction \B) = 0$. Since $\alpha$ vanishes
    locally at each point of $\U$, $\alpha = 0$, and $j^*$ is
    injective. Now, let us prove the surjectivity. Let
    $\alpha \in {\cG^\circ}^{\raisebox{-3.2pt}{\scriptsize
    *}}$. For any component $\G_i$ of $\G$, let us choose an
    element $g_i \in \G_i$, and the identity for the
    identity component. Let $\P : \U \to \G$ be a plot, an let
    us assume that $\U$ is connected. So, $\P(\U)$ is
    contained in one connected component of $\G$, let us say
    the component $\G_i$. Let us define then, $\bar
    \alpha(\P) = \alpha(\eR(g_i^{-1}) \circ \P)$. Since
    $\eR(g_i^{-1}) \circ \P (r) \in \G^\circ$ for all $r \in
    \U$, this is well defined. Now, since any plot is the sum
    of its restrictions on the components of its domain, the
    map $\bar \alpha$ extends naturally to every plot of $\G$.
    Now, let $\P : \U \to \G$ be a plot, let $\V$ be a
    domain and $\F \in \Cinfty(\V,\U)$. Let $s_0 \in
    \V$, let $\V_0$ be the  component of $s_0$ in $\V$, let
    $r_0 = \F(s_0)$ and $\U_0$ be the component of $r_0$
    in $\U$. Let $\G_i$ be the component of $\P \circ \F(s_0)
    = \P(r_0)$ in $\G$. We have $\bar \alpha((\P \circ \F)
    \restriction \V_0) = \bar \alpha((\P \restriction \U_0)
    \circ (\F \restriction \V_0)) = \alpha(\eR(g_i^{-1})
    \circ (\P \restriction \U_0) \circ (\F \restriction
    \V_0)) = \alpha([\eR(g_i^{-1}) \circ (\P \restriction
    \U_0)] \circ (\F \restriction \V_0)) = (\F \restriction
    \V_0)^*[\alpha(\eR(g_i^{-1}) \circ (\P \restriction
    \U_0)] = (\F \restriction
    \V_0)^*[\bar \alpha(\P \restriction
    \U_0)]$. Hence, $\bar \alpha (\F \circ \P) =_{\rm
    loc} \F^*(\bar \alpha(\P))$, and if it is satisfied
    locally then it is satisfied globally, thus $\bar \alpha (\F
    \circ \P) = \F^*(\bar \alpha(\P))$. The map $\bar \alpha$
    is a well defined differential $1$-form on $\G$. Let us
    check now that $\bar \alpha$ is invariant by left
    multiplication. Let $g \in \G$, let $\P : \U \to \G$ be a
    plot, let $r_0 \in \U$, let $\U_0$ be the component of
    $r_0$ in $\U$, let $\G_i$ be the component of $\P(r_0)$
    in $\G$, so $\P(\U_0) \subset \G_i$. We have
    $\eL(g)^*(\bar \alpha(\P \restriction \U_0)) = \bar
    \alpha(\eL(g) \circ (\P \restriction \U_0)) =
    \alpha(\eR(g_i^{-1}) \circ \eL(g) \circ (\P
    \restriction \U_0)) = \alpha( \eL(g) \circ \eR(g_i^{-1})
    \circ (\P \restriction \U_0)) =
    [\eL(g)^*(\alpha)] (\eR(g_i^{-1}) \circ (\P \restriction
    \U_0)) = \alpha (\eR(g_i^{-1}) \circ (\P \restriction
    \U_0)) = \bar \alpha (\P \restriction
    \U_0)$. So, $\eL(g)^*(\bar \alpha)(\P) =_{\rm loc}
    \bar \alpha(\P)$, and hence globally. Thus,
    $\eL(g)^*(\bar \alpha) = \bar \alpha$. Therefore, $\bar \alpha$
    is an element of $\cG^*$ which coincides with $\alpha$ on
    $\G^\circ$. Since the pullback is a smooth operation,
    $j^*$ is a smooth linear bijection, on the other hand, since the
    multiplication by the base points $g_i \in \G_i$ is smooth, the process
    of extension $\alpha \mapsto \bar\alpha$ is smooth and $j^*$ is an 
    isomorphism of diffeological vector spaces.
  \end{proof}
  
  \begin{article}\artlabel{Momenta of coverings of diffeological groups}
    \addcontentsline{toc}{section}{\small\hspace{10pt} Momenta of coverings of diffeological
    groups} 
    \label{Momenta-of-covering-of-diffeological-groups} 
    Let $\G$ be a diffeological group, let $\pr :
    \widehat \G \to \G$ be some group covering, see
    \art{Covering-diffeological-groups}. Let $\cG^*$ and
    $\widehat \cG^*$ be the spaces of momenta of $\G$ and
    $\widehat \G$. Then, the pullback $\pr^* : \cG^* \to \widehat \cG^*$ 
    is an isomorphism.  
  \end{article} %% Momenta-of-covering-of-diffeological-groups
  
  \begin{proof}
    Thanks to \art{Momenta-and-connectedness}, it is
    sufficient to assume that $\widehat \G$ and $\G$ are
    connected, and thanks to
    \art{Covering-diffeological-groups}, it is sufficient
    to check it for the universal covering $\pi : \tilde \G \to \G$. 
    Now, $\pi^*$ is obviously linear, let us show
    that $\pi^*$ is surjective. Let $\tilde\alpha \in \tilde\cG^*$. 
    The group $\G$ is isomorphic to
    $\tilde \G/\pi_1(\G)$, with respect to the left
    action of $\pi_1(\G)$, that is, $\tilde g \sim k \tilde
    g$, for all $k \in \pi_1(\G)$. Now, let $\tilde \alpha
    \in \tilde\cG^*$, $\tilde \alpha$ is left invariant
    by $\tilde \G$, thus by $\pi_1(\G)$, that is, for
    all $k \in \pi_1(\G)$, $\eL(k)^*(\tilde \alpha) = \tilde
    \alpha$. But, since $\pi_1(\G) = \ker(\pi)$ is discrete,
    this is sufficient for the existence of a $1$-form $\alpha$
    on $\G$ such that $\tilde \alpha = \pi^*(\alpha)$
    \art{Pushing-forms-onto-quotients}, and the 
    map $\tilde \alpha = \alpha$ is smooth. 
    Now, let $\tilde g \in \tilde \G$ and $g = \pi(\tilde g)$.
    Since $\pi$ is a homomorphism, $\pi \circ \eL(\tilde g) =
    \eL(g) \circ \pi$. Thus, on the one hand we have $\eL(\tilde
    g)^*(\tilde \alpha) = \eL(\tilde g)^*(\pi^*(\alpha)) =
    (\pi \circ \eL(\tilde g))^*(\alpha) = (\eL(g) \circ
    \pi)^*(\alpha) = \pi^*(\eL(g)^*(\alpha))$, and on the
    other hand $\eL(\tilde g)^*(\tilde \alpha) =
    \tilde \alpha = \pi^*(\alpha)$. Hence,
    $\pi^*(\eL(g)^*(\alpha)) = \pi^*(\alpha)$. But since
    $\pi$ is a subduction, $\eL(g)^*(\alpha) = \alpha$,
    $\alpha \in \cG^*$ and the map $\pi^*$ is surjective.
    Next, let $\tilde \alpha$ and $\tilde \beta$ be such that
    $\pi^*(\tilde \alpha) = \pi^*(\tilde \beta)$. Since
    $\pi$ is a subduction, $\tilde \alpha = \tilde \beta$,
    $\pi^*$ is injective and thus bijective. 
    Finally, $\pi^* : \cG^* \to \tilde \cG^*$ is a smooth linear
    bijection, its inverse is smooth, therefore it is 
    an isomorphism of diffeological vector spaces.
  \end{proof}
  
  \begin{article}\artlabel{Linear coadjoint action and coadjoint orbits}
    \addcontentsline{toc}{section}{\small\hspace{10pt} Linear coadjoint action and coadjoint orbits} 
    \label{Linear-coadjoint-action} 
    Let $\G$ be a diffeological group and let $\cG^*$ be the space of its
    momenta. The pushforward $\Ad(g)_*(\alpha)$ of a momentum
    $\alpha \in \cG^*$, by the adjoint action of any element
    $g$ of $\G$, is again a momentum of $\G$, that is, again a
    left-invariant $1$-form. This defines a linear smooth
    action of $\G$ on $\cG^*$ called  {\em
    coadjoint action}, and denoted by $\Ad_*$,
    $$ 
    \Ad_* : (g,\alpha) \mapsto \Ad(g)_*(\alpha) =
    \Ad(g^{-1})^*(\alpha).
    $$
    We check immediately that for all $g$, $g'$ in $\G$,
    $\Ad_*(gg') = \Ad_*(g) \circ \Ad_*(g')$, and
    that $\Ad_*(g)$ is linear. Note that, since
    $\alpha$ is left-invariant, $\Ad_*(g)(\alpha) = 
    \eR(g)^*(\alpha)$.
    The orbit of $\alpha$ by $\G$ is by definition
    a {\em coadjoint orbit\/} of $\G$,
    it will be denoted by
    $$
    \cO_\alpha \mbox{ or } \Ad_*(\G)(\alpha) =
    \{ \Ad_*(g)(\alpha) \mid g\in \G \}. 
    $$
    The orbit $\cO_\alpha$ can be regarded as a subset of
    $\cG^*$, but also as the quotient of the group $\G$ by the
    stabilizer of the momentum $\alpha$, 
    $$
    \cO_\alpha \simeq \G/\Stab_\G(\alpha), \mbox{ with } 
    \Stab_\G(\alpha) = \{ g \in \G \mid \Ad(g)_*(\alpha) =
    \alpha \}.
    $$
    \Note~The orbit $\cO_\alpha$ can be equipped
    with the subset diffeology of the functional diffeology of
    $\cG^*$, or with the quotient diffeology of $\G$. 
    For a Hausdorff second countable Lie group it has been proved that these diffeologies
    coincide \cite{IZK10}, but there is
    no reason {\em a priori\/} that these two diffeologies coincide
    in general. It could be interesting however to understand in which conditions they do.
  \end{article} %% Linear-coadjoint-action
  
  \begin{article}\artlabel{Affine coadjoint actions and $(\Gamma,\theta)$-coadjoint orbits} 
    \addcontentsline{toc}{section}{\small\hspace{10pt} Affine coadjoint actions and $(\Gamma,\theta)$-coadjoint orbits} 
    \label{Affine-coadjoint-actions-and-orbits} 
    Let $\G$ be a diffeological group, and $\cG^*$ be the space of its
    momenta. Let $\Gamma \subset \cG^*$ be a subgroup of
    $(\cG^*,+)$, invariant by the coadjoint action $\Ad_*$,
    that is,
    $$
    \Ad_*(g)(\Gamma) \subset \Gamma,
    $$
    for all $g \in \G$. Then, the coadjoint action of $\G$ on $\cG^*$ projects to
    the quotient $\cG^*\!/\Gamma$, regarded as an Abelian group,
    as a smooth action denoted by $\Ad_*^\Gamma$,
    $$
    \Ad_*^\Gamma(g)(\tau) = \class(\Ad_*(g)(\mu))
    \qmbox{with} \tau = \class(\mu) \in \cG^*\!/\Gamma,
    $$ 
    for all $g \in \G$ and all $\tau \in
    \cG^*\!/\Gamma$.
    Now, let $\theta$ be a smooth map, from $\G$
    to the space $\cG^*\!/\Gamma$, such that
    $$
    \theta(g g') = \Ad_*^\Gamma(g)(\theta(g')) + \theta(g),
    $$
    for all  $g,g' \in \G$.
    Such maps are formally known, in the literature, as
    twisted {\em $1$-cocycles} of $\G$ with values in
    $\cG^*\!/\Gamma$ \cite{Kir76}. We shall call them
    cocycles of $\G$, with values in $\cG^*\!/\Gamma$, or
    simply $(\cG^*\!/\Gamma)$-cocycles. A cocycle $\theta$ is
    a {\em coboundary} if there exists a constant $c \in
    \cG^*\!/\Gamma$, such that $\theta = \Delta c$, with
    $$
    \Delta c : g \mapsto \Ad_*^\Gamma(g)(c) -c.
    $$
    Cocycles modulo coboundaries define a cohomology group
    denoted by $\H^1(\G,\cG^*\!/\Gamma)$. Every such cocycle
    $\theta$ defines a new action of $\G$ on $\cG^*\!/\Gamma$ by
    $$
    \Ad^{\Gamma,\theta}_* : (g, \tau) \mapsto
    \Ad_*^\Gamma(g)(\tau) + \theta(g).
    $$
    The orbits for these actions will be called the
    {\em $(\Gamma,\theta)$-coadjoint orbits} of $\G$. If
    $\Gamma = \{0\}$ we shall call them simply
    $\theta$-coadjoint orbits. If $\theta = 0$ we shall call
    them simply $\Gamma$-coadjoint orbits. And, if $\Gamma =
    \{0\}$ and $\theta = 0$ we find again the ordinary
    coadjoint orbits defined in  \art{Linear-coadjoint-action}.
  \end{article} %% Affine-coadjoint-actions-and-orbits
  
  \begin{article}\artlabel{Closed momenta of a diffeological group}
    \addcontentsline{toc}{section}{\small\hspace{10pt} Closed momenta of a diffeological group} 
    \label{Closed-momenta-of-a-diffeological-group} Let $\G$
    be a diffeological group, and $\cG^*$ be its space of
    momenta. Let us denote by $\ZG$ the subset of closed
    momenta of $\G$,  and by $\BG$ the subset of exact
    momenta of $\G$, that is,
    $$
    \ZG = \ZDR^1(\G) \cap \cG^* \qmbox{and} \BG =
    \BDR^1(\G) \cap \cG^*.
    $$
    \alinea{1.}~Let us assume that $\G$ is connected, and let
    $\tilde \G$ be its universal covering. By
    factorization, the Chain-Homotopy operator defines a
    canonical De Rham isomorphism $\ek$, from the space of
    closed momenta $\ZG$ to the vector space
    $\DHom(\tilde \G,\RR)$, that is, for all $\zeta \in \ZG$,
    $$
    \ek(\zeta) = [\tilde g \mapsto
    \CHK\zeta(p)], \qmbox{where} 
    \CHK\zeta(p) =
    \int_{p} \zeta
    \qmbox{and} \tilde g = \class(p).
    $$ 
    Here, we have denoted by $\class(p)$  the
    fixed-ends homotopy class of the path $p \in
    \Paths(\G,\id_\G)$. The subspace of exact momenta
    $\BG$ identifies, through the
    isomorphism $\ek$, with the subspace $\DHom(\G,\RR)$,
    $$
    \ZG \simeq \DHom(\tilde \G,\RR)
    \qmbox{and} 
    \BG \simeq \DHom(\G,\RR).
    $$
    \alinea{2.}~Let $\G$ be a diffeological group, connected or not. 
    Let $\zeta \in \cG^*$, if $\zeta$ is closed then $\zeta$ is $\Ad_*$ invariant. 
    For all $\zeta \in \cG^*$, 
    $$
    d\zeta = 0 
    \qmbox{implies} \Ad_*(g)(\zeta) = \zeta, \mbox{ for
    all } g \in \G.
    $$
    \Note~Every homomorphism from a 
    group $\G$ to an Abelian group factorizes through the
    {\em abelianized group} $\Ab{\G} = \G/[\G,\G]$, where
    $[\G,\G]$ is the normal subgroup of the commutators of
    $\G$.  Thus actually, $\ZG \simeq \DHom(\Ab{\tilde\G},\RR)$ and $\BG \simeq
    \DHom(\Ab{\G},\RR)$.
  \end{article} %% Closed-momenta-of-a-diffeological-group
  
  \begin{proof}
  1. Let $\pi : \tilde \G \to
    \G$ be the universal covering defined in 
    \art{Covering-diffeological-groups}. Since $\tilde
    \G$ is simply connected, every closed $1$-form is exact
    \art{Closed-1-forms-vanishing-on-loops}. Thus, for every
    $\zeta \in \ZG$, the pullback $\pi^*(\zeta)$ is exact.
    Let $\F$ be a primitive of $\pi^*(\alpha)$, that is,
    $d\F = \pi^*(\alpha)$. We can even set uniquely $\F$ by
    choosing $\F(\id_{\tilde \G}) = 0$. Actually $\F$ is
    defined by integrating the form $\zeta$ along the paths
    starting at the identity, that is, $\F = \ek(\zeta)$.
    Since $\alpha$ is left-invariant and since the projection
    $\pi$ commutes with the left actions, on $\G$ and
    $\tilde \G$, $\pi^*(\alpha)$ is left invariant. So,
    for every $\tilde g \in \tilde \G$, $d[\F \circ
    \eL(\tilde g)] = d\F$. Since $\tilde \G$ is connected,
    for every $\tilde g$, $\tilde g'$ in $\tilde \G$,
    $\F(\tilde g \tilde g') = \F(\tilde g') + f(\tilde g)$,
    where $f$ is a smooth real function. But since
    $\F(\id_\G) = 0$, $f(\tilde g) = \F(\tilde g)$, and $\F$
    is a smooth homomorphism from $\tilde \G$ to $\RR$.
    So, for every closed momentum $\zeta \in \ZG$, there
    exists a unique homomorphism $\F \in
    \DHom(\tilde \G, \RR)$ such that $\zeta =
    \pi_*(d\F)$. The homomorphism $\ek$ is thus injective,
    and it is obviously surjective. Now, if $\zeta$ is exact,
    that is, if $\zeta = df$, then $\F = \pi^*(f)$. So,
    $\ek(\BG) = \pi^*(\DHom(\G,\RR)) \simeq
    \DHom(\G,\RR)$.
    
   \alinea{2.}~Thanks to \art{Momenta-and-connectedness} we can
    assume that $\G$ is connected. Now, for every $\tilde g$,
    $\tilde g'$ in $\tilde \G$, $\F(\tilde g \tilde g'
    \tilde g^{-1}) = \F(\tilde g')$, that is, $\F \circ
    \Ad(\tilde g) = \Ad(\tilde g)^*(\F) = \F$, for all
    $\tilde g \in \tilde \G$. Thus, $d[\Ad(\tilde g)^*(\F)]
    = d\F$, or $\Ad^*(\tilde g)(\pi^*(\zeta)) = \pi^*(\zeta)$,
    or $(\pi \circ \Ad(\tilde g))^*(\zeta) = \pi^*(\zeta)$.
    But $\pi \circ \Ad(\tilde g) =  \Ad(g) \circ \pi$,  where
    $g = \pi(\tilde g)$. Hence, $\pi^*(\Ad(g)^*(\zeta)) =
    \pi^*(\zeta)$, and since $\pi$ is a subduction,
    $\Ad(g)^*(\zeta) = \zeta$, that is,
    $\Ad_*(g)(\zeta) = \zeta$.
  \end{proof}
  
  \begin{article}\artlabel{The value of a momentum} 
    \addcontentsline{toc}{section}{\small\hspace{10pt} The value of a momentum} 
    \label{The-value-of-a-momentum}
    Let $\G$ be a diffeological group. Every momentum
    $\alpha$ of $\G$ is
    characterized by its value at the identity
    \art{The-values-of-a-differential-form}. In other
    words, if the value $\alpha_\id$ of $\alpha$ at the
    identity $\id_\G \in \G$ vanishes, then $\alpha$ is zero. 
    Equivalently, $\alpha = 0$ if and only if for
    any 1-plot $\F$ of $\G$, centered at the identity,
    $\alpha(\F)(0) = 0$.
    Equivalently, two momenta of $\G$, $\alpha$ and $\beta$,
    coincide if and only if, for any 1-plot $\F$ of $\G$
    centered at the identity, $\alpha(\F)(0) =
    \beta(\F)(0)$.
    
    \Note~It is tempting to reduce this
    proposition for $\F$ being a germ of a
    $1$-parameter subgroup of $\G$. But, if it is clear that
    this is possible for Lie groups, it is not clear that it is
    still the case for any diffeological group. 
  \end{article} %% The-value-of-a-momentum
  
  \begin{proof}
    Let us consider an $n$-plot $\P: \U \to \G$ and $r
    \in \U$. Let
    $\T_r$ be the translation $\T_r: s \mapsto  s+r$, and
    let us denote $g = \P(r)$. The parametrization 
    $\bar\P = \eL(g^{-1}) \circ \P \circ \T_r$, defined on 
    $\bar\U = \T_r^{-1}(\U)$, is a plot of $\G$, centered at
    the identity: $0 \in \bar\U$ and 
    $\bar\P(0) = \id_\G$. So, $\P = \eL(g) \circ \bar\P \circ \T_{-r}$, and 
    $\alpha(\P)_r = \alpha(\eL(g) \circ \bar\P \circ \T_{-r})_r = 
    (\eL(g) \circ \bar\P \circ \T_{-r})^*(\alpha)_r = 
    \T_{-r}^* \circ \bar\P^* \circ \eL(g)^*(\alpha)_r$. But
    $\eL(g)^*(\alpha) = \alpha$, thus
    $\alpha(\P)_r =  \T_{-r}^* \circ \bar\P^*(\alpha)_r = 
    \T_{-r}^*[\bar\P^*(\alpha)]_r = \bar\P^*(\alpha)_0 = 0$.
    Hence, if $\alpha(\F)(0) = 0$ for every 1-plot $\F$
    centered at $\id_\G$, then $\alpha(\P)(r) = 0$, for every
    plot $\P$ and every $r \in \Dom(\P)$, that is, 
    $\alpha = 0$. Therefore, the momenta
    of $\G$ are characterized by their values at the
    identity. Now, we know that
    every $1$-form is characterized by its values on the
    $1$-plot \art{The-k-forms-are-defined-by-the-k-plots},
    what completes the proposition.
  \end{proof}
  
  \begin{article}\artlabel{Equivalence between right and left momenta} 
    \addcontentsline{toc}{section}{\small\hspace{10pt} Equivalence between right and left
    momenta} 
    \label{Equivalence-between-right-and-left-momenta} Let
    $\G$ be a diffeological group, and let $\cG^\star$ denote
    the space of {\em right momenta} of the group $\G$, that
    is, the space of $1$-forms of $\G$ invariant by the right
    multiplication,
    $$ \cG^\star
    = \{ \alpha \in \Omega^1(\G) \mid \mbox{For all } g \in
    \G, \ \eR(g)^*(\alpha) = \alpha \}.
    $$
    There exists a natural linear isomorphism 
    $\flip : \cG^* \to \cG^\star$ equivariant with respect to the coadjoint
    action, $\Ad_*(g) \circ \flip = \flip \circ\Ad_*(g)$, for all $g \in \G$.
    In other words, there is no reason to prefer left or
    right momenta of a diffeological group. The particular
    role of left momenta comes because we are
    dealing with actions of groups and not anti-actions. This situation is 
    summarized by the following commutative diagram.
    
    \begin{center}
    \begin{tikzpicture}[ auto]
      \node (A)  at (0,0) {$\cG^*$};
      \node (B)  at (3,0) {$\cG^\star$};
      \node (C)  at (0,-2) {$\cG^*$};
      \node (D)  at (3,-2) {$\cG^\star$};
      \draw[->] (A) to node {$\flip$} (B);
      \draw[->] (A) to node [swap] {$\Ad_*(g)$} (C);
      \draw[->] (B) to node {$\Ad_*(g)$} (D);
      \draw[->] (C) to node [swap] {$\flip$} (D);
    \end{tikzpicture}
    \end{center}
  \end{article} %% Equivalence-between-right-and-left-momenta
  
  \begin{proof}
    Let us denote by a dot the multiplication in $\G$. Let
    $\alpha$ be any left $p$-momentum of $\G$. Let $\P : \U
    \to \G$ be an $n$-plot. Let $\bar
    \alpha(\P)$ be defined by
    $$ \bar \alpha(\P)_r = \alpha\left[s \mapsto \P(s)
    \cdot \P(r)^{-1}\right]_{s=r}.
    $$
    where $r$ belongs to $\U$. Let us show that $
    \bar{\alpha}$ defines a $p$-form of $\G$. First
    of all let us remark that $\bar \alpha(\P)$ is the
    restriction of the $1$-form $\alpha((s,r) \mapsto
    \P(s)\cdot \P(r)^{-1})$ to the diagonal $s=r$. Thus,
    $\bar \alpha(\P)$ is a smooth $1$-form of $\U$. 
    Now, let us check that $\bar\alpha$ is a well
    defined differential $1$-form on $\G$. let
    $\F : \V \to \U$ be a smooth $m$-parametrization, $v$
    be a point of $\V$ and $\delta v$ be a vector of
    $\RR^m$. We have
    \begin{eqnarray*}
      \bar{\alpha}(\P \circ \F)_v(\delta v) 
      & = & \alpha\left[s \mapsto (\P \circ \F)(s) \cdot (\P \circ \F)(v)^{-1}\right]_v(\delta v) \\  
      & = & \alpha\left[s \mapsto \F(s) \mapsto (\P \circ \F)(s) \cdot (\P \circ \F)(v)^{-1}\right]_v(\delta v) \\  
      & = & \alpha\left[s \mapsto r = \F(s) \mapsto \P(r) \cdot \P(\F(v))^{-1}\right]_v(\delta v) \\  
      & = & \alpha\left(\left[r \mapsto \P(r) \cdot \P(\F(v))^{-1}\right] \circ \F\right)_v(\delta v) \\ 
      & = & \F^*\left[\alpha\left(r \mapsto \P(r) \cdot \P(\F(v))^{-1}\right)\right]_v(\delta v) \\
      & = & \alpha\left[r \mapsto \P(r) \cdot \P(\F(v))^{-1}\right]_{\F(v)} (\D(\F)(v)(\delta v)) \\
      & = & \bar{\alpha}(\P)_{\F(v)}(\D(\F)(v)(\delta v))  \\
      & = & \F^*\left[\bar \alpha(\P)\right]_v(\delta v). 
    \end{eqnarray*}
    %
    Next, let us check that the $1$-form $\bar{\alpha}$ is
    right-invariant, that is, $\bar \alpha \in \cG^\star$. For
    all $g \in \G$, we have
    \begin{eqnarray*}
      \eR(g)^*(\bar\alpha)(\P)_r 
      & = &
      \bar\alpha(\eR(g) \circ \P)_r \\
      & = & \alpha\left[s \mapsto 
      (\eR(g) \circ \P)(s) \cdot (\eR(g) \circ
      \P)(r)^{-1}\right]_{s=r} \\
      & = & \alpha\left[s \mapsto
      \P(s) \cdot g \cdot (\P(r) \cdot
      g)^{-1}\right]_{s=r} \\
      & = & \alpha\left[s \mapsto \P(s) \cdot g \cdot g^{-1}
      \cdot \P(r)^{-1}\right]_{s=r} \\
      & = & \alpha\left[s \mapsto \P(s) \cdot
      \P(r)^{-1}\right]_{s=r} \\
      & = & \bar{\alpha}(\P)_r.
    \end{eqnarray*}
    Thus, we just defined a map $\flip : \alpha \mapsto \bar
    \alpha$, from $\cG^*$ to $\cG^\star$. Let us check now
    that $\flip$ is bijective. Let $\beta = \bar \alpha$. Let
    $\P : \U \to \G$ be a plot, and let us define $\bar
    \beta$ by $\bar \beta(\P)(r) = \beta [s \mapsto \P(r)^{-1} \cdot \P(s)](s=r)$, 
    for all $r \in \U$. We have then,
    \begin{eqnarray*}
      \bar\beta(\P)_r & = & \beta \left[s \mapsto \P(r)^{-1} \cdot \P(s)\right]_{s=r} \\
      & = & \bar \alpha \left[s \mapsto \P(r)^{-1} \cdot \P(s)\right]_{s=r} \\
      & = & \alpha \left[s \mapsto \P(r)^{-1} \cdot \P(s) \cdot \P(r)^{-1} \cdot \P(r) \right]_{s=r} 
%      & = & \alpha \left[s \mapsto \P(r)^{-1} \cdot \P(s) \right]_{s=r}  \\
%      & = & \eL(\P(r)^{-1})^*(\alpha) \left[s \mapsto \P(s) \right]_{s=r} \\
%      & = & \alpha(\P)_r. 
    \end{eqnarray*}
    \begin{eqnarray*}
     \phantom{\bar\beta(\P)_r} % & = & \beta \left[s \mapsto \P(r)^{-1} \cdot \P(s)\right]_{s=r} \\
%      & = & \bar \alpha \left[s \mapsto \P(r)^{-1} \cdot \P(s)\right]_{s=r} \\
%      & = & \alpha \left[s \mapsto \P(r)^{-1} \cdot \P(s) \cdot \P(r)^{-1} \cdot \P(r) \right]_{s=r} 
      & = & \alpha \left[s \mapsto \P(r)^{-1} \cdot \P(s) \right]_{s=r} \\ 
      & = & \eL(\P(r)^{-1})^*(\alpha) \left[s \mapsto \P(s) \right]_{s=r} \\
      & = & \alpha(\P)_r. 
    \end{eqnarray*}
    Hence, $\bar \beta = \alpha$. Thus, $\flip$ is bijective and is clearly linear. 
    Therefore, $\flip$ is a
    linear isomorphism from $\cG^*$ to $\cG^\star$. It is easy
    to check then that it is a smooth isomorphism.
    Let us check now that $\flip$ is equivariant under
    the coadjoint action. Let $\alpha \in \cG^*$, let
    $\P: \U \to \G$ be a plot and $r \in \U$. On the one
    hand we have
    \begin{eqnarray*}
      \flip[\Ad(g)^*(\alpha)](\P)_r &=&
      \flip[\eR(g)^*(\alpha)](\P)_r  \\
      & = & \eR(g)^*(\alpha)[s \mapsto 
      \P(s) \cdot \P(r)^{-1}]_{s=r}\\
      & = & \alpha[s \mapsto \P(s) \cdot \P(r)^{-1}
      \cdot g]_{s=r}, 
    \end{eqnarray*}
    and on the other hand,
    %
    \begin{eqnarray*}
      [\Ad(g)^*(\flip(\alpha))](\P)_r & = &
      [\eL(g)_*(\flip(\alpha))](\P)_r \\
      & = &
      \flip(\alpha)(\eL(g^{-1}) \circ \P)_r \\
      & = & \alpha[s \mapsto (\eL(g^{-1}) \circ \P)(s) \cdot
      (\eL(g^{-1}) \circ \P)(r))^{-1}]_{s=r} \\
      & = & \alpha[s \mapsto g^{-1} \cdot \P(s) \cdot
      \P(r)^{-1} \cdot g]_{s=r} \\
      & = & \eL(g^{-1})^*(\alpha)[s
      \mapsto \P(s) \cdot \P(r)^{-1} \cdot g]_{s=r} \\
      & = & \alpha[s \mapsto \P(s) \cdot \P(r)^{-1} \cdot
      g]_{s=r}. 
    \end{eqnarray*}
    Therefore, $\flip \circ \Ad(g)^* = \Ad(g)^* \circ \flip$
    for all $g \in \G$.
  \end{proof}
  
  %%%%%%%%%%%%%%%%%%%%%%%%%%%%%%%%%%%%%%%%%%%%%%%%%%%%%%%%%%
  %
  %   Exercises  
  %
  %%%%%%%%%%%%%%%%%%%%%%%%%%%%%%%%%%%%%%%%%%%%%%%%%%%%%%%%%%
  
  \Exercises
  
  \begin{exercise}[Pullback of 1-forms by multiplication]
    \label{Pullback-of-1-forms-by-multiplication}  
    Let $\G$ be a diffeological group and $\alpha \in \DForms^1(\G)$.
    Let $\mul: \G \times \G \to \G$ be the multiplication map,
    $\mul(a,b) = ab$. Let $\P : \U \to \G$ be 
    an $n$-plot, and  $\Q: \V \to \G$ be a $m$-plot. Let
    $\P \times \Q$  be the $(n+m)$-plot of $\G \times
    \G$ defined by $(r,s) \mapsto (\P(r),\Q(s))$. Let
    $r \in \U$ and $s \in \V$. Let $\delta r \in \RR^n$, and
    $\delta s \in \RR^m$. Show that 
    $$ 
    \mul^*(\alpha)(\P \times \Q)_{r \choose s}\vect{\delta r \\ \delta s} = 
    \eL(\P(r))^*(\alpha)(\Q)_s(\delta s) + \eR(\Q(s))^*(\alpha)(\P)_r(\delta r), 
    $$
    and, in particular,  
    $
    \alpha[r \mapsto \P(r)\cdot \Q(r)]_r =
    \eL(\P(r))^*(\alpha)(\Q)_r + \eR(\Q(r))^*(\alpha)(\P)_r 
    $.
  \end{exercise} %% Pullback-of-1-forms-by-multiplication
  
  \begin{exercise}[Liouville form on groups]
    \label{Liouville-form-on-groups} 
    Let $\G$ be a diffeological group and $\cG^*$ be its space of momenta.
    Let $\G \times \cG^*$ be the diffeological product where $\cG^*$ is equipped with
    its functional diffeology. Let $\lambda$ be the map
    defined for every $n$-plot $\Q: \U \to \G \times \cG^*$ by
    $$
    \lambda(\Q)_r(\delta r) = \A(r)(\P)_r(\delta r),
    $$ 
    where $r\in\U$, $\Q(r) = (\P(r),\A(r))$ and
    $\delta r \in \RR^n$. 
%    Note that since $\A(r)$ is a $1$-form of $\G$, it can be evaluated on the plot $\P$
%    at the point $r$ and applied to the vector $\delta r$.

    \Question{1)}~Show that $\lambda$ is a
      differential $1$-form on the product $\G\times\cG^*$.
    
      \Question{2)}~Let $\alpha \in \cG^*$ and $j_{\alpha} : \G \to \G \times \cG^*$ 
      defined by $j_{\alpha} : g \mapsto (g,\alpha)$. 
      Show that $j_{\alpha}^{*}(\lambda) = \alpha$.
      
      \Question{3)}~Show that the form $\lambda$ is invariant by the following action
      of $\G$ on  $\G \times \cG^*$
      $$ 
      \mbox{for all } g' \in \G, \quad g'_{{\G}\times\cG^*}: 
      (g,\alpha) \mapsto (\Ad(g')(g),\Ad_*(g')(\alpha)).
      $$
    Note the similitude of this construction with the
    general construction of the Liouville form
    for any diffeological space
    \art{The-Liouville-forms-on-the-spaces-of-p-forms}.
  \end{exercise} %% Liouville-form-on-groups
  
