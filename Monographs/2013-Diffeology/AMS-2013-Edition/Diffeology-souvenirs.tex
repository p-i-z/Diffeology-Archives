%%%%%%%%%%%%%%%%%%%%%%%%%%%%%%%%%%%%%%%%%%%%%%%%%%%%%%%%%%
%% 
%% MARK: Souvenirs
%% 
%%%%%%%%%%%%%%%%%%%%%%%%%%%%%%%%%%%%%%%%%%%%%%%%%%%%%%%%%%
  
\chapter*{Afterword}

\begin{chaphead}
  I was a student of Jean-Marie Souriau, working on my
  doctoral dissertation, when he introduced \guillemots{Diffeology}. 
  I remember well, we used to gather for a seminar 
  at that time~--- the beginning of the eighties~--- 
  every Tuesday, at the Center for Theoretical Physics, in
  Marseille's Luminy campus. Jean-Marie was trying to generalize his
  quantization procedure to a certain kind of coadjoint orbits of infinite dimensional 
  groups of diffeomorphisms. 
  He wanted to regard these groups of diffeomorphisms as Lie groups,
  like everybody,
  but he also wanted to avoid topological finessing, feeling
  that that was not essential for this goal. He invented
  then a lighter \guillemots{differentiable} structure on groups
  of diffeomorphisms. These groups quickly became autonomous
  objects. I mean: he gave up groups of diffeomorphisms for
  abstract groups, equipped with an abstract differential
  structure. He called them \guillemots{groupes diff\'erentiels},
  this was the first name for the future diffeological groups.
  
  \alinea{\bf \guillemots{Differential spaces} are born} 
  Listening to Jean-Marie talking about his differential groups,
  I had the feeling that these structures, the axiomatics of
  differential groups, could be easily extended to any
  set, not necessarily groups, and I remember a
  particularly hot discussion about this question in the
  Luminy campus cafeteria. It was during a break
  in our seminar. We were there, the whole group: JMS (as
  we call him), Jimmy Elhadad, Christian Duval, Paul
  Donato, Henry-Hugues Fliche, Roland Triay and myself.
  Souriau denied the interest of considering anything else
  than orbits of differential groups (Souriau was really,
  but really, \guillemots{group-oriented}), and I decided when
  I'd get the time --- I was working on the classification
  of $\SO(3)$-symplectic manifolds which has nothing to do
  with diffeology --- to generalize his axiomatics for any
  sets. But I never got the opportunity to do it. Some
  times later, days or weeks, I don't remember exactly, he
  outlined the general theory of \guillemots{espaces
  diff\'erentiels} as he called them. I would have liked
  to do it, anyway... I must say that, at that time, these
  constructions appeared to us, his students, as a fine
  construction but so general that it could not turn out
  into great results, it could give at most some intellectual
  satisfaction. We were dubitative. 
  I decided to forget differential spaces and stay
  focused on \guillemots{real maths}, the classification of
  $\SO(3)$-symplectic manifolds. I went to
  Moscow, spent a year there, and came back 
  with a complete classification in dimension 4 and
  some general results in any dimension. This work
  represented for me a probable doctoral thesis. It was
  the first global classification theorem in symplectic
  geometry after the homogeneous case, the
  famous Kirillov-Kostant-Souriau theorem which states that any
  homogeneous symplectic manifold is a covering of some
  coadjoint orbit. But Jean-Marie didn't pay any attention
  to my work, looking away from it, as he was completely
  absorbed by his \guillemots{differential spaces}. I was really
  disappointed, I thought that this work deserved to become
  my doctorat. At the same time, Paul Donato gave a
  general construction of the universal covering for any
  quotient of \guillemots{differential groups}, that is, the
  universal covering of any homogeneous \guillemots{differential
  space}. This construction became his doctoral thesis. I decided
  then to give up, for a moment, symplectic geometry and
  to get into the world of differential spaces, since 
  it was the only subject about which JMS was able, or willing, to talk at that time.
  
  \alinea{\bf The coming of the irrational torus} It was the year
  1984, we were taking part in a conference about
  symplectic geometry, in Lyon, when we decided, together
  with Paul, to test diffeology on the {\em irrational torus},
  the quotient of the 2-torus by an irrational line. This
  quotient is not a manifold but remains a diffeological
  space, moreover a diffeological group. We decided to call
  it $\T_\alpha$, where $\alpha$ is the slope of the line.
  The interest for this example came, of course, from the
  Denjoy-Poincar\'e flow about which we heard so much
  during this conference. What had diffeology to say
  about this group, for which topology is completely dry? We
  used the techniques worked out by Paul and computed its
  homotopy groups, we found $\ZZ + \alpha \ZZ \subset \RR$ 
  for the fundamental group and zero for the higher ones.
  The real line  $\RR$ itself appeared as the universal
  covering of $\T_\alpha$. I remember how we were excited
  by this computation, as we didn't believe really in the
  capabilities of diffeology for saying anything serious about such
  \guillemots{singular} spaces or groups. Don't forget that differential spaces
  had been introduced to study infinite dimensional groups and not 
  singular quotients.
  We continued to explore
  this group and found that, as diffeological space,
  $\T_\alpha$ is characterized by $\alpha$, up to a
  conjugation by $\GL(2,\ZZ)$, and we found that the
  components of the group of diffeomorphisms of $\T_\alpha$
  distinguish the cases where $\alpha$ is quadratic or not. It
  became clear that diffeology was not such a trivial theory
  and deserved to be more developed. At the same time,
  Alain Connes introduced the first elements of
  non-commutative geometry, and applied them to the
  irrational flow on the torus --- our favorite example ---
  and his techniques didn't give anything more (in fact
  less) than the diffeological approach, which we
  considered more in the spirit of ordinary geometry. We were in a good position 
  to know the application of Connes' theory
  to irrational flows as he had many fans, in the Center
  for Theoretical Physics at that time, developing his
  ideas.
  
  All in all, this example convinced me that diffeology was a
  good tool, not as weak as it seemed to be. 
  And I decided to continue to explore this path. 
  The result of the computation of the homotopy group
  of $\T_\alpha$ made me think that everything was as if
  the irrational flow was a true fibration of the 2-torus:
  the fiber $\RR$ being contractible the homotopy of the
  quotient $\T_\alpha$ had to be the same as the total
  space $\T^2$, 
  and one should avoid Paul's group specific techniques to get it.
  But, of course, $\T_\alpha$ being
  topologically trivial it could not be an ordinary locally
  trivial fibration. I decided to investigate this question and,
  finally, gave a definition of diffeological fiber
  bundles, which are not locally trivial, but locally
  trivial along the plots ---~the smooth
  parametrizations defining the diffeology. I
  showed two important things for me: the first one was
  that the quotient of a diffeological group by any subgroup 
  is a diffeological fibration, and thus $\T^2 \to \T_\alpha$. 
  The second point was that diffeological fibrations satisfy the exact
  homotopy sequence. I was done, I understood why the
  homotopy of $\T_\alpha$, computed with the techniques
  elaborated by Paul, gave the homotopy of $\T^2$, because
  of the exact homotopy sequence. I spent one year on
  this job, and I returned to Jean-Marie with that
  and some examples. He agreed to listen to me
  and decided that it could be my dissertation. I defended it in
  November 1985, and became since then completely involved in
  the diffeology adventure. 
  
  \alinea{\bf Differential, differentiable or diffeological spaces?} The
  choice of the wording \guillemots{diffe\-ren\-tial spaces}
  or \guillemots{differential groups} was not very happy, because
  \guillemots{differential} is already used in maths and has some
  kind of copyright, especially \guillemots{differential groups}
  which are groups with an operation of derivation. This was
  quoted often to us. I remember Daniel Kastler insisting
  that JMS change this name. From time to time we tried to
  find something else, without success. Finally, it was
  during the defense of Paul's thesis, if memory serves
  me right, when Van Est suggested the word
  \guillemots{diff\'eologie} like \guillemots{topologie} as a replacement
  of \guillemots{diff\'erentiel}. We found the word accurate and
  we decided to use it, and \guillemots{espaces diff\'erentiels}
  became \guillemots{espaces diff\'eologiques}. There was a damper,
  however, \guillemots{diff\'erentiel} as well as
  \guillemots{topologique} have four spoken syllables when
  \guillemots{diff\'eologique} has five. Anyway, I used and abused
  this new denomination, many friends laughed at me, and 
  one of them once told me: your 
  \guillemots{dix f\'ees au logis} --- which means \guillemots{ten fairies
  at home} ---~since then, there is no time when I use
  diffeology without thinking of these ten fairies waiting
  at home... Later, Daniel Bennequin pointed out to me that
  Kuo-Tsai Chen, in his work about {\em Iterated path
  integrals} \cite{Che77}, in the '70s, defined
  \guillemots{differentiable spaces} which looked a lot like
  \guillemots{diffeological spaces}. I got to the library,
  read Chen's paper and drew the conclusion that our
  \guillemots{diffeological spaces} were just equivalent to
  Chen's \guillemots{differentiable spaces}, the slight difference
  in the definition giving, however, the same category. I
  was very disappointed, I was working on a subject I thought
  really new and it appeared to be known and already worked
  out. I decided to drop \guillemots{diffeology} for
  \guillemots{differentiable} and to give honor to Chen, but my
  attempt to use Chen's vocabulary aborted, the word
  \guillemots{diffeology} had already moved into practice, having
  myself helped to popularize it. However, I must admit
  that, although Chen's and Souriau's axiomatics lead to the
  same category, Souriau's choice is better adapted
  to our geometrical point of view. Defining plots on open domains,
  rather than on standard simplices, changes dramatically
  the scope of the theory. 
  
  \alinea{\bf Last word?} I would add some words about the use or misuse of
  diffeology. Some friends have expressed their skepticism
  about diffeology, and told me that they are waiting for
  diffeology to prove something great. Well, I don't know
  any theory proving anything, but I know mathematicians
  proving theorems. Let me put it differently, number theory
  doesn't prove any theorem, mathematicians solve problems
  raised by number theory. A theory is just a framework to
  express questions and pose problems, 
  it is a playground. 
  The solutions of these problems depend on the skill of the
  mathematicians who are interested in them. As a framework to
  formulate questions in differential geometry, I think
  diffeology is a very good one, it offers good tools,
  simple axioms, simple vocabulary, simple but rich
  objects, it is a stable category, and it opens a wide field of
  research. Now, I understand my friends, there are so many
  attempts to extend the usual category of differential
  geometry, and so much expectations, that it is legitimate to be doubtful.
  Nevertheless, I think that we have now enough convincing
  examples, simple or more elaborated, for which
  diffeology brings concrete and formal results. And this
  is an encouragement to persist on this path,
  developing new diffeological tools, and perhaps to prove
  some day, some great theorem :).
  
  \alinea{\em By the time I began this book, Jean-Marie Souriau was alive and well.
  He asked me frequently about my progress. He was eager to know if 
  people were buying his theory and he was happy when I could say sometimes that, 
  yes, some people in Tel-Aviv or in Texas, mentioned it in some papers or discussed it
  on some web forum. 
  Now, when I'm finishing this book and writing the last sentences, 
  Jean-Marie is no longer with us. He will not see the book published and complete.
  It is sad, Diffeology was his last program, in which he had strong expectations
  regarding geometric quantization. I am not sure if diffeology 
  will fulfill his expectations, but I am sure that it is now a mature theory,
  and I dedicate this work to his memory. 
  Whether it is the right framework to achieve Souriau's quantization program 
  is still an open question.}
\end{chaphead}
