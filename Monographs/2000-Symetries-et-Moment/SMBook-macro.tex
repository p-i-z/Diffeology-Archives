%%%%%%%%%%%%%%%%%%%%%%%%%%%%%%%%%%%%%%%%%%%%%%%%%%%%%%%%%%
%%
%% Fichier de style 'symétries et moment'.
%%
%% Version nettoyée et modernisée, Décembre 2025.
%%
%%%%%%%%%%%%%%%%%%%%%%%%%%%%%%%%%%%%%%%%%%%%%%%%%%%%%%%%%%

\typeout{Fichier de macros 'symetries et moment' (version nettoyee)}
\typeout{Version du fevrier 2000, modifiee en Decembre 2025}

% %%%%%%%%%%%%%%%%%%%%%%%%%%%%%%%%%%%%%%%%%%%%%%%%%%%%%%%%%%%%%%%%%%%%%%%%%%%%%%%
%
%% MARK: Required Packages
%
% %%%%%%%%%%%%%%%%%%%%%%%%%%%%%%%%%%%%%%%%%%%%%%%%%%%%%%%%%%%%%%%%%%%%%%%%%%%%%%%%

%\usepackage{amsmath, amssymb}
\usepackage{pifont} % For Zapf Dingbats symbols.
\usepackage{layout} % For the \calclayout command.
\usepackage[width=170mm, height=240mm, cam, noinfo, center]{crop}

\usepackage{graphicx}      % The package that provides \includegraphics
\usepackage{epstopdf}      % Automatically converts .ps files to .pdf for pdflatex

% Pour page de couverture
\usepackage{pdfpages}

% Pour figures et diagrammes
% http://texdoc.net/texmf-dist/doc/latex/tikz-cd/tikz-cd-doc.pdf
\usepackage{tikz-cd}
\usetikzlibrary{calc} % Utile pour des calculs de position avancés
\tikzcdset{
    arrow style=tikz,
    diagrams={>={Straight Barb[scale=0.8]}} % Style de flèche standard et élégant
}

%% --- Font and Typography Packages ---

\usepackage{microtype}
% The mathdesign package provides the Utopia font.
% Note: It redefines many math symbols and may conflict with symbols
% from amssymb if they are used. The current text seems compatible.
\usepackage[cal=scr,uppercase=upright,greeklowercase=upright,expert,utopia]{mathdesign}
\linespread{1.2}

%% --- Hyperlinks ---
% Load hyperref last for best compatibility.
\usepackage[hidelinks]{hyperref}

%% --- Captions ---
% Load caption package
\usepackage[margin=0pt, font=small, labelfont=sc, justification=centering, labelformat=simple]{caption}

\renewcommand{\thefigure}{\roman{figure}}

% %%%%%%%%%%%%%%%%%%%%%%%%%%%%%%%%%%%%%%%%%%%%%%%%%%%%%%%%%%%%%%%%%%%%%%%%%%%%%%%
%
%% MARK: Layout and Page Geometry
%
% %%%%%%%%%%%%%%%%%%%%%%%%%%%%%%%%%%%%%%%%%%%%%%%%%%%%%%%%%%%%%%%%%%%%%%%%%%%%%%%%

\paperheight 240mm
\paperwidth 160mm
\marginparwidth = 0pt

\setlength{\textwidth}{125mm}
\setlength{\textheight}{180mm}
\calclayout

\setlength{\parindent}{0em}
\setlength{\parskip}{1ex}
\linespread{1.1}

% %%%%%%%%%%%%%%%%%%%%%%%%%%%%%%%%%%%%%%%%%%%%%%%%%%%%%%%%%%%%%%%%%%%%%%%%%%%%%%%
%
%% MARK: Theorem and Proof Environments
%
% %%%%%%%%%%%%%%%%%%%%%%%%%%%%%%%%%%%%%%%%%%%%%%%%%%%%%%%%%%%%%%%%%%%%%%%%%%%%%%%%

%% End-of-proof and end-of-note symbols.
\newcommand{\TheEndOfProof}{\rule{1mm}{2.5mm}}
\newcommand{\TheEndOfNote}{\ding{119}} % Replaces plain TeX \font definition.

%% Unnumbered theorem-like environments.
\newtheorem{theoreme}{{\sc Th\'eor\`eme.}}
\renewcommand{\thetheoreme}{}

\newtheorem{lemme}{{\sc Lemme.}}
\renewcommand{\thelemme}{}

\newtheorem{corollaire}{{\sc Corollaire.}}
\renewcommand{\thecorollaire}{}

\newtheorem{proposition}{{\sc Proposition.}}
\renewcommand{\theproposition}{}

\newtheorem{definition}{{\sc D\'efinition.}}
\renewcommand{\thedefinition}{}

\newtheorem{exercice}{{\sc Exercice.}}
\renewcommand{\theexercice}{}

\newtheorem{ddemonstration}{{\sc D\'emonstration}}
\renewcommand{\theddemonstration}{}

%% Proof environment.
%\newenvironment{demonstration}{\n
%    \begin{ddemonstration}\n
%}{\n
%    \nolinebreak \TheEndOfProof \end{ddemonstration}\n
%}\n
\newenvironment{demonstration}{
    \textsc{Démonstration.}
}{
    \nolinebreak \TheEndOfProof
}

%% Note-like environments.
\newenvironment{exemple}{
    \medskip\noindent{\sc Exemple.\ }
}{
    \nolinebreak\TheEndOfNote%\medskip
}

\newenvironment{conclusion}{
    \medskip\noindent{\sc Conclusion.\ }
}{
    \nolinebreak\TheEndOfNote%\medskip
}

\newenvironment{note}{
    \medskip\noindent{\sc Note.\ }
}{
    \nolinebreak\TheEndOfNote%\medskip
}

\newenvironment{remarque}{
    \medskip\noindent{\sc Remarque.}
}{
    \nolinebreak\TheEndOfNote%\medskip
}

\newenvironment{notation}{
    \medskip\noindent{\sc Notations.}
}{
    \nolinebreak\TheEndOfNote%\medskip
}

%% Renumbering chapter
\renewcommand\thechapter{\Roman{chapter}}
\numberwithin{equation}{chapter}

% %%%%%%%%%%%%%%%%%%%%%%%%%%%%%%%%%%%%%%%%%%%%%%%%%%%%%%%%%%%%%%%%%%%%%%%%%%%%%%%
%
%% MARK: Font and Text Macros
%
% %%%%%%%%%%%%%%%%%%%%%%%%%%%%%%%%%%%%%%%%%%%%%%%%%%%%%%%%%%%%%%%%%%%%%%%%%%%%%%%%

%% --- Boldface Abbreviations (used) ---
\renewcommand{\AA}{\mathbf{A}}
\newcommand{\CC}{\mathbf{C}}
\newcommand{\EE}{\mathbf{E}}
\newcommand{\FF}{\mathbf{F}}
\newcommand{\HH}{\mathbf{H}}
\newcommand{\LL}{\mathbf{L}}
\newcommand{\NN}{\mathbf{N}}
\newcommand{\RR}{\mathbf{R}}
\newcommand{\TT}{\mathbf{T}}
\newcommand{\ZZ}{\mathbf{Z}}

\newcommand{\rr}{\mathbf{r}}
\newcommand{\vv}{\mathbf{v}}

%% --- Calligraphic Abbreviations (used) ---
\newcommand{\cD}{\mathcal{D}}
\newcommand{\cG}{\mathcal{G}}
\newcommand{\cH}{\mathcal{H}}
\newcommand{\cK}{\mathcal{K}}
\newcommand{\cM}{\mathcal{M}}
\newcommand{\cT}{\mathcal{T}}
\newcommand{\cV}{\mathcal{V}}

%% --- Typographic Helpers ---
\newcommand{\cad}{c'est-\`a-dire }
\newcommand{\ie}{{\em i.e.}\ }
\newcommand{\cf}{{\em cf.}\ }

% %%%%%%%%%%%%%%%%%%%%%%%%%%%%%%%%%%%%%%%%%%%%%%%%%%%%%%%%%%%%%%%%%%%%%%%%%%%%%%%
%
%% MARK: Mathematical Macros
%
% %%%%%%%%%%%%%%%%%%%%%%%%%%%%%%%%%%%%%%%%%%%%%%%%%%%%%%%%%%%%%%%%%%%%%%%%%%%%%%%%

%% --- Redefined Standard Commands ---
\renewcommand{\emptyset}{\varnothing}

%% --- Brackets, Norms, etc. ---
\newcommand{\abs}[1]{\left\vert #1 \right\vert}
\newcommand{\norme}[1]{\left\Vert #1 \right\Vert}
\newcommand{\norm}[1]{\norme{#1}}
\def\scal(#1,#2){\left\langle #1,#2\right\rangle}
\newcommand{\vect}[1]{\left(\begin{array}{c}#1\end{array}\right)}
\def\mat#1,#2,#3,#4{\left(\begin{array}{cc} #1 & #2 \\ #3 & #4\end{array}\right)}
\newcommand{\mor}[1]{\mathrel{\mathop{\longrightarrow}\limits^{#1}}} 

%% --- Mathematical Operators (using amsmath) ---
\DeclareMathOperator{\ad}{ad}
\DeclareMathOperator{\Ad}{Ad}
\DeclareMathOperator{\aire}{aire}
\DeclareMathOperator{\Arc}{Arc}
\DeclareMathOperator{\Aut}{Aut}
\DeclareMathOperator{\aut}{aut}
\DeclareMathOperator{\but}{but}
\DeclareMathOperator{\cst}{cst}
\DeclareMathOperator{\cl}{cl}
\DeclareMathOperator{\Cinf}{C^\infty}
\DeclareMathOperator{\Diff}{Diff}
\DeclareMathOperator{\Ext}{Ext}
\DeclareMathOperator{\grad}{grad}
\DeclareMathOperator{\GL}{GL}
\DeclareMathOperator{\Ham}{Ham}
\DeclareMathOperator{\ham}{ham}
\DeclareMathOperator{\Hom}{Hom}
\DeclareMathOperator{\im}{im}
\DeclareMathOperator{\Imm}{Imm}
\DeclareMathOperator{\Isom}{Isom}
\DeclareMathOperator{\Lac}{Lac} % \Long is a LaTeX command
\DeclareMathOperator{\Longi}{Long} % \Long is a LaTeX command
\DeclareMathOperator{\Ogroup}{O} % \O is a LaTeX command
\DeclareMathOperator{\Orth}{Orth}
\DeclareMathOperator{\pcirc}{pc}
\DeclareMathOperator{\PSL}{PSL}
\DeclareMathOperator{\slin}{sl} % \sl is a LaTeX command
\DeclareMathOperator{\SL}{SL}
\DeclareMathOperator{\SO}{SO}
\DeclareMathOperator{\so}{so}
\DeclareMathOperator{\surf}{surf}
\DeclareMathOperator{\source}{src}
\DeclareMathOperator{\Sp}{Sp}
\DeclareMathOperator{\supp}{supp}
\DeclareMathOperator{\SU}{SU}
\DeclareMathOperator{\Sympl}{Sympl}
\DeclareMathOperator{\Vol}{Vol}
\DeclareMathOperator{\vol}{vol}

%% --- Miscellaneous Math Macros ---
\newcommand{\DLie}{\text{\pounds}}
\newcommand{\HdR}{H_{\mbox{$\scriptstyle {\rm dR}$}}}
\newcommand{\ZdR}{Z_{\mbox{$\scriptstyle {\rm dR}$}}}
\newcommand{\id}{\mathbf{1}}
\newcommand{\modulo}{{\em modulo}}
\newcommand{\no}{\hbox{n${}^\circ$}}
\newcommand{\tq}{\mathbin|}
\newcommand{\tr}{\mathrm{Tr}} % Using \mathrm for better spacing.
\newcommand{\U}{\mathrm{U}}
\newcommand{\undemi}{\frac{1}{2}} % Modernized from old TeX.
\newcommand{\xo}{x_0} % Modernized from old TeX.
\newcommand{\hs}{\hspace{2mm}}
\newcommand{\module}[1]{\left\vert #1 \right\vert}


% %%%%%%%%%%%%%%%%%%%%%%%%%%%%%%%%%%%%%%%%%%%%%%%%%%%%%%%%%%%%%%%%%%%%%%%%%%%%%%%
%
%% MARK: Complex Macros (Arrays, etc.)
%
% %%%%%%%%%%%%%%%%%%%%%%%%%%%%%%%%%%%%%%%%%%%%%%%%%%%%%%%%%%%%%%%%%%%%%%%%%%%%%%%%

\newcommand{\map}[5]{
    \begin{array}{rrccc}
        #1 : & #2 & \longrightarrow & #3 \\
             & #4 & \longmapsto     & #5
    \end{array}
}

\newcommand {\equaldef}{\mathrel{\mathop{\kern 0pt =}^{\text{\scriptsize def}}}}

%% --- Document-specific Settings ---
% Allows page breaks within long multi-line equations (e.g., align).
\allowdisplaybreaks
