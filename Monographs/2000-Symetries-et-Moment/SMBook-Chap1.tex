%%====================================================================
% MARK: - Chapter 1: Lagrange et les origines du calcul symplectique
%%====================================================================

\chapter{Lagrange et les origines du calcul symplectique}

Entre 1808 et 1811,
Lagrange développe une {\em théorie de la variation des constantes} appliquée aux problèmes de la mécanique \cite{Lagrange2,Lagrange3,Lagrange4},
une application particulière de la {\em théorie générale de la variation des constantes} qu'il a introduite en 1775 \cite{Lagrange6}.
C'est l'acte de naissance du calcul {\em symplectique}\footnote{Voir l'annexe \ref{geomsymp} pour une introduction à la géométrie symplectique.},
terme qui ne sera inventé qu'en 1946 par Hermann Weyl\footnote{Voir la note historique à la fin de ce chapitre.} \cite{Weyl1}.

Le but que poursuit alors Lagrange est la généralisation d'un théorème de Laplace sur la stabilité séculaire du grand axe de l'orbite elliptique d'une planète,
perturbée par l'attraction d'autres corps célestes.
Depuis Kepler on sait {\em résoudre} explicitement le problème des éphémérides des planètes\index{éphémérides},
c'est-à-dire,
calculer avec une précision aussi grande que l'on veut la position de la terre (ou de toute autre planète) connaissant sa position et sa vitesse à un instant donné,
à condition toutefois de seulement considérer l'attraction du Soleil et de négliger l'influence des autres planètes.
Mais bien que ce savoir soit important,
il est largement insuffisant pour ce qui est du mouvement réel des planètes.
L'influence des autres planètes sur la Terre est-elle vraiment négligeable,
et ne va-t-elle pas à terme déstabiliser sa trajectoire et l'expulser aux confins de l'espace?

Il faut donc traiter le problème dans sa globalité:
calculer la position d'une planète quelconque,
connaissant les positions et vitesses de toutes les planètes,
et en ne négligeant l'influence d'aucune d'entre elles.
La difficulté de cette question donne le vertige,
et on ne sait y répondre,
encore actuellement,
ni analytiquement ni même numériquement.
On pourrait croire,
en effet,
qu'avec l'avènement de l'ordinateur cette question soit devenue académique:
pourquoi ne pas intégrer naïvement les équations du mouvement par une méthode numérique quelconque?
Malheureusement,
si les erreurs d'approximations,
inévitables dans ce genre de calcul,
sont négligeables sur un bref intervalle de temps,
elles deviennent catastrophiques à long terme.
Cette incertitude sur la position de la planète n'a rien à voir avec une éventuelle situation chaotique du système (le système à deux corps est d'ailleurs parfaitement {\em intégrable\index{système intégrable}} dans tous les sens raisonnables que l'on veut bien donner à ce mot),
elle est simplement la conséquence de l'accumulation des erreurs commises par l'ordinateur lors de l'intégration numérique des équations du mouvement.

L'existence d'une méthode analytique d'intégration du mouvement est donc capitale pour résoudre convenablement cette question.
Si cette remarque est vraie pour le problème à deux corps,
elle l'est {\em a fortiori} pour le problème à $n$ corps (\ie un nombre quelconque de planètes en interactions).
Or,
comme nous l'avons déjà dit,
nous ne connaissons toujours aucune méthode analytique satisfaisante susceptible de résoudre cette question.
Lagrange a contourné cette difficulté en appliquant de façon astucieuse sa méthode de {\em la variation des constantes} aux problèmes de la mécanique analytique.
Décrivons rapidement ce dont il s'agit.

{\bf L'espace des mouvements kepleriens:}\index{mouvements kepleriens}
considérons d'abord un corps matériel (une planète) attiré par un centre fixe (le Soleil) selon la loi de la gravitation universelle.
Les équations différentielles qui décrivent son mouvement sont d'ordre deux dans l'espace à trois dimensions:
il faudra donc six {\em constantes d'intégration}\footnote{À l'époque de Lagrange on utilisait ce terme de {\em constantes d'intégration};
nous parlons aujourd'hui d'{\em espace de solutions}\index{constantes d'intégration}.
Par exemple,
l'équation différentielle ordinaire réelle $dx/dt=x$ a toutes ses solutions de la forme $x(t)=c\exp(t)$,
où $c$ est une constante arbitraire --- la fameuse constante d'intégration.
Or $c$ caractérise justement cette solution.} pour le décrire.
D'après Newton,
nous savons que la trajectoire de ce corps est une ellipse\index{ellipse}\footnote{Si Kepler a découvert le mouvement elliptique des planètes,
c'est Newton qui l'a «déduit» de la loi de la gravitation universelle qui porte son nom.
Pour une discussion plus approfondie sur ce sujet voir la thèse de François de Gandt \cite{deGandt1}.},
dont le foyer \index{foyer} est le centre d'attraction\footnote{Les caractéristiques géométriques de cette ellipse étant,
par ailleurs,
liées aux position et vitesse initiales du corps.}.
Pour décrire complètement cette ellipse,
il nous faut d'abord connaître le plan dans lequel elle s'inscrit (le plan de l'orbite):
on peut le repérer par le vecteur unitaire qui lui est orthogonal,
ce qui fait deux paramètres.
Pour définir l'ellipse dans son plan on peut choisir la position du deuxième foyer,
ce qui donne deux nouveaux paramètres,
et la longueur de l'ellipse,
soit au total cinq paramètres pour situer et décrire la trajectoire du corps dans l'espace.

Connaissant ces cinq paramètres nous pouvons tracer alors l'ellipse par la {\em méthode du jardinier}:
plantons dans le plan de l'orbite deux piquets,
un à chacun des foyers,
entourons ces deux piquets d'une corde de la longueur donnée,
puis à l'aide d'un bâton tendons cette corde et faisons lui parcourir (corde tendue) un tour complet;
la figure ainsi obtenue est l'ellipse en question.
Mais si ces cinq paramètres suffisent à définir complètement la trajectoire du corps céleste,
ils ne suffisent pas à déterminer son {\em mouvement}.
En effet,
comment déterminer la position de la planète à chaque instant sur sa trajectoire si nous ne connaissons pas sa position à une origine des temps arbitraire?
Ou encore la date de son passage\footnote{C'est un peu comme si nous connaissions le trajet d'un autobus sans connaitre l'heure de ses passages aux arrêts,
cela ne nous serait de peu d'utilité.} à l'aphélie?
C'est le sixième paramètre qu'il faut introduire et que les astronomes appellent l'{\em époque}\index{époque}.

Nous aurions pu tout aussi bien choisir six autres paramètres:
par exemple les position et vitesse initiales de la planète à l'origine des temps.
Ils définissent aussi,
de façon unique,
le mouvement de la planète.
Seul le caractère pratique de tel ou tel ensemble de paramètres peut influencer notre choix.
Les astronomes appellent {\em éléments kepleriens}\index{éléments kepleriens} de la planète un ensemble de six paramètres caractérisant son mouvement,
cinq pour la figure de l'ellipse et un autre pour la loi horaire:
l'époque.
L'ensemble des mouvements de la planète considérés indépendamment du choix des paramètres qui nous servent à les décrire\footnote{Un ensemble muni de différentes familles de paramètres servant à repérer ses éléments est appelée une {\em variété}.
Par son utilisation des constantes d'intégration,
pour repérer les différents mouvements de la planète,
Lagrange traite de façon moderne l'ensemble des mouvements de la planète comme une variété.} sera appelé {\em espace des mouvements kepleriens}\index{espace des mouvements}.
Un point de cet espace,
que nous noterons $\cK$ par la suite,
représente donc un {\em mouvement keplerien}\index{mouvements kepleriens} dans son intégralité:
sa trajectoire\index{trajectoire} et sa loi horaire.
La façon de le repérer par six constantes est une question de commodité qui peut être dictée par les conditions particulières du problème que l'on étudie.

%% TODO: The user needs to provide the modern image file for this figure.
\begin{figure}[ht]
  \fbox{\includegraphics{figures/fig-choc.pdf}}
  \caption{Méthode de la variation des constantes}
  \label{perturbation}
 \end{figure}

\label{page_mvc}
{\bf La méthode de la variation des constantes:}\index{variation des constantes}
établissons maintenant la méthode de la variation des constantes telle qu'elle a été introduite par Lagrange.
Lorsque la planète suit un mouvement keplerien,
son état est complètement caractérisé par les 6 éléments kepleriens de son orbite\index{orbite} qui définissent,
nous l'avons dit,
à la fois la figure de l'ellipse et sa loi horaire.
Ce mouvement est un point $m$ de l'espace $\cK$.
Supposons maintenant que la planète,
qui suit le mouvement keplerien $m$,
subisse un choc instantané dû à l'impact d'un astéroïde.
Après le choc,
elle suivra encore un mouvement keplerien $m'$ différent du précédent.
C'est une autre ellipse\footnote{Si le choc n'a pas été trop violent!} parcourue selon une nouvelle loi horaire.
Le mouvement (perturbé) de cette planète est donc décrit par son mouvement $m$ avant le choc,
son mouvement $m'$ après le choc et l'instant $t$ du choc.

Supposons ensuite que la planète subisse une série de chocs de ce type.
Le mouvement réel de la planète est donc décrit par une courbe dans l'espace des mouvements kepleriens,
discontinue et constante par morceaux,
chaque morceau de courbe décrivant le mouvement keplerien de la planète entre deux chocs successifs.
En étendant ce raisonnement,
Lagrange assimile l'interaction des autres planètes du système à une série infinie de chocs «infiniment petits et continuels».
Il décrit ainsi le mouvement réel de la planète perturbée par une courbe,
cette fois différentiable,
tracée dans son espace des mouvements kepleriens.
C'est ce que représente,
de façon rudimentaire,
la figure précédente (fig. \ref{perturbation}).
C'est en précisant l'équation différentielle de cette courbe\footnote{Cette équation est connue aujourd'hui sous le nom d'{\em équation de Hamilton},
mais Sir W.R. Hamilton n'avait que six ans lorsque Lagrange la publia pour la première fois.} qu'il a fait apparaître la {\em structure symplectique}\index{symplectique} de l'{\em espace des mouvements}.
Il a donné l'expression des composantes de la forme symplectique de l'espace des mouvements kepleriens dans le système de coordonnées que sont les {\em éléments de la planète}.
Il en a déduit entre autre la stabilité séculaire\index{stabilité séculaire} du grand axe des planètes.

J'ai essayé,
dans ce chapitre,
d'être le plus fidèle possible aux textes de Lagrange,
désirant par là mettre en évidence le processus qui lui a permis,
en voulant résoudre le problème du système des planètes,
d'élaborer les premiers éléments de calcul symplectique.

\begin{note}
  {\sc(Parenthèses et crochets de Lagrange)}
  C'est le 22 ao\^ut 1808 que Lagrange présente à l'Institut de France son {\em Mémoire sur la théorie des variations des éléments des planètes} \cite{Lagrange2} où apparaissent pour la première fois ce que l'on nomme aujourd'hui\index{parenthèses de Lagrange} les {\em parenthèses de Lagrange},
  qui sont en termes modernes les composantes covariantes de la {\em forme symplectique} naturelle,
  de l'espace des mouvements d'une planète.
  Elles apparaissent comme les coefficients des variations des éléments de la planète dans l'expression de la force de perturbation,
  rapportée aux éléments kepleriens.
  Ce mémoire a été suivi de celui {\em Sur la théorie générale de la variation des constantes arbitraires} \cite{Lagrange3} présenté le 13 mars 1809,
  où Lagrange généralise sa méthode à tous les problèmes de mécanique.
  Ce mémoire contient en annexe une version notablement simplifiée,
  et définitive,
  de ses calculs.
  Le 19 février 1810,
  il publie un texte \cite{Lagrange4} contenant les formules explicites d'inversion des {\em parenthèses} introduites dans son premier texte de 1808\footnote{Il y a,
  à ce propos,
  une interaction entre Lagrange et Poisson.
  Elle est commentée dans l'annexe \ref{AnnLP}.}.
  Ces coefficients apparaissent sous forme de crochets,
  les {\em crochets de Lagrange},
  qui sont les composantes contravariantes de la forme symplectique naturelle de l'espace des éléments kepleriens de la planète.
  Ils apparaissent comme les coefficients de la force de perturbation,
  rapportée aux éléments de la planète,
  dans l'expression de la variation des éléments kepleriens.
  C'est à partir de ces textes que Lagrange a écrit les chapitres relatifs à la dynamique dans la deuxième édition\footnote{Dans la seconde partie,
  de la cinquième à la septième section} de son traité de {\em Mécanique Analytique} \cite{Lagrange1},
  publié en 1815,
  après sa mort\footnote{Entretemps,
  sous l'influence d'un échange avec Poisson,
  les {\em parenthèses} sont devenues des {\em crochets} et vice-versa,
  ce qui rend la lecture comparée de ces textes malaisée.}.
\end{note}

\begin{note}
  {\sc(Le groupe symplectique)}
  Dans son ouvrage sur les groupes classiques \cite{Weyl1},
  Hermann Weyl baptise {\em groupe symplectique}\index{groupe symplectique} le groupe des transformations linéaires de $\RR^{2n}$ qui préservent la forme bilinéaire antisymétrique canonique\footnote{Soient $X=(u,v)$ et $X'=(u',v')$ deux vecteurs de $\RR^{2n}=\RR^n \times\RR^n$,
  la forme symplectique\index{forme symplectique} canonique $\omega$ est définie par
  $$
    \omega(X,X')=u.v'-u'.v,
    $$
  où le point désigne le produit scalaire;
  autrement dit:
  $$
    \omega(X,X')=\sum_{i=1}^n u_iv'_i-u'_i.v_i
    $$
  Le groupe symplectique est le groupe des transformation linéaires $A$ de $\RR^{2n}$ tel que
  $$
    \omega(AX,AX')=\omega(X,X')
    $$
  pour tout $X\in\RR^{2n}$.
  Pour une présentation plus générale de le géométrie symplectique voir l'annexe \ref{geomsymp}.}.
  Les relations étroites entre la structure définie par $\omega$ et la structure complexe ($\RR^{2n}\sim \CC^n$) lui fait choisir le mot {\em symplectique} [du grec {\em sum-plektikos}],
  transposition de {\em complexe} [du latin {\em com-plexus}] pour désigner ce groupe,
  le mot {\em complexe} étant par ailleurs réservé.
  $$
    \begin{array}{rccc}
    \mbox{symplectique :} & \mbox{sum} & - & \mbox{plecticos} \\
    & \updownarrow & & \updownarrow \\
    \mbox{complexe :}     & \mbox{com} & - & \mbox{plexus} \\
    \end{array}
    $$
  Le suffixe {\em plekticos} $\sim$ {\em plexus} signifiant {\em tenir},
  {\em entrelacer}\ldots l'idée de {\em complexe},
  comme {\em symplectique},
  sous-entend l'existence de plusieurs types d'objets (ici deux) maintenus ensemble dans une même structure.
  De façon rapide et en anticipant sur la suite,
  on peut dire que la {\em complexité} représente la dualité {\em réel--imaginaire},
  et la {\em symplecticité} la dualité {\em position--vitesse}.
  Voilà ce qu'en dit lui même Weyl \cite[p. 165]{Weyl1}:
  \begin{quote}
    The name ``complex group'' formerly advocated by me in allusion to line complexes,
    as these are defined by the vanishing of antisymetric bilinear forms,
    has become more and more embarrassing through collision with the word ``complex'' in the connotation of complex number.
    I therefore propose to replace it by the corresponding Greek adjective ``symplectic''.
    Dickson calls the group the ``Abelian linear group'' in homage to Abel who first studied it.
  \end{quote}
  En ce qui concerne la notion actuelle de {\em géométrie symplectique},
  au sens de la théorie des variétés différentielles munies d'une forme symplectique,
  il semble que ce soit J.-M. Souriau qui l'ait introduite en 1953 dans son article {\em Géométrie symplectique différentielle, applications} \cite{Souriau20}.
  Dans un article plus récent {\em La structure symplectique de la mécanique décrite par Lagrange en 1811} \cite{Souriau9} le même auteur décrit un autre aspect des relations entre la géométrie symplectique et la mécanique de Lagrange.
\end{note}

%%====================================================================
\section{Espace des mouvements d'une planète}
%%====================================================================

Pour comprendre et apprécier la méthode de la variation des constantes développée par Lagrange,
il est nécessaire de bien connaître la résolution du problème à deux corps\index{problème à deux corps}.
Nous allons en donner un bref résumé dans ce qui suit.

%% TODO: The user needs to provide the modern image file for this figure.
% \begin{figure}[ht]\n
% \centerline{\includegraphics{figures/Newton.pdf}}\n
% \caption{Isaac Newton}\label{Newton}\n
% \end{figure}\n

Depuis Newton,
on sait que les mouvements d'un point matériel (une planète) autour d'un centre fixe (le Soleil) sont décrits par l'équation différentielle\footnote{Il faudrait en toute rigueur multiplier $\rr$ par la constante d'attraction solaire,
mais nous choisirons les unités de telle sorte qu'elle soit égale à 1.} suivante:
\begin{equation}
  \label{eqNewton1}
  {d^2\rr \over dt^2} = - {\rr \over r^3},
\end{equation}
où $\rr$ désigne un vecteur non nul de l'espace $\RR^3$ et $r$ son module.
Même si cela tient plus à l'histoire des sciences qu'à la science elle-même,
il est intéressant de savoir comment Newton est arrivé à cette équation.
Le problème posé à l'époque de Newton consistait à prouver que le mouvement des planètes était dû à des forces centripètes qui {\em décroissent comme le carré de la distance}.
A ce moment du développement des sciences,
le terme de {\em force}\index{force} était pris au sens d'{\em effort},
comme l'effort que déploie un cheval pour tirer une charrette.
Pour résoudre cette question,
Newton devait d'abord préciser le sens qu'il donnait au mot {\em force},
ce qu'il fit sous la forme de principes qu'il énonça d'abord dans son {\em De Motu} \cite{Newton0,deGandt1}:
\begin{itemize}
  \item Laissées à elles-mêmes les planètes suivraient un mouvement rectiligne uniforme (le principe de l'inertie\index{inertie}).
  \item L'incurvation de leur trajectoire est due à une force extérieure qui les dirige vers le Soleil.
  \item Pour évaluer la force,
  il faut mesurer l'incurvation,
  \cad la différence entre la trajectoire rectiligne virtuelle et la trajectoire incurvée réelle.
\end{itemize}
Ce troisième principe deviendra,
dans ses {\em Principia Mathematica} \cite{Newton1,Newton2},
la deuxième loi de la mécanique\index{mécanique} sous la forme suivante:
\begin{itemize}
  \item Le changement de mouvement est proportionnel à la force motrice imprimée.
\end{itemize}
Newton admet les trois lois de Kepler\index{lois de Kepler}:
\begin{itemize}
  \item Les orbites des planètes sont des ellipses avec le Soleil pour foyer\index{foyer}.
  \item Les aires\index{loi des aires} balayées par le rayon joignant le Soleil à la planète sont proportionnelles au temps de parcours.
  \item La période du mouvement de la planète autour du Soleil est proportionnelle à la racine carrée du cube du grand axe.
\end{itemize}

\begin{figure}[ht]
  \centerline{\includegraphics{figures/fig-ellipse.pdf}}
  \caption{Le mouvement elliptique d'une planète}
  \label{ellipseFig}
\end{figure}

Supposons que la planète se trouve en un point $P$ de l'ellipse à un instant donné et au point $Q$ après un temps $t$.
Si la planète avait été libre de toute interaction elle aurait suivi la tangente à l'ellipse passant par $P$,
mais puisque sa trajectoire a été incurvée vers le Soleil,
cela signifie qu'il lui a imprimé une {\em force centripète\index{force centripète}}:
dirigée vers le centre.
Cette force doit être mesurée par la déflexion de la planète,
\cad par la distance dont elle est {\em tombée} sur le Soleil,
autrement dit par la longueur du segment $PR$,
où $R$ est la projection sur le segment $SP$,
parallèlement à la tangente en $P$,
du point $Q$ (fig. \ref{ellipseFig}).
Newton suppose alors que la planète tombe vraiment sur le Soleil,
comme un corps pesant\footnote{Il n'est peut-être pas inutile d'insister sur cet aspect du raisonnement de Newton,
puisque c'est là le caractère {\em universel} de sa loi de la gravitation:
{\em comme les pommes, les planètes tombent}.}.
Il peut donc appliquer la loi de la chute des corps connue depuis Galilée:
{\em la proportionnalité de la hauteur de chute avec le carré du temps de la chute},
\cad:
\begin{equation}
  PR = \undemi g t^2,
\end{equation}
où $g$ mesure la {\em pesanteur} au point $P$.
C'est ce nombre qu'il faut calculer pour connaître l'intensité de la force d'attraction du Soleil.
Le temps $t$ est,
d'après la deuxième loi de Kepler,
proportionnel à l'aire du secteur balayé.
Lorsque le temps $t$ est petit,
cette aire est à peu près égale à celle du triangle $SQP$:
\begin{equation}
  t \sim \mbox{aire}(SPQ) = c {SP\times QT\over 2},
\end{equation}
où $c$ est une constante de proportionnalité dépendant {\em a priori} de l'ellipse.
La valeur de cette constante $c$ s'obtient en appliquant la deuxième loi de Kepler à une orbite complète,
et en utilisant la troisième loi de Kepler,
\ie «la période $\tau$ de la planète est proportionnelle à la racine carrée du cube du grand axe».
Comme lors d'une révolution complète de la planète l'aire balayée est l'aire de l'ellipse toute entière,
nous avons donc les deux égalités suivantes\index{lois de Kepler}
$$
  \left\{
  \begin{array}{llll}
  \tau & = & c\times\mbox{aire de l'ellipse} = c\times \pi a b, & \mbox{(deuxième loi de Kepler)} \\
  &   & \mbox{et}                                        &                                 \\
  \tau & = & k \sqrt{a^3}                                     & \mbox{(troisième loi de Kepler)} \\
  \end{array}
  \right.
  $$
où $a$ est le demi grand axe de l'ellipse et $b$ l'autre demi grand axe,
$k$ étant une constante dépendant des planètes et des unités de mesure.
L'aire des ellipses étant par ailleurs connue par la formule $\mbox{aire}=\pi a b$.
Ces deux dernières expressions de $\tau$ nous permettent d'obtenir une valeur approchée de la constante de gravitation cherchée,
acceptable pour les temps petits:
\begin{equation}
  g \sim k' \times {b^2\over a} \times {PR\over SP^2\times QT^2}.
\end{equation}
où $k'$ est une nouvelle constante intermédiaire.
Mais cette approximation ne donne la valeur exacte de $g$ qu'à la limite où les points $Q$ et $P$ se confondent,
ce que nous notons $Q\to P$.
Auparavant,
il est nécessaire d'établir un résultat intermédiaire tiré de la géométrie des ellipses dont nous épargnons le calcul au lecteur:
\begin{equation}
  \lim_{Q\to P} {PR \over QT^2} = {a\over 2 b^2},
\end{equation}
qui permet de conclure:
\begin{equation}
  g= {K\over SP^2},
\end{equation}
où $K$ est une constante ne dépendant que des masses des planètes considérées.
Le coefficient de gravitation $g$,
l'accélération de la planète,
ne dépend donc pas des paramètres de l'ellipse,
mais seulement du lieu géométrique de la planète et des masses en présence.
Cette propriété remarquable mise en évidence par Newton peut s'exprimer ainsi:
\begin{quote}
  «La force centripète exercée par le Soleil sur la planète est inversement proportionnelle au carré de la distance de la planète au Soleil».
\end{quote}
\noindent C'est à partir de ce résultat que Newton a précisé la notion de force comme variation de la quantité de mouvement et qu'il a établi ses équations générales de la mécanique que nous connaissons aujourd'hui\footnote{Ces équations sont souvent présentées de façon obscure dans les manuels de l'enseignement secondaire.
L'élève ne sait pas toujours ce qui est donné:
l'expression de la force $F$ comme une fonction $F(t,\rr,\vv)$ de certaines variables indépendantes,
le temps $t$ la position $\rr$ et la vitesse $\vv$,
et ce qu'il faut rechercher:
la loi horaire,
ou plutôt les lois horaires,
$t\mapsto \rr(t)$ telles qu'à chaque instant $t$ l'équation différentielle $F(t,\rr(t),\vv(t))=d^2\rr(t)/dt^2$ soit satisfaite.
Parfois même l'absence de réflexion sur le choix de la fonction $F$ obscurcit encore davantage le discours.} sous la forme condensée\index{équation de Newton}
$$
  F=m\gamma.
  $$
\begin{remarque}
  Certains historiens des sciences (comme François de Gandt par exemple \cite{deGandt1}) pensent que Newton n'a pas vraiment répondu au problème qui avait été posé à la communauté scientifique de l'époque,
  en particulier par Sir Christopher Wren,
  de la {\em Royal Society}.
  Ce problème,
  que l'on appelle {\em problème direct} peut être énoncé comme ceci:
  «Montrer que les seules trajectoires possibles d'un corps soumis à une attraction inversement proportionnelle au carré de la distance sont les sections coniques ayant le centre attracteur pour foyer».
  Nous avons pu vérifier qu'en utilisant les trois lois de Kepler,
  Newton déduit effectivement que la planète est attiré par le Soleil selon une loi en inverse du carré de la distance qui les séparent.
  Ce qui est plutôt l'énoncé réciproque.
  Mais il ne faut pas oublier que ces mots:
  «force»,
  «attraction»\ldots n'ont pas de sens à l'époque,
  à peine un sens commun,
  vague,
  mais pas le sens précis,
  mathématique,
  opératoire que justement Newton leur a donné.
  En d'autres termes,
  c'est Newton lui-même qui pose la vraie question et qui y répond.
  C'est là sa contribution essentielle à la fondation de la mécanique moderne,
  d'avoir introduit les notions opératoires de force,
  d'attraction etc...
  sans lesquelles les questions évoquées plus haut sont infondées.
  Toujours d'après F. de Gandt \cite{deGandt1},
  il semble que ce soit Jean Bernoulli qui,
  trente and plus tard,
  ait réellement résolu le problème direct.
  Mais
  \begin{quote}
    Ces critiques et ces preuves nouvelles font cependant apparaître le mérite de Newton et l'apport irremplaçable des {\em Principia}.
    En effet si Jean Bernouilli corrige et améliore Newton,
    c'est en s'appuyant sur les résultats et les procédés des {\em Principia}.
  \end{quote}
  c'est ce qu'écrit justement F. de Gandt dans sa thèse \cite{deGandt1}.
\end{remarque}

Démontrons le théorème de Newton par les méthodes modernes du calcul linéaire.
Auparavant,
fixons les quelques notations que nous utiliserons par la suite.
Nous noterons par des lettres grasses $\rr,\vv$\ldots les vecteurs,
\cad les colonnes (ou parfois lignes lorsqu'il s'agit de co-vecteurs) de nombres réels:
$$
  \rr=\vect{r_1\\ r_2\\ r_3}\quad\vv=\vect{v_1\\ v_2 \\ v_3}\quad\ldots
  $$
Les composantes des vecteurs seront notées normalement,
par des lettres indexées $r_i$, $v_i$ \ldots.
Nous utiliserons la même lettre qui désigne un vecteur mais non grasse et sans indice pour désigner sa norme (par exemple $r$ désigne la norme de $\rr$).

Transformons maintenant l'équation différentielle de Newton (\ref{eqNewton1}) en un système du premier ordre dans $[\RR^3-\{0\}]\times\RR^3$;
les {\em mouvements} de la planète deviennent les solutions de
\begin{equation}
  {d\rr\over dt}= \vv, \qquad {d\vv\over dt}= - {\rr\over r^3}, \qquad r=\norm{r}.
\end{equation}
Comme on le sait [lire aussi l'annexe \ref{AnnH}],
l'énergie totale du système est conservée le long du mouvement\footnote{Comme on peut le lire dans l'annexe \ref{AnnH},
l'énergie \index{énergie} totale d'un point matériel dans un champ de force est la somme de son énergie cinétique $mv^2/ 2$,
où $v^2$ est le carré de la norme du vecteur vitesse $\vv$ et $m$ la masse du corps,
et de l'énergie potentielle $U$ dont dérive la force $\FF$ exercée sur le corps,
\cad $F=(F_i)_{i=1}^3$ et $F_i= \partial U / \partial r_i$.
L'énergie totale notée en général $E$ est donc définie par $E= mv^2/2 +U$.
La vertu essentielle de cette fonction du temps de la position et de la vitesse est d'être préservée le long du mouvement.
On dit que c'est une {\em constante du mouvement}.
Un des objectifs de ce livre est de montrer,
entre autre,
pourquoi la conservation de cette fonction,
le long des mouvements,
est la conséquence d'une symétrie:
le décalage du temps.}.
Les astronomes appellent {\em constante des forces vives} le double de l'énergie\index{forces vives} que l'on la notera\footnote{Suivant les notations de Lagrange.} $f$:
\begin{equation}
  \label{fv}
  f = {v^2} - {2 \over r}.
\end{equation}
D'autre part,
comme la force d'attraction gravitationnelle est centrale,
le moment cinétique\index{moment cinétique} $\LL$ est lui aussi conservé\footnote{Le signe $\wedge$ représente le produit vectoriel ordinaire de $\RR^3$,
$(\rr\wedge\vv)_1=r_2v_3-r_3v_2$\ldots}:
\begin{equation}
  \LL=\rr\wedge \vv.
\end{equation}
De cette invariance on déduit que le mouvement de la planète s'effectue dans le plan orthogonal à $\LL$.

%% TODO: The user needs to provide the modern image file for this figure.
 \begin{figure}[t]
  \centerline{\includegraphics{figures/fig-trajectoire.pdf}}
  \caption{L'orbite de la planète P}
  \label{FigOrbite}
 \end{figure}

Un autre vecteur est miraculeusement conservé\footnote{Contrairement aux autres constantes du mouvement ce vecteur n'est associé à aucun symétrie évidente (\cad spatio-temporelle).
Il est lié à une symétrie du groupe $\SO(4)$ qui n'apparait que lors de la régularisation du problème à deux corps,
voir par exemple \cite{Souriau8}.} le long du mouvement,
on peut le vérifier directement;
c'est {\em le vecteur de Laplace}\index{vecteur de Laplace}:
\begin{equation}
  \EE=\LL\wedge \vv +{\rr\over r}.
\end{equation}
On déduit de cet invariant supplémentaire les trajectoires des planètes.
En effet,
on a immédiatement,
en prenant la norme carrée du vecteur $\EE$ et le produit scalaire de $\EE$ par $\LL$:
\begin{equation}
  \label{EqEL}
  E^2=1+fL^2 \quad \mbox{et} \quad \EE.\LL=0.
\end{equation}
Le vecteur $\EE$ est donc perpendiculaire au moment cinétique à tout instant,
et se situe donc dans le plan du mouvement.
En prenant le produit scalaire de $\EE$ par $\rr$ on obtient l'équation suivante,
vérifiée le long du mouvement:
\begin{equation}
  \EE.\rr+L^2=r.
\end{equation}
Soit $\phi$ l'angle entre $\EE$ et $\rr$ (voir figure \ref{FigAExc}),
de sorte que:
$$
  \EE\cdot\rr=Er\cos\phi
  $$
on obtient alors:
\begin{equation}
  E r\cos\phi+L^2=r \quad \mbox{ou encore} \quad r={L^2\over1-E\cos\phi}.
\end{equation}
On reconnait ainsi l'équation,
exprimée en coordonnées polaires $(r,\phi)$,
d'une conique de paramètre $L^2$,
d'excentricité $E$ et d'axe la direction du vecteur $\EE$.
Les astronomes appellent l'angle $\phi$ l'{\em anomalie vraie}\index{anomalie vraie}\footnote{Dans ce contexte,
le terme {\em anomalie} signifie simplement {\em paramètre}.}.
Le vecteur $\EE$ pourrait s'appeler le {\em vecteur d'excentricité}\index{excentricité}.
Les trajectoires de la planète sont donc des sections coniques\index{section conique},
avec le Soleil pour foyer\index{foyer}.
Leur nature dépend essentiellement du signe de l'énergie totale,
comme le montre la formule\footnote{Le lecteur peut vérifier que les trajectoires elliptiques correspondent à une excentricité $E<1$,
la norme de $\rr$ étant alors encadrée par $L^2/(1+E)$ et $L^2/(1-E)$.
Lorsque $E=1$,
le rayon $\rr$ est infini pour $\phi=0$ ou $2\pi$,
mais la vitesse du point est alors nulle,
comme on peut le vérifier par le calcul:
l'orbite parabolique est en quelque sorte {\em tangente à l'infini}.
Dans le troisième cas $E>1$,
l'infini est atteint pour $\phi = \pm\arccos(1/E)$ et la vitesse du point à l'infini n'est pas nulle.} (\ref{EqEL}),
les trois cas possibles sont:
\begin{itemize}
  \item $f<0 \Rightarrow E<1$,
  l'excentricité est inférieure à 1:
  l'orbite est elliptique\index{orbite elliptique}.
  \item $f=0 \Rightarrow E=1$,
  l'excentricité vaut 1:
  l'orbite est parabolique\index{orbite parabolique}.
  \item $f>0 \Rightarrow E>1$,
  l'excentricité est supérieure à 1:
  l'orbite est hyperbolique\index{orbite hyperbolique}.
\end{itemize}
Dans le cas des orbites elliptiques,
on trouve tout de suite la valeur du demi-grand axe,
noté $a$.
En effet,
les points de l'ellipse les plus éloignés du foyer sont atteints pour $\phi=0$ et $\phi=\pi$ ce qui donne les deux valeurs de $r$:
$r_0=L^2/(1+E)$ et $r_\pi=L^2/(1-E)$;
la somme de ces deux valeurs est égale au grand axe,
donc $2a =L^2/(1+E) +L^2/(1-E)$,
d'où on déduit que $a=L^2/(1-E^2)$,
ce qui donne grâce à l'équation (\ref{EqEL}):
\begin{equation}
  a = -{1\over f}.
\end{equation}
Nous pouvons décrire complètement la variété des mouvements kepleriens elliptiques ($f<0$) si l'on exclut les chutes sur le centre,
\cad si on se restreint à $\LL\neq 0$.
Une trajectoire elliptique est bien définie par les deux vecteurs $\LL$ et $\EE$,
le vecteur $\EE$ donnant à la fois l'excentricité et l'axe de la conique,
le plan étant défini comme l'orthogonal de $\LL$ et le paramètre de l'ellipse valant $L^2$.
Autrement dit,
l'espace des trajectoires kepleriennes elliptiques est équivalent à l'ensemble des couples de vecteurs $(\EE,\LL)\in \RR^3\times \RR^3$ tels que:
\begin{equation}
  E<1, \quad L\neq 0 \quad \mbox{et} \quad \EE.\LL=0.
\end{equation}
C'est une sous-variété,
de dimension 5,
de $\RR^3\times\RR^3$.
Ce n'est pas encore l'espace des mouvements kepleriens elliptiques\index{mouvements kepleriens}:
il nous faut pouvoir calculer la position de la planète à chaque instant.
On pourrait,
pour cela,
choisir la position de la planète sur son orbite (\cad l'anomalie vraie) à l'{\em instant zéro}.
Mais ce choix donne lieu à des calculs pénibles qui ne seront pas faits ici.
On considère plutôt le vecteur qui joint l'origine du cercle circonscrit à l'ellipse au point $\AA$ de ce cercle qui a la même projection orthogonale,
sur l'axe dirigé par $\EE$,
que la planète $P$ (voir figure \ref{FigAExc}).

%% TODO: The user needs to provide the modern image file for this figure.
 \begin{figure}[thbp]
  \centerline{\includegraphics{figures/fig-AExc.pdf}}
  \caption{L'anomalie excentrique}
  \label{FigAExc}
 \end{figure}

Nous obtenons ainsi une description géométrique de l'espace des mouvements kepleriens (ou plutôt d'un ouvert de l'espace des mouvements kepleriens,
puisque nous avons supprimé l'ensemble des chutes sur le foyer\index{foyer}\footnote{Le problème des chutes sur le foyer donne lieu à la {\em régularisation} du problème de Kepler qui est un autre sujet très intéressant lié au problème à deux corps et qui sera peut-être inclus dans une version future de ce livre.}\ie les mouvement correspondants aux conditions initiales $\LL=\rr\wedge\vv=0$).
Comme nous pouvons le constater directement,
nous retrouvons les 6 paramètres nécessaires,
ainsi que cela était précédemment indiqué.
Ce vecteur,
ou plus précisément l'angle $\theta$ qu'il fait avec l'axe de l'ellipse,
est appelé {\em anomalie excentrique}\index{anomalie excentrique}\footnote{Comme le montre la figure l'{\em anomalie excentrique} doit son nom à ce qu'il est le paramètre excentré de l'ellipse,
le «vrai» centre,
le centre d'attraction,
étant bien entendu le foyer.},
et a été introduit par Kepler.
L'intérêt de l'anomalie excentrique est l'équation différentielle suivante (l'{\em \'Equation de Kepler})\index{équation de Kepler},
que ce paramètre vérifie le long du mouvement:
\begin{equation}
  \label{aediff}
  dt = \sqrt{a^3}\left[\vphantom{\sqrt{a^3}}1-E\cos \theta\right] d\theta,
\end{equation}
et qui donne,
par intégration,
une nouvelle constante du mouvement:
\begin{equation}
  \label{eqKepler}
  c = t - \sqrt{a^3}\left[\vphantom{\sqrt{a^3}}\theta -E\sin \theta \right].
\end{equation}
Cette constante est donc égale à la valeur de $t$ pour $\theta =0$,
\cad à la date du passage de la planète à l'aphélie.
C'est ce paramètre que les astronomes appellent l'{\em époque}\index{époque} de la planète,
et qu'ils choisissent à la place de l'anomalie excentrique à l'instant zéro\footnote{En réalité ce paramètre est mal défini puisque le mouvement de la planète est périodique.
Il n'est vraiment défini que modulo $\sqrt{a^3}$ (la période du mouvement).
Il faudrait plutôt choisir $C=\exp(2i\pi c/\sqrt{a^3})$,
ce qui est équivalent au choix de $\AA$ à l'instant zéro.}.

\begin{demonstration}
  {\sc (\'Equation de Kepler)}
  Plaçons nous dans le repère orthonormé d'origine le centre de symétrie de l'ellipse,
  et d'ordonnée l'axe du vecteur excentricité $\EE$.
  L'équation d'un point courant de l'ellipse et sa vitesse sont données,
  en coordonnées polaires,
  par:
  \begin{equation}
    P=a\left(
    \begin{array}{cc}
      & \cos \theta \\
      \sqrt{1-E^2} & \sin \theta
    \end{array}
    \right) \quad \vv=a\left(
    \begin{array}{cc}
      & -\sin \theta \\
      \sqrt{1-E^2} & \cos \theta
    \end{array}
    \right) {d\theta\over dt}.
  \end{equation}
  où $a$ est le demi-grand axe et $E<1$ l'excentricité (la norme du vecteur $\EE$).
  Le carré de la vitesse vaut donc
  $$
    v^2=\norm{\vv}^2=a^2(1-E^2\cos^2\theta)\left({d\theta\over dt}\right)^2.
    $$
  En comparant cette expression avec la constante des forces vives donnée par la formule (\ref{fv}):
  $v^2=f+2/r$,
  en se rappelant que $f=-1/a$,
  et en utilisant la formule suivante donnant la distance $r$ du foyer $S$ \index{foyer} au point courant $P$
  $$
    r= a\sqrt{1+E\cos\theta},
    $$
  on obtient l'équation de Kepler annoncée.
\end{demonstration}

\begin{remarque}
  Les mouvements kepleriens sont donc définis par les valeurs de l'époque,
  du moment cinétique et du vecteur de Laplace.
  Mais il est évident,
  puisque tous les mouvements elliptiques sont périodiques,
  que cet espace des mouvements kepleriens est aussi l'ensemble des conditions initiales à l'instant $t=0$,
  \cad l'ouvert de $\RR^3\times \RR^3$ des couples $(\rr,\vv)$ vérifiant:
  \begin{equation}
    \rr\wedge \vv\neq 0 \quad \mbox{et} \quad v^2 - {2\over r} <0.
  \end{equation}
  La représentation d'un mouvement keplerien par ses conditions initiales ou par ses caractéristiques géométriques est {\em a priori} affaire de goût.
  Nous verrons toutefois que certaines représentations sont plus pratiques que d'autres.
  Dans sa {\em Mécanique},
  Lagrange choisit les six {\em éléments kepleriens}\index{éléments kepleriens} $(a,b,c,h,i,k)$,
  où $a$ est la valeur du demi-grand axe (l'inverse de la constante des forces vives au signe près)\index{forces vives},
  $b$ le paramètre de l'ellipse (le carré du moment cinétique),
  $c$ l'époque.
  Les éléments $h$, $i$ et $k$ déterminent le plan de l'orbite et l'axe de l'ellipse dans ce plan:
  $i$ est l'inclinaison du plan de l'orbite par rapport à un plan de référence,
  $h$ est la {\em longitude des n\oe uds},
  \cad l'angle que fait la trace du plan de l'orbite sur le plan de référence (la {\em ligne des n\oe uds})\index{n\oe uds (ligne des)},
  et $k$ est la {\em longitude du périhélie}\index{périhélie},
  \cad l'angle que fait l'axe de l'ellipse avec la ligne des n\oe uds.
\end{remarque}

%%====================================================================
\section{La méthode de la variation des constantes}
\label{parvarconst}
%%====================================================================

Maintenant que nous avons bien compris et résolu\footnote{C'est-à-dire que nous avons ramené le problème à un exercice de calcul numérique.} le problème à deux corps (au moins en ce qui concerne les mouvements réguliers),
il nous reste à traiter le problème à deux corps perturbé,
et introduire ainsi les premiers calculs symplectiques comme l'a fait Lagrange.
Nous nous bornerons,
comme lui\footnote{Même s'il évoque explicitement la possibilité d'une modification de l'orbite de l'ellipse à la parabole,
ou à l'hyperbole.},
aux perturbations des orbites elliptiques\index{perturbation};
les astronomes se sont naturellement toujours intéressés davantage aux perturbations des mouvements elliptiques puisqu'ils concernent,
en particulier,
les mouvements de notre planète.
\'Evidemment,
cela exclut les fortes perturbations qui auraient pour conséquence de transformer notre orbite elliptique en une orbite hyperbolique,
cas pour lequel d'autres techniques sont à mettre en \oe uvre.

Nous avons déjà expliqué (dans le deuxième chapitre) la méthode de la variation des constantes\index{variation des constantes} (voir page \pageref{page_mvc}).
Rappelons succinctement de quoi il s'agit.
Mais plutôt que de répéter ce que nous avons déjà écrit,
laissons la plume à Lagrange \cite[tome II, p.58]{Lagrange1}:
\begin{quote}
  «Un des premiers et des plus beaux résultats de la Théorie de Newton,
  sur le système du monde,
  consiste en ce que toutes les orbites des corps célestes sont de même nature,
  et ne diffèrent entre elles qu'à raison de la force de projection que ces corps peuvent être supposés avoir reçue dans l'origine des choses.
  Il suit de là que,
  si une planète ou une comète venait à recevoir une impulsion étrangère quelconque,
  son orbite en serait dérangée;
  mais il n'y aurait que les éléments,
  qui sont les constantes arbitraires de l'équation,
  qui pourraient changer:
  c'est ainsi que l'orbite circulaire ou elliptique d'une planète pourrait devenir parabolique ou même hyperbolique,
  ce qui transformerait la planète en comète.
  Il en est de même de tous les problèmes de Mécanique.
  Comme les constantes arbitraires introduites par les intégrations dépendent de l'état initial du système,
  qui peut être placé dans un instant quelconque,
  si l'on suppose que les corps viennent à recevoir pendant leur mouvement des impulsions quelconques,
  les vitesses produites par ces impulsions étant composées avec les vitesses déjà acquises par les corps,
  pourront être regardées comme des vitesses initiales,
  et ne feront que changer les valeurs des constantes.
  Et si au lieu d'impulsions finies,
  qui n'agissent que dans un instant,
  on suppose les impulsions infiniment petites,
  mais dont l'action soit continuelle,
  les mêmes constantes deviendront tout à fait variables,
  et serviront à déterminer l'effet de ces sortes de forces,
  qu'il faudra regarder comme des forces perturbatrices.»\index{force perturbatrice}
\end{quote}
Ainsi,
étant donnée une condition initiale du système,
l'influence des forces perturbatrices se traduit sur l'espace des mouvements --- l'espace des constantes d'intégration --- par une courbe.
Cette courbe,
tracée pour le cas d'une planète dans l'espace des mouvements kepleriens\footnote{Les points de cette courbe sont des mouvements kepleriens dont la trajectoire est elliptique,
d'après nos hypothèses.
Pour des raisons bien compréhensibles,
ces ellipses sont appelées par Lagrange les {\em ellipses osculatrices} au mouvement réel.} (non perturbés),
exprime le mouvement réel de la planète.
Il s'agit donc d'en exprimer l'équation,
et éventuellement d'en extraire quelques renseignements,
comme par exemple la stabilité du grand axe.
Ce résultat a été découvert par Laplace en 1773\footnote{Ce théorème de Laplace dit que,
sous certaines conditions sur les masses en présence,
si les perturbations de la planète sont suffisamment petites,
le grand axe de la famille d'ellipses osculatrices au mouvement perturbé ne contient pas de termes linéaires en temps mais uniquement des termes périodiques.
Autrement dit,
faiblement perturbé,
le grand axe des ellipses osculatrices est borné dans le temps,
ce qui est plutôt rassurant.}.

Plutôt que de présenter cette méthode dans le seul cas des planètes,
nous allons montrer maintenant comment Lagrange l'a inclus dans le cadre général de sa méthode de la variation des constantes pour tous les systèmes de la mécanique.
Introduisons les quelques notations utilisées par la suite.

\begin{notation}
  De façon générale,
  on note
  $$
    D(x\mapsto y)(x) = {\partial y \over\partial x}
    $$
  l'application linéaire tangente d'une application différentiable $x\mapsto y$.
  Lorsque le but est $\RR$ et la source $\RR^3$ comme c'est le cas ici pour $[\vv\mapsto a]$,
  l'application linéaire tangente est un covecteur (une ligne de nombres):
  $$
    {\partial a \over\partial \vv} = \left[
    {\partial a \over\partial v_1} \quad {\partial a \over\partial v_2} \quad {\partial a\over\partial v_3}
    \right].
    $$
  Lorsqu'il s'agit d'une application de $\RR$ dans $\RR^3$,
  comme c'est le cas ici pour $[t\mapsto \vv]$ l'application linéaire est un vecteur (une colonne de nombres),
  de telle sorte que le produit de l'un par l'autre soit donné par la somme:
  $$
    {\partial a\over\partial\vv}\cdot{d\vv\over dt}=\sum_{i=1}^3{\partial a \over\partial v_i}{dv_i\over dt}.
    $$
  Ces notations seront utilisées tout le long de ce livre.
  Nous omettrons parfois le point de multiplication.
\end{notation}

Revenons maintenant au système des planètes.
Supposons donc qu'une planète subisse de façon continue une série de chocs infiniment petits.
Ces chocs se traduisent par une variation instantanée de la vitesse,
sans conséquence sur sa position.
Si on désigne par $a$ un élément quelconque de la planète (pas nécessairement le demi grand axe),
le calcul différentiel donne:
\begin{equation}
  \label{eq_da}
  {da\over dt} = {\partial a\over \partial \vv}\cdot{d\vv\over dt}.
\end{equation}
En remarquant que le vecteur $d\vv/dt$ représente exactement la force perturbatrice $X$ exercée sur la planète à l'instant $t$ au point (vecteur) $\rr$,
la variation infinitésimale de l'élément $a$,
sous l'effet de la perturbation,
peut s'écrire:
\begin{equation}
  \label{eq_B}
  {da\over dt} = {\partial a\over \partial \vv}\cdot X.
\end{equation}
Le mouvement vrai est ainsi décrit par la courbe intégrale de cette équation,
tracée dans l'espace des éléments de la planète.
Cette famille d'ellipses est appelée {\em famille d'ellipses osculatrices} \index{ellipse osculatrice} du mouvement perturbé.
Supposons maintenant que la force perturbatrice $X$ dérive d'un potentiel $\Omega$,
\index{potentiel} \cad que,
par définition\footnote{C'est vraisemblablement Lagrange qui a introduit cette propriété/définition des forces dérivant de potentiel.
En termes mathématiques,
sans connotation physique nous disons que la force $X$ est la {\em différentielle} (ou la {\em dérivée extérieure}) d'une fonction scalaire $\Omega$.
Cela pose un certain nombre de questions sur la nature géométrique de cet objet qu'est la force,
mais nous n'aurons pas ce débat ici.
Il est plus ou moins stérile et largement dépassé par le cadre moderne de la mécanique symplectique.
Il suffit juste,
pour ce qui nous concerne ici,
de considérer les choses comme Lagrange lui-même les a considérées:
simplement.}:
\begin{equation}
  X = {\partial \Omega \over \partial \rr}
\end{equation}
Supposons de plus,
comme Lagrange,
que ce {\em potentiel de perturbation} $\Omega$ \index{potentiel de perturbation} n'est fonction que de $\rr$.
Ce qui,
dit autrement,
s'écrit:
\begin{equation}
  {\partial \Omega \over \partial \vv} =0.
\end{equation}
À ce stade,
Lagrange emploie l'astuce suivante:
ajouter $0={\partial \Omega / \partial \vv}$ à $da/dt$,
pour éliminer dès le départ des termes symétriques ennuyeux et pour faire apparaître tout de suite la structure antisymétrique \index{antisymétrique} qui s'imposait par le calcul.
Il aura fallu à Lagrange trois articles sur le même sujet (voir appendice \ref{AnnLP}) pour arriver à ce raccourci et cette présentation épurée.
Nous avons donc:
\begin{eqnarray}
  \label{eq_A}
  {da\over dt} & = & {\partial a\over \partial \vv}\cdot X \nonumber \\
  & = & {\partial a\over \partial \vv}\cdot {\partial \Omega \over\partial \rr} \nonumber \\
  & = & {\partial a\over \partial \vv}\cdot {\partial \Omega \over\partial \rr}-{\partial a\over \partial \rr}\cdot {\partial \Omega\over \partial \vv} \nonumber \\
  & = & \sum_{i=1}^3{\partial a\over \partial\vv^i}{\partial\Omega \over \partial \rr^i} - {\partial a\over \partial\rr^i}{\partial \Omega \over \partial \vv^i}.
\end{eqnarray}
C'est maintenant,
avec cette transformation astucieuse de Lagrange,
que commence la véritable histoire,
d'où sortira la géométrie symplectique.
Mais il faut aller plus loin:
puisque l'application $(t,\rr,\vv)\mapsto (t,a,b,c,h,i,k)$ est un difféomorphisme,
le potentiel de perturbation peut être considéré aussi bien comme une fonction de $\rr$ que comme une fonction du temps $t$ et des éléments $(a,b,c,h,i,k)$ de la planète.
En utilisant les méthodes du calcul différentiel,
que maîtrisait parfaitement Lagrange,
on remplace l'expression de ${\partial \Omega / \partial \rr}$:
\begin{equation}
  {\partial \Omega \over \partial \rr} = {\partial \Omega \over \partial a}{\partial a\over\partial \rr} + {\partial \Omega \over \partial b}{\partial b \over\partial \rr} + \mbox{etc.},
\end{equation}
et de ${\partial \Omega / \partial \vv}$:
\begin{equation}
  {\partial \Omega \over \partial \vv} = {\partial \Omega \over \partial a}{\partial a\over\partial \vv} + {\partial \Omega \over \partial b}{\partial b \over\partial \vv} + \mbox{etc.},
\end{equation}
dans l'équation (\ref{eq_A}),
pour obtenir une nouvelle expression de $da/dt$:
\begin{equation}
  \label{eq_C}
  {da\over dt} = (a,b){\partial \Omega \over\partial b} + (a,c){\partial \Omega \over\partial c} + \mbox{etc.}
\end{equation}
où les parenthèses \index{parenthèses de Lagrange}$(a,b)$, $(a,c)$, \ldots, désignent les fonctions de $(t,\rr,\vv)$ définies par:
\begin{equation}
  (a,b) = \sum_{i=1}^3 {\partial a\over \partial \vv^i}{\partial b\over \partial \rr^i} - {\partial b\over \partial\vv^i}{\partial a\over \partial \rr^i}.
\end{equation}
Il en est de même pour les quatorze autres parenthèses\footnote{Pour 6 constantes du mouvement,
il y a $6\times 6 = 36$ valeurs de parenthèses possibles;
par antisymétrie,
les six parenthèses diagonales $(x,x)$,
où $x=a,\ldots,k$,
sont nulles et les 30 autres s'organisent deux par deux $(x,y)=-(y,x)$,
ce qui donne seulement 15 valeurs différentes possibles non nulles.}\label{36}.
Les termes $\partial \Omega /\partial a$, $\partial \Omega /\partial b$ etc. intervenant dans cette formule peuvent être considérés comme les {\em forces de perturbations}\index{force perturbatrice} rapportées aux variables $(a,b,c,h,i,k)$.
Les coefficients des forces de perturbation exprimées dans les variables $(a,b,c,h,i,k)$ sont appelés aujourd'hui {\em parenthèses de Lagrange}\index{parenthèses de Lagrange}\footnote{Ce sont aussi les {\em crochets de Poisson} des fonctions coordonnées $a,\ldots, k$.}.

L'expression formelle (\ref{eq_B}) de la variation $da/dt$ est beaucoup plus simple que celle (\ref{eq_C}) à laquelle nous avons abouti après toutes ces transformations.
On est en droit de se demander quel intérêt nous avons eu à effectuer ces transformations.
La réponse est contenue dans le théorème suivant de Lagrange,
où l'on considère le difféomorphisme $(t,\rr,\vv)\mapsto (t,a,b,c,h,i,k)$.

\begin{theoreme}
  {\sc (Lagrange)}\index{théorème de Lagrange}
  Les parenthèses $(a,b)$, $(a,c)$, etc. considérées comme des fonctions de $(t,a,b,c,h,i,k)$ ne sont fonction que des éléments $(a,b,c,h,i,k)$.
\end{theoreme}

À ce propos Lagrange écrit exactement \cite[tome II, p. 73]{Lagrange1}:
\begin{quote}
  « Ainsi la variation de $a$ sera représentée par une formule qui ne contiendra que les différences partielles de $\Omega$ par rapport à $b$, $c$, etc.,
  multipliées chacune par une fonction de $a$, $b$, $c$, etc.,
  sans $t$.
  Et la même chose aura lieu à l'égard des variations des autres constantes arbitraires $b$, $c$, $h$, etc.»
\end{quote}

\begin{note}
  Lagrange a donné successivement plusieurs démonstrations de ce théorème,
  le généralisant et le simplifiant chaque fois davantage.
  Il l'énonce la première fois,
  dans le cadre du mouvement des planètes,
  dans son mémoire de 1808 \cite{Lagrange2}.
  Il le généralise ensuite à tous les problèmes de la mécanique dans son mémoire de 1809 \cite{Lagrange3},
  qu'il complète par l'inversion explicite de ses parenthèses dans son mémoire de 1810 \cite{Lagrange4} (voir annexe \ref{AnnLP} pour plus de détails).
  Dans son mémoire de 1809,
  Lagrange donne en annexe une construction épurée,
  simplifiée qui ne comporte plus que quelques pages.
  L'énoncé particulier que nous avons donné plus haut est extrait de sa {\em Mécanique Analytique} publiée en 1811 \cite{Lagrange1}.
  On peut voir dans ce théorème les prémisses du théorème connu aujourd'hui des étudiants sous la forme suivante:
  {\em le crochet de Poisson\index{crochets de Poisson} de deux constantes du mouvement est encore une constante du mouvement}\ldots
\end{note}

Aussitôt énoncé son théorème,
Lagrange fait remarquer\footnote{Cette présentation respecte la {\em Mécanique Analytique} de Lagrange même si historiquement Lagrange a d'abord défini ses crochets dans l'article de 1808,
où ils apparaissent comme les coordonnées des forces $\partial\Omega / \partial a$ en fonction des variations des éléments.
Ils sont notés,
dans cet article,
comme des parenthèses.} que la formule (\ref{eq_C}) donnant l'expression de la variation des éléments de la planète en fonction des forces de perturbations s'inverse,
et note que:
\begin{equation}
  \label{eq_D}
  {\partial \Omega \over \partial a} = [a,b]{db\over dt} + [a,c]{dc\over dt} + \mbox{etc.},
\end{equation}
où les crochets\index{crochets de Lagrange} $[a,b]$, $[a,c]$, \ldots, ne sont eux-mêmes fonction que des éléments $(a,b,c,h,i,k)$,
et sont explicitement donnés par:
\begin{equation}
  [a,b] = \sum_{i=1}^3{\partial \rr^i \over \partial a}{\partial\vv^i \over \partial b} - {\partial \vv^i\over \partial a}{\partial \rr^i \over \partial b},\quad \mbox{etc.}
\end{equation}
Dans cette dernière équation les vecteurs $\rr$ et $\vv$ sont considérés comme fonctions de $t$ et des éléments $(a,b,c,h,i,k)$.
Ainsi le mouvement de la planète perturbée est décrit par l'équation différentielle (\ref{eq_C}) sur l'espace des mouvements de la planète non perturbée,
ou si l'on préfère sur l'{\em espace des constantes d'intégration} du système non perturbé.
C'est évidemment là l'origine du nom donné par Lagrange à sa méthode:
{\em la méthode de la variation des constantes}.
En effet,
la variation des constantes d'intégration du système non perturbé décrit le mouvement réel du système perturbé.

\begin{note}
  Cette méthode\index{variation des constantes} est évidemment de même nature que la méthode du même nom développée par Lagrange entre 1774 et 1779,
  à la fois pour comprendre la nature des {\em solutions particulières des équations différentielles} \cite{Lagrange5,Lagrange7} que pour résoudre les systèmes différentiels linéaires inhomogènes \cite[Remarque 5, pp.159--165]{Lagrange6}.
  C'est dans ce dernier mémoire {\em Sur les suites récurrentes\ldots}\footnote{Ce mémoire n'a que peu à voir avec la méthode de la variation des constantes.
  Lagrange le dit lui même:
  «{\em Quoique ce ne soit pas ici le lieu de nous occuper de cette matière, je vais néanmoins en traiter en peu de mots, me réservant de le faire ailleurs avec plus d'étendue.}»} que Lagrange expose de façon formelle sa méthode sur la variation des constantes,
  méthode qu'il n'avait fait qu'ébaucher dans \cite{Lagrange5},
  mais qu'il avait déjà abondamment utilisée.
  Dans le cas des équations linéaires inhomogènes \cite{Lagrange6},
  la partie non homogène est traitée comme une perturbation de la partie linéaire.
  L'espace des solutions du système linéaire est un espace vectoriel dont chaque point est un ensemble de constantes d'intégration.
  Le terme non linéaire du système initial définit sur cet espace vectoriel un nouveau système différentiel,
  équivalent au premier,
  mais qui porte sur les {\em constantes d'intégration} du système linéaire.
  Il est intéressant de noter à ce propos cette remarque de Lagrange \cite[p. 163]{Lagrange6}:
  \begin{quote}
    « J'avoue que l'intégration des équations en $a$, $b$, $c$,\ldots et x sera le plus souvent très difficile,
    du moins aussi difficile que celle de l'équation proposée [...] mais le grand usage de la méthode précédente est pour intégrer par approximation les équations dont on connait déjà l'intégrale complète à peu près,
    c'est-à-dire en négligeant les quantités qu'on regarde comme très petites.»
  \end{quote}
  Il achève sa remarque 5 de son article {\em Sur les suites récurrentes\ldots} par ce paragraphe prémonitoire,
  treize ans avant son premier mémoire sur la variation des constantes appliquée au système des planètes:
  \begin{quote}
    « Il est visible au reste que cette méthode,
    que je ne fais qu'exposer ici en passant,
    peut s'appliquer également au cas où l'on aurait plusieurs équations différentielles entre plusieurs variables dont on connaitrait les intégrales complètes approchées,
    c'est-à-dire en y négligeant des quantités supposées très petites.
    Elle sera par conséquent fort utile pour calculer les mouvements des planètes en tant qu'ils sont altérés par leur action mutuelle,
    puisqu'en faisant abstraction de cette action la solution complète du problème est connue;
    et il est bon de remarquer que,
    comme dans ce cas les constantes $a$, $b$, $c$,\ldots représentent ce qu'on nomme les {\em éléments des planètes},
    notre méthode donnera immédiatement les variations de ces éléments provenantes de l'action que les planètes exercent les unes sur les autres.»
  \end{quote}
  On peut se demander quelle est alors la différence entre cette méthode,
  introduite dans les années 1770,
  et son application au cas du système des planètes ?
  Elle relève principalement du type de système traité.
  En appliquant sa méthode générale de la variation des constantes aux systèmes différentiels spécifiques de la mécanique,
  Lagrange fait apparaître une structure particulière,
  qui n'existe pas dans le cas général et qui est à l'origine de la géométrie symplectique.
  Cette structure,
  caractérisée par les crochets et parenthèses qu'il a défini,
  Lagrange va savoir en tirer profit,
  comme il l'espérait,
  dans l'étude de la stabilité du grand axe des planètes,
  comme nous allons le voir maintenant.
\end{note}

%%====================================================================
\section{Application à la stabilité séculaire du grand axe}
\label{parstabsec}
%%====================================================================

Nous sommes en mesure maintenant de déduire,
de toutes ces transformations et manipulations algébriques,
le théorème de Lagrange sur la stabilité\index{stabilité (du grand axe)} du grand axe des planètes.
Appliquons la formule (\ref{eq_D}) à l'époque $c$:
\begin{equation}
  \label{eqforcpert}
  {\partial \Omega \over \partial c} =[c,a]{da\over dt} + [c,b]{db\over dt} + \cdots + [c,k]{dk\over dt}.
\end{equation}
Lagrange a calculé l'ensemble de ces crochets\index{crochets de Lagrange}.
Les calculs sont pénibles et parfois même douteux,
mais on peut le croire lorsqu'il affirme que les crochets $[c,b]$, $[c,h]$, $[c,i]$, $[c,k]$ sont nuls;
il reste alors:
\begin{equation}
  [c,a] = -1/2a^2 \quad \mbox{d'où} \quad {\partial\Omega\over\partial c} = -{1\over2a^2}{da\over dt}.
\end{equation}
Si on se rappelle alors que le demi-grand axe $a$ est égal à $-1/f$,
où la constante des forces vives $f$ est le double de l'énergie $H$\footnote{La lettre $H$ a été choisie par Lagrange en l'honneur de Huygens et non de Hamilton,
voir \cite[tome I, pages 217--226 et 267--270]{Lagrange1};
voir également l'annexe A.} du mouvement keplerien,
on obtient:
\begin{equation}
  \label{eqLagrange}
  {dH\over dt} = -{\partial \Omega \over \partial c}.
\end{equation}
Cette formule est en réalité très générale et Lagrange l'établit pour tous les problèmes de mécanique analytique conservatifs\index{système conservatif} \cite{Lagrange3}.
Comme nous l'avons déjà dit,
le potentiel de perturbation $\Omega$ (fonction de $\rr$) est considéré comme fonction de $t$ et des éléments kepleriens $(a,b,c,h,i,k)$.
Mais le temps n'intervient dans $\Omega$ que par $t-c$;
plus précisément $\Omega$ n'est fonction que de $(a,b,t-c,h,i,k)$.
En effet,
dans des coordonnées orthonormées du plan de l'orbite,
en prenant pour axe des $x$ l'axe porté par le vecteur $\EE$ et en posant $\rr=(x,y)$ (voir fig. \ref{FigAExc}),
on a:
\begin{equation}
  x= a\sqrt{1-{b\over a}} + a\cos(\theta) \quad \mbox{et}\quad y=\sqrt{ab}\sin(\theta),
\end{equation}
où l'anomalie excentrique \index{anomalie excentrique} $\theta$ est donnée en inversant la formule (\ref{eqKepler}) de Kepler.
On peut préciser davantage les choses en notant $\phi_E$ la fonction:
\begin{equation}
  \phi_E : \theta\mapsto \theta - E\sin(\theta)\quad\mbox{avec} \quad E = \sqrt{1-{b\over a}}.
\end{equation}
Cette fonction est inversible (car $E<1$) et on peut écrire:
\begin{equation}
  \label{eq_xy1}
  x= a\sqrt{1-{b\over a}} +a\cos\left[  \phi_E^{-1}\left(  {t-c\over a^{3/2}}  \right)  \right]
\end{equation}
et
\begin{equation}
  \label{eq_xy2}
  y=\sqrt{ab}\sin\left[  \phi_E^{-1}\left(  {t-c\over a^{3/2}}  \right)  \right].
\end{equation}
On en déduit,
d'une part,
une nouvelle expression de (\ref{eqLagrange}) donnant la variation de l'énergie\index{énergie} $H$:
\begin{equation}
  \label{eq_dH}
  {dH\over dt} = {\partial \Omega \over \partial t}.
\end{equation}
On constate,
d'autre part,
que la fonction $\Omega$ est périodique en $t-c$ (voir (\ref{eq_xy1}) et (\ref{eq_xy2})),
de période $2\pi a^{3/2}$.
Le potentiel peut alors se développer en série trigonométrique,
qu'on appelle aujourd'hui {\em série de Fourier}\index{série de Fourier}.
Il est intéressant de noter ce que Lagrange écrit explicitement\footnote{Lagrange et Fourier étaient contemporains,
quelle a été leur influence mutuelle?} à ce propos \cite[pages 735--736]{Lagrange2}:
\begin{quote}
  « Comme les valeurs des coordonnées peuvent être réduites en série de sinus et cosinus,
  il est facile de voir que la fonction $\Omega$ pourra être réduite en une série de sinus et cosinus;
  ces sinus et cosinus ayant pour coefficients des fonctions des éléments $a$, $b$, $c$, etc.»
\end{quote}
Cette série de Fourier s'écrit aujourd'hui de la manière suivante:
\begin{equation}
  \Omega = \sum_k A_k \exp {i k(t-c) \over a^{3/2}}.
\end{equation}
Les coefficients $A_k$ étant des fonctions uniquement des éléments de l'orbite $a$, $b$, $h$, $i$, $k$,
l'équation (\ref{eq_dH}) devient alors:
\begin{equation}
  {dH\over dt} = -{\partial \Omega \over \partial c} = \sum_{k\neq 0} {ikA_k\over a^{3/2}} \exp {i k(t-c) \over a^{3/2}}.
\end{equation}
Ainsi que l'énonce Lagrange,
la première approximation consiste à regarder dans la fonction $\Omega$ tous ces éléments comme constants \cite[page 736]{Lagrange2} --- \ie à considérer,
à l'intérieur des fonctions $A_k$,
les éléments de l'orbite comme constants.
Sans vouloir commenter la validité de cette affirmation,
on obtient ensuite par intégration:
\begin{equation}
  H(t) \sim H_0 + \sum_{k\neq 0} A_k \exp{ik(t-c) \over a^{3/2}}.
\end{equation}
À ce premier ordre d'approximation,
la fonction $H$ (et donc le grand axe $a=-1/2H$) ne contient pas de terme linéaire en $t$ (qu'on appelle le {\em terme séculaire\footnote{Car leur présence entraîne des perturbations sensibles au long des siècles.}}) mais seulement des termes périodiques.
Nous venons de re-démontrer le théorème de stabilité du grand axe de Lagrange.
Laissons le encore une fois s'exprimer \cite[page 736]{Lagrange2}:

\begin{theoreme}
  {\sc(Lagrange)}\index{théorème de Lagrange}
  Les grands axes des planètes ne peuvent être sujets qu'à des variations périodiques,
  et non à des variations croissant comme le temps.
\end{theoreme}

Ce théorème n'est qu'une application particulière des méthodes de la variation des constantes introduite par Lagrange.
Elle ne concerne,
telle qu'elle est présentée ici,
que la première approximation (démontrée la première fois,
mais par d'autres méthodes,
par Laplace en 1773).
Son véritable théorème sur la stabilité séculaire des grands axes des planètes (où il étend véritablement le résultat de Laplace) est plus profond,
subtil et délicat car il prend en compte le mouvement de toutes les planètes (consulter par exemple \cite{Sternberg1}).
Nous ne le présenterons pas ici,
car le calcul est plus compliqué même si le principe est le même:
écrire le mouvement perturbé dans l'espace des solutions des mouvements non perturbés.
Malgré tout,
même si la méthode de Lagrange est correcte (nous en verrons plus loin une justification moderne),
ses approximations sont douteuses et méritent davantage de soin.
C'est ce qu'apporteront par la suite des mathématiciens comme Poincaré et ses successeurs.

L'importance de cette nouvelle méthode introduite par Lagrange,
outre qu'elle formule de façon élégante les principes de la mécanique analytique --- en introduisant la structure symplectique de l'espace des mouvements kepleriens --- facilite aussi le calcul des autres {\em inégalités}\index{inégalités (des mouvements)}\footnote{C'est ainsi qu'on appelait les variations des éléments de l'orbite dues aux perturbations extérieures.}.
C'est ce qui l'a rendu célèbre puisque Lagrange montre que la variation de l'angle du périhélie de Jupiter,
observée par les astronomes (mais non encore expliquée à l'époque),
est périodique.
Il en calculera la période ($\sim 900$ ans si on croit Sternberg \cite{Sternberg1}).
La résolution des inégalités du mouvement des planètes et le calcul des perturbations marque l'origine du calcul symplectique en mécanique.
L'histoire a montré qu'il en est le cadre idéal.

%%====================================================================
\section{Structure de l'espace des mouvements kepleriens}
%%====================================================================

Ces crochets $[a,b]$, $[a,c]$, \ldots, fonctions seulement des éléments kepleriens $a$, $b$, $c$ etc. possèdent trois propriétés remarquables qui sont à la base de l'axiomatique de la géométrie symplectique.

\begin{enumerate}
  \item[\bf 1.]{\bf Antisymétrie.} Ils sont {\em anti-symétriques}:
  \begin{equation}
    [a,b] = -[b,a], \quad [a,c] = -[c,a], \quad \mbox{etc.}
  \end{equation}
  \item[\bf 2.]{\bf Inversibilité.} La matrice $\omega$ définie par la famille de crochets:
  \begin{equation}
    \omega_{ab}=[a,b], \quad \omega_{ac}=[a,c],\quad \mbox{etc.}
  \end{equation}
  est inversible,
  et son inverse est la matrice des parenthèses de Lagrange:
  \begin{equation}
    \left(\omega^{-1}\right)_{ab}=(a,b), \quad \left(\omega^{-1}\right)_{ac}=(a,c), \quad \mbox{etc.}
  \end{equation}
  \item[\bf 3.]{\bf Fermeture.} Pour tous les triplets d'éléments $(a,b,c)$,
  $(a,b,h)$, \ldots, $(i,h,k)$ l'équation aux dérivées partielles suivante est vérifiée:
  \begin{equation}
    {\partial [b,c] \over \partial a} + {\partial [c,a] \over \partial b} + {\partial [a,b] \over\partial c} =0, \quad \mbox{etc.}
  \end{equation}
\end{enumerate}

\noindent Ces trois propriétés font de la matrice $\omega$ ce qu'on appelle aujourd'hui une {\em forme symplectique}\index{forme symplectique}.
Sans vouloir nous attarder sur les définitions formelles,
disons seulement qu'une {\em forme différentielle}\index{forme différentielle} définie sur un ouvert d'un espace numérique est une application qui à chaque point de cet ouvert associe une application multilinéaire alternée.
Par exemple,
une $2$-forme $\omega$ définie sur un ouvert de $\RR^{2n}$ sera caractérisée par $n(n-1)/2$ fonctions $\omega_{ij}$,
de telle sorte que:
\begin{equation}
  \omega(x)(X,Y) = \sum_{i,j} \omega_{ij}(x) X^i Y^j,
\end{equation}
où $x$ est un point de l'ouvert de définition,
$X=(X^i)$ et $Y=(Y^i)$ deux vecteurs de $\RR^{2n}$,
les indices $i$ et $j$ variant de 1 à $2n$.
On dit que la 2-forme différentielle $\omega$ est {\em symplectique} si elle est {\em non dégénérée} en chaque point et si elle est {\em fermée},
c'est-à-dire\footnote{Cette formulation n'est pas très parlante.
Dire qu'une forme différentielle $\omega$ est fermée signifie précisément qu'elle est {\em localement exacte}:
pour tout point $x$ il existe un voisinage $U$ et une forme différentielle $\alpha$ tels que $\omega\mid U=d\alpha$,
\cad:$\omega_{ij}=\partial_i\alpha_j-\partial_j\alpha_i$.} si:
\begin{equation}
  \partial_i\omega_{jk} +\partial_j\omega_{ki} +\partial_k\omega_{ij} =0,
\end{equation}
pour tout triplet d'indices $i,j,k$;
on note $d\omega =0$.

Les trois propriétés que nous avons énoncées plus haut font des crochets de Lagrange les composantes,
covariantes\index{composantes covariantes}\index{composantes contravariantes} et contravariantes,
d'une forme symplectique sur l'espace des mouvements kepleriens de la planète.
Les deux premières propriétés ont été soulignées explicitement par Lagrange,
même s'il ne pouvait considérer à son époque ces crochets comme les éléments d'une matrice,
{\em a fortiori} d'une $2$-forme différentielle.
Quant à la propriété de fermeture (la troisième propriété),
il ne l'évoque pas.
En effet,
son importance n'est apparue que plus tard,
avec la formalisation du calcul différentiel.
Du point de vue de la mécanique,
cette dernière propriété est la conséquence de l'existence du potentiel $\Omega$ des forces de perturbation:
$X=\partial \Omega / \partial \rr$.

Lagrange a explicitement calculé la valeur de ses crochets,
c'est-à-dire les composantes de la forme symplectique,
au nombre de quinze\footnote{Il y a {\em a priori} 36 composantes de la forme $\omega$ mais pour des raisons d'antisymétrie,
que nous avons déjà évoquées (note page \pageref{36}),
seules 15 de ces composantes sont indépendantes.}.
Il en donne les expressions dans diverses cartes de l'espace des mouvements kepleriens,
\cad pour divers choix d'éléments kepleriens caractérisant les mouvements de la planète.
Il n'y a pas grand intérêt à donner ici l'ensemble de ces expressions que l'on peut trouver dans \cite{Lagrange2} et \cite{Lagrange1}.

\begin{remarque}
  Lagrange note que l'on peut toujours choisir,
  comme constantes d'intégration,
  les positions et les vitesses à un instant donné plutôt que les éléments de la planète.
  L'expression des parenthèses et des crochets s'en trouve alors notablement simplifiée.
  En effet,
  dans ce cas les seuls crochets non nuls sont:
  \begin{equation}
    [\vv_i,\rr_i] = 1, \quad i=1,2,3.
  \end{equation}
  Comme on le voit,
  les variables se regroupent par deux:
  $\rr_i$ avec $\vv_i$ et leurs crochets sont constants.
  Cette forme symplectique définie de façon générale sur $\RR^n\times\RR^n$ est appelée aujourd'hui {\em forme symplectique canonique}\index{forme canonique}.
  Le {\em Théorème de Darboux}\index{théorème de Darboux} (voir annexe \ref{geomsymp}) dit que toute forme symplectique possède au moins localement des coordonnées canoniques.
  Mais Lagrange,
  même s'il dit qu'«il y aurait toujours de l'avantage à utiliser ces constantes à la place des autres constantes $a$, $b$, $c$, etc.» \cite[volume II, page 76]{Lagrange1},
  n'a pratiquement pas utilisé ces coordonnées canoniques.
  En particulier,
  la carte $(a,b,c,h,i,k)$ n'est pas canonique.
\end{remarque}

Revenons à la méthode de la variation des constantes telle qu'elle est présentée plus haut,
et en particulier à la formule (\ref{eq_da}).
Nous pouvons en donner une justification en termes actuels.
Considérons l'espace $Y$ des conditions initiales du système étudié,
\cad l'espace des triplets $y=(t,\rr,\vv)$ où $t\in \RR$,
$\rr\in\RR^3-\{0\}$ et $\vv\in \RR^3$.
Les solutions de l'équation différentielle
\begin{equation}
  {d\rr\over dt} = \vv \quad \mbox{et} \quad {d\vv\over dt} = - {\rr\over r^3} +X,
\end{equation}
sont les courbes intégrales du feuilletage défini sur $Y$ par:
\begin{equation}
  y\mapsto \RR\cdot\xi \quad \mbox{avec}\quad \xi=\vect{1 \\ \vv \\ -\rr/r^3+X}.
\end{equation}
Le vecteur $\xi$ se décompose en $\xi_0 + \chi$:
\begin{equation}
  \xi_0 = \vect{1 \\ \vv \\ -\rr/r^3} \quad \mbox{et}\quad\chi = \vect{0 \\ 0 \\ X}.
\end{equation}
L'espace des mouvements kepleriens est l'espace quotient $\cK=Y/\RR\cdot\xi_0$,
\cad l'espace des courbes intégrales du feuilletage $y\mapsto\RR\cdot\xi_0$.

%% TODO: The user needs to provide the modern image file for this figure.
 \begin{figure}[t]
  \centerline{\includegraphics{figures/fig-perturbKepler.pdf}}
  \caption{Projection de $Y$ sur $\cK$}
  \label{figure3}
 \end{figure}

Considérons alors une feuille du feuilletage $y\mapsto \RR\cdot \xi$ passant par $y=(t,\rr,\vv)$.
Cette courbe se projette sur l'espace des mouvements kepleriens $\cK$;
son équation est alors:
\begin{equation}
  {dm\over dt} = D\pi_y(\xi)=D\pi_y(\xi_0) +D\pi_y(\chi),
\end{equation}
où $\pi : y\mapsto m$ est la projection de $Y$ sur son quotient et $D$ désigne l'application linéaire tangente.
Or,
par construction,
$D\pi_y(\xi_0)=0$;
il reste donc $dm/dt= D\pi_y(\chi)$.
Un petit dessin vaut parfois mieux qu'un long discours (voir figure \ref{figure3}).
C'est la famille d'équations (\ref{eq_B}).
Enfin,
transformée en la famille d'équations (\ref{eq_C}),
elle s'écrit encore\footnote{$\omega$ est considérée comme une matrice,
$\omega^{-1}$ désigne la matrice inverse.}:
\begin{equation}
  {dm\over dt} = \omega^{-1}\left({d\Omega}\right),
\end{equation}
où $d\Omega$ est la différentielle de $\Omega$,
sa dérivée extérieure,
calculée au point $m$ évidemment.
La forme $\omega$ peut être considérée comme une application de l'espace des vecteurs sur celui des covecteurs:
l'application réciproque $\omega^{-1}$ associe donc à une forme différentielle (par exemple $d\omega$) un vecteur tangent.
C'est le sens qu'il faut donner à cette notation\footnote{Les physiciens utilisent parfois la notation $\nabla\Omega$ pour désigner la dérivée extérieure de la fonction $\Omega$.}.
Par analogie avec le cas euclidien,
comme $\omega$ est inversible,
on appelle {\em gradient symplectique}\index{gradient symplectique} de la fonction $\Omega$ le champ de vecteurs $\omega^{-1}(d\Omega)$,
et on note:
\begin{equation}
  \grad(\Omega)\mbox{ ou } \grad_\omega(\Omega) \equaldef \omega^{-1}\left({d\Omega}\right).
\end{equation}
Avec ces conventions de langage,
l'équation différentielle qui décrit la variation des constantes prend la forme suivante:
\begin{equation}
  {dm\over dt} = \grad(\Omega).
\end{equation}
C'est sous cette forme qu'on la retrouve fréquemment dans les ouvrages actuels de mécanique.
L'évolution du mouvement $m$,
perturbé par le potentiel $\Omega$,
est donc la courbe intégrale du gradient symplectique du potentiel de perturbation.

%%====================================================================
\section{Une remarque de Poincaré}
%%====================================================================

La partie la moins précise du travail de Lagrange concerne sûrement la méthode d'approximation utilisée.
Je voudrais à ce propos souligner qu'hormis ces méthodes d'approximation les conclusions de Lagrange sont rigoureusement établies,
même si la présentation qu'il en a faite,
et que j'ai essayé de reproduire ici,
ne respecte pas les canons actuels de la rigueur mathématique.
En ce sens,
les transformations qu'il apporte aux équations initiales ne sont pas d'une grande utilité puisque celles qu'il obtient leur sont absolument équivalentes.
Laissons parler Lagrange encore une fois:
\begin{quote}
  «Ainsi on peut regarder les équations précédentes entre les nouvelles variables $a$, $b$, $c$, etc. comme les transformées des équations en $x$, $y$, $z$;
  mais ces transformations seraient peu utiles pour la solution générale du problème.
  Leur grande utilité est lorsque la solution rigoureuse est impossible,
  et que les forces perturbatrices sont très petites;
  elles fournissent alors un moyen d'approximation.»
\end{quote}
Mais la justification de ces méthodes a occupé un grand nombre de mathématiciens après lui et non des moindres.
Poincaré\index{Poincaré} soulignait ainsi dans l'introduction de sa célèbre {\em Nouvelle mécanique céleste} \cite{Poincare1}:
\begin{quote}
  «Ces méthodes qui consistent à développer les coordonnées des astres suivant les puissances des masses,
  ont en effet un caractère commun qui s'oppose à leur emploi pour le calcul des éphémérides à longue échéance.
  Les séries obtenues contiennent des termes dits {\em séculaires},
  où le temps sort des signes des sinus et cosinus,
  et il en résulte que leur convergence pourrait devenir douteuse si l'on donnait à ce temps $t$ une grande valeur.
  La présence de ces termes séculaires ne tient pas à la nature du problème,
  mais seulement à la méthode employée.
  Il est facile de se rendre compte,
  en effet,
  que si la véritable expression d'une coordonnée contient un terme en $\sin \alpha mt$,
  $\alpha$ étant une constante et $m$ l'une des masses,
  on trouvera quand on voudra développer suivant les puissances de $m$,
  des termes séculaires $\alpha mt - \alpha^3m^3t^3/6 +\cdots$ et la présence de ces termes donnerait une idée très fausse de la véritable forme de la fonction étudiée.»
\end{quote}
Cette objection est sans nul doute très pertinente et a conduit,
notamment grâce aux travaux de Poincaré,
au développement de la géométrie symplectique --- en particulier en ce qui concerne son application à la mécanique.
De nouvelles théories sont nées,
comme par exemple la théorie des systèmes complètement intégrables et de leur perturbation qui a donné le fameux théorème\footnote{Théorème difficile,
qui nécessite beaucoup de précautions dans l'énoncé et d'expertise dans ses applications.} de Kolmogorov--Arnold--Moser,
sur la stabilité de nombreux mouvements après perturbation (voir \cite{Arnold0} et \cite{Arnold1}).

\newpage\null\vfill

%% TODO: The user needs to provide the modern image file for this figure.
 \begin{figure}[h]
  \centerline{\includegraphics[width = 1\textwidth]{figures/fig-portraits.pdf}}
  \caption{Galerie}
  \label{portraits}
 \end{figure}

\vfill
