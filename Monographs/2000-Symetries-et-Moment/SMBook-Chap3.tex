%%====================================================================
% MARK: - Chapter 3: Quelques exemples
%%====================================================================

\chapter{Quelques exemples}

Le cas le plus simple de lagrangien homogène est celui défini par la norme associée à une métrique riemannienne positive:
la variété $Q$ est munie d'une métrique riemannienne $g$ et $l(q,\dot q) = \norm{\dot q}$.
Les solutions d'un tel problème variationnel sont appelées {\em géodésiques} de la variété $Q$:
ce sont les extrémales de la longueur.
Dans le cas (rare) où l'espace des géodésiques est une variété,
c'est donc une variété symplectique.
Nous allons l'expliciter ici,
pour deux exemples simples:
les géodésiques de la sphère $S^2$ et celles du disque de Poincaré $H^2$.
Nous avons déjà vu au paragraphe précédent comment la variété des droites affines de l'espace $\RR^2$,
ses géodésiques,
était munie d'une structure de variété symplectique.
Ces trois exemples:
le plan,
la sphère et le disque réalisent les trois cas de courbure constante:
nulle,
positive et négative.

%%====================================================================
\section{Les géodésiques de la sphère}
\index{géodésiques de la sphère}
%%====================================================================

L'exemple bien connu\footnote{Si nous marchons sur la Terre toujours {\em tout droit} dans une direction donnée,
nous passerons inévitablement par les antipodes de notre point de départ,
et donc par les antipodes de chacun des points par lesquels nous passons.} des géodésiques de la sphère nous servira ici d'illustration à la méthode générale que nous venons de présenter.
La sphère $S^2$ est la sous-variété des vecteurs unitaires de $\RR^3$:
\begin{equation}
  S^2 =\{x\in \RR^3 \mid \norm{x}=1\}.
\end{equation}
en un point $x$,
l'espace tangent s'identifie naturellement avec le sous-espace vectoriel de $\RR^3$,
orthogonal à $x$:
\begin{equation}
  T_xS^2 \simeq \{v\in \RR^3 \mid \scal(x,v)=0\}.
\end{equation}
La sphère $S^2$ est munie de sa métrique usuelle,
induite par la métrique de $\RR^3$.
Autrement dit,
pour tout couple de vecteurs $v$,
$v'$ orthogonaux à $x$,
on pose simplement:
\begin{equation}
  g_x(v,v')= \scal(v,v').
\end{equation}
Le lagrangien est donné par:
\begin{equation}
  l(x,v) = \norm{v}.
\end{equation}
On en déduit l'application de Legendre $P$:
\begin{equation}
  P : (x,v) \mapsto \left(x, {\bar v\over \norm{v}} \right),
\end{equation}
où la barre désigne le vecteur transposé.
La sous-variété $Y$ d'équation $l=1$ est le fibré unitaire tangent à la sphère $S^2$,
\cad:
\begin{eqnarray}
  Y & = & \{(x,u)\in TS^2 \mid \norm{v}=1\} \\
  \nonumber
  & = & \{(x,u)\in \RR^3\times \RR^3 \mid\scal(x,x)=\scal(u,u)=1 \mbox{ et }\scal(x,u)=0 \},
\end{eqnarray}
et la forme de Cartan est simplement:
\begin{equation}
  \varpi(\delta y) = \scal(u,\delta x),\quad \mbox{avec} \quad y=(x,u).
\end{equation}
La forme présymplectique $d\varpi$ définie sur $Y$,
dont on doit calculer le noyau,
s'écrit:
\begin{equation}
  d\varpi (\delta y,\delta'y) = \scal(\delta u,\delta'x) - \scal(\delta'u,\delta x).
\end{equation}
En utilisant la méthode des multiplicateurs de Lagrange exposée plus haut,
on obtient les équations du noyau:
\begin{equation}
  {dy\over ds}\in \ker d\varpi \quad \Leftrightarrow \quad {dx\over ds} = \alpha u, \quad {du\over ds} = -\alpha x, \quad \alpha \in \RR.
\end{equation}

%% TODO: The user will provide the modern image file for this figure.
%% \begin{figure}[t]
%%  \centerline{\includegraphics{figures/fig-sphere.pdf}}
%%  \caption{Les géodésiques de la sphère $S^2$}
%%  \label{sphere}
%% \end{figure}

Sans méthode générale pour résoudre cette équation,
on peut remarquer que le plan $m$ engendré par $x$ et $u$ est constant le long de chaque solution.
En effet,
le plan $m$ étant entièrement caractérisé par le produit vectoriel $x\wedge u$,
on peut écrire directement:
\begin{equation}
  m = x\wedge u \quad \mbox{et} \quad {dm \over ds} = 0.
\end{equation}
La solution est donc entièrement contenue dans l'intersection de $S^2$ avec $m$.
Comme cette intersection est une courbe et que nous cherchons une courbe,
cela ne peut être que celle-là.
Autrement dit,
$m$ représente la solution de condition initiale $(x,u)$.
Réciproquement,
tout plan de $\RR^3$ définit une solution;
donc l'espace des géodésiques de la sphère $S^2$ est l'espace des plans orientés de $\RR^3$.
Il faut noter à ce propos que l'involution $(x,u)\mapsto (x,-u)$ ne préserve pas la 2-forme $d\varpi$,
mais l'inverse.
La solution de condition initiale $(x,u)$ est donc différente de la solution $(x,-u)$.
Il est important de préciser que,
comme courbe de sur $S^2$,
la solution de condition initiale $(x,u)$ est orientée par l'orientation du plan $m$.
Autrement dit,
la bonne réalisation de la variété des géodésiques (orientées) de $S^2$ est celle que nous avons donnée plus haut,
définie par l'application:
\begin{equation}
  \mu : Y = US^2 \to \RR^3,\quad (x,u) \mapsto m=x\wedge u.
\end{equation}
La variété des géodésiques de $S^2$ est l'image par $\mu$ de $Y$,
\cad:
\begin{equation}
  \cG_{S^2} = \mu (US^2) = S^2.
\end{equation}
Il reste à exprimer la forme symplectique associée.
C'est un exercice d'algèbre linéaire de montrer que:
\begin{equation}
  \label{fsS2}
  \omega_m(\delta m,\delta'm) = -\scal(m,\delta m\wedge \delta'm) = \det(m,\delta m,\delta'm).
\end{equation}
C'est l'élément de surface de la sphère $S^2$.
Autrement dit,
la variété des géodésiques sur la sphère $S^2$ munie du produit scalaire ordinaire est la sphère unité de $\RR^3$ munie de la restriction de la forme d'aire canonique (au signe près).

\begin{exercice}
  Que se passe-t-il pour une métrique homothétique $ag$,
  $a>0$?
  Vérifier le signe de la formule précédente (\ref{fsS2}).
\end{exercice}

\begin{exercice}
  Traiter le cas des géodésiques de la sphère unité de $\RR^n$.
\end{exercice}

%%====================================================================
\section{Les géodésiques du disque de Poincaré}
\label{exdisquepoincare}
%%====================================================================

L'exemple des géodésiques du disque de Poincaré\index{géodésiques du disque $H^2$} est moins intuitif que celui de la sphère $S^2$ et l'intuition nous faisant défaut,
la méthode générale est ici notre seul guide.
Nous allons donner maintenant l'un des innombrables modèles de l'espace hyperbolique de dimension 2 et la construction de ses géodésiques.
Cet espace a $\PSL(2,\RR)$ comme groupe d'isométries directes.
C'est le groupe des matrices réelles $2\times 2$ de déterminant $+1$ modulo $\ZZ/2\ZZ$:
\begin{equation}
  A= \mat a,b,c,d \mbox{ tel que } ad-bc=+1 \mbox{ et } A\sim -A.
\end{equation}
Nous allons partir directement de ce groupe pour reconstruire à la fois l'espace hyperbolique et sa variété des géodésiques.
L'algèbre de Lie de $\SL(2,\RR)$,
\cad son espace tangent en l'identité (dans l'espace vectoriel des matrices $2\times 2$),
est le sous-espace vectoriel des matrices $2\times 2$ de trace nulle,
noté $\sl(2,\RR)$.
Tout élément $X$ de $\sl(2,\RR)$ s'écrit donc:
\begin{equation}
  X=\mat\alpha,\beta,\gamma,{-\alpha}.
\end{equation}
On définit sur $\sl(2,\RR)$ la forme bilinéaire suivante:
\begin{equation}
  \label{defnormeDP}
  \scal(X,X')=\undemi \tr(XX') = \alpha\alpha' + {\beta\gamma' + \gamma\beta' \over 2}.
\end{equation}
Il est facile de vérifier qu'elle est non dégénérée et de signature $++-$.
Il suffit par exemple d'identifier la matrice $X$ avec le vecteur,
noté par la même lettre,
$X=(x,y,z)$ défini par:
\begin{equation}
  x=\alpha \quad y={\beta +\gamma \over 2} \quad z={\beta -\gamma \over 2}.
\end{equation}
La réciproque de cette correspondance est évidemment donnée par:
\begin{equation}
  \alpha = x \quad \beta=y+z \quad \gamma=y-z.
\end{equation}
La forme bilinéaire définie plus haut devient:
\begin{equation}
  \scal(X,X')= xx'+yy'-zz'.
\end{equation}
On dit que c'est un pseudo-produit scalaire,
encore appelé «produit scalaire lorentzien».
Par analogie avec avec le cas euclidien,
on note $\norm{X}^2= \scal(X,X)$ bien que cette forme ne soit pas définie positive.
La sous-variété de $\sl(2,\RR)$ d'équation $\norm{X}^2=-1$,
\cad $x^2+y^2-z^2=-1$,
est un hyperboloïde de révolution (autour de $oz$) à deux nappes,
asymptote au cône $C$ d'équation $z^2=x^2+y^2$.
Nous noterons $H^2$ la nappe contenant la matrice
\begin{equation}
  X_0 = \mat 0,1,{-1},0 \sim \vect{0\\ 0\\ 1},
\end{equation}
représentée par le vecteur $(0,0,1)$.
La forme bilinéaire restreinte à $H^2$ est alors définie positive.
On peut le vérifier directement ou encore utiliser l'action adjointe de $\SL(2,\RR)$,
\cad
\begin{equation}
  (A,X)\mapsto AXA^{-1}.
\end{equation}
Il est clair que le produit pseudo-scalaire est invariant par cette action et donc que $H^2$ est stable.
Par un argument sur la dimension,
il est possible de montrer que $H^2$ est homogène sous l'action de $\SL(2,\RR)$,
autrement dit que pour tout couple de points $X$,
$X'$ dans $H^2$ il existe une matrice $A\in \SL(2,\RR)$ telle que $AX=X'$.
Il suffit alors de vérifier que le produit pseudo-scalaire est positif sur l'espace tangent à $H^2$ au point $X_0$.
Soit $UH^2$ le fibré unitaire tangent à $H^2$:
\begin{equation}
  UH^2=\{(X,U)\mid X\in H^2 \mbox{ et } \scal(X,U)=0\}.
\end{equation}
De la même façon que pour la sphère,
la forme de Cartan s'écrit:
\begin{equation}
  \varpi(\delta Z) = \scal(U,\delta X) = \undemi\tr(U\delta X),\quad \mbox{avec} \quad Z=(X,U).
\end{equation}
On en déduit l'expression de la forme présymplectique:
\begin{equation}
  \omega(\delta Z,\delta'Z) = \undemi\tr(\delta U\delta'X-\delta'U\delta X).
\end{equation}
En utilisant la même méthode que pour la sphère,
on obtient les équations différentielles des géodésiques de cette métrique riemannienne sur $H^2$:
\begin{equation}
  {dX\over ds} = \alpha U \quad \mbox{et} \quad {dU\over ds} = \alpha X,\quad \alpha >0.
\end{equation}
On peut remarquer sur ces équations que la courbure de $H^2$ est égale à $-1$.
Considérons alors la matrice $J\in \slin(2,\RR)$ définie par:
\begin{equation}
  J= \frac{1}{2} [U,X] = \frac{1}{2}\left(UX-XU\right);
\end{equation}
elle est conservée le long des géodésiques;
en effet:
\begin{equation}
  \frac{dJ}{ds} \propto \left[\frac{dU}{ds}, X\right] + \left[U,\frac{dX}{ds}\right] \propto [X,X]+[U,U] = 0.
\end{equation}
D'autre part,
$J$ est à la fois orthogonal (pour le produit pseudo-scalaire) à $X$ et à $U$ (utiliser les propriétés de la trace).
La trajectoire géodésique appartient donc à l'intersection de $H^2$ et du plan orthogonal à $J$,
et comme cette intersection est une courbe,
c'est donc la géodésique elle-même.

\begin{figure}[t]
  \begin{center}
    \includegraphics[width=0.8\textwidth]{figures/fig-disquePoincare.pdf}
  \end{center}
  \caption{Les géodésiques de l'espace hyperbolique $H^2$.}
\end{figure}

Ainsi les géodésiques de $H^2$ sont obtenues comme les intersections de $H^2$ avec les plans passant par l'origine.
Notons que l'application $(X,U)\mapsto J$ nous donne un moyen de réaliser l'espace des géodésiques de $H^2$ comme une sous-variété de $\slin(2,\RR)$.
En effet,
on peut vérifier\footnote{On peut faire le calcul général ou bien utiliser la transitivité de $\SL(2,\RR)$ sur $H^2$ et se placer au point $X_0$.} que $\norm{J}^2=1$:
il suffit de se placer au point $X_0$.
D'autre part,
tout plan orthogonal à un $J$ tel que $\norm{J}^2=1$ coupe $H^2$ (on résout l'équation $J=[U,X]$).
L'espace des géodésiques de $H^2$ s'identifie donc avec la sous-variété des matrices $J$ de $\slin(2,\RR)$ d'équation $\norm{J}^2=1$,
\cad $x^2+y^2-z^2=1$.
C'est un hyperboloïde de révolution à une nappe,
asymptote au cône $C$,
dont les traces avec les plans d'équation $z=c$ sont des cercles de rayon $\sqrt{1+c^2}$.
C'est un autre espace homogène (orbite adjointe) de $\SL(2,\RR)$.
La variété des géodésiques de $H^2$ est donc difféomorphe au cylindre $S^1\times \RR$.
On peut retrouver cette identification avec le cylindre en considérant la {\em représentation de Klein} qui consiste à projeter $H^2$ sur le disque $K^2$ d'équations $z=1$ et $x^2+y^2<1$ suivant les droites passant par l'origine et intérieures au cône $C$.
Les géodésiques sont alors des segments de droites,
traces sur $K^2$ des plans passant par l'origine.
L'espace des géodésiques est donc difféomorphe à la variété des droites,
\cad au cylindre
\begin{equation}
  \cG_{H^2} = \{J\in \RR^3 \mid \norm{J}^2=1\} \simeq TS^1.
\end{equation}
Le lecteur perspicace aura remarqué sans doute l'analogie entre cette construction et celle de la variété des géodésiques de la sphère $S^2$,
notamment dans l'introduction du vecteur $J$,
équivalent de $m$ dans le cas $S^2$.
En effet,
dans ce cas la fonction $(x,u)\mapsto m$ est le {\em moment cinétique} relatif au groupe des isométries $\SO(3)$ pour la sphère $S^2$,
et $(X,U)\mapsto J$ est l'équivalent du moment cinétique du groupe $\SL(2,\RR)$ pour $H^2$ --- en réalité,
si $L$ désigne le {\em moment} associé à l'action de $\SL(2,\RR)$ sur $H^2$,
alors $L=[X,U]=-2J$.
Dans ces deux cas c'est l'abondance de symétries qui rend la construction de l'espace des géodésiques aussi simple (et algébrique).

\begin{exercice}
  Montrer que la forme symplectique s'écrit:
      \begin{equation}
    \omega_J(\delta J,\delta'J) = \frac{1}{2} \scal(J,[\delta J,\delta'J]) = \frac{1}{4}\tr(J[\delta J,\delta'J]).
  \end{equation}
      Utiliser le fait que $(U,J,X)$ est une base «orthonormée» de $\slin(2,\RR)$ et décomposer les vecteurs dans cette base.
  Utiliser cette méthode pour les géodésiques de la sphère $S^2$,
  en identifiant $\RR^3$ avec l'algèbre de Lie de $SO(3)$,
  ensemble des matrices réelles $3\times 3$ antisymétriques,
  grâce au produit vectoriel:
      \begin{equation}
    \label{prodvect}
    X=\vect{x\\ y\\ z} \mapsto j(X)=\left(
    \begin{array}{ccc}
      0  & -z & y  \\
      z  & 0  & -x \\
      -y & x  & 0
    \end{array}
    \right)
  \end{equation}
\end{exercice}

%%====================================================================
\section{Les mouvements du solide}
%%====================================================================

Bien que l'exemple des mouvements d'un solide\index{solide} dans $\RR^3$ ne soit pas du même genre que les précédents,
il n'est toutefois pas sans rapport.
C'est,
de plus,
le terrain privilégié de la théorie symplectique des systèmes complètement intégrables.
Comme on ne trouve pas toujours les développements qui aboutissent aux équations du mouvement des solides dans les ouvrages classiques,
nous les détaillons ici.
Définissons d'abord ce qu'est un solide.
Intuitivement,
c'est un ensemble de points $r=(r_1,\cdots,r_N)$ dont les distances mutuelles sont fixes:
\begin{equation}
  \norm{r_i-r_j} = \cst,\quad \forall i,j=1\cdots N.
\end{equation}
Chacun de ces points est soumis à une force $F_i=F_i(t,r_1,\cdots,r_N)$ et vérifie donc les équations de Newton:
\begin{equation}
  {dr_i\over dt} = v_i, \quad m_i{dv_i\over dt} = F_i, \quad F_i = -{\partial U\over \partial r_i},
\end{equation}
mais en respectant la contrainte solide.
Comment traiter cette contrainte,
c'est ce que nous allons voir.
La construction que nous avons donnée à partir du principe des travaux virtuels,
qui conduit à transformer cette famille d'équations en un problème variationnel,
est encore valable,
avec comme lagrangien:
\begin{equation}
  L(t,r,v) = \sum_{i=1}^N {m_iv_i^2\over 2} -U,
\end{equation}
où
\begin{equation}
  r=(r_1,\cdots,r_N) \quad \mbox{et} \quad v=(v_1,\cdots,v_N)\in \RR^{3N}.
\end{equation}
Nous cherchons toujours les solutions du problème:
\begin{equation}
  \delta a = 0, \quad \mbox{avec}\quad a =\int_a^b L(t,r(t),v(t)) \ dt,
\end{equation}
mais avec les contraintes suivantes sur les variations:
\begin{equation}
  \delta \norm{r_i-r_j}=0.
\end{equation}
Si l'on veut éviter maintenant des calculs pénibles et quasiment inextricables,
il faut remarquer que la condition de rigidité signifie qu'à chaque instant il existe une isométrie $g_t$ de l'espace euclidien $\RR^3$ telle que:
\begin{equation}
  r_i(t)= g_t (r_i), \quad \forall i=1,\ldots, N.
\end{equation}
Rappelons que le groupe des isométries de $\RR^3$,
que nous noterons $E$,
est appelé {\em groupe des déplacements euclidiens}\index{déplacements euclidiens},
qu'un élément $g\in E$ est défini par une rotation $A\in \SO(3)$ et un vecteur $T\in \RR^3$,
et que son action sur $\RR^3$ est donnée par:
\begin{equation}
  g=(A,T)\in E,\quad r\in \RR^3 \quad g(r)=Ar+T.
\end{equation}
En d'autres termes,
la contrainte solide signifie simplement que le point
$$
  r=(r_1,\cdots,r_N)
  $$
évolue sur une orbite du groupe $E$ agissant sur $\RR^{3N}$.
Si le nombre de points est suffisant et s'ils ne sont pas tous alignés,
alors le déplacement $g_t$ qui envoie les $r_i$ sur $r_i(t)$ est unique.
Autrement dit,
à tout chemin $t\mapsto r=(r_1,\cdots,r_N)$ dans $\RR^{3N}$ est associé un seul chemin $t\mapsto g(t)=g_t$ dans $E$.
De plus ce chemin est différentiable comme l'est $t\mapsto r$.
Toute variation $\delta r$ compatible avec les contraintes est obtenue par une variation quelconque du chemin $t\mapsto g_t$:
\begin{equation}
  \delta r_i(t)=\delta g(t) (r_i).
\end{equation}
En choisissant une origine $r$ de l'orbite,
et puisque
\begin{equation}
  \label{eqAcTEucl}
  v(t) = {dg(t)\over dt}(r),
\end{equation}
on peut écrire l'action $a$ directement sur le groupe $E$
\begin{equation}
  a =\int_a^b \tilde L (t,g(t),\nu(t)) \ dt,
\end{equation}
où $\tilde L$ est l'image réciproque de $L$ par l'application définie sur $\RR\times TE$ à valeurs dans $\RR\times T\RR^{3N}$:
\begin{equation}
  \label{eqAcTEucl2}
  \hat r : (t,g,\nu) \mapsto (t,g(r),\nu( r)),\quad g\in E,\quad \nu\in T_gE,
\end{equation}
\cad:
\begin{equation}
  \tilde L(t,g,\nu)=L(t,g(r),\nu(r)).
\end{equation}
Dans le cas qui nous préoccupe le vecteur $\nu$ est défini par deux vecteurs $\alpha \in T_A\SO(3)$ et $\tau\in \RR^3$ de telle sorte que:
\begin{equation}
  \nu=(\alpha ,\tau), \quad r_i\in \RR^3 \quad \Rightarrow \quad \nu(r_i) = \alpha r_i+\tau.
\end{equation}
Nous sommes donc ramenés à la situation générale d'un problème variationnel avec,
comme variété de configuration,
le groupe de Lie des déplacements euclidiens de l'espace $\RR^3$.
Nous avons donc à notre disposition toutes les constructions qui ont été décrites jusqu'à présent,
en particulier l'expression de la forme de Cartan $\varpi$ sous sa forme non homogène (\ref{CartanNonHom}):
\begin{equation}
  \varpi(\delta y) = \tilde p(\delta g) -\tilde h\delta t,\quad \tilde y=(t,g,\nu),
\end{equation}
où:
\begin{equation}
  \tilde p= d\tilde L_g\in T^*_gE, \quad \mbox{et} \quad \tilde h(t,g,\nu) = h(t,g(r),\nu(r)).
\end{equation}
On obtient immédiatement la valeur de $\tilde L$:
\begin{equation}
  \tilde L(t,g,\nu) = \frac{1}{2} \tr \alpha J \bar \alpha + \scal(\alpha c,\tau) + {m\over 2}\norm{\tau}^2 -\tilde U,
\end{equation}
la matrice $J$ étant la «matrice des moments»:
\begin{equation}
  J=\sum_{i=1}^N m_i r_i\bar r_i,
\end{equation}
le vecteur $c$ et le scalaire $m$
\begin{equation}
  c = \sum m_i r_i,\quad m=\sum_{i=1}^N m_i,
\end{equation}
étant respectivement le centre de gravité du solide et sa masse totale.
Enfin,
$\tilde U(t,g) = U(t,g(r))$ est le potentiel des forces résultantes sur le solide.
La forme linéaire $p$ a deux composantes,
que nous noterons $p_{\scriptstyle{\SO(3)}}$ et $p_{{\scriptstyle\bf R}^3}$,
qui ont pour valeur:
\begin{equation}
  p_{\scriptstyle{\SO(3)}}(\delta A) = \tr[(J\bar\alpha +\tau \bar c) \delta A] \quad \mbox{et}\quad p_{{\scriptstyle\bf R}^3}(\delta \tau)= \scal(m\tau +\alpha c,\delta\tau).
\end{equation}
La construction de la forme de Cartan dans le cas général des solides libres n'est pas difficile mais fastidieuse,
et nous nous contenterons,
à partir de maintenant,
du cas des mouvements du solides autour d'un point fixe.
Le groupe qui intervient alors n'est plus le groupe des déplacements euclidiens mais son sous-groupe $SO(3)$.
Les termes dépendants de $\tau$ disparaissent et il reste:
\begin{equation}
  \tilde L(t,g,\nu) = \frac{1}{2} \tr \alpha J \bar \alpha -\tilde U.
\end{equation}
La matrice $J$ est définie positive,
et définit donc une métrique riemannienne sur $SO(3)$;
notons:
\begin{equation}
  \scal(\alpha,\alpha')_J = \tr \alpha J\bar\alpha', \quad \tilde L = \frac{1}{2} \norm{\alpha}_J^2 -\tilde U.
\end{equation}
La forme de Cartan devient:
\begin{equation}
  \varpi(\delta y) = \scal(\alpha,\delta A)_J - \tilde h \delta t, \quad \mbox{avec} \quad \tilde h = \frac{1}{2} \norm{\alpha}^2_J +\tilde U, \quad y=(t,A,\alpha).
\end{equation}
Notons alors:
\begin{equation}
  \tilde F = - \grad_J\tilde U \in T_A\SO(3),
\end{equation}
où $\grad_J$ désigne le gradient au sens de la norme définie par $J$,
$\tilde F$ étant la force induite sur $\SO(3)$ par le potentiel $\tilde U$.
La dérivée extérieure de $\varpi$ s'écrit simplement:
\begin{equation}
  d\varpi(\delta y,\delta'y) = \scal(\delta A-\alpha \delta t,\delta'\alpha -\tilde F\delta't)_J - \scal(\delta'A-\alpha \delta't,\delta\alpha -\tilde F\delta t)_J.
\end{equation}
On peut déjà remarquer sur cette expression que si la force exercée sur le solide est nulle (donc $\tilde F=0$),
les équations que l'on obtient sont celles des géodésiques sur le groupe de $\SO(3)$ pour la métrique définie par $J$.
En particulier,
si $J=\id$ les mouvements solides sont les groupes à un paramètre de $SO(3)$.
Autrement dit,
les mouvements d'un solide s'interprètent comme les mouvements d'un point $A$ dans l'espace des matrices $M(\RR^3)$,
muni de la métrique définie par $J$,
soumis à une force $\tilde F$,
et astreint à se mouvoir sur la sous-variété $\SO(3)\subset M(\RR^3)$.
Notons encore que si le terme de force est suffisamment régulier,
la variété symplectique des mouvements solides est le tangent du groupe $\SO(3)$ muni de la forme symplectique image réciproque de la forme de Liouville par la métrique $J$.
Il suffit en effet de faire $t=cst$ et on obtient:
\begin{equation}
  \omega (\delta x,\delta'x) = \scal(\delta A,\delta'\alpha)_J - \scal(\delta'A,\delta\alpha)_J, \quad x=(A,\alpha)\in T\SO(3).
\end{equation}
Le traitement des problèmes de ce type est généralement difficile du fait de la métrique et du plongement de la sous-variété de configuration.
Mais dans le cas d'un groupe,
comme ici,
on peut trivialiser l'espace tangent;
$T\SO(3)$ est isomorphe au produit $\SO(3)\times \so(3)$ grâce à la forme de Maurer-Cartan définie par:
\begin{equation}
  \theta : T_A\SO(3) \to \so(3), \theta(A,\alpha) = (A,Z=A^{-1}\alpha).
\end{equation}
Dans cette trivialisation,
en notant $x=(a,Z)\in \SO(3)\times \so(3)$ et $\delta x$ un vecteur tangent à $X$,
la forme de Cartan devient:
\begin{equation}
  \varpi(\delta x) = \tr A^{-1}\delta A J \bar Z - \left(\frac{1}{2} \tr Z J \bar Z + \tilde U\right) \delta t,
\end{equation}
et sa dérivée extérieure:
\begin{eqnarray}
  d\varpi(\delta x,\delta'x) & = & \tr(\xi'-Z\delta't)J(\delta \bar Z-\bar f\delta t) \\
  \nonumber
  & - & \tr(\xi-Z\delta t)J(\delta'Z-\bar f \delta't) \\
  \nonumber
  & + & \tr[\xi',\xi]J\bar Z,
\end{eqnarray}
où on a défini:
\begin{equation}
  \xi = A^{-1}\delta A, \xi'= A^{-1}\delta'A \in \so(3), \quad f = A^{-1}\tilde F.
\end{equation}
On identifie enfin l'algèbre de Lie $\so(3)$ avec $\RR^3$ grâce à l'opérateur $j$ (définition \ref{prodvect}),
et on obtient:
\begin{eqnarray}
  d\varpi(\delta x,\delta'x) & = & \scal(\Omega'-\zeta \delta' t, \delta \zeta -\phi \delta t)_I - \scal(\Omega-\zeta \delta t, \delta' \zeta -\phi \delta' t)_I \\
  \nonumber
  & + & \scal(\zeta ,\Omega'\wedge \Omega )_I
\end{eqnarray}
où $\scal(\ ,\ )_I$ désigne le produit scalaire dans $\RR^3$ associé au {\em tenseur d'inertie} du solide
\begin{equation}
  I= J-\tr J,
\end{equation}
et où $\Omega$,
$\Omega'$,
$\zeta$ et $\phi$ représentent respectivement $\xi$,
$\xi'$,
$Z$ et $f$:
\begin{equation}
  \xi = j(\Omega)\quad \xi'=j(\Omega') \quad Z=j(\zeta) \quad f= j(\phi).
\end{equation}
Le calcul du noyau de $d\varpi$ est alors immédiat:
il est engendré par le champ de vecteurs solution des équation:
\begin{equation}
  \Omega = \zeta \quad \mbox{et} \quad \frac{d\zeta}{dt} = \phi + I^{-1}(I\zeta \wedge \zeta).
\end{equation}
Ce sont les équations du mouvement du solide\index{solide}.
On a l'habitude de les présenter dans une base diagonalisant le tenseur d'inertie,
soit $\lambda_i$ ($i=1,2,3$) les valeurs propres de $I$,
le système précédent est équivalent au suivant:
\begin{equation}
  \frac{d\zeta_i}{dt} = {\lambda_j-\lambda_k\over \lambda_i}\ \zeta_j\zeta_k, \quad (i,j,k)= \pcirc(1,2,3).
\end{equation}

\begin{exercice}
  Généraliser cette construction pour un groupe de Lie quelconque $G$ et établir les équations du mouvement.
\end{exercice}

\begin{exercice}
  Traiter le problème de 4 points dans $\RR^3$ contraints à se mouvoir en préservant le volume du tétraèdre qu'ils délimitent;
  utiliser l'action du groupe $\SL(3,\RR)$.
\end{exercice}

\begin{remarque}
  Lorsque l'on veut décrire le solide par une distribution continue de points et non plus par une distribution finie,
  on le représente par un compact $S\subset\RR^3$,
  muni d'une densité de matière,
  \cad une mesure $dn$.
  L'espace de configuration n'est plus $Q=\RR\times \RR^{3N}$ mais $Q=\RR\times \Cinf(S,\RR^3)$.
  Chaque application $r:S\to \RR^3$ devient un état du solide.
  Le lagrangien du problème est modifié:
      \begin{equation}
    \label{defLinfini}
    L(t,r,v) = \int_S \left[\frac{1}{2} m\norm{v}^2 - U\right] dn,
  \end{equation}
      où $v$,
  comme $r$,
  est une application de $S$ dans $T_r\RR^3=\RR^3$,
  représentant la distribution des vitesses des constituants du solide;
  $m$ est une fonction réelle positive définie sur $S$,
  représentant la distribution de masse.
  Les forces sont locales,
  ce qui explique le terme de potentiel sous forme d'intégrale.
  On en déduit la forme de Cartan associée:
      \begin{equation}
    \label{fCartInfini}
    \varpi(\delta y) = \int_S m\scal(v,\delta r)dn - h\delta t,\quad h = \int_S \left[{\frac{1}{2}} m\norm{v}^2 + U\right] dn.
  \end{equation}
      Il faut noter que cette forme de Cartan est définie sur un espace de dimension infinie $\RR\times \Cinf(S,T\RR^3)$,
  où $\Cinf(S,T\RR^3)=T\Cinf(S,\RR^3)$.
  La contrainte solide se traite de la même manière que pour le cas fini.
  Les configurations solides sont encore une orbite du groupe d'Euclide,
  agissant naturellement sur $\Cinf(S,\RR^3)$ par composition:
      \begin{equation}
    \label{defActG}
    g\in E,\quad r \in \Cinf(S,\RR^3) : \quad g(r) = g\circ r.
  \end{equation}
      Les mouvements du solide continu sont les solutions des mêmes équations que pour une distribution finie de points,
  mais avec la matrice des moments suivante:
      \begin{equation}
    J=\int_S m r\bar r \ dn.
  \end{equation}
      On peut traiter,
  de la même manière,
  d'autres types de contrainte en remplaçant le groupe des déplacements euclidiens par un autre sous-groupe des difféomorphismes de $\RR^3$.
  Par exemple,
  pour traiter les mouvements d'un fluide incompressible on choisira le sous-groupe des difféomorphismes qui préservent le volume.
  Une grande partie de la théorie des systèmes complètement intégrables trouve son origine dans la recherche des solutions du mouvement des solides.
  Le lecteur pourra consulter avec profit le beau livre de Michèle Audin \cite{Audin3}.
\end{remarque}
