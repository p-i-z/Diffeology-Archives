%%====================================================================
% MARK: - Chapter 2: Les principes de la mécanique analytique
%%====================================================================

\chapter{Les principes de la mécanique analytique}

Comme nous l'avons montré au chapitre précédent,
les origines de la géométrie symplectique remontent à Lagrange et à son souci de structurer les constructions de la mécanique qu'il avait développées pour répondre à des questions autant concrètes,
comme celles,
par exemple,
sur la stabilité des planètes,
que théoriques,
comme celles posées par Euler sur le calcul des variations.
Nous essayons dans ce chapitre d'en donner une présentation moderne synthétique,
à partir du principe des travaux virtuels,
ainsi que le fait Lagrange.

%%====================================================================
\section{Le principe des travaux virtuels}
%%====================================================================

Il s'agit de décrire le mouvement d'une particule de masse $m$ repérée,
dans l'espace,
par sa position $r$.
Cette particule subit une force extérieure $F$,
au sens de Newton,
comme il a été expliqué plus haut.
Il s'agit donc de résoudre l'équation différentielle ordinaire:
\begin{equation}
  F(t,r(t),v(t))=m{dv(t)\over dt}, \quad v(t)={dr(t)\over dt}.
\end{equation}
Changeons de siècle et interprétons les équations de Newton à partir du «principe des travaux virtuels\index{principe des travaux virtuels}» de d'Alembert:
les équilibres d'un point\index{point matériel} matériel,
soumis à l'influence de forces $F_i$,
sont les solutions de l'équation:
\begin{equation}
  \sum_iF_i\cdot\delta r= 0,
\end{equation}
où $\delta r$ est une variation\index{variation}\footnote{étant donné un point $x$ d'une variété $X$,
ce que nous appelons variation (au sens moderne) est un vecteur tangent $\delta x$ à $X$,
au point $x$,
formellement:
$\delta x\in T_xX$.
Cette notation ancienne a été réhabilitée par Souriau \cite{Souriau5}.} de $r$,
compatible avec les contraintes\index{contrainte (compatibilité)} existantes.
Nous verrons plus loin,
par des exemples,
comment se traite la question des contraintes dans ce genre de problème.

Ce principe des équilibres se nomme {\em principe des travaux virtuels}.
Pour employer le mode de pensée et le langage des mécaniciens:
le {\em travail}\index{travail (d'une force)} d'une force $F$ est,
par définition,
le produit (scalaire) de la force par le déplacement de son point d'appui;
la somme $\sum_iF_i\cdot\delta r$ représente donc le travail de l'ensemble des forces $F_i$,
agissant en $r$,
pour un déplacement {\em virtuel} infinitésimal $\delta r$.
Supposer l'équilibre,
c'est supposer que le travail du système de forces est infinitésimalement nul pour tout déplacement virtuel du point d'appui $\delta r$,
compatible avec les contraintes.
Comme tout principe de ce genre,
le principe des travaux virtuels a pour but de remplacer la répétition de l'analyse par la méthode,
permettant ainsi à notre esprit de se dégager des problèmes déjà résolus pour se concentrer sur les nouveaux.
L'expression de l'équilibre par la formule $\sum_iF_i\cdot\delta r= 0$ trouve son intérêt dans le traitement des contraintes;
par exemple,
si l'on veut traiter l'équilibre d'un point matériel sur une surface d'équation $f={\rm const.}$,
l'équation $\sum_iF_i\cdot\delta r= 0$ couplée à l'équation de contrainte $\grad f \cdot \delta r=0$ donnera immédiatement comme résultat:
«$\sum_iF_i$ proportionnel à $\grad f$»,
où le gradient est ici le gradient ordinaire associé à la métrique euclidienne,
comme le montre l'exemple particulier suivant.

\begin{exemple}
  \label{exsphr}
  Considérons une bille de métal roulant librement sur une sphère\index{sphère} aimantée $S$ ({\em contrainte} à ne pouvoir s'en détacher).
  Supposons que cette bille soit soumise à une gravitation $F$ constante,
  dont la direction s'appelle la verticale.
  Soit $r$ le point occupé par la bille sur la sphère $S$.
  Ce point sera en équilibre,
  d'après le principe des travaux virtuels,
  si et seulement si $F\cdot \delta r = 0$ pour toute variation $\delta r$ du point de contact $r$,
  compatible avec la contrainte,
  \cad pour tout vecteur tangent $\delta r$ au point $r\in S$.
  Il faut donc introduire cette contrainte dans l'équation du problème:
  être tangent à la sphère $S$ au point $r$,
  c'est être orthogonal à $r$,
  autrement dit $F\cdot \delta r = 0$ pour tout $\delta r$ tel que $r\cdot \delta r=0$.
  On déduit donc que $F$ et $r$ sont nécessairement colinéaires.
  Il y a deux points où $F$ et $r$ sont colinéaires:
  les points haut et bas de la sphère $S$.
  Ce sont les deux points d'équilibre de la bille.
  Un autre type de considérations permet de qualifier le point haut d'{\em équilibre instable} et le point bas d'{\em équilibre stable}.
  Voilà illustré,
  dans ce cas particulier,
  comment la méthode repose l'esprit.
\end{exemple}

Lagrange va traiter le problème de dynamique qui se pose à lui --- trouver les solutions des équations de Newton,
\cad trouver un mouvement,
une dynamique --- en le ramenant à un problème de statique\footnote{Lagrange écrit explicitement \cite[volume I p. 223]{Lagrange1}:
\begin{quote}
  «Le {\em Traité de Dynamique} de d'Alembert,
  qui parut en 1743,
  mit fin à ces espèces de défis,
  en offrant une méthode directe et générale pour résoudre,
  ou du moins pour mettre en équation tous les problèmes de la Dynamique que l'on peut imaginer.
  Cette méthode réduit toutes les lois du mouvement des corps à celles de leur équilibre,
  et ramène ainsi la Dynamique à la Statique.»
\end{quote}
Pour ramener les lois de la dynamique à celles de la statique,
Lagrange ne s'est pas borné à reprendre les méthodes de ses prédécesseurs mais il les a généralisées en les simplifiant et en les unifiant.},
de recherche d'équilibre.
Pour cela,
il considère le vecteur
\begin{equation}
  I=-m{dv\over dt}
\end{equation}
comme la {\em force de l'inertie}\index{force d'inertie} à laquelle est soumise la particule lors de son mouvement.
Le système des équations de Newton devient alors une simple condition d'équilibre,
les forces extérieures équilibrant les forces d'inertie à chaque instant.
Le temps s'impose ainsi comme un nouveau paramètre:
l'équilibre n'est plus statique mais dynamique.
En appliquant alors le principe des travaux virtuels les équations de Newton deviennent:
\begin{equation}
  (F+I)\cdot\delta r=0,
\end{equation}
en tout point $r$ du mouvement et pour tout $\delta r$,
compatible avec les contraintes.
Comme Lagrange,
supposons maintenant que le mobile se déplace sans contrainte et que la force $F$ dérive\footnote{Lagrange remarque que cette propriété,
vérifiée pour les systèmes de points en interaction gravitationnelle,
lui permet de démontrer le théorème de conservation de la force vive.
Il en a ensuite fait une condition des lois de la mécanique elle-même.
Lagrange note $Pdp+Qdq+Rdr+\ldots$ le travail des forces extérieures;
il écrit précisément \cite[Tome I, pp. 290--291]{Lagrange1}:
\begin{quote}
  «\ldots Mais cette opération devient encore plus facile,
  lorsque les forces sont telles,
  que la somme des moments,
  \cad la quantité
  $$
    Pdp+Qdq+Rdr+\ldots
    $$
  est intégrable,
  ce qui,
  comme nous l'avons déjà observé,
  est proprement le cas de la nature.»
\end{quote}}
d'un potentiel\footnote{La force $F$ est alors considérée comme un covecteur plutôt que comme un vecteur.} $U$:
\begin{equation}
  F= - {\partial U \over \partial r}.
\end{equation}
On transforme ensuite l'équation des travaux virtuels par quelques manipulations algébriques:
\begin{eqnarray*}
  (F+I).\delta r & = & -{\partial U\over \partial r}(\delta r) - m{dv\over dt}\cdot\delta r \\
  & = & -{\partial U\over \partial r}(\delta r) - {d\over dt}(mv\delta r) + mv{d\over dt}(\delta r) \\
  & = & -{\partial U\over \partial r}(\delta r) - {d\over dt}(mv\delta r) + mv\delta v \\
  & = & \delta \left({mv^2\over 2} - U\right) - {d\over dt}(mv\delta r).
\end{eqnarray*}
Définissons alors:
\begin{equation}
  L= {mv^2\over 2} - U,
\end{equation}
fonction de $(t,r,v)$ définie sur $\RR\times \RR^3\times \RR^3$,
ou tout au moins sur un ouvert de cet espace.
Cette fonction s'appelle le {\em lagrangien}\index{lagrangien} du problème,
elle a été introduite par Lagrange à la fin du dix-huitième siècle,
dans sa {\em Mécanique Analytique} \cite[Tome I, pp. 299--300]{Lagrange1},
où elle est notée $Z$.
L'équation de Newton devient après ce passage à travers le filtre du principe de d'Alembert\index{principe de d'Alembert}:
\begin{equation}
  \delta L = {d\over dt}(mv\delta r).
\end{equation}
On a l'habitude de présenter cette équation sous sa version intégrale,
connue sous le nom de «principe de moindre action» (de Maupertuis, de d'Alembert\ldots).
On intègre les deux membres de cette équation le long d'un intervalle de temps quelconque $[t_a,t_b]$,
et l'on suppose que la variation $\delta r$ est nulle aux extrémités;
on obtient la condition équivalente suivante:
\begin{equation}
  \delta \int_{t_a}^{t_b} L dt = 0, \quad \mbox{pour tout} \quad t\mapsto \delta r, \quad \mbox{tel que} \quad \delta r(t_a)=\delta r(t_b)=0.
\end{equation}
Le lecteur peut vérifier lui-même que les solutions de cette {\em équation aux variations}\index{équation aux variations} sont exactement les solutions de l'équation de Newton originale.
Nous avons simplement interprété les solutions de l'équation de Newton comme les {\em courbes extrémales}\index{courbes extrémales} de l'{\em action lagrangienne}\index{action lagrangienne}:
\begin{equation}
  A([t\mapsto r])=\int_{t_a}^{t_b}L\big(t,r(t),v(t)\big) dt \quad \mbox{avec} \quad v(t) ={dr\over dt}.
\end{equation}
Ainsi,
le principe du calcul des variations consiste à sélectionner,
selon un principe d'{\em extremum},
les {\em mouvement réels} de la particule,
parmi tous ses {\em mouvements virtuels},
\cad tous les chemins possibles $t\mapsto r$.

\begin{remarque}
  Telle quelle cette construction n'a que peu d'intérêt,
  sinon celui d'offrir un principe --- que l'on peut apprécier diversement --- aux lois de la mécanique newtonienne.
  Mathématiquement parlant,
  le passage des équations de Newton aux équations aux variations de d'Alembert--Lagrange n'offre aucun intérêt,
  le problème restant inchangé et la difficulté identique.
  Peut-être le choix de tel lagrangien plutôt qu'un autre a-t-il pu sembler,
  dans certaines conditions,
  plus justifié que le choix explicite de l'expression d'une force?
  Ou bien l'utilisation de la terminologie quasi-mystique «moindre action» peut-elle forcer la foi?
  Le fait est que cette construction a le mérite de nous conduire à la {\em structure symplectique} de l'espace des solutions de ces équations,
  et de mettre en évidence la structure très particulière de cet espace qui aurait pu rester cachée si nous en étions restés à l'expression initiale des équations de Newton.
\end{remarque}

Nous allons maintenant abandonner la lecture rigoureuse des \oe uvres de Lagrange pour adopter un ton un peu plus moderne.
Elargissons dès à présent le cadre des équations de la mécanique en changeant l'espace vectoriel $\RR^3$ pour une variété quelconque $X$,
et définissons précisément les différents objets que nous venons d'introduire:

\begin{definition}
  On appelle {\em lagrangien}\index{lagrangien} (d'ordre 1),
  sur une variété différentielle $X$,
  toute fonction différentiable $L$ définie sur un ouvert $V$ de $\RR\times TX$ à valeurs réelles.
  On appelle {\em action} d'un chemin $\gamma: [a,b]\to X$ l'intégrale:
      \begin{equation}
    A(\gamma) =\int_a^b L\big(t,\gamma (t), \dot\gamma (t)\big) dt.
  \end{equation}
      On appelle {\em solution du problème variationnel} défini par le lagrangien $L$,
  tout chemin $\gamma$ extrémal pour l'action $A$,
  \cad tel que:
      \begin{equation}
    \delta A = 0 \quad \forall \delta \gamma \quad \mbox{avec} \quad \delta\gamma(a)=\delta\gamma(b)=0.
  \end{equation}
\end{definition}

Puisque nous introduisons ici pour la première fois cette notation,
il est peut-être nécessaire d'en préciser le sens.
Une variation $\delta\gamma$ d'un chemin $\gamma$ est le chemin obtenu en dérivant une famille à un paramètre $\gamma_s$ par rapport à $s$,
pour la valeur $s=0$,
autrement dit:
\begin{equation}
  \delta\gamma = \left.{\partial \gamma_s \over \partial s}\right\vert_{s=0}.
\end{equation}
Cette variation est donc,
elle aussi,
un chemin mais tracé dans le fibré tangent $TX$,
au-dessus de $\gamma$,
\cad tel que:
$\delta\gamma(t)\in T_{\gamma(t)}X$.
Puisque $A$ est fonction de $\gamma$,
la famille $\gamma_s$ définit une famille d'actions $A_s=A(\gamma_s)$,
la variation $\delta A$ étant évidemment le résultat de la dérivation:
\begin{equation}
  \delta A = \left.{\partial A(\gamma_s) \over \partial s}\right\vert_{s=0}.
\end{equation}
On présente souvent les équations de ce problème variationnel sous la forme des équations obtenues en calculant naïvement la variation de l'action:
\begin{eqnarray}
  \nonumber
  \delta A & = & \int_a^b \left[{\partial L \over \partial r}(\delta r) + {\partial L \over \partial v}(\delta v) \right] dt \\
  & = & \int_a^b \left[{\partial L \over \partial r} - {d\over dt}\left({\partial L \over \partial v}\right) \right](\delta r) dt + \int_a^b \left[ {d\over dt}\left({\partial L \over \partial v} (\delta r)\right)\right] dt.
\end{eqnarray}
Le dernier terme de cette identité est nul puisque la variation est supposée à support compact;
il reste donc le système:
\begin{equation}
  \delta A = 0 \quad \Leftrightarrow \quad {\partial L \over \partial r} - {d\over dt}\left({\partial L \over \partial v}\right) =0.
\end{equation}
Ces équations sont communément appelées {\em équations d'Euler-Lagrange}.

\begin{remarque}
  Malheureusement si,
  comme on le verra plus loin,
  on peut donner facilement un sens intrinsèque au terme ${\partial L / \partial v}$,
  il n'en est pas de même du terme ${\partial L / \partial r}$.
  En effet,
  s'il avait un sens ce terme désignerait la dérivée partielle du lagrangien $L$ à $v$ fixé:
      \begin{equation}
    \label{eqEulerLagrange}
    {\partial L \over \partial r}(u) = \lim_{\varepsilon\to 0}{L(r+\varepsilon u, v) - L(r,v) \over \varepsilon}.
  \end{equation}
      Or --- il est important de le souligner --- si $v\in T_rX$ alors $v\notin T_{r+\varepsilon u}X$,
  autrement dit:
  cette formule (\ref{eqEulerLagrange}) n'a pas de sens,
  ou alors localement,
  dans les coordonnées d'une carte.
  On pourrait,
  pour éviter ce problème,
  identifier les espaces tangents voisins (définir ce qu'on appelle une {\em connexion}) mais il n'est jamais bon d'introduire une structure notoirement étrangère au problème que l'on traite.
  Nous allons voir immédiatement comment nous pouvons avantageusement nous en passer.
\end{remarque}

%%====================================================================
\section{Homogénéisation du lagrangien}
\label{parHomLag}
%%====================================================================

Même si cela est équivalent,
il est parfois commode de chercher les solutions d'un problème variationnel sous forme de courbes non paramétrées dans l'espace des conditions initiales $(t,x,v)$ plutôt que sous la forme de courbes paramétrées:
$t\mapsto x(t)$,
dans l'espace $X$.
L'espace des $(t,r,v)$ est l'espace naturel de définition du lagrangien $L$ et il est donc tout aussi naturel d'y rechercher les solutions du problème variationnel associé.
Nous le noterons $Y$ et l'appellerons l'{\em espace d'évolution du système}\index{espace d'évolution},
ou encore {\em espace des conditions initiales}\index{conditions initiales}.
Choisissons un paramétrage auxiliaire $s\mapsto \big(t(s),x(s),v(s)\big)$ de la courbe $\{(t,x(t),v(t))\}_t$,
définie dans $Y$ par le chemin $\gamma: t\mapsto x(t)$.
Introduisons les dérivées de $(t,x,v)$ par rapport à $s$ et notons:
\begin{equation}
  \dot t ={dt\over ds}, \quad \dot x ={dx\over ds}, \quad \dot v ={dv\over ds},
\end{equation}
de telle sorte que:
\begin{equation}
  A=\int_{s_a}^{s_b} L\left(t,x,\dot {x\over \ddot}\right)\dot t ds = \int_{s_a}^{s_b} l(q,\dot q) ds,
\end{equation}
avec:
\begin{equation}
  q= \vect{t\\ x},\quad \dot q= \vect{\dot t \\ \dot x},\quad \mbox{et} \quad l(q,\dot q) = L\left(t,r,{\dot r \over \dot t}\right)\dot t.
\end{equation}
Notons par la lettre $Q$ l'espace de la variable $q=(t,x)$ lorsque $x$ parcourt $X$ et $t$ parcourt $\RR$.
Le lagrangien $l$ est maintenant défini sur l'espace tangent\index{espace tangent}\footnote{Les bases de la géométrie différentielle ordinaires sont supposées connues du lecteur,
sinon nous renvoyons aux ouvrages classiques.} $TQ$,
un ouvert de $\RR^4\times\RR^4$;
il ne dépend pas du paramètre $s$ et il est évidemment homogène de degré 1.
Nous pouvons lui appliquer la formule d'Euler:
\begin{equation}
  l(q,\dot q) = p(\dot q), \quad \mbox{où} \quad p= {\partial l\over \partial \dot q}.
\end{equation}
La notation $\partial l/ \partial \dot q$ désigne l'application linéaire tangente de la restriction $l\mid T_qQ$.
C'est donc une application linéaire bien définie,
de $T_qQ$ à valeur réelle,
c'est-à-dire un élément de l'espace vectoriel dual\index{espace dual}\index{espace cotangent} $T^*_qQ$.
Nous noterons $P$ l'application $(q,\dot q)\mapsto (q,p)$,
et l'appellerons {\em application de Legendre}\index{application de Legendre}\footnote{Attention,
cette dénomination n'est pas usuelle.}:
\begin{equation}
  P : TQ \to T^*Q, \quad P(q,\dot q) = (q,p) \quad p={\partial l\over \partial \dot q}=D(l\vert_{T_qQ})_{\dot q}\in T_q^*Q.
\end{equation}
Arrêtons nous un instant sur cette formule.
À ce stade,
nous pouvons oublier comment nous en sommes arrivés là,
oublier notre particule de départ,
l'homogénéisation du lagrangien\index{homogénéisation (du lagrangien)},
et considérer que:
\begin{enumerate}
  \item $q$ est un point courant d'une variété quelconque $Q$,
  \item $\dot q$ est un vecteur tangent à $Q$ au point $q$,
  \item $l$ est une fonction réelle,
  homogène\footnote{Nous dirons homogène pour positivement homogène,
  \cad homogène par multiplication d'un nombre strictement positif } de degré 1,
  définie sur un ouvert de l'espace tangent $TQ$ privé de la section nulle.
\end{enumerate}
Ce sera le cadre général du calcul lagrangien des variations comme nous l'entendons.
Cet espace $Q$ est appelé en mécanique {\em espace des configurations}\index{espace des configurations},
car chaque point $q\in Q$ représente un état possible du système étudié:
une {\em configuration}.
Dans le cas d'une particule,
c'est,
comme nous l'avons vu,
$\RR\times\RR^3$.
Ce sera $\RR\times(\RR^3)^N$ dans le cas d'un système à $N$ particules en interaction,
$\RR\times SO(3)$ dans le cas d'un solide etc.
Dans chacun de ces cas,
un mouvement virtuel est une courbe (non paramétrée) de $Q$,
et parmi eux se trouvent les mouvements réels.

Revenons maintenant à l'écriture de l'action lagrangienne $A$ le long de la courbe $s\mapsto (q,\dot q)$:
\begin{equation}
  A=\int_{s_a}^{s_b}p(\dot q) ds = \int_{s_a}^{s_b}p \left({dq \over ds}\right) ds.
\end{equation}
Considérons la forme\footnote{$\lambda$ est ce que l'on appelle une {\em forme différentielle} sur $Q$,
\cad une fonction différentiable définie sur $TQ$ à valeurs réelles,
et linéaire en restriction à chaque espace tangent $T_qQ$ lorsque $q$ parcourt $Q$.} de Liouville $\lambda$ définie,
sur l'espace cotangent\footnote{Rappelons que l'espace cotangent $T^*Q$ à $Q$,
est l'espace de toutes les formes linéaires définies sur chaque espace tangent $T_qQ$,
lorsque $q$ parcourt $Q$.
Cet espace est naturellement {\em fibré} par la projection {\em point base} qui consiste à associer à une forme linéaire $\alpha$ le point base de l'espace tangent sur lequel elle est définie.
On note formellement $\pi:T^*Q\to Q$.
Rappelons si nécessaire qu'une forme linéaire $\alpha$ définie sur un espace vectoriel réel $E$ est une application linéaire $\alpha: E\to \RR$.
Une forme différentielle $\alpha$ sur $Q$ est une application particulière (une section) de $Q$ dans $T^*Q$.} $T^*Q$ à $Q$,
par:
\begin{equation}
  \label{fl}
  \forall y=(q,p)\in T^*Q,\quad \forall \delta y\in T_yT^*Q \quad : \quad \lambda (\delta y)=p(\delta q),
\end{equation}
où $\delta q$ est la projection du vecteur $\delta y$ sur $Q$;
c'est donc un vecteur tangent à $Q$ au point $q$,
et que nous évaluons sur la 1-forme $p$.
Soit $\varpi$ l'image réciproque de $\lambda$ sur $TQ$,
par l'application de Legendre définie plus haut:
\begin{equation}
  \varpi = P^*\lambda.
\end{equation}
Cette forme $\varpi$ est appelée {\em forme de Cartan}\index{forme de Cartan} associée au lagrangien $l$.
Introduisons le {\em relevé holonome}\index{relevé holonome} du chemin $\gamma$,
qui décrit le mouvement virtuel du système,
sur lequel se fait l'intégration du lagrangien:

\begin{definition}
  Soit $\gamma: s\mapsto q$ un chemin dans $Q$;
  nous appellerons {\em relevé holonome} de $\gamma$ le chemin $s\mapsto(q,\dot q)$ dans $TQ$ tel que $\dot q=dq/ds$.
\end{definition}

L'action $A$ s'écrit alors sous forme condensée:
\begin{equation}
  A=\int_{\bar \gamma} \varpi.
\end{equation}
Toute variation de $\gamma$ entraîne une variation de l'action,
donnée par la formule suivante (voir annexe \ref{AnnStokes}):
\begin{equation}
  \delta A = \int_{\bar\gamma} d\varpi(\delta \bar\gamma) + \int_{\bar\gamma} \varpi(\delta\bar\gamma),
\end{equation}
où $\delta \bar\gamma$ est la variation du relevé holonome $\bar\gamma$ associée à la variation $\delta\gamma$.
Il est peut-être utile de préciser le sens des notations utilisées:
\begin{enumerate}
  \item $\gamma$ est en réalité {\em une courbe} de $Q$:
  c'est un arc $s\mapsto q$,
  {\em modulo} reparamétrage.
  \item $\delta \gamma$ est une {\em variation}\index{variation} de $\gamma$:
  c'est une courbe $s\mapsto \delta q$ du tangent $TQ$,
  telle que $\delta q\in T_qQ$;
  on dit que c'est un {\em relevé} de $\gamma$ dans $TQ$.
  \item La courbe $\delta \bar\gamma$ n'est pas un relevé quelconque dans $TTQ$ de $\bar\gamma$:
  c'est le relevé holonome de la variation $\delta \gamma$,
  autrement dit:
      \begin{equation}
    \delta \bar\gamma : s\mapsto {d\over ds}[\delta \gamma (s)].
  \end{equation}
      Nous dirons que c'est une {\em variation holonome} de $\bar\gamma $.
  \item $d\varpi(\delta\bar\gamma)$ est le contracté de $d\varpi$ avec la variation $\delta\bar\gamma$:
  c'est une 1-forme concentrée sur la courbe $\bar\gamma$.
\end{enumerate}
Compte tenu de ces précisions,
la variation $\delta A$ s'écrit encore:
\begin{equation}
  \label{varAc2}
  \delta A = \int_{s_a}^{s_b} d\varpi\left(\delta \bar\gamma(s),{d\over ds}\bar\gamma(s)\right) ds + [\varpi(\delta \bar\gamma (s))]_{s_a}^{s_b}.
\end{equation}

\begin{remarque}
  Cette écriture est relativement compliquée du fait du nombre d'espaces mis enjeux par cette construction,
  qui va de $Q$ (l'espace de configuration) et son tangent $TQ$ pour la définition du lagrangien et de la forme de Cartan,
  jusqu'à l'espace tangent $TTQ$ de $TQ$ qui héberge les variations nécessaires.
  Mais cette relative complexité ne doit pas nous effrayer,
  car en réalité les choses sont simples.
  Utilisons une carte locale de $Q$,
  $q=(q_i)$;
  soit $p=(p_i)$ les coordonnées de $p$ dans la base duale de la base associée à cette carte.
  Si le chemin $\gamma$ est contenu complètement dans le domaine de cette carte,
  la variation de l'action est donnée par:
      \begin{equation}
    \label{varAc2+}
    \delta A = \int_{s_a}^{s_b} \sum_{i=1}^n [\delta p_i{dq_i\over ds} - {dp_i\over ds}\delta q_i] ds + \sum_{i=1}^n [p_i\delta q_i]_{s_a}^{s_b}.
  \end{equation}
      La lettre $n$ désigne la dimension de l'espace $Q$.
\end{remarque}

Revenons à la formule précédente (\ref{varAc2}) de la variation de l'action,
et notons que:

\begin{proposition}
  Une condition suffisante pour que $\gamma$ soit une solution du problème est que $\bar\gamma$ soit tangent en tout point au noyau de $d\varpi$,
  \cad qu'il soit contenu dans son {\em feuilletage caractéristique}.
\end{proposition}

\begin{demonstration}
  En effet,
  si $\bar\gamma$ est dans le feuilletage caractéristique de $d\varpi$,
  \cad si:
      \begin{equation}
    {d\over ds}\bar\gamma(s)\in \ker d\varpi,
  \end{equation}
      alors la variation $\delta A$ est nulle pour toute variation arbitraire de $\bar\gamma$ (formule \ref{varAc2}),
  et {\em a fortiori} pour des variations holonomes.
\end{demonstration}

Il faut noter que le feuilletage caractéristique de $d\varpi$ est au moins de dimension deux sur $TQ$;
en effet,
$l$ étant homogène de degré un sur $TQ$,
l'application de Legendre est homogène de degré zéro,
\cad invariante sous l'action des dilatations positives $(q,\dot q)\mapsto (q,\alpha\dot q)$,
avec $\alpha >0$:
\begin{equation}
  P(q,\alpha \dot q) = P(q,\dot q) \quad \forall (q,\dot q)\in TQ-Q \quad \forall\alpha >0.
\end{equation}
L'invariance de l'application de Legendre $P$ nous conduit naturellement à introduire un nouvel espace,
sur lequel P est définie:
l'{\em espace des directions tangentes}\index{espaces des directions} $SQ=(TQ-Q)/]0,\infty[$.

%%====================================================================
\section{L'espace des directions tangentes}
%%====================================================================

La forme de Cartan $\varpi$ est donc invariante par les dilatations positives agissant sur chaque espace tangent $T_qQ-\{0\}$ en tout point $q\in Q$;
d'autre part,
les demi-droites engendrées par ces dilations sont dans le noyau de $\varpi$.
On en conclut (par la formule de Cartan par exemple) que ces demi-droites sont dans le noyau de $d\varpi$.
Puisque $d\varpi$ est antisymétrique,
son rang est pair.
La dimension de $TQ$ étant paire,
son noyau est de dimension paire et donc au moins égal à deux.
Mais nous venons de montrer que nous pouvons nous débarrasser de cette direction inessentielle dans le noyau de $d\varpi$,
qui ne correspond qu'à la liberté qui nous est donnée par l'homogénéité du lagrangien $l$ de reparamétrer les solutions du problème.
Nous considérerons donc l'espace des {\em demi-droites tangentes}\index{demi-droites},
que nous appelons aussi {\em directions tangentes},
\cad le quotient de $TQ$,
privé de la section nulle,
par les dilatations positives $(q,\dot q)\mapsto (q,\alpha\dot q)$,
$\alpha >0$.
Cet espace est encore une variété fibrée sur $Q$;
chaque fibre au dessus d'un point $q$ est l'espace des demi-droites tangentes en $q$;
nous noterons $[\dot q]$ la demi-droite engendrée par $\dot q$,
$\dot q\neq 0$.
Nous noterons $S_qQ$ l'espace des demi-droites tangentes au point $q$,
$S$ étant mis pour {\em sphère} puisque $S_qQ$ est topologiquement une sphère.
Nous noterons $SQ$ le quotient total $[TQ-Q]/]0,\infty[$,
et nous l'appellerons l'{\em espace des directions tangentes};
il est de dimension $2n-1$;
nous noterons $\pi$ la projection de $TQ-Q$ sur $SQ$:
\begin{equation}
  SQ= [TQ-Q]/]0,\infty[ \quad \begin{array}{rrccc}
    \pi : & TQ-Q          & \longrightarrow & SQ           \\
    & \pi(q,\dot q) & \longmapsto     & (q,[\dot q])
  \end{array}
\end{equation}
Puisque l'application de Legendre est invariante par dilatation,
elle se factorise à travers $SQ$:
\begin{equation}
  P_S: SQ \to T^*Q \quad P=P_S\circ \pi : P_S(q,[\dot q])=P(q,\dot q)=(q,p).
\end{equation}
Nous noterons $\varpi_S$ l'image directe sur $SQ$,
par $\pi$,
de la forme de Cartan $\varpi$ définie sur $TQ-Q$,
mais nous conserverons l'appellation {\em forme de Cartan}:
\begin{equation}
  \varpi_S = \pi_*\varpi \quad \mbox{ou} \quad \varpi = \pi^*\varpi_S.
\end{equation}
À tout chemin $\gamma$,
nous pouvons associer son relevé holonome sur $SQ$,
obtenu par projection du relevé holonome sur $TQ$:
c'est la courbe des directions tangentes de $\gamma$.
Nous le noterons $[\bar\gamma]$.
L'expression (\ref{varAc2}) de la variation de l'action le long du chemin $\gamma$ peut-être considérée comme s'appliquant au relevé holonome de $\gamma$ dans $SQ$:
\begin{equation}
  \label{varAc3}
  \delta A = \int_{s_a}^{s_b} d\varpi_S \left(\delta [\bar\gamma(s)], {d\over ds}[\bar\gamma(s)]\right) ds + [\varpi([\delta\bar\gamma (s)])]_{s_a}^{s_b}.
\end{equation}
La variété $SQ$ étant de dimension impaire,
le noyau de $d\varpi_S$ est au moins de dimension un.
Supposons qu'il soit de dimension constante exactement égale à un,
alors on a:

\begin{proposition}
  \label{feuillCar1}
  Soit $l$ un lagrangien homogène de degré un défini sur le tangent d'un espace de configuration $Q$.
  Si la dimension du feuilletage caractéristique de $d\varpi_S$,
  définie sur $SQ$,
  est constante et égale à 1,
  les solutions du problème variationnel associé à $l$ sont les courbes intégrales du feuilletage caractéristique de $d\varpi_S$.
\end{proposition}

\begin{demonstration}
  Les courbes cherchées sont les solutions d'un système différentiel du premier ordre sur $TQ$ comme le montrent les équations d'Euler-Lagrange,
  écrites dans une carte locale de $TQ$:
  $$
    \delta A = 0 \quad \Leftrightarrow \quad {dq\over ds}= \dot q \quad {d\dot q\over ds} = {d\over ds} \left({\partial l\over \partial \dot q}\right).
    $$
  Ainsi,
  par chaque point $(q,[\dot q])\in SQ$ passe une solution unique,
  mais comme nous l'avons vu plus haut,
  la courbe intégrale du feuilletage caractéristique passant par ce point est solution.
  C'est donc la solution cherchée.
  La condition suffisante énoncée plus haut est ainsi devenue nécessaire.
\end{demonstration}

Parce que cette situation particulière est importante,
on introduit un nouveau terme de vocabulaire:

\begin{definition}
  On appelle {\em forme présymplectique}\index{présymplectique} toute 2-forme fermée dont le noyau est de dimension constante.
\end{definition}

\begin{remarque}
  Je voudrais attirer l'attention du lecteur sur le seul point délicat de cette proposition.
  Le problème que nous avons à résoudre est un cas particulier du problème général suivant:
  soit $X$ une variété différentielle et $\beta$ une 1-forme sur $X$,
  appelons {\em action d'un chemin} $\gamma$ de $X$ l'intégrale de $\beta$ sur $\gamma$
      \begin{equation}
    A=\int_\gamma \beta,
  \end{equation}
      La variation de l'action,
  associée à une variation $\delta \gamma$,
  est donnée par la formule générale (voir annexe \ref{AnnStokes}):
      \begin{equation}
    \delta A = \int_\gamma d\beta(\delta \gamma) + [\beta(\delta \gamma)]_{\partial \gamma}.
  \end{equation}
      Il est évident que l'action est extrémale si et seulement si $\gamma$ est tangent en tout point au feuilletage caractéristique de $d\beta$.
  Mais dans ce cas la variation du chemin $\delta \gamma$ est arbitraire:
  c'est un relevé quelconque de $\gamma$ dans $TX$.
  Or,
  dans le cas qui nous occupe (celui d'un lagrangien $l$ défini sur $TQ$) la variation est holonome.
  Nous obtenons un résultat identique seulement parce que nous avons fait l'hypothèse que le feuilletage caractéristique de $d\varpi_S$ est constant,
  de dimension 1,
  et parce que les courbes que nous cherchons sont des solutions d'un système différentiel ordinaire.
  Cela implique en particulier que,
  dans ce cas,
  les solutions du problème général,
  pour $X=TQ$ et $\beta =\varpi_S$,
  sont nécessairement les relevés holonomes de chemins dans $Q$.
\end{remarque}

Ainsi que nous venons de le voir,
dans tous les cas,
l'espace intéressant est l'espace des directions tangentes $SQ$,
ou plutôt l'ouvert de définition $Y$ de la forme de Cartan,
que nous noterons alors $\varpi_Y$.
Comme cet espace est important nous lui donnerons un nom.

\begin{definition}
  Nous appellerons {\em espace d'évolution}\index{espace d'évolution} du système défini par le lagrangien $l$,
  le plus grand ouvert $Y$ de $SQ$,
  sur lequel est définie l'application de Legendre $P_S$,
  et donc la forme de Cartan $\varpi_S$.
\end{definition}

%%====================================================================
\section{La structure symplectique}
%%====================================================================

Nous pouvons faire apparaître maintenant la structure symplectique promise,
définie sur l'espace des mouvements du système.
Nous renvoyons à l'annexe \ref{geomsymp} pour une brève description des objets formels de la géométrie symplectique.
Nous considérerons le cas évoqué au paragraphe précédent d'une 2-forme $d\varpi_Y$ présymplectique.
Rappelons qu'alors une solution,
un mouvement du système,
est une courbe intégrale de son feuilletage caractéristique.
L'espace des mouvements est donc l'ensemble de ces courbes intégrales et nous le noterons:
\begin{equation}
  \cM = Y/\ker d\varpi_Y.
\end{equation}
Cet espace de feuilles n'a pas toujours de bonnes propriétés topologiques,
mais si c'est le cas:
si $\cM$ est une variété telle que la projection associée
\begin{equation}
  \pi:Y\to \cM
\end{equation}
soit une submersion,
alors $\cM$ est naturellement muni d'une unique 2-forme fermée $\omega$ vérifiant:
\begin{equation}
  \pi^*\omega = d\varpi_Y, \quad \ker \omega = \{0\}.
\end{equation}
Rappelons que (voir annexe \ref{geomsymp}):

\begin{definition}
  On appelle forme {\em symplectique}\index{symplectique} toute 2-forme fermée non dégénérée,
  et {\em variété symplectique} toute variété munie d'une forme symplectique.
\end{definition}

\begin{proposition}
  Soit $Q$ une variété et $l$ un lagrangien homogène de degré 1.
  Soit $\varpi_Y$ la forme de Cartan associée,
  définie sur l'espace d'évolution $Y\subset SQ$.
  Si $d\varpi_Y$ est présymplectique et si l'espace des solution $\cM=Y/\ker d\varpi_Y$ est une variété,
  alors il existe sur $\cM$ une forme symplectique $\omega$ définie par $\pi^*\omega = d\varpi_Y$.
\end{proposition}

\begin{demonstration}
  Pour définir $\omega$ en un point $m$ de $\cM$,
  il suffit de donner la valeur qu'elle prend sur deux vecteurs tangents $\delta m$ et $\delta'm$,
  soit:
      \begin{equation}
    \label{defOmega}
    \omega(\delta m,\delta'm) = d\varpi_Y(\delta y,\delta'y),
  \end{equation}
      où
      \begin{equation}
    \label{defOmega-2}
    \pi(y)=m, \pi_*(\delta y)=\delta m, \pi_*(\delta' y)=\delta' m.
  \end{equation}
      Il faut simplement vérifier que cette valeur ne dépend pas du représentant $y$ de $m$ ni des vecteurs tangents $\delta y$ et $\delta'y$,
  pourvu que $\pi_*(\delta y)=\delta m$ et $\pi_*(\delta' y)=\delta' m$.
  Nous avons supposé que la projection $\pi$ est une submersion,
  ce qui implique en particulier que l'on peut trouver une section $\varphi$ de la projection $\pi$,
  définie sur un voisinage $U$ de $m$,
  qui envoie $m$ sur $y$.
  L'image $\varphi(U)$ de cette section est une sous-variété transverse au feuilletage,
  qui coupe chaque feuille qu'elle rencontre en un point et un seul.
  Le théorème de redressement du flot affirme qu'il existe alors un voisinage $V\subset Y$ de $y$,
  difféomorphe au produit d'un intervalle $]-\varepsilon,\varepsilon[$ par $\varphi(U)$,
  qui identifie les segments du type $]-\varepsilon,\varepsilon[\times \{y\}$ à l'intersection de la feuille passant par $y$ avec $V$.
  Grâce à cette identification,
  le vecteur $\delta y$ qui relève $\delta m$ s'écrit $\delta y= (\delta s,\delta m)$,
  et de même $\delta' y= (\delta' s,\delta' m)$.
  Il est clair ainsi que $d\varpi_Y(\delta y,\delta'y)$ ne dépend pas des composantes $\delta s$ et $\delta's$ qui sont dans le noyau de $d\varpi_Y$.
  Il reste à vérifier l'indépendance par rapport au point $y$ choisi,
  ce qui est encore une conséquence du théorème du redressement du flot.
  Soit $z$ un autre point de la feuille $m$,
  $W$ et $\psi$ l'analogue,
  pour le point $z$,
  de $V$ et $\varphi$.
  Pour que $\delta z$ relève $\delta m$ il faut évidemment que $\delta z=(*,\delta m)$,
  où $*$ est quelconque.
\end{demonstration}

\begin{remarque}
  \label{rem_action}
  Il faut à la fois apprécier cette structure qui nous est offerte avec les solutions du problème et relativiser son importance,
  \cad comprendre ce qu'elle représente du problème dont elle est issue.
  Il faut noter en particulier que ce n'est pas parce que $\omega$ est la projection (image directe) d'une forme exacte qu'elle est elle-même exacte.
  Nous verrons quelques exemples du contraire.
  Mais il vrai que localement,
  pour tout mouvement $m$,
  il existe un voisinage $U$ de $m$ sur lequel $\omega$ est exacte;
  nous pouvons d'ailleurs prendre le même voisinage que précédemment sur lequel $\omega=\varphi^*(d\varpi_Y)$.
  Si nous pensons que le rôle d'une 2-forme est d'être intégrée sur une deux-chaîne,
  et si cette deux-chaîne est suffisamment petite pour être contenue dans un voisinage de trivialisation comme $U$ alors
  $$
    \int_\sigma\omega=\int_{\varphi(\sigma)} d\varpi_Y=\int_{\varphi(\partial \sigma)}\varpi_Y.
    $$
  Notons $\gamma$ le contour d'intégration,
  $\gamma =\partial \varphi(\sigma )=\varphi(\partial \sigma)$;
  alors:
      \begin{equation}
    \int_\sigma\omega = \int_\gamma \varpi_Y = \int_\gamma l = A(\gamma).
  \end{equation}
      Autrement dit:
  la {\em forme symplectique c'est l'action};
  ou plutôt,
  c'est tout ce qu'il est possible de se rappeler de l'action originale du système lorsque l'on ne considère que l'espace de ses mouvements.
  Nous verrons plus loin comment cette remarque nous permet de démontrer facilement,
  par exemple,
  la formule de Crofton sur la mesure des droites qui coupent un domaine,
  et donc de l'étendre au cas général d'un problème variationnel.
\end{remarque}

Examinons maintenant sur quelques exemples ce que nous donne cette construction.

\begin{exemple}
  ({\sc Le point matériel})\label{exPtm}\index{point matériel}
  Revenons au point matériel de masse $m$,
  $Q=\RR\times \RR^3$.
  Dans les variables $(t,r,v)\in\RR\times\RR^3\times\RR^3$,
  le lagrangien $L$ vaut:
      \begin{equation}
    L= {mv^2\over 2} -U.
  \end{equation}
      Il induit sur $TQ$ le lagrangien:
      \begin{equation}
    l = {m\dot r^2\over \raisebox{-2pt}{$2\dot t$}} - U\dot t, \quad q=(t,r)\in Q, \quad \dot q=(\dot t,\dot r)\in T_qQ,\quad \dot t\neq 0.
  \end{equation}
      L'application $(q,\dot q)\mapsto p$ est donnée par:
      \begin{equation}
    (q,\dot q)\mapsto p = \left[
    -{m\dot r^2 \over \raisebox{-2pt}{$2\dot t^2$}} - U \quad {m\dot r \over \raisebox{-2pt}{$\dot t$}}
    \right].
  \end{equation}
      L'espace d'évolution est défini par:
      \begin{equation}
    Y=\{(q,\dot q) \mid \dot t\neq 0\}\sim \RR\times T\RR^3,
  \end{equation}
      et s'identifie avec $\RR\times T\RR^3$ en faisant $\dot t=1$;
  la forme de Cartan devient:
      \begin{equation}
    \varpi_Y = mv\cdot dr - h dt \quad \mbox{avec} \quad h = {mv^2\over 2} +U.
  \end{equation}
      La fonction $h$ est appelée {\em hamiltonien du système}\index{hamiltonien}.
  En se rappelant que $\partial U/\partial r=-F$,
  la dérivée extérieure $d\varpi_Y$ vaut:
      \begin{equation}
    d\varpi_Y(\delta y,\delta'y)= \scal(m\delta v-F\delta t,\delta'r-v\delta't) - \scal(m\delta' v-F\delta' t,\delta r-v\delta t),
  \end{equation}
      où $\delta y$ et $\delta'y$ sont deux vecteurs tangents à $Y$ au point $y=(t,r,v)$.
  Comme on peut le constater,
  on retrouve bien les équations de Newton comme équations du noyau de $d\varpi_Y$:
      \begin{equation}
    \delta y = {dy\over ds}\in \ker d\varpi_Y \quad \Leftrightarrow \quad \delta r=v\delta t \quad \mbox{et} \quad m\delta v=F\delta t.
  \end{equation}
      Les courbes intégrales du feuilletage caractéristique de $d\varpi_Y$,
  ou ce qui est équivalent,
  leur projection $y\mapsto (t, r)$ sur $\RR\times\RR^4$,
  sont les mouvements de la particule.
\end{exemple}

\begin{exemple}
  ({\sc La variété des droites})\index{variété des droites}
  L'exemple de la variété des droites est un cas particulier d'espace de géodésiques que nous verrons en toute généralité plus loin.
  Soit $Q=\RR^n$ et $l$ le lagrangien homogène défini sur $TQ\sim \RR^n\times \RR^n$ par:
  \begin{equation}
    l(q,\dot q) = \norm{\dot q}.
  \end{equation}
  Le symbole double barre $\norm{ }$ désigne la norme euclidienne usuelle.
  L'application de Legendre est donnée par:
  \begin{equation}
    P(q,\dot q) = (q,\bar u) \quad \mbox{avec} \quad u = {\dot q \over \norm{\dot q}},
  \end{equation}
  où $\bar u$ désigne le covecteur transposé du vecteur $u$ pour la métrique euclidienne.
  Nous voyons que nous pouvons naturellement identifier l'espace $SQ$ avec le produit direct $\RR^n\times S^{n-1}$,
  $Y=SQ$.
  La forme de Cartan s'écrit:
  \begin{equation}
    \varpi_Y = \bar u dq, \quad y=(q,u)\in Y.
  \end{equation}
  Sa dérivée extérieure est donnée par:
  \begin{equation}
    d\varpi_Y (\delta y,\delta'y) = \scal(\delta u,\delta'q) - \scal(\delta' u,\delta q).
  \end{equation}
  Un vecteur $\delta y$ tangent à $Y$ au point $y$ est dans le noyau de $d\varpi_Y$ si:
  \begin{equation}
    \scal(\delta u,\delta'q) - \scal(\delta'u,\delta q) = 0 \quad \forall \delta'u \in T_uS^{n-1}, \forall \delta'q.
  \end{equation}
  Pour résoudre ce type d'équation avec contrainte,
  on fixe le vecteur tangent $\delta y$ et on considère $d\varpi(\delta y,\cdot)$ comme une 1-forme linéaire,
  que l'on peut noter $d\varpi(\delta y)$.
  Ensuite on remarque que $\delta'u \in T_uS^{n-1}$ est équivalent à $\bar u\delta' u=0$,
  l'équation précédente devenant alors:
  \begin{equation}
    \delta'u\in \ker \bar u \Rightarrow (\delta'q,\delta'u)\in \ker d\varpi(\delta y).
  \end{equation}
  On utilise maintenant le {\em théorème des multiplicateurs de Lagrange}:\index{multiplicateur de Lagrange}
    \begin{theoreme}
    {\rm ({\sc Lagrange})}
    Soit $A: E\to F$ et $B:E\to G$ deux opérateurs linéaires si $\ker A \subset \ker B$ alors il existe $C: F\to G$,
    linéaire,
    tel que $B=CA$.
  \end{theoreme}
    Cela nous permet d'écrire:
  \begin{equation}
    \delta y = (\delta q,\delta u) \in \ker d\varpi \quad \Rightarrow \quad \exists \alpha \in \RR : \delta q= \alpha u, \delta u = 0.
  \end{equation}
  Autrement dit,
  le feuilletage caractéristique de $d\varpi$ est engendré par le champ de vecteurs $\xi$ suivant:
  \begin{equation}
    \ker d\varpi = \RR \xi \quad \mbox{avec} \quad \xi(q,u) =(u, 0).
  \end{equation}
  La courbe intégrale $\Delta(q,u)$ passant par le point $y=(q,u)$ est la droite affine de $\RR^n$,
  de vecteur directeur $u$ et passant par $q$:
  \begin{equation}
    \Delta(q,u) = \{(q+su,u)\in Y \mid s\in \RR\}.
  \end{equation}
  L'espace $\cM$ des solutions du problème variationnel en question est donc l'espace des droites affines de $\RR^n$.
  Pour écrire la forme symplectique,
  caractérisons la droite $\Delta(q,u)$ par $u$,
  évidemment,
  puisqu'il est conservé,
  et par la projection orthogonale $x$ de $q$ sur l'hyperplan orthogonal à $u$,
  autrement dit:
  \begin{equation}
    \Delta(u,q) \sim (u,x) \quad \mbox{avec} \quad x= [\id - u\bar u] q.
  \end{equation}
  Cela nous permet d'identifier $\cM$ avec le fibré tangent à la sphère $S^{n-1}$:
  \begin{equation}
    \cM \simeq TS^{n-1},
  \end{equation}
  qui peut aussi être considéré comme une sous-variété de $Y$,
  section de la projection canonique $\pi:Y\to \cM$.
  Pour déterminer la forme symplectique,
  il suffit de restreindre la forme $d\varpi$ à $TS^{n-1}$:
  \begin{equation}
    \omega(\delta m,\delta'm) = \scal(\delta u,\delta'x) - \scal(\delta'u,\delta x) \quad \mbox{avec}\quad m=(u,x)\in TS^{n-1}.
  \end{equation}
  Cet exemple est le plus simple des exemples d'espaces de géodésiques,
  mais nous en verrons d'autres par la suite.
\end{exemple}

\begin{exemple}
  ({\sc Théorème de Crofton})\index{théorème de Crofton}
  Le théorème de Crofton (dans une version simplifiée) démontre l'existence d'une mesure $\mu$,
  unique à un facteur multiplicatif près,
  définie sur l'espace des droites du plan $\RR^2$,
  invariante par le groupe des déplacements euclidiens $x\mapsto Ax+b$,
  où $A\in \O(2)$ et $b\in\RR^2$.
  Cette situation est facile à généraliser à $\RR^n$.
  On peut montrer ensuite (c'est le théorème proprement dit) que la mesure des droites qui coupent un domaine convexe compact $\Omega$ est proportionnelle au périmètre du bord de $\Omega$.
  Nous allons voir comment ce théorème,
  à l'apparence mystérieuse,
  est une conséquence,
  presque triviale,
  de la nature symplectique de la variété des droites.

    %% TODO: The user needs to provide the modern image file for this figure.
    %% \begin{figure}[ht]
    %% \centerline{\includegraphics{figures/fig-droites.pdf}}
    %% \caption{La variété des droites du plan}
    %% \label{droites}
    %% \end{figure}

    Les constructions précédentes nous indiquent qu'il existe,
  sur l'espace $\cD_n$ des droites orientées de $\RR^n$,
  une forme symplectique naturelle $\omega$.
  Puisque la variété $\cD_n$ est de dimension $2(n-1)$,
  et que la forme $\omega$ est non-dégénérée,
  la puissance extérieure $\vol_\omega=\omega^{\wedge (n-1)}$ est une forme volume de $\cD_n$.
  Ce volume oriente donc naturellement la variété des droites de $\RR^n$.
  Dans notre cas $n=2$,
  le volume est la forme symplectique elle-même.
  Il faut noter à ce propos que la variété des droites (orientées!) du plan qui est,
  nous l'avons vu,
  l'espace tangent au cercle $S^1$,
  est difféomorphe au cylindre\footnote{À ce propos,
  le lecteur peut vérifier que la variété des droites non-orientées est le ruban de M\"obius,
  lui même non-orientable.} $S^1\times\RR$.
  Considérons maintenant un domaine $\Omega$ strictement convexe et compact du plan $\RR^2$,
  bordé par une courbe fermée $\Sigma=\partial\Omega$.
  Soit $\tilde\Omega$ l'ensemble des droites qui coupent $\Omega$;
  c'est un domaine compact de la variété des droites $\cD_2$.
  On peut en effet obtenir $\tilde\Omega$ comme l'image,
  dans $\cD_2$,
  du produit $\Sigma\times\Sigma$ par l'application $j_\Omega$ qui à deux points $(x,y)\in\Sigma\times\Sigma$ associe la droite unique qui passe par ces deux points.
  Cette application est un plongement et n'est définie que sur le produit $\Sigma\times\Sigma$ privé de la diagonale $\Sigma$.
  Son image est un cylindre dont le bord est constitué de deux composantes:
  les sous-variétés des droites tangentes à $\Sigma$,
  orientées dans un sens et dans l'autre.
  Ces deux bords sont obtenus comme les différentes limites,
  $\lim_{y\to x}j_\Omega(y,x)$ et $\lim_{y\to x}j_\Omega(x,y)$,
  quand $x$ parcourt $\Sigma$.
  La mesure des droites qui coupent le domaine $\Omega$ est donc l'intégrale
  \begin{equation}
    \mu(\tilde\Omega)={1\over 2}\int_{\tilde\Omega} \omega = {1\over 2}\int_{\tilde\Omega} d\lambda = {1\over 2}\int_{\partial \tilde\Omega}\lambda
  \end{equation}
  où $\lambda$ est la primitive $\bar u dx$ de $\omega$,
  $u\in S^1$ et $x\in \RR^2$ avec $u.x=0$.
  Le coefficient 1/2 vient de ce que nous ne considérons que les droites {\em modulo} orientation.
  Rappelons que le couple $(u,x)$ représente la droite $D=\{x+tu \mid t\in \RR\}$,
  orientée par $u$.
  Orientons la courbe $\Sigma$,
  \cad choisissons un vecteur unitaire $r\mapsto u$,
  où $r\in \Sigma$.
  Le bord $\partial \tilde\Omega$ du domaine des droites qui coupent $\Omega$ est donc la réunion des images des deux applications:
  \begin{equation}
    r\mapsto (u,x=[\id -u\bar u]r) \mbox{ et } r\mapsto (-u,x=[\id -u\bar u]r).
  \end{equation}
  Ayant convenablement orienté le cylindre $\cD_2$,
  il vient alors:
  \begin{equation}
    \mu(\tilde\Omega) = \int_\Sigma \bar u d([\id -u \bar u] r) = \int_\Sigma \bar u dr - \int_\Sigma \bar u d(u \bar u r);
  \end{equation}
  or:
  \begin{equation}
    \int_\Sigma \bar u d(u \bar u r) = \int_\Sigma (\bar u r)\bar u du + \int_\Sigma d(\bar u r),
  \end{equation}
  et $\bar u du = 0$ car $u$ est unitaire;
  comme la courbe $\Sigma$ est supposée fermée:
  \begin{equation}
    \int_\Sigma d(\bar u r) = \bar u r\mid_{\partial\Sigma} =0.
  \end{equation}
  Il reste donc enfin:
  \begin{equation}
    \mu(\tilde\Omega) = {1\over 2}\int_{\tilde\Omega} \omega = \int_\Sigma \bar u dr =\Longi(\Sigma).
  \end{equation}

    %% TODO: The user needs to provide the modern image file for this figure.
    %% \begin{figure}[t]
    %% \centerline{\includegraphics{figures/fig-droitesII.pdf}}
    %% \caption{Le domaine des droites qui coupent un convexe}
    %% \label{droites_sigma}
    %% \end{figure}

    C'est ce que nous voulions vérifier:
  la mesure des droites qui coupent un domaine strictement convexe compact à bord lisse est proportionnelle au périmètre de son bord.
  Ce qui précède est une spécialisation,
  dans ce cas de la variété des droites du plan,
  de l'argument général de la remarque de la page \pageref{rem_action}.
  Dans le cas général des droites de $\RR^n$,
  on montre de la même manière que la mesure des droites qui coupent un domaine strictement convexe,
  compact à bord lisse est proportionnelle au volume de son bord.
  Les domaines non strictement convexes sont obtenus,
  comme d'habitude,
  par des procédés de limite.
\end{exemple}

\begin{exercice}
  Considérons un domaine non nécessairement convexe $D$,
  obtenu par plongement $\Cinf$ du disque $D^2\subset\RR^2$ dans $\RR^2$ affine.
  Soit $\nu_D$ la fonction suivante,
  définie sur la variété des droites affines non orientées de $\RR^2$:
  $\nu_D(\Delta)$ est la moitié du nombre des points d'intersection de la droite affine $\Delta\subset\RR^2$ avec le bord $\Sigma=\partial D$,
  ce que l'on note aussi:
  $$
    \nu_D(\Delta) = \sharp(\Delta\cap\Sigma).
    $$
  En adaptant les méthodes du cas convexe,
  montrer que,
  sous certaines conditions techniques sur le bord $\Sigma$,
  l'intégrale de la fonction $\nu_D$ sur l'espace des droites non orientées de $\RR^2$ donne encore la longueur du bord $\Sigma$.
\end{exercice}

%%====================================================================
\section{Le point de vue hamiltonien}
\index{hamiltonien}
%%====================================================================

L'intérêt de la construction précédente d'homogénéisation du lagrangien (s'il n'est pas déjà homogène) réside en partie dans ce qu'elle introduit naturellement l'espace des directions tangentes,
qui est essentiel dans la description du problème,
sa résolution et l'introduction de la structure symplectique sur l'espace des solutions.
Mais c'est une variété abstraite,
obtenue par quotient,
et difficile à manipuler concrètement.
Il est plus agréable de travailler sur le fibré tangent.
Une solution consisterait à choisir une section de la projection\footnote{Dans ce paragraphe,
nous supposerons que le lagrangien $l$ est défini sur tout $TQ-Q$,
\cad $Y=SQ$.
Il est facile de l'adapter au cas où $Y$ est seulement un ouvert de $SQ$.} $\pi:TQ-Q\to SQ$.
Cela est possible en effet puisque la fibration principale $\pi$ est à groupe contractile.
On pourrait aussi munir la variété $Q$ d'une métrique riemannienne auxiliaire et choisir le sous-fibré unitaire pour cette métrique.
Mais cette méthode introduirait un objet extérieur au problème,
ce qui n'est pas une bonne idée.
Une autre méthode,
plus naturelle,
consiste à choisir un niveau du lagrangien $l$,
par exemple,
si la valeur $1$ est atteinte:
\begin{equation}
  \Sigma =l^{-1}(1).
\end{equation}
Mais pour que $\Sigma$ soit une bonne section de $\pi$,
il faut s'assurer de deux choses:
\begin{enumerate}
  \item Chaque demi-droite $\alpha \dot q$,
  $\alpha >0 $ et $\dot q\neq 0$,
  coupe $\Sigma$ en un point et un seul.
  \item L'hypersurface $\Sigma$ est une sous-variété de $TQ-Q$,
  \cad que son application linéaire tangente $dl$ ne s'annule nulle part.
\end{enumerate}
La première condition s'interprète simplement:

\begin{proposition}
  Pour que $\Sigma = l^{-1}(1)$ soit une section de la projection $\pi: TQ-Q\to SQ$,
  il faut et il suffit que le lagrangien $l$ soit toujours strictement positif.
\end{proposition}

\begin{demonstration}
  Supposons $l>0$;
  soit $v\in T_qQ-Q$ et $\alpha = l(v)$;
  puisque $\alpha >0$ notons $u= v/\alpha$;
  par homogénéité $l(u)=1$ et donc $u\in\Sigma$.
  Soit $v'$ un autre point de la demi-droite $\RR^*_+v$,
  alors $v'=\beta v$,
  $\beta >0$;
  si $l(v)=l(v')$ alors $\beta =1$,
  $v=v'$.
  Ainsi,
  si $l>0$,
  $\sigma$ coupe chaque demi-droite en un point et un seul.
  Réciproquement,
  soit $v\in T_qQ-Q$,
  soit $u$ le point d'intersection de la demi-droite dirigée par $v$ et de $\Sigma$,
  alors $v=\alpha u$ avec $\alpha >0$ et donc $l(v)=l(\alpha u)=\alpha l(u)=\alpha$ est strictement positif.
\end{demonstration}

Remarquons que si $l$ est une fonction strictement positive,
alors $p=\partial l/\partial \dot q$ ne s'annule jamais,
puisque $l=p(\dot q)$.
On en déduit donc $dl\neq 0$ et l'on a la proposition suivante:

\begin{proposition}
  Si $\Sigma = l^{-1}(1)$ est une section de la projection $\pi:TQ-Q\to SQ$,
  alors c'est une sous-variété lisse de $TQ-Q$.
\end{proposition}

Nous regrouperons les deux propositions précédentes en une seule:

\begin{proposition}
  Pour que $\Sigma = l^{-1}(1)$ soit une section lisse de la projection $\pi: TQ-Q\to SQ$,
  il faut et il suffit que le lagrangien $l$ soit toujours strictement positif.
\end{proposition}

Si cette condition de positivité sur le lagrangien $l$ est réalisée,
nous pouvons identifier $SQ$,
\cad $Y$ à $W$;
c'est ce que nous ferons ici.
En particulier,
la forme différentielle $\varpi_Y$ devient la restriction $\varpi\mid Y$,
sa dérivée extérieure la restriction de sa dérivée extérieure,
etc.
Mais que devient,
dans ce cas,
la condition d'immersion sur l'application de Legendre qui nous permet de munir l'espace des solutions du problème de sa structure symplectique?
Notons d'abord que l'application de Legendre est une immersion si et seulement si sa restriction à chaque fibre $Y_q= T_qQ\cap Y=l_q^{-1}(1)$ est une immersion.
La fibre $Y_q$ est une sous-variété d'un espace vectoriel;
il est donc légitime d'évoquer sa convexité:

\begin{proposition}
  \label{lagPosHom}
  Soit $l$ un lagrangien homogène strictement positif défini sur $TQ-Q$,
  et $Y=l^{-1}(1)$.
  La restriction de l'application de Legendre $P$ à $Y$ est une immersion si et seulement si chaque fibre $Y_q=Y\cap T_qQ$,
  au dessus de $q\in Q$,
  est strictement convexe,
  c'est alors un plongement.
\end{proposition}

\begin{demonstration}
  Notons $E=T_qQ$,
  $S=Y_q$,
  $x$ un point courant de $E$ et $p=dl_q(x)$ la valeur de l'application de Legendre au point $x$.
  Considérons l'application $H$ qui à $x\in S$ associe l'hyperplan tangent $h=\ker p=T_xS$,
  orienté par le vecteur sortant $x$.
  Puisque l'application de Legendre est une immersion,
  $H$ est un difféomorphisme local de $S$ dans la variété des hyperplans orientés de $E$,
  \cad dans la sphère $S^{n-1}$,
  où $n=\dim Q$.
  D'autre part,
  l'application $H$ est surjective:
  par compacité de $S$,
  on peut trouver,
  pour tout hyperplan $h$,
  un hyperplan affine parallèle à $h$ qui ne coupe pas $S$;
  on le rapproche ensuite radialement de $S$:
  il existe un moment où il sera tangent à $S$.
  L'application $H$ est donc un revêtement de $S^{n-1}$.
  Si $n>2$,
  $S^{n-1}$ est simplement connexe,
  $H$ est alors un difféomorphisme global et $S$ est convexe.
  Pour $n=2$,
  on remarque d'une part que ce que nous avons dit implique que $S$ est localement convexe;
  d'autre part,
  $S$ étant radiale,
  le degré de $H$ ne peut être qu'égal à 1,
  et donc $H$ est encore un difféomorphisme global.
\end{demonstration}

Dans les conditions de la proposition précédente,
tout ce qui a été dit sur $Y$ peut être transposé sur l'image $Y^*=P(Y)$;
la forme de Cartan $\varpi$ est alors remplacée par la forme de Liouville $\lambda$ (voir la définition \ref{fl}):

\begin{proposition}
  Soit $l$ un lagrangien homogène strictement positif,
  défini sur une variété $Q$.
  Si l'application de Legendre $P$,
  restreinte à $Y$,
  est une immersion alors les solutions du problème variationnel associé sont en bijection avec les caractéristiques,
  sur l'hypersurface $Y^*=P(Y)$,
  de la dérivée extérieure $d\lambda$ de la forme de Liouville $\lambda$ du cotangent $T^*Q$.
\end{proposition}

\begin{remarque}
  La variété $Y^*$ peut être réalisée comme le niveau 1 d'une fonction $l^*$ définie sur $T^*Q$.
  Nous pouvons d'ailleurs choisir pour $l^*$ l'application réelle homogène de degré 1 définie sur $T^*Q-Q$ qui vaut 1 sur $Y^*$.
  En voici une construction géométrique.
  De façon générale,
  soit $E$ un espace vectoriel,
  $l$ une fonction réelle,
  strictement positive,
  homogène de degré un définie sur $E$ telle que sa différentielle restreinte à $S=l^{-1}(1)$ soit une immersion.
  Associons à toute forme linéaire $p\in E^*$ le point $\dot q$ de $E$ tel que $T_{\dot q}S=\ker p$.
  Ce point est unique par stricte convexité de $S$,
  posons alors:
      \begin{equation}
    \label{def_trans_legendre}
    l^*(p)=p(\dot q),\quad \ker p = T_{\dot q}(S).
  \end{equation}
      Cette fonction est évidemment positive,
  homogène de degré 1,
  et vaut 1 sur $S$.
  Il suffit maintenant de poser $E=T_qQ$ et de faire varier $q\in Q$:
  la fonction $l^*$ est bien homogène de degré 1 et vaut 1 sur $Y^*$.
  Nous pouvons faire de cette construction une définition:
      \begin{definition}
    Nous appellerons {\em transformée homogène de Legendre} de la fonction $l$ la fonction $l^*$ définie par la formule précédente (\ref{def_trans_legendre}),
    \cad:
    $$
      l^*(p)=p(\dot q),\quad \ker p = T_{\dot q}(S).
      $$
  \end{definition}
      Cette transformée de Legendre n'est pas la transformée de Legendre ordinaire de $l$,
  qui est infinie dans le cas des fonctions homogènes de degré un,
  mais,
  comme nous l'avons vu,
  elle est peut être construite de manière analogue.
\end{remarque}

Cette construction est le point de vue dual du point de vue lagrangien dont nous avons parlé jusqu'à présent:
on l'appelle {\em point de vue hamiltonien}.
C'est un cas particulier de la construction générale suivante.
Soit $T^*Q$ l'espace cotangent de $Q$,
$\lambda$ sa forme de Liouville,
sa dérivée extérieure $d\lambda$ est alors symplectique;
en effet,
elle est évidemment fermée puisque exacte mais elle est aussi non dégénérée.
Cela se vérifie directement,
si $(q,p)$ désigne un point courant de $T^*Q$,
$\lambda = pdq$ et $d\lambda$ s'exprime,
dans une carte locale,
de la manière suivante:
\begin{equation}
  d\lambda = \sum_{i=1}^n dp_i\wedge dq_i,
\end{equation}
où $n=\dim Q$;
$d\lambda$ est clairement non dégénérée.
Considérons maintenant une variété symplectique quelconque $(X,\omega)$ et une fonction réelle $f$ définie sur $X$.
Soit $\Sigma$ une hypersurface de niveau $\Sigma=f^{-1}(c)$ et supposons que $\Sigma$ soit une sous-variété de $X$,
\cad $df\neq 0$ en tout point de $\Sigma$.
La restriction de $\omega$ à $\Sigma$ est dégénérée,
ne serait-ce que parce que $\dim \Sigma = 2n-1$ est impaire.
On peut exprimer plus précisément son noyau;
il faut résoudre l'équation:
\begin{equation}
  \delta y\in \ker \omega \mid \Sigma \quad \Leftrightarrow \quad \omega(\delta y, \delta'y)=0, \forall \delta'y \mbox{ tel que }df(\delta'y)=0.
\end{equation}
En appliquant la méthode des multiplicateurs de Lagrange,
on obtient:
\begin{equation}
  \ker \omega \mid\Sigma = \RR\grad_\omega(f) \quad \mbox{avec}\quad grad_\omega(f) = \omega^{-1}(df).
\end{equation}
La notation $\grad_\omega$ mérite peut-être quelque précision:
$\omega$ est une forme non dégénérée,
que l'on peut interpréter,
en tout point $x\in X$,
comme un isomorphisme linéaire entre $T_xX$ et $T_x^*X$:
au vecteur $v\in T_xX$ on associe la forme linéaire $\omega_x(v,\cdot)$;
$\omega_x^{-1}$ est l'isomorphisme linéaire inverse.
Le vecteur $\grad_\omega(f)$ est appelé {\em gradient symplectique}\index{gradient symplectique} de $f$.
Ainsi,
le feuilletage caractéristique de $f$ sur $\Sigma$ est de dimension $1$,
et ses feuilles sont les courbes intégrales du champ de vecteur $\grad_\omega(f)$.
C'est ce cadre général que nous avons retrouvé en construisant la sous-variété $Y^*=P(Y)$,
associé au lagrangien $l$,
et appliqué à $X=T^*Q$ et $f=l^*$.

La construction que nous venons d'exposer n'est évidemment pas la plus générale qui nous permette d'identifier l'espace $SQ$ quotient du tangent $TQ$ avec une sous-variété $W\subset TQ$.
Les conditions exigées sur le lagrangien $l$ sont très particulières et pas toujours vérifiées.
Si l'on prend par exemple l'homogénéisé d'un lagrangien $L(t,x,v)$,
comme dans le paragraphe \ref{parHomLag},
$$
  l(q,\dot q) = L(t,x,\dot x/\dot t)\dot t
  $$
où $q=(t,x)$ et $\dot q= (\dot t,\dot x)$,
la condition de positivité de $l$,
\cad de $L$ (car on prend $\dot t>0$),
n'est généralement pas vérifiée:
pour une particule dans un champ de forces $L(t,x,v) =mv^2/2 +U(t,x)$ la positivité de $L$ est équivalente à celle du potentiel $U$.
Mais dans ce cas la sous-variété privilégiée qui réalise l'espace des directions,
ou plutôt un ouvert de cet espace,
est la sous-variété d'équation $\dot t=1$.
Comme cette situation est la plus générale des situations particulières,
précisons-la davantage.

Revenons avant l'homogénéisation du lagrangien:
soit $X$ une variété et un lagrangien $L$ défini sur $\RR\times TX$,
$q=(t,x)$ et $\dot q= (\dot t,\dot x)$ les variables de l'homogénéisation,
et $l=L(t,x,\dot x/\dot t)\dot t$ le lagrangien homogénéisé.
Choisissons naturellement comme section de l'espace des directions tangentes $SQ$ la nappe d'équation $\dot t=1$;
l'application de Legendre s'exprime sans difficulté:
\begin{equation}
  P(t,x,v) = \left[
  l - {\partial l\over \partial v}(v) \quad {\partial l\over \partial v}
  \right],
\end{equation}
de telle sorte que la forme de Cartan s'écrive:
\begin{equation}
  \label{CartanNonHom}
  \varpi(\delta y) = p(\delta x) - h\delta t, \quad \mbox{avec}\quad p={\partial l\over \partial v} \quad \mbox{et}\quad h= {\partial l\over \partial v}(v) -l.
\end{equation}
On reconnaît en $h$ la transformée de Legendre du lagrangien $L$ et la forme classique de la forme de Cartan du problème variationnel associé.
Dans ce cas,
si la 2-forme $d\varpi$ est présymplectique,
les sections à temps constant sont des cartes de l'espace des solutions.
Si le problème est suffisamment régulier,
ces cartes sont globales,
autrement dit l'espace des solutions est alors isomorphe à l'espace tangent $TX$,
muni de l'image réciproque de la dérivée extérieure de la forme de Liouville $d\lambda$ par l'application $(x,v)\mapsto p$;
on peut écrire:
\begin{equation}
  \omega = dp\wedge dx.
\end{equation}
C'était justement la situation de l'exemple de la particule.
