%%====================================================================
% MARK: - Chapter 0: Introduction
%%====================================================================

\chapter*{Introduction}

Ce livre est le texte d'un cours donné entre 1996 et 1997 à des étudiants de second et troisième cycles allemands dans le cadre des écoles d'été du {\em Studienstiftung}.
Je profite de l'occasion qu'il m'est faite ici pour remercier le professeur Alan Huckleberry qui m'a invité à y participer et qui a permis ainsi la rédaction de ces notes.

Il existe déjà beaucoup de traités de mécanique analytique/symplectique et l'on peut légitimement se demander pourquoi en rajouter ?
La question est pertinente\ldots

Parmi tous ces livres de mécanique,
j'en vois deux qui peuvent faire office de références:
{\em Structure des systèmes dynamiques} de Jean-Marie Souriau \cite{Souriau5} et {\em Méthodes mathématiques de la mécanique classique} de Vladimir~I.~Arnold \cite{Arnold0}.
Sans oublier un livre plus ancien,
qu'il faut impérativement inclure dans cette courte liste :
la {\em Mécanique Analytique} de Joseph-Louis Lagrange \cite{Lagrange1},
édition de 1811/1815.
Même si cela peut paraître désuet;
la mécanique symplectique moderne en est,
il n'y a pas de doute,
l'héritière directe.
La lecture de ce chef-d'œuvre est une source de compréhension profonde pour celui qui veut dépasser le simple stade du savoir-faire technique.

Ces livres sont excellents et peuvent satisfaire la curiosité de chacun comme ils ont satisfait la mienne.
Ils sont tous,
à leur manière,
complets;
cependant,
ils laissent dans l'ombre quelques constructions qui méritaient,
selon moi,
d'être développées davantage.
C'est ce qui a motivé la rédaction de ce livre.

Dans son ouvrage,
Souriau consacre un tout petit chapitre à définir la structure symplectique de l'espace des mouvements d'un système dynamique à partir d'un lagrangien.
Il préfère l'introduire directement à partir du système des forces en présence --- à la Newton,
j'ai envie de dire.
Pourtant,
contrairement à ce que l'on trouve dans la littérature,
sa méthode --- à partir d'un lagrangien homogène --- est ingénieuse et élégante.
Elle évite de nombreuses constructions fastidieuses que l'on trouve habituellement dans les traités de mécanique symplectique.
Elle introduit directement le meilleur cadre pour l'étude des géodésiques d'une variété riemannienne,
ou de systèmes analogues.
En même temps cette méthode ne se résume pas à la construction hamiltonienne ordinaire.
Elle fait naître des questions (auxquelles je n'ai pas répondu dans ce livre) sur les situations critiques de lagrangiens hors du cadre régulier des hypothèses.

L'approche «lagrangienne» des systèmes de la mécanique est suffisamment répandue en physique théorique pour justifier un développement particulier de cette méthode:
c'est ce que j'ai essayé de faire dans la deuxième partie de ce cours.
J'espère qu'en développant cette idée je ne l'ai pas étouffée,
ce qui est souvent le risque de ce genre d'entreprise.

Mais au fond pourquoi faire de la mécanique symplectique?
C'est aussi une question que l'on peut se poser,
surtout après en avoir fait sa préoccupation quasi-quotidienne pendant de longues années.
Une façon de se rassurer,
de se convaincre que l'on n'a pas perdu son temps,
est de se retourner vers l'histoire et l'interroger :
qui a introduit la géométrie symplectique,
pourquoi,
quelles questions a-t-elle permis de résoudre?
C'est cette réponse,
historique,
que j'ai essayé de traduire dans le premier chapitre.

C'est Lagrange qui a introduit les premiers éléments de calcul symplectique entre 1808 et 1811.
Cette structure s'est imposée d'elle-même lorsqu'il a voulu étendre un résultat de Laplace sur la stabilité du grand axe des planètes.
C'est donc vers Lagrange qu'il faut se tourner pour comprendre le sens de cette théorie,
dont le développement moderne,
actuel,
continue d'apporter de nouvelles réponses à d'anciennes questions.

La réduction des systèmes dynamiques possédant des symétries particulières est un de ces problèmes que la géométrie symplectique moderne a permis de résoudre d'une façon tout à fait nouvelle.
À cette question est attaché aujourd'hui le nom d'Emmy Noether et le théorème célèbre qui porte son nom.
Mais la version moderne de son théorème est la construction de l'{\em application moment},
associée naturellement à l'action d'un groupe de Lie préservant une 2-forme fermée.
La nature de cette application moment est assez claire lorsqu'il s'agit du cadre originel du théorème de Noether:
lorsque la 2-forme est issue d'un lagrangien invariant.
Elle est plus obscure dans le cas général de la mécanique symplectique.
Cette (relative) obscurité se retrouve dans les diverses propriétés de variance de cette application moment et des différentes cohomologies qui lui sont associées.

Il suffirait d'«intégrer» la 2-forme en question pour que tout s'éclaire et que la situation particulière du théorème de Noether devienne la situation générale,
et c'est bien ce que l'on peut et doit faire.
Mais bien s\^ur tout a un prix,
et cette simplification nous demandera d'introduire des espaces et groupes bizarres,
quotients irrationnels de la droite réelle,
de construire avec ces groupes des fibrés au-dessus de nos variétés,
espaces qui ne sont plus eux-mêmes tout à fait des variétés.
Mais qu'importe,
à la fin l'application moment se révèle dans toute sa simplicité;
c'est en tous les cas mon opinion,
même si la simplicité est une notion bien subjective et changeante.

Voilà,
ce sont quelques leçons,
plutôt destinées à aiguiser votre curiosité sur le sujet qu'à prétendre la satisfaire.

\vfill
{\hfill {Cruis, Février 2000} \par \hfill {\tt patrick.iglesias@laposte.net}}
