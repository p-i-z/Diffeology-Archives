%%====================================================================
% MARK: - Chapter 4: Invariance du lagrangien et théorème de Nœther
%%====================================================================

\chapter{Invariance du lagrangien et théorème de N\oe ther}

Comme nous l'avons vu dans les exemples précédents,
les groupes de symé\-tries des problèmes nous ont permis de réaliser simplement l'espace des solutions du système et de lui fournir sa structure symplectique.
Cela n'est pas d\^u au hasard,
c'est une conséquence du théorème de Emmy N\oe ther.
Ce théorème associe à tout groupe de symétrie du lagrangien une famille d'intégrales premières.
Dans certaines conditions,
ces intégrales premières sont suffisantes pour séparer les solutions,
comme pour le cas des exemples que nous avons rencontrés précé\-demment.
Nous verrons ensuite à quelle classe de lagrangiens,
presque invariants,
nous pouvons étendre cette construction.
Dans ce cas,
il apparait une classe de cohomologie non triviale,
associée à cette {\em presque invariance}.
Nous retrouverons cette classe de cohomologie au chapitre suivant,
dans l'étude générale de l'application moment.

%%====================================================================
\section{Lagrangien invariant par un groupe de sy\-mé\-tries}
\index{symétries}
%%====================================================================

Considérons d'abord un groupe de Lie\index{groupe de Lie} $G$ qui agit sur une variété de configuration $Q$,
et son action dérivée sur $TQ$:
\begin{equation}
  g : (q,\dot q) \mapsto \big(g(q), dg_q(\dot q)\big).
\end{equation}
Soit $L$ un lagrangien homogène défini sur $TQ$.
Supposons que l'action de $G$ préserve le lagrangien:
\begin{equation}
	\forall g\in G \quad L(q,\dot q)=L(g(q),dg_q(\dot q)).
\end{equation}
L'application de Legendre est alors équivariante sous l'action de $G$:
\begin{equation}
	\forall \delta \dot q\in T_qQ \quad dL_q(\delta \dot q) =  dL_{g(q)}(dg_q(\delta \dot q)),
\end{equation}
\cad :
\begin{equation}\label{varP}
	P(g(q),dg_q(\dot q)) = P(q,\dot q) dg_q^{-1}.
\end{equation}
La forme de Cartan $\varpi$ est donc invariante sous l'action dérivée de $G$:
\begin{equation}
	\forall g\in G \quad g^*\varpi = \varpi.
\end{equation}
En effet,
soit $x=(q,\dot q)\in TQ$ et $\delta x\in T_xQ$,
évaluons $g^*\varpi$ sur $\delta x$:
\begin{eqnarray}
	(g^*\varpi)_x(\delta x) & = & \varpi_{g(x)}(dg_x(\delta x)) \\ \nonumber
	& = & P(g(q),dg_q(\dot q))(dg_q(\delta q)) \\ \nonumber
	& = & P(q,\dot q) dg_q^{-1}(dg_q(\delta q)) \\ \nonumber
	& = & P(q,\dot q)(\delta q)\\ \nonumber
	& = & \varpi_x(\delta x)
\end{eqnarray}
Soit $Z$ un vecteur de l'algèbre de Lie $\cG$ de $G$,
soit $Z_Q$ et $Z_{TQ}$ les champs de vecteurs fondamentaux associés,
définis sur $Q$ et $TQ$:
\begin{equation}
	Z_Q(q) = D(g\mapsto g_Q(q))(\id)(Z), \quad Z_{TQ}(x) = D(g\mapsto g_{TX}(x))(\id)(Z).
\end{equation}
Puisque $\varpi$ est invariante sous l'action de $G$,
sa dérivée de Lie par rapport à tout champ de vecteur fondamental est nulle:
\begin{equation}
	\forall Z\in \cG : \quad \DLie_Z\varpi =0
\end{equation}
en appliquant à $\varpi$ la formule de Cartan:
\begin{equation}
	\DLie_Z\varpi = d\varpi(Z_{TQ},\cdot) + d[\varpi(Z_{TQ})],
\end{equation}
on obtient:
\begin{equation}\label{defmom1}
	d\varpi(Z_{TQ}(x),\cdot) = - d[\mu\cdot Z], \quad \mu(x)=  [Z \mapsto \varpi(Z_{TQ}(x))].
\end{equation}
L'application $\mu$ est évidemment à valeur dans le dual $\cG^*$ de l'algèbre de Lie de $G$,
puisque $(Z+Z')_{TQ}= Z_{TQ}+Z'_{TQ}$.
Elle est évidemment différentiable:
$\mu\in C^{\infty}(Y,\cG^*)$,
et s'exprime simplement gr\^ace à l'application de Legendre.
Cette construction donne lieu à la définition suivante.

\begin{definition}
	L'application $\mu$ définie par:
	\begin{equation}\label{defmom2}
	 \mu(q,\dot q)\cdot Z = P(q, \dot q)(Z_Q(q)),
	\end{equation}
	est appelée {\em application moment}\index{application moment}\index{moment} de l'action du groupe $G$ sur $Y$.
\end{definition}

Par linéarité,
l'action de $G$ sur $TQ$ passe sur l'espace des directions tangentes $Y=SQ$;
d'autre part,
l'application moment $\mu$ est clairement invariante par les dilatations:
elle est donc définie aussi sur $Y$,
son intért majeur résidant alors dans le théorème suivant.

\begin{theoreme}[{\sc Emmy N\oe ther}]\index{théorème de N\oe ther}
  Soit $Q$ une variété différentiable,
  $L$ un la\-gran\-gien homogène défini sur $TQ$,
  $\varpi$ la forme de Cartan associée.
  Soit $G$ un groupe de Lie agissant sur $Q$ et préservant le lagrangien $L$.
  Si la dérivée extérieure de la forme de Cartan définie sur $Y=SQ$ est présymplectique,
  alors l'application moment est constante sur les solutions du problème variationnel associé à $L$.
\end{theoreme}

\begin{demonstration}
  On sait que les solutions du problème variationnel sont les caractéristiques\index{caractéristiques} de $d\varpi$ sur $Y$;
  soit $s\mapsto y$ un paramétrage d'une caractéristique $m$,
  $dy/ds\in \ker d\varpi$;
  gr\^ace à la formule (\ref{defmom1}) on a:
  \begin{equation}
	\forall Z\in \cG^* \quad 0 = d\varpi(Z_Y,dy/ds) = - d[\mu(y).Z](dy/ds),
  \end{equation}
	donc $\mu$ est constant le long de $m$.
\end{demonstration}

\begin{remarque}
  L'application moment est la construction,
  dans le cadre de la mécani\-que variationnelle,
  qui associe symétries et invariants.
\end{remarque}

\begin{exercice}
  Calculer l'application moment dans les exemples des géodési\-ques de la sphère $S^2$ et du disque de Poincaré $H^2$.
  Que peut-on dire sur la construction qui a été donnée des espaces des géodésiques,
  dans chacun de ces cas?
\end{exercice}

\begin{remarque}
  La démonstration du théorème de N\oe ther ne nécessite pas que l'action du groupe $G$ sur $TQ$ soit relevée d'une action sur la base $Q$;
  elle s'adapte donc parfaitement au cas d'un groupe agissant sur $TQ$ qui préserve le lagrangien sans autre condition,
  mais la formule du moment (\ref{defmom2}) n'est alors plus valable telle quelle.
\end{remarque}

Si le lagrangien $L$ est invariant sous l'action de $G$,
nous avons vu que la forme de Cartan est elle-mme invariante.
Plus précisément on a

\begin{proposition}
  La forme de Cartan est invariante sous l'action du groupe $G$ si et seulement si le lagrangien est invariant.
\end{proposition}

\begin{demonstration}
  C'est une conséquence immédiate de la formule (\ref{varP}),
  appliquée au vecteur $\delta \dot q= \dot q$:
  \begin{equation}
	P(q,\dot q)(\dot q) =	P(g(q),dg_q(\dot q))dg_q(\dot q) \Rightarrow L(q,\dot q) = 	L(g(q),dg_q(\dot q)),
  \end{equation}
	puisque $L(q,\dot q)= P(q ,\dot q)(\dot q)$.
\end{demonstration}

%%====================================================================
\section{Lagrangien presque invariant}
\label{parLagPrInv}
%%====================================================================

Dans le paragraphe précédent,
nous avons noté que l'invariance du lagrangien par un groupe $G$,
agissant sur $Q$,
entrane l'invariance de la forme de Cartan et donc l'existence d'une application moment constantes sur les solutions du problème.
En d'autres termes,
l'application moment est une fonction définie,
non seulement sur l'espace des conditions initiales $Y$,
mais aussi sur l'espace des solution $\cM=Y/\ker d\varpi$;
nous la noterons de la mme lettre $\mu$:
\begin{equation}
	\mu : \cM \mapsto \cG^*.
\end{equation}
On peut supposer que $\cM$ est une variété,
bien que cela ne soit pas tout à fait nécessaire.
Le groupe $G$ préservant $d\varpi$ préserve son feuilletage caractéristique et  agit donc en réalité sur l'espace des solutions en préservant sa structure symplectique $\omega=\pi_*d\varpi$,
où $\pi$ est la projection naturelle de $Y$ sur son quotient:
\begin{equation}
  \forall g\in G : \quad g^*\omega =\omega.
\end{equation}
L'application moment $\mu$ vérifie alors,
sur $\cM$,
l'identité suivante:
\begin{equation}\label{defmom3}
	\forall Z\in \cG : \quad \omega(Z_{\cM},\cdot) = -d\mu\cdot Z.
\end{equation}
Cette formule est d'ailleurs la définition originale de l'application moment,
dans le cadre général des actions de groupes de Lie sur les variétés symplectiques.
Et comme on peut le constater,
il n'est pas nécessaire pour en arriver là que le groupe $G$ préserve la forme de Cartan $\varpi$,
\cad le lagrangien $L$.
Il suffit que pour tout $Z\in \cG$,
la forme différentielle $d\varpi(Z_{TQ},\cdot)$ soit exacte.
Ce sera le cas,
en particulier s'il existe une fonction $F$ définie sur $G\times TQ$ à valeurs réelles telle que :
\begin{equation}\label{defF}
	g^*\varpi = \varpi + dF(g,\cdot).
\end{equation}
En notant que
\begin{equation}
	\DLie_Z \varpi = df\cdot Z
\end{equation}
où l'application $f$,
définie sur $TQ$ à valeurs dans $\cG^*$,
est donnée par:
\begin{equation}
	f = D(g\mapsto F(g,\cdot))(\id),
\end{equation}
et en utilisant la formule de Cartan pour la dérivée de Lie de $\varpi$,
on déduit l'expression du moment $\mu$:
\begin{equation}\label{defmom4}
	\mu\cdot Z=\varpi(Z_{TQ}) - f\cdot Z = P\cdot Z_Q(q) -f\cdot Z, \quad  \forall Z\in \cG.
\end{equation}
Tout cela donne lieu à cette proposition/définition:

\begin{proposition}
  S'il existe une fonction $F:G\times TQ\to \RR$ telle que,
  pour tout $g\in B$:
  $g^*\varpi = \varpi + dF(g,\cdot)$,
  alors pour tout $Z\in \cG$:
  $d\varpi(Z_{\cM},\cdot) = -d\mu\cdot Z$.
  La fonction $\mu$ est constante sur les caractéristiques de $d\varpi$.
  Elle est encore appelée application moment.
\end{proposition}

\begin{remarque}{\sc(Défaut d'invariance)}\label{varAc}\index{défaut d'invariance}
	Cette application $F$ traduit en réalité le défaut d'invariance de l'action $a$,
	associée à $L$.
	En effet,
	on peut toujours fixer la constante additive de $F$ de telle sorte que:
	\begin{equation}
	a(\gamma)=\int_{\bar \gamma} \varpi \quad \Rightarrow \quad
	a(g_*(\gamma)) =  a(\gamma) + F(g)(q,\dot q)
	\end{equation}
	où $\gamma$ est un chemin quelconque qui relie un point fixe $q_0=\gamma(0)$ à $q=\gamma(1)$,
	tel que $\dot q_0=\dot \gamma(0)=0$ et $\dot q=\dot\gamma(1)$.
	Montrer que la différence $a(g_*(\gamma))-a(\gamma)$ est indépendante du chemin $\gamma$ est d'ailleurs une méthode pour vérifier que l'on est dans les conditions supposées.
\end{remarque}

Cette application $F$ n'est pas quelconque;
en utilisant la loi de groupe on déduit son comportement par rapport à la multiplication dans $G$:
\begin{equation}
	(gg')^*\varpi = g^{'*}(g^*\varpi)\quad \Rightarrow \quad d [F(gg') - F(g)\circ g - F(g')]=0.
\end{equation}
Si on suppose alors $Q$ connexe,
il existe une fonction $c$ définie sur le produit $G\times G$ à valeurs réelles telle que :
\begin{equation}\label{varF}
	F(gg') = F(g)\circ g' + F(g') +c(g,g').
\end{equation}
Mais là encore,
cette fonction n'est pas quelconque;
gr\^ace à l'associativité de l'action du groupe,
on établit l'identité:
\begin{equation}
	c(gg',g'') + c(g,g') = c(g,g'g'') + c(g',g'').
\end{equation}
Cette fonction $c$ est donc un {\em deux-cocycle}\index{cocycle} du groupe $G$ à valeurs réelles:
\begin{equation}\label{cocFc}
	c \in Z^2(G,\RR).
\end{equation}
Ce cocycle attaché au défaut d'invariance du lagrangien n'est pas bien défini.
La fonction $F$ n'étant définie qu'à une constante près,
nous aurions pu choisir:
\begin{equation}
	F'(g) = F(g) + k(g), \quad k\in C^\infty(G,\RR).
\end{equation}
Le cocycle associé $c'$ est alors relié à $c$ par:
\begin{equation}
	c'(g,g') = c(g,g') + k(gg')-k(g)-k(g') = c(g,g') +dk(g,g'),
\end{equation}
où $dk$ est le {\em cobord}\index{cobord} de $k$;
$c'$ est donc cohomologue à $c$.
Ce n'est pas le cocycle $c$ qui est défini par le défaut d'invariance du lagrangien mais sa classe $\sigma$:
\begin{equation}
	\sigma = [c]\in H^2(G,\RR).
\end{equation}
Ce cocycle,
ou plut\^ot son cocycle dérivé,
est aussi relié à la variance du moment $\mu$ par le groupe $G$.
Plaçons nous dans le cadre général d'une 2-forme fermée $\omega$ définie sur une variété connexe $X$,
munie d'une action d'un groupe de Lie $G$ et possédant un moment $\mu$,
défini par la formule(\ref{defmom3}).
Soit $Z\in \cG$ et $Z_X$ son champ de vecteur fondamental sur $X$,
notons la variance du champ $Z_X$ sous l'action de $G$:
\begin{equation}\label{varVectFond}
	Z_X(g(x)) = dg_x[\ad(g^{-1}(Z)_X(x)],
\end{equation}
où $\ad$ désigne l'action adjointe de $G$ sur son algèbre de Lie;
elle est définie par:
\begin{equation}\label{varZ}
	\ad(g)(Z) = D(L_g\circ R_{g^{-1}})(\id)(Z),
\end{equation}
où $R_g$ et $L_g$ désignent les multiplications à droite et à gauche par $g$ dans $G$.
Prenons l'image réciproque par $g$,
de chaque terme de l'équation (\ref{defmom3}):
\begin{equation}
	g^*(\omega (Z_X,\cdot )) = - dg^*\mu\cdot Z.
\end{equation}
En appliquant ensuite la formule (\ref{varZ}) et en utilisant l'invariance de $\omega$ par $G$ on obtient:
\begin{equation}
	g^*(\omega (Z_X,\cdot )) = \omega(\ad(g^{-1}(Z)_X,\cdot),
\end{equation}
d'où on déduit,
en appliquant la définition du moment:
\begin{equation}
	d[(g^*\mu - \ad^*(g)(\mu))\cdot Z]=0,\quad \forall Z\in \cG,
\end{equation}
où $\ad^*(g)$ désigne l'action coadjointe de $G$ sur le dual $\cG^*$ de l'algèbre de Lie de $G$:
\begin{equation}
	\ad^*(g)(\mu) : Z \mapsto \mu\cdot \ad(g^{-1})(Z).
\end{equation}
L'application $g^*\mu - \ad^*(g)(\mu)$ est donc constante sur $X$.
Il existe ainsi une application $\Theta$ définie sur $G$ à valeur dans le dual $\cG^*$ telle que:
\begin{equation}
	g^*\mu = \ad^*(g)(\mu) + \Theta(g).
\end{equation}
Sa variance par rapport au groupe $G$ est donnée par:
\begin{equation}
	\Theta (gg') = \ad^*(g)[\Theta(g')] + \Theta(g).
\end{equation}
On reconnat ici un un-cocycle du groupe $G$ à valeurs dans le dual de son algèbre de Lie.
Il existe une relation entre ce un-cocycle $\Theta$ et le deux-cocycle $c$ dont il a été question précédemment (éq. \ref{varF}),
donnée par:
\begin{equation}
	\Theta(g)(Z) = {\partial \over \partial t}\left\{c\left(g^{-1}e^{tZ},g\right) - c\left(g,g^{-1}e^{tZ}\right)\right\}_{t=0}.
\end{equation}
Notons $x=(q,\dot q)\in TQ$.
Gr\^ace à la formule (\ref{varZ}) et à l'hypothèse (éq. \ref{defF}),
on établit une première valeur de $\Theta = g^*\mu -\ad^*(g)(\mu)$:
\begin{equation}
	\Theta(g)\cdot Z = dF(g,\cdot)_x(\ad(g^{-1})(Z)_{TQ}(x)) + f(x)\cdot\ad(g^{-1})(Z) - f(g(x))\cdot Z.
\end{equation}
On calcule alors la variance de la fonction $f$,
dérivée de $F$ en l'identité:
\begin{equation}
	f(g(x))\cdot Z= dF(\cdot,x))_g[d(R_g)_{\scriptstyle\id}(Z)] - dc(\cdot,g))_{\scriptstyle\id}(Z),
\end{equation}
puis,
gr\^ace à la formule (\ref{varF}) de $F$,
on établit ensuite:
\begin{eqnarray}
	dF(g,\cdot)_x(\ad(g^{-1})(Z)_{TQ}(x)) & = & dF(\cdot,x))_g [d(R_g)_{\scriptstyle\id}(Z)] \\ \nonumber
	& - & f(x)\cdot \ad(g^{-1})(Z) \\ \nonumber
	& - & dc(g,\cdot)_{\scriptstyle\id}(\ad(g^{-1}(Z)).
\end{eqnarray}
Il ne nous reste plus qu'à assembler les morceaux pour trouver:
\begin{equation}
	\Theta (g)\cdot Z = dc(\cdot,g))_{\scriptstyle\id}(Z) - dc(g,\cdot)_{\scriptstyle\id}(\ad(g^{-1}(Z)),
\end{equation}
ce qui,
après quelques manipulations supplémentaires,
donne le résultat annoncé.
On dit que $\Theta$ est le {\em cocycle dérivé} de $c$.
Cette construction est générale à la théorie de la cohomologie des groupes.

\begin{remarque}
  Lorsque le lagrangien $L$ est invariant par le groupe $G$,
  le cocycle $c$ est alors trivial;
  le moment $\mu$ est donc équivariant,
  ce qui peut tre vérifié directement.
\end{remarque}

\begin{exemple}
  Gr\^ace à cette construction,
  Souriau \cite{Souriau5} a montré que la masse totale d'un système dynamique galiléen libre peut tre définie comme la classe de cohomologie associée au défaut d'équivariance du moment,
  pour l'action du groupe de Galilée.
  Précisons qu'un {\em système dynamique galiléen libre} est une variété symplectique $(X,\omega)$ munie d'une {\em action hamiltonienne}\index{action hamiltonienne} du groupe de Galilée\index{groupe de Galilée} et qu'une action symplectique est dite {\em hamiltonienne} si elle possède un moment $\mu$.
  Le groupe de Galilée est défini par son action sur $\RR\times \RR^3$ par:
  \begin{equation}
	q=\vect{r\\ t} \mapsto g(q)=\vect{Ar+bt+c \\ t+e }, \quad
	g=\left(
	\begin{array}{ccc}
	  A & b & c \\
	  0 & 1 & e \\
	  0 & 0 & 1
	\end{array}
	\right)
  \end{equation}
	avec $A\in \SO(3)$,
	$b,c\in\RR^3$,
	$e\in \RR$,
	la loi de groupe étant la loi de multiplication des matrices.
	Considérons le cas particulier d'une assemblée de $N$ particules sans interactions de lagrangien homogénéisé:
	\begin{equation}
	L(q,\dot q) = \sum_{i=1}^N \frac{m_i \norm{\dot x}^2}{2\dot t}.
	\end{equation}
	On peut vérifier directement,
	en utilisant les résultats de l'exemple du point matériel (page \pageref{exPtm}),
	que la dérivée extérieure de la forme de Cartan associée est invariante par le groupe de Galilée.
	En utilisant la remarque sur le défaut d'invariance (page \pageref{varAc}),
	on établit immédiatement que la fonction $F$ est donnée par:
	\begin{equation}
	F(g,q) = \scal(A\sum_{i=1}^N m_i r_i,b) + \sum_{i=1}^N m_i \ {b^2 t\over 2}.
	\end{equation}
	On déduit ensuite l'expression du cocycle $c$ associé:
	\begin{equation}
	c = \sum_{i=1}^N m_i \ c_0 ,\quad \mbox{avec} \quad c_0(g,g')=  \scal(Ac',b) + {b^2e'\over 2}.
	\end{equation}
	Bargmann a montré que la cohomologie du groupe de Galilée est de di\-men\-sion~1;
	on peut vérifier,
	d'autre part,
	que le cocycle $c_0$ représente une base de cette cohomologie.
	Le cocycle $c$ est donc nécessairement proportionnel à $c_0$,
	mais on remarque que le coefficient de proportionnalité est la masse totale du système.
	Il est donc légitime de définir,
	de manière générale,
	la masse totale d'un système dynamique galiléen isolé comme la classe de cohomologie du cocycle $c$ associé à l'action du groupe de Galilée,
	indépendamment du fait que la variété symplectique $(X,\omega)$ soit obtenu comme l'espace des solutions d'un problème variationnel.
	On montre d'autre part que la cohomologie du groupe de Poincaré (le groupe des isométries de l'espace de Minkowski) est nulle.
	Cela explique en partie la difficulté qu'il y a à définir la masse d'un système relativiste.
\end{exemple}

L'{\em homomorphisme de Calabi\ }\index{homomorphisme de Calabi} est un homomorphisme réel du groupe des dif\-fé\-omor\-phis\-mes qui préservent l'aire du plan $\RR^2$.
La valeur de cet homomorphisme sur un difféomorphisme est donc invariante par conjugaison:
c'est l'{\em inva\-riant de Calabi}.
Bien qu'il sorte légèrement du cadre de ce paragraphe,
nous allons montrer comment cet homomorphisme est relié à la construction générale que nous avons exposée plus haut.

\begin{exemple}({\sc L'invariant de Calabi}) \label{exCalabi}
	Considérons une variété symplectique exac\-te $(X,\omega)$ avec $\omega = d\varpi$.
	Nous supposerons $X$ simplement connexe:
	$\pi_1(X)=0$.
	Soit $G$ un groupe de Lie agissant sur $X$ par symplectomorphismes:
	$g^*\omega = \omega$.
	Puisque $X$ est simplement connexe,
	il existe une fonction $F:G\times X\to\RR$,
	telle que pour tout $g\in G$:
	$g^*\varpi = \varpi + dF(g)$.
	Nous pouvons mme choisir $F$ différentiable:
	choisissons une origine $o\in X$;
	alors l'intégrale
	$$\int_\gamma g^*\varpi - \varpi$$
	ne dépend de $\gamma$ que par son extrémité $x=\gamma(1)$.
	En effet,
	pour toute variation $\delta \gamma$ nulle au bord ($\delta\gamma(0)=\delta \gamma(1)=0$),
	on a:
	\begin{eqnarray}
	\delta \int_\gamma g^*\varpi - \varpi & = & \int_\gamma d[g^*\varpi	-\varpi](\delta\gamma) \\ \nonumber
	& = & \int_\gamma [g^*d\varpi - d\varpi](\delta \gamma) \\ \nonumber
	& = & \int_\gamma [g^*\omega - \omega ](\delta \gamma) \\ \nonumber
	& = & 0.
	\end{eqnarray}
	Cette intégrale est donc constante sur les composantes connexes des arcs d'extré\-mi\-té $x$,
	\cad sur les classes d'homotopie des arcs pointés en $o$.
	Or nous avons justement supposé $X$ simplement connexe;
	cette intégrale ne dépend donc que de $x$.
	Nous pouvons alors définir $F$,
	sans ambiguté,
	par:
	\begin{equation}
	F(g,x) = \int_o^x g^*\varpi - \varpi.
	\end{equation}
	
	\begin{exercice}
	Montrer que la fonction $F(g,\cdot)$ est différentiable.
	\end{exercice}
	
	En appliquant encore la formule de la variation de l'intégrale le long du chemin $\gamma$,
	mais cette fois avec $\delta\gamma(1)=\delta x$,
	on vérifie que $F$ est une solution de l'équation:
	\begin{equation}\label{eqF}
	g^*\varpi = \varpi + dF(g),
	\end{equation}
	toutes les autres se déduisant de $F$ par une constante additive.
	Soit $c$ le cocycle réel de $G$ qui lui est associé par la formule (\ref{varF}).
	Choisissons maintenant pour $X$ l'espace vectoriel $\RR^n\times \RR^n$,
	muni de sa forme symplectique ordinaire et choisissons pour $G$ le groupe des difféomor\-phis\-mes symplectiques à support compact:
	\begin{equation}
	g\in G \quad \Leftrightarrow \quad \exists K \mbox{ compact} :\quad  g \mid \RR^n\times \RR^n-K = \id.
	\end{equation}
	Bien que $G$ ne soit pas un groupe de Lie de dimension finie,
	ce qui a été dit plus haut est toujours vrai.
	Nous pouvons remarquer que pour $g$ donné la fonction $F(g)$ est constante en dehors du support de $g$;
	en effet si $x$ et $x'$ sont deux points à l'extérieur du support de $g$,
	ils peuvent tre joints par un chemin toujours contenu à l'extérieur du support de $g$ sur lequel $g^*\varpi=\varpi$,
	et donc:
	\begin{equation}
	\forall x,x'\notin \supp(g) : \quad F(g,x)=F(g,x').
	\end{equation}
	On peut définir alors une application $k:G\to \RR$ par:
	\begin{equation}
	k(g) = F(g,x), \quad x\notin \supp(g).
	\end{equation}
	Cette fonction est différentiable dans un sens que nous ne préciserons pas ici et qui est suffisant pour justifier ce qui suit.
	Définissons alors une nouvelle solution $\tilde F$ de l'équation (\ref{eqF}) par:
	\begin{equation}
	\tilde F(g,x) = F(g,x)-k(g);
	\end{equation}
	elle vérifie,
	par construction:
	\begin{equation}
	\forall g\in G, \forall x\notin \supp(g) : \quad \tilde F(g,x) = 0.
	\end{equation}
	En particulier,
	le cocycle associé $\tilde c$ est nul:
	\begin{eqnarray}
	\tilde c(g,g') & = & \tilde F(gg',x)-\tilde F(g,g'(x))-\tilde F(g',x) \\ \nonumber
	& = & \tilde F (gg',\infty) - \tilde F(g,g'(\infty)) - \tilde F (g',\infty)=0.
	\end{eqnarray}
	Pour tout $g\in G$,
	$\tilde F(g)$ est une fonction réelle à support compact que l'on peut donc intégrer sur $X=\RR^n\times \RR^n$,
	soit:
	\begin{equation}
	\kappa(g)= \int_X \tilde F (g,x) \vol, \quad \vol=\omega^{\wedge n}.
	\end{equation}
	Cette fonction $\kappa$ est un homomorphisme du groupe des difféomorphismes symplectiques à valeurs réelles;
	en effet:
	\begin{equation}
	\kappa(gg') = \int_X \tilde F (gg',x) \vol = \int_X \tilde F (g,g'(x)) \vol + \int_X \tilde F(g',x)) \vol,
	\end{equation}
	mais puisque $g^*\omega = \omega$ alors $g^*\vol = \vol$,
	et par un changement de variable:
	\begin{equation}
	\int_X \tilde F (g,g'(x)) \vol = \int_{g(X)} \tilde F (g,x)\ g^*\vol= \int_X \tilde F (g,x) \vol,
	\end{equation}
	d'où on conclut:
	\begin{equation}
	\kappa(gg')=\int_X \tilde F(g,x) \vol + \int_X \tilde F(g',x) \vol = \kappa(g) + \kappa(g').
	\end{equation}
	Cet homomorphisme est appelé {\em invariant de Calabi}.
\end{exemple}

\begin{exercice}
  On considère $\RR^2$ muni de sa forme symplectique ordinaire $\omega = dx\wedge dy$.
  On note $(\rho,\theta)$ les coordonnées polaires.
  \begin{enumerate}
  \item[1.] Donner l'expression de la forme symplectique $\omega$ en coordonnées polaires.
  \end{enumerate}
	On considère une fonction réelle positive $\phi$,
	définie sur l'intervalle $]0,\infty[$,
	à support compact,
	contenue dans l'intervalle $]1,2[$.
	\begin{enumerate}
	\item[2.] Vérifier que l'application $g:(\rho,\theta)\mapsto (\rho, \theta + \phi(\rho))$ définit un symplectomorphisme (à support compact) de $\RR^2$.
	\item[3.] Calculer $\kappa(g)$,
	et en déduire que l'invariant de Calabi n'est pas trivial.
	\item[4.] Généraliser à $\RR^n\times \RR^n$.
	\end{enumerate}
\end{exercice}

\begin{exercice}
	Essayer d'établir le cadre le plus général de la construction de l'invariant de Calabi.
\end{exercice}
