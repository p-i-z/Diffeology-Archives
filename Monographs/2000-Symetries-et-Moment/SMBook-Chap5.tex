%%%%%%%%%%%%%%%%%%%%%%%%%%
%%
%% MARK: Chapter 5
%%
%%%%%%%%%%%%%%%%%%%%%%%%%%

\chapter{L'application moment}

L'application moment\index{application moment}\index{moment} que nous avons introduite sous plusieurs formes dans le chapitre précédent est un outil essentiel dans l'étude des actions symplectiques de groupe de Lie.
Il n'est pas nécessaire,
en réalité,
de considérer les variétés symplectiques,
mais seulement les variétés munies d'une 2-forme fermée.
Comme le théorème de N{\oe}ther le fait remarquer,
l'intérêt du moment en mécanique est justement d'être conservé le long des caractéristiques de $\omega$.
Mais le moment n'est pas vraiment un objet de la géométrie des 2-formes fermées invariantes.
Je voudrais insister sur cet aspect:
on présente souvent le moment comme un objet de géométrie symplectique,
mais c'est une erreur.
Comme nous allons le montrer dans ce chapitre,
le moment est un objet de la géométrie des 1-formes différentielles invariantes.

Nous considérerons une 2-forme fermée $\omega$ définie sur une variété $X$,
munie d'une action différentiable d'un groupe de Lie $G$,
supposé connexe.
On pourra être conduit à étendre la notion de groupe de Lie à la dimension infinie en considérant des sous-groupes du groupe des difféomorphismes préservant la 2-forme $\omega$.
Dans ce cas,
nous considérerons comme algèbre de Lie du groupe l'algèbre de Lie des champs de vecteurs dont la dérivée de Lie annule $\omega$.
Nous le préciserons à chaque fois.
Mais avant tout,
donnons une première définition formelle,
dans ce cadre général,
de l'application moment.

\begin{definition}
  Soit $X$ une variété différentiable munie d'une 2-forme fermée $\omega$,
  invariante sous l'action d'un groupe de Lie $G$.
  On dit que l'action de $G$ est {\em hamiltonienne} si,
  pour tout $Z\in \cG$,
  la 1-forme fermée $\omega(Z_X,\cdot)$ est exacte.
\end{definition}

On peut choisir la primitive de $\omega(Z_X,\cdot)$ de telle sorte qu'elle dépende linéairement de $Z$,
comme le dit la proposition suivante,
qui est aussi la définition originale de l'application moment introduite par J.-M.~Souriau \cite{Souriau5}.
Nous verrons au paragraphe suivant une autre définition du moment qui évite l'introduction d'une base de l'algèbre de Lie.

\begin{proposition}
  [\sc Application Moment]\label{appmomdef}
  Si l'action de $G$ est hamiltonienne,
  il existe une application $\mu\in \Cinf(X,\cG^*)$ telle que:
  %
  \begin{equation}
    \omega(Z_X,\cdot) = -d\mu\cdot Z.
  \end{equation}
  %
  Une telle application est appelée {\em application moment}\index{application moment}\index{moment}.
\end{proposition}


\begin{demonstration}
  Soit $Z_i$,
  $i=1,\ldots,k$,
  une base de l'algèbre de Lie $\cG$ de $G$,
  $G$ étant supposé de dimension finie.
  Par hypothèse,
  il existe une famille de fonctions $\mu_i\in \Cinf(X,\RR)$,
  telles que $\omega(Z_{iX},\cdot)= -d\mu_i$.
  Soit $Z=\sum_i a_iZ_i$ un vecteur quelconque de $\cG$:
  $\omega(Z_X,\cdot) =\omega(\sum_i a_i Z_{iX},\cdot) = \sum_ia_i\omega( Z_{iX},\cdot) = -\sum_i a_id\mu_i$;
  définissons $\mu\cdot Z=\sum_ia_i\mu_i$ et il vient $\omega(Z_X,\cdot) = -d\mu\cdot Z$.
\end{demonstration}

Il faut noter que si $X$ est connexe,
deux moments ne diffèrent évidemment que d'une constante.
Pour des exemples d'utilisation de l'application moment dans la classification des variétés symplectiques,
le lecteur pourra consulter les articles suivants \cite{Iglesias6,Iglesias6bis},
\cite{Delzant1},
\cite{Delzant2} ou encore le livre de M. Audin \cite{Audin1b} qui intègre à peu près tout ce que l'on sait aujourd'hui sur cette question.

%%%%%%%%%%%%%%%%%%%%%%%%%%%% paragraphe 1 %%%%%%%%%%%%%%%%%%%%%%%%%%
\section{Conditions d'existence d'une application moment}

Une des premières questions que l'on se pose lorsqu'on s'intéresse à l'application moment est celle des conditions de son existence.
Soit $Z$ un élément de l'algèbre de Lie $\cG$ de $G$,
on a:
%
\begin{equation}
  \DLie_{Z_X}\omega = 0 \quad \Leftrightarrow \quad d[\omega({Z_X},\cdot)]=0.
\end{equation}
%
Soit $\Phi$ l'application définie sur $\cG$ à valeurs dans $\HdR^1(X)$ qui à $Z$ associe la classe de cohomologie de $\omega({Z_X},\cdot)$:
%
\begin{equation}
  \Phi : \cG \to \HdR^1(X) : \quad \Phi(Z) = [\omega({Z_X},\cdot)].
\end{equation}
%
Cette application est évidemment linéaire.
Supposer l'action de $G$ {\em hamiltonienne},
\cad supposer qu'il existe un moment $\mu$ pour cette action,
se traduit par $\Phi =0$.
D'autre part,
le groupe $G$ agit sur l'algèbre de Lie par l'action adjointe.
Grâce à la variance de la 1-forme $\omega({Z_X},\cdot)$ par rapport à l'action du groupe $G$,
que nous rappelons:
%
\begin{equation}
  g^*(\omega ({Z_X},\cdot))=\omega (\ad(g^{-1})(Z)_X,\cdot),
\end{equation}
%
on déduit:
%
\begin{equation}
  \Phi(\ad(g)(Z)) = [g_*(\omega ({Z_X},\cdot))],
\end{equation}
%
mais comme le groupe est supposé connexe:
$$
  \Phi(\ad(g)(Z)) = [g_*(\omega({Z_X},\cdot))] = [\omega ({Z_X},\cdot)] = \Phi(Z),
  $$
c'est-à-dire:
%
\begin{equation}
  \forall g\in G :\quad \Phi\circ\ad(g)=\Phi.
\end{equation}
%
En dérivant cette fonction par rapport à $g$ on remarque qu'elle s'annule sur l'algèbre de Lie dérivée,
autrement dit qu'elle se projette en une application $\phi$ définie sur l'espace vectoriel quotient $\cG/[\cG,\cG]=H_1(\cG,\RR)$.
En termes de cohomologie,
$\phi$ est une classe de cohomologie de $\cG$ à valeurs dans $\HdR^1(X)$:
%
\begin{equation}
  \phi \in H^1(\cG,\HdR^1(X))=H^1(\cG,\RR)\otimes \HdR^1(X).
\end{equation}
%
Cette classe représente donc l'obstruction\index{obstruction (au moment)} à l'existence du moment pour l'action de $G$.
On remarque en particulier deux conditions suffisantes pour l'existence du moment:
d'une part,
de façon évidente,
si le premier groupe de cohomologie de Rham de $X$ est nul,
d'autre part si le premier groupe de cohomologie de $G$ est nul,
\cad si $\cG=[\cG,\cG]$,
en particulier si $G$ est semi-simple.

Dans ce paragraphe,
nous avons situé l'habitat naturel de l'obstruction $\phi$ à l'existence du moment.
Mais cette obstruction peut être calculée simplement,
dans tous les cas,
comme un sous-groupe de $\cG^*$,
comme nous l'indiquons dans le paragraphe suivant.

%%%%%%%%%%%%%%%%%%%%%%%%%%%% paragraphe 1bis %%%%%%%%%%%%%%%%%%%%%%%%%%
\section{Définition alternative du moment}

Dans le paragraphe précédent nous avons introduit le lieu naturel de l'obstruction à l'existence du moment comme l'espace $H^1(\cG,\RR)\otimes \HdR^1(X)$.
Voici une façon directe de présenter cette obstruction comme un sous-groupe de $\cG^*$,
et qui permet en même temps d'élargir la notion d'application moment.

Considérons l'action d'un groupe de Lie $G$ sur une variété $X$ préservant une 2-forme fermée $\omega$,
la variété $X$ étant supposée connexe.
Choisissons un point base $\xo\in X$ et soit $\Arc(X,\xo)$ l'espace des arcs de $X$ basés en $\xo$:
\begin{equation}
  \gamma\in\Arc(X,\xo) \quad \Rightarrow \quad \gamma\in \Arc(X) \mbox{ et }\gamma(0)=\xo.
\end{equation}
Les arcs considérés,
éléments de $\Arc(X)$,
sont les applications différentiables par morceaux de $[0,1]$ dans $X$.
La différentiabilité par morceaux est requise pour autoriser les opérations de juxtaposition.
Rappelons que le juxtaposé $\gamma\vee\gamma'$ de deux arcs $\gamma$ et $\gamma'$,
basés en $\xo$,
est l'arc,
basé en $\xo$,
défini par:
\begin{equation}
  \gamma\vee\gamma' = \left\{
  \begin{array}{ll}
    \gamma(2t)   & 0\leq t \leq {1\over 2} \\
    \gamma(2t-1) & {1\over 2} \leq t \leq 1
  \end{array}
  \right.
\end{equation}
Rappelons aussi que l'{\em opposé} d'un arc $\gamma$ est l'arc noté $\bar\gamma$,
défini par:
\begin{equation}
  \bar\gamma= \gamma(1-t).
\end{equation}
Soit $Z$ un élément de l'algèbre de Lie $\cG$,
$Z_X$ le champ de vecteurs fondamental sur $X$,
associé à $Z$,
et $\omega(Z_X)$ le contracté de $\omega$ par $Z_X$.
Considérons l'intégrale suivante,
dépendante de $Z\in\cG$ et de $\gamma\in\Arc(X,\xo)$:
\begin{equation}
  \int_\gamma \omega(Z_X) = \int_0^1 \omega\left(Z_X(\gamma(t),{d\gamma(t)\over dt}\right)dt.
\end{equation}
La dépendance en $Z$ est évidemment linéaire;
la fonction $[Z\mapsto \int_\gamma\omega(Z_X)]$ est donc naturellement à valeurs dans $\cG^*$;
nous définissons ainsi l'application:
\begin{equation}
  \renewcommand{\arraystretch}{1.5}
  \begin{array}{rrll}
    \Psi : & \Arc(X,\xo) & \longrightarrow & \cG^*                                          \\
    & \gamma      & \longmapsto     & \left[Z \mapsto\int_\gamma\omega(Z_X)\right]
  \end{array}
  \renewcommand{\arraystretch}{1}
\end{equation}
Cette fonction est additive pour la juxtaposition des arcs:
on vérifie,
par un simple changement de paramètre sous le signe somme,
que
\begin{equation}
  \Psi(\gamma\vee\gamma')=\Psi(\gamma)+\Psi(\gamma').
\end{equation}
Considérons maintenant deux arcs $\gamma$ et $\gamma'$ qui aboutissent tous les deux au point $x=\gamma(1)=\gamma'(1)$;
on a clairement,
en notant que $\Psi(\bar\gamma)=-\Psi(\gamma)$:
$$
  \begin{array}{ccl}
  \Psi(\gamma') & = & \Psi(\gamma)+[\Psi(\gamma')-\Psi(\gamma)] \\
  & = & \Psi(\gamma)+\Psi(\ell), \mbox{ avec } \ell = \gamma'\vee\bar\gamma
  \end{array}
  $$
où $\ell$ est un lacet pointé en $\xo$.
Définissons alors $\Gamma_\omega$ comme l'image par $\Psi$ du sous-espace des lacets de $X$,
pointés en $\xo$
\begin{equation}
  \Gamma_\omega = \Psi \Big(\Lac(X,\xo)\big) = \left\{\int_\ell \omega(Z_X)\mid \ell\in \Lac(X,\xo) \right\}\subset \cG^*.
\end{equation}
Puisque $\Psi$ est additive par juxtaposition des arcs,
son image $\Gamma_\omega$ est un sous-groupe additif de $\cG^*$.
On peut alors définir l'application $\psi$ par passage au quotient de $\Psi$ sur $\cG^*/\Gamma_\omega$ par:
\begin{equation}
  \label{momgene}
  \renewcommand{\arraystretch}{1.5}
  \begin{array}{rrll}
    \psi : & X & \longrightarrow & \cG^*/\Gamma_\omega                                              \\
    & x & \longmapsto     & \left[Z \mapsto\int_{\xo}^x\omega(Z_X)\right]_{\Gamma_\omega}
  \end{array}
  \renewcommand{\arraystretch}{1}
\end{equation}
où $[\ldots]_{\Gamma_\omega}$ représente la classe d'équivalence dans le groupe quotient $\cG^*/\Gamma_\omega$.
D'autre part,
pour tout $Z\in \cG$,
la 1-forme $\omega(Z_X)$ est fermée,
et $\Psi(\ell)$ ne dépend que de la classe d'homologie $[\ell]\in H_1(X,\ZZ)$ du lacet $\ell$.
Le groupe $\Gamma_\omega$ est donc l'image du groupe $H_1(X,\ZZ)$ dans $\cG^*$ par la projection $h_\omega$ de $\Psi\vert_{\Lac(X,\xo)}$,
qui est évidemment un morphisme de groupe abélien:
\begin{equation}
  \forall c\in H_1(X,\ZZ), \quad h_\omega(c) = \Psi(\ell) \mbox{ si } c= [\ell], \quad \Gamma_\omega =h_\omega(H_1(X,\ZZ)).
\end{equation}
Voici quelques propriétés immédiates de cette construction:
\begin{enumerate}
  \item L'action de $G$ est hamiltonienne si et seulement si $\Gamma_\omega=\{0\}$,
  auquel cas $\psi$ est le moment.
  \item Dans tous les cas,
  le quotient $\cG^*/\Gamma_\omega$ est un groupe abélien.
  Ce n'est un groupe de Lie que lorsque $\Gamma_\omega$ est fermé;
  dans ce cas $\cG^*/\Gamma_\omega$ est le produit d'un tore par un espace vectoriel réel.
  Sinon,
  $\cG^*/\Gamma_\omega$ doit être considéré comme un {\em groupe différentiable} au sens général des {\em espaces différentiables} (voir annexe \ref{AnnED}).
  \item Dans tous les cas,
  l'application $\psi$ est différentiable,
  même lorsque $\Gamma_\omega$ n'est pas fermé dans $\cG^*$.
  Cela signifie simplement que,
  pour tout $x\in X$,
  il existe un voisinage $U$ de $x$ et une application différentiable $F:U\to \cG^*$ telle que pour tout $y\in U$:
  $\psi(y)=[F(y)]_{\Gamma_\omega}$,
  ce qui peut s'interpréter aussi comme une simple conséquence de ce qu'une forme fermée est localement exacte.
  \item L'application $\psi$ est invariante sur les caractéristiques\footnote{J'appelle sous-variété {\em caractéristique} de la 2-forme $\omega$ toute sous-variété de $X$ connexe dont l'espace tangent est,
  en tout point,
  contenu dans le noyau de $\omega$.
  Un arc caractéristique est un arc différentiable tangent en tout point au noyau de $\omega$.} de la 2-forme $\omega$:
  étant donnés deux points $x$ et $y$ sur une même caractéristique,
  il suffit de joindre $\xo$ à $x$ par un chemin quelconque,
  et de joindre $\xo$ à $y$ en juxtaposant au chemin précédent un arc joignant $x$ à $y$ tracé entièrement dans la caractéristique.
\end{enumerate}

\begin{exemple}
  Considérons l'exemple suivant:
  $X$ est le tore $T^2=\RR^2/\ZZ^2$;
  la forme $\omega$ est le volume ordinaire $\omega = dx\wedge dy$ (\modulo\ un abus de langage courant).
  Le groupe $T^2$ agit sur lui-même par translation,
  en préservant le volume $\omega$.
  Considérons les deux générateurs canoniques de $H_1(T^2,\ZZ)$:
  \begin{equation}
    a =\left[t\mapsto \left([t] ,1\right)\right] \quad b =\left[t\mapsto \left(1,[t]\right)\right],
  \end{equation}
  où $t$ varie de $0$ à $1$.
  L'algèbre de Lie $\cT^2$ de $T^2$ est évidemment $\RR^2$ et son dual $\cT^{2*}$ est aussi identifié à $\RR^2$ grâce au produit scalaire ordinaire.
  Les images des générateurs $a$ et $b$ de $H_1(T^2,\ZZ)$ par l'homomorphisme $h_\omega$ sont données par:
  \begin{equation}
    h_\omega(a) =(0, -1) \quad h_\omega(b) =(1, 0).
  \end{equation}
  Ainsi,
  l'image $\Gamma_\omega = h_\omega(H_1(T^2,\ZZ))$ est le réseau ordinaire $\ZZ^2\subset\cT^{2*}$,
  engendré par $h_\omega(a)$ et $h_\omega(b)$.
  L'application $\psi$ est un difféomorphisme,
  et le lecteur peut vérifier directement que:
  \begin{equation}
    \psi[x,y] = \big[x h_\omega(a) + y h_\omega(b)\big]\in T^2.
  \end{equation}
  L'espace quotient $\cT^{2*}/\Gamma_\omega$ est donc isomorphe à $T^2$.
\end{exemple}

\begin{exemple}
  Dans l'exemple précédent,
  considérons l'action du groupe additif $\RR$ définie par restriction de l'action de $T^2$ ci-dessus:
  \begin{equation}
    t : [x,y] \mapsto [(x,y) + t (\alpha,\beta)],
  \end{equation}
  où $\alpha$ et $\beta$ sont des réels quelconques.
  Le dual de $\RR$ étant naturellement identifié à $\RR$ lui-même grâce au produit scalaire,
  le groupe $\Gamma_\omega = h_\omega(H_1(T^2,\ZZ))$ est le sous-groupe:
  \begin{equation}
    \Gamma_\omega = \alpha\ZZ +\beta\ZZ \in \RR.
  \end{equation}
  Dans ce cas,
  le quotient $\RR/\Gamma_\omega$ est ou bien un cercle si $\alpha$ et $\beta$ sont commensurables ou bien un tore irrationnel.
\end{exemple}

\begin{exemple}
  On considère le tore $\TT^3=\RR^3/\ZZ^3$.
  L'espace tangent en tout point $X\in\TT^3$ est naturellement identifié à $\RR^3$.
  Soit $\omega$ la 2-forme suivante:
  $$
    \omega_X(u,v) = \scal(\Omega,u\wedge v)
    $$
  où $\Omega$ est un vecteur non nul de $\RR^3$,
  et $\wedge$ désigne le produit vectoriel de $\RR^3$.
  Le groupe $\RR^3$ agit naturellement sur $\TT^3$ par translation en préservant la forme $\omega$:
  $$
    \left\{
    \begin{array}{l}
    \forall \zeta=(r,s,t)\in\RR^3, \forall X=[x,y,z]\in \TT^3, \zeta_{\TT^3}(X) = [x+r,y+s,z+t] \\
    \zeta^*\omega = \omega.
    \end{array}
    \right.
    $$
  L'algèbre de Lie de $\RR^3$ est naturellement identifiée à $\RR^3$,
  ainsi que son dual grâce au produit scalaire ordinaire.
  Tout chemin basé en $[0,0,0]$ dans $\TT^3$ est homotope à un chemin du type $\gamma_X: t\mapsto [tX]$,
  où $X=(x,y,z)\in\RR^3$.
  La fonction $\Psi$ est donnée par:
  $$
    \Psi(\gamma_X)=[Z\mapsto -\scal(\Omega\wedge X,Z)] \quad \mbox{\ie} \quad \Psi(\gamma_X) = -\Omega\wedge X.
    $$
  Ce qui nous donne immédiatement le groupe $\Gamma_\omega$ et le «{\em moment généralisé}» $\psi$:
  $$
    \Gamma_\omega = \{\Omega\wedge \ell \mid \ell\in \ZZ^3 \} \quad \mbox{et} \quad \psi[X] =[\Omega\wedge X]_{\Gamma_\omega}.
    $$
  Soit $\cM$ l'image de $\TT^3$ par le «moment» $\psi$.
  D'après ce qui précède,
  $\cM$ est le quotient de l'espace vectoriel $\Omega^{\perp} = j(\Omega)(\RR^3)$ par le sous-groupe $j(\Omega)(\ZZ^3)$:
  $$
    \cM=\psi(\TT^3) = j(\Omega)(\RR^3)/j(\Omega)(\ZZ^3).
    $$
  L'opérateur $j$ désignant le produit vectoriel,
  $j(\Omega)(X)=\Omega\wedge X$.
  Précisons davantage la structure de $\cM$.
  Soit $\Omega=(a,b,c)$;
  puisque $\Omega\neq 0$,
  une de ses trois coordonnées $a$,
  $b$ ou $c$ est non nulle;
  supposons que ce soit $c$.
  Nous pouvons alors identifier $\Omega^\perp$ avec le plan $z=0$ par projection $(x,y,z)\mapsto (x,y)$.
  Le sous-groupe $j(\Omega)(\ZZ^3)$ se projette alors sur le plan $z=0$ comme le sous-groupe:
  $$
    j(\Omega)(\ZZ^3) \sim\left\{\vect{bm-cl \\ ck-am}\in\RR^2 \mbox{ tel que } \vect{k\\ l\\m}\in\ZZ^3\right\}.
    $$
  Ainsi,
  l'image du moment est isomorphe au quotient de $\RR^2$ par la relation:
  $$
    \vect{x\\ y} \sim \vect{x\\ y} + \vect{bm-cl \\ ck-am} = \vect{x\\ y} + c \vect{-l \\ k} + m \vect{b\\ -a}.
    $$
  Le quotient s'obtient donc en deux étapes:
  
  1) on prend d'abord le quotient de $\RR^2$ par le réseau $c\ZZ^2$ et on obtient alors un tore ordinaire (de côté $c$) $\TT^2=\RR^2/c\ZZ^2$;
  
  2) on prend ensuite le quotient de ce tore par $\ZZ$,
  agissant selon $m : [x,y]\mapsto [x+mb,y-ma]$,
  $m\in \ZZ$.
  Ce quotient $\TT^2/\ZZ$ fibre sur le quotient $\TT^2/\RR$,
  où $\RR$ agit sur $\TT^2$ par $t : [x,y]\mapsto[x+tb,y-ta]$,
  $t\in \RR$,
  avec pour fibre $S^1=\RR/\ZZ$.
  Supposons alors que les nombres $a$ et $b$ soient rationnellement indépendants,
  et laissons l'autre cas comme exercice pour le lecteur.
  Soit $\alpha = b/a$,
  le quotient $\TT^2/\RR$ est équivalent à $T_\alpha =\RR/(\ZZ+\alpha \ZZ)$;
  en effet,
  il suffit de remonter au revêtement $\RR^2$ et de caractériser chaque orbite de $\RR\times \ZZ^2$ sur $\RR^2$ par sa trace sur l'axe des $x$ (grâce à la projection $(x,y)\mapsto x + \alpha y$).
  Ainsi,
  l'espace des moments de $\RR^3$,
  pour la 2-forme fermée $\omega$,
  est un fibré en cercle sur un tore irrationnel $\TT_\alpha$.
  La trivialité de ce fibré est liée à la commensurabilité des coefficients $a$,
  $b$,
  $c$.
  Nous laissons au lecteur le soin de préciser cela.
  On vérifie d'autre part que l'application moment réalise le quotient symplectique de $\TT^3$ par le noyau de $\omega$;
  autrement dit,
  l'espace des moments --- ce fibré en cercle $S^1$ sur un tore $\TT_\alpha$ --- est muni de la structure «symplectique» naturelle d'espace de caractéristiques d'une 2-forme présymplectique.
\end{exemple}

\begin{remarque}
  Cette construction nous encourage à étendre la notion de moment d'un groupe en définissant de façon générale le {\em Moment} d'un groupe de Lie $G$ qui agit sur une variété $X$ en préservant une 2-forme fermée $\omega$,
  comme l'application $\psi$ à valeur dans $\cG^*/\Gamma_\omega$,
  telle qu'elle est définie plus haut (\ref{momgene}).
  Cette définition n'est évidemment plus compatible avec la définition originale avec laquelle elle s'accorde seulement si $\Gamma_\omega$ est nul.
  Mais la persistance du théorème de N{\oe}ther rend cet élargissement de la définition du moment tout à fait raisonnable et même souhaitable.
  On pourrait alors conserver l'adjectif {\em hamiltonien} uniquement pour désigner les actions de groupes $G$ sur $(X,\omega)$ tels que $\Gamma_\omega$ soit nul.
  Cette construction a été utilisée dans le cas particulier de $G=S^1$,
  pour introduire une certaine notion de «moment à valeurs dans un groupe»,
  mais cette notion (à part le cas précis de ce paragraphe) est loin d'être claire et nous ne l'évoquerons pas plus.
\end{remarque}

%%%%%%%%%%%%%%%%%%%%%%%%%%%% paragraphe 2 %%%%%%%%%%%%%%%%%%%%%%%%%%
\section{Le cas particulier des formes exactes}

Un cas particulier important,
nous l'avons vu avec les problèmes variationnels,
est lorsque la 2-forme $\omega$ est exacte\index{forme exacte}:
$\omega = d\varpi$.
Pour tout élément $g\in G$,
la 1-forme $g^*\varpi -\varpi$ est fermée puisque $g^*d\varpi =0$;
on a donc une application
%
\begin{equation}
  \beta :G\to \ZdR^1(X) \quad \mbox{définie par} \quad \beta : g\mapsto g^*\varpi - \varpi,
\end{equation}
%
telle que
%
\begin{equation}
  \beta(gg') = g'^*\beta(g) + \beta (g').
\end{equation}
%
On dit que $\beta$ est un un-cocycle de $G$ dans $\ZdR^1(X)$.
De cette définition de $\beta$ on tire:
%
\begin{equation}
  \DLie_Z\varpi = \left.{\partial e^{tZ*}\varpi \over \partial t}\right\vert_{t=0} = d\beta_{\scriptstyle \id}\cdot Z
\end{equation}
%
où $d\beta_{\scriptstyle \id}$ désigne la dérivée de $\beta$ calculée en l'identité:
$D(\beta)(\id)$).
On déduit donc que l'action de $G$ est hamiltonienne si et seulement si $d\beta_{\scriptstyle \id}\cdot Z$ est exacte pour tout $Z\in \cG$,
mais on a:


\begin{proposition}
  Soit $X$ une variété différentiable munie d'une action d'un groupe de Lie connexe $G$.
  Soit $\varpi$ une 1-forme sur $X$ telle que sa dérivée extérieure soit invariante par $G$,
  soit $\beta: g \mapsto g^*\varpi -\varpi\in \ZdR^1(X)$.
  L'action de $G$ est hamiltonienne pour $d\varpi$ si et seulement si $\beta(g)$ est exacte pour tout $g\in G$.
\end{proposition}


\begin{demonstration}
  Il est évident que si $\beta(g)$ est exacte pour tout $g$,
  $d\beta_{\scriptstyle \id}\cdot Z$ sera exacte pour tout $Z\in \cG$.
  Réciproquement,
  supposons que $d\beta_{\scriptstyle \id}\cdot Z$ soit exacte pour tout $Z\in \cG$.
  Par connexité de $G$ la classe de cohomologie de $g'^*\beta(g)$ est égale à la classe de cohomologie de $\beta(g)$;
  l'application $g\mapsto [\beta(g)]$ étant donc un homomorphisme de $G$ dans $\ZdR^1(X)$,
  il se projette donc sur l'abélianisé $G/[G,G]$ de $G$.
  Soit $[\beta]$ cette application:
  %
  $$
    \begin{tikzcd}[column sep=normal, row sep=normal, every label/.append style = {font = \small}]
    G \arrow[r,"\beta"] \arrow[d] & \ZdR^1(X) \arrow[d] \\
    G/[G,G] \arrow[r,swap,"{[\beta]}"] & \HdR^1(X)
    \end{tikzcd}
    $$
  %
  Mais le groupe de Lie $G$ étant connexe et de dimension finie,
  $G/[G,G]$ est le produit direct d'un tore $T^k$ par un espace vectoriel $\RR^l$.
  L'homomorphisme $\beta$ est évidemment nul sur le facteur $T^k$,
  et reste un homomorphisme de $\RR^l$ dans $\HdR^1(X)$,
  \cad une application linéaire,
  puisque tout cela est différentiable.
  L'hypothèse revient à supposer $d[\beta]_0(T_0\RR^l) = \{0\}$,
  mais $[\beta]$ est linéaire donc $[\beta](\RR^l) = \{0\}$,
  or $[\beta](T^k)=\{0\}$,
  d'où $[\beta](G/[G,G])=\{0\}$;
  autrement dit,
  $\beta(g)$ est exacte pour tout $g\in G$.
\end{demonstration}

\begin{remarque}
  Par définition de la cohomologie des groupes l'application $[\beta]$ est un un-cocycle de $G$ à valeurs dans $\HdR^1(X)$:
  
  
  \begin{equation}
    [\beta]\in H^1(G,\HdR^1(X))=H^1(G,\RR)\otimes \HdR^1(X).
  \end{equation}
  
  
  Nous venons de dire que l'action est hamiltonienne si et seulement si ce cocycle est nul.
  On retrouve ainsi,
  d'une autre façon,
  l'obstruction précédente à l'existence du moment.
\end{remarque}

Nous pouvons appliquer maintenant tout ce qui a été dit au paragraphe~\ref{parLagPrInv}.
L'action de $G$ est hamiltonienne si et seulement si il existe une fonction $F:G\mapsto \Cinf(G\times X,\RR)$ telle que:
%
\begin{equation}
  g^*\varpi = \varpi + dF(g,\cdot),
\end{equation}
%
Le défaut d'invariance de $\varpi$ est encore défini par le deux-cocycle $c$ de $G$:
%
\begin{equation}
  c(g,g') = F(gg') - g'^*F(g) - F(g'),
\end{equation}
%
et le moment $\mu$ de l'action de $G$ est encore donné par:
%
\begin{equation}
  \mu\cdot Z = \varpi(Z_X) - f\cdot Z ,\quad \mbox{avec} \quad f: x \mapsto dF(\cdot,x)_{\scriptstyle\id}\in \cG^*.
\end{equation}
%
Les formules établies au paragraphe \ref{parLagPrInv} pour le défaut d'équivariance $\Theta$ du moment $\mu$ et sa relation avec $c$ sont encore valables dans ce cadre.

Nous allons analyser plus en détail la nature du moment dans les trois différents cas qui se présentent:


\begin{enumerate}
  \item La primitive $\varpi$ est invariante par $G$.
  \item La primitive $\varpi$ n'est pas invariante par $G$,
  mais le cocycle $c$ est trivial,
  \item La primitive $\varpi$ n'est pas invariante par $G$ et $c$ n'est pas trivial.
\end{enumerate}



\subsection*{Cas 1. La primitive $\mathbf{\varpi}$ est invariante par $\mathbf{G}$}

Supposons que la primitive $\varpi$ de $\omega$ soit invariante par $G$,
le moment $\mu$ est alors donné par:
%
\begin{equation}
  \mu : x\mapsto \mu(x)= [Z\mapsto \varpi_x(Z_X(x))].
\end{equation}
%
Nous pouvons aussi interpréter le moment $\mu$ directement sur le groupe $G$.
Considérons l'application $\hat x : G\to X$ telle que $\hat x(g)= g(x)$;
soit $\varpi^x=\hat x^*\varpi$,
c'est une 1-forme sur $G$ invariante par l'action à gauche.
En effet:
$L_g^*\varpi^x = L_g^*\hat x^*\varpi = (\hat x\circ L_g)^*\varpi = (g\circ \hat x)^*\varpi = \hat x^* g^*\varpi = \hat x^*\varpi = \varpi^x$.
C'est donc un élément du dual de l'algèbre de Lie de $G$ par définition même,
puisque l'algèbre de Lie de $G$ est l'espace des champs de vecteurs sur $G$ invariants à gauche et son dual l'espace des 1-formes invariantes à gauche\footnote{On peut d'ailleurs définir le {\em dual-de-l'algèbre-de-Lie},
en un seul mot,
directement comme l'espace des 1-formes de $G$,
invariantes par l'action à gauche.
Cela permet des généralisations au cas des groupes pour lesquels l'algèbre de Lie est mal définie.
On peut préférer le nom d'{\em espace des moments}.}.
La 1-forme $\varpi^x$ est donc définie par sa valeur en l'identité,
soit:
$\mu^x=\varpi^x_\id$,
tel que $\varpi^x=\mu^x\circ\theta$,
$\theta$ étant la forme de Maurer-Cartan.
Il est clair que $\mu^x=\mu(x)$.
Dans ce cas le moment est simplement l'application:
%
\begin{equation}
  \mu : x \mapsto \mu(x) = \varpi^x = \hat x^*\varpi.
\end{equation}

\begin{remarque}
  Il est important d'insister sur ce cas,
  car il est l'essence même de l'application moment.
  Contrairement à ce qu'on peut penser et écrire (même ici),
  le moment n'est pas un objet de la géométrie des 2-formes différentielles fermées invariantes,
  encore moins un objet de la géométrie symplectique:
  \begin{center}
    {\it Le Moment est un objet de la géométrie des 1-formes différentielles invariantes.}
  \end{center}
  \noindent C'est ce que nous nous efforcerons de montrer jusqu'à la fin de ces notes et nous définirons formellement:
  
  
  \begin{definition}
    [\sc Moment des 1-formes]\label{def0appmom}\index{moment des 1-formes}
    Soit $\alpha$ une 1-forme différentielle définie sur une variété $X$.
    Soit $G$ un groupe de Lie agissant différentiablement sur $X$.
    Si $\alpha$ est invariante par $G$,
    nous appellerons {\em moment} de $\alpha$ (sous l'action de $G$) la fonction $\mu : X\mapsto \cG^*$ définie par $\mu(x)=\hat x^*\alpha=[Z\mapsto \alpha_x(Z_X(x))]$.
  \end{definition}
  
  
  Il faut noter que dans ce cas l'application moment est bien définie,
  et pas seulement à une constante près.
  La remarque précédente en appelle une nouvelle:
  il n'est pas nécessaire de se restreindre à un groupe de Lie préservant la forme $\alpha$;
  on peut immédiatement considérer le moment de $\alpha$ défini pour le groupe $G_\alpha$ de tous les difféomorphismes de $X$ préservant $\alpha$.
\end{remarque}



\subsection*{Cas 2. La primitive $\mathbf{\varpi}$ n'est pas invariante par $\mathbf{G}$, mais le cocycle $\mathbf{c}$ est trivial}

Supposons que la primitive $\varpi$ ne soit pas invariante par $G$,
la forme $\varpi^x$ n'est alors,
pas davantage,
invariante par l'action à gauche de $G$.
Notons $F^x$ l'application réelle de G:
$F(\cdot,x)$;
on a immédiatement $L_g^*\varpi^x= \varpi^x+d[\hat x^*F(g,\cdot)]$,
et les identités sur $F$ et $c$ se traduisent par:
%
\begin{equation}
  x^*F(g,\cdot) = L_g^*F^x-F^x-c(g,\cdot).
\end{equation}
%
Si,
comme c'est ici l'hypothèse,
$c\equiv 0$ alors $L_g^*\varpi^x = \varpi^x +d[L_g^*F^x-F^x]$.
Autrement dit,
la 1-forme différentielle $\tilde \varpi^x = \varpi^x- dF^x$ est invariante par l'action à gauche de $G$:
elle appartient au dual de l'algèbre de Lie de $G$,
et sa valeur en l'identité nous donne exactement le moment $\mu$ (tel qu'il est défini à la page \pageref{appmomdef}):
pour tout $Z\in \cG$,
$\tilde \varpi^x_\id(Z) = \varpi(Z_X) -f\cdot Z$.
En d'autres termes
%
\begin{equation}
  \mu : x \mapsto \tilde \varpi^x= \varpi^x-dF^x= x^*\varpi- dF(\cdot,x).
\end{equation}
%



\subsection*{Cas 3. La primitive $\mathbf{\varpi}$ n'est pas invariante par $\mathbf{G}$ et $\mathbf{c}$ n'est pas trivial}

Plaçons nous maintenant dans le cas général.
La primitive $\varpi$ n'est pas invariante et le cocycle $c$ n'est pas trivial.
Utilisons donc ce cocycle $c$ qui nous est offert pour l'extension de $G$ par $\RR$ suivante:
%
\begin{equation}
  \tilde G = G\times \RR, \quad (g,t)\cdot(g',t') = (gg',t+t'+c(g,g')).
\end{equation}
%
La condition de cocycle sur $c$ signifie exactement que $\tilde G$ est un groupe.
C'est une extension centrale de $G$.
On voit tout de suite que le sous-groupe $\{\id\}\times \RR$ commute avec tous les éléments de $G$.
Utilisons maintenant la fonction $F$,
dont $c$ est issu,
pour définir une action de $\tilde G$ sur le produit direct $\tilde X= X\times \RR$:
%
\begin{equation}
  \vect{g\\ t}\cdot \vect{x\\ s} = \vect{g(x)\\ s+t-F(g,x)}.
\end{equation}
%
Là encore,
les propriétés conjuguées de $F$ et de $c$ nous permettent de vérifier que l'on a bien défini une action de $\tilde G$ sur $\tilde X$.
L'image réciproque $\tilde \omega$ de $\omega$ sur $\tilde X$ est évidemment exacte,
mais choisissons comme primitive:
%
\begin{equation}
  \tilde \varpi = \varpi + ds.
\end{equation}
%
Nous laissons en exercice le soin de vérifier que les signes ont été correctement choisis pour que $\tilde\varpi$ soit invariante par $\tilde G$:
%
\begin{equation}
  \forall \tilde g\in \tilde G:\quad \tilde g^*\tilde \varpi=\tilde \varpi.
\end{equation}
%
Le moment de l'action de $\tilde G$ sur $\tilde X$ est donc donné,
grâce à ce qui précède,
par:
%
\begin{equation}
  \tilde \mu : \tilde x \mapsto \tilde x^*\tilde\varpi.
\end{equation}
%
Mais l'action de $\tilde G$ induite sur $X$ est l'action de $G$ elle-même,
le moment de $G$ coïncide alors avec celui de $\tilde G$,
autrement dit;
%
\begin{equation}
  \mu(x) = \tilde \mu(\tilde x) \quad \mbox{pour tout}\quad \tilde x\mapsto x.
\end{equation}
%

\noindent On pourrait résumer cette analyse par la proposition suivante:

\begin{remarque}
  Dans le cas des 2-formes $\omega$ exactes $\omega=d\varpi$,
  la nature de l'application moment est toujours la même:
  c'est l'application qui à $x\in X$ associe la 1-forme invariante à gauche $\hat x^*\varpi$ sur le groupe $G$,
  ou si nécessaire sur une extension centrale bien choisie $\tilde G$ de $G$.
  Nous allons voir comment nous pouvons adapter cette proposition au cas général.
\end{remarque}

%%%%%%%%%%%%%%%%%%%%%%%%%%%% paragraphe 3 %%%%%%%%%%%%%%%%%%%%%%%%%%
\section{Classes de cohomologie associées au moment}

Il est bon de préciser de quels objets dépendent vraiment ces cocycles que nous avons introduits jusqu'à présent,
ou plutôt leur classe de cohomologie.
Comme nous l'avons déjà remarqué,
la classe de cohomologie de $\Theta$,
le défaut d'équivariance du moment,
ne dépend que de $\omega$ et de l'action de $G$,
mais elle est indépendante du choix de la primitive $\varpi$.
Il n'en est pas de même pour la classe de cohomologie du deux-cocycle $c$.
Elle dépend explicitement du choix de $\varpi$.
En effet,
considérons une autre primitive $\varpi'$ de $\omega$,
alors $\varpi'=\varpi + b$ où $b$ est une 1-forme fermée quelconque de $X$.
Puisque $b$ est fermée et que $G$ est connexe,
il existe une application $\phi: G\to \Cinf(X,\RR)$ telle que $g^*b= b + d\phi(g)$,
et donc:
%
\begin{equation}
  F'(g) = F(g) + \phi(g), \quad \mbox{avec} \quad g^*b= b + d\phi(g).
\end{equation}
%
Par définition,
la fonction $\phi$ définit un deux-cocycle réel de $G$,
%
\begin{equation}
  e(g,g')= \phi (gg')-g'^*(\phi(g)) -\phi(g').
\end{equation}
%
Puisque $X$ est connexe,
la classe de cohomologie $[e]\in H^2(G,\RR)$ ne dépend que de la classe de cohomologie $[b]\in \HdR^1(X)$ et non du choix de $\phi$.
Nous venons de définir un homomorphisme par:
%
\begin{equation}
  \chi: \HdR^1(X) \to H^2(G,\RR), \quad \mbox{avec} \quad \chi([b]) = [e].
\end{equation}
%
Il n'y a aucune raison a priori pour que cet homomorphisme soit nul.
La classe de cohomologie de $c$ dépend non seulement de $\omega$ mais du choix de la primitive $\varpi$.
Mais on peut remarquer que la classe de $[c]$,
modulo le sous-espace vectoriel $\chi(\HdR^1(X))\subset H^2(G,\RR)$,
est justement caractérisée par la classe $[\Theta]\in H^1(G,\cG^*)$.

Pour comprendre davantage la nature de ces objets,
introduisons un cadre un peu plus général et plus géométrique.
Considérons toujours une variété différentiable connexe $X$,
munie d'une action différentiable d'un groupe de Lie connexe $G$.
Considérons un deux-cocycle $c$ de $G$ à valeurs dans un groupe abélien $A$ et son extension centrale associée,
définie par la multiplication
$$
  (g,a)\cdot(g',a') = (gg',a+a'+c(g,g')), \quad g,g'\in G, \quad a,a'\in A.
  $$
Nous allons réaliser ce groupe comme un groupe de difféomorphismes,
comme un sous-groupe d'automorphismes du {\em fibré d'homologie}\index{fibré d'homologie} de $X$.
Commençons par construire ce fibré.

Soit $C_k(X,A)$ l'ensemble des $k$-chaînes simpliciales de $X$ à valeurs dans le groupe abélien $A$,
et $\partial _k$ l'opérateur {\em bord} de $C_k(X,A)$ dans $C_{k-1}(X,A)$.
Rappelons qu'une $k$-chaîne simpliciale est une combinaison linéaire de simplexes différentiables formelle à coefficients dans $A$,
et que les groupes d'homologie $H_k(X,A)$ sont définis par:
%
\begin{eqnarray}
  H_k(X,A) & = & Z_k(X,A)/B_k(X,A), \\
  \nonumber
  & \mbox{avec} & \\
  \nonumber
  Z_k(X,A)=\ker \partial _k & \mbox{et} & B_k(X,A) =\im \partial _{k+1}.
\end{eqnarray}
%
Les éléments de $Z_k(X,A)$ sont appelés {\em $k$-cycles} simpliciaux de $X$ à valeurs dans $A$,
et ceux $B_k(X,A)$ les {\em $k$-bords}.
On a en particulier pour les 0-chaînes:
%
\begin{equation}
  c = \sum_{x} n_x x,\quad n_x\in A, \quad x\in X, \quad \partial _0c=0,
\end{equation}
%
où les $n_x$ sont tous nuls à l'exception d'un nombre fini;
et pour les 1-chaînes:
%
\begin{equation}
  c = \sum_{\gamma} n_\gamma \gamma \quad n_\gamma\in A, \quad \gamma \in \Arc(X), \quad \partial _1c = \sum_\gamma n_\gamma (\gamma(1)-\gamma(0)).
\end{equation}
%
Puisque $B_1(X,A)\subset Z_1(X,A)$,
le sous-groupe quotient $C_1(X,A)/B_1(X,A)$ fibre sur $C_1(X,A)/Z_1(X,A)$,
avec pour fibre $Z_1(X,A)/B_1(X,A)$,
\cad $H_1(X,A)$;
mais $C_1(X,A)/Z_1(X,A)=C_1(X,A)/\ker \partial _1$,
\cad
$$
  C_1(X,A)/Z_1(X,A) = \partial _1C_1(X,A).
  $$
Par connexité de $X$,
toute 0-chaîne de la forme $\sum_{x,y} n_{x,y} (x-y)$ est un élément de $\im \partial _1$,
puisqu'on peut toujours trouver un arc $\gamma$ tel que $x=\gamma(1)$ et $y=\gamma(0)$.
Ainsi,
on montre que $C_1(X,A)/Z_1(X,A)$ est le groupe abélien libre engendré par le produit direct $X\times X$,
en associant à $\sum_{x,y} n_{x,y} (x-y)$,
l'élément $\sum_{x,y} n_{(x,y)} (x,y)$.
On a donc construit,
au-dessus de $C_0(X\times X,A)$,
un fibré principal de groupe structural $A$.
Pointons $X$ par une origine $o$ et considérons l'injection de $X$ dans $C_0(X\times X,A)$ définie par $\hat o : x\mapsto (o,x)$.
Soit $\hat X$ l'image réciproque,
par $\hat o$,
de la fibration principale $C_1(X,A)/B_1(X,A)\to C_1(X,A)/Z_1(X,A)$:
%
\begin{equation}
  \begin{tikzcd}[column sep=normal, row sep=normal, every label/.append style = {font = \small}]
    \hat X \arrow[r] \arrow[d,swap,"{H_1(X,A)}"] & C_1(X,A)/B_1(X,A) \arrow[d,"{H_1(X,A)}"] \\
    X \arrow[r]                                   & C_1(X,A)/Z_1(X,A)
  \end{tikzcd}
\end{equation}
%
Nous ne démontrerons pas la proposition suivante:

\begin{proposition}
  L'image réciproque,
  par l'injection $\hat o : x \mapsto (o,x)$,
  du fibré principal $C_1(X,A)/B_1(X,A)\to C_1(X,A)/Z_1(X,A)$ est une variété différentiable connexe $\hat X_A$ fibrée principalement sur $X$,
  de groupe structural $H_1(X,A)$.
  On l'appelle le revêtement d'homologie\index{revêtement d'homologie} de $X$.
\end{proposition}

La fonction de ce revêtement est de rendre exacte toute 1-forme fermée.
Choisissons $A=\ZZ$,
et notons simplement $\hat X= \hat X_{{\scriptstyle\bf Z}}$ et $\pi$ sa projection sur $X$:
%
\begin{equation}
  \pi : \hat X \to X,\quad \pi : x-o + c' \mapsto x, \quad \mbox{avec} \quad \partial c'=0.
\end{equation}
%
Soit $\alpha$ une 1-forme fermée de $X$,
considérons l'homomorphisme de $h_\alpha:C_1(X,\ZZ)\to \RR$ obtenu par intégration de $\alpha$ le long des chaînes:
%
\begin{equation}
  h_\alpha : C_1(X,\ZZ) \to \RR,\quad h_\alpha \left(\sum_\gamma n_\gamma \gamma\right) = \sum_\gamma n_\gamma \int_\gamma \alpha.
\end{equation}
%
Pour toute variation $\delta c$ de la chaîne $c$,
on a grâce,
à la formule de Stokes,
et parce que $\alpha$ est fermée:
%
\begin{equation}
  \delta h_\alpha (c) = \sum_\gamma n_\gamma [\alpha(\delta \gamma(1) - \alpha(\delta \gamma(0)].
\end{equation}
%
Si la chaîne $c$ représente $\hat x\in \hat X$,
alors $c\sim x- o + c'$ avec $\partial _1c'=0$;
on déduit de ce qui précède que:
%
\begin{equation}
  \delta h_\alpha (\hat x) = \alpha (\delta x) \quad \Leftrightarrow \quad dh_\alpha = \pi^*\alpha.
\end{equation}
%
On a intégré de cette façon $\alpha$ sur le revêtement d'homologie $\hat X$.
Le groupe $H_1(X,\ZZ)$ agit additivement sur $H_1(X,\RR)$ par inclusion de $\ZZ$ dans $\RR$,
les chaînes à coefficients dans $\ZZ$ étant en particulier à coefficients dans $\RR$.
Cette action peut ne pas être libre,
et son noyau est appelé le groupe de torsion de $H_1(X,\ZZ)$.
Mais on peut vérifier que:
%
\begin{equation}
  H_1(X,\RR)/H_1(X,\ZZ) = H_1(X,S^1).
\end{equation}
%
Considérons alors le produit fibré:
%
\begin{equation}
  \hat X\times_{H_1(X,{\scriptstyle \bf Z})}H_1(X,\RR) \to X.
\end{equation}
%
Puisque l'action de $H_1(X,\ZZ)$ est libre sur $\hat X$ et que ces groupes sont commutatifs,
cette fibration est principale de groupe $H_1(X,\RR)$.
Mais le groupe $H_1(X,\RR)$ étant en réalité un espace vectoriel sur $\RR$,
cette fibration principale est donc triviale.
Il existe une section de cette projection,
ou,
ce qui revient au même,
une application $T:\hat X\to H_1(X,\RR)$ équivariante sous l'action de $H_1(X,\ZZ)$:
%
\begin{equation}
  \hat T(\hat x + k) = \hat T(x) + k,\quad k\in H_1(X,\ZZ).
\end{equation}
%
On en déduit ainsi l'existence d'une application
$$
  T : X\to H_1(X,S^1)=H_1(X,\RR)/H_1(X,\ZZ)
  $$
telle que le diagramme suivant commute:
%
\begin{equation}
  \begin{tikzcd}
    \hat X \arrow[r, "\hat T"] \arrow[d, "\pi"'] & H_1(X,\RR) \arrow[d, "\pi_1"] \\
    X \arrow[r, "T"'] & H_1(X,S^1)
  \end{tikzcd}
\end{equation}
%
Supposons maintenant que $H_1(X,\RR)$ soit de dimension finie $m$;
alors $H_1(X,S^1)$ est un tore $\TT^m$,
un groupe dont l'algèbre de Lie est justement $H_1(X,\RR)$.
Soit $\theta$ sa forme de Maurer-Cartan;
elle est fermée mais non exacte puisque c'est la projection de l'identité de $H_1(X,\RR)$.
Son image réciproque sur $X$ est une 1-forme fermée que nous noterons $\Lambda$:%
\begin{equation}
  \Lambda = T^*\theta, \quad \theta = \pi_{1*}\id,\quad \theta \in \ZdR^1(X,H_1(X,\RR)).
\end{equation}

\begin{remarque}
  Cette 1-forme n'est pas définie de façon unique à cause du choix de $\hat T$,
  mais elle a un caractère universel dans la mesure où elle représente,
  en un seul objet,
  les classes de cohomologie de toutes les 1-formes fermées de $X$.
  Considérons une 1-forme fermée $\alpha$ et soit $h_\alpha$ l'homomorphisme qu'elle induit sur $H_1(X,\RR)$ par intégration,
  alors la 1-forme différentielle fermée:
  %
  \begin{equation}
    \Lambda _\alpha = h_\alpha \circ \Lambda : \hat x \mapsto \Lambda_\alpha(\hat x) = \int_\alpha \hat T(\hat x)
  \end{equation}
  %
  lui est cohomologue. {}
  En effet,
  l'application $h_\alpha\circ\hat T-h_\alpha$ est invariante sous l'action de $H_1(X,\ZZ)$:
  soit $\hat x\in \hat X$ et $k\in H_1(X,\ZZ)$,
  $h_\alpha[\hat T(\hat x +k)] -h_\alpha (\hat x + k) = h_\alpha[\hat T(\hat x) + k] - h_\alpha(\hat x) -h_\alpha(k) = h_\alpha[ \hat T(\hat x)] +h_\alpha(k) - h_\alpha(\hat x) -h_\alpha(k) = h_\alpha[\hat T(\hat x)] - h_\alpha(\hat x)$.
  L'application $h_\alpha\circ\hat T-h_\alpha$ est donc définie sur $X$.
  Soit $f_\alpha (x) =h_\alpha[\hat T(\hat x)]-h_\alpha(\hat x)$,
  pour tout $\hat x$ au dessus de $x$,
  \cad $\int_{\hat x}\Lambda_\alpha = \int_{\hat x} \alpha + f_\alpha(x)$;
  dérivée,
  cette identité donne:
  $\Lambda_\alpha = \alpha +df_\alpha$.
  D'autre part,
  soit $\chi: H_1(X,\RR)\to \RR$ un homomorphisme et $\alpha = \chi\circ T$,
  c'est une 1-forme fermée vérifiant $h_\alpha =\chi$.
  Cette 1-forme universelle $\Lambda$ nous a permis de démontrer la proposition suivante:
  
  \begin{theoreme}[{\sc De Rham}]\index{théorème de de Rham}
    L'homomorphisme défini sur $\HdR^1(X,\RR)$ par intégration des formes,
    à valeurs dans $H^1(X,\RR)=\Hom(H_1(X,\RR),\RR)$,
    est un isomorphisme.
  \end{theoreme}
  
  C'est un cas particulier du véritable théorème de De Rham,
  concernant les formes fermées de tout degré.
\end{remarque}

Revenons maintenant au revêtement d'homologie $\hat X$,
et au groupe $G$ agissant sur $X$.
Considérons le sous-groupe des difféomorphismes de $\hat X$ commutant avec l'action de $H_1(X,\ZZ)$;
c'est le groupe des automorphismes de $\hat X$,
que l'on note:
%
\begin{equation}
  \Aut(\hat X) = \{ \phi \in \Diff(\hat X) \mid \phi(\hat x+k)= \phi(\hat x)+k, k\in H_1(X,\ZZ) \}.
\end{equation}
%
La projection de $\Aut(\hat X)$ sur $\Diff(X)$ a pour noyau $H_1(X,\ZZ)$;
en effet,
la projection de $\hat X$ sur $X$ étant une fibration principale,
l'injection $(k,\hat x)\mapsto (\hat x, \hat x +k)$ est une immersion,
donc si $\phi$ est dans le noyau de la projection $\Aut(\hat X)\to \Diff(X)$,
l'application $\hat x\mapsto \kappa(\hat x)$ telle que $\phi(\hat x) =\hat x + \kappa(\hat x)$ est différentiable;
mais $\kappa$ est à valeurs dans $H_1(X,\ZZ)$ qui est discret;
puisque $\hat X$ est connexe,
cette application est constante.
Puisque par définition de $\Aut(\hat X)$,
$H_1(X,\ZZ)$ commute avec tous ses éléments,
on en déduit une extension centrale:
%
\begin{equation}
  \id\to H_1(X,\ZZ)\to \Aut(\hat X) \to \Diff(X).
\end{equation}
%
On en déduit alors,
parce que $G$ est connexe et parce que $\hat X$ est un revêtement de $X$,
une extension centrale de $G$ par $H_1(X,\ZZ)$:
%
\begin{equation}
  \id\to H_1(X,\ZZ)\to \hat G \to G\to \id.
\end{equation}
%
Nous pouvons lui associer par produit fibré une {\em extension centrale}\index{extension centrale} de $G$ par $H^1(X,\RR)$:
$\hat G_\RR= \hat G \times_{H_1(X,\ZZ)}H_1(X,\RR)$.
À cette nouvelle extension centrale peut être associé un 2-cocycle de $G$ à valeurs dans l'espace vectoriel $H_1(X,\RR)$.
Nous pouvons d'ailleurs le deviner.
Soit $g\in G$,
nous savons,
puisque $G$ est connexe et que $\Lambda$ est fermée,
que $g^*\Lambda -\Lambda$ est exacte.
Nous pouvons définir ici,
comme dans le cas des formes différentielles réelles avec lequel il n'y a aucune différence essentielle:
%
\begin{equation}
  \Phi \in \Cinf(G\times X,H_1(X,\RR)), \quad g^*\Lambda = \Lambda + d\Phi(g,\cdot).
\end{equation}
%
De façon identique au cas réel,
cette fonction $\Phi$ définit un 2-cocycle\index{cocycle} $C$:
%
\begin{equation}
  C\in H^2(G,H_1(X,\RR)), \quad \Phi(gg') = g'^*\Phi(g) + \Phi(g') + C(g,g').
\end{equation}
%
\begin{exercice}
  Vérifier que ce 2-cocycle $C$ définit l'extension centrale $\hat G$ de $G$.
\end{exercice}

Soit $\alpha$ une 1-forme différentielle réelle fermée,
$F_\alpha$ la fonction exprimant le défaut d'invariance de $\alpha$ et $c_\alpha$ le deux-cocycle réel de $G$ associé.
Les relations entre,
d'une part,
$\Phi$ et $F_\alpha$ et,
d'autre part,
$C$ et $c_\alpha$ sont évidemment obtenues par intégration:
%
\begin{equation}
  F_\alpha(g,x)= \int_{\Phi(g,x)}\alpha, \quad c_\alpha(g,g') = \int_{C(g,g')}\alpha.
\end{equation}
%
Il suffit de se souvenir que $\Lambda_\alpha = h_\alpha \circ \Lambda$ est cohomologue à $\alpha$.
L'extension $\hat G_\RR^\alpha$ de $G$,
associée à $c_\alpha$,
est le quotient de $\hat G_\RR$ par le noyau de $h_\alpha$:
\begin{equation}
  \begin{tikzcd}
    \hat G_\RR \arrow[rr] \arrow[dr, swap, "{H_1(X,\RR)}"] & {} & \hat G_\RR^\alpha = \hat G_\RR/\ker h_\alpha \arrow[dl, "{H_1(X,\RR)}/\ker h_\alpha"] \\
    {} &  G  & {}
  \end{tikzcd}
\end{equation}
\noindent Voilà donc l'interprétation géométrique de cette classe de cohomologie $\sigma$ associée au choix d'une primitive particulière de la 2-forme fermée $\omega$.

%%%%%%%%%%%%%%%%%%%%%%%%%%%% paragraphe 4 %%%%%%%%%%%%%%%%%%%%%%%%%%
\section{Le cas général des 2-formes fermées quelconques}

Nous avons vu que l'application moment est tout à fait naturelle lorsque l'on considère l'action d'un groupe de Lie $G$ sur une variété $X$ préservant une 1-forme $\varpi$.
C'est l'application définie sur $X$ à valeurs dans le dual $\cG^*$ de l'algèbre de Lie $\cG$ de $G$ qui au point $x\in X$ associe la 1-forme invariante à gauche $\hat x^*\varpi$,
où $\hat x$ est l'{\it application orbite}\index{application orbite} $\hat x : g\mapsto g(x)$.
Nous avons pu encore remarquer que,
si la 1-forme $\varpi$ n'est pas invariante par $G$ mais que sa dérivée extérieure $\omega=d\varpi$ l'est,
on pouvait définir aussi le moment de façon analogue,
en considérant non plus la 1-forme $\varpi$ mais la forme de connexion $\tilde\varpi= \varpi + ds$,
définie sur le fibré principal trivial $Y=X\times \RR$.
Le groupe agissant sur $Y$ étant l'extension centrale $\tilde G$ de $G$ par $\RR$ définie grâce au cocycle $c$ et associée au défaut d'invariance $F$,
défini par $g^*\varpi = \varpi + dF(g)$.
Comme nous avons pu le constater,
cette extension centrale est le groupe des automorphismes du fibré principal $(Y,\tilde\varpi)$.
Le moment est alors obtenu comme l'application qui à un point $y=(x,t)\in Y$ associe la 1-forme invariante à gauche $y^*\tilde\varpi$,
élément du dual $\tilde\cG^*$ de l'algèbre de Lie $\tilde\cG$ de $\tilde G$.
Le fait que l'extension soit centrale indique que,
{\em a priori},
cette fonction de $y=(x,t)$ ne dépend que de $x$.
D'autre part le moment de $G$,
et non celui de $\tilde G$,
se retrouve en projetant $\tilde\cG^*$ sur $\cG^*$ le long de la section $t=0$.

Nous allons voir comment cette construction peut nous servir de guide et comment on peut l'adapter au cas général des 2-formes fermées non exactes.
Pour cela nous allons montrer qu'il existe au moins un fibré principal $Y$ au dessus de $X$,
structuré par le {\em tore des périodes} $T_\omega$ de la forme $\omega$,
et muni d'une connexion de courbure $\omega$.
De tels fibrés seront appelés,
par la suite,
{\em fibrés des périodes} de la forme $\omega$.



\subsection{Intégration d'une 2-forme fermée}

Considérons donc de façon générale une $2$-forme fermée $\omega$ sur une variété différentiable connexe $X$.
Appelons {\em groupe des périodes}\index{groupe des périodes} de la forme $\omega$ le sous-groupe additif de $\RR$ défini par:
%
\begin{equation}
  P_\omega = \left\{\int_\sigma \omega \mid \sigma \in H_2(X,\ZZ) \right\}.
\end{equation}
\begin{definition}
  Nous appellerons {\em tore des périodes}\index{tore des périodes} de la forme $\omega$ le quotient $T_\omega$ de $\RR$ par son groupe des périodes $P_\omega$.
  On notera $\cl_\omega$ la projection de $\RR$ sur le tore $T_\omega$:
  %
  \begin{equation}
    0\longrightarrow P_\omega \longrightarrow \RR\mor{\cl_\omega} T_\omega \longrightarrow 1.
  \end{equation}
\end{definition}
Le tore des périodes $T_\omega$ est évidemment un groupe abélien,
puisque c'est le quotient du groupe additif $\RR$ par le sous-groupe $P_\omega$.
Nous noterons additivement sa loi de groupe,
même si lorsqu'il est égal à $S^1$,
l'usage veut qu'elle soit notée multiplicativement.
Mais le tore des périodes est rarement un groupe de Lie:
$T_\omega$ est une variété uniquement dans l'un des deux cas suivants:
\begin{itemize}
  \item[1)] le groupe des périodes est trivial:
  $P_\omega=\{0\}$,
  alors $T_\omega=\RR$ et la forme $\omega$ est exacte
  \item[2)] le groupe des périodes est isomorphe à $\ZZ$,
  il existe un réel $a\neq 0$ tel que $P_\omega = a\ZZ$,
  alors $T_\omega\simeq S^1$.
\end{itemize}
Dans ces deux cas nous dirons que la forme $\omega$ est {\em entière}.
Dans les autres cas,
\cad lorsque le groupe $P_\omega$ est engendré par au moins deux éléments,
le tore des périodes sera muni de sa structure de {\em groupe différentiable}\index{groupe différentiable}\index{groupe difféologique}\footnote{Voir annexe \ref{AnnED} pour la définition précise de la structure d'{\em espace} et de {\em groupe différentiable}.}.
En particulier,
une application différentiable d'une variété $V$ dans le tore $T_\omega$ est la projection,
au moins localement,
d'une application différentiable de $V$ dans $\RR$.
Autrement dit:
$\varphi : V\to T_\omega$ est dite différentiable si et seulement si,
pour tout $x\in V$,
il existe un voisinage $U\subset V$ de $x$ et $\psi : U \to \RR$,
différentiable,
telle que $\varphi\mid U = \cl_\omega \circ \psi$.

Toute forme différentielle invariante sur $T_\omega$ est proportionnelle à $\theta$,
sa $1$-forme de Maurer-Cartan,
définie par:
%
\begin{equation}
  \cl_\omega^*\theta = dt.
\end{equation}
%
Comme le tore des périodes $T_\omega$,
les objets suivants doivent être compris au sens de la théorie des {\em espaces différentiables}\index{espace différentiable}.

Nous noterons $\Arc(X)$ l'espace des arcs\index{espace des arcs} d'une variété (ou d'un espace différentiable) $X$,
\cad l'espace des applications différentiables de $\RR$ dans $X$.
Nous munirons l'espace des arcs de $X$ de sa structure différentiable fonctionnelle.
Ses {\em plaques} (ou {\em paramétrages différentiables})\index{paramétrage différentiable} sont les applications $\varphi:U\to\Arc(X)$,
où $U$ est un ouvert de $\RR^p$,
telles que $(r,t)\mapsto\varphi(r)(t)$ définie sur $U\times \RR$ dans $X$ soit différentiable.
Nous définirons les applications {\em source} et {\em but}:\index{application source}\index{application but}
%
\begin{eqnarray}
  \nonumber
  \source : \Arc(X) \to X & \mbox{avec} & \source(\gamma) = \gamma(0) \\
  \but : \Arc(X) \to X    & \mbox{avec} & \but(\gamma) = \gamma(1).
\end{eqnarray}
%
Ces applications sont évidemment différentiables.
Pour toute plaque $\varphi : U\to \Arc(X)$ la composée $\but\circ\varphi:U\to X$ est:
$\but\circ\varphi(r)=\varphi(r)(1)$ qui est la restriction à $U\times\{1\}$ d'une application différentiable définie sur $U\times \RR$ (de même pour l'application source).

Nous noterons $\Arc(X,\xo)$ l'espace des arcs pointés en $\xo$,
\cad le sous-espace des arcs de $X$ tels que $\gamma(0)=\xo$:
%
\begin{equation}
  \Arc(X,\xo) = \{\gamma\in \Arc(X) \mid \source(\gamma)=\xo\}.
\end{equation}
%
Sa structure différentiable est évidemment sa structure de partie de $\Arc(X)$.
Mais il faut noter que cet espace est contractile,
ce qui est une propriété utilisée de façon importante dans la suite.
En effet,
l'application:
%
\begin{equation}
  \rho : \RR\times\Arc(X,\xo) \to \Arc(X,\xo) \quad \rho(s)(\gamma) = [t\mapsto \gamma(st)],
\end{equation}
%
est une rétraction de déformation de $\Arc(X,\xo)$ sur l'arc constant $t\mapsto \xo$.

Nous noterons aussi $\Lac(X,\xo)$ l'espace des lacets de $X$ basés en $\xo$.
De même que $\Arc(X,\xo)$,
il est muni de sa structure différentiable de partie de $\Arc(X)$.

Nous dirons que deux arcs $\gamma$ et $\gamma'$ sont {\em homologues}\index{arcs homologues} si leur différence borde une $2$-chaîne singulière;
nous noterons:
%
\begin{equation}
  \gamma \sim \gamma' \quad \Leftrightarrow \quad \exists \sigma \quad \partial \sigma = \gamma' - \gamma.
\end{equation}
%
Le quotient des arcs pointés par la relation d'homologie sera noté $\hat X$:
c'est le {\em revêtement d'homologie} de $X$.
Son groupe structural,
quotient de $\Lac(X,\xo)$ par la relation d'homologie,
est isomorphe au groupe $H_1(X,\ZZ)$.
En particulier si $H_1(X,\ZZ)=\{0\}$ alors deux arcs pointés en $\xo$ ont même but si et seulement si ils bordent une 2-chaîne.

Nous avons introduit les objets nécessaires à la démonstration de la proposition suivante.

\begin{proposition}
  \label{prop1}
  Soit $\omega$ une $2$-forme fermée sur une variété différentiable connexe $X$,
  dont le premier groupe d'homologie entière est nul:
  $H_1(X,\ZZ) =0$.
  Il existe alors un fibré principal $\pi : Y\to X$,
  de base $X$,
  de groupe structural le tore des périodes $T_\omega$,
  muni d'une forme de connexion $\lambda$ de courbure $\omega$.
  La structure $(Y,\lambda)$ est unique à équivalence près.
\end{proposition}
\begin{demonstration}
  Démontrons d'abord l'existence.
  Relevons la relation d'équivalence,
  définie plus haut sur les arcs,
  au produit $\Arc(X,\xo)\times T_\omega$:
  %
  \begin{equation}
    \label{defrelequiv}
    (\gamma,z) \sim (\gamma',z') \quad \Leftrightarrow \quad \gamma \sim \gamma' \mbox{ et } z' = z + \cl_\omega \int_\sigma \omega \quad \mbox{où} \quad \partial \sigma =\gamma'-\gamma.
  \end{equation}
  %
  Il est facile de vérifier que c'est bien une relation d'équivalence.
  Soit $Y$ le quotient de $\Arc(X,\xo)\times T_\omega$ par cette relation d'équivalence.
  Puisque $H_1(X,\ZZ)=0$ alors $\hat X=X$ et $Y$ est fibré principalement sur $X$,
  de groupe structural $T_\omega$.
  Soit $K$ l'opérateur de chaîne-homotopie tel qu'il est défini dans l'annexe \ref{AnnOCH},
  et $\alpha$ la $1$-forme définie sur le produit $\Arc(X,\xo)\times T_\omega$ par:
  %
  \begin{equation}
    \label{defalpha}
    \alpha = K\omega \oplus \theta.
  \end{equation}
  %
  C'est évidemment une forme de connexion pour l'action naturelle de $T_\omega$,
  de courbure $\but^*\omega$,
  puisque $K\omega$ est une primitive de $\but^*\omega$,
  où $\but$ désigne l'application {\em but} de $\Arc(X,\xo)$ sur $X$ qui à $\gamma$ associe $\gamma(1)$.
  En appliquant la proposition sur les quotients de formes différentiables (voir page \pageref{propformquot}),
  le passage de la forme $\alpha$ au quotient $Y$ est assuré par le lemme suivant:
  \begin{lemme}
    Soit $p$ la projection de $\Arc(X,\xo)\times T_\omega$ sur son quotient $Y$.
    Si deux paramétrages différentiables (plaques) $P$ et $P'$ de $\Arc(X,\xo)\times T_\omega$ vérifient $p\circ P=p\circ P'$,
    alors $P^*\alpha = P'^*\alpha$.
  \end{lemme}
  \begin{demonstration}
    Soit $U$ le domaine des plaques $P$ et $P'$ et $r\in U$,
    notons $P(r)=(P_X(r),P_T(r))$ où $P_X$ est une plaque de $\Arc(X,\xo)$ et $P_T$ une plaque de $T_\omega$.
    On ne perd rien en généralité en supposant que $P_T$ se relève globalement sur $\RR$ en une plaque $Q\in \Cinf(U,\RR)$ (\ie $P_T=\cl_\omega\circ Q$).
    On a donc $P^*(K\omega \oplus \theta) = P_X^*(K\omega) + P_T^*\theta$,
    \cad $P_X^*(K\omega) + dQ$.
    Notons $\gamma =P(r)$ et $\delta\gamma = D(P)(r)(\delta r)$ où $\delta r$ est un vecteur tangent à $r\in U$;
    par définition (\cf annexe \ref{AnnOCH}):
    $$
      P_X^*(K\omega)_r(\delta r)= \int_0^1\omega_{\gamma(t)}(\dot\gamma(t),\delta \gamma(t))\ dt.
      $$
    D'autre part,
    $p\circ P=p\circ P'$ implique l'existence,
    pour tout $r\in U$,
    d'une $2$-chaîne $\sigma_r$ telle que $\partial \sigma_r= P'(r)-P(r)$.
    On peut vérifier,
    en restreignant si nécessaire le domaine $U$,
    que $\sigma_r$ peut être choisie différentiablement.
    On a alors
    $$
      \cl_\omega(Q'(r)) = \cl_\omega(Q(r)) + \cl_\omega \int_{\sigma_r}\omega,
      $$
    \cad $Q'(r)=Q(r) + \int_{\sigma_r}\omega +l(r)$,
    où $l:U\to P_\omega$ est différentiable.
    Comme $P\omega$ est {\em différentiablement discret} toute application différentiable de $U$ dans $\RR$ à valeur dans $P_\omega$ est localement constante,
    et on en déduit que $l$ est constante et donc que:
    $$
      dQ'(r)=dQ(r)+d\left[\int_{\sigma_r} \omega\right].
      $$
    En utilisant le théorème de Stokes (voir annexe \ref{AnnStokes}),
    qui peut s'écrire aussi
    $$
      d\left[\int_{\sigma} \omega\right] = \int_{\sigma} d\omega (\cdot) + \int_{\partial \sigma} \omega(\cdot),
      $$
    et l'expression précédente de $P_X^*(K\omega)$,
    on déduit finalement:
    $$
      dQ'-dQ= -P_X^*(K\omega) + {P'_X}^*(K\omega) \quad \Rightarrow \quad P^*[K\omega \oplus \theta] = {P'}^*[K\omega \oplus \theta].
      $$
    C'est ce qu'il fallait démontrer.
  \end{demonstration}
  Démontrons l'unicité de cette construction.
  Soit $(Y',\lambda')$ une autre structure répondant aux hypothèses.
  L'image réciproque de $Y$ par $\but : \Arc(X,\xo) \to X$ est triviale,
  car $\Arc(X,\xo)$ est contractile et muni d'une connexion (cette propriété remplace la paracompacité dans le cas des variétés).
  L'image de la section nulle au-dessus de $\Arc(X,\xo)$ définit une application $\psi$ de $\Arc(X,\xo)$ dans $Y$.
  On vérifie alors que la projection $(\gamma, z)\mapsto z\psi(\gamma)$ réalise le quotient de $\Arc(X,\xo)\times T_\omega$ par la relation d'équivalence définie plus haut et donc que $Y$ et $Y'$ sont équivalents en tant que fibrés principaux.
  Il est toujours possible,
  d'autre part,
  de choisir un isomorphisme entre $\but^*(Y')$ et le produit $\Arc(X,\xo)\times T_\omega$ de telle sorte que l'image réciproque de $\lambda'$ coïncide avec la forme $\alpha$.
\end{demonstration}

\begin{remarque}
  Le relevé à $\Arc(X,\xo)$ de la relation d'équivalence sur les arcs définit un {\em cocycle} qu'on pourrait appeler le {\em cocycle d'arcs} associé à $\omega$.
  C'est la fonction $f_\omega$ définie sur les couples d'arcs homologues par:
  %
  \begin{equation}
    \label{cocarc}
    f_\omega(\gamma,\gamma') = \cl_\omega \int_\sigma \omega \quad \mbox{où} \quad \partial \sigma =\gamma'-\gamma.
  \end{equation}
  %
  L'intégrale dépend de la chaîne bordant $\gamma'-\gamma$,
  mais pas sa classe dans $T_\omega$.
  Le choix de cette terminologie résulte de l'identité suivante,
  vérifiée pour tout triplet d'arcs homologues:
  %
  \begin{equation}
    f_\omega(\gamma,\gamma') + f_\omega(\gamma',\gamma'') + f_\omega(\gamma'',\gamma) =0.
  \end{equation}
  %
  La fonction $f_\omega$ est un {\em cobord} s'il existe $\mu: \Arc(X,\xo) \to T_\omega$ différentiable,
  telle que $f_\omega(\gamma,\gamma') = \mu(\gamma')-\mu(\gamma)$,
  ce qui se produit lorsque $\omega =d\alpha$,
  on a alors $\mu (\gamma) = \int_\gamma \alpha$.
\end{remarque}

Nous utiliserons le lemme précédent pour démontrer le théorème plus général suivant:

\begin{theoreme}
  Pour toute $2$-forme fermée $\omega$ définie sur une variété différentiable connexe $X$,
  il existe un fibré principal $\pi : Y \to X$ de groupe structural $T_\omega$ muni d'une forme de connexion $\lambda $ de courbure $\omega$.
  Un tel fibré sera appelé {\em fibré des périodes} de la $2$-forme $\omega$.
  Les fibrés des périodes de la $2$-forme $\omega$ sont classés,
  à équivalence de fibré principal près,
  par le premier groupe d'extension $\Ext(H_1(X,\ZZ),P_\omega)$.
\end{theoreme}
\begin{demonstration}
  Comme dans la démonstration précédente,
  nous considérons la relation d'équivalence définie par (\ref{defrelequiv}) sur le produit $\Arc(X,\xo)\times T_\omega$,
  ainsi que la $1$-forme $\alpha$ (définition \ref{defalpha}).
  L'espace quotient $\hat Y$ est alors fibré sur le revêtement d'homologie $\hat X$ de $X$,
  de groupe $T_\omega$.
  Il est muni d'une connexion $\hat \lambda$ de courbure $\hat \omega$,
  image réciproque de $\omega$ par la projection naturelle $\hat X \to X$.
  En vertu de la proposition précédente,
  le fibré des périodes $(\hat Y \hat \lambda)$ est unique à équivalence près.
  Considérons une section $s$ de $H_1(X,\ZZ)$ dans l'espace des lacets $\Lac(X,\xo)$.
  Soit $\phi$ le $2$-cocycle de groupe défini sur $H_1(X,\ZZ)$,
  à valeurs dans $T_\omega$:
  %
  \begin{equation}
    \phi (h,h') = f_\omega (s(h+ h'), s(h)+ s(h')).
  \end{equation}
  %
  Ce cocycle étant symétrique,
  il définit une extension abélienne $\Gamma$ de $H_1(X,\ZZ)$ par $T_\omega$.
  Cette extension agit naturellement sur $\hat Y$ par:
  %
  \begin{equation}
    (h,\tau)[\gamma,z] = [s(h)+ \gamma, z+ \tau],
  \end{equation}
  %
  où $l+ \gamma$ désigne la juxtaposition du lacet $l$ avec l'arc $\gamma$,
  et les crochets indiquent les classes d'équivalences.
  Mais le groupe $T_\omega$ étant divisible,
  l'extension est triviale (voir par exemple \cite{KargMerz}) et le groupe $\Gamma$ est isomorphe au produit direct de $H_1(X,\ZZ)$ par $T_\omega$.
  Tout isomorphisme permet de définir une action de $H_1(X,\ZZ)$ sur $\hat Y$,
  dont le quotient $Y = \hat Y/H_1(X,\ZZ)$ est un fibré principal de groupe structural $T_\omega$ sur $X$.
  On vérifie alors que la forme de connexion $\hat \lambda$,
  puisqu'elle est invariante par ces actions de $H_1(X,\ZZ)$,
  passe sur $Y$ en une forme de connexion $\lambda$ de courbure $\omega$.
  Le fibré $Y\to X$,
  muni de la forme de connexion $\lambda$ est construit par quotients successifs:
  $$
    \begin{tikzcd}
    \Arc(X,x_o)\times T_\omega \arrow[r] \arrow[d] & \hat Y \arrow[r] \arrow[d] & Y \arrow[d] \\
    \Arc(X,x_o) \arrow[r] & \hat X \arrow[r] & X
    \end{tikzcd}
    $$
  Deux tels isomorphismes entre $\Gamma$ et $H^1(X,\ZZ)$ ne diffèrent que d'un élément du groupe $\Hom(H_1(X,\ZZ), T_\omega)$,
  c'est-à-dire d'un élément de $H^1(X,T_\omega)$.
  Grâce à l'unicité du fibré $(\hat Y, \hat \lambda)$,
  nous avons trouvé de cette façon tous les fibrés des périodes de $(X,\omega)$.
  Soit alors $\pi : Y \to X$ et $\pi':Y'\to X$ deux de ces fibrés obtenus par quotient de $\hat Y$ à partir des actions $\rho$ et $\rho'$ de $H_1(X,\ZZ)$.
  Ces actions diffèrent d'un élément $r\in \Hom(H_1(X,\ZZ), T_\omega)$.
  Il y a donc une surjection naturelle $\sigma$ de l'espace des classes de fibrés des périodes de $\omega$ sur $\Hom(H_1(X,\ZZ), T_\omega)$.
  On peut facilement vérifier,
  en prenant les images réciproques de $Y$ et $Y'$ par la projection naturelle de $\hat X$ sur $X$,
  qu'ils sont équivalents si et seulement si il existe une application $\zeta : \hat X \to T_\omega$ telle que $r(k) = \zeta(k\hat x) - \zeta (\hat x)$,
  pour tout $k\in H_1(X,\ZZ)$.
  La fonction $\zeta$ définit alors une $1$-forme fermée $\varepsilon$ sur $X$,
  par $\zeta^*\theta =\pi^*\varepsilon$,
  qui définit à son tour (par intégration sur le revêtement d'homologie) une fonction $F$ de $\hat X$ dans $\RR$,
  telle que $\zeta^*\theta =dF$;
  on en déduit $\zeta = \cl_\omega \circ F$.
  Autrement dit:
  le noyau de la surjection $\sigma$ est le sous groupe des homomorphisme de $H_1(X,\ZZ)$ dans $T_\omega$ provenant d'un homomorphisme de $H_1(X,\ZZ)$ dans $\RR$.
  Comme $\RR$ est divisible,
  on conclut,
  en utilisant la suite exacte du foncteur $\Ext$ {\em \cite{MacLane2}},
  que le co-noyau de la flèche naturelle $\Hom(H_1(X,\ZZ),\RR) \to \Hom(H_1(X,\ZZ),T_\omega)$ est exactement le premier groupe d'extension de $H_1(X,\ZZ)$ dans le groupe des périodes $P_\omega$,
  c'est-à-dire:
  $\Ext(H_1(X,\ZZ),P_\omega)$.
\end{demonstration}

\begin{remarque}
  \label{remsanstorsion}
  Si le groupe d'homologie $H_1(X,\ZZ)$ est sans torsion et si $X$ est une variété compacte,
  il y a unicité du fibré des périodes de la $2$-forme $\omega$.
  En effet,
  tout homomorphisme de $H_1(X,\ZZ)$ dans $T_\omega$ est alors la projection d'un homomorphisme à valeurs réelles.
\end{remarque}

\begin{remarque}
  \'Etant donné un fibré des périodes $\pi : Y \to X$,
  les formes de connexions inéquivalentes sur $Y$ de courbure $\omega$ sont classées par $H^1(X,\RR)$.
  L'ensemble des couples $(Y,\lambda)$ qui «intègrent» $(X,\omega)$ est donc classé,
  à isomorphisme près,
  par $H^1(X,T_\omega)$.
  Ce que nous dit le théorème précédent est que cette classification se scinde grâce à la suite exacte:
  %
  \begin{multline*}
    0  \to  \Hom(H_1(X,\ZZ),P_\omega) \to \Hom(H_1(X,\ZZ),\RR) \to \nonumber \\
    \to  \Hom(H_1(X,\ZZ),T_\omega) \to \Ext(H_1(X,\ZZ),P_\omega) \to 0
  \end{multline*}
  %
  Le groupe $\Ext(H_1(X,\ZZ),P_\omega)$ classe les fibrés des périodes,
  et le groupe dual $\Hom(H_1(X,\ZZ), \RR)$,
  c'est-à-dire $H^1(X,\RR)$,
  classe les formes de connexion.
\end{remarque}

\begin{remarque}
  Parce que les cocycles d'arcs de deux $2$-formes fermées cohomologues sont cohomologues,
  leurs fibrés des périodes sont équivalents.
  Autrement dit,
  la classe de cohomologie de la forme $\omega$ est,
  en un certain sens,
  la première «classe de Chern» de ces fibrés des périodes.
\end{remarque}

\begin{remarque}
  On peut être surpris que des fibrés des périodes existent pour toute forme fermée,
  même multiples les unes des autres par un réel non entier.
  Mais,
  c'est un exercice de montrer que pour $\omega = s\omega_0$,
  où $s\in \RR$,
  le fibré des périodes de $\omega_0$ est équivalent comme fibré à celui de $\omega$;
  mais attention:
  les groupes $T_\omega$ et $T_{\omega _0}$ ne sont pas identiques,
  seulement isomorphes.
  La différence avec la situation classique des $2$-formes entières est qu'il n'y a pas d'identification possible,
  {\em a priori},
  des tores des périodes pour les $2$-formes fermées quelconques.
  Une telle identification,
  dans le cas entier,
  consiste à fixer à l'avance la longueur de la période.
\end{remarque}



\subsection{Les difféomorphismes hamiltoniens revisités}
\index{difféomorphisme hamiltonien}

Considérons une variété différentiable $X$ munie d'une 2-forme fermée $\omega$ et $(Y,\lambda)$ un de ses fibrés des périodes;
nous noterons $\pi:X\to Y$ la projection.
Considérons le groupe des automorphismes $\Aut(Y,\lambda)$ des difféomorphismes de $Y$ qui commutent avec l'action du tore des périodes et qui préservent $\lambda$:
%****
\begin{equation}
  \Aut(Y,\lambda)=\{\varphi\in \Diff(Y) \mid \forall \tau\in T_\omega :\varphi\circ \tau_Y=\tau_Y\circ\varphi, \varphi^*\lambda=\lambda\}.
\end{equation}
%
Soit $\varphi$ un automorphisme de $(Y,\lambda)$;
grâce à la commutativité avec l'action de $T_\omega$,
il existe un difféomorphisme $\phi$ de $X$ défini par $\pi\circ\varphi=\phi\circ\pi$.
Nous dirons que $\phi$ est la {\em projection} de $\varphi$ et noterons $\phi=\pi_*\varphi$.
Le difféomorphisme $\phi$ préserve évidemment la courbure $\omega$ de $\lambda$:
$\phi^*\omega=\omega$.
%
\[
  \begin{tikzcd}
  Y \arrow[r, "\varphi"] \arrow[d, "\pi"'] & Y \arrow[d, "\pi"] \\
  X \arrow[r, "\phi=\pi_*\varphi"'] & X
  \end{tikzcd}
  \]
%
Le sous-groupe des difféomorphismes de $X$ préservant la 2-forme $\omega$ sera notée $\Diff(X,\omega)$:
%****
\begin{equation}
  \Diff(X,\omega)=\{\phi\in \Diff(X) \mid \phi^*\omega=\omega \}.
\end{equation}
%
Le groupe des périodes $T_\omega$ s'injecte,
par son action,
dans le groupe des automorphismes de $(Y,\lambda)$.
Il est clair qu'ainsi $T_\omega$ est dans le noyau de la projection $\pi_*$.
Nous avons plus précisément:
\begin{proposition}
  Le noyau de la projection $\pi_*:\Aut(Y,\lambda) \to \Diff(X,\omega)$ est le tore des périodes $T_\omega$.
  On a donc la suite exacte d'homomorphismes:
  
  
  \begin{equation}
    0 \longrightarrow T_\omega \longrightarrow \Aut(Y,\lambda) \xrightarrow{\displaystyle \pi_*} \Diff(X,\omega).
  \end{equation}
  
  
\end{proposition}
\begin{demonstration}
  Soit $\varphi$ un automorphisme de $(Y,\lambda)$ se projetant sur l'identité de $X$.
  Puisque $\pi:Y\to X$ est une fibration principale,
  il existe une application différentiable $\tau:Y\to T_\omega$ telle que $\varphi(y)= \tau(y).y$.
  L'identité $\varphi^*\lambda=\lambda$ signifie que pour toute plaque $P$ de $Y$ on a $P^*(\varphi^*\lambda)=P^*(\lambda)$.
  Considérons alors l'application $y\mapsto \tau(y).y$ comme la composition des applications:
  
  
  \begin{equation}
    \begin{array}{ccccc}
      Y & \to     & T_\omega\times Y    & \to     & Y         \\
      y & \mapsto & \vect{\tau(y) \\ y} & \mapsto & \tau(y).y
    \end{array}
  \end{equation}
  
  
  Nous pouvons écrire:
  
  
  \begin{equation}
    \varphi^*\lambda = \left[y\mapsto\vect{\tau(y)\\ y}\right]^*\circ \left[\vect{\tau \\ y}\mapsto\tau.y\right]^*\lambda.
  \end{equation}
  %****
  Soit $\hat y:T_\omega\to Y$ l'application orbite $\hat y: \tau\mapsto \tau.y$,
  $\tau_Y$ désignant toujours l'action de $\tau\in T_\omega$ sur $Y$,
  on a:
  
  
  \begin{equation}
    \left(\left[\vect{\tau \\ y}\mapsto\tau.y\right]^*\lambda\right)_{(\tau,y)} =(\hat y^*\lambda)_\tau \oplus (\tau_Y^*\lambda)_y = (\hat y^*\lambda)_\tau \oplus \lambda_y,
  \end{equation}
  %****
  On déduit ainsi:
  
  
  \begin{equation}
    \varphi^*\lambda_y = \left[y\mapsto \vect{\tau(y) \\ y}\right]^* ((\hat y^*\lambda)_{\tau(y)} \oplus \lambda_y)=(\tau^*\circ\hat y^* \lambda)_y + \lambda_y.
  \end{equation}
  %****
  On déduit ainsi de l'identité $\varphi^*\lambda=\lambda$ que $\tau^*(\hat y^* \lambda)=0$;
  mais puisque $\lambda$ est une forme de connexion sur $Y$:
  $y^*\lambda=\theta$ --- où,
  rappelons-le,
  $\theta$ est la forme de Maurer-Cartan sur $T_\omega$ ---,
  alors la seule solution pour que $\tau^*\theta=0$ est que $\tau={\rm cste}$.
  Autrement dit,
  il existe $\tau\in T_\omega$ tel que $\varphi(y) = \tau.y$.
  Le noyau de $\pi_*:\Aut(Y,\lambda)\to\Diff(X,\omega)$ est donc bien le tore des périodes $T_\omega$.
\end{demonstration}
\noindent On dit aussi,
en utilisant le langage des physiciens,
que les seules «transformations de jauges»\index{transformation de jauge} sont les transformations de «première espèce».
Soit $\mu_Y$ le moment de $\lambda$ par rapport à $\Aut(Y,\lambda)$,
au sens de la définition du moment des 1-formes (page \pageref{def0appmom}).
Il est défini sur $Y$,
à valeurs dans l'espace vectoriel des 1-formes différentielles invariantes par multiplication à gauche\footnote{La multiplication à gauche dans $\Aut(Y,\lambda)$ par un élément donné $f$:
$L_f(\varphi)=\varphi\circ f^{-1}$ est une application du groupe $\Aut(Y,\lambda)$ dans lui-même.
À ce titre,
on peut considérer l'image réciproque,
par $L_f$,
de toute forme différentielle $\alpha$ définie sur $\Aut(Y,\lambda)$.
Ici,
les termes sont à considérer au sens des espaces différentiables.
Il est clair que les 1-formes différentielles de $\Aut(Y,\lambda)$ invariantes par multiplication à gauche forment un sous-espace vectoriel de l'espace des 1-formes de $\Aut(Y,\lambda)$.
On peut l'appeler par commodité le {\em dual-de-l'algèbre-de-Lie} sans qu'il soit nécessaire de définir {\em a priori} l'algèbre de Lie de $\Aut(Y,\lambda)$.
Nous lui préférons le terme d'{\em espace des moments}.} du groupe $\Aut(Y,\lambda)$.
Nous noterons $\aut^*(Y,\lambda)$ cet espace vectoriel.
%
\begin{equation}
  \mu_Y : Y \to \aut^*(Y,\lambda) \quad \mu_Y : y \mapsto \hat y^*\lambda.
\end{equation}
%****
\begin{definition}
  Nous dirons qu'un difféomorphisme $\phi$ de $X$,
  préservant $\omega$,
  est un {\em difféomorphisme hamiltonien} s'il est l'image d'un automorphisme $\varphi$ du fibré des périodes $(Y,\lambda)$;
  nous noterons $\Ham_Y(X,\omega)\subset\Diff(X,\omega)$ le groupe des difféomorphismes hamiltoniens.
  %
  \begin{equation}
    \label{extppl}
    \{1\} \longrightarrow T_\omega \longrightarrow \Aut(Y,\lambda) \xrightarrow{\displaystyle \pi_*} \Ham_Y(X,\omega)\longrightarrow \{1\}
  \end{equation}
  
  
  L'action d'un groupe de Lie $G$ sur $X$ préservant $\omega$ sera dite {\em hamiltonienne} si l'image de $G$ dans $\Diff(X,\omega)$ est toute entière contenue dans $\Ham_Y(X,\omega)$.
\end{definition}
\noindent Cette définition appelle immédiatement la remarque suivante:

\begin{remarque}
  L'extension décrite ci-dessus (suite exacte \ref{extppl}) est centrale\footnote{Rappelons qu'une extension $1\to A\to \Gamma\to G\to 1$ de $G$ par $A$ est dite centrale si le groupe $A$ commute avec $\Gamma$,
  c'est-à-dire si tout élément de $A$ commute avec tous les éléments de $\Gamma$;
  ce qui implique,
  en particulier,
  que $A$ est abélien.},
  autrement dit,
  le groupe $\Aut(Y,\lambda)$ est une extension centrale du groupe $\Ham_Y(X,\omega)$ par le tore des périodes $T_\alpha$.
\end{remarque}

On en déduit immédiatement que le moment $\mu_Y$ de $Aut(Y,\lambda)$,
défini plus haut,
est constant sur les fibres de la projection $\pi$.
C'est une conséquence immédiate de la commutativité du tore des périodes $T_\omega\subset\Aut(Y,\lambda)$.
Autrement dit,
il existe une application $\mu_X$ définie sur $X$ à valeurs dans $\aut^*(Y,\lambda)$ définie par $\mu_X(x)=\mu_Y(y)$,
où $\pi(y)=x$.
$$
  \begin{tikzcd}[column sep=normal, row sep=normal, every label/.append style = {font = \small}]
  Y \arrow[dr,"\mu_Y"] \arrow[dd,swap,"\pi"] & {} \\
  {} & \aut^*(Y,\lambda) \\
  X \arrow[ru,swap,"\mu_X"] & {}
  \end{tikzcd}
  $$
Mais cela ne suffit pas encore à définir le moment de l'action de $\Ham_Y(X,\omega)$;
en effet $\mu_X$ est à valeurs dans $\aut^*(Y,\lambda)$;
or le moment de $\Ham_Y(X,\omega)$ doit bien être défini sur $X$,
mais à valeurs dans $\ham^*(X,\omega)$,
espace vectoriel des 1-formes invariantes de $\Ham_Y(X,\omega)$.
Il nous faut alors utiliser la remarque suivante.

\begin{remarque}
  La suite exacte de groupes (\ref{extppl}) se renverse pour donner une suite exacte de «duaux d'algèbre de Lie\footnote{Là encore il faut entendre duaux-d'algèbre-de-Lie,
  en un seul mot.}»:
  %
  \begin{equation}
    0 \longrightarrow \ham^*_Y(X,\omega) \longrightarrow \aut^*(Y,\lambda) \xrightarrow{\displaystyle i^*} \tau^*_\omega\longrightarrow 0
  \end{equation}
  
  
  En effet,
  l'injection $i$ de $T_\omega$ dans $\Aut(Y,\lambda)$ induit naturellement,
  par restriction,
  une projection de $\aut^*(Y,\lambda)$ sur $\tau^*_\omega$.
  Le noyau de cette projection est bien constitué des formes différentielles qui «s'annulent» sur les fibres.
  \'Etant invariantes et nulles le long des fibres,
  elles sont donc issues de la base $\Ham_Y(X,\omega)$ (voir annexe \ref{AnnED}).
\end{remarque}

Cette dernière remarque va nous permettre de définir un moment pour l'action de $\Ham_Y(X,\omega)$ sur $X$.
En effet,
$\mu_X(x)$ n'est jamais dans $\ham^*_Y(X,\omega)$ puisque justement $i^*(\mu_X(x))=i^*(\hat y^*\lambda)=(\hat y\circ i)^*\lambda = \theta$,
où $\theta$ désigne la forme de Maurer-Cartan du tore $T_\omega$.
Cela est une conséquence directe\footnote{Par définition même.} de ce que $\lambda$ est une forme de connexion sur $Y$.
Ainsi le moment $\mu_X$ est à valeurs dans un sous-espace affine de $\aut^*(Y,\lambda)$,
parallèle à $\ham^*_Y(X,\omega)$.
Il suffit donc de retrancher la valeur du moment $\mu_X$ en un point quelconque $o\in X$ pour être à valeurs dans $\ham^*_Y(X,\omega)$.
Nous définirons donc un moment de l'action de $\Ham_Y(X,\omega)$ sur $X$ de la façon suivante.
\begin{definition}
  Soit $o\in X$ un point de $X$,
  nous appellerons {\em moment} attaché à $o$,
  du groupe $\Ham_Y(X,\omega)$,
  l'application:
  $\mu_o : X \to \ham^*_Y(X,\omega)$ définie par:
  
  
  \begin{equation}
    \forall x\in X \quad \mu_o(x) = \mu_X(x)-\mu_X(o).
  \end{equation}
  
  
\end{definition}
Autrement dit:
soit $y_o$ un point de $Y$ au-dessus de $o\in X$,
et $y$ un point au-dessus de $x$,
le moment $\mu_o$ a la valeur:
%
\begin{equation}
  \mu_o(x) =\hat y^*\lambda - \hat y_o^*\lambda.
\end{equation}
%
Il faut maintenant se convaincre que cette définition est bien conforme à ce que l'on connait,
par ailleurs,
de l'application moment.
Nous l'établirons donc dans le cadre de l'action hamiltonienne (au sens défini plus haut) d'un groupe de Lie $G$ sur $X$.
Nous avons alors,
par définition,
une extension centrale:
%
\begin{equation}
  \{1\} \longrightarrow T_\omega \longrightarrow \tilde G \xrightarrow{\displaystyle \pi_*} G \longrightarrow \{1\}.
\end{equation}
%
Considérons alors un homomorphisme $\tilde h:\RR\to\tilde G$ et appliquons la formule de Cartan (voir \ref{formCartan}) à $\tilde h$ et $\lambda$:
%
\begin{equation}
  \DLie_{\tilde h}\lambda = d[i_{\tilde h}(\lambda)] + i_{\tilde h}[d\lambda],
\end{equation}
%
où,
rappelons-le:
$$
  \DLie_h\lambda= \left.{\partial h(t)^*\lambda \over \partial t}\right\vert_{t=0}.
  $$
Mais comme $\tilde h$ est justement,
pour tout $t$,
un automorphisme de $(Y,\lambda)$:
$\DLie_h\lambda=0$,
il reste l'identité:
%
\begin{equation}
  i_{\tilde h}[d\lambda]= - d[i_{\tilde h}(\lambda)].
\end{equation}
%
Soit alors $h=\pi_*\circ \tilde h$ l'homomorphisme de $\RR$ dans $G$,
projection de $\tilde h$.
Utilisons ensuite le fait que $\omega$ est la courbure de la connexion $\lambda$:
$d\lambda = \pi^*\omega$,
nous obtenons:
$i_{\tilde h}[d\lambda]=i_h(\omega)$.
D'autre part,
$i_{\tilde h}(\lambda)$ est une fonction de $Y$,
dont la valeur en $y\in Y$ est donnée par:
$i_{\tilde h}(\lambda)(y) = (\hat y_{\tilde h}^*\lambda)_0(1)$ (voir formule (\ref{remcontr1form}));
il suffit de vérifier que $\hat y_{\tilde h}=\hat y \circ \tilde h$ et on déduit immédiatement $i_{\tilde h}(\lambda)(y) = \tilde h^*(\hat y^*\lambda)_0(1) = i_{\tilde h}(\mu_Y(y))$.
Ainsi,
en utilisant la suite d'identités:
$d[i_{\tilde h}(\mu_Y(y))]= d[i_{\tilde h}(\mu_Y(y))-i_{\tilde h}(\mu_Y(y_o))] = d[i_{\tilde h}(\mu_o(x)]=d[i_{h}(\mu_o(x)]$,
on obtient l'expression finale:
%
\begin{equation}
  i_{h}[\omega]= - d[i_{h}(\mu_o(x))].
\end{equation}
%
On peut reconnaître sur cette dernière formule l'expression originale du moment tel qu'il a été défini par J.-M. Souriau \cite{Souriau5}.
Plus précisément en notant $Z_X$ le champ de vecteur associé à $h$ sur $X$:
$Z_X(x)= (\partial h(t)(x)/\partial t)_{t=0}$ l'expression précédente devient:
%
\begin{equation}
  \omega(Z_X,\cdot)= - d[\mu_o\cdot Z].
\end{equation}
%
Il ne nous reste plus,
avant de clore ce chapitre,
qu'à retrouver grâce à cette définition de l'application moment ses différentes variances par rapport à l'action de $G$.
Considérons donc un élément $g\in G$ (on peut supposer si on désire que $G=\Diff(X,\omega)$).
Soit $\tilde g\in \tilde G$ se projetant sur $G$.
On a%
\footnote{Rappelons que $R_k$ désigne,
sur un groupe,
la multiplication à droite par l'élément $k$.}:


\protected\def\tg{\tilde{}\!\!\!g}
\begin{eqnarray*}
  \mu_o(g(x)) & = & \widehat{\ \ \tg(y)}^*\lambda - \hat y_o^*\lambda \\ %% formule bricolée parcequ'il doit y avoir un bug dans l'implémentation de \widehat
  & = & (\hat y \circ R_{\tilde g})^*\lambda - \hat y_o^*\lambda \\
  & = & R_{\tilde g}^*(\hat y^*\lambda) - \hat y_o^*\lambda \\
  & = & R_{\tilde g}^*(\hat y^*\lambda -\hat y_o^*\lambda) + R_{\tilde g}^*(\hat y_o^*\lambda) - \hat y_o^*\lambda.
\end{eqnarray*}


Remarquons alors que:
$R_{\tilde g}^*(\hat y^*\lambda -\hat y_o^*\lambda) = R_{\tilde g}^*(\mu_o(x)) = R_{g}^*(\mu_o(x))$.
D'autre part,
$R_{g}^*(\mu_o(x)) = \Ad^*(g)(\mu_o(x))$,
par définition même de l'action coadjointe de $G$ sur le dual de son algèbre de Lie $\cG^*$.
Notons encore que $R_{\tilde g}^*(\hat y_o^*\lambda) - \hat y_o^*\lambda$ est un élément de $\cG^*$ ne dépendant que de $g$;
nous le noterons:
%
\begin{equation}
  \Theta(g)= R_{\tilde g}^*(\hat y_o^*\lambda) - \hat y_o^*\lambda.
\end{equation}
%
Il vient alors l'expression de la variance de $\mu_o$ par rapport au groupe $G$:
%
\begin{equation}
  \mu_o(g(x)) = \ad^*(g)(\mu_o(x)) + \Theta(g).
\end{equation}
%
Il est inutile de préciser que,
par construction même,
la fonction $\Theta$,
de $G$ dans $\cG^*$ est un 1-cocycle,
que l'on appelle le «défaut d'équivariance» du moment:
%
\begin{equation}
  \theta\in H^1(G,\cG^*), \quad \Theta(gg') = Ad^*(g')(\Theta(g)) +\Theta(g').
\end{equation}
%
Ce cocycle n'est pas nécessairement trivial,
même si on peut se laisser tromper par sa définition.
Il le serait (trivial) s'il existait une 1-forme $\alpha\in\cG^*$ telle que $\Theta(g)= \Ad^*(g)(\alpha)-\alpha$;
or dans la définition de $\Theta$,
c'est une 1-forme de $\tilde\cG^*$ qui intervient et non de $\cG^*$.
Si par hasard le cocycle $\Theta$ est trivial,
on peut alors ajouter une constante au moment $\mu$ de façon à le rendre équivariant\footnote{Pour une discussion générale sur la variance du moment et des exemples fameux de défaut d'équivariance (par exemple la masse totale d'un système dynamique),
voir \cite{Souriau5}.}.

\begin{remarque}
  Une condition évidente pour que le défaut d'équivariance $\Theta$ de l'action hamiltonienne d'un groupe de Lie $G$ soit trivial est que ce groupe ait un point fixe.
  En effet,
  supposons que $a\in X$ soit un point fixe:
  $g(a)=a$ pour tout $g\in G$.
  On a alors d'une part:
  $\mu_o(g(a)) = \Ad^*(g)(\mu_o(a)) + \Theta(g)$ et d'autre part:
  $\mu_o(g(a)) = \mu_o(a)\Rightarrow\Theta(g)=\mu_o(a) - \Ad^*(g)(\mu_o(a))$;
  en posant ensuite $\nu=-\mu_o(a)\in \cG^*$,
  il vient:
  $\Theta(g)= \Ad^*(g)(\nu)-\nu$.
  Ce qui prouve que le moment peut être choisi équivariant.
\end{remarque}

\begin{conclusion}
  Dans tous les cas où un groupe de Lie $G$ agit sur une variété $X$ en préservant une 2-forme fermée $\omega$,
  nous avons ramené la construction de l'{\em application moment} au cas le plus simple possible,
  celui d'une 1-forme invariante par un groupe de Lie,
  cas que nous avons largement commenté précédemment.
  Mais,
  pour cela nous avons dû:
  \begin{itemize}
    \item[1)] {\em intégrer} la 2-forme fermée invariante $\omega$ en construisant le fibré des périodes $Y$ associé pour l'y relever et l'y intégrer;
    \item[2)] sélectionner parmi les difféomorphismes $g\in G$ préservant $\omega$ les {\em difféomorphismes hamiltoniens},
    ceux qui se relèvent au fibré des périodes $Y$.
  \end{itemize}
  \noindent Une fois le moment défini sur $Y$,
  les problèmes de variance et de cohomologie associée se réduisent alors à des choix différents de réduction.
  Nous allons illustrer ces constructions générales dans les quelques exemples suivants.
\end{conclusion}

\subsection{Quelques exemples}

Illustrons maintenant les constructions précédentes par quelques exemples,
en laissant au lecteur intéressé et curieux le plaisir d'en construire d'autres\ldots

\textbf{Exemple 1 : Le tore $\TT^3$.}

Pour notre premier exemple,
considérons le tore\index{tore} $\TT^3$,
quotient de $\RR^3$ par le réseau $\ZZ^3$.
Nous noterons $X=(x,y,z)$ un point de $\RR^3$ et $[x,y,z]$ sa classe dans $\TT^3$.
Considérons la forme $\alpha$ sur $\RR^3$ définie par
%
\begin{equation}
  \alpha = aydz + bzdx + cxdy,\quad a,b,c \in \RR^3.
\end{equation}
%
Cette forme ne passe pas sur $\TT^3$,
mais sa dérivée extérieure $d\alpha$ oui.
Soit $\omega$ l'image de $d\alpha$ sur $\TT^3$,
que nous noterons de la manière suivante:
%
\begin{equation}
  \omega = [d\alpha] = [ady\wedge dz + bdz\wedge dx + cdx\wedge dy].
\end{equation}
%
Le groupe des périodes de la forme $\omega$ est le sous-groupe de $\RR$ engendré par $a$,
$b$ et $c$.
On a donc:
%
\begin{equation}
  P_\omega = a\ZZ + b\ZZ + c\ZZ\subset \RR.
\end{equation}
%
Plusieurs possibilités s'offrent alors:
le groupe des périodes peut être isomorphe à $\ZZ$,
$\ZZ^2$ ou $\ZZ^3$ suivant les propriétés de commensurabilité des coefficients $a$,
$b$ et $c$.
Supposons,
par exemple,
que pour tout triplet $l, m, n$ d'entiers relatifs $la+mb+nc=0 \Rightarrow l=m=n=0$;
alors $a$,
$b$ et $c$ sont incommensurables et le groupe des périodes est égal à $a\ZZ \oplus b\ZZ \oplus c\ZZ\subset\RR$.
Dans tous les cas,
ce qui suit est valable.

Dans le cas du tore $\TT^3$,
le fibré des périodes $Y$ est unique à isomorphisme près.
D'autre part,
l'image réciproque du fibré des périodes par la projection canonique $\RR^3\to \TT^3$ est trivial,
et donc isomorphe à $\RR^3\times T_\omega$.
Il peut donc se reconstruire par quotient en relevant l'action de $\ZZ^3$ sur $\RR^3\times T_\omega$.
Cette action est du type:
%
\begin{equation}
  L \vect{X\\ \tau} = \vect{X+L \\ \tau +\phi(X,L)}, \quad L=\vect{l\\ m\\ n}\in\ZZ^3,\quad \tau=[t]\in T_\omega,
\end{equation}
%
où l'action sur $T_\omega$ est notée additivement et où $\phi$ est un cocycle,
c'est-à-dire qu'il vérifie la propriété suivante:
%
\begin{equation}
  \label{propT3cocycle}
  \phi(X,L+L') = \phi(X+L,L') +\phi(X,L').
\end{equation}
%
Pour expliciter $\phi$,
introduisons la forme de connexion $\Lambda$,
définie sur $\RR^3\times T_\omega$,
image réciproque de la «forme d'intégration» $\lambda$ de $\omega$,
définie sur $Y$:
%
\begin{equation}
  \Lambda =\alpha + d\tau = aydz + bzdx + cxdy + d\tau.
\end{equation}
%
La condition d'invariance $L^*\Lambda = \Lambda$ impose tout de suite:
%
\begin{equation}
  amdz+bndx+cldy+d\phi(X,L)=0.
\end{equation}
%
On en déduit alors l'expression du cocycle\footnote{Vérifier que la propriété (\ref{propT3cocycle}) est bien satisfaite.} $\phi$:
%
\begin{equation}
  \phi(X,L) =- [amz+bnx+cly]\in T_\omega.
\end{equation}
%
Ainsi le fibré des périodes de la 2-forme $\omega$ au dessus de $\TT^3$ est le quotient:
%
\begin{equation}
  Y = \RR^3\times T_\omega / \ZZ^3, \quad \vect{X \\ \tau} \sim L\vect{X \\ \tau} = \vect{X+L \\ \tau - \left[\bar L WX\right]},
\end{equation}
%
où la barre désigne la transposée et où on a posé:
%
\begin{equation}
  W = \left(
  \begin{array}{ccc}
    0 & c & 0 \\
    0 & 0 & a \\
    b & 0 & 0
  \end{array}
  \right).
\end{equation}
%
Avec ces notations,
la forme de connexion $\Lambda$ s'écrit aussi:
%
\begin{equation}
  \Lambda = \bar X W dX + d\tau.
\end{equation}
%
Les translations du tore $\TT^3$ ($\sim \TT^3$) préservent la 2-forme $\omega$,
la vérification étant immédiate.
Nous allons montrer que cette action n'est pas hamiltonienne,
autrement dit qu'elle ne se relève pas en un automorphisme de $Y$,
sauf pour le seul sous-groupe engendré par le vecteur $(a,b,c)$.
Considérons donc la translation associée à un vecteur $V\in\RR^3$,
et supposons que cette action soit hamiltonienne.
Commençons par relever cette translation de $\TT^3$ sur $\RR^3$,
soit $\phi(X) = X +V$.
Le relevé de $\varphi$ sur $Y$ se relève sur $\RR^3\times T_\omega$ en:
%
\begin{equation}
  \psi\vect{X\\ \tau} = \vect{X+V \\ \tau + f(X,V)},
\end{equation}
%
mais $\psi$ préserve la forme de connexion:
$\psi^*\Lambda=\Lambda$.
Un calcul analogue aux précédents donne la valeur de $f$:
%
\begin{equation}
  f(X,V) = - \bar VWX + \tau_o \quad \Rightarrow \quad \psi\vect{X\\ \tau} =\vect{X+V\\\tau - \bar VWX + \tau_o}.
\end{equation}
%
Il nous reste à vérifier que cet automorphisme de $(\RR^3\times T_\omega,\Lambda)$ passe bien au quotient en un automorphisme de $(Y,\lambda)$,
c'est-à-dire qu'il commute avec l'action de $\ZZ^3$;
il est immédiat de vérifier que:
%
\begin{equation}
  \psi\circ L \vect{X\\ \tau} = L\circ \psi \vect{X\\ \tau} \Rightarrow \bar LWV = \bar VWL \quad \forall L\in \ZZ^3.
\end{equation}
%
Autrement dit,
$\phi$ se relève sur $\RR^3\times T_\omega$ en un automorphisme équivariant si et seulement si:
%
\begin{equation}
  \bar V[\bar W-W]L=0, \forall L\in \ZZ^3.
\end{equation}
%
Mais $\bar W-W$ est une matrice antisymétrique de $\RR^3$ et vaut:
%
\begin{equation}
  \bar W-W = j(\Omega),\quad \Omega = \vect{a\\ b\\ c},
\end{equation}
%
où $j$ désigne le produit vectoriel de $\RR^3$.
Ainsi,
la translation par le vecteur $V$ n'est hamiltonienne que si:
%
\begin{equation}
  \forall L\in \ZZ^3 : \scal(V,\Omega\wedge L)= (V\wedge\Omega, L)= 0 \quad \Rightarrow \quad V \propto \Omega.
\end{equation}
%
Remarquons toutefois que le moment associé au groupe des translations engendré par $\Omega$ est nul.
En effet,
soit $y=(X,\tau)\in\RR^3\times T_\omega$,
on a:
%
\begin{equation}
  \hat y : s \mapsto s\Omega\vect{X\\ \tau} = \vect{X + s\Omega \\ \tau - s \bar\Omega W X}, \quad s\in \RR,
\end{equation}
%
où l'on a choisi la constante additive nulle.
Le calcul de $\hat y^*\Lambda$ est immédiat:
%
\begin{equation}
  \hat y^*\Lambda_{s=0}(1) = \bar X[W-\bar W]\Omega =\scal(X,\Omega\wedge\Omega)=0.
\end{equation}
%
La raison de cela est simplement que le groupe des translations engendré par $\omega$ est feuilletage caractéristique de la 2-forme $\omega$,
ce qu'il est facile de vérifier directement.

\textbf{Exemple 2 : Le groupe de Heisenberg.}

L'exemple suivant est fameux car il définit un groupe important en physique théorique:
le {\em groupe de Heisenberg}\index{groupe de Heisenberg}.
Considérons le plan vectoriel $X=\RR^n\times \RR^n$ muni de la forme symplectique canonique $\omega=\sum_{i=1}^n dy^i\wedge dx^i$.
Considérons $\RR^n\times \RR^n$ comme agissant par translation sur lui-même,
notons:
%
\begin{equation}
  T_U(X) = X + U, \mbox{ avec } X = \vect{x\\ y} \mbox{ et } U=\vect{u\\ v}.
\end{equation}
%
Cette action est évidemment symplectique:
$T_U^*\omega=\omega$;
elle est hamiltonienne et nous allons calculer son moment par la méthode exposée précédemment.
La forme $\omega$ est exacte;
choisissons comme primitive\footnote{On omet la somme sur l'indice $i$ sans risque de confusion.}:
%
\begin{equation}
  \alpha = \undemi[ydx-xdy] .
\end{equation}
%
Le fibré des périodes $Y$ est le produit direct $\RR^n\times \RR^n\times \RR$,
notons:
%
\begin{equation}
  y=\vect{x\\ y\\ t}\in Y=\RR^n\times \RR^n \times \RR.
\end{equation}
%
La forme de connexion $\lambda$ de courbure $\omega$ est évidemment donnée par:
%
\begin{equation}
  \lambda = \alpha + dt = \undemi[ydx-xdy] + dt.
\end{equation}
%
Soit $\varphi$ un relevé d'une translation,
on a nécessairement:


$$
  \varphi \vect{X\\ t} = \vect{X+U \\ t+ F(X,U)},
  $$


la fonction $F$ étant déterminée par la condition d'invariance:
$\varphi^*\lambda=\lambda$.
On obtient après un petit calcul:
%
\begin{equation}
  \varphi^*\lambda=\lambda \quad \Rightarrow \quad F(X,U) = \undemi \omega(X,U) +c.
\end{equation}
%
Le difféomorphisme $\varphi$ est donc caractérisé par le couple $(U,c)$,
ce que nous écrirons de la manière suivante:
%
\begin{equation}
  \vect{U\\ c}\vect{X\\ t} = \vect{X+U \\ t+ c + {1\over 2}\omega(X,U)}.
\end{equation}
%
L'extension de $\RR^{2n}$ par $\RR$ que nous venons de construire ainsi s'appelle le {\em groupe de Heisenberg}:
%
\begin{equation}
  0\longrightarrow \RR \longrightarrow {\rm Heisenberg}\longrightarrow\RR^{2n} \longrightarrow 0.
\end{equation}
%
La loi de groupe s'explicite en composant deux relevés ou,
si l'on préfère,
en ré-interprétant la formule précédente,
car dans cet exemple l'espace et le groupe se confondent:
%
\begin{equation}
  \vect{U\\ c}\vect{U'\\ c'} = \vect{U+U' \\ c+ c' + {1\over 2}\omega(U,U')}.
\end{equation}
%
Le calcul explicite du moment est une formalité.
Pour que les choses soient claires,
notons deux ou trois formules nécessaires à ce calcul.
Soit $(Z,\epsilon)$ un élément de l'algèbre de Lie du groupe de Heisenberg (soit $\cH$),
son action infinitésimale sur $\RR^{2n}$ est donnée par:
%
\begin{equation}
  \vect{Z \\ \epsilon}_Y\vect{X\\ t} = \vect{Z\\ \epsilon +{1\over 2}\omega(X,Z)}, \quad \vect{Z \\ \epsilon}\in \cH.
\end{equation}
%
Le calcul des moments respectifs donne alors:
%
\begin{equation}
  \hat y^*\lambda_{U=0 \atop c=0}\vect{Z\\ \epsilon}= \omega(X,Z) + \epsilon \quad \mbox{ et } \quad \hat y_o^*\lambda_{U=0 \atop c=0} \vect{Z\\ \epsilon} = \epsilon,
\end{equation}
%
d'où le résultat:
%
\begin{equation}
  \mu_o(X) = \hat y^*\lambda - \hat y_o^*\lambda = \omega(X,\cdot)= [Z\mapsto \omega(X,Z)].
\end{equation}
%

\textbf{Exemple 3 : Le cocycle de Bott-Thurston.}
\index{cocycle de Bott-Thurston}
Dans l'exemple qui suit,
la 2-forme est encore exacte.
Nous nous trouvons dans un cas analogue au précédent,
mais l'espace sur lequel est définie cette forme différentielle est de dimension infinie.
Nous laissons au lecteur,
comme exercice de difféologie,
le soin de vérifier que les affirmations qui suivent sont fondées.
L'intérêt de cet exemple est qu'il fait apparaître une version du groupe de Virasoro,
groupe qui a récemment connu une certaine vogue chez les physiciens.
On voit apparaître le {\em cocycle de Bott-Thurston} et la {\em dérivée Schwarzienne} ainsi que le {\em cocycle de Gelfand-Fuchs} comme différents produits de cet exemple.

Considérons l'espace $\Imm(S^1,\RR^2)$ des immersions du cercle\index{immersion du cercle} $S^1$ dans $\RR^2$.
Il est pointé par le plongement ordinaire $\iota : S^1 \hookrightarrow \RR^2$.
Nous le munissons de la $1$-forme suivante\footnote{Exercice:
donner la définition de $\alpha$ au sens des espaces difféologiques.}:
%
\begin{equation}
  \alpha(\delta\gamma) = \int {1 \over \norm{\gamma'}^2}\scal(\gamma'',\, \delta \gamma'),
\end{equation}
%
où $\delta\gamma$ est un «vecteur tangent» en $\gamma$ à l'espace $\Imm(S^1,\RR^2)$,
\cad un relèvement dans $T\RR^2$ de l'immersion $\gamma$:
$\delta\gamma(z)\in T_{\gamma(z)}\RR^2$;
d'autre part $\gamma'$ désigne la dérivée par rapport au paramètre de l'immersion,
et l'intégration se fait sur le cercle $S^1$.
Soit $\omega = d\alpha$.
Considérons l'(anti)-action naturelle des difféomorphismes du cercle $S^1$ sur $\Imm(S^1)$ par composition:
%
\begin{equation}
  \forall \phi\in\Diff(S^1), \quad \phi(\gamma)= \gamma\circ \phi.
\end{equation}
%
On vérifie que
$$
  \phi^*\alpha = \alpha+ dF(\phi)
  $$
pour tout difféomorphisme $\phi \in \Diff^\circ(S^1)$,
avec:
%
\begin{equation}
  F(\phi) : \gamma \mapsto \int \log\norm{(\gamma \circ \phi)'}\ d\log\phi'.
\end{equation}
%
La $2$-forme $\omega$ est donc invariante par $\Diff^\circ(S^1)$,
et en voici une expression:
%
\begin{eqnarray}
  \omega(\delta\gamma,\delta'\gamma) = \int & \displaystyle{1 \over \mbox{\raisebox{-2pt}{$\norm{\gamma'}^2$}}} & \left\{ \langle \delta\gamma'' - \delta[\log\norm{\gamma'}^2]\gamma'',\delta'\gamma'\rangle\right.
  \nonumber \\
  &                                                                 & \left. - \langle \delta'\gamma'' - \delta'[\log\norm{\gamma'}^2]\gamma'', \delta\gamma' \rangle\right\}.
\end{eqnarray}
%
L'action de $\Diff(S^1)$ est donc évidemment hamiltonienne.
Son extension centrale par $\RR$ est définie par:
%
\begin{equation}
  \vect{\phi\\c}\vect{\gamma \\ t} = \vect{ \gamma\circ \phi \\ \displaystyle{t + c - \int\log\norm{(\gamma\circ \phi)'}\ d\log\phi'}}.
\end{equation}
%%
La loi de groupe,
obtenue par composition,
est donnée par:
%
\begin{equation}
  \vect{\phi\\c}\vect{\psi \\ e} = \vect{ \phi\cdot\psi \\ c +e + K_\omega (\phi,\psi)},
\end{equation}
%%
où $\phi\cdot\psi=\psi\circ\phi$,
et $K_\omega$ étant le 2-cocycle suivant,
de $\Diff(S^1)$ dans $\RR$:
%
\begin{equation}
  K_\omega (\phi,\psi) =\int \log(\psi \circ \phi )' \ d\log\phi'.
\end{equation}
%
On reconnait en $K_\omega$ l'expression du {\em cocycle de Bott-Thurston} \cite{Bott1} qui décrit la seule extension centrale non triviale du groupe $\Diff^\circ(S^1)$ par $\RR$.
Un calcul ordinaire donne ensuite l'expression du moment associé à cette action de $\Diff(S^1)$:
%
\begin{equation}
  J(\gamma) : \eta \mapsto \int \left[ {\norm{\gamma''}^2 \over \mbox{\raisebox{-2pt}{$\norm{\gamma'}^2$}}} -\Big(\log\norm{\gamma'}^2\Big)'' \right] \eta.
\end{equation}
%
Nous laissons en exercice le calcul du défaut d'équivariance $\Theta$.
Il faut toutefois faire attention que $\int\phi_*\eta = \int \eta (\phi')^2$;
il vient immédiatement:
%
\begin{equation}
  \Theta (\phi) : \eta \mapsto \int {3\phi''^2 - 2\phi'''\phi' \over \phi'^2}\ \eta
\end{equation}
%
On reconnait l'expression de la {\em dérivée Schwarzienne} dont la classe de cohomologie représente le défaut d'équivariance du moment $J$ de cette $2$-forme $\omega$.
Il est facile ensuite de calculer le cocycle infinitésimal,
que l'on reconnait évidemment comme le {\em cocycle de Gelfand-Fuchs}\index{cocycle de Gelfand-Fuchs} \cite{GelfandFuchs}:
%
\begin{equation}
  k_\omega(\xi,\eta) = \int \xi''\eta'-\eta''\xi'.
\end{equation}
%

\textbf{Exemple 4 : Les courbes à courbure prescrite.}

Nous allons achever cette série d'exemples par un retour au disque de Poincaré\index{disque de Poincaré},
que nous avons introduit au chapitre \ref{exdisquepoincare}.
Nous en reprenons les notations.
Rappelons que nous identifions le disque de Poincaré $H^2$ avec la nappe supérieure de l'hyperboloïde à deux nappes,
dans l'algèbre de Lie $\sl(2,\RR)\sim \RR^3$ du groupe $\SL(2,\RR)$,
d'équation $x^2+y^2-z^2=-1$;
rappelons aussi que cette nappe est l'orbite adjointe du point:
%
$$
  X_0 = \mat 0,1,{-1},0.
  $$
%
Considérons le fibré des directions tangentes à $H^2$,
c'est-à-dire le fibré des demi-droites tangentes au disque.
C'est une variété sur laquelle agit naturellement le groupe $\SL(2,\RR)$.
Grâce à la métrique de Poincaré,
nous l'identifions au fibré unitaire tangent $UH^2$,
\cad à l'ensemble des couples $(X,U)$ où $U$ est un vecteur tangent à $H^2$ de carré 1,
\ie $\norm{U}^2=1$ (voir la définition \ref{defnormeDP} de la métrique de Poincaré).
Dans le paragraphe \ref{exdisquepoincare},
nous avons étudié le feuilletage géodésique du disque de Poincaré,
\cad le système dynamique défini par le feuilletage caractéristique de la 2-forme fermée $\omega=d\alpha$ avec $\alpha = \bar U dX$,
définie sur $UH^2$;
nous avons remarqué que cette 2-forme est invariante par l'action adjointe de $\SL(2,\RR)$.
Nous allons étudier maintenant le système dynamique défini par le feuilletage caractéristique de la 2-forme fermée la plus générale invariante par $\SL(2,\RR)$ définie sur le fibré des directions de $H^2$ (\cad sur $UH^2$).

Notre première remarque sera que le fibré unitaire tangent est en bijection avec le groupe $\SL(2,\RR)$ lui-même.
En effet,
nous savons que l'action adjointe de $\SL(2,\RR)$ est transitive sur $H^2$.
Pour tout $X\in H^2$,
il existe $A\in SL(2,\RR)$ tel que $AX_0A^{-1}=X$.
Soit $A$ et $B$ deux éléments de $\SL(2,\RR)$ tels que:
$X=AX_0A^{-1}= BX_0B^{-1}$,
alors $B=AM$ et $M\in\SO(2,\RR)\subset\SL(2,\RR)$.
En effet,
par hypothèse $MX_0M^{-1}=X_0$,
c'est-à-dire $MX_0=X_0M$.
On vérifie immédiatement que cette condition équivaut à:
$$
  M=\mat a,b,{-b},a, \quad \mbox{avec} \quad a^2+b^2=1 \quad \Rightarrow \quad M\in\SO(2,\RR).
  $$
Introduisons alors le vecteur $U_0$ tangent à $H^2$ au point $X_0$:
%
\begin{equation}
  U_0=\mat 1,0,0,{-1},\quad U_0\in U_{X_0}H^2.
\end{equation}
%
Considérons ensuite un couple $(X,U)\in UH^2$ et soit $A\in\SL(2,\RR)$ tel que:
$AX_0A^-1=X$.
Soit $U'_0$ le vecteur tel que $AU'_0A^{-1}=U$.
Le vecteur $U'_0$ est un vecteur unitaire dans le plan tangent à $H^2$ au point $X_0$;
il existe un élément unique $M\in\SO(2)$ tel que $MU_0M^{-1}=U'_0$ (le vérifier).
On pose alors $B=AM$ et l'on a $BX_0B^{-1}=X$ et $BU_0B^{-1}=U$,
$B$ étant unique.
L'équivalence:
%
\begin{equation}
  UH^2=\{(X,U)\mid X\in H^2, \norm{U}^2=1, \scal(X,U)=0\} \sim \SL(2,\RR)
\end{equation}
%
est formellement définie par:
%
\begin{equation}
  UH^2\ni (X,U)\mapsto A\in\SL(2,\RR) \Leftrightarrow AX_0A^{-1} = X \mbox{ et } AU_0A^{-1} = U.
\end{equation}
%
Notons ici que l'action adjointe de $\SL(2,\RR)$ sur $UH^2$ se traduit simplement par la multiplication à gauche sur $\SL(2,\RR)$,
grâce à l'identification:
$$
  (X,U)\sim A \Leftrightarrow (MXM^{-1},MUM^{-1})\sim MA.
  $$
Nous devons donc chercher l'expression la plus générale de 2-forme fermée sur $UH^2$ invariante par l'action adjointe;
ou,
ce qui revient au même,
l'expression la plus générale de 2-forme fermée sur $\SL(2,\RR)$ invariante par la multiplication à gauche.
Nous allons immédiatement vérifier que ces formes constituent un espace vectoriel de dimension 3.
Orientons d'abord l'algèbre de Lie $\sl(2,\RR)$:
soit {\em vol} la forme volume définie par:
%
\begin{equation}
  \vol(A,B,C) = -\undemi \scal(A,[B,C]).
\end{equation}
%
Elle est telle que la base faite des vecteurs
%
$$
  U_0=(1,0,0),\quad J_0= (0,1,0) = \undemi [U_0,X_0] \quad \mbox{et} \quad X_0=(0,0,1)
  $$
%
soit directe et de volume 1.
Soit $C^2$ la nappe hyperboloïde des vecteurs de $\sl(2,\RR)$ de carré scalaire 1.
Soit $\surf_{H^2}$ et $\surf_{C^2}$ les formes de surfaces associées au volume $\vol$ sur les sous-variétés $H^2$ et $C^2$ de $\sl(2,\RR)$,
définies par
%
$$
  \surf_{\Sigma}(x)(u,v) = \vol(x,u,v),\quad x\in\Sigma, (u,v)\in T_x\Sigma^2,\quad \Sigma=H^2 \mbox{ ou } C^2.
  $$
%
Avec ces notations,
la 2-forme la plus générale invariante définie sur $UH^2\sim \SL(2,\RR)$ dépend de trois constantes $a$,
$b$,
$c$ et s'exprime de la façon suivante:


\begin{eqnarray}
  \omega(\delta Z,\delta'Z) & = & a[\scal(\delta U,\delta'X) -\scal(\delta'U,\delta X)]\nonumber\\
  & + & b \surf_{H^2}(X)(\delta X,\delta'X) \nonumber \\
  & + & c \surf_{C^2}(U)(\delta U,\delta'U).
\end{eqnarray}


En effet,
toute 2-forme différentielle sur $\SL(2,\RR)$ invariante par multiplication à gauche est entièrement définie par sa valeur en l'identité.
L'espace des 2-formes invariantes sur $\SL(2,\RR)$ est donc au plus de dimension 3,
dimension de l'espace des 2-formes linéaires sur $\sl(2,\RR)=T_\id\SL(2,\RR)$.
Il suffit donc de montrer que la 2-forme précédente est nulle si et seulement si $a=b=c=0$.
Cela se vérifie facilement en introduisant le vecteur $J=(1/2)[U,X]$,
et en explicitant la forme $\omega$ sur la base «orthonormée» $(U,J,X)$.
On vérifie ensuite qu'elle est bien fermée puisque le premier terme est une différentielle et les deux autres termes les images réciproques des éléments de surface sur $H^2$ et $C^2$ par les deux projections $(X,U)\mapsto X$ et $(X,U)\mapsto U$.

Il est alors facile de déterminer le noyau de $\omega$,
qui est de dimension 1 (si $\omega$ est non-nulle évidemment):
%
\begin{equation}
  \label{sysdynsl2}
  {dZ\over ds}\in \ker \omega \Leftrightarrow {dX\over ds} = \lambda (aU-cJ) \mbox{ et } {dU\over ds} =\lambda (aX+bJ), \lambda\in \RR.
\end{equation}
%
Pour que le feuilletage caractéristique de $\omega$ soit {\em holonome}\index{holonome},
\cad pour qu'il soit solution d'une équation différentielle du second degré sur $H^2$,
il faut que $c=0$.
Le noyau est alors engendré par le champ de vecteurs suivant:
%
\begin{equation}
  \label{eqsolsysdyn}
  \left\{
  \begin{array}{rcl}
    \displaystyle{dX\over ds} & = & aU \\
    \\
    \displaystyle{dU\over ds} & = & aX+bJ
  \end{array}
  \right.
\end{equation}
Alors $U$ est bien la vitesse unitaire du système dynamique.
C'est ce que nous allons supposer dorénavant:
%
\begin{equation}
  \label{sysdynsl2form}
  \omega(\delta Z,\delta'Z) = a[\scal(\delta U,\delta'X) -\scal(\delta'U,\delta X)] + b \surf_{H^2}(X)(\delta X,\delta'X).
\end{equation}
%
Comme nous pouvons le voir sur les équations précédentes (\ref{eqsolsysdyn}),
les solutions sont les courbes de $H^2$ de courbure constante $k=-b$,
la courbure étant donnée par $k = -J\cdot(dU/ds)$ où $U=dX/ds$ est la vitesse unitaire et $J$ le vecteur normal,
orienté positivement,
de la courbe.

Pour résoudre explicitement ce système,
nous allons utiliser le fait que l'action de $\SL(2,\RR)$,
définie plus haut,
est hamiltonienne.
Pour calculer le moment $\mu$,
précisons l'action infinitésimale de $\SL(2,\RR)$ sur $UH^2$,
naturellement définie par:
%
\begin{equation}
  \zeta\in \sl(2,\RR), \quad \zeta_{UH^2}(X,U) = ([\zeta,X],[\zeta,U]).
\end{equation}
%
Le moment est obtenu en résolvant l'équation différentielle:
%
\begin{equation}
  \omega(\zeta_{UH^2}(X,U),\delta(X,U)) = -\delta(\mu(X,U)\cdot \zeta).
\end{equation}
%
Ce qui donne:
%
\begin{equation}
  \mu(X,U) = a[X,U] + 2b X + \cst.
\end{equation}
%
Si on choisit la constante nulle et que l'on introduit le vecteur J défini plus haut,
il vient:
%
\begin{equation}
  \mu(X,U) = 2bX-2aJ \quad \Rightarrow \quad \norm{\mu(X,U)}^2 = 4(a^2-b^2).
\end{equation}
%
D'après le théorème de N{\oe}ther,
nous savons que $\mu$ est constant sur les solutions et par homogénéité nous déduisons que (les composantes de) la surface dans $\sl(2,\RR)$ définie par l'équation $\norm{\mu}^2=4(a^2-b^2)$ réalise(nt),
grâce à l'application moment,
l'espace des mouvement du système dynamique défini par les caractéristiques de $\omega$.
Nous voyons que trois cas se distinguent,
correspondant aux trois types d'orbites coadjointes\footnote{Les différentes composantes des hyperboloïdes à deux nappes ou composantes du cône isotrope correspondent aux solutions orientées dans un sens ou dans un autre.} de $\SL(2,\RR)$:
%
\begin{equation}
  \begin{array}{rcl}
    \module{a}<\module{b} & : & \mbox{cas elliptique -- hyperboloïdes à deux nappes} \\
    \module{a}=\module{b} & : & \mbox{cas parabolique -- cône isotrope}\\
    \module{a}>\module{b} & : & \mbox{cas hyperbolique -- hyperboloïdes à une nappe}
  \end{array}
\end{equation}
%
Autrement dit,
l'application moment nous permet d'interpréter chaque orbite coadjointe comme l'espace des solutions d'un système dynamique du type (\ref{sysdynsl2}),
défini comme les caractéristiques de la 2-forme $\SL(2,\RR)$ invariante (\ref{sysdynsl2form}).
Si on note $m=a$ et $qB=b$,
on peut interpréter physiquement ce système comme le mouvement d'une particule test de masse $m$ de charge électrique $q$,
soumise à la gravitation représentée par la métrique de Poincaré et à un champ électromagnétique «constant»\footnote{C'est-à-dire orthogonal au disque et de carré scalaire constant.} B.

Nous pouvons,
grâce au moment $\mu$,
analyser plus précisément la nature géométrique des caractéristiques de la forme $\omega$ (ce que nous appelons les {\em mouvements} du système).
Soit $M=\mu(X_0,U_0)$ la valeur du moment le long du mouvement de conditions initiales $(X_0,U_0)$.
C'est un point de la nappe normique $4(a^2-b^2)$.
On remarque qu'en tout point du mouvement $M\cdot X = -2b$ et $M\cdot U = 0$.
Autrement dit,
le mouvement est entièrement contenu dans le plan affine $P_{-2b}$ d'équation $M\cdot X = -2b$.
Puisque d'autre part le mouvement est une courbe de la nappe $H^2$,
c'est exactement l'intersection $P_{-2b}\cap H^2$.
La nature de l'intersection de $P$ avec $H^2$,
plus précisément le nombre de points à l'infini de cette intersection (0,1 ou 2) nous dit dans quel cas (elliptique, parabolique ou hyperbolique) nous nous trouvons.
Il est intéressant d'expliciter ces courbes dans le modèle du disque de Poincaré.
Soit
%
\renewcommand{\arraystretch}{1.5}
\begin{equation}
  X=\vect{x\\ y\\ z} \mapsto \renewcommand{\arraystretch}{2} Y = \vect{\alpha =\displaystyle{x\over 1+z} \\ \beta =\displaystyle{{y}\over 1+z}}.
\end{equation}
\renewcommand{\arraystretch}{1}
%
La projection stéréographique de la nappe $H^2$ sur le plan $z=0$,
dont l'image est le disque de Poincaré d'équation $\norm{Y}^2<1$,
l'inverse étant donné par:
%
\begin{equation}
  x={2\alpha \over 1-\norm{Y}^2} \quad y={2\beta \over 1-\norm{Y}^2} \quad z={1+ \norm{Y}^2 \over 1-\norm{Y}^2}.
\end{equation}
%
Notons $M=(r, s, t)$;
l'intersection $P_{-2b}\cap H^2$ dans le plan $z=0$ se projette,
grâce à la projection stéréographique ci-dessus,
comme la courbe d'équation:
%
\begin{eqnarray}
  M\cdot X                                                                               & = & -2b \nonumber \\
  rx+sy-tz                                                                                & = & -2b \nonumber \\
  r{2\alpha\over 1-\norm{Y}^2} + s{2\beta\over 1-\norm{Y}^2} -t{1+\norm{Y}^2 \over 1-\norm{Y}^2} & = & -2b \nonumber \\
  2(\alpha r + \beta s) - t(1+\norm{Y}^2)                                                  & = & -2b(1-\norm{Y}^2). \nonumber
\end{eqnarray}
%
Notons alors
%
$$
  M= \vect{T \\ t}, \quad T=\vect{r\\ s}\in \RR^2.
  $$
%
En supposant $2b+t\neq 0$,
l'équation du mouvement dans le plan devient alors,
en regroupant les termes comme il se doit:
%
\begin{eqnarray}
  \label{eqsol}
  2T\cdot Y-t({1+ \norm{Y}^2}) & = & -2b ({1-\norm{Y}^2}) \nonumber \\
  \norm{Y-C}^2                & = & \norm{C}^2 + {2b-t\over 2b+t} \nonumber \\
  \norm{Y-C}^2                & = & \rho^2
\end{eqnarray}
%
où on a posé:
%
\begin{equation}
  C = {T\over 2b+t} \quad \mbox{et} \quad \rho = \left|{2a\over 2b+t}\right|.
\end{equation}
%
On reconnait ainsi l'équation d'un cercle de centre $C$ et de rayon $\rho$.
Nous pouvons retrouver les trois types de solution elliptique,
parabolique et hyperbolique en calculant le nombre de points d'intersection du cercle défini par (\ref{eqsol}) avec le bord du disque $\norm{Y}=1$,
\cad le nombre de points à l'infini de la courbe (\ref{eqsol}).
Le calcul est laissé en exercice au lecteur:
%
\begin{eqnarray}
  \mbox{0 point à l'infini} & : & \module{a} < \module{b} \mbox{elliptique} \nonumber \\
  \mbox{1 point à l'infini} & : & \module{a} = \module{b} \mbox{parabolique} \\
  \mbox{2 point à l'infini} & : & \module{a} > \module{b} \mbox{hyperbolique}. \nonumber
\end{eqnarray}
%
Dans le premier cas,
les solutions sont des cercles complètement contenus dans le disque;
dans le second cas ils sont tangents à l'infini:
ce sont les horicycles du disque de Poincaré;
enfin dans le troisième cas,
il coupe le bord à l'infini en deux points distincts.

On peut vérifier (exercice) que dans le cas hyperbolique,
\ie $\module{a} > \module{b}$,
l'angle que fait la trajectoire avec le bord à l'infini est constant.
En orientant correctement les trajectoires et le bord,
on trouve que cet angle $\phi$ est donné (au signe près) par:
%
\begin{equation}
  \cos(\phi) = {b\over a}.
\end{equation}
%
On retrouve en particulier,
lorsque $b=0$ (le champ magnétique est nul),
la représentation des géodésiques du disque de Poincaré comme les cercles qui coupent le bord à l'infini à angle droit.
Si on pose $b=qB$ et $a=m$ comme précédemment,
alors $\cos(\phi) = qB/m$ est la {\em pulsation cyclotron}\footnote{Merci à Christian Duval pour m'avoir précisé ce fait.} de la particule.

Lorsque $2b+t=0$,
les mouvements sont solutions de l'équation $Y\cdot T = -2b$,
mais d'autre part $\norm{M}^2=\norm{T}^2-t^2=4(a^2-b^2)\Rightarrow \norm{T}=2\module{a}$.
On en déduit que les mouvements sont les droites affines qui passent à la distance $\module{b/a}$ de l'origine.
Autrement dit,
les trajectoires rectilignes,
solutions du système dynamique,
sont les tangentes au cercle de rayon $\module{b/a}$.

Nous venons de voir comment l'application moment nous a permis de traiter complètement l'étude des courbes à courbure prescrite sur le disque de Poincaré,
et comment chaque espèce de ces courbes est naturellement l'espace des caractéristiques d'une 2-forme fermée $\SL(2,\RR)$-invariante sur le fibré des directions du disque $H^2$ et hérite donc naturellement d'une structure de variété symplectique.
Grâce à l'application moment,
en accord avec le théorème de Kirillov-Souriau,
chacun de ces espaces est identifié à une {\em orbite coadjointe}\index{orbite coadjointe} de $\SL(2,\RR)$ selon la valeur courbure.
